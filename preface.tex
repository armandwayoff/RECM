\pageblanche

\chapter*{Préface}
\addcontentsline{toc}{chapter}{Préface}  

%Ce recueil est né d'une simple liste des exercices dits "classiques" (dans le sens où leur résolution invoque un raisonnement / une méthode à maîtriser où s'ils sont la démonstration de résultats hors du programme des CPGE) de mathématiques. \\
%Cette a rapidement été complétée par des remarques et les démarches de résolution des exerices. \\
%Ce recueil est maintement un fourre-tout; certains exerices ayant une correction complète et d'autres pratiquement aucune. \\
%Notes historiques, références à mes lectures personelles, schémas ... \\
% Ce recueil fait parfois des disgressions hors programme mais c pour el plésire.

A faire:

cf tableau blanc, \\
notations: espace entre Ker et rg, décider si je mets des () ou pas \\
normaliser les références aux exerices du TD. \\
Rappeler la liste des sous parties avec chaque début de chapitre. \\
Trouver un moyen d'intégrer des schémas quiver.\\
Style table des matières \\
Minitoc\\
Faire des cadres def, prop, théo, ... avec des titres \\
Programmer graphique constante euler \\
Pb environnement preuve \\
Annale 6/04 X/ENS PSI 2018 Partie 5 \\
Format déterminants démo vandermonde \\
format déterminants démo distance gram \\
14.12 donner la définition d'une chaîne de Markov \\
14.16 faire un graphe de le fonction indicatrice d'euler \\
Intégrale de lesbegue : schéma \\
14.17 : schéma \\
usepackage{braket} Set https://tex.stackexchange.com/questions/561231/latex-commands-for-number-sets-with-and-without-zero \\
Symbole privé de \\
\url{https://tex.stackexchange.com/questions/36039/automatic-size-adjustment-for-nested-parentheses} \\
centrer les figures

\begin{flushright}
	Armand \textsc{Wayoff}
\end{flushright}

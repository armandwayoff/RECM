\usepackage[dvipsnames]{xcolor}
\usepackage{ragged2e}

% \usepackage{quiver}

% Pour la fonction Gamma
\usepackage{pgfplots}
\pgfplotsset{compat=newest}
%

% Pour la figure homothétie
\usepackage{tkz-euclide}
%

\usepackage{hyperref}
\usepackage{multicol}
\usepgfplotslibrary{fillbetween}

\usepackage{tikz}
\usetikzlibrary{automata, 
                decorations.pathreplacing, % pour les curly braces de la fig du déter
                calligraphy, % pour les curly braces de la fig du déter
                arrows, % customizing arrows
                positioning, % positioning nodes
                calc,
                bending % figure racines troisième de l'unité
                }

% Pour pouvoir enlever l'indentation dans un itemize
%https://tex.stackexchange.com/questions/131637/no-indentation-for-non-item-within-itemize
\newcommand\NoIndent[1]{%
  \par\vbox{\parbox[t]{\linewidth}{#1}}%
}

%%%
\usepackage[outline]{contour} % glow around text
\contourlength{1.0pt}
%%%

% Pour les longues équations (théorème d'intégration par parties généralisées) https://tex.stackexchange.com/questions/3782/how-can-i-split-an-equation-over-two-or-more-lines
%\usepackage{breqn}

% Pour l'arbre de Calkin-Wilf
\usepackage{forest}
%%%%%%%%%%%%%%%%%%%%%%%

\tikzset{node distance=3.5cm, % Minimum distance between two nodes. Change if necessary.
    every state/.style={ % Sets the properties for each state
    semithick,
    fill=red!20},
    initial text={},     % No label on start arrow
    double distance=1pt, % Adjust appearance of accept states
    every edge/.style={  % Sets the properties for each transition
    draw,
    ->,>=stealth',     % Makes edges directed with bold arrowheads
    auto,
    semithick}}
          
\tikzset{
    block/.style={
    draw, 
    rectangle, 
    minimum height=1.5cm, 
    minimum width=3cm, align=center
    }, 
    line/.style={->,>=latex'}
}
           
\usepackage{tikz-3dplot}

% Pour le diagramme quiver
\usepackage{tikz-cd}
% 

\usepackage{fancyhdr}
\usepackage{amsfonts} 
\usepackage{amsmath}
\usepackage{amssymb}
\usepackage{mathtools}
\usepackage{bm}
\usepackage{stmaryrd}
\usepackage{mathrsfs}
\usepackage{enumitem}
\usepackage{xurl}

% Notations comme les sujets de Centrale
% \usepackage[mathscr]{eucal}

% Pour les matrices par blocs
\newcommand{\rvline}{\hspace*{-\arraycolsep}\vline\hspace*{-\arraycolsep}}

% Figures
\usepackage{physics}
\usepackage{mathdots}
\usepackage{cancel}
\usepackage{color}
\usepackage{siunitx}
\usepackage{array}
\usepackage{multirow}
\usepackage{gensymb}
\usepackage{tabularx}
\usepackage{extarrows}

% \renewcommand\qedsymbol{$\blacksquare$}

%\newenvironment{preuve}[1][\proofname]{%
%  \begin{proof}[#1]$ $\par\nobreak\ignorespaces
%}{%
%  \end{proof}
%}

\newenvironment{preuve}
  {\begin{proof}[\textbf{\blacksquare\ \textsc{Démonstration.}}]}
  {\end{proof}}

\newenvironment{solution}
  {\renewcommand\qedsymbol{$\lhd$}\begin{proof}[\blacktriangleright \textbf{\textsc{Solution.}}]}
  {\end{proof}}

\newenvironment{elem_sol}
  {\renewcommand\qedsymbol{$\lhd$}\begin{proof}[\blacktriangleright \textbf{\textsc{Éléments de solution.}}]}
  {\end{proof}}


\usepackage[most]{tcolorbox}

\newtcolorbox[auto counter,number within=chapter]{defi}[1][]{%
  enhanced jigsaw, % better frame drawing
  borderline west={2pt}{0pt}{YellowOrange}, % straight vertical line at the left edge
  sharp corners, % No rounded corners
  boxrule=0pt, % no real frame,
  fonttitle={\bfseries},
  % colback=YellowOrange!10!white,
  colback=white,
  coltitle={YellowOrange},  % Black colour for title
  title={Définition \thetcbcounter.\ },  % Fixed title
  attach title to upper, % Move the title into the box
  #1
}

\newtcolorbox[auto counter,number within=chapter]{theo}[1][]{%
  enhanced jigsaw, % better frame drawing
  borderline west={2pt}{0pt}{RoyalBlue}, % straight vertical line at the left edge
  sharp corners, % No rounded corners
  boxrule=0pt, % no real frame,
  fonttitle=\bfseries,
  fontupper=\slshape,
  % colback=RoyalBlue!10!white,
  colback=white,
  coltitle={RoyalBlue},  % Black colour for title
  title={Théorème \thetcbcounter.\ },  % Fixed title
  attach title to upper, % Move the title into the box
  #1
}

\newtcolorbox[auto counter,number within=chapter]{prop}[1][]{%
  enhanced jigsaw, % better frame drawing
  borderline west={2pt}{0pt}{RoyalBlue}, % straight vertical line at the left edge
  sharp corners, % No rounded corners
  boxrule=0pt, % no real frame,
  fonttitle={\bfseries},
  % colback=RoyalBlue!10!white,
  colback=white,
  coltitle={RoyalBlue},  % Black colour for title
  title={Proposition \thetcbcounter.\ },  % Fixed title
  attach title to upper, % Move the title into the box
  #1
}

\newtcolorbox[auto counter,number within=chapter]{corol}[1][]{%
  enhanced jigsaw, % better frame drawing
  borderline west={2pt}{0pt}{RoyalBlue}, % straight vertical line at the left edge
  sharp corners, % No rounded corners
  boxrule=0pt, % no real frame,
  fonttitle={\bfseries},
  % colback=RoyalBlue!10!white,
  colback=white,
  coltitle={RoyalBlue},  % Black colour for title
  title={Corollaire \thetcbcounter.\ },  % Fixed title
  attach title to upper, % Move the title into the box
  #1
}

\newtcolorbox[auto counter,number within=chapter]{lemme}[1][]{%
  enhanced jigsaw, % better frame drawing
  borderline west={2pt}{0pt}{RoyalBlue}, % straight vertical line at the left edge
  sharp corners, % No rounded corners
  boxrule=0pt, % no real frame,
  fonttitle={\bfseries},
  % colback=RoyalBlue!10!white,
  colback=white,
  coltitle={RoyalBlue},  % Black colour for title
  title={Lemme \thetcbcounter.\ },  % Fixed title
  attach title to upper, % Move the title into the box
  #1
}

\newtcolorbox{box_titre}[2][]{
    lower separated=false,
    colback=blue!2!white,
    colframe=gray,
    colbacktitle=blue!20!white,
    coltitle=black,
    enhanced jigsaw,
    sharp corners,
    borderline west={2pt}{0pt}{RoyalBlue}, % straight vertical line at the left edge
    boxrule=0pt, % no real frame,
    attach boxed title to top left={yshift=-0.1in,xshift=0.15in},
    boxed title style={boxrule=0pt,colframe=white,},
title=#2,#1}

\newtcolorbox[auto counter,number within=chapter]{remarque}[1][]{%
  enhanced jigsaw, % better frame drawing
  borderline west={2pt}{0pt}{ForestGreen}, % straight vertical line at the left edge
  sharp corners, % No rounded corners
  boxrule=0pt, % no real frame,
  fonttitle={\bfseries},
  % colback=ForestGreen!10!white,
  colback=white,
  coltitle={ForestGreen},  % Black colour for title
  title={Remarque \thetcbcounter.\ },  % Fixed title
  attach title to upper, % Move the title into the box
  #1
}

\newtcolorbox[auto counter,number within=chapter]{exercice}[1][]{%
  enhanced jigsaw, % better frame drawing
  borderline west={2pt}{0pt}{BrickRed}, % straight vertical line at the left edge
  sharp corners, % No rounded corners
  boxrule=0pt, % no real frame,
  fonttitle={\bfseries},
  % colback=red!10!white,
  colback=white,
  coltitle={BrickRed},  % Black colour for title
  title={Exercice \thetcbcounter.\ },  % Fixed title
  attach title to upper, % Move the title into the box
  #1
}

\newtcolorbox[auto counter,number within=chapter]{methode}[1][]{%
  enhanced jigsaw, % better frame drawing
  borderline west={2pt}{0pt}{darkgray}, % straight vertical line at the left edge
  sharp corners, % No rounded corners
  boxrule=0pt, % no real frame,
  fonttitle={\bfseries},
  colback=white,
  coltitle={darkgray},  % Black colour for title
  title={Méthode \thetcbcounter.\ },  % Fixed title
  attach title to upper, % Move the title into the box
  #1
}
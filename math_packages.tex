\usepackage[dvipsnames]{xcolor}
\usepackage{ragged2e}

\usepackage{subcaption}

\usepackage{pdfpages}

\usepackage{fontawesome} % pour les icônes

% Pour l'index des notations
\usepackage{longtable}

%%%%%%%%%%%%%%%%%%%%%%%%
% Pour le tableau des transformées de Laplace
\usepackage{adjustbox}
\usepackage{tabularray}

%%%%%%%%%%%%%%%%%%%%%%%
% Commande pour citer un théorème
\usepackage{ifthen}
\newcommand{\theoremeutilise}[1]{%
  \ifthenelse{\isodd{\thepage}}%
    {#1 $\bigstar$}% If the page number is even
    {$\bigstar$ #1}% If the page number is odd
}

%%%%%%%%%%%%%%%%%%
% Pour créer des annexes par chapitre
% https://tex.stackexchange.com/questions/120716/appendix-after-each-chapter
\usepackage{appendix}
\usepackage{chngcntr}

% Start of subappendices environment
\AtBeginEnvironment{subappendices}{%
\newpage
% \addcontentsline{toc}{section}{Annexes}
\counterwithin{figure}{section}
\counterwithin{table}{section}
}

% End of subappendices environment
\AtEndEnvironment{subappendices}{%
\counterwithout{figure}{section}
\counterwithout{table}{section}
}
%%%%%%%%%%%%%%%%%%

% Pour la fonction Gamma
\usepackage{pgfplots}
\pgfplotsset{compat=newest}
\usepgfplotslibrary{groupplots}

% Pour la figure homothétie
\usepackage{tkz-euclide}

\usepackage{hyperref}
\hypersetup{
    colorlinks=true,
    linkcolor=blue,
    filecolor=magenta,      
    urlcolor=cyan
}
\usepackage{multicol}
\usepgfplotslibrary{fillbetween}

\usepackage{tikz}
\usetikzlibrary{automata, 
                decorations.pathreplacing, % pour les curly braces de la fig du déterminant
                calligraphy, % pour les curly braces de la fig du déterminant
                arrows, % customizing arrows
                positioning, % positioning nodes
                calc,
                bending, % figure racines troisième de l'unité
                matrix, % critère de nilpotence par la trace
                patterns
                }
% Pour le schéma de la preuve de la densité des matrices diagonalisables
\usetikzlibrary{decorations.markings,arrows}
\tikzset{
    arrowMe/.style={
        postaction=decorate,
        decoration={
            markings,
            mark=at position .45 with {\arrow[thick]{#1}}
        }
    }
}
                
% Pour la figure de projection orthogonale
%\usepackage[active,tightpage]{preview}

% Pour pouvoir enlever l'indentation dans un itemize
% https://tex.stackexchange.com/questions/131637/no-indentation-for-non-item-within-itemize
\newcommand\NoIndent[1]{%
  \par\vbox{\parbox[t]{\linewidth}{#1}}%
}

\usepackage[outline]{contour} % glow around text
\contourlength{1.0pt}

% Pour les longues équations (théorème d'intégration par parties généralisées) https://tex.stackexchange.com/questions/3782/how-can-i-split-an-equation-over-two-or-more-lines
%\usepackage{breqn}

% Pour l'arbre de Calkin-Wilf
\usepackage{forest}

\tikzset{node distance=3.5cm, % Minimum distance between two nodes. Change if necessary.
    every state/.style={ % Sets the properties for each state
    semithick,
    fill=red!20},
    initial text={},     % No label on start arrow
    double distance=1pt, % Adjust appearance of accept states
    every edge/.style={  % Sets the properties for each transition
    draw,
    ->,>=stealth',     % Makes edges directed with bold arrowheads
    auto,
    semithick}}
          
\tikzset{
    block/.style={
    draw, 
    rectangle, 
    minimum height=1.5cm, 
    minimum width=3cm, align=center
    }, 
    line/.style={->,>=latex'}
}
           
\usepackage{tikz-3dplot}

% Pour le diagramme quiver
\usepackage{tikz-cd}
\usepackage{quiver}

% Scale pour le diagramme sur le red des endos
% https://tex.stackexchange.com/questions/325297/how-to-scale-a-tikzcd-diagram
\tikzcdset{scale cd/.style={every label/.append style={scale=#1},
    cells={nodes={scale=#1}}}}

%%%% Commente par Alain car conflit, deja charge par koa ? %%%%
% \usepackage{fancyhdr}
\usepackage{amsfonts} 
\usepackage{amsmath}
\usepackage{amsthm}
\usepackage{amssymb}

% \usepackage{mathtools}
\usepackage{bm}
\usepackage{stmaryrd}
%\usepackage{mathrsfs}
\usepackage{euscript}
\usepackage{enumitem}
\usepackage{tasks}
\usepackage{xurl}
\usepackage{dsfont}  % Pour la fonction indicatrice \mathds{1}

\usepackage{cfr-lm} % https://tex.stackexchange.com/questions/301699/oldstyle-numbers-in-body-text-computer-modern

% Pour les matrices par blocs
\newcommand{\rvline}{\hspace*{-\arraycolsep}\vline\hspace*{-\arraycolsep}}

% Figures
\usepackage{physics}
\usepackage{mathdots}
\usepackage{cancel}
\usepackage{color}
\usepackage{siunitx}
\usepackage{array}
\usepackage{multirow}
\usepackage{gensymb}
\usepackage{tabularx}
\usepackage{extarrows}

%\newenvironment{preuve}[1][\proofname]{%
%  \begin{proof}[#1]$ $\par\nobreak\ignorespaces
%}{%
%  \end{proof}
%}

% \newenvironment{preuve}
%   {\begin{proof}[$\blacksquare$\ \textbf{\textsc{\emph{Démonstration}}}]}
%   {\end{proof}}
  
% \newenvironment{elem_preuve}
%   {\begin{proof}[$\blacksquare$\ \textbf{\textsc{\emph{Éléments de démonstration}}}]}
%   {\end{proof}}

% \newenvironment{solution}
%   {\renewcommand\qedsymbol{$\lhd$}\begin{proof}[$\blacktriangleright$ \textbf{\textsc{\emph{Solution}}}]}
%   {\end{proof}}

% \newenvironment{elem_sol}
%   {\renewcommand\qedsymbol{$\lhd$}\begin{proof}[$\blacktriangleright$ \textbf{\textsc{\emph{Éléments de solution}}}]}
%   {\end{proof}}

%----------------------------------------------------------------------------------------
%	PACKAGES AND OTHER DOCUMENT CONFIGURATIONS
%----------------------------------------------------------------------------------------

\documentclass[
	a4paper, % Page size
	fontsize=10pt, % Base font size
	twoside=true, % Use different layouts for even and odd pages (in particular, if twoside=true, the margin column will be always on the outside)
	% open=any, % If twoside=true, uncomment this to force new chapters to start on any page, not only on right (odd) pages
	chapterentrydots=true, % Uncomment to output dots from the chapter name to the page number in the table of contents
	numbers=noenddot, % Comment to output dots after chapter numbers; the most common values for this option are: enddot, noenddot and auto (see the KOMAScript documentation for an in-depth explanation)
]{kaobook}

\usepackage[french]{babel}

% guillements
\usepackage[
    left = \flqq{},% 
    right = \frqq{},% 
    leftsub = \flq{},% 
    rightsub = \frq{} %
]{dirtytalk}


\usepackage[T1]{fontenc} % Output font encoding for international characters

% Load packages for testing
\usepackage{blindtext}
%\usepackage{showframe} % Uncomment to show boxes around the text area, margin, header and footer
%\usepackage{showlabels} % Uncomment to output the content of \label commands to the document where they are used

% Pour faire des pages blanches
\usepackage{afterpage}
\newcommand\pageblanche{
    \null
    \thispagestyle{empty}
    \addtocounter{page}{-1}
    \newpage
}

% Load the bibliography package
\usepackage{kaobiblio}
\addbibresource{main.bib} % Bibliography file

% Load mathematical packages for theorems and related environments
\usepackage[framed=true]{kaotheorems}

% Load the package for hyperreferences
\usepackage{kaorefs}

\usepackage[dvipsnames]{xcolor}
\usepackage{ragged2e}

% Pour la fonction Gamma
\usepackage{pgfplots}
\pgfplotsset{compat=newest}

% Pour la figure homothétie
\usepackage{tkz-euclide}

\usepackage{hyperref}
\usepackage{multicol}
\usepgfplotslibrary{fillbetween}

\usepackage{tikz}
\usetikzlibrary{automata, 
                decorations.pathreplacing, % pour les curly braces de la fig du déterminant
                calligraphy, % pour les curly braces de la fig du déterminant
                arrows, % customizing arrows
                positioning, % positioning nodes
                calc,
                bending, % figure racines troisième de l'unité
                matrix % critère de nilpotence par la trace
                }
% Pour le schéma de la preuve de la densité des matrices diagonalisables
\usetikzlibrary{decorations.markings,arrows}
\tikzset{
    arrowMe/.style={
        postaction=decorate,
        decoration={
            markings,
            mark=at position .45 with {\arrow[thick]{#1}}
        }
    }
}
                
% Pour la figure de projection orthogonale
%\usepackage[active,tightpage]{preview}

% Pour pouvoir enlever l'indentation dans un itemize
% https://tex.stackexchange.com/questions/131637/no-indentation-for-non-item-within-itemize
\newcommand\NoIndent[1]{%
  \par\vbox{\parbox[t]{\linewidth}{#1}}%
}

\usepackage[outline]{contour} % glow around text
\contourlength{1.0pt}

% Pour les longues équations (théorème d'intégration par parties généralisées) https://tex.stackexchange.com/questions/3782/how-can-i-split-an-equation-over-two-or-more-lines
%\usepackage{breqn}

% Pour l'arbre de Calkin-Wilf
\usepackage{forest}

\tikzset{node distance=3.5cm, % Minimum distance between two nodes. Change if necessary.
    every state/.style={ % Sets the properties for each state
    semithick,
    fill=red!20},
    initial text={},     % No label on start arrow
    double distance=1pt, % Adjust appearance of accept states
    every edge/.style={  % Sets the properties for each transition
    draw,
    ->,>=stealth',     % Makes edges directed with bold arrowheads
    auto,
    semithick}}
          
\tikzset{
    block/.style={
    draw, 
    rectangle, 
    minimum height=1.5cm, 
    minimum width=3cm, align=center
    }, 
    line/.style={->,>=latex'}
}
           
\usepackage{tikz-3dplot}

% Pour le diagramme quiver
\usepackage{tikz-cd}
\usepackage{quiver}

% Scale pour le diagramme sur le red des endos
% https://tex.stackexchange.com/questions/325297/how-to-scale-a-tikzcd-diagram
\tikzcdset{scale cd/.style={every label/.append style={scale=#1},
    cells={nodes={scale=#1}}}}

%%%% Commente par Alain car conflit, deja charge par koa ? %%%%
% \usepackage{fancyhdr}
\usepackage{amsfonts} 
\usepackage{amsmath}
\usepackage{amsthm}
\usepackage{amssymb}

% \usepackage{mathtools}
\usepackage{bm}
\usepackage{stmaryrd}
%\usepackage{mathrsfs}
\usepackage{euscript}
\usepackage{enumitem}
\usepackage{xurl}
\usepackage{dsfont}  % Pour la fonction indicatrice \mathds{1}


\usepackage{cfr-lm} % https://tex.stackexchange.com/questions/301699/oldstyle-numbers-in-body-text-computer-modern

% Pour les matrices par blocs
\newcommand{\rvline}{\hspace*{-\arraycolsep}\vline\hspace*{-\arraycolsep}}

% Figures
\usepackage{physics}
\usepackage{mathdots}
\usepackage{cancel}
\usepackage{color}
\usepackage{siunitx}
\usepackage{array}
\usepackage{multirow}
\usepackage{gensymb}
\usepackage{tabularx}
\usepackage{extarrows}

%\newenvironment{preuve}[1][\proofname]{%
%  \begin{proof}[#1]$ $\par\nobreak\ignorespaces
%}{%
%  \end{proof}
%}

\newenvironment{preuve}
  {\begin{proof}[$\blacksquare$\ \textbf{\textsc{\emph{Démonstration}}}]}
  {\end{proof}}
  
\newenvironment{elem_preuve}
  {\begin{proof}[$\blacksquare$\ \textbf{\textsc{\emph{Éléments de démonstration}}}]}
  {\end{proof}}

\newenvironment{solution}
  {\renewcommand\qedsymbol{$\lhd$}\begin{proof}[$\blacktriangleright$ \textbf{\textsc{\emph{Solution}}}]}
  {\end{proof}}

\newenvironment{elem_sol}
  {\renewcommand\qedsymbol{$\lhd$}\begin{proof}[$\blacktriangleright$ \textbf{\textsc{\emph{Éléments de solution}}}]}
  {\end{proof}}

\DeclareMathOperator{\ch}{ch}

\newcommand{\me}{\mathrm{e}}
\newcommand{\mi}{\mathrm{i}}
\newcommand{\mj}{\mathrm{j}} % racine troisième de l'unité

\newcommand{\R}{\mathbb{R}}
\newcommand{\Rp}{\R_+}
\renewcommand{\Re}{\R^\star}
\newcommand{\Rpe}{\Re_+}
\renewcommand{\C}{\mathbb{C}}
\renewcommand{\Ce}{\C^\star}
\newcommand{\K}{\mathbb{K}}
\newcommand{\Ke}{\K^\star}
\newcommand{\N}{\mathbb{N}}
\newcommand{\Ne}{\N^\star}
\newcommand{\Z}{\mathbb{Z}}
\newcommand{\Q}{\mathbb{Q}}
\newcommand{\A}{\mathbb{A}} % Nombres algébriques
\newcommand{\Premier}{\mathbb{P}} % Nombres premiers

\renewcommand{\d}{\, \mathrm{d}}

\newcommand{\Ninf}[1]{\Vert #1 \Vert_\infty}
\newcommand{\norme}[1]{\Vert #1 \Vert}
\newcommand{\segN}[2]{\llbracket #1 , #2 \rrbracket}

% Complexes
\newcommand{\Reel}{\mathrm{Re}}

\newcommand{\ptnclegras}[1]{\textbf{#1}}
\newcommand{\ptnclecadre}[1]{\boxed{#1}}

% Algèbre
% \newcommand{\Gl}{\mathscr{G}\kern-0.16em\ell}
\newcommand{\Gl}{\mathrm{GL}}
\newcommand{\I}{\mathrm{I}}
\newcommand{\Id}{\mathrm{Id}}
\newcommand{\Rg}{\mathrm{Rg}\,}
\newcommand{\Ker}{\mathrm{Ker}\,}
\newcommand{\Vect}{\mathrm{Vect}}
\newcommand{\Sp}{\mathrm{Sp}}
\newcommand{\M}{\mathscr{M}}
\newcommand{\Endo}{\mathscr{L}}
\newcommand{\Trsp}[1]{#1^\top}
\newcommand{\Inv}[1]{#1^{-1}}
\renewcommand{\Tr}{\mathrm{Tr}}
\newcommand{\com}{\mathrm{com}}
\newcommand{\Mat}{\mathrm{Mat}}
\newcommand{\Diag}{\mathrm{Diag}}

% Probas
\newcommand{\E}{\mathbf{E}}
\newcommand{\V}{\mathbf{V}}
\renewcommand{\P}{\mathbf{P}}

% Suites remarquables
\newcommand{\Leg}{\mathrm{L}}
\newcommand{\Lag}{\mathrm{L}}
\newcommand{\Hilb}{\mathrm{H}}
\newcommand{\Hermite}{\mathrm{H}}
\newcommand{\Bern}{\mathrm{B}}
\newcommand{\Tcheby}{\mathrm{T}}
\newcommand{\Wallis}{\mathrm{W}}
\newcommand{\Cauchy}{\mathrm{C}}
\newcommand{\Gram}{\mathrm{G}}
\newcommand{\Bernstein}{\mathrm{B}}
\newcommand{\Vandermonde}{\mathrm{V}}

% Symboles
%https://tex.stackexchange.com/questions/4216/how-to-typeset-correctly
% :=
\newcommand*{\defeq}{\mathrel{\vcenter{\baselineskip0.5ex \lineskiplimit0pt
                     \hbox{\scriptsize.}\hbox{\scriptsize.}}}%
                     =}

% https://tex.stackexchange.com/questions/349747/equivalence-of-sequence-sim-with-some-text-under      

\newcommand{\isEquivTo}[1]{%
  \mathpalette\isEquivToInner{#1}%
}
\newcommand{\isEquivToInner}[2]{%
  \ifx#1\displaystyle
    \underset{#2}{\sim}
  \else
    \sim_{#2}
  \fi
}

\graphicspath{{examples/documentation/images/}{images/}} % Paths in which to look for images

\makeindex[columns=3, title=Alphabetical Index, intoc] % Make LaTeX produce the files required to compile the index

\makeglossaries % Make LaTeX produce the files required to compile the glossary
\input{glossary.tex} % Include the glossary definitions

\makenomenclature % Make LaTeX produce the files required to compile the nomenclature

% Reset sidenote counter at chapters
%\counterwithin*{sidenote}{chapter}

%----------------------------------------------------------------------------------------
\begin{document}

%----------------------------------------------------------------------------------------
%	BOOK INFORMATION
%----------------------------------------------------------------------------------------

% \titlehead{The \texttt{kaobook} class}
% \subject{Use this document as a template}

\title[Le Recueil des Exercices Classiques de Mathématiques]{\textrm{\textsc{Le Recueil des Exercices Classiques de Mathématiques}}}
% \subtitle{corrigés}

\author[Armand Wayoff]{\textsc{Armand Wayoff}}

\date{}

% \publishers{Calvage \& Mounet}

%----------------------------------------------------------------------------------------

\frontmatter % Denotes the start of the pre-document content, uses roman numerals

%----------------------------------------------------------------------------------------
%	OPENING PAGE
%----------------------------------------------------------------------------------------

%\makeatletter
%\extratitle{
%	% In the title page, the title is vspaced by 9.5\baselineskip
%	\vspace*{9\baselineskip}
%	\vspace*{\parskip}
%	\begin{center}
%		% In the title page, \huge is set after the komafont for title
%		\usekomafont{title}\huge\@title
%	\end{center}
%}
%\makeatother

\maketitle

%----------------------------------------------------------------------------------------
%	DEDICATION
%----------------------------------------------------------------------------------------

\pageblanche

\begin{LARGE}
\emph{\say{ Celui qui aura goûté, qui aura vu, ne fût-ce que de loin, la splendide harmonie des lois naturelles, sera mieux disposé à faire peu de cas de ses petits intérêts égoïstes ; il aura un idéal qu’il aimera mieux que lui-même, et c’est là le seul terrain sur lequel on puisse bâtir une morale. Pour cet idéal, il travaillera sans marchander sa peine et sans attendre aucune de ces grossières récompenses qui sont tout pour certains hommes ; et quand il aura pris ainsi l’habitude du désintéressement, cette habitude le suivra partout ; sa vie entière en restera comme parfumée. }}
\begin{flushright}
\textsc{--- Henri Poincaré}, \emph{Dernières pensées}
\end{flushright}
\end{LARGE}

\begin{figure}[h]
    \includegraphics[width=10cm]{henri_poincare.jpg}
\end{figure}

%----------------------------------------------------------------------------------------
%	PREFACE
%----------------------------------------------------------------------------------------

\pageblanche

\chapter*{Préface}
\addcontentsline{toc}{chapter}{Préface}  

%Ce recueil est né d'une simple liste des exercices dits "classiques" (dans le sens où leur résolution invoque un raisonnement / une méthode à maîtriser où s'ils sont la démonstration de résultats hors du programme des CPGE) de mathématiques. \\
%Cette a rapidement été complétée par des remarques et les démarches de résolution des exerices. \\
%Ce recueil est maintement un fourre-tout; certains exerices ayant une correction complète et d'autres pratiquement aucune. \\
%Notes historiques, références à mes lectures personelles, schémas ... \\
% Ce recueil fait parfois des disgressions hors programme mais c pour el plésire.

A faire:


\begin{flushright}
	Armand \textsc{Wayoff}
\end{flushright}

\index{preface}

%----------------------------------------------------------------------------------------
%	TABLE OF CONTENTS & LIST OF FIGURES/TABLES
%----------------------------------------------------------------------------------------

\begingroup % Local scope for the following commands

% Define the style for the TOC, LOF, and LOT
%\setstretch{1} % Uncomment to modify line spacing in the ToC
%\hypersetup{linkcolor=blue} % Uncomment to set the colour of links in the ToC
\setlength{\textheight}{230\hscale} % Manually adjust the height of the ToC pages

% Turn on compatibility mode for the etoc package
\etocstandarddisplaystyle % "toc display" as if etoc was not loaded
\etocstandardlines % "toc lines" as if etoc was not loaded

\tableofcontents % Output the table of contents

% \listoffigures % Output the list of figures

% Comment both of the following lines to have the LOF and the LOT on different pages
\let\cleardoublepage\bigskip
\let\clearpage\bigskip

% \listoftables % Output the list of tables

\endgroup

%----------------------------------------------------------------------------------------
%	MAIN BODY
%----------------------------------------------------------------------------------------

\mainmatter % Denotes the start of the main document content, resets page numbering and uses arabic numbers
\setchapterstyle{kao} % Choose the default chapter heading style

\pagelayout{wide} % No margins
\addpart{Algèbre}
\pagelayout{margin} % Restore margins


\chapter{Polynômes}
\labch{polynomes}

\marginnote{Texte issu de \cite{oraux_x_ens_1}}

\begin{marginfigure}[5cm]
    \centering
    \includegraphics{images/jerome_cardan.jpg}
    \caption*{\centering Jérome \textsc{Cardan} (1501 - 1576)}
\end{marginfigure}

\textsl{
La théorie des équations polynomiales, qui précède de loin la définition formelle des polynômes, a été le propos essentiel de l'algèbre jusqu'au \textsc{xix}$^\me$ siècle. Elle est à l'origine de nombreuses notions: corps, nombres algébriques \dots Son développement est lié aux extensions successives de la notion de nombre: introduction des nombres négatifs, des nombres irrationnels, puis des nombres complexes. \\ 
Dès la plus haute Antiquité, on rencontre des exemples de résolutions d'équations. Les Babyloniens savent résoudre l'équation du second degré et les Grecs en font la base même de leur géométrie. \\
Après l'Antiquité, il faudra attendre le \textsc{xvi}$^\me$ siècle pour que des progrès substantiels apparaissent, dus à l'école italienne. \textsc{Scipone del Ferro}, \textsc{Tartaglia} et \textsc{Cardan} apportent la solution de l'équation du troisième dégré. L'équation générale est ramenée à la forme réduite $x^3 + px + q = 0$ \note, dont une solution s'écrit
$$x = \sqrt[3]{-\frac{q}{2} + \sqrt{\frac{q^2}{4} + \frac{p^3}{27}}} + \sqrt[3]{-\frac{q}{2} - \sqrt{\frac{q^2}{4} + \frac{p^3}{27}}}.$$
\marginnote[5.5cm]{
    \note Soient $(a, b, c) \in \C^3$ et $P$ le polynôme défini par:
    $$\forall z \in \C,\ P(z) \defeq z^3 + az^2 + bz + c.$$
    On observe que pour $h \in \C$,
    \begin{align*}
        P(z + h) = z^3 + (3h + a)&z^2 + (3h^2 + 2ha + b)z \\
        & + h^3 + ah^2 + bh + c.
    \end{align*}
    Pour $h = -\frac{a}{3}$, on définit le polynôme $Q$ par
    $$\forall z \in \C,\ Q(z) \defeq P(z + h) = z^3 + pz + q$$
    où $p \defeq -\frac{a^2}{3}+b$, $q \defeq \frac{2a^3}{27} - \frac{ab}{3} + c$.
    On a alors
    $$P(z) = 0 \Longleftrightarrow Q(z-h) = 0.$$
    La recherche des solutions de l'équation $P(z) = 0$ est ainsi ramenée au problème analogue pour l'équation $Q(z) = 0$.
}
Cette solution soulève des difficultés: si $\frac{q^2}{4} + \frac{p^3}{27}$ est négatif, cas où l'équation a des racines -- on le sait depuis \textsc{Archimède} -- on ne peut pas calculer $x$. Pour lever la difficulté, \textsc{Cardan} introduit timidement de nouveaux nombres, \say{ impossibles } ou \say{ imaginaires }. \textsc{Ferrari} et \textsc{Bombelli} résolvent l'équation du quatrième degré. \\
Grâce à l'école italienne, le théorie générale des équations algébriques se précise. L'équation étant mise sous le forme $P(x) = 0$, on prend conscience de l'importance du dégré de $P$ pour le nombre de solutions. On découvre que si $a$ est une racine de $P$, on peut factoriser par $x-a$. Les relations entre les coefficients et les fonctions symétriques des racines d'un polynôme apparaissent chez \textsc{Viète} (1540-1603), mais c'est \textsc{Girard} qui en 1629 leur donne toute leur extension. Suivi par \textsc{Newton}, il exprime les sommes des puissances des racines en fonction des coefficients. L'étude des fonctions symétriques des racines va se développer au \textsc{xvii}$^\me$ siècle avec \textsc{Waring} et au \textsc{xix}$^\me$ siècle avec \textsc{Cauchy}. \\
Au \textsc{xvii}$^\me$ siècle, la majorité des mathématiciens est convaincue qu'une équation de degré $n$ possède $n$ racines, celles-ci pouvant ne pas être réelles, mais il faut attendre \textsc{d'Alembert} pour trouver en 1724 une définition précise des nombres complexes (sous la forme $a + \sqrt{-1} b$). En 1799, \textsc{Gauss} fournit plusieurs preuves rigoureuses du \say{ théorème fondamental de l'algèbre } ou \say{ théorème de \textsc{d'Alembert}-\textsc{Gauss} }. \\
Des progrès sont réalisés également dans l'étude du nombre de racines réelles, et de leur signe. En 1637, \textsc{Descartes} énonce la règle qui porte son nom sur le nombre de racines positives d'un polynôme. On trouve dans \text{l'Algèbre} de \textsc{Rolle} (1690), la propriété suivante: entre deux solutions de l'équation $P(x)=0$, il existe au moins une solution de l'équation $P'(x)=0$. C'est \textsc{Sturm} qui formule, en 1829, les résultats les plus précis sur le nombre de racines réelles d'un polynôme. \\
Après les succès le l'école italienne au \textsc{xvi}$^\me$ siècle, les mathématiciens se sont attachés à trouver des formules analogues pour les dégrés suivants. Les réflexions sur cette question prennent un tour nouveau avec les travaux de \textsc{Lagrange} (1771), qui étudie les permutations des racines d'une équation laissant invariantes certaines fonctions de ces racines. Ces idées sont approfondies par \textsc{Cauchy} et \textsc{Ruffini}. \textsc{Abel} donne une démonstration rigoureuse de l'impossibilité de résoudre par radicaux l'équation générale de dégré $5$ en 1829. Enfin, en introduisant la notion de groupe, \textsc{Galois} énonce la condition générale à laquelle satisfait toute équation résoluble par radicaux (1831). \\
La distinction entre les nombres algébriques, racines d'un polynôme à coefficients entiers, et les autres qu'on nomme transcendants, date du \textsc{xvii}$^\me$ siècle, mais il faut attendre 1844 pour que \textsc{Liouville} démontre l'existence de nombres transcendants et plus longtemps encore pour que soit démontrée le trascendance de $\me$ (par \textsc{Hermite} en 1872) et celle de $\pi$ (par \textsc{Lindemann} en 1882). \\
Quant à la définition formelle des polynômes et à l'étude de leur structure, elles chemineront tout au long du \textsc{xix}$^\me$ siècle au rythme lent du processus d'axiomatisation de l'algèbre: par exemple, \textsc{Dedekind} introduit la notion de corps et définit les idéaux vers 1870.
}

\newpage

\section{Polynômes de \textsc{Legendre}}
\begin{defi}{Polynômes de \textsc{Legendre}}
    On appelle famille des \emph{polynômes de \textsc{Legendre}} la suite de polynômes $\suite{\Leg}{n}{n \in \Ne}$ définie par
    $$\Leg_n(X) \defeq \frac{1}{2^n} \sum_{k=0}^{n} \binom{n}{k}^2 (X-1)^{n-k}(X+1)^{k}.$$
\end{defi}

\marginnote[0cm]{
    Voir aussi...
}

\begin{exercice}
    \marginnote[0cm]{Source : \cite{exos_oraux} p. 25} 
    \begin{enumerate}
        \item Montrer que $\Leg_n(X) = \frac{1}{2^n \fact{n}} \left[ \big(X^2-1\big)^n \right]^{(n)}$. En déduire la parité de $\Leg_n$ et l'égalité $\sum\limits_{k=0}^n \binom{n}{k}^2 = \binom{2n}{n}$.
        \item Soit $n \in \Ne$, montrer que $\Leg_n(X)$ est scindé à racines simples dans $\interoo{-1}{1}$. 
        \item Montrer que pour tout $n \in \N$, $$\big(X^2-1\big) \Leg''_n(X) + 2X \Leg'_n(X) = n(n+1) \Leg_n.$$
    \end{enumerate}
\end{exercice}  

\begin{elem_sol}
    Pour montrer que $\Leg_n(X)$ est scindé à racines simples dans $\interoo{-1}{1}$, raisonner par récurrence et penser à \textsc{Rolle}. 
\end{elem_sol}

\begin{marginfigure}[-11.5cm]
    \centering
	\input{illustrations/i_polynomes_legendre}
	\caption*{\centering Les premiers polynômes de \textsc{Legendre}}
	\small
	\begin{align*}
	    \color{blue} \Leg_0 = 1 \\
	    \color{red} \Leg_1 = x \\
	    \color{green} \Leg_2 = \frac{1}{2}\big(3x^2-1\big) \\
	    \color{purple} \Leg_3 = \frac{1}{2}\big(5x^3-3x\big) \\
	    \color{black} \Leg_4 = \frac{1}{8}\big(35x^4-30x^2+3\big) \\
	    \color{orange} \Leg_5 = \frac{1}{8}\big(63x^5-70x^3+15x\big)
	\end{align*}
\end{marginfigure}



\section{Polynômes de \textsc{Hilbert}}
\begin{defi}
    On appelle famille des polynômes de \textsc{Hilbert} la suite de polynômes $(\Hilb_n)$ définie par
    $$\Hilb_0 \defeq 1,\ \forall n \in \Ne,\ \Hilb_n \defeq \frac{X(X-1)\cdots(X-n+1)}{n!}.$$
\end{defi}

Voir aussi \nameref{polynome_hilbert} dans la partie algèbre linéaire. 

\begin{exercice}
    \cite{exos_oraux} p.27
    \begin{enumerate}
        \item Montrer que pour tout $(k, n) \in \N \times \Z$, $\Hilb_k(n) \in \Z$.
        \item Soient $n \in \Ne$ et $Q \in \R_n[X]$. Montrer que les deux assertions suivantes sont équivalentes:
        \begin{enumerate}[label=(\roman*)]
            \item $Q(\Z) \subset \Z$,
            \item $\forall m \in \llbracket 0, n \rrbracket, Q(m) \in \Z$.
        \end{enumerate}
    \end{enumerate}
\end{exercice}

\section{Polynômes de \textsc{Bernoulli}}
\marginnote[0cm]{Lire \cite{calcul_infinitesimal} page 297}
Note d'un thème (cf. bureau) \\

Il arrive fréquemment que le calcul exact d’une intégrale soit difficile, voire
impossible pour certaines fonctions et il est courant, dans ce cas, de chercher à
approcher la valeur de l’intégrale en utilisant des polynômes comme les polynômes de \textsc{Bernoulli} par exemple.

\begin{defi}
    Les polynômes de \textsc{Bernoulli} sont l'unique suite de polynômes $(\Bern_n)_{n \in \N}$ telle que:
    \begin{align*}
        &\Bern_0 = 1, \\
        \forall n \in \N,\ &\Bern'_{n+1} = (n+1)\Bern_n, \\
        \forall n \in \Ne,\ &\int_{0}^{1} \Bern_n(x) \d x = 0.
    \end{align*}
\end{defi}

\begin{defi}
    Pour tout $n \geqslant 0$, on pose $\mathrm{b}_n \defeq \Bern_n(0)$. La suite de réels $(\mathrm{b}_n)_n$ est appelée suite des nombres de \textsc{Bernoulli}.
\end{defi}  

\begin{exercice}
    \begin{enumerate}
        \item Quel est le degré de $\Bern_n(X)$ pour $n \geqslant 0$. 
        \item Montrer que, pour tout $n \geqslant 2$, $\Bern_n(0) = \Bern_n(1)$.
        \item Soient $n \in \N$ et $x \in \R$. Montrer que 
        $$\Bern_n(x) = \sum_{k=0}^n \binom{n}{k} \mathrm{b}_{n-k} x^k.$$
        \item En déduire, pour $n \geqslant 1$, une expression de $\mathrm{b}_n$ en fonction de $\mathrm{b}_0, \dots, \mathrm{b}_{n-1}$.
        \item Montrer que la suite $(\mathrm{b}_n)_n$ est une suite de rationnels et que, pour $n \geqslant 0$, les polynômes $\Bern_n(X)$ sont à coefficients rationnels.
        \item Montrer que pour tout $n \geqslant 0$, $(-1)^n \Bern_n(1-X) = \Bern_n(X)$.
        \item En déduire que 
        $$
        \begin{cases}
            \forall n \geqslant 1, \mathrm{b}_{2n+1} = 0, \\
            \forall n \geqslant 0, \Bern_{2n+1}(\frac{1}{2}) = 0.
        \end{cases}
        $$
    \end{enumerate}    
\end{exercice}


\section{Polynômes de \textsc{Tchebychev}}
\begin{defi}
    Les polynômes de \textsc{Tchebychev} de première espèce sont les uniques polynômes $(\Tcheby_n)_{n \geqslant 0}$ définis sur $]-1, 1[$ par
    $$\forall \theta \in \R,\ \Tcheby_n (\cos \theta) = \cos(n \theta).$$
\end{defi}

\begin{elem_preuve}
    \begin{description}
        \item[Existence] Montrer la relation de récurrence $\Tcheby_{n+2} = 2X \Tcheby_{n+1} - \Tcheby_n$. 
        \item[Unicité] L'unicité de $\Tcheby_n$ pour $n$ fixé est garantie pas l'identification de \ptnclegras{deux polynômes coïncidant} sur $]-1, 1[$. 
    \end{description}
\end{elem_preuve}


\begin{marginfigure}[-8.5cm]
    \centering
	\begin{tikzpicture}
    \begin{axis}[width=6.5cm,
        axis lines=middle,
        grid=major,
        xmin=-1.1, xmax=1.1,
        ymin=-1.1, ymax=1.1,
        % xlabel=$x$, xlabel style={right},
        % ylabel=$y$, ylabel style={above},
        tick style={thick},
        ticklabel style={font=\normalsize},
        xtick={-1, 0, 1}, 
        ytick={-1, 0, 1},
        % legend entries={0.5x},
            legend style={
            at={(1.05,0.4)},
            anchor=north,
            legend columns=1},
            legend cell align={left}
    ]
    
    \def\a{-1.1}
    \def\b{1.1}
    
    \addplot[blue,thick,samples=100,domain=\a:\b] {1};
    \addplot[red,thick,samples=100,domain=\a:\b] {x};
    \addplot[green,thick,samples=100,domain=\a:\b] {2*x^2-1};
    \addplot[violet,thick,samples=100,domain=\a:\b] {4*x^3-3*x};
    \addplot[black,thick,samples=100,domain=\a:\b] {8*x^4-8*x^2+1};
    \addplot[orange,thick,samples=100,domain=\a:\b] {16*x^5-20*x^3+5*x};
    
   % \legend{$\Leg_0$, 
   %         $\Leg_1$,
%            $\Leg_2$,
%            $\Leg_3$,
 %           $\Leg_4$,
  %          $\Leg_5$
   %         }
    \end{axis}
\end{tikzpicture}
	\caption*{\centering Polynômes de \textsc{Tchebychev} de première espèce}
	\begin{align*}
	   	\color{blue} \Tcheby_0 = 1 \\
    	\color{red} \Tcheby_1 = x \\
    	\color{green} \Tcheby_2 = 2x^2-1 \\
    	\color{purple} \Tcheby_3 = 4x^3-3x \\
    	\color{black} \Tcheby_4 = 8x^4-8x^2+1 \\
    	\color{orange} \Tcheby_5 = 16x^5-20x^3+5x
	\end{align*}
\end{marginfigure}


\section{Polynômes scindés}
\begin{box_enonce}
    Soit $Q \in \R[X]$ et $a \in \R$. On suppose que $Q$ est scindé sur $\R[X]$, montrer que $Q'+aQ$ l'est aussi. 
\end{box_enonce}
Si $a=0$, appliquer le \textbf{théorème de \textsc{Rolle}} sur chacun des intervalles $[x_i, x_{i+1}]$ où $x_i$ et $x_{i+1}$ sont deux racines réelles consécutives de $Q$.
\begin{enumerate} 
    \item Poser $\varphi:t \mapsto \exp(at)Q(t)$.
    \item Démarche à revoir.
\end{enumerate}

\section{Équation polynomiale}
\cite{exos_oraux} p. 20 \\
\begin{exercice}
    Déterminer les polynômes $P \in \C[X]$ tels que:
    $$P(X^2) = P(X) P(X-1).$$
\end{exercice}

\chapter{Algèbre linéaire, matrices}
\labch{algebre_lineaire_matrices}

\textsl{Au \textsc{xviii}$^e$ siècle de développent la résolution des systèmes linéaires et la théorie des déterminants. Les raisonnements suggèrent rapidement le concept d'espace à $n$ dimensions. Mais il fallait oser un langage géométrique, alors qu'une interprétation sensible dans le plan ou l'espace faisait défaut pour $n > 3$. \\
De manière indépendante, \textsc{Cayley} en Angleterre et \textsc{Grassman} en Allemagne franchissent le par vers 1843-1845 et parlent d'espace à $n$ dimensions. Le point de vue de \textsc{Cayley} est issu directement de la géométrie analytique: un vecteur d'un espace à $n$ dimensions est un système de $n$ réels ou $n$ complexes. L'addition de deux vecteurs et la multiplication par un scalaire sont naturellement introduites par la généralisation de la dimension $3$. Pour parvenir vraiment à la notion d'espace vectoriel, il faut dégager le concept de sous-espace et de dimension d'un sous-espace. C'est ce que fera \textsc{Grassman} (professeur de lycée autodidacte en marge des milieux de la recherche) en cherchant à développer une analyse géométrique portant sur des calculs intrinsèques indépendants du choix des coordonnées. \textsc{Grassman} introduit le produit extérieur de deux vecteurs, la définition de l'indépendance linéaire, de la dimension d'un espace et démontre la relation fondementale
$$\dim V + \dim W = \dim (V + W) + \dim V \cap W.$$
Ces travaux eurent peu d'impact au début, mais ils furent repris par Henri \textsc{Poincaré} et Élie \textsc{Cartan} (notamment son "algèbre extérieure" en géométrie différentielle). \\
C'est en 1888 que \textsc{Peano} donnera la définition axiomatique d'un espace vectoriel réel. Jusqu'en 1930, le point de vue des matrices et des coordonnées prédomine par rapport au point de vue intrinsèque des espaces vectoriels.
}

\begin{marginfigure}[-13cm]
    \includegraphics{images/arthur_cayley.png}
    \caption{Arthur \textsc{Cayley}}
\end{marginfigure}

\begin{marginfigure}[-7cm]
    \includegraphics{images/hermann_grassmann.png}
    \caption{Hermann \textsc{Grassmann}}
\end{marginfigure}

\section{Produit d'endomorphismes nilpotents qui commutent}
\begin{itemize}
    \item Lemme à démontrer: \\
    si $u$ et $v$ sont deux endomorphismes qui commutent, $v$ étant nilpotent, alors soit $u=0$ soit $\Rg(uv) < \Rg(u)$.\\
    \textcolor{green}{Revoir la démonstration}
\end{itemize}

\section{Nilpotents}
\textcolor{green}{A compléter:} stricte croissance puis stationnarité des noyaux d'un nilpo (cf. corr DL n° 16 Q)8)a)).

\section{Centre de \texorpdfstring{$\M_n(\K)$}{l'espace des matrices carrées}}
\begin{defi}{Centre (algèbre)}
    Le \emph{centre} d'une structure algébrique est l'ensemble des éléments de cette structure qui commutent avec tous les autres éléments. 
\end{defi}

\begin{prop}{Centre de $\M_n(\K)$}
Le centre de $\M_n(\K)$, c'est-à-dire les matrices $A \in \M_n(\K)$ telles que pour toute matrice $B \in \M_n(\K), AB = BA$, est égal à l'ensemble des matrices scalaires.
\end{prop}

Nous voulons démontrer l'égalité de deux ensembles à savoir le centre de $\M_n(\K)$ et $\{ \lambda \I_n, \lambda \in \K \}$. Nous allons donc raisonner par double inclusion. 
\begin{preuve}
    \begin{itemize}
        \item[$(\subset)$] Posons $A \defeq (a_{i,j})_{1 \leqslant 1, j \leqslant n}$. Si la matrice $A$ appartient au centre de $\M_n(\K)$ alors, en particulier, elle commute avec les matrices élémentaires \note i.e. 
        \marginnote[0cm]{
            \begin{kaobox}[frametitle=\note Matrices élémentaires de $\M_{n,p}(\K)$]
                Pour tout $(i, j) \in \llbracket 1, n \rrbracket \times \llbracket 1, p \rrbracket$, on note $\mathrm{E}_{i,j}$ la matrice de taille $(n,p)$ dont tous les coefficients son nuls sauf le coefficient en ligne $i$, colonne $j$, qui est égal à $1$. Autrement dit,
                $$\mathrm{E}_{i,j} = (\delta_{k,i} \times \delta_{\ell, j})_{(k,\ell) \in \llbracket 1, n \rrbracket \times \llbracket 1, p \rrbracket}.$$
                On en déduit que 
                $$\mathrm{E}_{i,j} \times \mathrm{E}_{k, \ell} = \delta_{j,k} \mathrm{E}_{i, \ell}.$$
            \end{kaobox}
        }
        $$\forall (i, j) \in \llbracket 1, n \rrbracket^2, A \mathrm{E}_{i,j} = \mathrm{E}_{i,j} A.$$
        En décompasant la matrice $A$ dans la base des matrices élémentaires on obtient
        $$A \mathrm{E}_{i,j} = \sum_{1 \leqslant k, \ell \leqslant n} a_{k, \ell} \mathrm{E}_{k,\ell} \mathrm{E}_{i,j} = \sum_{k=1}^{n} a_{k,i} \mathrm{E}_{k,j},$$
        et
        $$\mathrm{E}_{i,j} A = \sum_{1 \leqslant k, \ell \leqslant n} a_{k, \ell} \mathrm{E}_{i,j} \mathrm{E}_{k,\ell} = \sum_{\ell=1}^{n} a_{j,\ell} \mathrm{E}_{i,\ell}.$$
        Puisque la famille $(\mathrm{E}_{i, j})_{1 \leqslant i, j \leqslant n}$ est libre, on peut identifier les coefficients des deux expressions et on déduit que pour tout $(i, k) \in \llbracket 1, n \rrbracket^2$ tel que $i \not= k, a_{i,k}=0$ et pour tout $(i,j) \in \llbracket 1, n \rrbracket^2, a_{i,i}=a_{j,j}$. \\
        Ainsi, si la matrice $A$ commute avec toutes les matrices, elle est nécessairement de la forme $A = \lambda \I_n$ où $\lambda \in \K$.
        \item[$(\supset)$] Réciproquement, pour tout $\lambda \in \K$ et toute matrice $B \in \M_n(\K), B \times (\lambda \I_n) = (\lambda \I_n) \times B$. 
    \end{itemize}
    On en déduit par double inclusion que le centre de $\M_n(\K)$ est égal à l'ensemble des matrices scalaires. 
\end{preuve}

\begin{methode}
    Lorsqu'il s'agit de montrer qu'une propriété est vraie \say{ pour toute matrice }, il est parfois utile de prendre des cas particuliers comme les \ptnclegras{matrices élémentaires} pour en déduire des informations sur les coefficients.
\end{methode}


\section{Semblables sur \texorpdfstring{$\C$, sur $\R$}{C, sur R}}
\begin{prop}{}
    Deux matrices réelles semblables dans $\M_n(\C)$ sont semblables dans $\M_n(\R)$.
\end{prop}

\begin{preuve}
    Soient $A$ et $B$ deux matrices réelles semblables dans $\M_n(\C)$. Alors il existe une matrice $P \defeq P_{\mathrm{r}} + \mi P_{\mathrm{i}} \in \Gl_n(\C)$ telle que $AP = PB$ soit $A P_{\mathrm{r}} + \mi A P_{\mathrm{i}} = P_{\mathrm{r}} B + \mi P_{\mathrm{i}} B$ et donc en identifiant parties réelle et imaginaire, $$A P_{\mathrm{r}} = P_{\mathrm{r}} B \text{ et } A P_{\mathrm{i}} = P_{\mathrm{i}} B.$$
    On en déduit que pour tout $x \in \R,\ A(P_{\mathrm{r}} + x P_{\mathrm{i}}) = (P_{\mathrm{r}} + x P_{\mathrm{i}})B$. On pose la fonction 
    $$\delta : z \in \C \mapsto \det(P_{\mathrm{r}} + z P_{\mathrm{i}}).$$ 
    La fonction $\delta$ est polynomiale et est non identiquement nulle car $\delta(\mi) = \det(P) \not=0$. On en déduit qu'il existe un réel $x_0$ tel que $\delta(x_0) = \det(P_{\mathrm{r}} + x_0 P_{\mathrm{i}}) \not=0$. \\
    Ainsi, en posant $\widetilde{P} \defeq P_{\mathrm{r}} + x_0 P_{\mathrm{i}} \in \Gl_n(\R)$ on obtient $A = \widetilde{P}B\Inv{\widetilde{P}}$.
\end{preuve}

\begin{remarque}
    Comme $\R \subset \C$, on en déduit que deux matrices sont semblables dans $\M_n(\C)$ si et seulement si elle le sont dans $\M_n(\R)$.
\end{remarque}

\begin{exercice}
    \marginnote[0cm]{Source : \cite{fmaalouf}}
    Soit $A \in \M_n(\R)$ telle que $\chi_A$ est scindé sur $\R$. Montrer que la matrice $A$ est diagonalisable dans $\M_n(\C)$ si et seulement si elle l'est dans $\M_n(\R)$.
\end{exercice}

\section{Noyaux itérés}
\begin{prop}{}
    Soit $E$ un espace vectoriel de dimension finie $n \in \Ne$. On considère $f \in \Endo(E)$.
    \begin{itemize}
        \item La suite $\left( \Ker(f^k) \right)_{k \in \N}$ est une suite croissante pour l'inclusion et stationnaire à partir d'un certain rang $r \in \llbracket 0, n \rrbracket$.
        \item La suite $\left( \Im(f^k) \right)_{k \in \N}$ est une suite décroissante pour l'inclusion et stationnaire à partir du même rang $r$. \\
        (la suite vient de \href{https://bibmath.net/dico/index.php?action=affiche&quoi=./n/noyauxiteres.html}{Noyaux itérés -- \textsf{Bibm@th.net}}) \\
        De plus
        $$\Ker(f^r) \oplus \Im(f^r) = E.$$
        Si on note $d_k \defeq \dim \big(\Ker(f^k) \big)$, alors pour tout $k \in \N$, 
        $$d_{k+1} - d_k \geqslant d_{k+2} - d_{k+1},$$
        autrement dit la suite de la différence des dimensions entre deux noyaux itérés consécutifs est décroissante. 
    \end{itemize}
\end{prop} 

Voir aussi énoncé de \cite{exos_oraux} p. 44.

\begin{demo}
    \begin{itemize}
        \item La monotonie des deux suites est triviale. 
        \item Bien précisier l'existe de $r$ en dimension finie. \\
        Soit $r$ le rang de stationnarité de la suite des noyaux itérés. Montrons que pour tout $k \in \N, \Ker(f^r) = \Ker(f^{r+k})$. \\
        D'après le premier item, l'une des deux inclusions est vérifiée par croissance de la suite des noyaux itérés. Montrons la deuxième. Soient $k \in \N$ et $x \in \Ker(f^{r+k+1})$. Alors $f^{r+k+1}(x) = f^{r+1} \big(f^k(x) \big) = 0$ soit $f^k(x) \in \Ker(f^{r+1}) = \Ker(f^r)$. Ainsi, $f^{r+k}(x) = 0$ et $x \in \Ker(f^{r+k})$.
        \item Montrons que $\Ker(f^r) \oplus \Im(f^r) = E$. \\
        D'après le théorème du rang, $\Rg(f^r) + \dim \Ker(f^r) = n$. Il reste à montrer que $\Ker(f^r) \cap \Im(f^r) = \{ 0 \}$. \\
        Soit $y \in \Ker(f^r) \cap \Im(f^r)$. Alors il existe $x \in E$ tel que $y = f^r(x)$. De plus, $f^r(y) = 0$ donc, en remplaçant $y$ par son expression, $f^{2r}(x) = 0$ i.e. $x \in \Ker(f^{2r})$, qui est égal à $\Ker(f^r)$ par définition de $r$. On en déduit que $y = f^r(x) = 0$. Ainsi $\Ker(f^r) \cap \Im(f^r) = \{ 0 \}$ et on a bien
        $$\Ker(f^r) \oplus \Im(f^r) = E.$$
    \end{itemize}
\end{demo}

\begin{remarque}
    La propriété de somme directe, n'est plus valable dans le cas d'un espace vectoriel de dimension infinie. En effet, dans $\R[X]$, l'application \emph{dérivée} met en défaut cette égalité. 
\end{remarque}

\begin{exercice}
    \marginnote[0cm]{Source : \cite{maths-france} Planche no 2. Révisions algèbre linéaire. Espaces vectoriels}
    Soient $E$ un espace vectoriel et $f$ un endomorphisme de $E$. Pour $k \in \N$, on pose $N_k \defeq \Ker(f^k)$ et $I_k \defeq \Im(f^k)$ puis $N \defeq \bigcup\limits_{k \in \N} N_k$ et $I \defeq \bigcap\limits_{k \in \N} I_k$. ($N$ est le nilespace de $f$ et $I$ le coeur de $f$).
    \begin{enumerate}
        \item 
        \begin{enumerate}
            \item Montrer que les suites $(N_k)_{k \in \N}$ et $(I_k)_{k \in \N}$ sont respectivement croissante et décroissante pour l'inclusion.
            \item Montrer que $N$ et $I$ sont stables par $f$. 
            \item Montrer que pour tout $k \in \N$, 
            $$(N_k = N_{k+1}) \implies (N_{k+1} = N_{k+2}).$$
        \end{enumerate}
        \item On suppose de plus que $\dim E = n$, $n \in \Ne$.
        \begin{enumerate}
            \item Soit 
            \begin{align*}
                A &\defeq \ens[\big]{ k \in \N \tq N_k = N_{k+1} } \\
                \text{et } B &\defeq \ens[\big]{k \in \N \tq I_k = I_{k+1}}.
            \end{align*}
            Montrer qu'il existe un entier $p$ inférieur à $n$ tel que $A = B =  \{ k \in \N \mid k \geqslant p \}$.
            \item Montrer que $E = N_p \oplus I_p$.
            \item Montrer que $f_{\vert N}$ est nilpotent et que $f_{\vert I} \in \Gl(I)$.
        \end{enumerate}
        \item Trouver des exemples où $A$ est vide et $B$ est non vide et où $A$ est non vide et $B$ est vide.
        \item Pour $k \in \N$, on pose $d_k \defeq \dim I_k$. Montrer que la suite $(d_k - d_{k+1})_{k \in \N}$ est décroissante. En déduire le sens de variation de la suite $\big( \dim N_{k+1} - \dim N_k \big)_{k \in \Ne}$.
    \end{enumerate}
\end{exercice}


\section{Indice de nilpotence en dimension finie} \label{indice_nilpotence}
\begin{box_enonce}
    Soit $E$ un $\K$-espace vectoriel de dimension $n \in \Ne$ et $u \in \Endo(E)$. On suppose que $u$ est nilpotent. \\
    Montrer que $\Id_E - u$ est bijective et déterminer son inverse.
\end{box_enonce}

En particulier, $\boxed{u^n = 0}$.
$$\mathrm{Id}_E = \Id_E - u^n = (\Id_E - u) \circ \left( \sum_{k = 0}^{n-1} u^k \right)$$

\section{Matrices de taille 3 d'ordre de nilpotence égal à 2}
\begin{exercice}
    Déterminer toutes les matrices $M \in \M_3(\K)$ telles que $M^2=0$.
\end{exercice}

\begin{solution}
    \marginnote[0cm]{La correction qui suit est issue de \cite{ellipses}}
    Si $M$ est la matrice nulle, $M$ est solution de l'équation. \\
    Supposons que $M \not= 0$. Soit $\varphi$ l'endomorphisme canoniquement associé à $M$. \\
    Comme $M$ est non nulle, existe donc un vecteur $x$ tel que $\varphi(x) \not= 0$. \\
    Comme $M^2 = 0$, $\Im \varphi \subset \Ker \varphi$. Par le théorème du rang, $\Rg \varphi + \dim \Ker \varphi = 3$ et comme $\Rg \varphi \geqslant 1$ (car $M \not=0$), nécessairement, $\Rg \varphi = 1$ et $\dim \Ker \varphi = 2$. \\
    Comme $\varphi(x) \in \Ker \varphi$, il existe $z \in \Ker \varphi$ tel que la famille $\{ \varphi(x), z \}$ soit une base de $\Ker \varphi$. La famille $\mathscr{F} = \{x, \varphi(x), z \}$ est libre (facile à montrer) et la matrice de $\varphi$ par rapport à cette base est 
    $A = 
    \begin{pmatrix}
    0 & 0 & 0 \\
    1 & 0 & 0 \\ 
    0 & 0 & 0
    \end{pmatrix}
    $. 
    Finalement, les matrices solutions sont
    $$\mathscr{S} = \left \{ \{0\} \cup \{\Inv{P} A P, P \in \Gl_3(\K) \} \right \}.$$
\end{solution}


\section{Applications de \texorpdfstring{$\M_n(\K) \to \K$}{l'espace des matrices carrées dans le corps K} conservant le produit}
\begin{defi}{Forme multiplicative}
    Soit $f$ une application de $\M_n(\K)$ dans $\K$. L'application $f$ est dite \emph{multiplicative} si pour tout matrices $A$ et $B$ de $\M_n(\K)$, $f(AB) = f(A)f(B)$.
\end{defi}

\begin{exercice}
    \marginnote[0cm]{\cite{exos_oraux} p. 53}
    Soit $n \in \Ne$ et $f$ une forme multiplicative de $\M_n(\K)$ dans $\K$ autre que les constantes $0$ et $1$. Montrer que $M \in \Gl_n(\K)$ si et seulement si $f(M) \not= 0$. 
\end{exercice}

\section{Matrices compagnon et commutant d'un cyclique}
\begin{defi}
    Soit $P(X) \defeq X^p + \sum\limits_{k=0}^{p-1} c_kX^k \in \K[X]$. On appelle \emph{matrice compagnon} de $P$ la matrice:
$$ C_P \defeq
\begin{pmatrix}
0 & 0 & \cdots & 0 & -c_0\\
1 & 0 & \cdots & 0 & -c_1\\
0 & 1 & \cdots & 0 & -c_2\\
\vdots & \vdots & \ddots & \vdots & \vdots\\
0 & 0 & \cdots & 1 & -c_{p-1}
\end{pmatrix}.
$$
\end{defi}

\begin{theo}
    Soit $P \in \K[X]$. Le polynôme $P$ est égal au polynôme caractéristique de sa matrice compagnon:
    $$\chi_{C_P}(X) = P(X).$$
\end{theo}   

\marginnote[2cm]{
    \begin{kaobox}[frametitle=Cofacteurs]
    \cite{acamanes} ch4
        Soient $A \in \M_n(\K)$ et $i, j \in \llbracket 1, n \rrbracket$. On note $\Delta_{i,j}$ le déterminant de la matrice de $\M_{n-1}(\K)$ obtenue à partir de la matrice $A$ en supprimant la ligne $i$ et la colonne $j$.
        \begin{itemize}
            \item Le \emph{mineur} d'indice $i,j$ de la matrice $A$ est $\Delta_{i,j}$.
            \item Le \emph{cofacteur} d'indice $i,j$ de la matrice $A$ est $(-1)^{i+j}\Delta_{i,j}$.
        \end{itemize}
    \end{kaobox}
    \begin{kaobox}[frametitle=Développement selon une ligne / colonne]
        Soit $A \defeq (a_{i,j})_{1 \leqslant i, j \leqslant n} \in \M_n(\K)$.
        \begin{itemize}
            \item $\forall j \in \llbracket 1, n \rrbracket, \det(A) = \sum\limits_{i=1}^n a_{i,j}(-1)^{i+j} \Delta_{i,j}$.
            \item $\forall i \in \llbracket 1, n \rrbracket, \det(A) = \sum\limits_{j=1}^n a_{i,j}(-1)^{i+j} \Delta_{i,j}$.
        \end{itemize}
    \end{kaobox}
}

\begin{preuve}
    Par définition,
    $$
    \chi_{C_P}(X) = \det(X \I_p - C_P) = 
    \begin{vmatrix}
        X & 0 & \cdots & 0 & c_0 \\
        -1 & X & & 0 & c_1 \\
        0 & -1 & \ddots & \vdots & \vdots \\
        \vdots & \ddots & \ddots & X & c_{d-2} \\
        0 & \cdots & 0 & -1 & X + c_{d-1}
    \end{vmatrix}.
    $$
    Notons $D_p(X, c_0, \dots, c_{p-1})$ ce déterminant. \\
    Le $(1,1)$-cofacteur de $(X \I_p - C_P)$ est $D_{p-1}(X, c_1, \dots, c_{p-1})$ et son $(1,p)$-cofacteur est $(-1)^{p+1} \delta$ où $\delta$ est le déterminant d'une matrice triangulaire supérieure de taille $d-1$ et dont tous les éléments valent $-1$; ainsi $\delta = (-1)^{p-1}$ et ce $(1,p)$-cofacteur vaut $1$. \\
    Le développement du déterminant $D_p(X, c_0, \dots, c_{p-1})$ par rapport à sa première ligne fournit donc la relation:
    $$D_p(X, c_0, \dots, c_{p-1}) = X D_{p-1}(X, c_1, \dots, c_{p-1}) + c_0.$$
    Comme $D_1(X, c_{p-1}) = X + c_{p-1}$,
    $$\det(X \I_p - C_P) = X \left(X \left(\cdots \left(X(X+c_{p-1}) + c_{p-2} \right) \cdots \right) + c_1 \right) + c_0.$$
    On reconnaît la construction de $P$ par le schéma de \textsc{Horner}. Ainsi le polynôme caractéristique de $C_P$ n'est autre que $P$. 
\end{preuve} 

\begin{defi}
    \marginnote[0cm]{\url{https://bibmath.net/dico/index.php?action=affiche&quoi=./c/cyclique.html}}
    Soit $E$ un $\K$-espace vectoriel de dimension finie $n$ et soit $u$ un endomorphisme de $E$. On dit que $u$ est \emph{cyclique} s'il existe $x \in E$ tel que $(x, u(x), \dots, u^{n-1}(x))$ soit une base de $E$. 
\end{defi}

Les endomorphismes cycliques admettent des matrices particulières dans la base précédente :

\begin{prop}
    Soit $u$ un endomorphisme de $E$. Alors $u$ est cyclique si et seulement s'il existe une base de $E$ dans laquelle la matrice de $u$ est la matrice compagnon de son polynôme caractéristique.
\end{prop}

\begin{prop}
    Soit $f$ un endomorphisme cyclique. Tout endomorphisme qui commute avec $f$ est un polynôme en $f$.
\end{prop}

\begin{preuve}
    
\end{preuve}

\begin{theo}
    Théorème de \textsc{Cayley}-\textsc{Hamilton} \\
    Le polynôme caractéristique est un polynôme annulateur.
\end{theo}

L'exercice suivant, issu du premier sujet de l'agrégation interne de 2022, démontre ce résultat. Cette preuve est basée sur le calcul du polynôme caractéristique d'une matrice compagnon et de l'étude du plus petit sous-espace stabilisé par une matrice et contenant un vecteur donné. \\

\marginnote[-3cm]{Note de \cite{contre-exemples} \\
    En recherchant l'inverse d'un quaternion, William \textsc{Hamilton} démontre, en 1853, le résultat pour la dimension 4 sans vraiment l'exprimer. Arthur \textsc{Cayley} énonce le résultat pour de matrices carrées d'ordre $n$, le démontre pour $n=2$, prétend l'avoir fait pour $n=3$ et dit qu'il ne lui semble pas nécessaire de le démontrer dans le cas général \dots Georg \textsc{Frobenius} fournit la première démonstration génrérale en 1878. 
}

\begin{exercice}
    Soit $p$ un entier strictement positif et soit $M$ une matrice de $\M_p(\C)$.
    \begin{enumerate}
        \item Étant donné un élément $x$ quelconque non nul de $\C^p$ on pose
        $$\mu \defeq \min \{ r \geqslant 1\ | (x, Mx, \dots, M^r x) \text{ est liée dans } \C^p\}.$$
        \item Montrer qu'il existe un élément $(\alpha_0, \dots, \alpha_{\mu-1}$ de $\C^{\mu}$ et une matrice $N$ de $\M_{p-\mu}(\C)$ tels que la matrice $M$ soit semblable à une matrice $M'$ de la forme suivante
        $$
        \begin{pmatrix}
        0 & \cdots & \cdots & 0 & -\alpha_0 & \star \\
        1 & 0 & & \vdots & -\alpha_1 & \star \\
        0 & 1 & \ddots & \vdots & \vdots & \vdots \\
        \vdots & \ddots & \ddots & 0 & -\alpha_{\mu-2} & \star \\
        0 & \cdots & 0 & 1 & -\alpha_{\mu-1} & \star \\
        O & \cdots & \cdots & O & O & N
        \end{pmatrix}
        $$
        où les $\star$ représentent des lignes d'éléments de $\C$ et les $O$ représentent des colonnes nulles. 
        \item Montrer que $\chi_M(M)x = 0$.
        \item Montrer que $\chi_M$ est un polynôme annulateur de $M$.
    \end{enumerate}
\end{exercice}

\begin{preuve}
    \marginnote[0cm]{Correction de la RMS 132 3}
\end{preuve}


\section{Caractérisation des homothéties}
\begin{tcolorbox}
    Soit $E$ un espace vectoriel et $f \in \Endo(E)$. Alors $f$ est une homothétie si et seulement si $(x, f(x))$ liée pour tout $x \in E$.
\end{tcolorbox}

\begin{itemize}
    \item ($\Leftarrow$) Poser $f:x \mapsto \lambda_x x$. Soient $x_0 \in E$ non nul et $x \in E$. Distinguer les cas où $x \in \Vect(x_0)$ et $x \not \in \Vect(x_0)$.
\end{itemize}

\section{Famille libre « engendrée » par un endomorphisme nilpotent}
Soit $E$ un espace vectoriel de dimension finie $n$ non nulle et $\varphi \in \mathscr{L}(E)$ un endomorphisme nilpotent d'indice de nilpotence égal à $p$. Montrer qu'il existe un vecteur $x_0 \in E$ tel que la famille $\mathscr{F}=(x_0, \varphi(x_0), \dots, \varphi^{p-1}(x_0))$ soit une famille libre. Montrer aussi que $p \leqslant n$. 

\begin{itemize}
    \item Première question:
    \begin{enumerate}
        \item Justifier l'existence de $x_0$.
        \item Revenir à la définition d'une famille libre et supposer par l'absurde que les $\lambda_i$ ne sont pas tous nuls. 
        \item En déduire une absurdité en composant l'expression par un endomorphisme bien choisi.
    \end{enumerate}
    \item Deuxième question:\\
        D'après le théorème de \textsc{Cayley}-\textsc{Hamilton}, $\chi_{\varphi}(\varphi) = 0$. \\
        Or comme $\varphi$ est nilpotent, $\Sp(\varphi) = \{0\}$. Donc $\chi_{\varphi}(\lambda) = \lambda^n$ et $\varphi^n = 0_{\Endo(E)}$. Enfin, l'ordre de nilpotence de $\varphi$ étant égal à $p$, $p \leqslant n$. 
\end{itemize}

\section{Polynômes de \textsc{Hilbert}} \label{polynome_hilbert}
\begin{prop}[$(\Hilb_i)_{0 \leqslant i \leqslant n}$ forme une base de $\C_n{[X]}$] \labprop{hilbert_base}
    La famille $(\Hilb_0, \dots, \Hilb_n)$ des $n+1$ premiers polynômes de \nom{Hilbert} forme une base de $\C_n[X]$.
\end{prop}

\begin{lemme} \lablemme{famille_deg_echelonnes_est_libre}
    Toute famille de polynômes non nuls à degrés échelonnés est libre.
\end{lemme}

\begin{demo}
    \source{\href{https://www.bibmath.net/ressources/justeunexo.php?id=815}{Polynômes à degrés échelonnés -- \textsf{Bibm@th.net}}}
    Soit $(P_1, \dots, P_n)$ une famille de polynômes de $\C_n[X]$ non nuls, à degrés échelonnés, i.e. $0 \leqslant \deg P_1 < \cdots < \deg P_n$. \\
    Soit $(\lambda_1, \dots, \lambda_n)$ des scalaires tels que
    \begin{equation}\tag{$\star$} \label{eq}
        \lambda_1 P_1 + \cdots + \lambda_n P_n = 0.
    \end{equation}
    Supposons par l'absurde que $\lambda_n \not= 0$. Alors le membre de gauche de l'égalité (\ref{eq}) est un polynôme de degré $\deg P_n \not= - \infty$ puisque tous les polynômes sont supposés non nuls. Ce membre ne peut donc pas être le polynôme nul. On aboutit à une contradiction et $\lambda_n = 0$. \\ 
    En itérant le raisonnement, on trouve successivement 
    $$\lambda_{n-1} = 0, \dots, \lambda_1 = 0$$
    ce qui assure la liberté de la famille $(P_1, \dots, P_n)$.
\end{demo}

\marginnote[-3cm]{
    \begin{methode}
        Penser au \reflemme{famille_deg_echelonnes_est_libre} pour montrer la liberté d'une famille de polynômes. 
    \end{methode}
}

Revenons à la démonstration de la \refprop{hilbert_base}.

\begin{demo}
    \marginnote[0cm]{
        \note Par définition, 
        $$\Hilb_0 \defeq 1$$
        $$\forall n \in \Ne,\ \Hilb_n \defeq \frac{1}{n!}X(X-1)\cdots(X-n+1).$$
        Donc pour tout $n \in \N$, $\deg \Hilb_n = n$.
    }
    Par construction, la famille $\mathscr{H} \defeq (\Hilb_0, \dots, \Hilb_n)$ est échelonnée en dégré \note ce qui assure sa liberté d'après le \reflemme{famille_deg_echelonnes_est_libre}. De plus, $|\mathscr{H}| = \dim \C_n[X]$ donc la famille $\mathscr{H}$ forme bien une base de $\C_n[X]$.
\end{demo}

\section{Polynômes de \textsc{Lagrange}} 
\marginnote[0cm]{\url{https://perso.math.univ-toulouse.fr/fdelebec/files/2018/03/chap01-L2.pdf}}
\subsection{Motivations de l'interpolation polynomiale}
En analyse numérique, une fonction $f$ inconnue explicitement est souvent connue seulement en certains points $x_0, \dots, x_d$, ou évaluable uniquement au moyen de l'appel à un code coûteux. \\
Mais dans de nombreux cas, on a besoin d'effectuer des opérations (dérivation, intégration, \dots) sur la fonction $f$. \\
On cherche donc à reconstruire cette fonction $f$ par une autre fonction $f_r$ simple et facile à évaluer à partir des données discrètes de $f$. On espère que le modèle $f_r$ ne sera pas trop éloigné de la fonction $f$ aux autres points. \\

Pourquoi utiliser des polynômes pour reconstruire la fonction $f$ ?

\begin{enumerate}
    \item Théorème d'approximation de \textsc{Weierstrass} (\textcolor{red}{lien}): pour toute fonction $f$ définie et continue sur un intervalle $[a, b]$ et pour tout $\varepsilon > 0$, il existe un polynôme $P$ tel que 
    $$\forall x \in [a, b],\ |f(x) - P(x)| < \varepsilon.$$
    Plus $\varepsilon$ est petit, plus le degré du polynôme est grand.
    \item La simplicité de l'évaluation d'un polynôme par le schéma de \textsc{Hörner}:
    $$\sum_{j=0}^n c_j x^j = \Big( \cdots \big( (c_n x + c_{n-1})x + c_{n-2} \big)x + \cdots c_1 \Big)x + c_0.$$
\end{enumerate}

\begin{marginfigure}[-1cm]
    \centering
    \begin{tikzpicture}
    \begin{axis}[width=6.5cm,
        axis lines=middle,
        inner axis line style={-latex},
        grid=major,
        xmin=-1.2, xmax=1.2,
        ymin=-1.1, ymax=1.1,
        % xlabel=$x$, xlabel style={right},
        % ylabel=$y$, ylabel style={above},
        %tick style={thick},
        %ticklabel style={font=\normalsize},
        xtick=\empty, 
        ytick=\empty,
        axis line style={-latex}
    ]
    
    \def\a{-1.1}
    \def\b{1.1}
    \def\colour{BrickRed}
    
    \addplot[red,thick,samples=100,domain=\a:\b] {
    (x+5/8)*(x-1/8)*(x-1/2)*(x-4/5)/((-7/8+5/8)*(-7/8-1/8)*(-7/8-1/2)*(-7/8-4/5))*(-1/2)
    + (x+7/8)*(x-1/8)*(x-1/2)*(x-4/5)/((-5/8+7/8)*(-5/8-1/8)*(-5/8-1/2)*(-5/8-4/5))*(-1/9)
    + (x+7/8)*(x+5/8)*(x-1/2)*(x-4/5)/((1/8+7/8)*(1/8+5/8)*(1/8-1/2)*(1/8-4/5))*(-1/8)
    + (x+7/8)*(x+5/8)*(x-1/8)*(x-4/5)/((1/2+7/8)*(1/2+5/8)*(1/2-1/8)*(1/2-4/5))*(1/2)
    + (x+7/8)*(x+5/8)*(x-1/8)*(x-1/2)/((4/5+7/8)*(4/5+5/8)*(4/5-1/8)*(4/5-1/2))*(2/9)
    };
    
    \addplot[\colour,mark=*] coordinates {(-7/8,-1/2)} node[left] {$M_0$};
    \addplot[\colour,mark=*] coordinates {(-5/8,-1/9)} node[below] {$M_1$};
    \addplot[\colour,mark=*] coordinates {(1/8,-1/8)} node[below] {\contour{white}{$M_2$}};
    \addplot[\colour,mark=*] coordinates {(1/2,1/2)} node[above] {\contour{white}{$M_3$}};
    \addplot[\colour,mark=*] coordinates {(4/5,2/9)} node[right] {$M_4$};
    
    \draw[blue, thick, dotted] (-7/8,-1/2) -- (-7/8, 0);
    \draw[blue, thick, dotted] (-7/8,-1/2) -- (0, -1/2);

    \draw[blue, thick, dotted] (-5/8,-1/9) -- (-5/8, 0);
    \draw[blue, thick, dotted] (-5/8,-1/9) -- (0, -1/9);

    \draw[blue, thick, dotted] (1/8,-1/8) -- (1/8, 0);
    \draw[blue, thick, dotted] (1/8,-1/8) -- (0,-1/8);

    \draw[blue, thick, dotted] (1/2,1/2) -- (1/2, 0) node[below] {$a_3$};
    \draw[blue, thick, dotted] (1/2,1/2) -- (0, 1/2) node[left] {$f_3$};
    
    \draw[blue, thick, dotted] (4/5,2/9) -- (4/5, 0);
    \draw[blue, thick, dotted] (4/5,2/9) -- (0, 2/9);
    
    \draw[black, thick] (-0.8,0.5) node[above] 
    {\footnotesize \contour{white}{{\parbox{2cm}{\centering Polynôme \\ interpolateur}}}} to [out=640,in=800] ($(-0.3,-1/4)$);
    \end{axis}
    
\end{tikzpicture}
\end{marginfigure}

Plus précisément, étant donnés $d+1$ points d'abscisses distinctes $M_i \defeq (a_i, f_i)$ pour $i \in \llbracket 0, d \rrbracket$ dans le plan, le problème de l'interpolation polynomiale consiste à trouver un polynôme de degré inférieur ou égal à $m$ dont le graphe passe par les $d+1$ points $M_i$.\\

\subsection{Interpolation lagrangienne}

Les polynômes de \textsc{Lagrange} permettent d'interpoler une série de $n+1$ points par un polynôme de degré $n$ qui passe exactement par ces points.

\begin{theo}{}
    Soit $n \in \N$. On considère $n + 1$ complexes deux à deux distincts, notés $x_0, \dots, x_n$. \\
    Pour tout $i \in \llbracket 0, n \rrbracket$, il existe un unique polynôme $\Lag_i \in \C_n[X]$ tel que 
    $$\forall j \in \llbracket 0, n \rrbracket,\ \Lag_i(x_j) = \delta_{i,j}.$$
    De plus,
        $$\forall (i, j) \in \llbracket 1, n \rrbracket^2,\ \Lag_i = \prod_{j \neq i} \frac{X-x_j}{x_i - x_j}.$$
\end{theo}

\marginnote[-4cm]{
    \begin{kaobox}[frametitle=Symbole de \textsc{Kronecker}]
    $$
    \delta_{i,j} \defeq \begin{cases}
    1 \quad \text{ si } i=j, \\
    0 \quad \text{ sinon}.
    \end{cases}
    $$
    \end{kaobox}
}

Voyons deux démonstrations. La première procède par construction et explicite la forme des polynômes de \textsc{Lagrange}, la deuxième passe par les propriétés des applications linéaires.
\begin{preuve}
    \marginnote[0cm]{\cite{maths-france}}
    \begin{itemize}
        \item[$\rhd$] On considère $n + 1$ complexes deux à deux distincts, notés $x_0, \dots, x_n$. \\
        Soit $i \in \llbracket 1, n \rrbracket$. Le polynôme $\Lag_i$ est de degré au plus $n$ et admet $n$ complexes deux à deux distincts $x_j$ pour racines, avec $j \not= i$. Alors, nécessairement, il existe une constante $C$ telle que 
        $$\Lag_i = C \prod_{i \not= j} (X-x_j).$$
        L'égalité $\Lag_i(x_i) = 1$ fournit $C = \left[ \prod\limits_{j \not=i}(x_i - x_j) \right]^{-1}$ et donc 
        $$\Lag_i = \prod_{j \neq i} \frac{X-x_j}{x_i - x_j}.$$
        \item[$\rhd$] Réciproquement, si pour tout $i \in \llbracket 0, n\rrbracket$ on pose $\Lag_i \defeq \prod\limits_{j \neq i} \frac{X-x_j}{x_i - x_j}$, alors le polynôme $\Lag_i$ est bien défini car les $x_j$ sont deux à deux distincts, est bien de degré $n$ et enfin les polynômes $\Lag_i$ vérifient clairement les égalités de dualité.
    \end{itemize}
\end{preuve}
\begin{preuve}
    \marginnote[0cm]{\url{https://www.youtube.com/watch?v=blB2SAYpobA}}
    On considère $n + 1$ complexes deux à deux distincts, notés $x_0, \dots, x_n$.
    \begin{alignat*}{2}
        \text{Soit } \varphi\ :\ \R_n[X]\ &\longrightarrow\ \R^{n+1}\\
        P\ &\longmapsto\ \big(P(x_0), \dots, P(x_n) \big).
    \end{alignat*}
    \begin{itemize}
        \item[$\rhd$] Soit $P \in \Ker \varphi$. Alors le polynôme $P$ a $n+1$ racines discintes. Or il est de degré inférieur à $n$ donc est le polynôme nul. On en déduit que $\varphi$ est injective ce qui assure l'\ptnclegras{unicité} des polynômes interpolateurs de \textsc{Lagrange}.
        \item[$\rhd$] Comme $\dim \R_n[X] = \dim \R^{n+1}$ et que l'application $\varphi$ est injective, c'est un isomorphisme \note.
        \item[$\rhd$] En particulier, l'application $\varphi$ est surjective ce qui assure l'\ptnclegras{existence} de ces polynômes. 
    \end{itemize}
    \marginnote[-4cm]{
        \begin{prop}
            \note Si $f$ est une application linéaire d'un espace de dimension finie $E$ dans un espace de dimension finie $F$ avec $\dim(E) = \dim(F)$ pour que $f$ soit un isomorphisme, il suffit que $f$ soit injective ou que $f$ soit surjective.
        \end{prop}
    }
\end{preuve}

\subsection{Coordonées d'un polynôme dans la base de \textsc{Lagrange}}

\begin{prop}
    La famille $(\Lag_0, \dots, \Lag_n)$ des $n+1$ premiers polynômes de \textsc{Lagrange} forme une base de $\C_n[X]$.
\end{prop}

\begin{preuve} 
    Par construction, la famille $\mathscr{L} \defeq (\Lag_0, \dots, \Lag_n)$ est échelonnée en dégré ce qui assure sa liberté d'après le \reflemme{famille_deg_echelonnes_est_libre}. De plus, $|\mathscr{L}| = \dim \C_n[X]$ donc la famille $\mathscr{L}$ forme bien une base de $\C_n[X]$.
\end{preuve}

\begin{prop}
Soit $(x_0, \dots, x_n)$, $n+1$ complexes deux à deux distincts et $(y_0, \dots, y_n)$, $n+1$ complexes. Il existe un et un seul polynôme $P \in \R_n[X]$ tel que 
$$\forall i \in \llbracket 0, n \rrbracket, P(x_i) = y_i.$$ 
Ce polynôme à pour expression dans la base de \textsc{Lagrange}
$$P = \sum_{i=0}^n y_i \Lag_i.$$
\end{prop}

\section{Polynômes d'interpolation de \textsc{Lagrange}, lien avec les déterminants de \textsc{Vandermonde}}
\cite{maths-france} \\
En appliquant la formule des coordonnées d'un polynôme de degré au plus $n$ dans la base $(\Lag_i)_{i \in \llbracket 0, n \rrbracket}$ au cas particulier où le polynôme $P$ est l'un des éléments de la base canonique $(X^j)_{j \in \llbracket 0, n \rrbracket}$ de $\C_n[X]$, on obtient $\sum\limits_{i=0}^{n} \Lag_i = 1$ et plus généralement, 
$$\boxed{\forall j \in \llbracket 0, n \rrbracket,\ X^j = \sum_{i=0}^{n} x_i ^j \Lag_i}.$$
Ainsi, 
\begin{prop}
    La matrice de passage de la base  $(\Lag_i)_{i \in \llbracket 0, n \rrbracket}$ à la base  $(X^j)_{j \in \llbracket 0, n \rrbracket}$ est la matrice de \textsc{Vandermonde} associée à la famille $(x_i)_{i \in \llbracket 0, n \rrbracket}$.
\end{prop}

\chapter{Déterminants}
\labch{determinants}

\textsl{L'utilisation des matrices et des déterminants trouve son origine dans l'étude systématique des systèmes linéaires menée à partir du \textsc{xvii}$^\e$ siècle. Alors que \textsc{Leibniz} et \textsc{Mac Laurin} avaient déjà introduit les notations à indices et résolu les systèmes à deux ou trois inconnues \textsc{Cramer}, en 1754, comprend que les solutions d'un systèmes linéaire s'expriment comme le quotient de deux expressions polynomiales multilinéaires des coefficients du système. Ces expressions représentent des déterminants mais ces derniers, étudiés notamment par \textsc{Vandermonde} et \textsc{Laplace} ne sont définis alors que par récurrence sur la taille (autrement dit par le développement par rapport à une rangée). On doit également à \textsc{Laplace} l'interprétation du déterminant en termes de volume. Par le suite, au début du \textsc{xix}$^\e$ siècle \textsc{Gauss}, dans ses recherches sur les formes quadratiques, représente les changements de base dans $\R^3$ à l'aide de tableaux de nombres (les matrices) et introduit le produit de deux de ces tableaux pour obtenir la composée de deux changements de bases. Cela devait suggérer en 1812 à \textsc{Cauchy} la règle générale du produit de deux déterminants; il lui revient d'imposer la terminologie moderne.} \\

\cite{objectif_agregation} p. 184.

\begin{marginfigure}[-7.6cm]
    \caption*{\centering Interprétation géométrique du déterminant en dimension $2$}
   \begin{tikzpicture}[
    label/.style={black},
    vector/.style={ultra thick,-latex}
  ]

  \def\xmin{-1} \def\xmax{5}
  \def\ymin{-1} \def\ymax{5}
  \def\deltax{0.4} \def\deltay{0.3}
  \def\gridscale{3}
  
  \def\spacing{0.15} 

  \def\colTrigHB{red} \def\colTrigGD{blue} \def\colRect{green}
  \def\opac{0.3}
  \def\colx{violet} \def\coly{cyan}

  \begin{scope}
    \coordinate (origin) at (0,0);
    
    % repère droit
    % \draw [very thick,->] (\xmin,0) -- (\xmax,0);
    % \draw [very thick,->] (0,\ymin) -- (0,\ymax);
    \clip [draw] (\xmin,\ymin) rectangle (\xmax,\ymax);
    
    % quadrillage droit
    \draw[style=help lines] (\xmin-\xmax,\ymin-\ymax) grid[step=\deltay/\deltax] (-\xmin+\xmax,-\ymin+\ymax);
    
    % triangle haut
    \filldraw [\colTrigHB!50, fill opacity=\opac] (\deltax * \gridscale, \gridscale) -- (\gridscale + \deltax * \gridscale, \gridscale + \deltay * \gridscale) -- (\deltax * \gridscale, \gridscale + \deltay * \gridscale) -- cycle;
    
    % triangle bas
    \filldraw [\colTrigHB!50, fill opacity=\opac] (origin) -- (\gridscale, \deltay * \gridscale) -- (\gridscale, 0) -- cycle;
    
    % triangle gauche
    \filldraw [\colTrigGD!50, fill opacity=\opac] (origin) -- (0, \gridscale) -- (\deltax * \gridscale, \gridscale) -- cycle;
    
    % triangle droit
    \filldraw [\colTrigGD!50, fill opacity=\opac] (\gridscale, \deltay * \gridscale) -- (\gridscale + \deltax * \gridscale, \gridscale + \deltay * \gridscale) -- (\gridscale + \deltax * \gridscale, \deltay * \gridscale) -- cycle;
    
    % rectangle haut gauche
    \filldraw[\colRect!50,fill opacity=\opac] (\deltax * \gridscale, \gridscale) rectangle (0, \gridscale + \deltay * \gridscale);
    
    % rectangle bas droite
    \filldraw[\colRect!50,fill opacity=\opac] (\gridscale, 0) rectangle (\gridscale + \deltax * \gridscale, \deltay * \gridscale);
    
    % curly brackets
    \draw [thick, decorate,
    decoration = {calligraphic brace, mirror, raise=1pt, amplitude=5pt}] (origin) --  (\gridscale, 0) node[pos=0.5pt,below=5pt,black]{$\textcolor{\colx}{a}$};
    
    \draw [thick, decorate,
    decoration = {calligraphic brace, mirror, raise=1pt, amplitude=5pt}] (\gridscale, 0) -- (\gridscale + \deltax * \gridscale, 0) node[pos=0.5pt,below=5pt,black]{$\textcolor{\coly}{c}$};
    
    \draw [thick, decorate,
    decoration = {calligraphic brace, mirror, raise=1pt, amplitude=5pt}] (\gridscale + \deltax * \gridscale, 0) --  (\gridscale + \deltax * \gridscale, \deltay * \gridscale) node[pos=0.5pt,right=5pt,black]{$\textcolor{\colx}{b}$};
    
    \draw [thick, decorate,
    decoration = {calligraphic brace, mirror, raise=1pt, amplitude=5pt}] (\gridscale + \deltax * \gridscale, \deltay * \gridscale) -- (\gridscale + \deltax * \gridscale, \gridscale + \deltay * \gridscale) node[pos=0.5pt,right=5pt,black]{$\textcolor{\coly}{d}$};
    
    % aires
    \node at (\gridscale + \deltax * \gridscale / 2, \deltay * \gridscale / 2) {$\textcolor{black}{bc}$};
    \node at (\gridscale + \deltax * \gridscale / 1.5, \gridscale / 2 + \deltay * \gridscale / 2) {$\displaystyle \frac{dc}{2}$};
    \node at (\gridscale / 1.5, \deltay * \gridscale / 3) {$ab/2$};
    \node at (\gridscale / 2 + \deltax * \gridscale / 2, \gridscale / 2 + \deltay * \gridscale / 2) {\Huge $\mathcal{A}$};


    \pgftransformcm{1}{\deltay}{\deltax}{1}{\pgfpoint{0}{0}}

    % quadrillage oblique
    \draw[style=help lines,dashed] (\xmin-\xmax,\ymin-\ymax) grid[step=\gridscale] (-\xmin+\xmax,-\ymin+\ymax);

    % noeuds
    \foreach \x in {\xmin,...,\xmax}{
        \foreach \y in {\ymin,...,\ymax}{
            \node[draw,circle,inner sep=1pt,fill] at (\gridscale*\x,\gridscale*\y) {};
          }
      }
      
    \filldraw[fill=yellow,fill opacity=\opac] (origin) rectangle (\gridscale,\gridscale);

    \draw [vector, \colx] (origin) -- (\gridscale,0) node [label,right=0] {$\textcolor{\colx}{\vec{x}}$};
    \draw [vector, \coly] (origin) -- (0,\gridscale) node [label,above=0] {$\textcolor{\coly}{\vec{y}}$};
    
    % je rajoute le point de l'origine
    \node[draw,circle,inner sep=1pt,fill] at (origin) {};
    \end{scope}
\end{tikzpicture}
\end{marginfigure}

\marginnote[-1cm]{
    $$
    \textcolor{violet}{\vec{x} = 
    \begin{pmatrix} 
        a \\ 
        b 
    \end{pmatrix}}
    \text{ et }
    \textcolor{cyan}{\vec{y} = 
    \begin{pmatrix} 
        c \\ 
        d 
    \end{pmatrix}}
    $$
    \begin{align*}
        \mathcal{A} &= (a + c)(b + d) - 2 \left( \frac{ab}{2} + bc + \frac{dc}{2} \right) \\
        &= ad - bc \\
        \mathcal{A} &= 
        \begin{vmatrix}
            \textcolor{violet}{a} & \textcolor{cyan}{c} \\
            \textcolor{violet}{b} & \textcolor{cyan}{d}
        \end{vmatrix}
    \end{align*}
}

\newpage

\section{Déterminant tridiagonal}
\begin{itemize}
    \item Avoir en tête la forme générale de la relation de récurrence du déterminant: 
    $$\boxed{D_n = aD_{n-1} - bc D_{n-2}}$$
    où $a$ constitue la diagonale principale et $b$ et $c$ les diagonales secondaires. 
    \item L'exercice 8 du chapitre 4 est un bon entraînement: pour tout $x \in \R$, déterminer:
    $$
        A_n(x)=\begin{vmatrix}
            2x & 1 & & 0\\
            1 & 2x & \ddots\\
             & \ddots & \ddots & 1\\
             0 & & 1 & 2x
        \end{vmatrix}.
    $$    
\end{itemize}

\section{Matrice circulante}
\begin{tcolorbox}
Une \href{https://fr.wikipedia.org/wiki/Matrice_circulante}{\emph{matrice circulante}} est une matrice carrée dans laquelle on passe d'une ligne à la suivante par permutation circulaire des coefficients:
$$
\mathrm{C}(c_0, \dots, c_{n-1})=
\begin{pmatrix}
c_0 & c_1 & c_2 & \cdots & c_{n-1} \\
c_{n-1} & c_0 & c_1 & \cdots & c_{n-2} \\
c_{n-2} & c_{n-1} & c_0 & \cdots & c_{n-3} \\
\vdots & \vdots & \vdots & \ddots & \vdots \\
c_1 & c_2 & c_3 & \cdots & c_0
\end{pmatrix}.
$$
\end{tcolorbox}

\begin{remarque}
    Une matrice circulante est un cas particulier de \href{https://fr.wikipedia.org/wiki/Matrice_de_Toeplitz}{matrice de \textsc{Toeplitz}}.
\end{remarque}

\begin{prop}
On pose $\omega = \me^{\mi \frac{2 \pi}{n}}$.
    $$\Sp(\mathrm{C}(c_0, \dots, c_{n-1})) = \left \{ \sum_{j=0}^{n-1} c_j \omega^j,\ k \in \llbracket 1, n-1 \rrbracket \right \}.$$
\end{prop}

\begin{preuve}
    Pour alléger les notations, on pose $\mathrm{C} := \mathrm{C}(c_0, \dots, c_{n-1})$. \\
    On pose 
    $$
    \mathrm{J} = 
    \begin{pmatrix}
    0 & 1 & 0 & \cdots & 0 \\
    \vdots  &   & \ddots & \\
    0 & & & & 1 \\
    1 & 0 & \cdots & \cdots & 0
    \end{pmatrix}
    $$
    
    (faire une remarque sur la structure de la matrice J et que J$^n = \I_n$) 
    
    $$\mathrm{C} = \sum_{k=0}^{n-1} c_k \mathrm{J}^k = \mathrm{P}_{\mathrm{C}}(\mathrm{J}).$$
    
    En développant par rapport à la première colonne, 
    \begin{align*}
        \chi_{\mathrm{J}}(X) &= 
        \begin{vmatrix}
            X & -1 & 0 & \cdots & 0 \\
            \vdots  &   & \ddots & \\
            0 & & & & -1 \\
            -1 & 0 & \cdots & \cdots & X
        \end{vmatrix} \\
        &= X \times X^{n-1} + (-1) \times (-1)^{n+1} \times (-1)^{n-1} \\
        \chi_{\mathrm{J}}(X) &= X^n-1. 
    \end{align*}
    Le polynôme caractéristique de $\mathrm{J}$ est scindé à racines simples sur $\C$ donc $\mathrm{J}$ est diagonalisable et en posant $\omega = \me^{\mi \frac{2 \pi}{n}}$, 
    $$\Sp(\mathrm{J}) = \mathbb{U}_n = \left \{ \omega^k,\ k \in \llbracket 0, n-1 \rrbracket \right \}.$$
    Ainsi, $\mathrm{J}$ est semblable à la matrice 
    $$
    \begin{pmatrix}
    1 & & & & \\
    & \omega & & & \\
    & & \omega^2 & & \\
    & & & \ddots & \\
    & & & & \omega^{n-1} \\
    \end{pmatrix}
    $$
    et donc comme $\mathrm{C} = \mathrm{P}_{\mathrm{C}}(\mathrm{J})$, la matrice $\mathrm{C}$ est semblable à  la matrice
    $$
    \begin{pmatrix}
    \mathrm{P}(1) & & & & \\
    & \mathrm{P}(\omega) & & & \\
    & & \mathrm{P}(\omega^2) & & \\
    & & & \ddots & \\
    & & & & \mathrm{P}(\omega^{n-1}) \\
    \end{pmatrix}.
    $$
    Ainsi, $\Sp(\mathrm{C}) = \left \{ \mathrm{P}(\omega^k),\ k \in \llbracket 0, n-1 \rrbracket \right \}$.
\end{preuve}

\begin{corol}
    $$\det(\mathrm{C}(c_0, \dots, c_{n-1})) = \prod_{j=0}^{n-1} \left( \sum_{k=0}^{n-1} c_k \exp \left( \mi \frac{2kj \pi}{n} \right) \right).$$
\end{corol}

\begin{preuve}
    Le déterminant d'une matrice est égal au produit de ses valeurs propres.
\end{preuve}


\section{Déterminant des \texorpdfstring{$|a_i - a_j|$}{|a_i - a_j|}}
\begin{exercice}
    \marginnote[0cm]{Source : \cite{exos_oraux} p. 71}
    Soient $n \geqslant 2$ et $(a_0, \dots, a_n) \in \R^{n+1}$. Calculer le déterminant de la matrice dont l'élément ligne $i$, colonne $j$ est $|a_{i-1} - a_{j-1}|$.
\end{exercice}

\section{Déterminant de \textsc{Vandermonde}, applications}
\begin{defi}
    Soit $(\alpha_1, \dots, \alpha_n)$ une famille de complexes. On définit la \emph{matrice de \textsc{Vandermonde}} de la famille $(\alpha_1, \dots, \alpha_n)$ par
    $$\Vandermonde(\alpha_1, \dots, \alpha_n) \defeq \begin{pmatrix}
    1 & \alpha_1 & \alpha_1^2 & \cdots & \alpha_1^{n-1} \\
    \vdots & \vdots & \vdots & \ddots & \vdots \\
    1 & \alpha_n & \alpha_n^2 & \cdots & \alpha_n^{n-1}
    \end{pmatrix}.$$
\end{defi}

\newcommand{\vandk}{
\left(\begin{gathered}
    \tikzpicture[every node/.style={anchor=south west}]
        \node[minimum width=2cm,minimum height=1.5cm] at (-0.16,0.625) {$\Vandermonde(\alpha_1, \dots, \alpha_{k-1})$};
        \node[minimum width=0.5cm,minimum height=0.5cm] at (0,0.05) {$1$};
        \node[minimum width=0.5cm,minimum height=0.5cm] at (0.4, 0) {$\alpha_{k}$};
        \node[minimum width=0.5cm,minimum height=0.5cm] at (0.95,0) {$\cdots$};
        \node[minimum width=0.5cm,minimum height=0.5cm] at (1.5,0) {$\alpha_{k}^{k-2}$};
        \node[minimum width=1cm,minimum height=0.5cm] at (2.5,0) {$\alpha_{k}^{k-1}$};
        \node[minimum width=1cm,minimum height=1cm] at (2.5,0.4) {$\alpha_{k-1}^{k-1}$};
        \node[minimum width=1cm,minimum height=1cm] at (2.5,1) {$\vdots$};
        \node[minimum width=1cm,minimum height=1cm] at (2.5,1.4) {$\alpha_1^{k-1}$};
        \draw (0,0.6) -- (2.5,0.6);
        \draw (2.5,0.6) -- (2.5,2.225);
        \draw[dashed] (2.5,0.6) -- (3.5,0.6);
        \draw[dashed] (2.5,0.6) -- (2.5,0);
    \endtikzpicture
    \end{gathered}\right)
}

\begin{remarque}
    Soit $(\alpha_1, \dots, \alpha_n)$ une famille de complexes. Les matrices de \textsc{Vandermonde} des familles $(\alpha_1, \dots, \alpha_k)$ pour $2 \leqslant k \leqslant n$ sont imbriquées les unes dans les autres de la manière suivante
    $$\Vandermonde(\alpha_1, \dots, \alpha_k) = \vandk.$$
\end{remarque}

\begin{prop}
    $$\det (\Vandermonde(\alpha_1, \dots, \alpha_n)) = \prod_{1\leqslant i < j \leqslant n}(\alpha_j - \alpha_i).$$
\end{prop}

\newcommand{\detvandnplusun}{
\left|\begin{gathered}
    \tikzpicture[every node/.style={anchor=south west}]
        \node[minimum width=2cm,minimum height=1.5cm] at (-0.16,0.625) {$\Vandermonde(\alpha_1, \dots, \alpha_n)$};
        \node[minimum width=0.5cm,minimum height=0.5cm] at (0,0.05) {$1$};
        \node[minimum width=0.5cm,minimum height=0.5cm] at (0.4, 0) {$\alpha_{n+1}$};
        \node[minimum width=0.5cm,minimum height=0.5cm] at (0.95,0) {$\cdots$};
        \node[minimum width=0.5cm,minimum height=0.5cm] at (1.5,0) {$\alpha_{n+1}^{n-1}$};
        \node[minimum width=1cm,minimum height=0.5cm] at (2.5,0) {$\alpha_{n+1}^n$};
        \node[minimum width=1cm,minimum height=1cm] at (2.5,0.4) {$\alpha_n^n$};
        \node[minimum width=1cm,minimum height=1cm] at (2.5,1) {$\vdots$};
        \node[minimum width=1cm,minimum height=1cm] at (2.5,1.4) {$\alpha_1^n$};
        \draw (0,0.6) -- (2.5,0.6);
        \draw (2.5,0.6) -- (2.5,2.225);
        \draw[dashed] (2.5,0.6) -- (3.5,0.6);
        \draw[dashed] (2.5,0.6) -- (2.5,0);
    \endtikzpicture
    \end{gathered}\right|
}

\newcommand{\detvandpoly}{
\left|\begin{gathered}
    \tikzpicture[every node/.style={anchor=south west}]
        \node[minimum width=2cm,minimum height=1.5cm] at (-0.16,0.625) {$\Vandermonde(\alpha_1, \dots, \alpha_n)$};
        \node[minimum width=0.5cm,minimum height=0.5cm] at (0,0.05) {$1$};
        \node[minimum width=0.5cm,minimum height=0.5cm] at (0.4, 0) {$\alpha_{n+1}$};
        \node[minimum width=0.5cm,minimum height=0.5cm] at (0.95,0) {$\cdots$};
        \node[minimum width=0.5cm,minimum height=0.5cm] at (1.5,0) {$\alpha_{n+1}^{n-1}$};
        \node[minimum width=1cm,minimum height=0.5cm] at (2.5,0) {$P(\alpha_{n+1})$};
        \node[minimum width=1cm,minimum height=1cm] at (2.5,0.4) {$P(\alpha_n)$};
        \node[minimum width=1cm,minimum height=1cm] at (2.5,1) {$\vdots$};
        \node[minimum width=1cm,minimum height=1cm] at (2.5,1.4) {$P(\alpha_1)$};
        \draw (0,0.6) -- (2.5,0.6);
        \draw (2.5,0.6) -- (2.5,2.225);
        \draw[dashed] (2.5,0.6) -- (3.5,0.6);
        \draw[dashed] (2.5,0.6) -- (2.5,0);
    \endtikzpicture
    \end{gathered}\right|
}

\newcommand{\detvandzero}{
\left|\begin{gathered}
    \tikzpicture[every node/.style={anchor=south west}]
        \node[minimum width=2cm,minimum height=1.5cm] at (-0.16,0.625) {$\Vandermonde(\alpha_1, \dots, \alpha_n)$};
        \node[minimum width=0.5cm,minimum height=0.5cm] at (0,0.05) {$1$};
        \node[minimum width=0.5cm,minimum height=0.5cm] at (0.4, 0) {$\alpha_{n+1}$};
        \node[minimum width=0.5cm,minimum height=0.5cm] at (0.95,0) {$\cdots$};
        \node[minimum width=0.5cm,minimum height=0.5cm] at (1.5,0) {$\alpha_{n+1}^{n-1}$};
        \node[minimum width=1cm,minimum height=0.5cm] at (2.5,0) {$P(\alpha_{n+1})$};
        \node[minimum width=1cm,minimum height=1cm] at (2.5,0.4) {$0$};
        \node[minimum width=1cm,minimum height=1cm] at (2.5,1) {$\vdots$};
        \node[minimum width=1cm,minimum height=1cm] at (2.5,1.4) {$0$};
        \draw (0,0.6) -- (2.5,0.6);
        \draw (2.5,0.6) -- (2.5,2.225);
        \draw[dashed] (2.5,0.6) -- (3.5,0.6);
        \draw[dashed] (2.5,0.6) -- (2.5,0);
    \endtikzpicture
    \end{gathered}\right|
}

\begin{preuve}
    Nous allons raisonner par récurrence sur la taille de la matrice. Pour tout $n \in \Ne$ on pose
    \begin{center}
            $\mathscr{P}_n$: \say{ Soit $(\alpha_1, \dots, \alpha_n) \in \C^n$, $\det (\Vandermonde(\alpha_1, \dots, \alpha_n)) = \prod\limits_{1\leqslant i < j \leqslant n}(\alpha_j - \alpha_i)$}.
        \end{center}
    \begin{itemize}
        \item[$\rhd$] L'initialisation pour $n = 1$ est triviale.
        \item[$\rhd$] Soit $n \in \Ne$. On suppose $\mathscr{P}_n$ vraie, montrons $\mathscr{P}_{n+1}$. \\ 
        Soit $(\alpha_1, \dots, \alpha_n, \alpha_{n+1})$ une famille de complexes. \\
        On pose $P(X) \defeq \prod\limits_{j=1}^n (X-\alpha_j) = X^n + \sum\limits_{k=0}^{n-1} p_k X^k$. \\
        D'après les propriétés du déterminant, en ajoutant les $n$ premières colonnes respectivement multipliée par $p_k$ à la dernière, on obtient
        \begin{align*}
            \detvandnplusun &= \detvandpoly \\
            &= \detvandzero \\
            &= \det(\Vandermonde(\alpha_1, \dots, \alpha_n)) P(\alpha_{n+1}) \\
            \text{par hypothèse de récurrence } &= \prod\limits_{1\leqslant i < j \leqslant n}(\alpha_j - \alpha_i) \times \prod\limits_{j=1}^n (\alpha_{n+1}-\alpha_j) \\
            \det(\Vandermonde(\alpha_1, \dots, \alpha_{n+1})) &= \prod_{1\leqslant i < j \leqslant n + 1}(\alpha_j - \alpha_i).
        \end{align*}
    \end{itemize}
\end{preuve}

\begin{corol}
    La famille $(\alpha_1, \dots, \alpha_n)$ est libre si et seulement si le déterminant de sa matrice de \textsc{Vandermonde} est non nul.
\end{corol}

\begin{preuve}
    C'est immédiat d'après l'expression du déterminant.
\end{preuve}

\begin{exercice}
    \marginnote[0cm]{\cite{exos_oraux} p. 77}
    Soient $\alpha_1 < \dots < \alpha_n$ des réels. Montrer que la famille des fonctions $f_j:t \mapsto \exp(\mi \alpha_j t)$, pour $j \in \llbracket 1,n \rrbracket$, est libre.
\end{exercice}

\begin{elem_sol}
    \begin{enumerate}
        \item Revenir à la définition d'une famille libre.
        \item Dériver successivement les relations et les sommer. 
        \item Faire apparaître une matrice de \textsc{Vandermonde} et un système de \textsc{Cramer}. 
    \end{enumerate}
\end{elem_sol}


\section{Inverse de la matrice de {\textsc{Vandermonde}}}
\begin{exercice}
    \marginnote[0cm]{\cite{exos_oraux} p.81}
    Soit $n \in \Ne$ et $x_1, \dots, x_n$ des complexes deux à deux distincts.
    \begin{enumerate}
        \item Montrer que l'application
        \begin{alignat*}{2}
            \varphi\ :\ \C_{n-1}[X]\ &\longrightarrow\ \C^n\\
            P\ &\longmapsto\ \big( P(x_1), \dots, P(x_n) \big)
        \end{alignat*}
        est un isomorphisme. Montrer que sa matrice dans les bases canoniques de départ et d'arrivée est 
        $$
        M_n(x_1, \dots, x_n) \defeq
        \begin{pmatrix}
            1 & x_1 & x_1^2 & \cdots & x_1^{n-1} \\
            1 & x_2 & x_2^2 & \cdots & x_2^{n-1} \\
            \vdots & \vdots & \vdots & & \vdots \\
            1 & x_n & x_n^2 & \cdots & x_n^{n-1}
        \end{pmatrix}.
        $$
        \item On note $\Lag_1(X), \dots, \Lag_n(X)$ les polynômes interpolteurs de \textsc{Lagrange} associés à $x_1, \dots, x_n$. Donner une relation entre les coefficients de $\Inv{M_n(x_1, \dots, x_n)}$ et ceux des polynômes $\Lag_i(X)$.
    \end{enumerate}
\end{exercice}

\section{Déterminant de \textsc{Cauchy}}
\begin{defi}{Déterminant de \textsc{Cauchy}}
    Le déterminant de \textsc{Cauchy} est un déterminant de taille $n$ et de terme général $\frac{1}{a_i+b_j}$, où les complexes $(a_1, \dots, a_n)$ et $(b_1, \dots, b_n)$ sont tels que pour tout $(i, j)$, $a_i+b_j \not= 0$.
    $$\Cauchy_n \defeq \begin{vmatrix}
        \frac{1}{a_1+b_1} & \frac{1}{a_1+b_2} & \cdots & \frac{1}{a_1+b_n} \\
        \frac{1}{a_2+b_1} & \frac{1}{a_2+b_2} & \cdots & \frac{1}{a_2+b_n} \\
        \vdots & \vdots & & \vdots \\
        \frac{1}{a_n+b_1} & \frac{1}{a_n+b_2} & \cdots & \frac{1}{a_n+b_n}
    \end{vmatrix}.$$
\end{defi}

\begin{prop}{Expression du déterminant de \textsc{Cauchy}}
    Le déterminant de \textsc{Cauchy} a pour expression
    $$\Cauchy_n = \frac{\prod\limits_{i<j}(a_j-a_i) \prod\limits_{i<j}(b_j-b_i)}{\prod\limits_{i,j}(a_i+b_j)}.$$
\end{prop}
        
\begin{preuve}
    On raisonne par récurrence sur $n$. \\
    Premièrement, faisons apparaître une ligne de 1 en multipliant toutes les colonnes par $a_n+b_j$. \\
    Par $n$-linéarité du déterminant,
    $$\Cauchy_n = \frac{1}{\prod\limits_{i=1}^{n}(a_n + b_i)} \begin{vmatrix}
        \frac{a_n+b_1}{a_1+b_1} & \cdots & \frac{a_n+b_n}{a_1+b_n} \\
        \vdots & & \vdots \\
        \frac{a_n+b_1}{a_{n-1}+b_1} & \cdots & \frac{a_n+b_n}{a_{n-1}+b_n} \\
        1 & \cdots & 1
    \end{vmatrix}.$$
    
    Ensuite, pour faire apparaître des 0 à la fin des $n-1$ premières colonnes, soustrayons la dernière colonne à toutes les autres. 
    
    $$
        \forall (i,j) \in \llbracket 1, n-1 \rrbracket^2, \quad \frac{a_n+b_i}{a_i+b_j} - \frac{a_n+b_n}{a_i+b_n} = \frac{(a_n - a_i)(b_n-b_j)}{(a_i + b_j)(a_i + b_n)}
    $$
    $$\Cauchy_n = \frac{1}{\prod\limits_{i=1}^{n}(a_n + b_i)} \begin{vmatrix}
        \frac{\textcolor{red}{(a_n - a_1)}\textcolor{blue}{(b_n-b_1)}}{(a_1 + b_1) \textcolor{green}{(a_1 + b_n)}} & \cdots & \frac{\textcolor{red}{(a_n - a_1)}\textcolor{blue}{(b_n-b_{n-1})}}{(a_1 + b_{n-1})\textcolor{green}{(a_1 + b_n)}} & \frac{a_n+b_n}{a_1+b_n} \\
        \vdots & & \vdots & \vdots \\
        \frac{\textcolor{red}{(a_n - a_{n-1})}\textcolor{blue}{(b_n-b_1)}}{(a_{n-1} + b_1)\textcolor{green}{(a_{n-1} + b_n)}} & \cdots & \frac{\textcolor{red}{(a_n - a_{n-1})}\textcolor{blue}{(b_n-b_{n-1})}}{(a_{n-1} + b_{n-1})\textcolor{green}{(a_{n-1} + b_n)}} & \frac{a_n+b_n}{a_{n-1}+b_n} \\
        0 & \cdots & 0 & 1
    \end{vmatrix}.$$
        
    Factorisons par les termes communs aux lignes et aux colonnes.
        
    $$\Cauchy_n =  \frac{\textcolor{red}{\prod\limits_{i=1}^{n-1} (a_n - a_i)} \textcolor{blue}{\prod\limits_{i = 1}^{n-1}(b_n - b_i)}}{\prod\limits_{i=1}^{n}(a_n + b_i) \textcolor{green}{\prod\limits_{i = 1}^{n - 1} (a_i + b_n)}} \begin{vmatrix}
        \frac{1}{a_1+b_1} & \cdots & \frac{1}{a_1+b_{n-1}} & \bigstar \\
        \vdots & & \vdots & \vdots \\
        \frac{1}{a_{n-1}+b_1} & \cdots & \frac{1}{a_{n-1}+b_{n-1}} & \bigstar \\
        0 & \cdots & 0 & 1
    \end{vmatrix}.$$
        
    Finalement, en développant par rapport à la dernière ligne on trouve la relation de récurrence:
    $$\Cauchy_n =  \frac{1}{a_n + b_n} \Bigg( \prod\limits_{i = 1}^{n-1}\frac{(a_n - a_i)(b_n - b_i)}{(a_n + b_i)(a_i + b_n)} \Bigg) \Cauchy_{n-1}$$
    
    Comme $\Cauchy_1 = \frac{1}{a_1+b_1}$, on obtient par récurrence
    $$\Cauchy_n = \frac{\prod\limits_{i<j}(a_j-a_i) \prod\limits_{i<j}(b_j-b_i)}{\prod\limits_{i,j}(a_i+b_j)}.$$
\end{preuve}

\begin{remarque}
    On peut aussi écrire
    $$\Cauchy_n = \frac{\Vandermonde(a_1, \dots, a_n) \Vandermonde(b_1, \dots, b_n)}{\prod\limits_{i,j}(a_i+b_j)}$$
    en notant $\Vandermonde(\alpha_1, \dots, \alpha_n)$ le déterminant de la matrice de \textsc{Vandermonde} de la famille $(\alpha_1, \dots, \alpha_n)$.
\end{remarque}

\begin{methode}
    La méthode pour calculer ce genre de déterminants \say{ compliqués } est toujours la même: 
    \begin{enumerate}
        \item Raisonner par récurrence sur la taille de la matrice.
        \item Faire des opérations élémentaires sur les lignes/colonnes pour faire apparaître une ligne/colonne composée de 1. 
        \item Soustraire les lignes/colonnes de manière à obtenir une ligne/colonne presque composée uniquement de 0 sauf pour un seul coefficient. 
        \item Développer par rapport à cette dernière et obtenir une relation de récurrence sur le déterminant. 
    \end{enumerate}
\end{methode}

\subsection{Matrice de \textsc{Hilbert}}

Le déterminant d'une matrice de \textsc{Hilbert} est un cas particulier du déterminant de \textsc{Cauchy}.

\begin{defi}{Matrice de \textsc{Hilbert}}
    Une matrice de \textsc{Hilbert} est une matrice carrée de terme général
    $$\Hilb_{i,j} \defeq {\frac {1}{i+j-1}}.$$
\end{defi}

\marginnote[0cm]{Source : \cite{exos_oraux} p. 82}
Les matrices de \textsc{Hilbert} sont souvent utilisées pour tester des ordinateurs ou leurs programmes censés inverser des matrices. Le fait que leur déterminant soit rapidement très petit rend l'inversion de la matrice numériquement difficile et c'est ce qui fait la qualité du test. 

\section{Déterminant par blocs}
\begin{exercice} 
À quelles conditions sur $A$, $B$, $C$ et $D$, des matrices carrées d'ordre $n$, a-t-on:
    $$
        \begin{vmatrix}
            A & B\\
            C & D\\
        \end{vmatrix} = \det(A \times D - B \times C) \text{ ?}
    $$ 
\end{exercice}
 
Rappel de cours:

\begin{prop}{}
    Soient $T_1, \dots, T_p$ des matrices carrées et 
    $$A \defeq
    \begin{pmatrix}
        T_1 & \star & \cdots & \star \\
        0 & T_2 & \ddots & \vdots \\
        \vdots & \ddots & \ddots & \star \\
        0 & \cdots & 0 & T_p
    \end{pmatrix}.
    $$
    Alors, 
    $$\det(A) = \prod_{i=1}^p \det(T_i).$$
\end{prop}

\begin{solution}
    D'après le rappel de cours, 
    $$\det(AD - BC) = \begin{vmatrix}
        AD - BC & \star \\
        0 & \I_n
    \end{vmatrix}.$$ 
    On \say{ remarque } que si les matrices $D$ et $C$ commutent et que si la matrice $D$ est inversible alors
    $$
    \begin{pmatrix}
        A & B \\
        C & D
    \end{pmatrix}
    \begin{pmatrix}
        D & 0 \\
        -C & \Inv{D}
    \end{pmatrix}
     = \begin{pmatrix}
         AD-BC & \star \\
         0 & \I_n
     \end{pmatrix}.
    $$
    ...
\end{solution} \chapter{Réduction des endomorphismes}
\labch{reduction_des_endomorphismes}

{\Large Diagonalisation} \\
\marginnote[0cm]{Le texte suivant est extrait de \cite{oraux_x_ens_2} p. 169.}
\begin{marginfigure}[1cm]
    \centering
    \includegraphics[width=5cm]{images/camille_jordan.jpg}
    \caption*{\centering Camille \nom{Jordan} (1838 - 1922)}
\end{marginfigure}
\textsl{On doit à Camille \nom{Jordan} de nombreux résultats sur la réduction des endomorphismes qu'il découvre notamment à travers l'étude des groupes. \\
Le problème fondamental de la réduction est bien celui de caractériser les classes de similitude de l'algèbre $\Endo(E)$ où $E$ est un $\K$-espace vectoriel de dimension finie ou, ce qui revient au même, les classes de similitudes de l'agèbre $\M_n(\K)$. La recherche d'une matrice la plus simple possible pour représenter un endomorphisme donné vise de multiples buts: calculer les puissances successives de cet endomorphisme, son commutant, résoudre des systèmes différentiels linéaires... Une idée naturelle pour essayer de \chevron{réduire} l'étude d'un endomorphisme $u$ donné à des choses plus simples consiste à essayer de décomposer l'espace vectoriel $E$ en une somme directe de sous-espaces non triviaux stables par $u$. Cela n'est évidemment pas toujours possible. Les sous-espaces stables les plus simples sont ceux sur lesquels $u$ coïncide avec une homothétie. On est ainsi naturellement amené à la notion de valeur propre. Si $\lambda$ est un scalaire, on s'intéresse donc au sous-espace $E_\lambda = \Ker(u - \lambda \Id_E)$ appelé sous-espace propre pour la valeur propre $\lambda$ lorsque celui-ci n'est pas nul. Le théorème de décomposition des noyaux nous assure que les différents sous-espaces propres d'un endomorphisme sont en somme directe. Le cas où la somme remplit tout l'espace $E$ mène à la notion d'endomorphisme diagonalisable: un tel endomorphisme peut être représenté par une matrice diagonale (il suffit de prendre une base formée de vecteurs propres). Pour les endomorphismes diagonalisables il est alors très facile de répondre à la question initiale de savoir quand ils sont semblables: il faut et suffit qu'ils aient les mêmes valeurs propres et que les espaces propres associés aient la même dimension. Il est aussi facile, en se ramenant à une matrice diagonale, de calculer les puissances d'un tel endomorphisme, son exponentielle (si on travaille sur un sous-corps de $\C$), son commutant...}

\newpage

\section{Matrice à diagonale dominante}
\input{chapitres/reduction_des_endomorphismes/matrice_a_diagonale_dominante}

\section{Matrices stochastiques}
\marginnote[3cm]{
    $$
    \begin{pmatrix}
    1/2 & 1/2 & 0 \\
    3/4 & 1/8 & 1/8 \\
    0 & 1/3 & 2/3
    \end{pmatrix}
    $$
}

\begin{marginfigure}[5cm]
    \centering
    \resizebox{5.5cm}{5.5cm}{%
    \begin{tikzpicture}
        \node[state] (s1) {1};
        \node[state, below right of=s1] (s2) {2};
        \node[state, below left of=s1] (s3) {3};
    
        \draw (s1) edge[loop above] node {$1/2$} (s1);
        \draw (s1) edge[bend left] node {$1/2$} (s2);
        %\draw (s1) edge[bend right, above left] node {0} (s3);
    
        \draw (s2) edge[bend left, above right] node {$3/4$} (s1);
        \draw (s2) edge[loop right] node {$1/8$} (s2);
        \draw (s2) edge[bend right] node {$1/8$} (s3);
    
        %\draw (s3) edge[bend right] node {0} (s1);
        \draw (s3) edge[bend right] node {$1/3$} (s2);
        \draw (s3) edge[loop left] node {$2/3$} (s3);
    \end{tikzpicture}    
}

  %  \begin{tikzpicture}
  %      \node[state] (s1) {État 1};
  %      \node[state, below right of=s1] (s2) {État 2};
   %     \node[state, below left of=s1] (s3) {État 3};
 %  
 %       \draw (s1) edge[loop above] node {$p_{1,1}$} (s1);
%        \draw (s1) edge[bend left] node {$p_{1,2}$} (s2);
%        \draw (s1) edge[bend right, above left] node {$p_{1,3}$} (s3);
    
 %       \draw (s2) edge[bend left, above right] node {$p_{2,1}$} (s1);
%        \draw (s2) edge[loop right] node {$p_{2,2}$} (s2);
%        \draw (s2) edge[bend right] node {$p_{2,3}$} (s3);
    
%        \draw (s3) edge[bend right] node {$p_{3,1}$} (s1);
%        \draw (s3) edge[bend right] node {$p_{3,2}$} (s2);
%        \draw (s3) edge[loop left] node {$p_{3,3}$} (s3);
%    \end{tikzpicture}    
    \caption*{\centering Une chaîne de \textsc{Markov} et sa matrice de transition.}
\end{marginfigure}

\marginnote[0cm]{Texte de \cite{oraux_x_ens_2} p. 59}
Les matrices stochastiques interviennent en probibilités. Si $X$ et $Y$ sont deux variables aléatoires à valeurs dans $E \defeq \llbracket 1, k \rrbracket$, alors la matrice $A \defeq (a_{i,j}) \in \M_k(\R)$ définie par $a_{i,j} \defeq \P(Y = j | X = i)$ est stochastique, ce qui par définition signifie qu'on a $a_{i,j} \geqslant 0\ (1 \leqslant i, j \leqslant k)$ et $\sum\limits_{j=1}^k a_{i,j} = 1\ (1 \leqslant i \leqslant k)$. \\
L'évolution d'un système susceptible de prendre un nombre fini d'états notés $1, \dots, k$ est représentée mathématiquement par une suite $(X_n)_{n \geqslant 0}$ de variables aléatoires à valeurs dans $E$. C'est ce qu'on appelle un processus aléatoire (ou stochastique). Si $X_{n+1}$ s'obtient à partir de la valeur de $X_n$ et d'un tirage au sort effectué selon une loi ne dépendant que de cette valeur, on dit que le processus est une chaîne de \textsc{Markov}. Les exemples abondent: marches aléatoires, fortune d'un joueur, modélisation de l'alternance des voyelles et des consonnes dans un poème de \textsc{Pouchkine} (par \textsc{Markov} lui-même), ou prévision (en probabilité) des états  successifs d'un signal pour améliorer la compression en traitement du signal (\textsc{Shannon}) \\
Techniquement, on dit qu'une suite de variables aléatoires $(X_n)$ est une chaîne de \textsc{Markov} si \say{ la loi de l'état $n+1$ conditionnelle au passé de dépend que de l'état antérieur $n$ }, ce qui se traduit par
$$\P(X_{n+1}=j | X_0=i_0, \dots, X_n = i_n) = \P(X_{n+1} = j | X_n = i_n).$$
Si la matrice $A \defeq (a_{i,j}) \in \M_k(\R)$ définie par $a_{i,j} \defeq \P(Y = j | X = i)$ est indépendante de $n$, on dit que la chaîne de \textsc{Markov} est stationnaire. Si, dans ce dernier cas, on pose $Y_n \defeq \begin{pmatrix} \P(X_n = 1) \\ \vdots \\ \P(X_n = k) \end{pmatrix}$, pour tout $n \geqslant 0$, on obtient $Y_{n+1} = A Y_n$ et donc $Y_n = A^n Y_0$. \\
Le comportement (probabiliste) d'une chaîne de \textsc{Markov} stationnaire, et notamment son comportement asymptotique, est donc entièrement décrit par la donnée de la loi initiale $Y_0$ et des puissances de la matrice $A$. 

\begin{defi}
    Une matrice stochastique (matrice de transition d'une \nameref{chaîne_markov}) est une matrice $P \in \M_n([0, 1])$ telle que pour tout $i \in \llbracket 1, n \rrbracket, \sum\limits_{j=1}^{n} p_{i,j} = 1$. \\ Autrement dit, chaque ligne de $P$ est une vecteur de probabilité. \\
    On dit que $P$ est \emph{doublement stochastique} si $P$ et $\Trsp{P}$ sont stochastiques.
\end{defi}

\begin{exercice}
    \footnote{Exercice 5, TD 11} \\
    Soit $P$ une matrice stochastique.
    \begin{enumerate}
        \item Montrer que $1$ est valeur propre de $P$.
        \item Soit $v = \Trsp{(v_1 \cdots v_n)}$ un vecteur propre associé à la valeur propre $1$. En considérant $|v_{i_0}| = \max\limits_{1 \leqslant i \leqslant n} |v_i|$, montrer que le sous-espace propre associé à $E_1$ est de dimension $1$.
        \item Montrer que si $\lambda \in \C$ est une valeur propre de $P$, alors $| \lambda | \leqslant 1$.
        \item Soit $\lambda \in \C$ une valeur propre de $P$ telle que $|\lambda| = 1$ et $\title{x}$ un vecteur propre associé.
        \begin{enumerate}
            \item Montrer qu'il existe un vecteur propre associé à $\lambda$ tel que $\Ninf{x} = 1$. 
            \item Montrer qu'il existe $i_0 \in \llbracket 1, n \rrbracket$ tel que $\left| \sum\limits_{j=1}^n p_{i_0,j} x_j \right| = 1$.
            \item Soit $\theta$ l'argument principal de $\sum\limits_{j=1}^n p_{i_0,j} x_j$. Montrer que pour tout $j \in \llbracket 1, n \rrbracket, \Reel \left( \me^{-\mi \theta} x_j \right) = 1$.
            \item En déduire que $\lambda = 1$.
        \end{enumerate}
    \end{enumerate}
\end{exercice}

\begin{solution}
    L'objectif de l'exercice est de montrer que $\boxed{\Sp_{\C}(P) = \{1 \} }$.
    \begin{enumerate}
        \item 1 est valeur propre évidente de $P$ de vecteur propre associé $v = (1, \dots, 1)^\top$.
        \item Montrer que $\dim E_1 = 1$.
        \begin{itemize}
            \item Appliquer la même méthode que la démonstration du lemme d'\textsc{Hadamard}. \\
            Soit $X = \Trsp{(x_1, \dots, x_n)} \in E_1$. Montrons que $X \in \Vect(v)$. \\
            On montre que $|x_{i_0}| = \left| \sum\limits_{j=1}^{n} p_{i_0, j} x_j \right| = \sum\limits_{j=1}^{n} p_{i_0, j} |x_j|$ et on écrit $|x_{i_0}| = |x_{i_0}| \sum\limits_{j=1}^{n} p_{i_0, j}$. D'où, en faisant la différence, pour tout $j \in \llbracket 1, n \rrbracket,\ |x_{i_0}| = |x_j|$. De plus d'après la première relation, il y égalité dans l'inégalité triangulaire et donc les $v_j$ sont \emph{positivement liées}. Finalement, pour tout $j \in \llbracket1, n \rrbracket,\ v_j = v_{i_0}$ soit $\dim E_1 = 1$.
        \end{itemize}
        \item Montrer que si $\lambda \in \C$ est une valeur propre de $P$, alors $|\lambda| \leqslant 1$. \\
        Poser $X = (x_1, \dots, x_n)^\top$ un vecteur propre associé et appliquer encore une fois la même méthode; poser $\displaystyle |x_{i_0}|= \max_{1 \leqslant i \leqslant n} |x_i|$, écrire en module la ligne $i_0$ de l'égalité $\lambda X = P X$, diviser par $|x_{i_0}|$ (qui est non nul d'après la question précédente) puis majorer par $1$. \\
            
        Pour les curieux, lire \cite{matrices} page 59. 
        
        \begin{prop}
            Le \nameref{rayon_spectral} stochastique est égal à $1$.
        \end{prop}
    
        \item Les questions suivantes (à détailler éventuellement) consiste encore au même jeu avec la ligne $i_0$ et les propriétés de matrices stochastiques. 
    \end{enumerate}
\end{solution}


\section{Endomorphismes semi-simples}
\begin{prop}{Critère de diagonalisabilité dans un $\R$-ev}
    Soit $E$ un $\R$-espace vectoriel de dimension finie, soit $f \in \Endo(E)$. Alors $f$ est diagonalisable si et seulement si tout sous-espace vectoriel de $E$ admet un supplémentaire stable par $f$.
\end{prop}

\begin{preuve}
    \marginnote[0cm]{Source : à compléter}
    \begin{itemize}
        \item[$(\Leftarrow)$] Supposons $f$ diagonalisable. Soit $F$ un sous-espace vectoriel quelconque de $E$. Soit $(e_1, \dots, e_m)$ une base de $F$. Puisqu'on a supposé $f$ diagonalisable, il exsite une base $(v_1, \dots, v_n)$ de $E$ formée de vecteurs propres pour $f$. D'après le théorème de la base incomplète, on peut compléter la base $(e_1, \dots, e_m)$ de $F$ en une base $(e_1, \dots, e_m, e_{m+1}, \dots, e_n)$ de $E$, où on rajouté uniquement des vecteurs de notre base de vecteurs propres (c'est-à-dire $e_{m+1}, \dots, e_n$ ont été pris parmi les $v_i$). En effet, la famille $(e_1, \dots, e_m)$ est libre, et elle est contenue dans la famille génératrice $(e_1, \dots, e_m, v_1, \dots, v_n)$, donc il existe une famille $\mathcal{F}$ de vecteurs de $E$, telle que 
        $$(e_1, \dots, e_m) \subseteq \mathcal{F} \subseteq (e_1, \dots, e_m, v_1, \dots, v_n)$$
        et $\mathcal{F}$ à la fois libre et génératrice. \\
        \textcolor{red}{il manque une explication supplémentaire} \\
        On prend alors $G \defeq \Vect(e_{m+1}, \dots, e_n)$. C'est un supplémentaire de $F$ dans $E$, et il est stable par $f$ car $e_{m+1}, \dots, e_n$ sont des vecteurs propres de $f$.
        \item[$(\Rightarrow)$] Réciproquement, supposons que tout sous-espace vectoriel de $E$ admet un supplémentaire stable par $f$. Considérons
        $$F \defeq \bigoplus_{\lambda \in \Sp f} \Ker(f - \lambda \Id)$$
        le sous-espace vectoriel de $E$ formé de la somme directe des sous-espaces propres de $f$. Si $f$ n'était pas diagonalisable, $F$ serait strictement inclus dans $E$. Soit $H$ un hyperplan de $E$ contenant $F$. Alors par hypothèse $H$ admet un supplémentaire stable par $f$. Ce supplémentaire est une droite, engendrée par un vecteur propre de $f$. Mais c'est une contradiction car tous les vecteurs propres de $f$ sont dans $H$. Ainsi, $f$ est diagonalisable. 
    \end{itemize}
\end{preuve}

Nous allons maintenant définir la notion d'endomorphisme semi-simple en relâchant un peu la
condition de l'exercice ci-dessus : nous allons seulement demander aux sous-espaces stables de posséder
un supplémentaire stable.

\begin{defi}{Endomorphisme semi-simple}
    Un endomorphisme $u$ est dit \emph{semi-simple} si tout sous-espace stable par $u$ admet un supplémentaire stable par $u$.
\end{defi}

\begin{prop}{Critère de diagonalisabilité sur un $\C$-ev}
    Soient $E$ un $\C$-espace vectoriel de dimension finie non nulle et $u \in \Endo(E)$. Alors l'endomorphisme $u$ est semi-simple si et seulement s'il est diagonalisable.
\end{prop}

\begin{preuve}
    \begin{itemize}
        \item[$(\Leftarrow)$] On suppose que l'endomorphisme $u$ est diagonalisable. \\
        Son polynôme caractéristique est scindé ce que l'on voit en mettant $u$ sous forme diagonale, et par invariance de $\chi$ par changement de base. \\
        Soit $F$ un sous-espace de $E$. Soit $(e_1, \dots, e_n)$ une base de vecteurs propres de $u$ et $(f_1, \dots, f_p)$ une base de $F$. Par le théorème de la base incomplète, on peut compléter la famille libre $(f_1, \dots, f_p)$ en une base de $E$ en rajoutant $n-p$ vecteurs parmi la base $(e_1, \dots, e_n)$, quitte à réindexer, on peut supposer que c'est $(e_{p+1}, \dots, e_n)$, ces vecteurs engendrent alors un sous-espace stable supplémentaire de $F$.
        \item[$(\Rightarrow)$] On construit une base de vecteurs propres de la manière suivante: prenons un hyperplan $H$ quelconque, il existe une droite stable supplémentaire, donc dirigée par un vecteur propre $e_1$. Si on a construit une famille libre de vecteurs propres $(e_1, \dots, e_k)$, on prend un hyperplan contenant $\Vect(e_1, \dots, e_k)$, et on trouve une droite stable $\Vect(e_{k+1})$ supplémentaire à $H$. On conclut par récurrence.
    \end{itemize}
\end{preuve}

Notes de la correction vue en cours.
\begin{preuve}
    \begin{itemize}
        \item[$(\Leftarrow)$] Soit $F$ un sev de $E$ stable par $f$.
        Posons $g \defeq f_{\vert F}$ qui est diagonalisable car $f$ l'est par hypothèse. \\
        Soit $\mathscr{B}_F$ une base de $F$ formée de vep de $g$. \\
        Soit $\mathscr{B}'$ une base de $E$ formée de vep de $f$ (qui existe car $f$ est diagonalisable). \\
        On complète $\mathscr{B}_F$ est une base $\mathscr{B}$ de E en prenant des vep $(\varepsilon_1, \cdots, \varepsilon_r)$ de $\mathscr{B}'$. On note $(\lambda_1, \dots, \lambda_r)$ les valeurs propres associées. \\
        On pose $G = \mathrm{Vect}(\varepsilon_1, \dots, \varepsilon_r)$. De cette manière, $F$ et $G$ sont supplémentaires. Montrons que $G$ est stable par $f$. \\
        Soit $x = \sum\limits_{i=1}^{r} \mu_i \varepsilon_i \in G$. Donc $f(x) = \sum\limits_{i=1}^{r} \mu_i f(\varepsilon_i) =  \sum\limits_{i=1}^{r} \mu_i \lambda_i \varepsilon_i \in G$ et $G$ est stable par $f$.
    
        \item[$(\Rightarrow)$] Montrons que $f$ est diagonalisable. On va montrer que $E = \bigoplus\limits_{\lambda \in \Sp(f)} E_\lambda (f)$.
        
        \begin{enumerate}
            \item \underline{Somme directe:} \\
            On pose $F = \bigoplus\limits_{\lambda \in \Sp(f)} E_\lambda (f)$ et $\Sp(f) = (\lambda_1, \dots, \lambda_r)$. \\
            Soit $x = \sum\limits_{i=1}^{r} x_i \in F$ où $x_i \in E_{\lambda_i}(f)$. Alors $f(x) = \sum\limits_{i=1}^{r} \underbrace{\lambda_i x_i}_{\in E_{\lambda_i}(f)} \in F$. Donc $F$ est stable par $f$.
            \item Montrons que $F = E$. \\
            Par hypothèse, $F$ admet un supplémentaire $G$ dans $\C^n$, stable par $f$. Montrons que $G = \{0\}$ en raisonnant par l'absurde. \\
            On pose $g = f_{\vert F}$. D'après le théorème de \textsc{D'Alembert-Gauss} sur $\C$, $g$ admet au moins une valeur propre $\mu \in \C$ de vep associé $x_\mu$. On montre que $x_\mu \in F \cap G$. Or $F$ et $G$ sont supplémentaires donc $x_\mu = 0_E$: contradiction. D'où le résultat. 
        \end{enumerate}
        \item Considérer $R_\theta$. \textcolor{green}{à revoir}
    \end{itemize}
\end{preuve}

\begin{exercice}
    Décrire un contre-exemple à la réciproque dans $\R$, en dimension 2.
\end{exercice}  

\section{Autour du commutant}
\begin{defi}[Commutant d'une matrice]
    Soient $A \in \M_n(\R)$ et $C(A) \defeq \ens[\big]{M \in \M_n(\R) \tq MA = AM }$.
\end{defi}

\begin{exercice}
    \source{\cite{exos_oraux} p. 119}
    Soit $E$ un $\K$-espace vectoriel de dimension finie et $u \in \Endo(E)$. Démontrer que $C(u)$ a une structure de $\K$-espace vectoriel puis que, si $u$ est diagonalisable:
    $$\dim C(u) = \smashoperator{\sum_{\lambda \in \Sp(u)}} \dim^2 E_\lambda(u).$$
\end{exercice}

\begin{solution}
    
\end{solution}

\begin{exercice}
    \source{\cite{acamanes} \href{https://acamanes.github.io/psi/psi_doc/exos_e11.pdf}{(Exercice 12 TD 11)}}
    Soit $A \in \M_n(\R)$.
    \begin{enumerate}
        \item 
        \begin{enumerate}
            \item Montrer que $C(A)$ est un sous-espace vectoriel de $\M_n(\R)$ stable par multiplication.
            \item Montrer que si $M \in C(A)$ et $M$ est inversible, alors $\Inv{M} \in C(A)$.
        \end{enumerate}
        \item Soit $D$ une matrice diagonale dont les coefficients diagonaux sont deux à deux distincts.
        \begin{enumerate}
            \item Déterminer $C(D)$.
            \item Montrer que $\big(\I_n, D, \dots, D^{n-1}\big)$ est une base de $C(D)$.
        \end{enumerate}
        \item On se limite au cas $n=2$.
        \begin{enumerate}
            \item Déterminer les matrices $A$ telles que $\dim C(A) = 4$.
            \item Montrer que $\dim C(A) \geqslant 2$. 
            \item On suppose que $\dim C(A) \geqslant 3$. En utilisant $F \defeq \Vect \big\{ \mathrm{E}_{1, 1}, \mathrm{E}_{1, 2} \big\}$ ou $G \defeq \Vect \big\{ \mathrm{E}_{2, 1}, \mathrm{E}_{2, 2} \big\}$, montrer que $A = \lambda \I_2$.
            \item Pour tout $A \in \M_2(\R)$, déterminer une base de $C(A)$.
        \end{enumerate}
    \end{enumerate}
\end{exercice}

\begin{enumerate}
    \item \emph{Montrer que C(A) est un sous-espace vectoriel de $\M_n(\R)$.} \\
    Au lieu de redémontrer les propriétés d'un sev, on peut voir $C(A)$ comme le \textbf{noyau de l'application linéaire} $M \mapsto MA - AM$ ce qui donne directement le résultat. 
    \item On veut montrer que $M^{-1} A = A M^{-1}$ i.e. $A = M A M^{-1}$ ce qui est vrai car $M A = A M$.
    \item
    \begin{itemize}
        \item $C(D) = \mathscr{D}_n$ (l'ensemble des matrices diagonales de taille $n$) \textcolor{red}{(ne pas oublier de montrer la double inclusion)}.
        \item Comme $| \mathscr{B} | = \dim C(D)$, il suffit de montrer la liberté de $\mathscr{B}$. \\
        Soit $(\lambda_0, \dots, \lambda_{n-1}) \in \R^n$ tel que $\sum\limits_{k=0}^{n-1} \lambda_k D^k = 0_n$. \\
        \textcolor{green}{Revoir le caractère générateur avec les polynômes d'interpolation.}
        \begin{itemize}
            \item Pour tout $i \in \llbracket 1, n \rrbracket$, $\sum\limits_{k=0}^{n-1} \lambda_k d_i = 0_n \quad (*)$. Donc le polynôme $P = \sum\limits_{k=0}^{n-1} \lambda_k X^k$ qui est de dégré $n-1$ et prossède $n$ racines distinctes et est donc le polynôme nul. On en déduit que les $\lambda_i$ sont tous nuls. La famille $\mathscr{B}$ est bien libre et forme une base de $C(D)$.
            \item Les relations $(*)$ forment un système de \nom{Vandermonde} de $n$ équations à $n$ inconnues. Comme les coefficients $d_i$ sont deux à deux distincts, le système est inversible et son unique solution est le vecteur colonne nul.
        \end{itemize}
    \end{itemize}
    \item On se limite au cas $n = 2$. 
    \begin{enumerate}
        \item Déterminer les matrices $A$ telles que $\dim C(A) = 4$. \\
        $C(A) = \M_2(\R)$ car $C(A) \subset \M_2(\R)$ et il y égalité des dimensions. \\
        \ptnclegras{Évaluer les commutant en les matrices de la base canoniques de $\M_2(\R)$}: on trouve que A est scalaire. \\
        \textcolor{red}{Ne pas oublier de montrer la réciproque}. 
        \item Montrer que $\dim C(A) \geqslant 2$. \\
        Si $A$ est scalaire, cf. question précédente. \\
        Sinon montrer que la famille $\big\{ \I_2, A \big\} \subset C(A)$ est libre. 
        \item Enoncé... \\
    \end{enumerate}
\end{enumerate}

\section{Que dire si \texorpdfstring{$M^2$}{M^2} est diagonalisable ?}
\begin{prop}
    Soit $n \in \Ne$ et $M \in \M_n(\C)$. On suppose que $M^2$ est diagonalisable. Alors $M$ est diagonalisable si et seulement si $\Ker(M) = \Ker(M^2)$.
\end{prop}

\begin{preuve}
    On note $f$ l'endomorphisme canoniquement associé à $M$. 
    \begin{itemize}
        \item 
        %$(\Rightarrow)$ On suppose que $M$ est diagonalisable (et donc $f$ aussi). Notons $(\lambda_1, \dots, \lambda_n) \in \C^n$ les valeurs propres de $f$. Il existe une base $\mathscr{B}$ telle que $\mathrm{Mat}_{\mathscr{B}}(f) = \mathrm{Diag}(\lambda_1, \dots, \lambda_n)$. Donc $\mathrm{Mat}_{\mathscr{B}}(f^2) = \mathrm{Diag}(\lambda_1^2, \dots, \lambda_n^2)$. \\
        Montrons que $\Ker(f^2) \subset \Ker(f)$ (l'autre inclusion est toujours vraie). \\
        Deux méthodes (je crois que la deuxième fonctionne...)
        \begin{itemize}
            \item On va montrer \textbf{l'égalité des dimensions}. Soit $\mathscr{B}$ une base de diagonalisation de $u$. \textbf{Le rang d'une matrice diagonale étant le nombre de coefficients diagonaux non nuls}, $\Rg \left(\mathrm{Mat}_\mathscr{B}(u^2) \right) = \Rg \left(\mathrm{Mat}_\mathscr{B}(u) \right)$ donc $\Rg(u^2) = \Rg(u)$. Par le \textbf{théorème du rang}, il s'ensuit que $\dim \Ker(u^2) = \dim \Ker(u)$.
            \item Soit $X \in \Ker(M^2)$ i.e. $M^2 X = 0\ (*)$. Montrons que $MX = 0$. L'idée est de \textbf{faire apparaître un produit scalaire} sur l'ensemble des vecteurs colonnes d'une matrice i.e. une produit de la forme $N^\top N$. \\
            Comme $M^2$ est diagonalisable, il existe $P$ inversible et $D$ diagonale telles que $M^2 = PDP^{-1}$. En remplaçant $M^2$ par cette expression dans $(*)$ puis en multipliant à gauche successivement par $P^{-1}$,  $(P^{-1})^\top$ et $X^\top$ on trouve $X^\top (P^{-1})^\top D^2 P^{-1} X = 0$ soit 
            $$(D P^{-1} X)^\top (D P^{-1} X) = 0.$$
            Comme $(C, C') \mapsto C^\top \times C'$ définit un produit scalaire sur l'espace des vecteurs colonnes, on a $D P^{-1} X = 0$ car il est orthogonal à lui-même et donc, en multipliant à gauche par $P$, on obtient bien $MX = 0$. 
        \end{itemize}
        \item $(\Leftarrow)$ On suppose que $\Ker(M) = \Ker(M^2)$. \\
        Raisonner par analyse-synthèse: soit $\lambda$ une valeur propre non nulle de $f^2$. Notons $\mu$ une racine carrée complexe de $\lambda$. Montrons que $E_{\lambda}(f^2) = E_{\mu}(f) \oplus E_{-\mu}(f)$. \\
        On pose $y = \frac{x}{2} + \frac{f(x)}{2 \mu}$ et $z = \frac{x}{2} - \frac{f(x)}{2 \mu}$. \\
        Comme $f^2$ est diagonalisable, $E$ est la somme directe des sous-espaces propres de $f^2$. On décompose chacun de ces sep comme ci-dessus et on en déduit que $E$ est la somme directe des sep de $f$ i.e. $f$ est diagonalisable. 
        \item $(\Leftarrow)$ \cite{reduc_des_endo} p. 100 \\
        Comme $u^2$ est diagonalisable, il existe $Q$ scindé à racines simples vérifiant $Q(0) \not= 0$ tel que $X Q(X)$ annule $u^2$, c'est-à-dire $u^2 \circ Q(u^2) = 0_{\Endo(E)}$. \\
        Alors, pour tout $x \in E$, $Q(u^2)(x) \in \Ker(u^2)$; or $\Ker(u^2) = \Ker(u)$ par hypothèse donc $Q(u^2)(x) \in \Ker(u)$ soit $u \circ Q(u^2) (x) = 0_E$. \\
        Ainsi, $XQ(X^2)$ est un polynôme annulateur de $u$. Il suffit de remarquer que ce polynôme est scindé à racines simples (car les racines complexes de $Q$ sont deux à deux distinctes et non nulles) pour conclure avec le critère algébrique de diagonalisabilité que $u$ est diagonalisable. \\
        \textsl{Une démonstration alternative consiste à montrer que, pour tout $\lambda \in \Ce$ de racines carrées distinctes $\mu$ et $\mu'$, le sous-espace propre $u^2$ associé à $\lambda$ se décompose avec les sous-espaces propres $u$:
        $$\Ker(\lambda \Id_E - u^2) = \Ker(\mu \Id_E - u) \oplus \Ker(\mu' \Id_E - u).$$
        La condition porte alors que le sous-espace propre de $u^2$ associé à $0$, c'est-à-dire $\Ker(u^2)$.
        }
    \end{itemize}
\end{preuve}


\section{Raciné carrée d'une matrice}
\url{https://share.miple.co/content/CtwFAB5leFp4M}

\begin{box_titre}{DS6}
    On note $\mathrm{Rac}(A) = \{ R \in \M_n(\R),\ R^2 = A \}$. \\
    $\blacktriangleright$ Soit $A \in \M_n(\R)$. $\mathrm{Rac}(A)$ est une partie fermée de $\M_n(\R)$. \\
    $\blacktriangleright$ $\mathrm{Rac}(\I_n)$ n'est pas une partie bornée de $\M_n(\R)$ pour $n \geqslant 3$. 
\end{box_titre}

\begin{box_titre}{Racine carrée de matrices symétriques positives}
    Pour tout $A \in \mathscr{S}^+(\R)$, il existe une unique matrice $B \in \mathscr{S}^+(\R)$ telle que $A = B^2$. 
\end{box_titre}

\underline{Exercice 11, TD 11:}\\
Soit $A = 
\begin{pmatrix}
    -1 & 2 & 3 \\
    0 & - 1 & 4 \\
    0 & 0 & 1
\end{pmatrix}. 
$ Montrer que $A$ n'a pas de racine dans $\M_3(\R)$. 

\section{Réduction d'une matrice creuse}
\begin{tcolorbox}
    Soit $n \geqslant 2$. On pose:
    $$
    A = 
        \begin{pmatrix}
              &        &   & c \\
              &  (0)   &   & \vdots \\
              &        &   & c \\
            b & \cdots & b & a
        \end{pmatrix}
        \in \M_n(\R).
    $$
    Étudier la possibilité de diagonaliser $A$ sur $\R$.
\end{tcolorbox}

\underline{Remarques:}\\
$\blacktriangleright$ Si $b = c$ alors $A$ est symétrique réelle donc diagonalisable. \\
$\blacktriangleright$ $A$ est au plus de rang 2. Donc par le \textbf{théorème du rang}, 0 est une valeur propre de $A$ de multiplicité au moins $n-2$.

\section{Vecteurs propres de \texorpdfstring{$\Trsp{\com(A)}$}{la transposée de la comatrice}}
\begin{exercice}
    Soit $A \in \M_n(\K)$. Montrer que 
    $$\Sp(A) \subset \Sp \big( \Trsp{\com(A)} \big).$$
\end{exercice}
\marginnote[0cm]{
    $$A \times\ \Trsp{\com(A)} = \Trsp{\com(A)} \times A = \det(A) \I_n$$
}
\begin{elem_sol}
    Soit $\lambda \in \Sp(A)$ et $X$ un vecteur propre associé. Alors $\lambda (BX) = \det(A) X$. 
        
    \begin{itemize}
        \item $\lambda \not = 0$: $BX = \frac{\det(A)}{\lambda}X$. 
        \item $\lambda = 0$: alors $\det(A) = 0$ et $\Rg(A) \leqslant n-1$. 
        \begin{itemize}
            \item $\Rg(A) \leqslant n-2$: supposer par l'absurde qu'il existe un déterminant mineur de $A$ non nul. 
            \item $\Rg(A) = n-1$: $X \in \Ker(A)$ qui est une droite vectorielle. 
            $$A \text{ et } B \text{ commutent donc } \Ker(A) \text{ est stable par } B$$
        \end{itemize}
    \end{itemize}
\end{elem_sol}


\section{Éléments propres de \texorpdfstring{$MN$}{MN}, de \texorpdfstring{$NM$}{NM}}
\begin{prop}{}
    Soient $M$ et $N$ dans $\M_n(\K)$.
    \begin{itemize}
        \item $$0 \in \Sp(MN) \iff 0 \in \Sp(NM)$$
        \item Soit $\lambda \in \Ke$,
        $$\dim E_\lambda(MN) = \dim E_\lambda(NM)$$
    \end{itemize}
\end{prop}

Soit $M, N \in \M_n(\K)$. 
\begin{enumerate}
    \item ...
    \item \emph{Soit $\lambda \in \Ke$, montrer que $\dim(E_\lambda (MN)) = \dim(E_\lambda (NM))$.} \\
    On remarque que si $X \in E_\lambda (MN)$ alors $NX \in E_\lambda (NM)$. On pose alors:
    $$
        \fonction[\varphi]{E_\lambda (MN)}{E_\lambda (NM)}{X}{NX}.
    $$
    On montre que $\varphi$ est injective et on en déduit que $\dim(E_\lambda (MN)) \leqslant \dim(E_\lambda (NM))$. Par symétrie des rôles de $M$ et de $N$, on montre l'inégalité dans le sens inverse et on en déduit l'égalité.
    \item ...
    \item ...
\end{enumerate}


% \section{Spectre de \texorpdfstring{$\Id_E-uv$ et $\Id_E-vu$}{IdE-uv et IdE - vu}} \label{spectre_I-uv_et_I-vu}
% \begin{exercice}
    \marginnote[0cm]{\cite{exos_oraux}}
    Soient $u$ et $v$ deux endomorphismes d'un espace $E$ de dimension finie $n \in \Ne$. Prouver que $\Id_E - uv$ et $\Id_E - vu$ ont les mêmes valeurs propres. \\
    En déduire que $\Id_E - uv$ est inversible si $\Id_E - vu$ l'est aussi, relier les inverses.
\end{exercice}

Soient $u$ et $v$ deux endomorphismes d'un espace $E$ de dimension $n$.

\begin{enumerate}
    \item Montrer que $\Id_E-uv$ et $\Id_E-vu$ ont les mêmes valeurs propres.
    \begin{itemize}
        \item Montrer que ces deux endomorphismes ont même polynôme caractéristique en posant les deux matrices
        $$
        A = 
        \begin{pmatrix}
            U & \lambda \I_n \\
            \I_n & V
        \end{pmatrix}
        \text{ et }
        B = 
        \begin{pmatrix}
            V & -\lambda \I_n \\
            -\I_n & 0_n
        \end{pmatrix}.
        $$
        \begin{align*}
            \det(AB) &= \det(BA) \\
            (-\lambda)^n \det(AB - \lambda \I_n) &= (-\lambda)^n \det(BA - \lambda \I_n)
        \end{align*}
    \end{itemize}
    \item En déduire que $\Id_E-uv$ est inversible si et seulement si $\Id_E-vu$ l'est, relier les inverses. 
    \begin{itemize}
        \item $f \in \Gl(E) \Longleftrightarrow 0 \not \in \Sp(f)$
        \item Évaluer le résultat précédent en $\lambda = 1$.
        \item Analogie avec l'inversion par sommation géométrique des endomorphismes nilpotents (cf. \nameref{indice_nilpotence})
    \end{itemize}
\end{enumerate}

\section{Réduction simultanée}
\begin{defi}[Endomorphismes codiagonalisables/cotrigonalisables]
    Soit $E$ un espace vectoriel de dimension finie. Soient $u$ et $v$ deux endomorphismes de $E$ diagonalisables (resp. trigonalisables). \\
    On dit que $u$ et $v$ sont \emph{codiagonalisables} (resp. \emph{cotrigonalisables}) s'ils sont diagonalisables (resp. trigonalisables) dans une même base de $E$.
\end{defi}

\begin{prop}[CNS de réduction simultanée]
    Soient $u$ et $v$ deux endomorphimes diagonalisables (resp. trigonalisables). Alors, 
    $$u, v \text{ codiagonalisables (resp. cotrigonalisables)} \iff u, v \text{ commutent}.$$
\end{prop}

\begin{demo} 
    \source{\cite{acamanes} (Thème \emph{Diagonalisation simultanée} Ch. 11)}
    Montrons la résultat pour la codiagonalisation. Raisonnons par double implication.
    \begin{itemize}
        \item[$(\Rightarrow)$] Supponsons que les endomorphismes $u$ et $v$ sont codiagonalisables. On note $D$ (resp. $\widetilde{D}$) la matrice diagonale de $u$ dans une base de $E$ (resp. celle de $v$ dans cette même base). Comme ces matrices sont diagonales, $D \times \widetilde{D} = \widetilde{D} \times D$ \note \marginnote[0cm]{\note Le produit de matrices diagonales commute.} et on en déduit que $u$ et $v$ commutent.
        \item[$(\Leftarrow)$] Supposons que les endomorphismes $u$ et $v$ commutent. \\ 
        Notons $\mathrm{Sp}(u) \defeq \ens[\big]{ \lambda_i \tq i \in \llbracket1, p \rrbracket }$. Comme $u$ est diagonalisable, 
        $$E = \bigoplus\limits_{i = 1}^{p} E_{\lambda_i}(u).$$
        Soit $i \in \llbracket 1, p \rrbracket$. Comme $v$ commute avec $u - \lambda_i \mathrm{Id}_E$ \note 
        \marginnote[0cm]{
            \begin{align*}
                \note v \circ (u - \lambda_i \Id_E) &= v \circ u - \lambda_i v \\
                u \text{ et } v \text{ commutent }&= u \circ v - \lambda_i v \\
                &= (u - \lambda_i \Id_E) \circ v.
            \end{align*}
        }, $E_{\lambda_i}(u) \defeq \Ker(u - \lambda_i \Id_E)$ est stable par $v$ \note. \\
        En notant $v_i$ l'endomorphisme induit par $v$ sur $E_{\lambda_i}(u)$, comme $v$ est diagonalisable, $v_i$ est aussi diagonalisable. Ainsi, il existe $(e_{i, 1}, \dots, e_{i, r_i})$ une base de $E_{\lambda_i}(u)$ formée de vecteurs propres de $v_i$. De plus, $e_{i, j} \in E_{\lambda_i}(u) = \Ker(u - \lambda_i \Id_E)$. Ainsi, $u(e_{i,j}) = \lambda_i e_{i,j}$ et $e_{i,j}$ est un vecteur propre de $u$. \\
        Finalement, $(e_{i, 1}, \dots, e_{i, r_i})_{1 \leqslant i \leqslant p}$ est une base de $E$ constituée de vecteurs propres de $u$ et de $v$. Ainsi, $u$ et $v$ sont diagonalisables dans cette même base. 
    \end{itemize}
\end{demo}

\begin{remarque}
    \source{\cite{objectif_agregation} p. 167}
    \begin{itemize}
        \item La proposition établit que la commutativité est une condition suffisante pour réduire simultanément plusieurs endomorphismes. Remarquez que dans le cas de la diagonalisabilité, elle est aussi nécessaire: si des endomorphismes sont codiagonalisables, alors ils commutent. 
        \item Le seul fait que deux endomorphismes commutent n'implique par leur codiagonalisabilité, ni leur cotrigonalisabilité! Il ne faut pas oublier l'hypothèse faite sur chacun des endomorphismes (diagonalisabilité ou trigonalisabilité).
        \item La preuve de la proposition généralise la démonstration classique de la trigonalisabilité d'un endomorphisme dont le polynôme caractéristique est scindé: on raisonne par récurrence sur la dimension de l'espace. 
        \item Réduire simultanément plusieurs endomorphismes est très utile lorsqu'il s'agit de les ajouter ou de les composer. La proposition permet de démontrer, par exemple, que la somme d'endomorphismes trigonalisables qui commutent est encore trigonalisable, ou que la composée d'endomorphismes diagonalisables qui commutent est encore diagonalisables. 
        \item Supposons que $E$ soit un $\C$-espace vectoriel muni d'un produit scalaire hermitien. Soit $(f_i)_{i \in I}$ une famille d'endomorphismes de $E$ (qui sont donc trigonalisables). Si les $f_i$ commutent deux à deux, alors il existe une base orthonormée de cotrigonalisabilité. 
    \end{itemize}
\end{remarque}

\marginnote[-5cm]{
    \begin{theo}{\note Commutativité \& Stabilité}
        Soient $\varphi$ et $\psi$ deux endomorphismes qui commutent. Alors $\Im \varphi$ et $\Ker{\varphi}$ sont stables par $\psi$.
    \end{theo}
}

\begin{exercice}
    Soient $A$ et $B$ deux matrices de $\M_n(\C)$. On pose $\Phi_{A,B}$ l'endomorphisme de $\M_n(\C)$ défini pour tout $M \in \M_n(\C)$ par
    $$\Phi_{A,B}(M) \defeq AM + MB.$$
    \begin{enumerate}
         \item On suppose que la matrice $A$ est diagonalisable et que $B = 0$. Montrer que $\Phi_{A, B}$ est diagonalisable.
        \item On suppose que les matrices $A$ et $B$ sont diagonalisables. Montrer que $\Phi_{A, B}$ est diagonalisable. 
    \end{enumerate}
\end{exercice}

\begin{solution}
    \begin{enumerate}
        \item \textbf{1$^\text{ère}$ méthode.} Nous allons utiliser l'équivalence
        $$A \text{ diagonalisable} \iff \exists P \in \mathrm{Ann}(A) \text{ \textsc{sars}}.$$
        On remarque que $\Phi_{A, 0}: M \mapsto AM$. On montre par récurrence que 
        $$\Phi_{A, 0}^r(M) = A^r M.$$
        Ainsi, si $P$ est un polynôme, $P(\Phi_{A, 0}) = \Phi_{P(A), 0}$ \note. \\
        \marginnote[-1cm]{
            \note On note $P = \sum\limits_{k=0}^n p_k X^k$. Alors
            \begin{align*}
                P(\Phi_{A, 0}) &= \sum_{k=0}^n p_k \Phi_{A, 0}^k \\
                &= \sum_{k=0}^n p_k A^k M \\
                &= \left( \sum_{k=0}^n p_k A^k \right) M \\
                &= \Phi_{P(A), 0}.
            \end{align*}
        }
        Comme $A$ est diagonalisable, il existe un polynôme $P$ scindé à racines simples qui est un polynôme annulateur de la matrice $A$. Ainsi, $P(\Phi_{A, 0}) = \Phi_{0, 0} = 0_{\Endo \left( \M_n(\C) \right)}$. Comme $P$ est un polynôme scindé à racines simples qui est annulateur de $\Phi_{A, 0}$, alors $\Phi_{A, 0}$ est diagonalisable. \\
        \textbf{2$^\text{e}$ méthode.} Nous allons utiliser l'équivalence 
        $$A \text{  diagonalisable} \iff \exists \text{ une base de $E$ formée de vep de $A$}.$$
        Comme $A$ est diagonalisable, il existe une base $(V_1, \dots, V_n)$ de $\M_{n,1}(\C)$ formée de vecteurs propres de $A$ respectivement associés aux valeurs propres $(\lambda_1, \dots, \lambda_n)$. On définit en colonnes la matrices $M_{k, \ell} \defeq [ 0 \cdots 0\ V_k\ 0 \cdots 0]$ où $V_k$ est la colonne $\ell$ de la matrice. En effectuant une multiplication par blocs, on vérifie que $\Phi_{A, 0}(M_{k, \ell}) = A M_{k, \ell} = \lambda_k M_{k, \ell}$. On vérifie ensuite aisément que $(M_{k, \ell})_{1 \leqslant k, \ell \leqslant n}$ est une base de $\M_n(\C)$. On a ainsi exhibé une base de $\M_n(\C)$ formée de vecteurs propres de $\Phi_{A, 0}$. 
        \item On remarque que $\Phi_{A, 0}$ et $\Phi_{0, B}$ commutent. \\
        D'après la question précédente, $\Phi_{A, 0}$ est diagonalisable et on montrerait de même que $\Phi_{0, B}$ est diagonalisable. Ainsi, d'après la propriété de diagonalisation simultanée, $\Phi_{A, 0}$ et $\Phi_{0, B}$ sont simultanément diagonalisables. \\
        En notant $(M_k)$ une base de vecteurs propres communs, alors
        $$\Phi_{A, B} (M_k) = \Phi_{A, 0} (M_k) + \Phi_{0, B} (M_k) = (\lambda_k + \mu_k) M_k$$
        et $(M_k)$ est une base de vecteurs propres de $\Phi_{A, B}$. Ainsi, $\Phi_{A, B}$ est diagonalisable. 
    \end{enumerate}
\end{solution}

\marginnote[-10cm]{
    \begin{kaobox}[frametitle=Décomposition de $E$ en somme de sous-espaces stables supplémentaire]
        Si $E$ est de dimension finie non nulle et $\chi_f = \prod\limits_{i=1}^{k}(f-\lambda_i)^{\alpha_i}$, alors
        $$E = \bigoplus_{i=1}^{k} \Ker(f-\lambda_i \Id_E)^{\alpha_i}.$$
        \begin{itemize}
            \item Les $\Ker(f-\lambda_i \Id_E)^{\alpha_i}$ sont supplémentaires et stables par $f$. Donc, dans tout base adaptée à cette décomposition, la matrice de $f$ est diagonale par blocs. 
            \item La restriction de $f$ à $\Ker(f-\lambda_i)^{\alpha_i}$ induit un endomorphisme $f_i$ de ce sous-espace. $f_i$ admet une et une seule valeur propre à savoir $\lambda_i$ et $f_i - \lambda_i \Id_{\Ker(f-\lambda_i)^{\alpha_i}}$ est nilpotente d'indice inférieur ou égal à $\alpha_i$. 
        \end{itemize}
    \end{kaobox}
    Source: fiche de \cite{maths-france}.
}

\section{Critère de nilpotence par la trace} \labsec{critere_de_nilpotence_par_la_trace}
Les résultats que nous allons montrer par la suite peuvent être résumés par le diagramme suivant:
\begin{figure*}[h!]
    $$
    \begin{tikzcd}
    	& {\text{si pour tout } k \in \llbracket 1, n-1 \rrbracket, \mathrm{Tr}(A^k)=0 \text{ et si \dots}} & {} \\
    	{A \text{ est nilpotente}} && {A \text{ est diagonalisable}}
    	\arrow["{\dots \mathrm{Tr}(A^n) \not= 0}"{description}, curve={height=6pt}, Rightarrow, from=1-2, to=2-3]
    	\arrow["{\dots \mathrm{Tr}(A^n)=0}"{description}, curve={height=-6pt}, Rightarrow, 2tail reversed, from=1-2, to=2-1]
    \end{tikzcd}
    $$
\end{figure*}

Intéressons-nous d'abord à la branche de gauche.
\begin{prop}{}
    Soit $A \in \M_n(\K)$. La matrice $A$ est nilpotente si et seulement si pour tout $k \in \llbracket 1, n \rrbracket, \Tr(A^k) = 0$.
\end{prop}
\begin{preuve}
    \marginnote[-1cm]{\url{http://vonbuhren.free.fr/Agregation/Developpements/dev_thm_burnside.pdf}}
    Raisonnons par double implication.
    \marginnote[1cm]{
        \note Soit $A$ une matrice semblable à
        $$
        \begin{pmatrix}
            \lambda_1 & \star & \cdots & \star \\
            0 & \lambda_2 & \cdots & \star \\
            \vdots & \ddots &\ddots & \vdots \\
            0 & \cdots & 0 & \lambda_n
        \end{pmatrix}.
        $$
        Alors la matrice $A^k$ est semblable à
        $$
        \begin{pmatrix}
            \lambda_1^k & \star & \cdots & \star \\
            0 & \lambda_2^k & \cdots & \star \\
            \vdots & \ddots &\ddots & \vdots \\
            0 & \cdots & 0 & \lambda_n^k
        \end{pmatrix}.
        $$
    }
    
    %\marginnote[3cm]{
    % https://tex.stackexchange.com/questions/343439/how-to-draw-this-special-matrix-with-two-diagonal-braces
    %}
    \begin{itemize}
        \item[$(\Rightarrow)$] Si la matrice $A$ est nilpotente, son spectre est réduit à $0$ et donc elle est semblable à une matrice strictement triangulaire $T$. Pour tout $k \in \llbracket 1, n \rrbracket$, la matrice $A^k$ est semblable à la matrice $T^k$ dont la diagonale est nulle \note. La trace étant un invariant de similitude, on en déduit que pour tout $k \in \llbracket 1, n \rrbracket$, $\Tr(A^k) = \Tr(T^k) = 0$.
        \item[$(\Leftarrow)$] Réciproquement, supposons que la matrice $A$ n'est pas nilpotente. On désigne par $\lambda_1, \dots, \lambda_r \in \C$ les valeurs propres non nulles deux à deux distinctes de $A$ (qui existent car le polynôme $\chi_A \in \C[X]$ est scindé) et $m_1, \dots, m_r \in \Ne$ leur multiplicité respective. En trigonalisant la matrice $A$, notre hypothèse équivaut à
        $$\forall k \in \llbracket 1, n \rrbracket,\ \sum_{i=1}^r m_i \lambda_i^k = 0.$$
        En effet, la valeur propre $\lambda_i$ est présente $m_i$ fois sur la diagonale. \\
        En particulier, en se limitant à $k \in \llbracket 1, r \rrbracket$, ces relations se traduisent matriciellement par
        $$
        \underbrace{
        \begin{pmatrix}
        \lambda_1 & \cdots & \lambda_r \\
        \vdots & & \vdots \\
        \lambda_1^r & \cdots & \lambda_r^r
        \end{pmatrix}
        }_{\defeq V}
        \underbrace{
        \begin{pmatrix}
            m_1 \\ \vdots \\ m_r
        \end{pmatrix}
        }_{\defeq X}
        = 
        \begin{pmatrix}
        0 \\ \vdots \\ 0
        \end{pmatrix}.
        $$
        
        %$$
        %A^k \sim
        %\begin{tikzpicture}[decoration={brace,amplitude=5pt},baseline=(current bounding box.west)]
        %\matrix (magic) [matrix of math nodes,left delimiter=(,right delimiter=)] {
        %\lambda_1^k \\
        %& \ddots \\
        %& & \lambda_1^k \\
        %& & & \ddots \\
        %& & & & \lambda_r^k \\
        %& & & & & \ddots \\
        %& & & & & & \lambda_r^k \\
        %};
        %\draw[decorate] (magic-1-1.north) -- (magic-3-3.north east) node[above=5pt,midway,sloped] {$m_1$};
        %\draw[decorate] (magic-5-5.north east) -- (magic-7-7.north east) node[above=5pt,midway,sloped] %{$m_r$};
        %\end{tikzpicture}
        %$$

        Ainsi $X \in \Ker(V)$. Or la matrice $V$ est une matrice de \textsc{Vandermonde} inversible car les $\lambda_i$ sont deux à deux distinctes par hypothèse. Nous aboutissons alors à une contradiction car le vecteur $X$ est non nul. On en déduit que la matrice $A$ est nilpotente. \\
        
        Voyons une autre démonstration pour la réciproque. \\ \marginnote[0cm]{\url{https://www.youtube.com/watch?v=d70IfThN_-A}}
        \item[$(\Leftarrow)$] Montrons par récurrence que la matrice $A$ est nilpotente. Plus précisément pour $n \in \N$ on note
        \begin{center}
            $\mathscr{P}_n$: \say{ Soit $A \in \M_n(\K)$. Si pour tout $k \in \llbracket 1, n \rrbracket, \Tr(A^k) = 0$ alors la matrice $A$ est nilpotente }.
        \end{center}
        Montrons d'abord le lemme suivant.
        \begin{lemme}
            Soit $A \in \M_n(\K)$ telle que pour tout $k \in \llbracket 1, n \rrbracket, \Tr(A^k) = 0$. Montrons que $0 \in \Sp(A)$.
        \end{lemme}
        On note $\chi_A(X) \defeq \sum\limits_{k=0}^n a_k X^k$. D'après le théorème de \textsc{Cayley}-\textsc{Hamilton}, $\sum\limits_{k=0}^n a_k A^k = 0$. En composant cette relation par la trace, qui est linéaire, et d'après les hypothèses, 
        $$a_n \times 0 + \cdots + a_1 \times 0 + a_0 \times n = 0.$$
        D'où $a_0 = 0$ et $\chi_A(0) = 0$ donc $0$ est racine du polynôme caractéristique de la matrice $A$ i.e. $0 \in \Sp(A)$. \\
        
        Revenons à la démonstration de la récurrence.
        \begin{itemize}
            \item[$\rhd$] Initialisation pour $n=1$ : $\Tr(A) = 0$ donc $A = 0$ et $A$ est nilpotente. 
            \item[$\rhd$] Hérédité: soit $A \in \M_n(\K)$ telle que pour tout $k \in \llbracket 1, n \rrbracket, \Tr(A^k) = 0$. \\
            D'après le lemme, $0 \in \Sp(A)$. Soit $u$ un vecteur propre de $A$ associé à $0$ et soit $(u, u_2, \dots, u_n)$ une base de $\M_{n,1}(\K)$. Alors en notant $B$ une matrice carrée de taille $n-1$, 
            $$A \sim 
            \begin{pmatrix}
            0 & \star & \cdots & \star \\
            \vdots & & B & \\
            0 & & &
            \end{pmatrix}
            \text{ et }
            A^k \sim 
            \begin{pmatrix}
            0 & \star & \cdots & \star \\
            \vdots & & B^k & \\
            0 & & &
            \end{pmatrix}.
            $$
            D'après les hypothèses sur la trace des $A^k$, pour tout $k \in \llbracket 1, n-1 \rrbracket, \Tr(B^k) = 0$. Ainsi, en appliquant $\mathscr{P}_{n-1}$, la matrice $B$ est nilpotente. On en déduit, d'après les opérations sur les matrices par blocs, que
            $$\chi_A(X) = X \chi_{B}(X) = X \times X^{n-1} = X^n.$$
            Ainsi, d'après le théorème de \textsc{Cayley}-\textsc{Hamilton}, la matrice $A$ est nilpotente. 
        \end{itemize}
    \end{itemize}
\end{preuve}

Intéressons-nous maintenant à la branche de droite du diagramme.
\begin{prop}{}
    Soit $A \in \M_n(\K)$. Si pour tout $k \in \llbracket 1, n-1 \rrbracket, \Tr(A^k) = 0$ et $\Tr(A^n) \not= 0$ alors la matrice $A$ est diagonalisable. 
\end{prop}
Le démonstration de ce résultat reprend la démarche de la première démonstration de la réciproque du résultat précédent.
\newcommand{\vandermondepartiel}{
\left(\begin{gathered}
    \tikzpicture[every node/.style={anchor=south west}]
        \node[minimum width=1.5cm,minimum height=0.5cm] at (0.125,1.25) {\LARGE $V_k$};
        
        \node[minimum width=0.5cm,minimum height=0.5cm] at (0,0) {$\star$};
        \node[minimum width=0.5cm,minimum height=0.5cm] at (0.55,0) {$\cdots$};
        \node[minimum width=0.5cm,minimum height=0.5cm] at (1.25,0) {$\star$};
        
        \node[minimum width=0.5cm,minimum height=0.5cm] at (0,0.375) {$\vdots$};
        \node[minimum width=0.5cm,minimum height=0.5cm] at (1.25,0.375) {$\vdots$};
        
        \node[minimum width=0.5cm,minimum height=0.5cm] at (0,0.75) {$\star$};
        \node[minimum width=0.5cm,minimum height=0.5cm] at (0.55,0.75) {$\cdots$};
        \node[minimum width=0.5cm,minimum height=0.5cm] at (1.25,0.75) {$\star$};

        \draw (0, 1.25) -- (1.75, 1.25);
    \endtikzpicture
    \end{gathered}\right)
}
\begin{preuve}
    \begin{itemize}
        \item Montrons d'abord que la matrice $A$ possède au moins une valeur propre non nulle. \\
        La matrice $A$ est trigonalisable dans $\C$ et on note $\lambda_1, \dots, \lambda_n$ ses valeurs propres. Alors la matrice $A^n$ est semblable à la matrice
        $$
        \begin{pmatrix}
            \lambda_1^n & \star & \cdots & \star \\
            0 & \lambda_2^n & \cdots & \star \\
            \vdots & \ddots &\ddots & \vdots \\
            0 & \cdots & 0 & \lambda_n^n
        \end{pmatrix}.
        $$
        Or, par hypothèse, $\Tr(A^n) \not= 0$ donc les $\lambda_i$ ne peuvent pas être tous nuls et la matrice $A$ possède au moins une valeur propre non nulle.
        \item Montrons maintenant que la matrice $A$ possède $n$ valeurs propres distinctes ce qui assurera sa diagonalisabilité. \\
        On note $\alpha_1, \dots, \alpha_k$ les valeurs propres non nulles deux à deux distinctes de $A$ (qui existent d'après le premier point), $n_1, \dots, n_k$ leurs multiplicités respectives et 
        $$
        V \defeq \begin{pmatrix}
            \alpha_1 & \cdots & \alpha_k \\
            \vdots & & \vdots \\
            \alpha_1^{n-1} & \cdots & \alpha_k^{n-1}
        \end{pmatrix}
        \in \M_{n-1,k}(\C).
        $$
        Montrons que $N \defeq \Trsp{\begin{pmatrix} n_1  \cdots n_k \end{pmatrix}} \in \Ker(V)$. On calcule
        \begin{align*}
            V N = 
            \begin{pmatrix}
                \sum\limits_{i=1}^n \alpha_i n_i \\ 
                \vdots \\ 
                \sum\limits_{i=1}^n \alpha_i^{n-1} n_i
            \end{pmatrix}
            = 
            \begin{pmatrix}
                \Tr{A} \\ \vdots \\ \Tr{A^{n-1}}
            \end{pmatrix}
            =
            0_n
        \end{align*}
        Supposons par l'absurde que $k < n$. \\
        Réécrivons la relation précédente en extrayant de la matrice $V$ une matrice de \textsc{Vandermonde} carrée de taille $k$ notée $V_k$:
        $$V N = \vandermondepartiel N = 0_n$$
        et
        $$V_k N = 0.$$
        Comme $V_k$ est une matrice de \textsc{Vandermonde} et que les $\alpha_1, \dots, \alpha_k$ sont deux à deux distincts alors elle est inversible et donc
        $$N = 0.$$
        Ainsi, la matrice $A$ ne possède pas de valeur propre non nulle ce qui est absurde d'après le premier point. \\
        On en déduit que $k \geqslant n$ et comme $k \leqslant n$, $k=n$. Ainsi, la matrice $A$ possède $n$ valeurs propres distinctes et est donc diagonalisable.
    \end{itemize}
\end{preuve}

On déduit des propositions \textcolor{red}{4.6} et \textcolor{red}{4.7} le résultat suivant.
\begin{prop}{}
    Soit $A \in \M_n(\K)$. Si pour tout $k \in \llbracket 1, n-1 \rrbracket, \Tr(A^k) = 0$ alors la matrice $A$ est nilpotente ou diagonalisable.
\end{prop}

\marginnote[0cm]{(\url{https://fr.wikipedia.org/wiki/Polynôme_caractéristique#Coefficients})
$$\note \det(X \I_n - M) = X^n - f_1(M) X^{n-1} + \cdots + (-1)^n f_n(M)$$
où, en notant $(\lambda_1, \dots, \lambda_n)$ les valeurs propres de $M$ prises avec multiplicité,
$$f_k(M) = s_k(\lambda_1, \dots, \lambda_n)$$
où $s_k$ désigne le $k$-ème polynôme symétrique élémentaire. \\
Grâce aux identités de \textsc{Newton}, les coefficients $f_k(M)$ s'expriment comme des fonctions polynomiales des sommes de \textsc{Newton} des valeurs propres:
$$\sum_{i=1}^n \lambda_i^j = \Tr(M^j).$$
}

\begin{preuve}
    \textcolor{red}{à revoir} \\
    \marginnote[0cm]{\cite{reduc_des_endo} p. 114}
    D'après les formules des sommes de \textsc{Newton} et les relations coefficients-racines \note, le polynôme caractéristique de la matrice $A$ a pour expression
    $$\chi_A = X^n + (-1)^n \det(A).$$
    Si $\det(A) = 0$, le théorème de \textsc{Cayley}-\textsc{Hamilton} assure que $A$ est nilpotente. Sinon, le polynôme caractéristique est scindé à racines simples donc $A$ est diagonalisable (et d'ailleurs, toutes les valeurs propres ont le même module).
\end{preuve}


\section{Supplémentaire stable}

\begin{exercice}
    \marginnote[0cm]{\cite{reduc_des_endo} p. 102}
    Soit $u$ un endomorphisme diagonalisable de $E$ et $F$ un sous-espace de $E$. Montrer que $F$ admet un supplémentaire stable par $u$. 
\end{exercice}

\begin{solution}
    Considérons une base $(e_1, \dots, e_p)$ de $F$ et complétons cette famille libre en une base $(e_1, \dots, e_n)$ de $E$ avec des vecteurs $e_{p+1}, \dots, e_n$ issus d'une base de diagonalisation de $u$. Commes les vecteurs $e_{p+1}, \dots, e_n$ sont des vecteurs propres associés à $u$, le sous-espace $\Vect(e_{p+1}, \dots, e_n)$ est stable par $u$. Comme c'est un supplémentaire de $F$, il répond à la question. 
\end{solution} 

\section{Endomorphisme \texorpdfstring{$\mathrm{ad}_u$}{ad_u}}
\marginnote[0cm]{\cite{reduc_des_endo} p.30}
\begin{defi}{Endomorphisme $\mathrm{ad}_u$}
    Soit $u$ un endomorphisme de $E$. L'endomorphisme $\mathrm{ad}_u$ de $\Endo(E)$ est défini par
    \begin{alignat*}{2}
        \mathrm{ad}_u\ :\ \Endo(E)\ &\longrightarrow\ \Endo(E)\\
        v\ &\longmapsto\ u \circ v - v \circ u
    \end{alignat*}
\end{defi}

Cet endomorphisme nous sert essentiellement à mesurer le défaut de commutativité. Une première remarque dans ce sens est que le commutant de $u$ est $\mathscr{C}(u) = \Ker(\mathrm{ad}_u)$. 

\begin{exercice}
    \marginnote[0cm]{\cite{reduc_des_endo} p. 103}
    Soit $A \in \M_n(\K)$ une matrice diagonalisable. Montrer que $\mathrm{ad}_A$ est diagonalisable.
\end{exercice}

A rajouter:
\begin{itemize}
    \item Matrices dont le spectre est un singleton
    \item Endomorphismes symétriques à valeurs propres positives
    \item Critère de non-diagonalisabilité sur $\C$
    \item Commutant d'un endomorphisme diagonalisable 
\end{itemize}

\newpage

\begin{figure*}
    \begin{Large}
    \begin{align*}
        f \text{ diagonalisable} &\Longleftrightarrow \exists \mathscr{B} \text{ base de } E \text{ telle que } \Mat_\mathscr{B}(f) \text{ diagonale} \\
        &\Longleftrightarrow \exists \mathscr{B} \text{ base de } E \text{ formée de vep de } f\\
        &\Longleftrightarrow E = \bigoplus_{\lambda \in \Sp(f)} E_\lambda(f) \\
        &\Longleftrightarrow \dim E = \sum_{\lambda \in \Sp(f)} \dim E_\lambda(f) \\
        &\Longleftrightarrow 
        \begin{cases}
        \chi_f \text{ est scindé} \\
        \forall \lambda \in \Sp(f), m_{\lambda} = \dim E_\lambda(f)
        \end{cases} \\
        &\Longleftrightarrow \prod_{\lambda \in \Sp(f)}(X-\lambda) \in \mathrm{Ann}(f)\\
        &\Longleftrightarrow \exists P \in \mathrm{Ann}(f) \text{ SARS} \\
        &\Longleftarrow f \text{ possède } n \text{ vap distinctes} \\
        &\Longleftarrow \chi_f \text{ SARS} \\
        &\Longleftarrow \Mat(f) \in \mathscr{S}_n(\R)
    \end{align*}
    \end{Large}
\end{figure*}

\begin{figure*}[h!]
        \begin{tikzcd}[scale cd=0.85, ampersand replacement=\&]
	\& {} \& {} \& {} \\
	\& {\parbox{3.0cm}{\centering $P \in \mathrm{Ann}(f)$ \\ $\mathrm{Sp}(f) \subset \mathrm{Rac}(P)$}} \& {f \in \mathrm{GL} (E) \Leftrightarrow 0 \notin \mathrm{Sp}(f)} \\
	\& {\parbox{5.1cm}{\centering \textcolor{red}{\textsc{Spectre}}\\ $\lambda\in \mathrm{Sp}(f) \Leftrightarrow \exists x \not=0_E, f(x)=\lambda x$}} \& {} \\
	{\parbox{6.0cm}{\centering \textcolor{red}{\textsc{Polynôme caractéristique}} \\ $\chi_f(\lambda) = \det(\lambda\mathrm{Id}_E-f)$ \\ $m_f(\lambda)$: ordre de la racine $\lambda$ dans $\chi_f$}} \&\& {\parbox{4.0cm}{\centering \textcolor{red}{\textsc{Sous-espace propre}} \\ $E_{\lambda}(f)= \mathrm{Ker}(\lambda\mathrm{Id}_E-f)$}} \\
	{\parbox{7.0cm}{\centering $g$ endomorphisme induit par $f$ \\ $\chi_g |\chi _f$ \\ $\chi_f(\lambda) = \lambda^n -\mathrm{Tr}(f) \lambda^{n-1} + \cdots + (-1)^n \det(f)$}} \&\& {\parbox{7.0cm}{\centering Les sep sont des sev stables par $f$ \\ $f \circ g = g \circ f \Rightarrow E_\lambda(f)$ stable par $g$}}
	\arrow["{\dim E_\lambda(f) \leqslant m_f(\lambda)}"{description}, tail reversed, from=4-1, to=4-3]
	\arrow["{\small{\lambda \in \mathrm{Sp}(f) \Leftrightarrow \dim E_\lambda(f) \geqslant 1}}"{description, pos=0.7}, curve={height=-24pt}, tail reversed, from=3-2, to=4-3]
	\arrow[tail reversed, from=4-3, to=5-3]
	\arrow[tail reversed, from=4-1, to=5-1]
	\arrow[curve={height=-12pt}, tail reversed, from=3-2, to=2-3]
	\arrow[from=2-2, to=3-2]
	\arrow["{\parbox{4cm}{\centering \textsc{Cayley}-\textsc{Hamilton} \\ \chi_f (f) = 0}}"{description}, curve={height=30pt}, tail reversed, from=2-2, to=4-1]
	\arrow["{\small{\contour{white}{$\lambda \in \mathrm{Sp}(f) \Leftrightarrow \chi_\lambda(f) = 0$}}}"{description}, shift left=2, curve={height=24pt}, tail reversed, from=3-2, to=4-1]
\end{tikzcd}

\end{figure*}
\chapter{Espaces préhilbertiens ou euclidiens}
\labch{espaces_prehilbertiens_ou_euclidiens}

Lorsque $E$ est un espace euclidien, le procédé de \textsc{Gram}-\textsc{Schmidt} permet, à partir d'une base adaptée à un drapeau total de $E$, d'obtenir une base orthonormale adaptée à ce même drapeau. \\
Si l'on combine avec le théorème de trigonalisation utilisant les drapeaux, on constate que tout endomorphisme trigonalisable peut être trigonalisé dans une base orthonormale. \\

%\marginnote[-3cm]{
%    \begin{kaobox}[frametitle=Drapeau]
%        (Wiki) \\
%        Un \emph{drapeau} d'un espace vectoriel $E$ de dimension finie est une suite finie strictement croissante de sous-espaces vectoriels de $E$, commençant par l'espace nul $\{0_E\}$ et se terminant par l'espace total $E$:
%       $$\{0_E\} = E_0 \subsetneq E_1 \subsetneq \cdots \subsetneq E_k = E.$$
%        Si $\dim(E)=n$ et si pour tout $i \in \llbracket 1, k \rrbracket$, $\dim(E_i)=i$, alors le drapeau est dit \emph{total}.
%    \end{kaobox}
%}


\newpage

\section{Déterminant de \textsc{Gram}} \label{matrice_gram}
\begin{defi}{Matrice de \textsc{Gram}}
    Soient $E$ un espace euclidien et $(x_1, \dots, x_p) \in E^n$. On définit la matrice de \textsc{Gram} par
    $$\Gram(x_1, \dots, x_p) \defeq \big( \langle x_i, x_j \rangle \big)_{i,j \in \llbracket 1, p \rrbracket}.$$
\end{defi}
Le déterminant de \textsc{Gram} permet de calculer des volumes et de tester l'indépendance linéaire d'une famille de vecteurs.
\begin{prop}{}
    $$\Rg \big( \Gram(x_1, \dots, x_n) \big) = \Rg(x_1, \dots, x_n)$$
\end{prop}

\begin{preuve}
    \marginnote[0cm]{Source : \cite{maths-france} Planche 6 Maths SPE}
    Soient $F \defeq \Vect(x_1, \dots, x_n)$ et $m \defeq \dim F$. Soient $\mathscr{B} \defeq (e_i)_{1 \leqslant i \leqslant m}$ une base orthonormée de $F$ puis $M$ la matrice de la famille $(x_j)_{1 \leqslant j \leqslant n}$ dans la base $\mathscr{B}$. La matrice $M$ est une matrice rectangulaire de format $(m, n)$. \\
    Soit $(i, j) \in \llbracket 1, m \rrbracket \times \llbracket 1, n \rrbracket$. Puisque la base $\mathscr{B}$ est orthonormée, le coefficient $[ \Trsp{M} M ]_{i,j}$ est 
    $$\sum_{k=1}^m m_{k,i} m_{k,j} = \langle x_i, x_j \rangle,$$
    et on a donc
    $$\Gram(x_1, \dots, x_n) = \Trsp{M} M.$$
    Puisque $\Rg(x_1, \dots, x_n) = \Rg M$, il s'agit de vérifier que $\Rg(\Trsp{M} M) = \Rg M$. Pour cela, montrons que les matrices $M$ et $\Trsp{M} M$ ont le même noyau. \\
    Soit $X \in \M_{n,1}(\R)$.
    \begin{align*}
        X \in \Ker M &\implies MX = 0 \\
        &\implies \Trsp{M}MX = 0 \\
        &\implies X \in \Ker(\Trsp{M}M)
    \end{align*}
    et aussi
    \begin{align*}
        X \in \Ker \big( \Trsp{M}M \big) &\implies \Trsp{M}MX = 0 \\
        &\implies \Trsp{X}\Trsp{M}MX = 0 \\
        &\implies \Trsp{(MX)}MX = 0 \\
        &\implies \norme{MX}^2 = 0 \\
        &\implies MX = 0 \\
        &\implies X \in \Ker M.
    \end{align*}
    Finalement, $\Ker(\Trsp{M}M) = \Ker M$ et donc, d'après le théorème du rang,
    $$\Rg(x_1, \dots, x_n) = \Rg M = \Rg(\Trsp{M}M) = \Rg \big( \Gram(x_1, \dots, x_n) \big).$$
\end{preuve}

\begin{prop}{}
     La famille $(x_1, \dots, x_p)$ est liée si et seulement si $\det \Gram(x_1, \dots, x_p) = 0$ et est libre si et seulement si $\det \Gram(x_1, \dots, x_p) > 0$.
\end{prop}

\begin{preuve}
    \marginnote[0cm]{Source : \cite{maths-france} Planche 6 Maths SPE}
    D'après (??) et puisque $\Trsp{M}M \in \M_n(\K)$,
    \begin{align*}
        (x_1, \dots, x_n) \text{ liée } &\iff \Rg (x_1, \dots, x_n) < n \\
        &\iff \Rg \Gram(x_1, \dots, x_n) < n \\
        &\iff \Gram(x_1, \dots, x_n) \not \in \Gl_n(\R) \\
        &\iff \det \Gram(x_1, \dots, x_n) = 0.
    \end{align*}
    De plus, quand la famille $(x_1, \dots, x_n)$ libre, avec les notations de (??), on a $m=n$ et la matrice $M$ est une matrice carrée, inversible. On peut donc écrire
    $$\det \Gram(x_1, \dots, x_n) = \det \big( \Trsp{M} M \big) = \det(M)^2 > 0.$$
\end{preuve}

\begin{theo}{Distance à un sous-espace vectoriel} \labthm{distance_a_un_sous_espace_vectoriel}
    Soit $E$ un espace préhilbertien. Soit $F$ un sous-espace vectoriel de $E$ de dimension finie $p \in \Ne$ et soit $(e_1, \dots, e_p)$ une base de $F$. Alors pour tout $x \in E$,
    $$d(x, F)^2 = \frac{\Gram(e_1, \dots, e_p, x)}{\Gram(e_1, \dots, e_p)}.$$
\end{theo}

\begin{preuve}
    Soit $\pi_F(x)$ le projeté orthogonal de $x$ sur $F$. \\
    Alors $d(x, F)^2 = \norme{x - \pi_F(x)}^2$ et par \textsc{Pythagore},
    $$\norme{x}^2 = \norme{\pi_F(x)}^2 + \norme{x - \pi_F(x)}^2.$$
    De plus, 
    $$\forall k \in \llbracket 1, p \rrbracket,\ \langle x , e_k \rangle = \langle \pi_F(x) , e_k \rangle.$$
    On obtient alors
    \begin{align*}
        \Gram(e_1, \dots, e_p, x) &= 
        \begin{vmatrix}
          \begin{matrix}
            & & \\
            & \langle e_i, e_j \rangle & \\
            & &
          \end{matrix}
          & \rvline & \langle e_i, x \rangle \\
        \hline
          \langle x, e_j \rangle & \rvline &
          \begin{matrix}
          \norme{x}^2
          \end{matrix}
        \end{vmatrix} \\
        &=
        \begin{vmatrix}
          \begin{matrix}
            & & \\
            & \langle e_i, e_j \rangle & \\
            & &
          \end{matrix}
          & \rvline & \langle e_i, \pi_F(x) \rangle + 0 \\
        \hline
          \langle x, e_j \rangle & \rvline &
          \begin{matrix}
          \norme{\pi_F(x)}^2 + \norme{x - \pi_F(x)}^2
          \end{matrix}
        \end{vmatrix} 
    \end{align*}
    \marginnote[0cm]{
        \note On écrit la dernière colonne sous la forme
        $$
        \begin{pmatrix}
            \langle e_1, \pi_F(x) \rangle \\
            \vdots \\
            \langle e_p, \pi_F(x) \rangle \\
            \norme{\pi_F(x)}^2
        \end{pmatrix}
        + 
        \begin{pmatrix}
            0 \\
            \vdots \\
            0 \\
            \norme{x - \pi_F(x)}^2
        \end{pmatrix}.
        $$
    }
    Par linéarité du déterminant par rapport à la dernière colonne \note on obtient
    $$\Gram(e_1, \dots, e_p, x) = \Gram \big(e_1, \dots, e_p, \pi_F(x) \big) + \norme{x - \pi_F(x)}^2 \Gram(e_1, \dots, e_p).$$
    Comme $\pi_F(x) \in \Vect(e_1, \dots, e_p)$, le premier terme est nul et donc 
    $$d(x, F)^2 = \frac{\Gram(e_1, \dots, e_p, x)}{\Gram(e_1, \dots, e_p)}.$$
\end{preuve}

\begin{corol} \labthm{inegalite_gram}
    Soit $(x_1, \dots, x_n) \in E^n$. Alors,
    $$\Gram(x_1, \dots, x_n) \leqslant \prod_{i=1}^n \norme{x_i}^2$$
    avec égalité si et seulement si la famille $(x_1, \dots, x_n)$ est orthogonale. 
\end{corol}

Compléter avec \cite{objectif_agregation} p. 185.

\begin{preuve}
    \marginnote[0cm]{Source : \href{http://vonbuhren.free.fr/Agregation/Developpements/dev_determinant_gram.pdf}{Développement : Déterminant de \textsc{Gram} -- Jérôme \textsc{Von Buhren}, \textsf{vonbuhren.free.fr}}}
    \begin{itemize}
        \item Si la famille $(x_1, \dots, x_n)$ est liée, le résultat est immédiat.
        \item Raisonnons par récurrence sur $n \in \Ne$ sur la propriété
        \begin{center}
            $\mathscr{P}_n$: \say{ pour toute famille libre $(x_1, \dots, x_n)$ de $E$, on a $\Gram(x_1, \dots, x_n) \leqslant \prod\limits_{i=1}^n \norme{x_i}^2$ }.
        \end{center}
        \begin{itemize}
            \item[$\rhd$] Initialisation pour $n = 1$: soit $x_1 \in E$. Par définition, $\Gram(x_1) = \langle x_1, x_1 \rangle = \norme{x_1}^2$ donc $\mathscr{P}_1$ est vérifiée.
            \item[$\rhd$] Hérédité: supponsons $\mathscr{P}_n$ vraie. Soit $(x_1, \dots, x_n, x_{n+1})$ une famille libre de $E$. En notant $F \defeq \Vect(x_1, \dots, x_n)$, il existe $(f, \pi_F) \in F \times F^\perp$ tel que $x_{n+1} = f + \pi_F$. Par le théorème \vrefthm{distance_a_un_sous_espace_vectoriel} et par $\mathscr{P}_n$, 
            $$\Gram(x_1, \dots, x_{n+1}) = \Gram(x_1, \dots, x_n) \norme{\pi_F} \leqslant \norme{x_1}^2 \cdots \norme{x_n}^2 \norme{x_{n+1}}^2$$
            car par le théorème de \textsc{Pythagore}, $\norme{\pi_F} \leqslant \norme{x_{n+1}}$. On conclut que $\mathscr{P}_{n+1}$ est vraie, d'où le résultat. 
        \end{itemize}
    \end{itemize}
    \textcolor{red}{cas d'égalité}
\end{preuve}

\begin{prop}{}
    La matrice de \textsc{Gram} est symétrique positive.
\end{prop}

\marginnote[0cm]{
    \begin{defi}{Matrices symétriques positives}
        L'ensemble des \emph{matrices symétriques positives} est noté $\mathscr{S}_n^+(\R)$. Une matrice $M \in \mathscr{S}_n^+(\R)$ équivaut à chacune des propriétés suivantes:
        \begin{itemize}
            \item pour tout $X \in \M_n(\R), \Trsp{X} M X \geqslant 0$,
            \item $\Sp(M) \subset \Rp.$
        \end{itemize}
    \end{defi}
}

\begin{preuve}
    \begin{itemize}
        \item La matrice de \textsc{Gram} est symétrique par symétrie du produit scalaire.
        \item Montrons la positivité de $\Gram$. Soit $X = \Trsp{(\alpha_1 \cdots \alpha_n)} \in \M_{n,1}(\R)$. Montrons que $\Trsp{X} \Gram X \geqslant 0$. 
        \begin{align*}
            \Trsp{X} \Gram X &= \sum_{i=1}^{n} \sum_{j=1}^{n} \langle x_i, x_j \rangle \alpha_i \alpha_j \\ 
            &= \sum_{i=1}^{n} \sum_{j=1}^{n} \langle \alpha_i x_i, \alpha_j x_j \rangle \\
            &= \left\Vert \sum_{i=1}^{n}x_i \alpha_i \right\Vert^2 \geqslant 0.
        \end{align*}
    \end{itemize}
   
    Ce qui montre bien que $\Gram$ est symétrique positive.
\end{preuve}


\section{Positivité de la matrice de \textsc{Hilbert}}
\marginnote[0cm]{(Planche n°14. Espaces euclidiens de \cite{maths-france})}
Si on interprète le terme général de la matrice de \textsc{Hilbert} (cf. \nameref{matrice_hilbert}) comme 
$$\Hilb_{i,j} = \int_{0}^{1} x^{i+j-2} \d x$$
on peut y reconnaître une \nameref{matrice_gram} pour les fonctions puissances et le produit scalaire usuel sur l'espace des fonctions de $[0, 1]$ dans $\R$ de carré intégrable. Puisque les fonctions puissances sont linéairement indépendantes, les matrices de \textsc{Hilbert} sont donc définies positives.


\section{Décompositions matricielles}
Voir le thème 23 de \cite{acamanes}.
\subsection{Décomposition d'\textsc{Iwasama}}
\begin{tcolorbox}
    Soient $n \in \Ne$ et $M \in \Gl_n(\R)$. Il existe un \textbf{unique} couple $(T, O)$ tel que:
    $$M = OT,$$
    avec $T$ triangulaire supérieure à coefficients diagonaux strictement positifs et $O$ matrice orthogonale. 
\end{tcolorbox}

Comme $M$ est inversible, c'est une matrice de changement de base. \\
Faire un encadré sur le procédé de G.S. \\
Le produit et l'inversibilité sont stables dans $\mathscr{T}_n^+$.

\begin{itemize}
    \item \underline{Existence:} \\
    On note $\mathscr{B}$ la base canonique de $\R^n$. Soit $\mathscr{C} = (C_1, \dots, C_n)$ la famille des vecteurs colonnes de $M$ exprimés dans $\mathscr{B}$. Comme $M$ est inversible, $\mathscr{C}$ forme un \textbf{base} de $\R^n$. Appliquons-lui le \textbf{procédé d'orthonormalisation de \textsc{Gram}-\textsc{Schmidt}}. \\
    Il existe une base orthonormée $\mathscr{B}_o = (o_1, \dots, o_n)$ telle que pour tout $i \in \llbracket 1, n \rrbracket$
    $$\mathrm{Vect}(C_1, \dots, C_i) = \mathrm{Vect}(o_1,\dots, o_i) \quad (1) \quad \text{et} \quad \langle C_i, o_i \rangle > 0 \quad (2).$$
    On écrit $M = P_{\mathscr{B} \to \mathscr{C}} = P_{\mathscr{B} \to \mathscr{B}_o} \times P_{\mathscr{B}_o \to \mathscr{C}} = OT$. \\
    Le caractère triangulaire de $T = P_{\mathscr{B}_o \to \mathscr{C}}$ vient de $(1)$ et la stricte positivité de sa diagonale de $(2)$.
    \item \underline{Unicité:} \textcolor{green}{à compléter} \\
    Soit $M = OT = O'T'$. $T$ est inversible. $(O')^{-1}O =T'\Inv{T}$. Le premier terme est une matrice orthogonale et le second triangulaire supérieure car ces deux ensembles sont des groupes multiplicatifs. $B = T'\Inv{T}$ est diagonale (schéma) de coeff...
\end{itemize} 

\subsection{Décomposition polaire d'une matrice}
Lire chapitre 7 de \cite{matrices} page 77.

\section{Inégalité d'\textsc{Hadamard}}
%\begin{marginfigure}[-1cm]
%    \includegraphics[width=5cm]{images/jacques_hadamard.jpg}
%    \caption{Jacques \textsc{Hadamard}}
%\end{marginfigure}

\begin{theo}{Inégalité d'\textsc{Hadamard}}
    Soit $M \in \M_n(\C)$ et soient $X_1, \dots, X_n$ les vecteurs colonnes de $M$. Alors,
    $$|\mathrm{det}(M)| \leqslant \prod_{i=1}^{n} \Vert X_i \Vert$$
    avec égalité si et seulement si la famille $(X_i)$ est orthogonale.
\end{theo}

%\begin{marginfigure}[-2cm]
%    \begin{tikzpicture}[scale=0.7]
    \node[block] (gram) {Déterminant \\ de \textsc{Gram}};
    \node[block, right of=s1] (iwasama) {Décomposition \\ d'\textsc{Iwasama}};
    \node[block, below right of=gram] (hadamard) {Inégalité \\ d'\textsc{Hadamard}};
    \draw (gram) edge[bend right, above left] node {} (hadamard);
    \draw (iwasama) edge[bend left, above right] node {} (hadamard);
\end{tikzpicture}    
%\end{marginfigure}

Voyons deux démonstrations de ce résultat; une première en utilisant la décomposition d'\textsc{Iwasama} et une deuxième le \emph{corollaire 5.1}.

\begin{preuve}
    Si la matrice $M$ n'est pas inversible, le résultat est immédiat. Supposons que $M$ est inversible. D'après la décomposition d'\textsc{Iwasama}, il existe une matrice $O \in \mathscr{O}_n(\C)$ et $T \defeq (t_{i,j})$, triangulaire supérieure dont les coefficients diagonaux sont strictement positifs telles que $M = OT$. D'après la multiplicité du déterminant, 
    $$\det(M) = \underbrace{\det(O)}_{= \pm 1} \det(T)$$
    donc
    \begin{equation} \label{det}
        |\det(M)| = |\det(T)| = \prod_{i=1}^{n} |t_{i,i}|.
    \end{equation}
    Par construction, pour tout $i \in \llbracket 1, n \rrbracket, t_{i,i} = \langle X_i, O_i \rangle$ où $O_i$ est un vecteur unitaire. D'après l'inégalité de \textsc{Cauchy}-\textsc{Schwarz}, pour tout $i \in \llbracket 1, n \rrbracket$, 
    $$|t_{i,i}| = |\langle X_i, O_i \rangle | \leqslant \norme{X_i} \underbrace{\norme{O_i}}_{=1}.$$
    Ainsi d'après (\ref{det}), 
    $$|\det(M)| \leqslant \prod_{i=1}^n \norme{X_i}.$$
\end{preuve}

\begin{preuve}
    Si la matrice $M$ n'est pas inversible, le résultat est immédiat. Supposons que $M$ est inversible. On a $\Trsp{M} M = \Gram(X_1, \dots, X_n)\ (\star)$, la matrice de \textsc{Gram} de la famille des colonnes de la matrice $M$. D'après le (\ref{inegalite_gram}), 
    $$\Gram(X_1, \dots, X_n) \leqslant \prod_{i=1}^n \norme{X_i}^2.$$
    Ainsi, en composant $(\star)$ par le déterminant, 
    $$\det(\Trsp{M}M) = \det(M)^2 = \det \Gram(X_1, \dots, X_n) \leqslant \prod_{i=1}^n \norme{X_i}^2.$$
    En passant à la racine on obtient l'inégalité d'\textsc{Hadamard}.
\end{preuve}

L'inégalité d'\textsc{Hadamard} nous apprend en fait que le volume du parallélotope défini par les vecteurs colonnes est inférieur ou égal au produit des normes de ses vecteurs et il y a égalité si et seulement si la matrice est diagonale, ou encore que le parallélotope est rectangle. 

\marginnote[-2cm]{
    \begin{kaobox}[frametitle=Parallélotope]
        Soit $(x_1, \dots, x_n)$ une famille libre. Le parallélotope engendré par cette famille est défini par
        $$P \defeq \left\{ x = \sum_{i=1}^n t_i x_i,\ \forall i\ t_i \in [0,1]\right\}.$$
    \end{kaobox}
}

\begin{prop}
    Soient $\mathscr{S}_n ^{++} (\R)$ l'ensemble des matrices réelles d'ordre $n$ symétriques à valeurs propres strictement positives et $A = (a_{i,j}) \in \mathscr{S}_n ^{++} (\R)$. Alors,
    $$\det(A) \leqslant \prod_{i=1}^{n} a_{i,i}.$$
\end{prop}

\begin{exercice}    
exercice 4, TD 14 \cite{acamanes}
\begin{enumerate}
    \item Soit $(\gamma_1, \dots, \gamma_n) \in (\Re)^n$. Montrer que $B = (\gamma_i \gamma_j a_{i,j}) \in \mathscr{S}_n^{++}(\R)$. 
    \item Montrer que $\det(A)^{1/n} \leqslant \frac{\Tr(A)}{n}$. \\
    \emph{On pourra utiliser l'inégalité arithmético-géométrique}.
    
    \marginnote[-2cm]{
    	\begin{kaobox}[frametitle=Inégalité arithmético-géométrique]
            Soient $n \in \Ne$ et $x_1, \dots, x_n$ des réels positifs. Alors, 
            $$\frac{x_1 + \cdots + x_n}{n} \geqslant \sqrt[n]{x_1 \cdots x_n}.$$
            Il y a égalité si et seulement si tous les $x_i$ sont égaux.
        \end{kaobox}
        Pour d'autres inégalités, lire le chapitre 16, p.117 de la deuxième édition de Raisonnements divins (en fr)
    }
    \item Montrer que pour tout $i \in \llbracket 1, n \rrbracket, a_{i,i} > 0$. On pose $\gamma_i = \frac{1}{\sqrt{a_{i,i}}}$. En déduire l'inégalité d'\textsc{Hadamard}.
\end{enumerate}
\end{exercice}


\section{Familles de polynômes orthogonaux}
\begin{exercice}
    Exercice 17 Ch 13 \cite{acamanes}. \\
    Soient $I$ un intervalle non vide de $\R$ et $w \in \mathscr{C}(I, \Rpe)$. On suppose que, pour tout entier naturel $n$, $\int_I |x|^n w(x) \d x$ converge. On note $\mathscr{H} = \left\{ f \in \mathscr{C}(I, \R),\ \int_I f^2 w \text{ converge} \right\}$. Pour tout $(P, Q) \in \R[X]^2$, on pose $\langle P, Q \rangle = \int_I P(t) Q(t) w(t) \d t$.
    \begin{enumerate}
        \item Montrer que $\langle \cdot, \cdot \rangle$ est un produit scalaire sur $\R[X]$.
        \item Montrer qu'il existe une suite $(P_n)_{n \in \N}$ de polynômes tels que 
        \begin{itemize}
            \item pour tout $n \in \N, \deg P_n = n$,
            \item pour tout $(n, m) \in \N^2, n \not= m, \langle P_n, P_m \rangle = 0$,
            \item pour tout $n \in \N$, $P_n$ est unitaire.
        \end{itemize}
        Soit $n$ un entier naturel.
        \item Montrer que $\Vect(P_0, \dots, P_n) = \R_n[X]$.
        \item Montrer que $P_{n+1} \in \R_n[X]^\perp$.
        \item \textbf{Racines.} On note $(\alpha_i)_{i \leqslant i \leqslant k}$ les racines de $P_n$ qui appartiennent à $\mathring{I}$ et qui sont de multiplicité impaire. On pose $\Q = \prod\limits_{i=1}^k (X - \alpha_i)$.
        \begin{enumerate}
            \item Déterminer le degré de $Q$.
            \item Déterminer le signe de $P_n Q$ sur $I$.
            \item En déduire que $k = n$ et que $P_n$ a toutes ses racines réelles et simples dans $\mathring{I}$.
        \end{enumerate}
        \item \textbf{Relation de récurrence.}
        \begin{enumerate}
            \item Montrer que $(P_0, \dots, P_{n-1}, X P_{n-1})$ forme une base de $\R_n[X]$. \\
        On note $P_n = \sum\limits_{k=0}^{n-1} \alpha_k P_k + \alpha_n X P_{n-1}$.
            \item Montrer que, pour tout $j \in \llbracket 0, n - 3 \rrbracket, \alpha_j = 0$.
            \item En déduire qu'il existe trois suites réelles $(a_n), (b_n)$ et $(c_n)$ telles que 
            $$\forall n \in \N,\ P_{n+2} = (a_n X + b_n) P_{n+1} + c_n P_n.$$
        \end{enumerate}
    \end{enumerate}
\end{exercice}


\section{Polynômes orthogonaux associés à un poids}
\begin{defi}{}
    Soit $E = \mathscr{C}^0 \big( [-1, 1], \R \big)$ et $w$ continue et intégrable sur $]-1, 1[$, vérifiant pour tout $ t \in ]-1, 1[,\ w(t) > 0$. On définit: 
    $$\forall (f,g) \in E^2,\ \langle f, g \rangle = \int_{-1}^{1} f(t)g(t)w(t) \d t.$$
\end{defi}

\section{Polynômes de \textsc{Legendre} (bis)}
$$\forall n \in \N, \Leg_n(X) = \frac{1}{2^n n!} U_n^{(n)}(X)$$
où $U_n(X) = (X^2-1)^n$.

\begin{enumerate}
    \item Montrer que $(\Leg_n)_{n \in \N)}$ est une famille orthogonale. \\
    $-1$ et $+1$ sont des racines d'ordre $n$ de $U_n$ donc:
    $$\forall i \in \llbracket 1, n-1 \rrbracket,\ U_n^{(n)}(-1) = U_n^{(n)}(1) = 0 \quad (*)$$
    On calcule $\int_{-1}^{1} U_n^{(n)}(t) \times U_m^{(m)}(t)\ \mathrm{d}t$ en faisant une \textbf{intégration par parties} un intégrant $U_m^{(m)}$. D'après $(*)$, le crochet est nul. On répète l'opération $n+1$ fois. On obtient alors en facteur dans l'intégrande $U_n^{(2n+1)} = 0$ car $\mathrm{deg}(U_n) = 2n$.
\end{enumerate}

\section{Rayon spectral d'une matrice} \label{rayon_spectral}
\begin{tcolorbox}
    Soient $n \geqslant 2, M \in \M_n(\C)$. On définit son \emph{rayon spectral}:
    $$\rho(M) = \max \{ |\lambda |,\ \lambda \in \Sp_{\C}(M) \}.$$
\end{tcolorbox}

% \section{Caractétisation des projecteurs orthogonaux}
%\begin{prop}{}
    Soient $E$ un espace euclidien et $p$ un projecteur de $E$. Alors $p$ est un projecteur orthogonal si et seulement si, pour tout $x \in E$,
    $$\norme{p(x)} \leqslant \norme{x}.$$
\end{prop}

\begin{preuve}
    \begin{itemize}
        \item[$(\Rightarrow)$] Il existe $F$ un sev de $E$ tel que $p$ soit la projection sur $F$ parallèlement à $F^\perp$. On décompose tout vecteur de $E$ comme la somme unique d'un élément de $F$ et de $F^\perp$ puis on applique le théorème de \textsc{Pythagore}. 
        \item[$(\Leftarrow)$]
        \begin{itemize}
            \item \marginnote[0cm]{(Exos incontournables SUP)} Raisonner par l'absurde. Soit $F$ et $G$ tels que $p$ soit la projection sur $F$ parallèlement à $G$. Considérer un vecteur de $G^\perp \setminus F$ et aboutir à une contradiction.
            \item \marginnote[0cm]{(Ellipses p.176)} Soit $p$ une projection telle que pour tout $x \in E, \norme{p(x)} \leqslant \norme{x}$. \\
            Nous allons poser un vecteur dont la composante selon $\Im p$ sera variable.
            Soit $x \in \Ker p$ et $y \in \Im p$, pour tout $t \in \R$, 
            \begin{align*}
                \norme{ty} \leqslant \norme{x + ty} &\Leftrightarrow t^2 \norme{y}^2 \leqslant \norme{x+ty}^2 \\
                &\Leftrightarrow t^2 \norme{y}^2 \leqslant \norme{x}^2 + t^2 \norme{y}^2 + 2t \langle x, y \rangle \\
                &\Leftrightarrow \norme{x}^2 + 2t \langle x, y \rangle \geqslant 0 \\
                &\Rightarrow \langle x, y \rangle = 0 \text{ car cette inégalité est vraie pour tout } t \in \R
            \end{align*}
            Donc $\Ker p$ et $\Im p$ sont orthogonaux et $p$ est un projecteur orthogonal.
        \end{itemize}
    \end{itemize}
\end{preuve}

% \begin{marginfigure}[-4cm]
    % % \tdplotsetmaincoords{70}{200}

% \end{marginfigure}


\section{Famille obtusangle}
\begin{defi}
    Soit $E$ un espace euclidien de dimension $n \geqslant 2$. Soit $(x_1, \dots, x_p)$ une famille de vecteurs de $E$. On dit que cette famille est \emph{obtusangle} si et seulement si pour tout $i \not= j, \langle x_i, x_j \rangle < 0$. 
\end{defi}

\begin{exercice0}
    Soit $E$ un espace vectoriel de dimension $n$ et soit $(x_1, \dots, x_p)$ une famille obtusangle de $E$. Montrer que $p \leqslant n + 1$. 
\end{exercice0}


\section{Exercice 6.28 du ELLIPSES}
\begin{exercice}
    Soit $A \in \M_{1,n} (\R)$. Montrer que $B = A^\top A$ est diagonalisable et déterminer une matrice diagonale semblable à $B$.
\end{exercice}

\begin{solution}
    \begin{itemize}
        \item On montre facilement que $B$ est symétrique et comme cette matrice est réelle, elle est diagonalisable.
        \item Toutes les lignes de $B$ sont proportionnelles, et colinéaires à $A$; donc $B$ est de rang $1$ et $\dim E_0(B) = n-1$; la deuxième valeur propre de $B$ est: $\mathrm{Tr}(B) = \sum\limits_{k = 1}^{n} a_k^2$ en posant $A = (a_1, \dots, a_n)$. Enfin, $\Diag \left(\sum\limits_{k = 1}^{n} a_k^2, 0, \dots, 0 \right)$ est semblable à $B$.
    \end{itemize}
\end{solution}


\section{Exercice}
\begin{exercice}
    Issu de la RMS 132 3. \\
    Agrégation Interne de Mathématiques (première épreuve 2022) \\
    Vrai ou faux: \say{ les matrices carrées et symétriques à coefficients dans $\C$ sont diagonalisables. }
\end{exercice}

\begin{solution}
    \newline
    \begin{lemme}
        Une matrice nilpotente non nulle n'est pas diagonalisable.
    \end{lemme}
    
    \begin{preuve}
        Soit $A \in \M_n(\R)$ une matrice nilpotente diagonalisable. \\
        Alors il existe $P \in \Gl_n(\R)$ et $D$ une matrice diagonale telles que $A = PD\Inv{P}$. Or $A$ est nilpotente donc il existe $p \in \N$ tel que $A^p = P D^p \Inv{P} = 0$. Donc $D^p = 0$ soit $D = 0$ et $A = 0$. \\
        On peut aussi dire qu'une matrice ayant une unique valeur propre (comme c'est la cas des matrices nilpotentes) est diagonalisable si et seulement si elle est diagonale.
    \end{preuve}
    Cette affirmation est fausse. \\
    En effet en taille $2$, la matrice $A \defeq \begin{pmatrix}
        1 & \mi \\
        \mi & -1
    \end{pmatrix}$ est symétrique et non nulle; elle vérifie $A^2 = 0$. Or d'après le lemme, une matrice nilpotente non nulle n'est pas diagonalisable. En taille $n > 2$ la matrice $B$ telle que $[B]_{i,j} = [A]_{i,j}$ si $1 \leqslant i, j \leqslant 2$ et $[B]_{i,j} = 0$ sinon est elle aussi symétrique, nilpotente et non nulle et n'est donc n'est pas diagonalisable. 
\end{solution}   

\marginnote[-2cm]{
    $$
    B \defeq
    \begin{pmatrix}
    1 & \mi & 0 & \cdots & 0 \\
    \mi & -1 & 0 & \cdots & 0 \\
    0 & 0 & 0 & \cdots & 0 \\
    \vdots & \vdots & \vdots & \ddots & \vdots \\
    0 & 0 & 0 & \cdots & 0
    \end{pmatrix}
    $$
}

A rajouter:
\begin{itemize}
    \item Extremums d'une fonction sur les fonctions continues
    \item Racine carrée d'un endomorphisme autoadjoint positif
    \item Endomorphismes et matrices antisymétriques
\end{itemize}
\chapter{Espaces vectoriels normés, suites}
\labch{espaces_vectoriels_normes_suites}

\section{\texorpdfstring{$\e$}{e} est irrationnel}
\begin{prop}{}
    Le nombre $\e \defeq \exp(1)$ est irrationnel.
\end{prop}
\marginnote[0cm]{Une version de la preuve est dans Proofs from the BOOK (p.47)}
La démonstration suivante est due à Joseph \textsc{Fourier} (1815).
\begin{preuve}
    Supposons par l'absurde qu'il existe deux entiers $a$ et $b$ non nuls tels que $\e = \frac{a}{b}$. Alors, pour tout $n \geqslant 0$,
    $$n! b\, \e = n!\, a.$$
    Le terme de droite est un entier et le terme de gauche s'écrit \note \marginnote[0cm]{$\displaystyle \note\ \e = \sum_{n=0}^{+ \infty} \frac{1}{n!}$}
    $$n! b \left(1 + \frac{1}{1!} + \frac{1}{2!} + \cdots + \frac{1}{n!} + \frac{1}{(n+1)!} + \cdots \right)$$
    qui se décompose en la somme d'un entier
    $$b n! \left(1 + \frac{1}{1!} + \frac{1}{2!} + \cdots + \frac{1}{n!} \right)$$
    et d'un second membre
    $$b \left( \frac{1}{n+1} + \frac{1}{(n+1)(n+2)} + \frac{1}{(n+1)(n+2)(n+3)}+ \cdots \right).$$
    Or ce second membre n'est pas entier car pour $n > 1$,
    \begin{align*}
        0 &< \frac{1}{n+1} + \frac{1}{(n+1)(n+2)} + \frac{1}{(n+1)(n+2)(n+3)} + \cdots \\
        & < \frac{1}{n+1} + \frac{1}{(n+1)^2} + \frac{1}{(n+1)^3} + \cdots = \frac{1}{n+1} \cdot \frac{1}{1-\frac{1}{n+1}} = \frac{1}{n}.
    \end{align*}
    Ainsi le membre de gauche n'est pas entier et on aboutit à une contradiction. On en déduit que le nombre $\e$ est irrationnel.
\end{preuve}

\marginnote{
J. \textsc{Liouville} montre en 1840 que $\e^2$ est également irrationnel. (à compléter)
Charles \textsc{Hermite} montre en 1873 que $\e$ est \emph{transcendant}
\begin{defi}{Nombre transcendant}
\end{defi}
}

\section{Autour du lemme de \textsc{Cesàro}}

\subsection{Lemme de \textsc{Cesàro}, application à \texorpdfstring{$u_{n+1}=\sin(u_n)$}{u_n+1 = sin(u_n)}}
\begin{itemize}
    \item Voir la démarche de la démonstration du \nameref{lemme_cesaro}.
    \item Voir l'énoncé de l'exercice. 
\end{itemize}

\subsection{Une variante de \textsc{Cesàro}} \label{variante_cesaro}
\begin{exercice}
    Soit $(u_n)_{n \in \N}$ une suite réelle convergeant vers $\ell$. On définit une suite $(v_n)_{n \in \N}$ par 
    $$\forall n \in \N, v_n \defeq \frac{1}{2^n} \sum_{k=0}^{n} \binom{n}{k} u_k.$$
    Montrer que $\displaystyle \lim_{n \rightarrow + \infty} v_n = \ell$.
\end{exercice}

\begin{elem_sol}
    Le méthode consiste à se ramener au cas où $\ell = 0$ en posant deux suites auxiliares $u_n'=u_n - \ell$ et $v_n' = v_n - \ell$. La démarche est en suite analogue à la démonstration du \nameref{lemme_cesaro}.
\end{elem_sol}


\section{Normes \texorpdfstring{$\ell^p$}{l^p} et inégalités} \label{normes_lp_et_inegalites}
La suite définie une norme sur $\K^n$ et établit une généralisation de l'inégalité de \textsc{Cauchy}-\textsc{Schwarz}. \\
Le produit scalaire sur $\K^n$ est noté $\langle \cdot, \cdot \rangle$.

\begin{defi}{}
    Soit $u \in \K^n$. Pour tout réel $p \geqslant 1$, on définit l'application $\Vert \bm{\cdot} \Vert_p$ par
    $$\Vert u \Vert_p \defeq \left (\sum_{k=1}^{n} |u_i|^p \right)^{1/p}.$$
\end{defi}

\marginnote[0cm]{
    \begin{defi}{Norme}
        L'application $N : E \to \Rp$ est une \emph{norme} sur $E$ si
        \begin{itemize}
            \item[(\textsc{i})] \textbf{Séparabilité :} $\forall x \in E, N(x) = 0 \Leftrightarrow x = 0_E$.
            \item[(\textsc{ii})] \textbf{Homogénéité :} $\forall x \in E, \forall \lambda \in \K, N(\lambda x) = |\lambda| N(x).$
            \item[(\textsc{iii})] \textbf{Inégalité triangulaire :} $\forall (x, y) \in E^2, N(x+y) \leqslant N(x) + N(y)$.
        \end{itemize}
    \end{defi}
}

\begin{prop}{Normes $\ell^p$}
    Pour $1 \leqslant p \leqslant \infty$, l'application $x \mapsto \Vert x \Vert_p$ définie une norme sur $\K^n$. 
\end{prop}

\begin{prop}{Inégalité de \textsc{Minkowski}}
    En particulier, on a l'\emph{inégalité de \textsc{Minkowski}}
    $$\Vert x + y \Vert_p \leqslant \Vert x \Vert_p + \Vert y \Vert_p.$$
\end{prop}

Pour montrer que l'application $\Vert \bm{\cdot} \Vert_p$ définie bien une norme sur $\K^n$, il faut entre autre montrer qu'elle vérifie l'inégalité triangulaire et pour montrer cela nous allons d'abord démontrer l'inégalité de \textsc{Hölder}:

\begin{prop}{Inégalité de \textsc{Hölder}}
    En outre, on a l'\emph{inégalité de \textsc{Hölder}}: soient deux réels $p > 1$ et $q > 1$ tels que $\frac{1}{p} + \frac{1}{q} = 1$ ($p$ et $q$ sont \emph{conjugués}). Pour tout $u, v \in \K^n$
    $$\big|\langle u, v \rangle\big| \leqslant \Vert u \Vert_p \Vert v \Vert_q.$$
\end{prop}

%\begin{marginfigure}[0cm]
%    \centering
%    \input{illustrations/i_inegalite_convexite}
%\end{marginfigure}

\begin{preuve}
    \marginnote[0cm]{Source : \href{https://bibmath.net/dico/index.php?action=affiche&quoi=./h/holder.html}{Inégalité de \textsc{Hölder}} -- \textsf{Bibm@th.net}}
    \begin{itemize}
        \item Un lemme fondamental: \\
        D'après la concavité de la fonction logarithme, 
        $$\forall (x, y) \in \Rpe^2,\ \ln \left( \frac{x^p}{p} + \frac{y^q}{q} \right) \geqslant \frac{\ln(x^p)}{p} + \frac{\ln(y^q)}{q} = \ln(xy).$$
        En passant à l'exponentielle, on obtient
        $$xy \leqslant \frac{x^p}{p} + \frac{y^q}{q}.$$
        \item Un cas particulier crutial: \\
        Nous démontrons pour le moment l'inégalité dans le cas où $\sum\limits_{k=1}^n |u_k|^p = 1$ et $\sum\limits_{k=1}^n |v_k|^q = 1$. \\
        D'après le lemme, pour tout $k \in \llbracket 1, n \rrbracket$,
        $$|u_k| |v_k| \leqslant \frac{|u_k|^p}{p} + \frac{|v_k|^q}{q}.$$
        En sommant cette inégalité pour $k$ allant de $1$ à $n$, on obtient le résultat.
        \item Raisonnement par homogénéité: \\
        Pour obtenir le cas général, il suffit d'appliquer la cas particulier avec 
        $$|u'_k| = \frac{u_k}{\Vert u \Vert_p} \text{ et } |v'_k| = \frac{v_k}{\Vert v \Vert_p}.$$
    \end{itemize}
\end{preuve}

\begin{remarque}
    Pour $p = q = 2$, on retrouve l'inégalité de \textsc{Cauchy}-\textsc{Schwarz}.
\end{remarque}

Montrons maintenant que $\Vert \bm{\cdot} \Vert_p$ définie bien une norme sur $\K^n$.

\begin{preuve}
    \marginnote[-1cm]{(lire aussi \emph{Chapitre 4 - Normes} page 39 \cite{matrices}) Exercice 4.81 page 377 \cite{oraux_x_ens_3}}
    Tout d'abord, on s'assure que $\Vert \bm{\cdot} \Vert_p$ est bien à valeurs positives.
    \begin{itemize}
        \item[(\textsc{i})] \textbf{Séparabilité :}
        \begin{itemize}
            \item[$(\Leftarrow)$] Immédiat.
            \item[$(\Rightarrow)$] Soit $u \in \K^n$ tel que $\Vert \lambda u \Vert_p = 0$. Alors
            $$\sum_{k=1}^{n} |\lambda u_i|^p = 0$$
            qui est une somme de termes positifs donc chacun des termes est nul et $$\forall k \in \llbracket 1, n \rrbracket,\ u_k = 0$$
            ce qui assure que $u = 0$.
        \end{itemize}
        \item[(\textsc{ii})] \textbf{Homogénéité :} soient $u \in \K^n$ et $\lambda \in \K$,
        $$\Vert \lambda u \Vert_p = \left (\sum_{k=1}^{n} |\lambda u_i|^p \right)^{1/p} ...$$
        \item[(\textsc{iii})] \textbf{Inégalité triangulaire :} 
        Soient $u, v \in \K^n$, montrons que 
        $$\Vert u + v \Vert_p \leqslant \Vert u \Vert_p \Vert v \Vert_p.$$
        Soit $k \in \llbracket 1, n \rrbracket$,
        $$|u_k + v_k|^p = |u_k| \times |u_k + v_k|^{p-1} + |v_k| \times |u_k + v_k|^{p-1}$$ 
        et sommer pour $k$ allant de $1$ à $n$. Appliquer le résultat précédent à chaque somme, factoriser et multiplier l'inégalité par une somme judicieuse. 
    \end{itemize}
\end{preuve}    

\begin{prop}{}
    \marginnote[0cm]{Source : Proposition 4.1.3. \cite{matrices}}
    Toutes les normes de $E = \K^n$ sont équivalentes. Par exemple:
    $$\Vert x \Vert_\infty \leqslant \Vert x \Vert_p \leqslant p^{1/p} \Vert x \Vert_\infty$$
\end{prop}





\pagelayout{wide} % No margins
\addpart{Analyse}
\pagelayout{margin} % Restore margins

\chapter{Suites \& Séries numériques}
\labch{suites_et_series_numeriques}

\emph{\say{ Les séries divergentes sont des inventions du diable, et c'est une honte que l'on ose fonder sur elles la moindre démonstration. On peut en tirer  tout ce qu'on veut quand on les emploie et ce sont elles qui ont produit tant d'échecs et tant de paradoxes. }}
\begin{flushright}
\textsc{--- Niels Abel}, \emph{Œuvres, 1881}
\end{flushright}

\newpage

\section{Lemme de \textsc{Cesàro}} \label{lemme_cesaro}
\begin{lemme}
    Soit $(u_n)_{n \in \Ne}$ une suite réelle ou complexe convergeant vers $\ell$.
    Alors la suite de terme général $\frac{1}{n} \sum\limits_{k=1}^{n} u_k$ converge aussi vers $\ell$.
\end{lemme}

\begin{preuve}
    Soit $\varepsilon > 0$. Comme la suite $(u_n)$ converge vers $\ell$, il existe un rang $n_0 \in \Ne$ tel que pour tout $n \geqslant n_0,\ |u_n - \ell| \leqslant \varepsilon$. \\
    Soit $n \geqslant n_0$,
    \begin{align*}
        \left| \frac{1}{n} \sum_{k=1}^n u_k - \ell \right| &= \left| \frac{1}{n} \sum_{k=1}^n (u_k - \ell) \right| \\
        \text{par l'inégalité triangulaire} &\leqslant \frac{1}{n} \sum_{k=1}^n |u_k - \ell| \\
        &\leqslant \frac{1}{n} \Bigg( \underbrace{\sum_{k=1}^{n_0-1} |u_k - \ell|}_{\defeq K} + \sum_{k=n_0}^n \underbrace{|u_k - \ell|}_{\leqslant \varepsilon} \Bigg) \\
        &\leqslant \frac{K}{n} + \varepsilon
    \end{align*}
    Or $\lim\limits_{n \to \infty} \frac{K}{n} = 0$ donc il existe un rang $n_1 \in \Ne$ tel que pour tout $n \geqslant n_1, \left| \frac{K}{n} \right| \leqslant \varepsilon$. \\
    Ainsi pour tout $n \geqslant \max \{ n_0, n_1 \}$, 
    $$\left| \frac{1}{n} \sum_{k=1}^n u_k - \ell \right| \leqslant 2 \varepsilon.$$
    On en déduit que la suite $\Bigg( \frac{1}{n} \sum\limits_{k=1}^{n} u_k \Bigg)_{n \in \Ne}$ converge vers $\ell$.
\end{preuve}

\begin{remarque}
    \textcolor{red}{à réécrire}
    Attention, la réciproque du lemme de \textsc{Cesàro} est fausse. Une suite $(u_n)$ peut converger au sens de \textsc{Cesàro} i.e. $\Bigg( \frac{1}{n} \sum\limits_{k=1}^{n} u_k \Bigg)_{n \in \Ne}$ converge sans pour autant que la suite $(u_n)$ converge. Par exemple, $(u_n) \defeq \left((-1)^n\right)_n$.
\end{remarque}

\section{Constante d'\textsc{Euler}}
\begin{tcolorbox}
    La constante d'\textsc{Euler} $\gamma$ est définie par:
    $$\gamma = \lim_{n \to \infty} \left(\sum_{k=1}^{n} \frac{1}{k} - \ln(n) \right) \approx 0,577 215 664 \dots$$
\end{tcolorbox}

\begin{enumerate}
    \item Poser $v_n = H_n - \ln(n)$
    \item Montrer que $v_{n+1}-v_n = \mathcal{O}\left(\frac{1}{n^2}\right)$\\
    On peut aussi montrer...
    \item ... la décroissance de la suite $(v_n)$. \\
    Soit $n \in \N$. 
    $$v_n - v_{n+1} = \ln(n+1) - \ln(n) - \frac{1}{n+1}$$
    Deux méthodes:
    \begin{itemize}
        \item On transforme $\ln(n+1) - \ln(n)$ en intégrale:
        $$v_n - v_{n+1} = \int_{n}^{n+1} \underbrace{\left( \frac{1}{t} - \frac{1}{n+1} \right)}_{\geqslant 0}\ \d t > 0.$$
        \item D'après le \textbf{théorème des accroissements finis}, il existe $c \in ]n, n+1[$ tel que 
        $$\ln(n+1) - \ln(n) = \ln'(c)((n+1) - n) = \frac{1}{c}$$
        d'où l'on tire que 
        $$v_n - v_{n+1} = \frac{1}{c} - \frac{1}{n+1} > 0.$$
    \end{itemize}
    \item ... que $\boxed{\forall n \in \Ne,\ \ln(n+1) \leqslant H_n \leqslant \ln(n) + 1}$ grâce à l'\textbf{encadrement de l'intégrale} sur $[k, k+1]$ de la fonction $t \mapsto \frac{1}{t}$.
\end{enumerate}

\section{Séries de \textsc{Bertrand}}
\begin{defi}{Séries de \textsc{Bertrand}}
    Soient $\alpha$ et $\beta$ deux réels. On nomme \emph{série de \textsc{Bertrand}} la série de terme général $\displaystyle \frac{1}{n^\alpha \ln^\beta (n)}$ pour $n \geqslant 2$. 
\end{defi}

\begin{theo}{}
    La série de \textsc{Bertrand} converge si et seulement si \begin{cases} \alpha > 1 \\
    \text{ou} \\ \alpha = 1 \text{ et } \beta > 1 \end{cases}.
\end{theo}

\begin{preuve}
    Distinguons trois cas selon les valeurs prises par $\alpha$:
    \begin{enumerate}
        \item[$\rhd$] si $\alpha > 1$, soit $\gamma \in ]1, \alpha[$. Par croissances comparées,
        $$\displaystyle \frac{1}{t^{\alpha} \ln^{\beta} (t)} = o_{+ \infty} \left( \frac{1}{t^{\gamma}} \right).$$
        Or, d'après le théorème de \textsc{Riemann}, la fonction $t \mapsto \frac{1}{t^\gamma}$ est intégrable sur $[2, +\infty[$ car $\gamma > 1$. Ainsi, en appliquant les théorèmes de comparaison, $\int_2^{+ \infty} f$ converge.
        \item[$\rhd$] si $\alpha < 1$, soit $\gamma \in ]\alpha, 1[$.Par croissances comparées,
        $t^{\gamma} f(t) \xrightarrow[t \to + \infty]{} + \infty$
        donc à partir d'un certain rang, $f(t) \geqslant \frac{1}{t^{\gamma}} > 0$. Or, d'après le théorème de \textsc{Riemann}, la fonction $t \mapsto \frac{1}{t^\gamma}$ n'est intégrable pas sur $[2, +\infty[$ car $\gamma < 1$. Ainsi, en appliquant les théorèmes de comparaison (les intégrandes sont positives), $\int_2^{+ \infty} f$ diverge.
        \item[$\rhd$] si $\alpha = 1$, revenons aux intégrales partielles:
        $$\int_{2}^{X} \frac{1}{t \ln^{\beta} (t)} \d t = 
        \begin{cases}
            \left[ \frac{\ln ^{1-\beta} (t)}{1-\beta} \right]_2 ^X & \text{si } \beta \not = 1, \\
            \left[\ln (\ln(t)) \right]_2 ^X & \text{si } \beta = 1.
        \end{cases}
        $$
        On en déduit que l'intégrale de la fonction $t \mapsto \frac{1}{t \ln^{\beta} (t)}$ converge sur $[2, + \infty[$ si et seulement si $\beta > 1$.
    \end{enumerate}
\end{preuve}


\begin{exercice}
    \marginnote[0cm]{\cite{acamanes}}
    On note $h : x \mapsto \sum\limits_{n=2}^{+ \infty} \frac{1}{n^x \ln n}$.
    \begin{enumerate}
        \item Étudier la continuité de $h$ sur son domaine de définition.
        \item Étudier les limites de $h$ aux bornes de son intervalle de définition.
        \item Déterminer des équivalents de $h$ aux bornes de son intervalle de définition.
    \end{enumerate}
\end{exercice}

\begin{solution}
\begin{enumerate}
    On note $f_x : t \mapsto \frac{1}{t^x \ln t}$.
    \item D'après le théorème de \textsc{Bertrand} sur les séries numériques, l'ensemble de définition de $h$ est $\mathcal{D}_h \defeq ]1, + \infty[$. \\
    Soit $a > 1$. On se place sur $I \defeq [a, +\infty[$. Pour tout $x \in I$,
    $$\left| \frac{1}{n^x \ln n} \right| \leqslant \frac{1}{n^\alpha \ln n}$$
    comme $a > 1$, d'après le théorème de \textsc{Bertrand} sur les séries numériques, la série du terme majorant converge et donc par théorème de comparaison, la suite $(f_x)$ converge normalement sur tout segment de la forme de $I$. On en déduit que $h$ est continue sur $\mathcal{D}_h$. 
    \begin{itemize}
        \item En $+ \infty$: comme la série des $f_x$ converge uniformément sur $[2, + \infty[$ (on aurait pu choisir une valeur que $2$), d'après le théorème de la double limite
        $$\lim_{x \to + \infty} h(x) = \sum_{n=2}^{+ \infty} \left[\lim_{x \to +\infty} f_x(n) \right] = 0.$$
        \item En $1^+$: On montre que la fonction $f_x$ est décroissante et donc $h$ aussi. On note $\ell \defeq \lim\limits_{1^+} h$. D'après le théorème de la limite monotone, $\ell \in \R \cup \{ + \infty \}$. \\
        Supponson que $\ell \in \R$, alors
        $$h(x) = \sum_{n=2}^{+\infty} \frac{1}{n^x \ln n} \geqslant \sum_{n=2}^N \frac{1}{n^x \ln n}$$
        et en passant à la limite quand $x$ tend vers $1^+$ dans l'inégalité (ce qui est licite) on obtient
        $$\ell \geqslant \sum_{n=2}^N \frac{1}{n \ln n}.$$
        Nous aboutissons donc à une contradiction car la somme minorante diverge quand $N$ tend vers $+ \infty$. Finalement,
        $$\lim_{1^+} h = + \infty.$$
    \end{itemize}
    \item 
    \begin{itemize}
        \item En $+ \infty$: on intuite que la premier terme de la somme domine les autres. On a
        $$2^x \ln(2) h(x) = 1 + \sum_{n=3}^{+\infty} \left(\frac{2}{n}\right)^x \frac{\ln 2}{\ln n}.$$
        On peut montrer(...) que la somme converge normalement sur $[2, +\infty[$. On en déduit que 
        $$h(x) \isEquivTo{+\infty} \frac{1}{2^x \ln 2}.$$
        \item En $1^+$: un encadrement par la méthode des rectangles permet de trouver
        $$\int_{3}^{+\infty} f_x(t) \d t + \frac{1}{2^x \ln 2} \leqslant h(x) \leqslant \int_{2}^{+\infty} f_x(t) \d t + \frac{1}{2^x \ln 2}.$$
        On en déduit que 
        $$h(x) \isEquivTo{1^+} \int_2^{+\infty} f_x(t) \d t$$
        soit après calculs (...)
        $$h(x) \isEquivTo{1^+} - \ln(x-1).$$
    \end{itemize}
\end{enumerate}
\end{solution}

\section{Deux sommes} \label{deux_sommes}
$\displaystyle \sum_{n=1}^{+\infty} \frac{(-1)^n}{n}$ et $\displaystyle \sum_{n=0}^{+\infty} \frac{(-1)^n}{2n+1}$

\begin{itemize}
    \item Exprimer les termes généraux avec une intégrale. 
    \item (1)$= -\ln(2)$, (2)$= \frac{\pi}{4}$.
\end{itemize}

\section{Sommation des relations de comparaison} \label{sommation_relations_comparaison}
\begin{prop}{}
    Soient $(a_n)_{n \in \Ne}$ et $(b_n)_{n \in \Ne}$ deux suites à valeurs positives telles que $a_n \sim b_n$.
    \begin{itemize}
        \item Si $ \sum a_n$ diverge, alors $\sum\limits_{k=1}^{n} a_k \sim \sum\limits_{k=1}^{n} b_k$. 
        \item Si $ \sum a_n$ converge, alors $\sum\limits_{k=n+1}^{+ \infty} a_k \sim \sum\limits_{k=n+1}^{+ \infty} b_k$. 
    \end{itemize}
\end{prop}

\emph{Il y a des résultats analogues si $a_n = o(b_n)$ ou si $a_n = \mathcal{O}(b_n)$.}

\begin{marginfigure}[3cm]
    \centering
    \caption*{\centering Diagramme de la démonstration}
    \begin{tikzcd}
    a_n \sim b_n \arrow[r, Rightarrow, "1"] & a_n - b_n = o(a_n) \arrow[d, Rightarrow, "2"]\\
    A_n \sim B_n \arrow[r, Leftarrow, "3"] & A_n - B_n = o (A_n)
    \end{tikzcd}
\end{marginfigure}

\begin{preuve}
    On suppose que $\sum a_n$ diverge. On sait que $a_n \sim b_n$ est équivalent à $a_n -b_n = o(b_n)$, autrement dit, pour $\varepsilon > 0$, il existe un rang $n_0$ à partir duquel $|a_n -b_n| \leqslant \varepsilon a_n$. \\
    Pour tout $n \in \Ne$, on note $A_n \defeq \sum\limits_{k=1}^n a_k$ et $B_n \defeq \sum\limits_{k=1}^n b_k$. \\
    Soit $n > n_0$,
    \begin{align*}
        |A_n - B_n| &= \left|\sum_{k=1}^n (a_k - b_k) \right| \\
        \text{par l'inégalité triangulaire} &\leqslant \sum_{k=1}^n |a_n-b_n| \\
        &\leqslant \underbrace{\sum_{k=1}^{n_0-1} |a_k - b_k|}_{\defeq C} + \sum_{k=n_0}^n \underbrace{|a_k - b_k|}_{\leqslant \varepsilon a_k} \\
        &\leqslant C + \varepsilon \sum_{k=n_0}^n a_k \\
        \text{comme $(a_n)$ est à valeurs positives} &\leqslant C + \varepsilon A_n
    \end{align*}
    Comme $\sum a_n$ est divergente et à valeurs positives, $A_n \longrightarrow +\infty$ et donc à partir d'un certain rang $n_1$, $A_n \varepsilon \geqslant C$. Ainsi pour $n \geqslant \max \{ n_0, n_1 \}$, $|A_n - B_n| \leqslant 2 \varepsilon A_n$. \\ 
    Donc $A_n - B_n = o(A_n)$ ce qui équivaut à $A_n \sim B_n$.
\end{preuve}


\section{Formule de \textsc{Stirling}}
\label{preuve_stirling}
\begin{theo}{}
    $$n! \sim \left(\frac{n}{\me}\right)^n \sqrt{2 \pi n}$$
\end{theo}

\begin{elem_preuve}
    \begin{enumerate}
        \item Montrer que $\left( \frac{n!\me^n}{\sqrt{n}n^n} \right)_{n \in \Ne}$ converge vers $\ell \in \R$
        \item Montrer que $\ell = \sqrt{2 \pi}$ en utilisant l'\nameref{integrale_wallis}:
        $$\Wallis_n \defeq \int_{0}^{\pi/2}\sin^n(t) \d t$$
        \begin{enumerate}
            \item Montrer que $(\Wallis_n)_{n \in \N}$ est décroissante
            \item Exprimer $\Wallis_{n+2}$ en fonction de $\Wallis_n$ grâce à une IPP: $$(n+2)\Wallis_{n+2} = (n+1)\Wallis_n$$
            \item Exprimer $\Wallis_{2p}$ et $\Wallis_{2p+1}$ en fonction de $p$:\\
            $$\Wallis_{2p} = \frac{\binom{2p}{p}}{2^{2p}}\frac{\pi}{2} \text{ et } \Wallis_{2p+1} = \frac{2^{2p} (p!)^2}{(2p+1)!}$$
            \item Utiliser les points (a) et (b) pour montrer que $\frac{\Wallis_n}{\Wallis_{n+1}} \longrightarrow 1$
            \item Utiliser les points (c) et (d) pour montrer que $\left ( \frac{2^n n!}{n ((2n)!)^2} \right)^4 \longrightarrow \pi$
            \item Utiliser le point 1. pour déterminer $\ell$
        \end{enumerate}
    \end{enumerate}
\end{elem_preuve}

\begin{preuve}
    Calculons $\Wallis_{n+2}$ en effectuant une intégration par parties. On pose $u(t) \defeq - \cos(t)$ et $v(t) \defeq \sin^{n+1}(t)$, toutes deux de classe $\mathscr{C}^1$ sur $\left[0, \frac{\pi}{2} \right]$. 
    \begin{align*}
        \Wallis_{n+2} &= \underbrace{\left[ -\cos(t) \sin^{n+1}(t) \right]_0^{\pi/2}}_{=0} + (n+1) \int_0^{\pi/2} \cos^2 (t) \sin^n(t) \d t \\
        &= (n+1) \int_0^{\pi/2} (1 - \sin^2(t)) \sin^n(t) \d t \\
        &= (n+1) \Wallis_n - (n+1) \Wallis_{n+2} \\
        \text{soit } (n+2) \Wallis_{n+2} &= (n+1) \Wallis_n.
\end{align*}
Soit $p \in \N$. D'après la relation précédente, 
\begin{figure*}[h!]
\begin{multicols}{2}
\begin{align*}
    \Wallis_{2p} &= \frac{2p-1}{2p} \Wallis_{2p-2} \\
    &= \frac{2p-1}{2p} \times \frac{2p-3}{2p-2} \times \cdots \times \frac{1}{2} \times \underbrace{\Wallis_0}_{=\pi/2} \\
    &= \frac{\prod\limits_{k=1}^p (2k+1)}{\prod\limits_{k=1}^{p+1} (2k)} \frac{\pi}{2} \\
    &= \frac{\left[\prod\limits_{k=1}^p (2k+1) \right] \times \left[ \prod\limits_{k=1}^{p+1} (2k) \right]}{\left[\prod\limits_{k=1}^{p+1} (2k) \right]^2} \frac{\pi}{2} \\
    \Wallis_{2p} &= \frac{(2p)!}{2^{2p}(p!)^2} \frac{\pi}{2}.
\end{align*}
\begin{align*}
    \Wallis_{2p+1} &= \frac{2p}{2p+1} \Wallis_{2p-1} \\
    &= \frac{2p}{2p+1} \times \frac{2p-2}{2p-1} \times \cdots \times \frac{2}{3} \times \underbrace{\Wallis_1}_{=1} \\
    &= \frac{\prod\limits_{k=1}^p (2k)}{\prod\limits_{k=0}^p (2k+1)} \\
    &= \frac{\left[ \prod\limits_{k=1}^p (2k) \right]^2}{\left[ \prod\limits_{k=0}^p (2k+1) \right] \left[ \prod\limits_{k=1}^p (2k) \right]} \\
    \Wallis_{2p+1} &= \frac{2^{2p}(p!)^2}{(2p+1)!}.
\end{align*}
\end{multicols}
\end{figure*}
\end{preuve}


\section{Règle de \textsc{Raabe-Duhamel}}
\begin{theo}{Règle de \textsc{d'Alembert}}
    On suppose que $\lim\limits_{n \to + \infty} \frac{u_{n+1}}{u_n} = \ell$.
    \begin{itemize}
        \item Si $\ell < 1$, alors $\sum u_n$ converge.
        \item Si $\ell > 1$, alors $\sum u_n$ diverge.
    \end{itemize}
\end{theo}

Lorsque $\ell = 1$, on ne peut pas conclure. En effet, 
$$\sum \frac{1}{n} \text{ diverge et } \sum \frac{1}{n^2} \text{ converge}.$$

\begin{theo}{}
    Soit $\alpha$ un réel et $(u_n)_{n \in \N}$ une suite de réels strictement positifs. On suppose que
    $$\displaystyle \frac{u_{n+1}}{u_n} = 1 - \frac{\alpha}{n} + \mathcal{O} \left( \frac{1}{n^2} \right).$$ Alors $\sum u_n$ converge si et seulement si $\alpha > 1$. 
\end{theo}
\marginnote[-1cm]{Voir exercice 3.43. \cite{oraux_x_ens_3}}
\begin{preuve}
    \begin{enumerate}
        \item[($\Rightarrow$)] Montrer que si $u_n=\frac{K}{n^{\alpha}}$ avec $K>0$ et $\alpha > 1$ alors $(u_n)$ vérifie la relation.
        \item[($\Leftarrow$)] Soit $(v_n)$ une suite vérifiant les hypothèses. Montrer qu'il existe $K>0$ tel que $v_n \sim \frac{K}{n^{\alpha}}$ avec $\alpha > 0$. Pour cela, étudier la série de terme général $\ln (v_n)$.
    \end{enumerate}
\end{preuve}

On ne peut pas conclure si $\alpha=1$.

\section{Suites sous-additive}
\emph{Exercice 2. TD I \cite{acamanes}}\\

Une suite $(u_n)_{n\geqslant1}$ est dite sous-additive si pour tout couple d'entiers non nuls $(n, m)$, $u_{n+m} \leqslant u_n + u_m$.\\
1.a)\\ 
\indent $(\Leftarrow)$ Étudier les cas où $n=m$.\\
\indent $(\Rightarrow)$ Étudier la fonction $f:x \mapsto 1+x^{\alpha} - (1+x)^{\alpha}$.\\
1.b)\\
\indent Étudier le cas où $m=1$ et montrer que $w_n = n w_1$.\\
2)\\
\indent Montrer que la suite $(v_n)$ est décroissante.\\ 
4.b)\\
\indent Soit $(n, m) \in (\Ne)^2$. D'après le théorème de la division euclidienne, il existe un unique couple $(k, r) \in \Ne \times \llbracket0, m-1 \rrbracket$ tel que $n=km+r$.\\
\indent Utiliser successivment la définition d'une suite sous-additive et les résultats des questions 3) et 4.a).

\section{Étude de la suite de terme général \texorpdfstring{$u_n = \left( \frac{1}{b-a} \int_{a}^{b} f(x)^n \d x \right)^{1/n}$}{égal à une intégrale}}
 \begin{exercice}
    \marginnote[0cm]{\emph{Exercice 9. TD I}}
    Soit $f$ une supposée continue et positive sur $[a, b]$. Étudier la suite de terme général $u_n = \left( \frac{1}{b-a} \int_{a}^{b} f(x)^n \d x \right)^{1/n}$.
 \end{exercice}

\begin{elem_sol}
    \begin{itemize}
        \item La démarche générale consiste à encadrer $u_n$. 
        \item \underline{Majoration:} $f$ est continue sur un segment donc est en particuler bornée par un réel positif $M$. On peut montrer que $u_n \leqslant M$ \emph{(ne pas oublier l'argument de la continuité lors du passage à l'intégrale)}.
        \item \underline{Minoration:} soit $\varepsilon > 0$, soit $x_0$ tel que $f(x_0) = M$. Comme $f$ est continue en $x_0$, il existe $[c, d] \subset [a, b]$ tel que $x_0 \in [c, d]$ et pour tout $x \in [c, d]$, $f(x) \geqslant M - \varepsilon$ \emph{(un dessin permet de bien comprendre la stratégie)}.\\
        On peut ensuite montrer que $u_n \geqslant \left(\frac{d-c}{b-a} \right)^{1/n}(M-\varepsilon) \xrightarrow[n \to + \infty]{} M-\varepsilon$.
        \item Finalement, $u_n \displaystyle \longrightarrow M = \max_{[ a, b ]} f = \Ninf{f}$.
    \end{itemize}
\end{elem_sol}


\section{Transformation d'\textsc{Abel}} \label{transformation_abel}
\begin{tcolorbox}
    Si $(\varepsilon_n)$ et $(v_n)$ sont des suites à valeurs réelles, si $(u_n)$ et $(S_n)$ sont les suites de termes généraux:
    $$u_n = \varepsilon_n v_n \text{ et } S_n = \sum_{k=0}^{n} \varepsilon_k,$$
    si $(S_n)$ est bornée dans $\R$, si $(v_n)$ converge vers $0$ et si la série $\sum |v_{n+1} - v_n|$ converge, alors la série $\sum u_n$ converge.
\end{tcolorbox}

\begin{itemize}
    \item Savoir énoncer et démontrer la \textbf{formule de sommation par parties} (analogie avec l'intégration par parties) (sans raisonner par récurrence).
    $$\boxed{\forall n \in \N,\ \sum_{k=0}^{n} v_k \varepsilon_k = \sum_{k=0}^{n-1} (v_k -v_{k+1})S_k + v_nS_n}$$
    \item Critère d'\textsc{Abel} et test de \textsc{Dirichlet}.
\end{itemize}

\section{Convergence et calcul de  \texorpdfstring{$\sum \frac{r}{2^r}$}{de la série de terme général r/2^r}}
Montrer la convergence de la série de terme général $\frac{r}{2^r}$ et prouver que $\sum\limits_{r=1}^{+ \infty} \frac{r}{2^r} = 2$. Deux méthodes de résolution sont possibles bien que la première soit plus élégante. 

\begin{itemize}
    \item La série $\sum \frac{1}{2^r}$ est une série géométrique absolument convergente. Ainsi, d'après le résultat sur les produits de \textsc{Cauchy}, 
    $$4 = \left( \sum_{r=0}^{+ \infty} \frac{1}{2^r} \right)^2 = \sum_{r=0}^{+ \infty} \sum_{k=0}^{r} \frac{1}{2^k \cdot 2^{r-k}} = \sum_{r=0}^{+ \infty} \frac{r}{2^{r-1}}.$$
    \item On peut également étudier la fonction $g : x \to \sum\limits_{r=0}^{n} \frac{x^r}{2^r}$.
\end{itemize}

\section{Suites du type \texorpdfstring{$f(x_n) = n$}{f(x_n) = n}}
\begin{exercice}    
    \marginnote[0cm]{\emph{Exercice X 1 p.226 de \cite{exos_oraux}}}
    Soit $n \in \N$, montrer que l'équation $x \me^x = n$ admet une unique solution $x_n \in \R$, en donner un équivalent puis un équivalent de $y_n = x_n - \ln(n)$.  
\end{exercice}

\begin{exercice}
    Soit $n \in \Ne$, montrer que l'équation $x + \ln(x) = n$ admet une unique solution $x_n \in \R$, en donner un équivalent.
\end{exercice}
    


\section{Suites définies implicitement}

\begin{methode}
    \marginnote[0cm]{Texte de \cite{oraux_x_ens_3} p. 181.}
    \begin{enumerate}
        \item Montrer l'existence de $x_n$
        \item Démontrer la convergence de la suite $x_n$
        \item Déterminer un équivalent 
        \item Déterminer un développement asymptotique de $x_n$ \\
        Pour cela il faut commencer par déterminer dans la relation qui définit $x_n$ quels sont les termes prépondérants. 
    \end{enumerate}
\end{methode}

\section{Sommation par paquets}
\begin{exercice}
    \marginnote[0cm]{Source : \cite{exos_oraux} p. 340}
    Soit $\sum\limits_{n \geqslant 1} u_n$ une série à termes réels positifs, telle que $(u_n)_{n \geqslant 1}$ est décroissante. Montrer que les séries $\sum\limits_{n \geqslant 1} u_n$ et $\sum\limits_{n \geqslant 2} 2^n u_{2^n}$ sont de même nature. 
\end{exercice}

\section{Plan d'étude des suites \texorpdfstring{$u_{n+1} = f(u_n)$}{u_(n+1) = f(u_n)}}

La fonction $f$ doit être continue. \\

\begin{enumerate}
    \item Trouver un invervalle $I$ stable par $f$ ($f(I) \subset I$). Le segment $I$ doit contenir, au moins à partir d'un certain rang, tous les termes de la suite.
    \item Recherche des points fixes de la fonction $f$ dans $I$.
    \item ...
\end{enumerate}

\section{Techniques classiques}
\marginnote[0cm]{Arnaud Guyader}
\subsection{Développements asymptotiques}

Dans de nombreuses situations, on conclut sur la nature d'une série en se ramenant à une série plus simple. On a vu que pour les séries à termers positifs, il suffit de se ramener à un équivalent. Ceci n'est plus le cas avec des séries à termes de signe quelconque. Par ailleurs, un équivalent correspond à une approximation au premier ordre, laquelle ne permet pas forcément de conclure. \\
Dans ces deux situations, il suffit souvent d'écrire un développement asymptotique du terme général, c'est-à-dire d'être plus précis dans l'approximation. Celui-ci est généralement en $\frac{1}{n}$ ou en $\frac{1}{\sqrt{n}}$ et s'arrête au premier terme absolument convergent, en $\frac{1}{n^2}$ ou $\frac{1}{n^{3/2}}$.

Exemples: \\
$$\sum_{n \geqslant 2} \ln \left(1 + \frac{(-1)^n}{\sqrt{n}}\right) \text{ est divergente}.$$
$$\sum_{n \geqslant 1} \left(\frac{1}{\sqrt{n}} - \sqrt{n} \sin \frac{1}{n} \right) \text{ est absolument convergente}.$$

\subsection{Groupements de termes}

Considérons une série numérique $\sum u_n$ dont on veut déterminer la nature. On commence par s'assurer que le terme général $(u_n)$ tend vers zéro, sinon la série est trivialement divergente. Ceci fait, il faudrait montrer que la suite $(s_N)$ des sommes partielles est convergente, ce qui n'est pas toujours facile. En particulier, il est parfois plus simple de montrer qu'une sous-suite de $(s_N)$ converge, par exemple en effectuant des regroupements de termes, et de conclure ensuite. \\
Cadre typique d'application: on réussit à montrer que $(s_{2N})$ converge, disons vers $s$. Alors pour la sous-suite $(s_{2N+1})$, il suffit d'écrire:
$$s_{2N+1} = s_{2N} + u_{2N+1},$$
et si la série ne diverge par trivialement, on a
$$\lim_{N \to + \infty} s_{2N+1} = \lim_{N \to + \infty} s_{2N} = s,$$
c'est-à-dire que 
$$\lim_{N \to \infty} s_N = s,$$
la série converge. 


\chapter{Intégration}
\labch{integration}

\section{Fonction intégrable et décroissante sur \texorpdfstring{$\R^+$}{R+}}
\begin{exercice}
    \marginnote[0cm]{Source : \cite{exos_oraux} p. 268}
    Soit $f : \Rp \to \R$ une fonction continue, décroissante et intégrable. Montrer que $x f(x) \xrightarrow[x \to +\infty]{} 0$.
\end{exercice}

\begin{elem_sol}
\todoarmand{Correction : exercice 3 de \url{http://ddmaths.free.fr/section115.html}}
\begin{itemize}
\item Montrer que $f$ tend vers 0 en utilisant sa décroissance et son intégrabilité. $f$ est donc à valeurs positives. 
\item Encadrer $x f(x)$ en écrivant $x=2 \cdot \frac{x}{2}$.
\end{itemize}
\end{elem_sol}

\todoinline{Concernant les fonctions décroissantes, on pourrait ajouter un exercice qui doit ressembler à (je ne l'ai jamais écrit...) : Soit $f$ décroissante, continue par morceaux et intégrable sur $]0, 1]$. Alors, $\lim\limits_{n\to+\infty} \sum\limits_{k=0}^n f\left(\frac{k}{n}\right) = \int_0^1 f(t) \d t$. Il se prête à une illustration graphique ! Je mets dans un dossier un article qui contient ce résultat. J'ai un exercice d'oral qui utilise ce résultat. À chercher ?}
    

\section{Calcul d'une intégrale impropre}
\begin{exercice}
    Calculer $\displaystyle \int_0^\pi \ln(\sin t) \d t.$
\end{exercice}

\begin{elem_sol}
    $=-\pi \ln(2)$
    \end{elem_sol}

\todoinline{En mettre un peu plus sur la démo ? J'ai la version suivante à relire et changer les dt (CCP-PSI-2016) : Soient $I = \int_0^{\pi/2} \ln(\sin(t)) \ dt$ et $J = \int_0^{\pi/2} \ln(\cos(t)) \ dt$.
1. Montrer que $I$ et $J$ sont convergentes et que $I = J$.
2. Calculer $I + J$ et en déduire $I$ et $J$.

Solution\\
1. La fonction $t \mapsto \ln(\sin(t))$ est continue sur $]0,\pi/2]$. De plus,
\[
\ln(\sin(t)) = \ln(t + o(t)) = \ln(t) + \ln(1 + o(1)) = o(\ln(t)).
\]
Ainsi, $t \mapsto \ln(\sin(t))$ est intégrable en $0$.

La formule de changement de variable, avec $\phi : u \mapsto \pi/2 - u$ assure la convergence de $J$ ainsi que l'égalité $I = J$.

2. Comme ces intégrales sont bien définies, en utilisant la relation de Chasles et la symétrie dans la dernière égalité,
\[
I + J = \int_0^{\pi/2} \ln\left(\frac{\sin(2t)}{2}\right) \ dt = \frac{1}{2} \int_0^\pi \ln(\sin(t)) \ dt - \frac{\pi}{2} \ln(2) = I - \frac{\pi}{2} \ln(2).
\]
Ainsi, $I = J = -\frac{\pi}{2} \ln(2)$.
}

    

\section{Intégrales de  \textsc{Bertrand}}
\begin{prop}
    Soient $(\alpha, \beta) \in  \R^2$ et $f:t \to \frac{1}{t^{\alpha} \ln^{\beta} (t)}$. Alors,
    $$\int_{2}^{+ \infty} f \text{ converge si et seulement si }
    \begin{cases}
    \alpha > 1 \\
    \text{ou}\\
    \alpha = 1 \text{ et } \beta > 1
    \end{cases}.
    $$
\end{prop}

\textcolor{red}{à rerédiger}
\begin{preuve}
    Distinguons trois cas selon les valeurs prises par $\alpha$:
    \begin{enumerate}
        \item si $\alpha > 1$, soit $\gamma \in ]1, \alpha[$. On peut montrer que $$\displaystyle \frac{1}{t^{\alpha} \ln^{\beta} (t)} = o_{+ \infty} \left( \frac{1}{t^{\gamma}} \right).$$
        \item si $\alpha < 1$, soit $\gamma \in ]\alpha, 1[$. On peut montrer que 
        $ t^{\gamma} f(t) \xrightarrow[t \to + \infty]{} + \infty $
        donc à partir d'un certain rang, $f(t) \geqslant \frac{1}{t^{\gamma}} > 0$.
        \item si $\alpha = 1$, revenir aux intégrales partielles. Connaître la primitive de $t \mapsto \frac{1}{t \ln^{\beta} (t)}$:
        $$\int_{2}^{X} \frac{1}{t \ln^{\beta} (t)} \d t = 
        \begin{cases}
            \left[ \frac{\ln ^{1-\beta} (t)}{1-\beta} \right]_2 ^X & \text{si } \beta \not = 1, \\
            \left[\ln (\ln(t)) \right]_2 ^X & \text{si } \beta = 1.
        \end{cases}
        $$
        On en déduit que l'intégrale de la fonction $t \mapsto \frac{1}{t \ln^{\beta} (t)}$ converge sur $[2, + \infty[$ si et seulement si $\beta > 1$.
    \end{enumerate}
\end{preuve}


\section{Une propriété géométique de l'intégrale}
\begin{exercice}
    Soit $f$ de classe $\mathscr{C}^1$ sur $[a, b]$ telle que $f'$ soit strictement positive sur $[a, b]$. Calculer:
    $$\int_{a}^{b} f(t) \d t + \int_{f(a)}^{f(b)} f^{-1}(t) \d t.$$
\end{exercice}

\begin{elem_sol}
    \begin{itemize}
    \item Comme $f' > 0$, alors $f$ est strictement croissante. De plus, $f$ est continue, donc $f$ réalise une bijection de $[a, b]$ sur $[f(a), f(b)]$.

    \item En effectuant le changement de variable $\phi : [a, b] \to [f(a), f(b)],\, u \mapsto f(u)$, alors $\phi$ est bien de classe $\mathscr{C}^1$ et
    \begin{align*}
    \displaystyle\int_{f(a)}^{f(b)} f^{-1}(t) \mathrm{d} t
    &= \displaystyle\int_a^b f^{-1}(f(u)) f'(u) \mathrm{d} u\\
    &= \displaystyle\int_a^b u f'(u) \mathrm{d} u\\
    &= \left[u f(u)\right]_a^b - \displaystyle\int_a^b f(u) \mathrm{d} u,
    \end{align*}
    où on a effectué une intégration par parties.
    \end{itemize}

    Finalement,
    \[
    \displaystyle\int_a^b f(t) \mathrm{d} t + \displaystyle\int_{f(a)}^{f(b)} f^{-1}(t) \mathrm{d} t = b f(b) - a f(a).
    \]
    
    % \item Calculer le deuxième terme en posant $t = f(u)$.
        % \item Éffectuer une IPP sur le deuxième terme pour conclure. 
        % Donner une interprétation géométrique.
    % \end{itemize}
\end{elem_sol}

\todoinline{Là, il y a un dessin à faire ;-) ! En gros, de mémoire, on peut dessiner un rectangle avec une symétrie par rapport à la première bissectrice. Je vais faire un schéma sur un papier que je mettrai dans le dossier.}

\includegraphics[width=0.5\textwidth]{./chapitres/integration/documents/propriete_geometrique.jpg}

\section{Permutation somme/intégrale}
Justifier les convergences, puis l'égalité:
$$\sum_{n=0}^{+ \infty} \frac{(-1)^n (2n+1)}{(2n+1)^2 + x^2} = \frac{1}{2} \int_{0}^{+ \infty} \frac{\cos (xt)}{\ch t}\ \d t.$$

\begin{itemize}
    \item $\frac{1}{2 \ch(t)} = \frac{\me^{-t}}{1 + \me^{-2t}}$
    \item ...
\end{itemize}

\section{Variante du lemme de \textsc{Lebesgue}}
\begin{prop}{}
    \marginnote[0cm]{Source : \cite{exos_oraux} p.280}
    Soit $f$ continue par morceaux sur le segment $[a,b]$, avec $a < b$,
    $$\lim_{n \to +\infty} \int_{a}^{b} f(t) | \sin (nt) | \d t = \frac{2}{\pi} \int_{a}^{b} f(t) \d t.$$
\end{prop}

\todoinline{Relire et compléter la correction. Trouver une application}

\todoarmand{Pour une application du lemme de Riemann-Lebesgue (qui n'est pas tout à fait l'énoncé d'au-dessus) : exercices 4 et 5 de \url{https://www.louboutin.org/LeSite20182019/mathematiques/Exercices/Ex11_1819.pdf}. Voir plus bas pour les énoncés}

\begin{elem_preuve}
    \begin{enumerate}
        \item On va montrer ce résultat dans le cas où \ptnclegras{$f$ est constante sur $[a, b]$} (véritable difficulté du problème) 
        \item On va ensuite montrer ce résultat dans le cas où \ptnclegras{$f$ est une fonction en escalier} en appliquant le résultat précédent sur chacun des intervalles de la subdivision de $[a, b]$.
        \item Finalement on va montrer le \ptnclegras{cas général} en encadrant $f$ par deux fonctions en escalier (méthode de l'intégrale de $\textsc{Riemann}$).
    \end{enumerate}
\end{elem_preuve}
\begin{preuve}
    \begin{enumerate}
        \item On pose $f = \lambda$. On va étudier la limite de l'intégrale
        $$I_n = \int_{a}^{b} \lambda | \sin(nt) | \d t = \frac{\lambda}{n} \int_{na}^{nb} | \sin(u) | \d u.$$
        L'idée est alors de découper l'intervalle $[a, b]$ en trois intervalles: des \textcolor{YellowGreen}{extrémités} où l'intégrale tendra vers $0$ puis un intervalle \textcolor{Salmon}{central} de longueur $k_n \pi$ qui sera simple à traiter. \\
        \textcolor{green}{A mieux rédiger...} \\
        On pose (qui existent pour $n \geqslant \frac{\pi}{b-a}$)
        $$c_n = \min( \pi \mathbb{Z} \cap [na, nb]) \text{ et } d_n = \max( \pi \mathbb{Z} \cap [na, nb]).$$
        $c_n \sim na$ et $d_n \sim nb$. 
        \item Aucune difficulté.
        \item Il existe deux fonctions en escalier $\varphi$ et $\psi$ telles que $\varphi \leqslant f \leqslant \psi$ et $\int_{a}^{b} (\psi - \varphi) \leqslant \varepsilon$.
        \begin{align*}
            & \left | \int_{a}^{b} f(t) | \sin (nt) | \d t - \frac{2}{\pi} \int_{a}^{b} f(t) \d t \right| \\
            \leqslant & \left | \int_{a}^{b} [f(t) - \varphi(t) ] | \sin(t) | \d t \right| + \left | \int_{a}^{b} \varphi(t) |\sin(t)| \d t - \frac{2}{\pi} \int_{a}^{b} \varphi(t) \d t \right| + \left| \frac{2}{\pi} \int_{a}^{b} [f(t) - \varphi(t)] \d t \right|
        \end{align*}
    \end{enumerate}
\end{preuve}

Exercices 3, 4, 5 de \url{https://www.louboutin.org/LeSite20182019/mathematiques/Exercices/Ex11_1819.pdf}

\begin{exercice}
\begin{enumerate}
    \item Soit $f$ une fonction de classe $\mathscr{C}^1$ sur l'intervalle $\interff{a}{b}$. Montrer que $\lim\limits_{\lambda \to \infty} \int_a^b \sin(\lambda t) f(t) \d t = 0$. 
    \item Énoncer sans démonstration un résultat analogue avec $\cos(\lambda t)$ et $\e^{\i \lambda t}$.
    \item Montrer que ce résultat reste valable pour une fonction en escalier. 
    \item En déduire qu'il est aussi valable pour une fonction continue par morceaux. 
    \item Montrer que si $f$ est intégrable sur l'intervalle $I$ alors $\lim\limits_{n \to +\infty} \int_I \sin(nt) f(t) \d t = 0$.
\end{enumerate}
\end{exercice}

\begin{exercice}
\begin{enumerate}
    \item Pour $t$ réel, calculer la somme $S_n(t) = \frac{1}{2} + \cos(t) + \cdots + \cos(nt)$. On écrira le résultat sous la forme $a \frac{\sin b}{\sin c}$. 
    \item Déterminer deux nombres réel $a$ et $b$ tels que pour tout $n$ entier non nul on ait
    \[
    \int_0^\pi \big(a t^2 + bt\big) \cos(nt) \d t = \frac{1}{n^2}.
    \]
    \item Montrer que la fonction valant $\frac{at^2 + bt}{\sin(t/2)}$ sur $\interoo{0}{\pi}$ peut être prolongée en une fonction de classe $\mathscr{C}^1$ sur $\interff{0}{\pi}$.
    \item Déterminer $\lim\limits_{n \to \infty} \int_0^n \big(a t^2 + bt \big) S_n(t) \d t$.
    \item Retrouver la valeur de $\sum\limits_{n=1}^\infty \frac{1}{n^2}$.
\end{enumerate}
\end{exercice}

\begin{solution}
\begin{enumerate}
\item Si $t = 2k \pi$ avec $k \in \Z$ alors $S_n(t) = n + \frac{1}{2}$. \\
Si $t \in \R \setminus 2 \pi \Z$, calculons plutôt $S_n(t) + \frac{1}{2}$ :
    \begin{align*}
    S_n(t) + \frac{1}{2} &= \sum_{k=0}^n \frac{\e^{\i k t} + \e^{-\i k t}}{2} \\
    &= \frac{1}{2} \left[\sum_{k=0}^n \big(\e^{\i t}\big)^k + \sum_{k=0}^n \big(\e^{-\i t}\big)^k \right] \\
    &= \frac{1}{2} \left[ \frac{1 - \e^{\i (n+1)t}}{1 - \e^{\i t}} + \frac{1 - \e^{-\i (n+1)t}}{1 - \e^{-\i t}} \right] \\
    &= \Reel \left( \frac{1 - \e^{\i (n+1)t}}{1 - \e^{\i t}} \right)
\end{align*}
Par la méthode de l'angle moitié, 
\begin{align*}
    \frac{1 - \e^{\i (n+1)t}}{1 - \e^{\i t}} &= \frac{\e^{\i\frac{(n+1)t}{2}} \Big( \e^{-\i\frac{(n+1)t}{2}} - \e^{\i\frac{(n+1)t}{2}} \Big)}{\e^{\i\frac{t}{2}} \big( \e^{-\i\frac{t}{2}} - \e^{\i\frac{t}{2}} \big)} \\
    &= \e^{\i \frac{nt}{2}} \frac{\sin \left( \frac{(n+1)t}{2} \right)}{\sin \left( \frac{t}{2} \right)}
\end{align*}
soit, 
\begin{align*}
    S_n(t) + \frac{1}{2} &= \cos \left(\frac{nt}{2}\right) \frac{\sin \left( \frac{(n+1)t}{2} \right)}{\sin \left( \frac{t}{2} \right)}.
\end{align*}
Nous pouvons simplifier le $\frac{1}{2}$ en utilisant la formule $2 \sin(a) \cos(b) =\sin(a+b) + \sin(a-b)$ vraie pour $a$ et $b$ réels. Nous en déduisons que 
\[
\cos \left(\frac{nt}{2}\right) \sin \left( \frac{(n+1)t}{2} \right) = \frac{1}{2} \left[ \sin \left( \frac{(2n+1)t}{2} \right) + \sin \left( \frac{t}{2} \right) \right].
\]
Finalement, pour $t \in \R \setminus 2 \pi \Z$,
\[
S_n(t) = \frac{\sin \left( \frac{(2n+1)t}{2} \right)}{2 \sin \left( \frac{t}{2} \right)}
\]
\item Soit $n$ un entier non nul, des intégrations par parties permettent de trouver
\[
\int_0^\pi t^2 \cos(n t) \d t = (-1)^n\frac{2\pi}{n^2} \quad \text{et} \quad \int_0^\pi t \cos(n t) \d t = \frac{(-1)^n - 1}{n^2}.
\]
On en déduit par linéarité de l'intégrale que 
\[
\int_0^\pi \big(a t^2 + bt\big) \cos(nt) \d t = \frac{1}{n^2} \big( (-1)^n 2 a \pi + (-1)^n b - b \big)
\]
et on obtient le résultat demandé pour 
\[
a = \frac{1}{2 \pi} \quad \text{et} \quad b = -1.
\]
\item 
%\item D'après la première question, 
%\begin{align*}
 %   \int_0^{n \pi} \big(a t^2 + bt \big) S_n(t) \d t &= \int_0^{n \pi} \big(a t^2 + bt \big) \frac{\sin \left( \frac{(2n+1)t}{2} \right)}{2 \sin \left( \frac{t}{2} \right)} \d t \\
 %   &= \sum_{k=0}^{n-1} \int_{k \pi}^{(k+1) \pi} \big(a t^2 + bt \big) \frac{\sin \left( \frac{(2n+1)t}{2} \right)}{2 \sin \left( \frac{t}{2} \right)} \d t \\
 %   &= \sum_{k=0}^{n-1} \int_{0}^{\pi} \big(a (t-k\pi)^2 + b(t-k\pi) \big) \frac{\sin \left( \frac{(2n+1)t}{2} - \frac{(2n+1)k\pi}{2} \right)}{2 \sin \left( \frac{t}{2} - \frac{k\pi}{2} \right)} \d t \\
  %  &= 
%\end{align*}
\end{enumerate}
\end{solution}

\begin{exercice}
\begin{enumerate}
    \item Justifier l'existence de l'intégrale $I = \int_0^{+\infty} \frac{\sin t}{t} \d t$. 
    \item Calculer $I_n = \int_0^\pi \frac{\sin \left(n + \frac{1}{2}\right)t}{2 \sin \frac{t}{2}} \d t$. \\
    \emph{Indication :} Calculer $I_{n+1} - I_n$. 
    \item Montrer que la fonction définie sur $\interof{0}{\pi}$ par $f(x) = \frac{1}{x} - \frac{1}{2 \sin \frac{x}{2}}$ peut être prolongée à $\interff{0}{\pi}$ en une fonction de classe $\mathscr{C}^1$. 
    \item Soit $f$ une fonction de classe $\mathscr{C}^1$ sur l'intervalle $\interff{a}{b}$. Montrer que 
    \[
    \lim_{\lambda \to +\infty} \int_a^b \sin(\lambda t) f(t) \d t = 0.
    \]
    \item En déduire la valeur de l'intégrale $I$.
\end{enumerate}
\end{exercice}


\section{Sommes de \textsc{Riemann} généralisées}
\begin{theo}
    \cite{acamanes} \\
    Pour tout entier naturel $n$ non nul, la \emph{somme de \textsc{Riemann}} associée à $f$ sur le segment $[a, b]$ est $S_n \defeq \frac{b-a}{n} \sum\limits_{k=0}^{n-1} f \left( a + k \frac{b-a}{n} \right)$. Si $f$ est continue par morceaux sur $[a, b]$, alors, 
    $$\lim_{n \to + \infty} S_n = \int_a^b f(t) \d t.$$
\end{theo}

\begin{marginfigure}[-1cm]
    %https://tex.stackexchange.com/questions/476702/riemann-sum-approaches-area-under-curve

\begin{tikzpicture}[scale=1,declare function={f(\x)=((1/3)*(\x)^(3)-3*(\x)^(2)+8*\x-3;}]
\coordinate (start) at (.8,{f(.8)});
\coordinate (x0) at (1,{f(1)});
\coordinate (x1) at (2,{f(2)});
\coordinate (x2) at (3,{f(3)});
\coordinate (x3) at (4,{f(4)});
\coordinate (x4) at (5,{f(5)});
\coordinate (end) at (5.05,{f(5.05)});
\draw[fill=teal!20!white] (1,0) rectangle (2,{f(1)});
\draw[fill=teal!20!white] (2,0) rectangle (3,{f(2)});
\draw[fill=teal!20!white] (3,0) rectangle (4,{f(3)});
\draw[fill=teal!20!white] (4,0) rectangle (5,{f(4)});
\draw (5,0)--(5,{f(5)});
\draw [-latex] (-0.5,0) -- (6,0) node (xaxis) [below] {$x$};
\draw [-latex] (0,-0.5) -- (0,5) node [left] {$y$};
\foreach \x/\xtext in {1/a=x^0_{4} ,2/x^1_{4}, 3/x^2_{4} , 4/x^3_{4} , 5/b=x^4_{4}}
 \draw[xshift=\x cm] (0pt,3pt) -- (0pt,0pt) 
node[below=2pt,fill=white,font=\normalsize]
  {$\xtext$};
\draw[domain=.5:5.3,samples=200,variable=\x,blue,thick] plot ({\x},{f(\x)});                 
\foreach \n in {0,1,2,3}
\draw[blue,fill=blue] (x\n) circle (2pt) node[font=\normalsize] {$ $};    
\draw[<->] (2,-1)--(3,-1) node[below,midway] {$\frac{b-a}{n}$};      
\end{tikzpicture}
\end{marginfigure}


\section{Intégration des relations de comparaisons}
\begin{prop}{}
    Soit $f: \Rp \rightarrow \C$ une fonction continue par morceaux et $g, h:\Rp \rightarrow \Rp$ deux fonctions continues par morceaux, strictement positives. On suppose que $f = o_{+\infty}(g)$ et $f \sim_{+\infty} h$.\\
    \begin{itemize}
        \item Si $g$ et $h$ ne sont pas intégrables sur $\Rp$,
        $$\int_{0}^{x} f = o_{+\infty} \left(\int_{0}^{x} g \right) \text{ et } \int_{0}^{x} f \sim_{+\infty} \int_{0}^{x} h.$$
        \item Si $g$ et $h$ sont intégrables sur $\Rp$,
        $$\int_{x}^{+\infty} f = o_{+\infty} \left(\int_{x}^{+\infty} g \right) \text{ et } \int_{x}^{+\infty} f \sim_{+\infty} \int_{x}^{+\infty} h.$$
    \end{itemize}
\end{prop} 

La démonstration est analogue à celle de la \nameref{sommation_relations_comparaison}

% \section{Transformée de \textsc{Fourier} de la loi normale}
% \input{chapitres/integration/transformee_de_fourier_de_la_loi_normale}

% \section{Intégrale à paramètre dans les bornes}
% \begin{itemize}
    \item On pose $f:x \mapsto \int_{x}^{x^2} \frac{\d t}{\ln (t)}$.
    $$\displaystyle f(x) = \int_{x}^{x^2} \frac{t}{t \ln (t)}\ \d t \leqslant x^2 \int_{x}^{x^2} \frac{\d t}{t \ln(t)} = x^2 \bigl[ \ln (\ln (t)) \bigr]_x^{x^2} = x^2 \ln(2)$$.
\end{itemize}

\section{Intégrale de \textsc{Dirichlet}}
\textcolor{red}{A revoir}
\begin{tcolorbox}
    L'intégrale de \textsc{Dirichlet} (1829) est l'intégrale de la fonction sinus cardinal sur la demi-droite des réels positifs
    $$\int_{0}^{+\infty} \frac{\sin x}{x}\ \d x = \frac{\pi}{2}.$$
\end{tcolorbox}

\begin{enumerate}
    \item Montrer que $\int_{0}^{1} \frac{\sin (t)}{t}\ \d t$ est convergente. 
    \item Deux méthodes.
    \begin{itemize}
        \item Montrer que la série de terme général $\int_{n \pi}^{(n+1) \pi} \frac{\sin(t)}{t}\ \d t$ est convergente. En déduire que $\int_{1}^{+ \infty} \frac{\sin(t)}{t}\ \d t$ converge. 
        \begin{enumerate}
            \item Montrer que $u_n = \int_{n \pi}^{(n+1) \pi} \frac{\sin(t)}{t}\ \d t$ est le terme général d'une série alternée. Donc $\sum u_n$ converge.\\
            Attention: on ne peut pas en déduire directement que $\sum\limits_{n=0}^{+ \infty} \int_{n \pi}^{(n+1) \pi} \frac{\sin(t)}{t}\ \d t = \int_{1}^{+ \infty} \frac{\sin(t)}{t}\ \d t$ car on n'a pas encore démontrer la convergence du deuxième membre (c.f. relation de \textsc{Chasles}).\\
            \item Il faut montrer la convergence de $\int_{\pi}^{x} \frac{\sin (t)}{t}\ \d t$. \textcolor{green}{A compléter.}
        \end{enumerate}
        \item On peut aussi procéder par intégration par parties en posant
        $$
        \begin{drcases}                
            u(t) = \frac{1}{t}\\
            v(t) = - \cos(t)
        \end{drcases}
        \mathscr{C}^1 \text{ sur } [1, +\infty].
        $$
        Bien présicer que $u(t)v(t)=-\frac{\cos(t)}{t}$ admet une limite finie en 1 et en $+ \infty$.\\
        \textcolor{red}{Les IPP préservent la régularité de l'intégrale MAIS ne préservent pas l'intégrabilité.}
    \end{itemize}
    \item On en déduit immédiatement que $\int_{0}^{+ \infty} \frac{\sin (t)}{t}\ \d t$ converge.
    \item De plus, on peut montrer que cette intégrale est semi-convergente (i.e. elle n'est pas intégrable sur $\Rp$). Pour cela, montrer que pour tout entier naturel $n$, $\int_{n \pi}^{(n+1) \pi} \frac{\sin(t)}{t}\ \d t \geqslant \frac{2}{(n+1) \pi}$. 
\end{enumerate}
    
\underline{Remarque:} La fonction sinus cardinal \textbf{n'est pas intégrable} sur $\Rpe$ (cf. le prochain exercice).

\section{Intégrale de \textsc{Gauss}}
\begin{prop}{}
    $$\int_{0}^{+\infty} \me^{-x^2} \d x = \frac{\sqrt{\pi}}{2}$$
\end{prop}

\begin{exercice}
    \marginnote[0cm]{\cite{maths-france} Planche no 13. Suites et séries d’intégrales}
    \begin{enumerate}
        \item \textbf{Première méthode:} \say{ à la main }. \\ 
        Pour $n \in \Ne$, on pose
        $$
        f_n(x) \defeq
        \begin{cases}
            \left(1 - \frac{x}{n} \right)^n &\text{si } x \in [0, n] \\
            0 &\text{si } x \geqslant n
        \end{cases}.
        $$
        Pour tout réel positif $x$, on pose $f(x) = \me^{-x^2}$.
        \begin{enumerate}
            \item Montrer que pour tout réel positif $x$, 
            $$|f(x) - f_n(x)| \leqslant \frac{1}{n \me}.$$
            \item À l'aide de la suite $(f_n)_{n \in \Ne}$, calculer l'intégrale de \textsc{Gauss}.
        \end{enumerate}
        \item \textbf{Deuxième méthode:} \say{ avec le théorème de convergence dominée }. \\
        Pour $n \in \Ne$, on pose
        $$
        f_n(x) \defeq
        \begin{cases}
            \left(1 - \frac{x^2}{n} \right)^n &\text{si } x \in [0, \sqrt{n}] \\
            0 &\text{si } x > \sqrt{n}
        \end{cases}.
        $$
        \begin{enumerate}
            \item Montrer que la suite $(f_n)_{n \in \Ne}$ converge simplement sur $\Rp$ vers la fonction $f:x \mapsto \me^{-x^2}$.
            \item À l'aide de la convergence dominée, calculer l'intégrale de \textsc{Gauss}.
        \end{enumerate}
    \end{enumerate}
\end{exercice}

\begin{solution}
\end{solution}

\section{Intégrale de \textsc{Wallis}} \label{integrale_wallis}
La page \url{https://fr.wikipedia.org/wiki/Intégrale_de_Wallis} est très complète. 

\begin{defi}{Intégrale de \textsc{Wallis}}
    $$\Wallis_n \defeq \int_{0}^{\frac{\pi}{2}} \sin^n x \d x = \int_{0}^{\frac{\pi}{2}} \cos^n x \d x$$
\end{defi}

\begin{prop}{} \labprop{prop_wallis}
    $$\Wallis_{2p} = \frac{\binom{2p}{p}}{2^{2p}}\frac{\pi}{2} \text{ et } \Wallis_{2p+1} = \frac{2^{2p} (p!)^2}{(2p+1)!}$$
    $$\Wallis_{n+1} \sim \Wallis_n \qquad \Wallis_n \sim \sqrt{\frac{\pi}{2n}}$$
    $$\Wallis_n \Wallis_{n+1} = \frac{\pi}{2(n+1)}$$
\end{prop}

\begin{preuve}
    Calculons $\Wallis_{n+2}$ en effectuant une intégration par parties. On pose $u(t) \defeq - \cos(t)$ et $v(t) \defeq \sin^{n+1}(t)$, toutes deux de classe $\mathscr{C}^1$ sur $\left[0, \frac{\pi}{2} \right]$. 
    \begin{align*}
        \Wallis_{n+2} &= \underbrace{\left[ -\cos(t) \sin^{n+1}(t) \right]_0^{\pi/2}}_{=0} + (n+1) \int_0^{\pi/2} \cos^2 (t) \sin^n(t) \d t \\
        &= (n+1) \int_0^{\pi/2} (1 - \sin^2(t)) \sin^n(t) \d t \\
        &= (n+1) \Wallis_n - (n+1) \Wallis_{n+2} \\
        \text{soit } (n+2) \Wallis_{n+2} &= (n+1) \Wallis_n.
\end{align*}
Soit $p \in \N$. D'après la relation précédente, 
\begin{figure*}[h!]
\begin{multicols}{2}
\begin{align*}
    \Wallis_{2p} &= \frac{2p-1}{2p} \Wallis_{2p-2} \\
    &= \frac{2p-1}{2p} \times \frac{2p-3}{2p-2} \times \cdots \times \frac{1}{2} \times \underbrace{\Wallis_0}_{=\pi/2} \\
    &= \frac{\prod\limits_{k=1}^p (2k+1)}{\prod\limits_{k=1}^{p+1} (2k)} \frac{\pi}{2} \\
    &= \frac{\left[\prod\limits_{k=1}^p (2k+1) \right] \times \left[ \prod\limits_{k=1}^{p+1} (2k) \right]}{\left[\prod\limits_{k=1}^{p+1} (2k) \right]^2} \frac{\pi}{2} \\
    \Wallis_{2p} &= \frac{(2p)!}{2^{2p}(p!)^2} \frac{\pi}{2}.
\end{align*}
\begin{align*}
    \Wallis_{2p+1} &= \frac{2p}{2p+1} \Wallis_{2p-1} \\
    &= \frac{2p}{2p+1} \times \frac{2p-2}{2p-1} \times \cdots \times \frac{2}{3} \times \underbrace{\Wallis_1}_{=1} \\
    &= \frac{\prod\limits_{k=1}^p (2k)}{\prod\limits_{k=0}^p (2k+1)} \\
    &= \frac{\left[ \prod\limits_{k=1}^p (2k) \right]^2}{\left[ \prod\limits_{k=0}^p (2k+1) \right] \left[ \prod\limits_{k=1}^p (2k) \right]} \\
    \Wallis_{2p+1} &= \frac{2^{2p}(p!)^2}{(2p+1)!}.
\end{align*}
\end{multicols}
\end{figure*}
\end{preuve}

\subsection{Séries génératrices}

\textcolor{red}{texte}

\begin{prop}{}
    \marginnote[0cm]{\url{https://fr.wikipedia.org/wiki/Intégrale_de_Wallis}}
    La série génératrice des termes pairs est 
    $$\sum_{p=0}^\infty \Wallis_{2p} x^{2p} = \frac{\pi}{2} \frac{1}{\sqrt{1-x^2}}.$$
    La série génératrice des termes impairs est 
    $$\sum_{p=0}^\infty \Wallis_{2p+1} x^{2p+1} = \frac{\arcsin x}{\sqrt{1-x^2}}.$$
\end{prop}

\begin{exercice}
    Soit $x \in ]0,1[$. Calculer $\sum\limits_{n=0}^\infty (-1)^n \Wallis_n$ puis $\sum\limits_{n=0}^\infty \Wallis_n x^n$.
\end{exercice}

\begin{solution}
    \marginnote[0cm]{fic00126}
    D'après \vrefprop{prop_wallis}, $\Wallis_n \sim \sqrt{\frac{\pi}{2n}}$ et la règle de \textsc{d'Alembert} fournit $R = 1$. Soit $x \in ]-1, 1[$. \\
    Pour tout $t \in \left[ 0, \frac{\pi}{2} \right]$ et tout entier naturel $n$, $|x^n \cos^n t| \leqslant |x|^n$. Comme la série numérique de terme général $|x|^n$ converge, la série de fonctions de terme général $t \mapsto x^n \cos^n t$ est normalement convergente et donc uniformément convergente sur le segment $\left[ 0, \frac{\pi}{2} \right]$. D'après le théorème d'intégration terme à terme sur un segment, 
    \begin{align*}
        \sum_{n=0}^{+ \infty} \Wallis_n x^n &= \sum_{n=0}^{+ \infty} \left[ x^n \int_0^{\pi/2} \cos^n t \d t \right] = \int_0^{\pi/2} \left( \sum_{n=0}^{+\infty} x^n \cos^n t \right) \d t \\
        &=\int_0^{\pi/2} \frac{1}{1 - x \cos t} \d t \\
        &= \int_0^1 \frac{1}{1 - x \frac{1-u^2}{1+u^2}} \frac{2}{1 + u^2} \d u \quad \text{en posant } u = \tan \frac{t}{2} \\
        &= 2 \int_0^1 \frac{1}{(1+x)u^2 + (1-x)} \d u \\
        &= 2 \times \frac{1}{1+x} \times \frac{1}{\sqrt{\frac{1-x}{1+x}}} \left[ \arctan \left( \frac{u}{\sqrt{\frac{1-x}{1+x}}} \right) \right]_0^1 \\
        \sum_{n=0}^{+ \infty} \Wallis_n x^n &= \frac{2}{\sqrt{1-x^2}} \arctan \sqrt{\frac{x+1}{x-1}}.
    \end{align*}
\end{solution}

\subsection{Calcul de l'intégrale de \textsc{Gauss}}
\marginnote[0cm]{\url{https://fr.wikipedia.org/wiki/Intégrale_de_Wallis}}
On peut aisément utiliser les intégrales de \textsc{Wallis} pour calculer l'intégrale de \text{Gauss}. \\
On utilise pour cela l'encadrement suivant, issu de la construction de la fonction exponentielle par la méthode d'\textsc{Euler}: pour tout entier $n > 0$ et tout réel $u \in ]-n, n[$, 
$$\left(1 + \frac{u}{n} \right)^n \leqslant \me^u \leqslant \left( 1 - \frac{u}{n} \right)^{-n}.$$
Posant alors $u = -x^2$, on obtient:
$$\int_0^{\sqrt{n}} \left( 1 - \frac{x^2}{n} \right)^n \d x \leqslant \int_0^{\sqrt{n}} \me^{-x^2} \d x \leqslant \int_0^{\sqrt{n}} \left( 1 + \frac{x^2}{n} \right)^{-n} \d x.$$
Or les intégrales d'encadrement sont liées aux intégrales de \textsc{Wallis}. Pour celle de gauche, il suffit de poser $x = \sqrt{n} \sin t$ ($t$ variant de $0$ à $\pi/2$). Quant à celle de droite, on peut poser $x = \sqrt{n} \tan t$ ($t$ variant de $0$ à $\pi/4$) puis majorer par l'intégrale de $0$ à $\pi/2$. On obtient ainsi:
$$\sqrt{n} \Wallis_{2n+1} \leqslant \int_0^{\sqrt{n}} \me^{-x^2} \d x \leqslant \sqrt{n} \Wallis_{2n-2}.$$
Par le théorème des gendarmes, on déduit alors de l'équivalent de $\Wallis_n$ que
$$\int_0^{+ \infty} \me^{-x^2} \d x = \frac{\sqrt{\pi}}{2}.$$

\subsection{Volume d'une boule en dimension \texorpdfstring{$n$}{n}}

\begin{exercice}
    \marginnote[0cm]{\cite{fmaalouf}}
    Pour $n \in \Ne$ et $R \in \Rpe$ on désigne par $V_n(R)$ le volume de la boule de $\R^n$ de centre $O$ et de rayon $R$, 
    $$V_n(R) \defeq \idotsint_{x_1^2 + \cdots + x_n^2 \leqslant R^2} \d x_1 \cdots \d x_n.$$
    Montrer que pour tout $p \in \Ne$, 
    $$V_{2p}(R) = \frac{\pi^p R^{2p}}{p!}.$$
\end{exercice}

\subsection{\textsc{Grain de raisin}: Produit de \textsc{Wallis}}

\begin{prop}{Produit de \textsc{Wallis}}
    $$\prod_{n=1}^{\infty} \frac{4n^2}{4n^2-1} = \frac{\pi}{2}$$
\end{prop}

\begin{preuve}
    Puisque $\Wallis_{2n} \sim \Wallis_{2n+1}$, 
    $$\lim_{n \to +\infty} \frac{\Wallis_{2n+1}}{\Wallis_{2n} / \frac{\pi}{2}} = \frac{\pi}{2}.$$
    Or d'après le calcul des intégrales de \textsc{Wallis}:
    $$\frac{\Wallis_{2n+1}}{\Wallis_{2n} / \frac{\pi}{2}} = \frac{\frac{2^{2p}(p!)^2}{(2p+1)!}}{\frac{(2p)!}{2^{2p}(p!)^2}} = \prod_{k=1}^n \frac{4k^2}{4k^2-1}.$$
\end{preuve}

\section{Intégrales eulériennes}
\subsection{Fonction Gamma d'\textsc{Euler}}

\begin{marginfigure}[5cm]
    \begin{tikzpicture}[]

\begin{axis}[
xmin = -4.9, xmax = 5.1, 
%ymin = -3.5, ymax = 3.5,  
restrict y to domain=-6:6,
axis lines = middle,
axis line style={-latex},  
xlabel={$x$}, 
ylabel={$\Gamma(x)$},
%enlarge x limits={upper={val=0.2}},
enlarge y limits=0.05,
x label style={at={(ticklabel* cs:1.00)}, inner sep=5pt, anchor=north},
y label style={at={(ticklabel* cs:1.00)}, inner sep=2pt, anchor=south east},
]

\addplot[color=red, samples=222, smooth, 
domain = 0:5] gnuplot{gamma(x)};

\foreach[evaluate={\N=\n-1}] \n in {0,...,-5}{%
\addplot[color=red, samples=555, smooth,  
domain = \n:\N] gnuplot{gamma(x)};
%
\addplot [domain=-6:6, samples=2, densely dashed, thin] (\N, x);
}%
\end{axis}
\end{tikzpicture}
    \caption{Graphe de la fonction Gamma}
\end{marginfigure}

\begin{defi}
    Pour tout $x$ > 0 réel, la \emph{fonction Gamma d'\textsc{Euler}} est définie par: 
    $$\Gamma(x) \defeq \int_{0}^{+\infty} t^{x-1} \me^{-t} \d t.$$
\end{defi}

\marginnote[-2mm]{Cette fonction, introduite en 1729 par le mathématicien suisse, prolonge la fonction factorielle à l'ensemble des nombres complexes (à l'exception des entiers négatifs).}

\begin{remarque}
    À un changement de variable près, la fonction $\Gamma$ est la \nameref{transformee_laplace} de la fonction $t \mapsto t^x$. 
\end{remarque} 
\underline{Propriétés principales:}
\begin{itemize}
    \item $\Gamma$ est définie si et seulement si $x>0$.
    \item Pour tout $x > 0$, $\Gamma(x+1) = x\Gamma(x)$.
    \item En particulier, $\boxed{\forall n \in \N$, $\Gamma(n+1) = n!}$. 
\end{itemize}
\begin{preuve}
    La fonction $f:(x,t) \mapsto t^{x-1} \me^{-t}$ est continue sur $]0, + \infty[$ comme produit de fonctions qui y sont continues. La fonction $f$ est donc intégrable sur tout segment de $]0, +\infty[$. Il reste à étudier son intégrabilité en $0$ et en $+ \infty$:
    \begin{itemize}
        \item En $+\infty:$ par croissances comparées, $t^{x-1} \me^{-t} = o_{+\infty} \left(\frac{1}{t^2} \right)$. D'après le théorème de comparaison des fonctions à termes positifs, $f$ est intégrable au voisinage de $+\infty$.
        \item En $0$: $f(x,t) \sim_0 t^{x-1}$ qui est intégrable d'après \textsc{Riemann} si et seulement si $1-x < 1$ i.e. si et seulement si $\boxed{x > 0}$.
    \end{itemize}
\end{preuve}
\underline{Dérivées successives:} \\
Utiliser une domination locale sur un segment $[a, A] \subset \R_+^\star$ par la fonction:
$$\varphi_k:t \mapsto 
\begin{cases}
    |\ln t |^k \me^{-t} t^{a-1}, & \text{si } t \in ]0, 1] \\
    |\ln t |^k \me^{-t} t^{A-1}, & \text{si } t > 1
\end{cases}
$$
$$\boxed{\forall k \in \N,\ \forall x \in \R_+^\star,\ \Gamma^{(k)}(x) = \int_{0}^{+\infty} (\ln t)^k t^{x-1} \me^{-t} \d t}$$

\underline{Exercice:} \url{https://share.miple.co/content/t8BIcXSjdEslq}

\subsection{Fonction bêta}
\begin{itemize}
    \item Pour tout $(p,q) \in \N^2$, on note 
    $$I_{p,q} \defeq \int_{0}^{1} x^p (1-x)^q \d x$$
    \begin{enumerate}
        \item Déterminer une relation entre $I_{p,q}$ et $I_{p+1, q-1}$ grâce à une IPP.
        \item En déduire l'expression de $I_{p,q}$ à l'aide de factorielles.
        $$\boxed{I_{p,q} = \frac{p! q!}{(p + q + 1)!}}$$
    \end{enumerate}
\end{itemize}
\url{https://fr.wikipedia.org/wiki/Intégrale_d'Euler} \\
\cite{calcul_infinitesimal} Chapitre IV, 3 Intégrales eulériennes, page 125.


\section{Théorème de \textsc{Fubini}}

\marginnote[-2mm]{Ce théorème a été démontré par le mathématicien italien Guido \textsc{Fubini} en 1907.}
\begin{theo}{\textsc{Fubini}}
    Soit $f: [a,b] \times [c, d] \to \K$ une application continue. Alors,
    $$\int_{a}^{b} \left ( \int_{c}^{d} f(x,y) \d y \right) \d x = \int_{c}^{d} \left ( \int_{a}^{b} f(x,y) \d x \right) \d y.$$
\end{theo}

\begin{marginfigure}[5cm]
    \centering
    \begin{tikzpicture}[
  x=(215:2em/sqrt 2), y=(0:2em), z=(90:2em),
  declare function={f(\x,\y)=((\x-3)^2+(-\y+3)^3)/8+3;}, 
  very thick, line join=round]
\draw [-stealth, black!75] (0,0,0) -- (5,0,0) node [below left] {$x$};
\draw [-stealth, black!75] (0,0,0) -- (0,5,0) node [below right] {$y$};
\draw [-stealth, black!75] (0,0,0) -- (0,0,5) node [right] {$z$};
\foreach \x in {1,...,4}
  \foreach \y [evaluate={\j=\x+.5; \i=\y+.5; \k=f(\j,\i);}] in {1,...,4}{
    \path [fill=black!50, draw=white] (\x, \y+1, 0) -- (\x+1, \y+1, 0) -- 
      (\x+1, \y+1, \k) -- (\x, \y+1, \k) -- cycle;
    \path [fill=black!25, draw=white] (\x+1, \y, 0) -- (\x+1, \y+1, 0) -- 
      (\x+1, \y+1, \k) -- (\x+1, \y, \k) -- cycle;
    \path [fill=black!10, draw=white] (\x, \y, \k)  -- (\x+1, \y, \k) -- 
      (\x+1, \y+1, \k) -- (\x, \y+1, \k) -- cycle;
  }
 \foreach \x in {1,...,4}
   \foreach \y in {1,...,4}{
 \draw [black, fill=black, fill opacity=0.125, 
    domain=0:1, samples=10, variable=\t] 
    plot (\x+\t, \y, {f(\x+\t,\y)}) -- 
    plot (\x+1, \y+\t, {f(\x+1,\y+\t)}) -- 
    plot (\x+1-\t, \y+1, {f(\x+1-\t,\y+1)}) --
    plot (\x, \y+1-\t, {f(\x,\y+1-\t)}) -- cycle;
  }
\end{tikzpicture}
    \caption*{\centering Cette figure ne correspond pas au théorème de \textsc{Fubini}}
\end{marginfigure}

Nous allons voir la démonstration de ce résultat sous forme d'exercice.

\begin{exercice}
    Pour tout $(x, t) \in [a, b] \times [c, d]$ on pose 
    $$\varphi(x, t) \defeq \int_{a}^{x} f(u, t) \d u.$$
    \begin{enumerate}
        \item Montrer que pour tout $x \in [a, b]$, l'application $t \mapsto \varphi(x, t)$ est continue sur $[c, d]$.
        \item On pose alors, pour tout $x  \in [a, b]$ 
        $$\psi(x) \defeq \int_{c}^{d} \varphi(x, t) \d t.$$
        Montrer que $\psi$ est de classe $\mathscr{C}^1$ sur $[a, b]$, préciser $\psi'$.
        \item En déduire:
        $$\forall x \in [a, b],\ \int_{a}^{x} \left ( \int_{c}^{d} f(u,t) \d t \right) \d u = \int_{c}^{d} \left ( \int_{a}^{x} f(u,t) \d u \right) \d t.$$
    \end{enumerate}
\end{exercice}

\marginnote[0cm]{Correction du sujet Mines Maths 2 PSI 2021 par Doc Solus.} 
\begin{solution}
    \begin{enumerate}
        \item Application du théorème de continuité des intégrales à paramètre. \\
        Pour la domination : $f$ est continue sur une partie fermée bornée de $\R^2$, donc d'après le théorème des bornes, $f$ est bornée sur $[a, b] \times [c, d]$ par une constante $M \in \Rp$.
        \item Application du théorème de dérivation des intégrales à paramètre à la fonction $x \mapsto \int_{c}^{d} \varphi(x, t) \d t$:
        \begin{itemize}
            \item $\forall t \in [c, d],\ x \mapsto \varphi(x, t)$ est de classe $\mathscr{C}^1$ sur $[a, b]$ car c'est la primitive s'annulant en $a$ de la fonction continue $x \mapsto f(x, t)$. 
            \item $\frac{\partial \varphi}{\partial x}(x, t) = f(x, t)$
            \item La domination se fait par le même constante $M$ que précédemment. 
        \end{itemize}
        $$\forall x \in [a, b] \quad \psi'(x) = \int_{c}^{d} f(x, t) \d t.$$
        \item Soit $x \in [a, b]$. D'une part,
        $$\psi(x) = \int_{c}^{d} \left ( \int_{a}^{x} f(u,t) \d u \right) \d t.$$
        D'autre part, d'après la question précédente et le théorème fondamental de l'analyse, 
        \begin{align*}
            \int_{a}^{x} \left ( \int_{c}^{d} f(u,t) \d t \right) \d u &= \int_{a}^{x} \psi'(u) \d u  = \psi(x) - \psi(a) \\
            \text{Or } \psi(a) &= \int_{c}^{d} \varphi(a, t)\ \d t \\
            \text{et } \forall t \in [c, d] \quad \varphi(a, t) &= \int_{a}^{a} f(u, t) \d u = 0
        \end{align*}
        d'où $\psi(a) = 0$ et le résultat. \\
        En particulier, pour $x = b$ on obtient le résultat final.
    \end{enumerate}
\end{solution}    


\section{Transformée de \textsc{Laplace}} 
\label{transformee_laplace}
\todoinline{Rechercher l'exercice corrigé sur les théorèmes des valeurs initiales / finales.}

La transformation de \textsc{Laplace} généralise la transformation de \textsc{Fourier} qui est également utilisée pour résoudre les équations différentielles : contrairement à cette dernière, elle tient compte des conditions initiales et peut ainsi être utilisée en théorie des vibrations mécaniques ou en électricité dans l'étude des régimes forcés sans négliger le régime transitoire. De manière générale, ses propriétés vis-à-vis de la dérivation permettent un traitement plus simple de certaines équations différentielles, et elle est de ce fait très utilisée en automatique. \\
Dans ce type d'analyse, la transformation de \textsc{Laplace} est souvent interprétée comme un passage du domaine temps, dans lequel les entrées et sorties sont des fonctions du temps, dans le domaine des fréquences, dans lequel les mêmes entrées et sorties sont des fonctions de la \say{ fréquence } (complexe) $p$. Ainsi; il est possible d'analyser simplement l'effet du système sur l'entrée pour donner la sortie en matière d'opérations algébriques simples (cf. théorie des fonctions de transfert en électronique ou en mécanique). 

\begin{defi}{Transformée de \textsc{Laplace}}
    Pour tout fonction $f \in \mathscr{C}(\Rp, \R)$, on note, lorsqu'elle converge, 
    $$\mathscr{L}(f)(p) \defeq \int_{0}^{+ \infty} \e^{-pt} f(t) \d t.$$
    La fonction $\mathscr{L}(f)$ est la \emph{transformée de \textsc{Laplace} de f}.
\end{defi}

\marginnote[0cm]{Sources : \cite{exos_oraux} + \cite{acamanes} (Exercice cerise Ch. 12)}
\underline{Démonstration du théorème de la valeur finale:}
\begin{itemize}
    \item Généralisation classique du théorème des bornes $\leadsto$ $f$ est bornée
    \item Changement de variable: $\varphi: u \mapsto \frac{u}{p}$
    \item Caractérisation séquentielle de la limite
    \item Théorème de convergence dominée
\end{itemize}



%\section{Exercice oral}
%\begin{exercice}
%    On pose
%    $$F:x \mapsto \int_x^{+\infty} \frac{\me^{-t}}{t} \d t.$$
%    \begin{enumerate}
%        \item Déterminer l'ensemble de définition de $F$. Étudier brièvement le comportement de la fonction $F$ et tracer sa courbe représentative.
%        \item Déterminer un  équivalent de $F$ en $+\infty$.
%        \item Montrer que $F(1) - F(x) - \ln x$ converge vers un réel. (pas sûr de cette question).
%    \end{enumerate}
%\end{exercice}

%\begin{enumerate}
%    \item $D = ]0, + \infty[$.
%    \item Intégration par parties ou comparaison série / intégrale: $F(x) \sim_{+\infty} \frac{\me^{x}}{x}$. \\
%    \url{https://www.bibmath.net/ressources/index.php?action=affiche&quoi=bde/analyse/integration/integralesimpropres&type=fexo} exercice 38. (à réécrire)\\
%    On remarque d'abord que $\int_{1}^{+\infty} \frac{\me^{-t}}{t} \d t$ converge: en effet, la fonction $t \mapsto \frac{\me^{-t}}{t}$ est continue et positive sur $[1, + \infty[$ et $\lim\limits_{t \to +\infty} t^2 \frac{\me^{-t}}{t} = 0$. On intègre ensuite par parties, en intégrant $t \mapsto \me^{-t}$ et en dérivant $\t \mapsto \frac{1}{t}$. On obtient, pour $x > 1$, 
%    $$\int_x^{+ \infty} \frac{\me^{-t}}{t} \d t = \left[ -\frac{\me^{-t}}{t} \right]_x^{+\infty} - \int_x^{+\infty} \frac{\me^{-t}}{t^2} \d t$$
%    $$\int_x^{+ \infty} \frac{\me^{-t}}{t} \d t= \frac{\me^{-x}}{x} - \int_x^{+\infty} \frac{\me^{-t}}{t^2} \d t.$$
%    Or, au voisinage de $+ \infty$, 
%    $$\frac{\me^{-t}}{t^2} = o\left( \frac{\me^{-t}}{t} \right).$$
%    Par intégration des relations de comparaison (les fonctions sont positives et intégrables), on trouve
%    $$\int_x^{+\infty} \frac{\me^{-t}}{t^2} \d t = o_{+\infty} \left( \int_x^{+\infty} \frac{\me^{-t}}{t} \d t \right).$$
%    On en déduit que
%    $$\int_x^{+\infty} \frac{\me^{-t}}{t} \d t \sim_{+\infty} \frac{\me^{-x}}{x}.$$
%    \item À faire.
%\end{enumerate}

\section{Version intégrale du lemme de \textsc{Cesàro}}
\begin{lemme}
    Soit $f$ une fonction continue telle que $\lim\limits_{+\infty} f = \ell$. Alors 
    $$\lim_{x \to + \infty} \frac{1}{x} \int_0^x f(t) \d t = \ell.$$
\end{lemme}

\begin{preuve}
    La démonstration est directement adaptée de celle de la version discrète. 
    Soit $\varepsilon > 0$. Comme la fonction $f$ converge vers $\ell$ en $+ \infty$, il existe $x_0 \in \Rp$ tel que pour tout $x \geqslant x_0,\ |f(x) - \ell| \leqslant \varepsilon$. \\
    Soit $x > x_0$,
    \begin{align*}
        \left| \frac{1}{x} \int_0^x f(t) \d t - \ell \right| &= \left| \frac{1}{x} \int_0^x (f(t) - \ell) \d t \right| \\
        \text{par l'inégalité triangulaire} &\leqslant \frac{1}{x} \int_0^x |f(t) - \ell| \d t \\
        &\leqslant \frac{1}{x} \Bigg( \underbrace{\int_{0}^{x_0} |f(t) - \ell| \d t}_{\defeq K} + \int_{x_0}^{x} \underbrace{|f(t) - \ell|}_{\leqslant \varepsilon} \d t \Bigg) \\
        &\leqslant \frac{K}{x} + \varepsilon
    \end{align*}
    Or $\lim\limits_{x \to \infty} \frac{K}{x} = 0$ donc il existe $x_1 \in \Rp$ tel que pour tout $x \geqslant x_1, \left| \frac{K}{x} \right| \leqslant \varepsilon$. \\
    Ainsi pour tout $x \geqslant \max \{ x_0, x_1 \}$, 
    $$\left| \frac{1}{x} \int_0^x f(t) \d t - \ell \right| \leqslant 2 \varepsilon.$$
    On en déduit le résultat. 
\end{preuve}
\chapter{Géométrie, courbes et surfaces}
\labch{geometrie_courbes_et_surfaces}

\begin{itemize}
    \item Géométrie élémentaire dans l'espace
    \item Réduction, tracé de coniques/quadratiques
    \item Orthoptique d'une parabole
    \item Courbe orthoptique de l'ellipse (cercle de \textsc{Monge})
    \item Tracé du pentagone régulier à la règle et au compas
\end{itemize}
\chapter{Fonctions d'une variable réelle}
\labch{fonction_une_variable_reelle}

\textsl{
(texte de \cite{oraux_x_ens_3})\\
À partir du \textsc{xvii}$^\me$ siècle, le développment du calcul infinitésimal, motivé par de nombreux problèmes de cinématique, de mécanique, ou de calcul des variations, fait de la \say{ fonction } l'objet central des mathématiques modernes, alors que jusque là, le \say{ nombre } était la base de l'édifice mathématique. Nous devons à \textsc{Bernoulli} et \textsc{Leibniz} le terme même de \say{ fonction }: pour \textsc{Bernoulli} (1698), une fonction de la variable $x$ est \say{ une quantité formée d'une manière quelconque à partir de $x$ et de constantes }. L'écriture $y = f(x)$ est introduite par \textsc{Euler} en 1734. Les fonctions sont représentées par des courbes dans le plan et \textsc{Euler} se demande si une courbe donnée correspond toujours à une fonction. C'est lui qui distingue les courbes continues, des courbes discontinues, qui sont le plus souvent, à cette époque, des graphes de fonctions continues par morceaux. Cette double conception des fonctions, comme expressions analytiques ou comme graphes du plan, ne sera pas vraiment éclaircie avant le \textsc{xix}$^\me$ siècle (c'est \textsc{Dirichlet} qui donnera la définition moderne d'une fonction comme correspondance; il proposera ainsi (en 1837) un exemple de fonction discontinue partout, la fonction $\chi$ définie par $\chi(x) = 1$ pour $x$ rationnel et $\chi(x)=0$ pour $x$ irrationnel). \textsc{Lagrange}, cherchant à établir les fondements de l'Analyse, s'en tient au point de vue formel et refuse de se référer à toute notion de limite. Ces hésitations empêchent les mathématiciens du \textsc{xvii}$^\me$ siècle  de mener jusqu'à leur achèvement certains de leurs travaux, comme l'étude de l'équation des cordes vibrantes. C'est la génération suivante, avec entre autres \textsc{Gauss}, \textsc{Cauchy}, \textsc{Bolzano} et \textsc{Abel}, qui donnera dans la première moitié du \textsc{xix}$^\me$ siècle un statut rigoureux aux notions de convergence, de continuité, \dots Quant au concept de limite d'une fonction numérique, on doit sans doute sa première définition précise à \textsc{Weierstrass}. 
}


\newpage

\section{Point fixe d'une fonction de \texorpdfstring{$[0, 1] \rightarrow [0, 1]$}{[0, 1] dans [0, 1]}}
Soit $f$ une fonction dérivable sur $[0, 1]$ telle que:
$$f(0) = f'(0) = f'(1) = 0 \text{ et } f(1) = 1.$$
Montrer qu'il existe $c \in ]0, 1[$ tel que $f(c) = c$. 

\begin{itemize}
    \item Poser $g(x)= f(x) - x$.
    \item L'objectif est de montrer que $g$ s'annule au moins une fois sur $[0, 1]$ en montrant l'existence de $x_0$ et $x_1$ dans $[0, 1]$ tels que $g(x_0) < 0$ et $g(x_1) > 0$ pour pouvoir appliquer le \textbf{théorème des valeurs intermédiaires}.
    \item Raisonner par l'absurde sur l'existence de $x_0$ et de $x_1$ et écrire la dérivée de $f$ comme la limite de son taux d'accroissement pour aboutir à des contradictions. 
\end{itemize}

\section{Convexité et signe}
\begin{exercice}
    Que peut-on dire sur une fonction concave et positive sur $\R$ ?
\end{exercice}

\begin{elem_sol}
    Faire un dessin...
\end{elem_sol}

\section{\textsc{Rolle} à l'infini}
\begin{theo}{}
    Soit $f: \Rp \to \R$, de classe $\mathscr{C}^2$ et telle que $f(x) \xrightarrow[x \to + \infty]{} f(0)$. Alors il existe $c \in \Rpe$ et $d \in \Rp$ tel que $f'(c) = f''(d) = 0$.
\end{theo}

\section{Théorème de \textsc{Darboux}}
\begin{theo}{\textsc{Darboux}}
    Soit $f$ une fonction réelle, dérivable sur un intervalle $[a, b]$. Pour tout réel $k$ compris entre $f'(a)$ et $f'(b)$, il existe un réel $c \in [a, b]$ tel que $c = f'(k)$.
\end{theo}

\underline{Fonction de \textsc{Darboux}:} \\
Fonction dérivable en tout point, mais dont la dérivée est discontinue en $0$:

\begin{alignat*}{2}
    \text{Soit } f\ :\ \R\ &\longrightarrow\ \R\\
    x\ &\longmapsto\ 
    \begin{cases}
        x^2 \sin \left( \frac{1}{x^2} \right) &\text{ si } x \not= 0,\\
        0 &\text{ sinon}.
    \end{cases}
\end{alignat*}

\section{Uniforme continuité et intégrale convergente}
\begin{exercice}
    Soit $f : \Rp \to \R$ uniformément continue telle que $\int_0^{+\infty} f$ converge. Montrer que $f(x) \xrightarrow[x \to + \infty]{} 0$.
\end{exercice}

\begin{solution}
    Soit $x \in \Rp$.
    \begin{align*}
        f(x) &= f(x) - \frac{1}{2\eta_{\varepsilon}} \int_{x-\eta_{\varepsilon}}^{x+\eta_{\varepsilon}} f(t) \d t + \frac{1}{2\eta_{\varepsilon}} \int_{x-\eta_{\varepsilon}}^{x+\eta_{\varepsilon}} f(t) \d t \\
        &= \frac{1}{2\eta_{\varepsilon}} \int_{x-\eta_{\varepsilon}}^{x+\eta_{\varepsilon}} \big(f(x) - f(t) \big) \d t + \frac{1}{2\eta_{\varepsilon}} \int_{x-\eta_{\varepsilon}}^{x+\eta_{\varepsilon}} f(t) \d t \\
        |f(x)| &\leqslant \frac{1}{2\eta_{\varepsilon}} \int_{x-\eta_{\varepsilon}}^{x+\eta_{\varepsilon}} \underbrace{\big|f(x) - f(t)\big|}_{\leqslant \varepsilon} \d t + \left| \frac{1}{2\eta_{\varepsilon}} \int_{x-\eta_{\varepsilon}}^{x+\eta_{\varepsilon}} f(t) \d t \right| \\
        &\leqslant \varepsilon + \underbrace{\left|\frac{1}{2\eta_{\varepsilon}} \int_{x-\eta_{\varepsilon}}^{x+\eta_{\varepsilon}} f(t) \d t \right|}_{\mathclap{\longrightarrow 0 \text{ d'après le critère de \textsc{Cauchy}...}}}
    \end{align*}
\end{solution}

\marginnote[-7cm]{
    \begin{defi}{Continuité uniforme}
        Soit $f$ une fonction de $\R$ dans $\R$. La fonction $f$ est \emph{uniformément continue} si
        $$\forall \varepsilon > 0 \quad \exists \eta_{\varepsilon} > 0 \quad \forall(x, y) \in \R^2,$$
        $$\left( |x-y| \leqslant \eta_{\varepsilon} \implies |f(x) - f(y)| \leqslant \varepsilon \right).$$
    \end{defi}
}

\section{Lemme de \textsc{Croft}}
\marginnote[0cm]{Sources : \cite{exos_oraux} p. 259 \& \cite{oraux_x_ens_3} p. 322}

\begin{lemme}
    Soit $f : \Rp \to \R$ telle que, pour tout $a > 0$, la suite $\big(f(na) \big)_{n \geqslant 0}$ tend vers 0. \\
    Montrer que si la fonction $f$ est uniformément continue, on a $\lim\limits_{x \to + \infty} f(x) = 0$.
\end{lemme}

Sources : correction principalement de \cite{oraux_x_ens_3} avec des précisions venant de \cite{exos_oraux}.

\begin{preuve}
    L'hypothèse signifie que $f$ tend vers $0$ selon toute suite arithmétique de la forme $(na)_{n \geqslant 0}$. Lorsque la fonction est uniformément continue, on peut contrôler son comportement entre deux termes consécutifs de la suite. Plus précisément, soit $\varepsilon > 0$. La continuité uniforme de $f$ permet de choisir $\eta_\varepsilon > 0$ tel que pour tout $(x, y) \in (\Rp)^2$, 
    $$|x-y| \leqslant \eta_\varepsilon \Rightarrow |f(x) - f(y)| \leqslant \varepsilon.$$
    Puisque $\eta_\varepsilon > 0$, la suite $\big(f(n\eta_\varepsilon)\big)_{n \geqslant 0}$ tend vers $0$ par hypothèse. Fixons $N$ tel que $|f(n \eta_\varepsilon)| \leqslant \varepsilon$ pour $n \geqslant N$. \\
    Soient $x \geqslant N \eta_\varepsilon$ et $n \defeq \min\ens[\big]{ k \in \N \tq x \leqslant k \eta_\varepsilon }$ qui est bien défini. Alors $|x-n\eta_\varepsilon| \leqslant \eta_\varepsilon$. On a alors $|f(x) - f(n \eta_\varepsilon)| \leqslant \varepsilon$ de sorte que par l'inégalité triangulaire, 
    \begin{align*}
        |f(x)| &\leqslant |f(x) - f(n \eta_\varepsilon)| + |f(n \eta_\varepsilon)| \\
        &\leqslant 2 \varepsilon.
    \end{align*}
    Ceci étant valable pour tout $x \geqslant N \eta_\varepsilon$, on a bien prouvé que $f$ tend vers $0$ en $+ \infty$.
\end{preuve}  

\begin{marginfigure}[-3cm]
    \begin{tikzpicture}[scale=5]
    \draw[->] [thick](-0.1,0) -- (0.8,0);
    
    \draw (0,0) node [above=5pt, red,fill=white]{$N \eta_\varepsilon$};
    \draw (0,0) node {$|$};
    
    \draw (0.16,0) node [above=7pt, red,fill=white]{$\cdots$};

    \draw (0.4,0) node [above=5pt, red,fill=white]{$(n-1) \eta_\varepsilon$};
    \draw (0.4,0) node {$|$};
    
    \draw [fill] (0.6,0) circle [radius=0.3pt];
    \draw (0.6,0) node [below=2pt, blue,fill=white]{$x$};

    \draw (0.7,0) node [above=5pt, red,fill=white]{$n \eta_\varepsilon$};
    \draw (0.7,0) node {$|$};
\end{tikzpicture}
\end{marginfigure}


\begin{exercice}
    \cite{exos_oraux} (p.261)
    Déterminer les fonctions $f \in \mathscr{C}^1(\R, \R)$ telles que $f \circ f = f$.
\end{exercice}

\begin{itemize}
    \item Equation fonctionnelle
\end{itemize}
\chapter{Suites et séries de fonctions}
\labch{suites_et_series_de_fonctions}

Le Calcul infinitésimal est l'apprentissage du maniement des \emph{inégalités} bien plus que des égalités, et on pourrait de résumer en trois mots:
\begin{center}
        MAJORER, MINORER, APPROCHER.
\end{center}

V.1 Écart de deux fonctions de \cite{calcul_infinitesimal} \\
De même qu'on cherche à \emph{approcher} un \emph{nombre} inconnu (défini par un procédé quelconque) à l'aide de nombres décimaux (ou rationnels), de même il est naturel en Analyse de chercher à \say{ approcher } une \emph{fonction} complexe inconnue (qui peut être définie par des procédés variés, somme de série, intégrale dépendant d'un paramètre, solution d'équation différentielle, etc.) à l'aide de fonctions que l'on considère comme \emph{connues} (polynômes, fonctions exponentielles, fonctions trigonométriques, etc.). Mais il faut préciser ce qu'on entend par \say{ approcher }, c'est-à-dire \say{ mesurer } en quelque sorte l'\say{ écart } de deux fonctions, de même que la valeur absolue $|x-y|$ mesure l'écart de deux nombres réels ou complexes. \\
L'idée la plus naturelle est que si une fonction $g$ \say{ approche } une fonction $f$ dans un ensemble $E$ où elles sont toutes deux définies, alors, pour chaque $x_0 \in E$, la \emph{valeur} $g(x_0)$ de $g$ doit \emph{approcher} la \emph{valeur} $f(x_0)$ de $f$ au sens usuel, c'est-à-dire que $|f(x_0)-g(x_0)|$ doit être \say{ petit }. Comme ceci doit avoir lieu en \emph{chaque} poit $x_0$ de $E$, on est conduit à prendre pour \say{ écart } de deux fonctions complexes $f$, $g$ définies dans $E$ le nombre
$$d(f, g) = \sup_{x \in E} |f(x)-g(x)|.$$
Lorsqu'il s'agit de fonctions \emph{réelles} $f, g$ définies dans un intervalle $E = [a, b]$ de $\R$, l'idée d'\say{ écart } que nous venons de définir peut se concrétiser graphiquement de la façon suivante: dire que $d(f,g) \leqslant \varepsilon$ signifie que pour tout $x \in E$ on a $g(x)-\varepsilon \leqslant f(x) \leqslant g(x) + \varepsilon$, c'est-à-dire que le graphe de $f$ est \emph{tout entier} contenu dans le \say{ bande } de demi-largeur $\varepsilon$ autour du graphe de $g$. \\
Pour distinguer cette idée d'\say{ approximations } d'autres notions que nous examinerons plus tard \textcolor{red}{à réécrire}, nous dirons qu'il s'agit d'\emph{approximation uniforme} d'une fonction par une autre dans un ensemble $E$ où elles sont toues deux définies; il est important de remarquer que cette notion \emph{dépend essentiellement} de l'ensemble $E$ que l'on considère: si $f$ et $g$ sont toutes deux définies dans un ensemble plus grand $E'$, la relation $|f(x) - g(x)| \leqslant \varepsilon$ pour $x \in E$ n'entraîne nullement $|f(x)-g(x)| \leqslant \varepsilon$ pour $x \in E'$. \\
Étant donnés deux ensembles de fonctions $\mathscr{F}$ (les fonctions \say{ inconnues }) et $\mathscr{G}$ (les fonctions \say{ connues }) toutes définies dans un même ensemble $E$, nous dirons pour abréger qu'\emph{on peut approcher uniformément dans} $E$ les fonctions de $\mathscr{F}$ par les fonctions de $\mathscr{G}$ si, pour toute fonction $f \in \mathscr{F}$ et \emph{tout nombre} $\varepsilon > 0$, il existe une fonction $g \in \mathscr{G}$ (dépendant de $f$ et de $\varepsilon$) telle que l'écart $d(f,g) \leqslant \varepsilon$, c'est-à-dire que 
$$|f(x) - g(x)| \leqslant \varepsilon \quad \text{pour tout } x \in E.$$


\begin{tikzpicture}
    \begin{axis}[width=6.5cm,
        axis lines=middle,
        grid=major,
        xmin=-0.1, xmax=1.1,
        ymin=-0.1, ymax=1.1,
        % xlabel=$x$, xlabel style={right},
        % ylabel=$y$, ylabel style={above},
        tick style={thick},
        ticklabel style={font=\normalsize},
        xtick={0, 1}, 
        ytick={0, 1},
        % legend entries={0.5x},
            legend style={
            at={(1.05,0.4)},
            anchor=north,
            legend columns=1},
            legend cell align={left}
    ]
    
    \def\a{-0.1}
    \def\b{1.1}
    \def\eps{0.5}
    
    \addplot[blue,thick,samples=100,domain=0:\b] {x^3+x-2} {};

    \end{axis}
\end{tikzpicture}

\newpage

\section{Théorème d'approximation de \textsc{Weierstrass}}
Aussi connu sous le nom  de \emph{théorème de \textsc{Stone-Weierstrass}}

\begin{theo}
    Toute fonction continue sur un segment $[a, b]$ de $\R$ à valeurs dans $\R$ ou $\C$ est limite uniforme sur $[a, b]$ d'une suite de polynômes.
\end{theo}

\begin{preuve}
    \cite{calcul_infinitesimal} page 157.
\end{preuve} 

L'exercice suivant montre qu'il existe une suite de polynômes $(P_n)$ qui converge uniformément vers la fonction racine carrée sur $[0, 1]$.

\begin{exercice}
    \emph{Exercice 5. TD Ch. VIII}\\
    Soit la suite de fonctions définie pour tout $x \in [0, 1]$ par
    $$
    \begin{cases}
        P_0(x) &= 0,\\
        P_{n+1}(x) &= P_n (x) + \frac{1}{2} \big( x-P_n (x)^2 \big).
    \end{cases}
    $$
    Montrer que $(P_n)$ converge uniformément vers une fonction $f$ sur $[0, 1]$.
\end{exercice}

\begin{marginfigure}[-5cm]
	\begin{tikzpicture}
    \begin{axis}[width=6.5cm,
        axis lines=middle,
        grid=major,
        xmin=-0.1, xmax=1.1,
        ymin=-0.1, ymax=1.1,
        % xlabel=$x$, xlabel style={right},
        % ylabel=$y$, ylabel style={above},
        tick style={thick},
        ticklabel style={font=\normalsize},
        xtick={0, 1}, 
        ytick={0, 1},
        % legend entries={0.5x},
            legend style={
            at={(1.05,0.4)},
            anchor=north,
            legend columns=1},
            legend cell align={left}
    ]
    
    \def\a{-0.1}
    \def\b{1.1}
    
    \addplot[blue,thick,samples=100,domain=0:\b] {x^(1/2)};
    \addplot[red,thick,samples=100,domain=\a:\b] {0};
    \addplot[red,thick,samples=100,domain=\a:\b] {x/2};
    \addplot[red,thick,samples=100,domain=\a:\b] {-1/8*x^2+x};
    \addplot[red,thick,samples=100,domain=\a:\b] {-1/128*x^4+1/8*x^3-5/8*x^2+3/2*x};
    \end{axis}
\end{tikzpicture}


%import matplotlib.pyplot as plt
%import numpy as np
%from numpy.polynomial import Polynomial

%PAS = 1e-3
%n = 8
%X = np.arange(0, 1, PAS)


%P = Polynomial([0])
%plt.plot(X, P(X))

%for k in range(n):
%    P = P + 1/2 * (Polynomial([0, 1]) - P ** 2)
%    plt.plot(X, P(X))

%racine = [np.sqrt(x) for x in X]
%plt.plot(X, racine, 'r')
%plt.show()

\end{marginfigure}

Peut-on génraliser à $\R$ ? \dots

\begin{prop}
    Si $(P_n)_{n \in \N}$ est une suite de polynômes convergeant uniformément sur $\R$ vers une fonction $f$, alors $f$ est un polynôme.
\end{prop}

\begin{preuve}
    Source : \cite{exos_oraux} \& \cite{maths-france}. \\
    Soit $(P_n)_{n \in \N}$ une suite de polynômes convergeant uniformément sur $\R$ vers une fonction $f$. \\
    D'après la critère de \textsc{Cauchy} uniforme, il existe un rang $n$ tel que pour tout $p \in \N$, 
    $$\Ninf{P_{n+p} - P_n} \leqslant 1.$$
    La fonction polynomiale $P_{n+p} - P_n$ est donc bornée sur $\R$ autrement dit elle est constante. On a alors,
    $$\forall (p, x) \in \N \times \R,\ P_{n+p}(x) = P_n(x) + P_{n+p}(0) - P_n(0) \xrightarrow[p \to + \infty]{} P_n(x) + f(0) - P_n(0)$$
    donc par unicité de la limite simple, $f : x \mapsto P_n(x) + f(0) - P_n(0)$, qui définit bien une fonction polynomiale. 
\end{preuve}

\section{Approximation polynomiale de \textsc{Bernstein}}
\emph{Exercice 10. TD VIII} \\
\underline{Démonstration:} \cite{calcul_infinitesimal} page 159.

\begin{tcolorbox}
    Soit $f$ une fonction $k$-lipschitzienne sur $[0, 1]$. Le polynôme de \textsc{Bernstein} d'ordre $n$ associé à $f$ est le polynôme
    $$\Bernstein_n(f)(x) = \sum_{k=0}^{n} f \left( \frac{k}{n} \right) \binom{n}{k} x^k (1-x)^{n-k}$$
    La suite $(\Bernstein_n(f))_n$ des polynômes de \textsc{Bernstein} converge uniformément vers $f$ sur $[0, 1]$. \\
    \textit{Ce résultat s'étend à toute fonction continue sur un segment à valeurs dans $\C$}.
\end{tcolorbox}

Le sujet X/ENS PSI 2018 propose une élégante démonstration de ce résultat d'analyse pure en passant par les probabilités. (Je crois que la version du TD est un peu différente).
    
\begin{box_enonce}
    
    Préciser les hypothèses sur $f$.
        
    Soit $x \in ]0, 1[$ et $n \in \Ne$. On considère $X_1, \dots, X_n$ des variables aléatoires mutuellement indépendantes et suivant toutes la même loi de \textsc{Bernoulli} de paramètre $x$. On pose
    $$S_n = \frac{X_1 + \cdots + X_n}{n}.$$
    \begin{enumerate}
        \item Exprimer $\E(S_n)$, $\V(S_n)$ et $\E(f(S_n))$ en fonction de $x$, $n$ et du polynôme $\Bernstein_n(f)$.
        \item En déduire les inégalités:
        $$\sum_{k=0}^{n} \left| x- \frac{k}{n} \right| \binom{n}{k} x^k (1-x)^{n-k} \leqslant \V(S_n)^{1/2} \leqslant \frac{1}{2\sqrt{n}}.$$
        \item Montrer que $\lambda^\alpha \leqslant 1+\lambda$ pour tout réel $\lambda > 0$ et en déduire l'inégalité:
        $$\left|x-\frac{k}{n} \right|^\alpha \leqslant n^{-\alpha/2} \Bigg(1 + \sqrt{n} \left|x - \frac{k}{n} \right| \Bigg)$$
        pour tout $x \in ]0, 1[, n \in \Ne$ et $k \in \llbracket 1, n \rrbracket$.
        \item Soit $n \in \Ne$. Montrer que 
        $$\Ninf{f-\Bernstein_n(f)} \leqslant \frac{3k}{2} \frac{1}{n^{\alpha/2}}.$$
        Conclure.
    \end{enumerate}
\end{box_enonce}

\section{Intégration d'une série de fonctions}
Soit $S:x \to \sum\limits_{n=1}^{+\infty} \frac{(-1)^n}{1+n^2 x^2}$.
\begin{itemize}
    \item Donner l'ensemble de définition de $S$ et donner un équivalent en $+\infty$.\\
    $\blacktriangleright$  $D_S = \Re$.\\
    $\blacktriangleright$ On se doute que $S$ \textbf{se comporte comme} $\frac{1}{x^2}$ en $+\infty$. L'idée est donc de déterminer la limite de $x^2 S(x)$ en $+\infty$. Le \textbf{théorème d'interversion des limites} permet d'affirmer que cette limite est finie et qu'elle est égale à $c = \sum\limits_{n=1}^{+\infty} \frac{(-1)^n}{n^2}$. Une séparation des termes pairs et impairs de la somme (et le résultat du \href{https://fr.wikipedia.org/wiki/Problème_de_Bâle}{problème de Bâle}) permet de montrer que $c = -\frac{\pi^2}{12}$ et donc $S(x) \sim_{+\infty} -\frac{\pi^2}{12x^2}$.
    \item Montrer que $S$ est intégrable sur $\Rpe$ et calculer $\int_0^{+\infty} S(t) \mathrm{d}t$.\\
    $\blacktriangleright$ Comme $S$ est une série alternée, on peut lui appliquer le \textbf{théorème des séries alternées} et écrire que 
    $$|S(x)| \leqslant \frac{1}{1+x^2} \text{ (majoration du reste d'ordre 1)}$$
    L'intégrabilité du majorant sur $\Rpe$ assure celle de $S$ sur cet ensemble. \textcolor{green}{Est-ce que l'intégrabilité sur R+* des termes de la somme et leur CU vers $S$ impliquent l'ingrabilité de $S$ sur cet ensemble ? c.f. théorème de la convergence dominée peut-être...}\\
    $\blacktriangleright$ L'interversion série/intégrale permet de montrer que $\int_0^{+\infty} S(t) \d t = \frac{\pi}{2} \sum\limits_{n=1}^{+\infty} \frac{(-1)^n}{n} = -\frac{\pi}{2} \ln(2)$ (c.f. \nameref{deux_sommes}).\\
    \textcolor{green}{Dans la correction (p. 374), on effectue le calcul sur une somme partielle et on détermine ensuite la limite de cette somme. J'avais naïvement travailler avec l'intégrale jusqu'en +infini et la somme aussi, \textbf{est-ce licite ?}}.
\end{itemize}

\section{Équivalent d'une série de fonctions}
\begin{exercice}
    On pose $f(x) = \sum\limits_{n=1}^{+\infty}\frac{x}{n(1+nx^2)}$. Donner un équivalent de $f$ et $0^+$.
\end{exercice}

\begin{solution}
    Éffectuer une comparaison série/intégrale aux termes de la somme pour encadrer $f$ (ne pas oublier de justifier l'intégrabilité des fonctions sur $[1, +\infty[$):
    $$\int_{2}^{+\infty} \frac{x}{t(1+tx^2)} \d t + \frac{x}{1+x^2} \leqslant f(x) \leqslant \int_{1}^{+\infty} \frac{x}{t(1+tx^2)} \d t + \frac{x}{1+x^2}.$$ 
    Décomposer les intégrandes en éléments simples et calculer les intégrales:
    $$-2x \ln(x) + o_0(x\ln(x)) \leqslant f(x) \leqslant -2x \ln(x) + o_0(x\ln(x))$$
    Finalement, 
    $$f(x) \isEquivTo{0^+} -2x\ln(x)$$
\end{solution}

\section{Série de fonctions continues dont la somme est discontinue}
\begin{prop}
    \marginnote[0cm]{\cite{contre-exemples} p.257}
    Soit la suite $(f_n)_{n \geqslant 0}$ d'applications continues de $[0,1]$ dans $\R$, de terme général
    \begin{alignat*}{2}
        f_n\ :\ [0,1]\ &\longrightarrow\ \R\\
        x\ &\longmapsto\ f_n(x) = (1-x)x^n.
    \end{alignat*}
    La somme des $f_n$ est discontinue.
\end{prop}

\begin{preuve}
    Pour tout point $x$ de $[0,1[$, la série $\sum f_n(x)$ est le produit par $(1-x)$ d'une série géométrique de raison $x$ où $0 \leqslant x < 1$, donc elle converge et sa somme est égale à $(1-x)/(1-x) = 1$. De plus $f_n(1) = 0$ pour tout entier naturel $n$. La série de fonctions $\sum f_n$ converge donc simplement sur le segment $[0,1]$ et sa somme est l'application 
    \begin{alignat*}{2}
        S\ :\ [0,1]\ &\longrightarrow\ \R\\
        x\ &\longmapsto\ S(x) =
        \begin{cases}
            1 &\text{ si } x \in [0,1[,\\
            0 &\text{ si} x = 1,
        \end{cases}
    \end{alignat*}
    qui est discontinue en $1$.
\end{preuve}


\section{Convergence uniforme d'une suite de fonctions polynomiales}
Soient $(P_n)_n$ une suite de fonctions polynomiales de $\R$ dans $\R$.\\
On suppose que $(P_n)_n$ converge uniformément vers une fonction $f$ sur $\R$. Montrer que $f$ est une fonction polynomiale. 
\begin{itemize}
    \item $(P_n)_n$ CU donc il existe $n \in \N$ tel que pour tout $p \in \N$, $\Vert P_{n+p}-P_n \Vert_{\infty} \leqslant 1$. 
    \item ...
\end{itemize}

\section{Fonction \texorpdfstring{$\zeta$}{zêta} alternée}
\marginnote[0cm]{\emph{Exercice 13. TD VIII}}
On définit la fonction $\zeta$ alternée $F$ comme suit
$$\boxed{F(x) = \sum_{n=1}^{+ \infty} \frac{(-1)^{n-1}}{n^x}}.$$
\begin{itemize}
    \item Déterminer l'ensemble de défintion de $F$ et trouver une relation entre $F$ et $\zeta$.
    \begin{itemize}
        \item Calculer $\zeta(x) - F(x)$ pour trouver que $\zeta(x) = \frac{1}{1-2^{1-x}} F(x)$.
    \end{itemize}
    \item Déterminer $\displaystyle \lim_{x \to 1} (x-1) \zeta(x)$.
    \begin{itemize}
        \item En comparant $\zeta$ à une intégrale, on peut montrer que $\frac{1}{1-x} \leqslant \zeta(x) \leqslant \frac{1}{1-x} + 1$.
    \end{itemize}
    \item On peut aussi procéder de la manière suivante: $\zeta$ est décroissante sur $]1, +\infty[$ donc $\zeta$ admet une limite $\ell \in \Rp \cup \{ + \infty \}$ en $1$.\\
        $\blacktriangleright$ Si $\ell \in \Rp$, par passage à la limite dans l'inégalité $\zeta(x) \geqslant \sum\limits_{n=1}^{N} \frac{1}{n^x}$, $$\ell \geqslant \sum\limits_{n=1}^{N} \frac{1}{n} \xrightarrow[N \to + \infty]{} +\infty.$$ 
        Donc $\ell = + \infty$ et $\displaystyle \lim_{x \to 1} \zeta(x) = +\infty$. 
    \item Déterminer $\displaystyle \lim_{x \to +\infty} F(x)$ ainsi qu'un équivalent de $\zeta$ en $+\infty$.
    \begin{itemize}
        \item On peut montrer que (\textcolor{green}{à détailler}) $\zeta(x) \sim_{+ \infty} 1$.
    \end{itemize}
\end{itemize}

\section{Fonctions continues nulle part dérivables}
\epigraph{\emph{``Je me détourne avec effroi et horreur de cette plaie lamentable des fonctions continues qui n'ont point de dérivées.''}}{--- Charles \textsc{Hermite} (1893)}
    
\epigraph{\emph{``C’est un cas où il est vraiment naturel de penser à ces fonctions continues sans dérivées que les mathématiciens ont imaginées, et que l’on regardait à tort comme de simples curiosités mathématiques, puisque l’expérience peut les suggérer.''}}{--- Jean \textsc{Perrin} \footnote{Physicien, chimiste et homme politique français (1870 - 1942), prix Nobel de physique 1926.}, au sujet du mouvement brownien}
    
- thème Ch. 8 \\
- DS MPSI 2015 \\
- \url{https://share.miple.co/content/XEZ7y9BayeSN1} \\
- \url{http://christophebertault.fr/documents/articles/Article - Une famille nombreuse de fonctions continues partout derivables nulle part.pdf}

\begin{itemize}
    \item Intégrale à paramètre vs. série de fonctions
    \item Développements asymptotiques de sommes de séries de fonctions
\end{itemize}

%\begin{figure}[!h]
%   \begin{floatrow}
%\ffigbox{\input{bolzano-lebesgue}}%
%        {\caption{first figure}\label{fig:example-1}}
%\hfill
%\ffigbox{\input{blanc-manger}}%
%        {\caption{second figure}\label{fig:example-2}}
%   \end{floatrow}
%\end{figure}

Pour découvrir d'autres \emph{courbes remarquables}, lire le paragraphe éponyme dans \cite{contre-exemples} page 350. 

\chapter{Séries entières}
\labch{series_entieres}

\section{Matrice et série entière}
\begin{exercice}
    Soit $m \in \Ne$ et $A \in \M_m(\R)$ vérifiant $A^3 + A = 0$.\\
    Montrer que son rang est pair. Étudier la convergence et la somme de $\sum\limits_{n=0}^{+\infty} x^n \Tr (A^n)$.
\end{exercice}

\begin{solution}
    D'après l'énoncé, le polynôme $P(X) \defeq X^3 + X$ est annulateur de la matrice $A$. Le polynôme $P = X(X-\i)(X+\i)$ est scindé à racines simples dans $\C$ donc la matrice $A$ est diagonalisable dans $\C$. \\
    La diagonalisabilité de la matrice $A$ équivaut à $\smashoperator{\sum\limits_{\lambda \in \Sp A}} \dim E_\lambda(A) = m$ soit $\dim E_0(A) + \dim_{-\i}(A) + \dim_{\i}(A) = 0$. Comme $E_0(A) = \Ker A$, d'après le théorème du rang, $\dim E_0(A) = m - \Rg A$. On sait aussi (\textcolor{red}{à détailler peut être}) que $p \defeq \dim E_{-\i}(A) = \dim E_{\i}(A)$. On obtient alors
    \begin{align*}
        & m - \Rg A + 2p = m \\
        \text{soit } & \Rg A = 2p.
    \end{align*}
\end{solution}

\section{Série génératrice des polynômes d'\textsc{Hermite}}
\begin{defi}{Polynômes d'\textsc{Hermite}}
    La suite des polynômes d'\textsc{Hermite}, notée $(\Hermite_n)_{n \in \N}$, est définie comme l'unique suite de polynômes réels tels que:
    $$\forall(x, t) \in \R^2,\ \exp(tx-t^2/2) = \sum_{n=0}^{+\infty} t^n \Hermite_n(x) \qquad (*)$$
\end{defi}

\begin{preuve}
    \begin{itemize}
    \item $\blacktriangleright$ L'existence se prouve en faisant le produit de \textsc{Cauchy} des développements en série entière de $\exp(tx)$ et de $\exp(-t^2/2)$.\\
    $\blacktriangleright$ L'unicité se justifie par l'unicité des coefficients d'un développement en série entière. 
    \item Pour montrer que pour tout $n \in \Ne,\ (n+1)\Hermite_{n+1}=X\Hermite_n-\Hermite_{n-1}$, il faut dériver $(*)$ par rapport à $t$, effectuer des changements d'indices et utiliser l'unicité des coefficients du DES. 
    \item On peut montrer en dérivant  la série de fonctions $\sum \big(x \mapsto t^n \Hermite_n(x) \big)$ terme à terme sur $\R$ que pour tout $n \in \Ne,\ \Hermite'_n=\Hermite_{n-1}$.\\
    La dérivation est justifiée par l'application du \ptnclegras{théorème de dérivation terme à terme}, en particulier montrer soigneusement la CN de $\sum f'_n$ sur tout segment $[-a, a] \subset \R$ (qui entraîne la CU sur $\R$) $\left(|\Hermite'_n(x)| \leqslant \e^{|x|} \right)$.\\
    On en déduit que $(\Hermite_n)_n$ forme une base de $\R[X]$.\\
    \end{itemize}
\end{preuve}


\section{Théorème abélien ou taubérien sur les séries numériques}
Soit $f(x) = \sum\limits_{n=0}^{+\infty} a_n x^n$ une série entière de rayon de convergence égal à $1$.
\begin{itemize}
    \item \underline{Un théorème abélien:} si $\sum a_n$ est convergente, alors $f$ est définie et continue en $1$.
    \item \underline{Un théorème taubérien:} si $(a_n)_n$ est à termes positifs et $f$ a une limite à gauche en $1$, alors $\sum a_n$ est convergente, de somme $\displaystyle \lim_{1^-}f$.
\end{itemize}

\section{Fonction non développable en série entière}
\cite{contre-exemples}
\begin{alignat*}{2}
    \text{Soit } f\ :\ \R\ &\longrightarrow\ \R\\
    x\ &\longmapsto\ 
    \begin{cases}
        0 &\text{ si } x \leqslant 0,\\
        \exp{\left(-\frac{1}{x}\right)} &\text{ sinon}.
    \end{cases}
\end{alignat*}
        

\section{Comparaison de séries entières au bord}
\begin{exercice}    
    Soient $\sum a_n$ et $\sum b_n$ des séries à termes positifs, on pose:
    $$f:x \mapsto \sum_{n=0}^{+\infty} a_n x^n \text{ et } g:x \mapsto \sum_{n=0}^{+\infty} b_n x^n.$$
    On suppose que les rayons de convergence valent $1$, que $\sum b_n$ diverge et que $a_n = o(b_n)$.\\
    Montrer que $g(x) \xrightarrow[x \to 1^-]{} + \infty$ et que $f = o_{1^-}(g)$.
\end{exercice}

\begin{elem_sol}
    \begin{itemize}
        \item Même méthode que pour la fonction zêta alternée.
        \item Soit $\varepsilon > 0$. Alors il existe $n_0 \in \N$ tel que pour tout $n \geqslant n_0$, $0 \leqslant a_n \leqslant \frac{\varepsilon}{2} b_n$. (détailler le dernier argument qui permet de conclure rigoureusement).
    \end{itemize}
\end{elem_sol}

\section{Développement en série entière}
La fonction $f$ définie par:
$$f(x) = \exp(x^2) \int_{x}^{+ \infty} \exp(-t^2)\ \d t$$
est-elle développable en série entière ? Calculer les coefficients du développement en série entière à l'aide de factorielles, on utilisera que $\int_{0}^{+\infty} \exp(-t^2)\ \d t = \frac{\sqrt{\pi}}{2}$. 
\begin{itemize}
    \item Bien justifier l'existence de $f$ sur $\R$.
    \item L'écriture de $f$ sous la forme:
    $$\forall x \in \R,\ f(x) = \exp(x^2) \times \left(\int_{0}^{+ \infty} \exp(-t^2)\ \d t + \int_{0}^{x} \exp(-t^2)\ \d t\right)$$
    permet de justifier que $f$ est DSE (comme primitive d'une fonction DSE et d'un produit de fonctions DSE).
    \item Remarquer que $f$ vérifie
    $$f' -2xf + 1 = 0$$
    \item On en déduit que 
    $$\forall p \in \N,\ a_{2p} = \frac{\sqrt{\pi}}{2p!} \text{ et } a_{2p+1} = -\frac{2^{2p} p!}{(2p+1)!}.$$
\end{itemize}


A rajouter:
\begin{itemize}
    \item Transformée de \textsc{Laplace} d'une série entière
    \item Formule de \textsc{Cauchy} et applications
    \item Fonction de \textsc{Bessel} et intégrale de \textsc{Wallis}
\end{itemize}

\chapter{Séries de \textsc{Fourier}}
\labch{series_de_fourier}

\begin{itemize}
    \item Inégalité isopérimétrique (ou de \textsc{Wirtinger})
    \item Calcul de $\zeta(2)$ et $\zeta(4)$
    \item Si la série de \textsc{Fourier} converge uniformément...
    \item Equation différentielle anticipante 
    \item Interversions de \textsc{Fourier}
\end{itemize}
\chapter{Équations différentielles \& Calcul différentiel }
\labch{equations_differentielles_et_calcul_differentiel}


\section{Lemme de \textsc{Gronwall}, application à une équation différentielle}
\begin{box_titre}{Lemme de \textsc{Gronwall}}
    Soient $\phi, \vaphi$ et $y$ trois fonctions continues sur un segment $[a, b]$, à valeurs positives et vérifiant l'inégalité 
    $$\forall t \in [a, b], \quad y(t) \leqslant \varphi(t) + \int_{a}^{t} \psi(s) y(s)\ \d s.$$
    Alors
    $$ \forall t \in [a, b], \quad y(t) \leqslant \varphi(t) + \int_{a}^{t} \varphi(s) \psi(s) \exp \left( \int_{s}^{t} \psi(u)\ \d u \right)\ \d s.$$
\end{box_titre}

\section{Solutions de $y'' + y = h$}
DM 22: \\
Si $h \in \mathscr{C}(\Rp, \R)$, $f_0 : t \in \Rp \mapsto \int_{0}^{t} h(u) \sin(t-u) \d u$ est solution de $(F_h)$.

\section{Le wronskien}
\input{chapitres/equations_differentielles_et_calcul_differentiel/le_wronskien}

\section{Relèvement angulaire}
\input{chapitres/equations_differentielles_et_calcul_differentiel/relevement_angulaire}

\begin{itemize}
    \item Système différentiel en $z = x + \mi y$
    \item Système différentiel antisymétrique
    \item Equation d'ordre 1 avec raccordement
    \item Variation des constantes
    \item Equation différentielle linéaire d'ordre 2 avec raccordement
    \item Variables séparables
    \item Équation non linéaire (de \textsc{Bernoulli})
    \item Intégrale de \textsc{Dirichlet} via une équation différentielle
    \item Méthode des moindres carrés
    \item Fonctions harmoniques
\end{itemize}

% \appendix % From here onwards, chapters are numbered with letters, as is the appendix convention

\pagelayout{wide} % No margins
\addpart{Probabilités \& Variables aléatoires}
\pagelayout{margin} % Restore margins

\section{Identité de \textsc{Vandermonde}}
\begin{exercice}
    \marginnote[0cm]{\cite{exos_oraux} p.81}
    Soit $n \in \Ne$ et $x_1, \dots, x_n$ des complexes deux à deux distincts.
    \begin{enumerate}
        \item Montrer que l'application
        \begin{alignat*}{2}
            \varphi\ :\ \C_{n-1}[X]\ &\longrightarrow\ \C^n\\
            P\ &\longmapsto\ \big( P(x_1), \dots, P(x_n) \big)
        \end{alignat*}
        est un isomorphisme. Montrer que sa matrice dans les bases canoniques de départ et d'arrivée est 
        $$
        M_n(x_1, \dots, x_n) \defeq
        \begin{pmatrix}
            1 & x_1 & x_1^2 & \cdots & x_1^{n-1} \\
            1 & x_2 & x_2^2 & \cdots & x_2^{n-1} \\
            \vdots & \vdots & \vdots & & \vdots \\
            1 & x_n & x_n^2 & \cdots & x_n^{n-1}
        \end{pmatrix}.
        $$
        \item On note $\Lag_1(X), \dots, \Lag_n(X)$ les polynômes interpolteurs de \textsc{Lagrange} associés à $x_1, \dots, x_n$. Donner une relation entre les coefficients de $\Inv{M_n(x_1, \dots, x_n)}$ et ceux des polynômes $\Lag_i(X)$.
    \end{enumerate}
\end{exercice}

\section{Dénombrement des applications strictement croissantes}
Calcul du nombre d'applications strictement croissantes de $\llbracket 1, p \rrbracket$ dans $\llbracket 1, n \rrbracket$.

\begin{itemize}
    \item Réponse: $\displaystyle \binom{n}{p}$
\end{itemize}

\section{Dénombrement des applications croissantes}
Calcul du nombre d'applications croissantes de $\llbracket 1, p \rrbracket$ dans $\llbracket 1, n \rrbracket$.

\begin{itemize}
    \item Réponse: $\displaystyle \binom{n + p - 1}{p}$
    \item "Démonstration": représenter les éléments de l'ensemble de départ par des \emph{barres} qu'il faut placer entre les \emph{cases} de l'ensemble d'arrivée. 
\end{itemize}

\section{Dénombrement des surjections} \label{denombrement_surjections}
Calcul du nombre $S(p,n)$ de surjections de $\llbracket 1, p \rrbracket$ dans $\llbracket 1, n \rrbracket$. \\

\url{https://fr.wikipedia.org/wiki/Principe_d'inclusion-exclusion}

\begin{itemize}
    \item Étudier des cas particuliers
    \item Montrer que $n^p = \sum\limits_{k=0}^{n} \binom{n}{k} S(p,k)$
    \item En déduire que $S(p,n) = (-1)^n \sum\limits_{j=1}^{n} \binom{n}{j} j^p$.
\end{itemize}

\section{Loi d'un maximum/minimum}
\input{chapitres/probabilites_et_variables_aleatoires/loi_un_maximum_minimum}

\section{Lemmes de \textsc{Borel-Cantelli}}
\begin{tcolorbox}
    Si la somme des probabilités d'une suite $(A_n)_{n \in \N}$ d'événements d'un espace probabilisé $(\Omega, \mathscr{F}, \mathbb{P})$ est finie, alors la probabilité qu'une infinité d'entre eux se réalisent simultanément est nulle.
\end{tcolorbox}

\begin{itemize}
    \item \url{https://www.youtube.com/watch?v=Yw2qk42EZcM}
    \item \url{https://www.youtube.com/watch?v=2GqPQY-mBpk}
    \item \cite{intro_graph_alea} II) §5 page 34.
\end{itemize}

\section{Identité de \textsc{Wald}}
Soient $(X_n)_{n \in \Ne}$ une suite de variables aléatoires, \textbf{mutuellement indépendantes}, de même loi à valeurs dans $\N$, et $T$ une variable aléatoire à valeurs dans $\N$ indépendante des précédentes. La famille $(T, X_n)_{n \in \Ne}$ est une famille de variables aléatoires mutuellement indépendantes.\\
On note $G_X$ la fonction génératrice commune à toutes les $X_n$.\\
Pour $n \in \N$ et $\omega \in \Omega$, on pose $S_n(\omega) = \sum\limits_{k=1}^{n} X_k(\omega)$ et $S_0(\omega) = 0$, puis, $S(\omega) = S_{T(\omega)}(\omega)$.\\
Alors on a $\boxed{G_S = G_T \circ G_X}$ et $\boxed{\E[S] = \E[T] \E[X]}$.

\section{Chaîne de \textsc{Markov}} \label{chaîne_markov}
% \input{illustrations/i_chaine_de_markov}

\section{Inégalités de concentration (et transformées de \textsc{Laplace})}
\begin{itemize}
    \item Inégalité de \textsc{Markov}
    \item Inégalité de \textsc{Bienaymé-Tchebychev}
    \item Loi faible des grands nombres
\end{itemize}

\section{Calcul de \texorpdfstring{$\E \left [ \left (\sum\limits_{i=1}^{n} X_i \right)^4 \right]$}{espérance de la somme puissance 4 de v.a.}}
\begin{exercice}
    On suppose que $(X_i)_{i \in \Ne}$ est une famille de variables aléatoires iid. Calculer $\E \left [ \left (\sum\limits_{i=1}^{n} X_i \right)^4 \right]$.
\end{exercice}

\begin{elem_sol}
    Écrire la somme comme une somme sur quatre indices, utiliser la linéarité de l'espérance i.e. $\sum\limits_{1 \leqslant i, j, k, l \leqslant n} \E[X_i X_j X_k X_l]$ et distinguer les cas suivants:
    \begin{itemize}
        \item $|\{i, j, k, l \}| = 4$ (tous les indices sont deux à deux distints). Par l'indépendance des v.a., $\E[X_i X_j X_k X_l] = \E[X_i]^4$.
        \item $|\{i, j, k, l \}| = 3$ (deux des indices sont égaux): $\E[X_i X_j X_k X_l] = \E[X_i^2]\E[X_k]\E[X_l]$.
        \item $|\{i, j, k, l \}| = 2$ (trois des indices sont égaux): $\E[X_i X_j X_k X_l] = \E[X_i^3]\E[X_l]$.
        \item $|\{i, j, k, l \}| = 1$ (tous les indices sont égaux): $\E[X_i X_j X_k X_l] = \E[X_i^4]$.
    \end{itemize}
\end{elem_sol}

\section{Exercice d'oral}
\cite{acamanes}
\begin{exercice}
    Soit $(\Omega, \mathscr{A}, \P)$ un espace probabilisé. 
    \begin{enumerate}
        \item Soit $B$ un ensemble non vide et $(A_{\beta})_{\beta \in B}$ une famille d'éléments deux à deux disjoints de $\mathscr{A}$ telle que pour tout $\beta \in B, \P(A_\beta) > 0$. Montrer que $B$ est au plus dénombrable. 
        \item Soit $X$ une variable aléatoire indépendante d'elle même. Montrer que $X$ est constante. 
    \end{enumerate}
\end{exercice} 

\marginnote{
\begin{methode}
    Écrire l'ensemble sous la forme d'une union finie ou dénombrable d'ensembles dénombrables ou finis.
\end{methode}
}

\begin{solution}
\begin{enumerate}
    \item Soit $I$ un ensemble non vide. 
    $$I = \bigcup_{n \in \Ne} \underbrace{\left \{ \beta \in B, \P(A_\beta) \geqslant \frac{1}{n} \right \}}_{\defeq I_n}.$$
    Soient $n, p \in \Ne$ et $(\beta_1, \dots, \beta_p) \in (I_n)^p$ deux à deux distincts. Alors
    \begin{align*}
        1 \geqslant \P \left( \bigsqcup_{i=1}^p A_{\beta_i} \right) &= \sum_{i=1}^p \P(A_{\beta_i}) \geqslant \sum_{i=1}^p \frac{1}{n}
    \end{align*}
    donc $p \leqslant n$. \\
    Ainsi, $I_n$ est fini et $|I_n| \leqslant n$ i.e. $I$ est au plus dénombrable. 
    \item \textcolor{red}{À vérifier} Soit $\omega \in X(\Omega)$. Par indépendance de $X$ avec elle-même,
    $$\P(X = \omega) = \P(\{X=\omega\} \cap \{X=\omega\})= \P(X=\omega)^2.$$
    On en déduit que $\P(X=\omega) \in \{ 0, 1 \}$. \\
    De plus, $\sum\limits_{\omega \in X(\Omega)} \P(X=\omega) = 1$ et donc il existe un unique $\omega_0 \in X(\Omega)$ tel que $X=\omega_0$ presque sûrement i.e. $X$ est presque sûrement constante. 
\end{enumerate}
\end{solution}

\section{Discontinuités des fonctions monotones}
\begin{tcolorbox}
    Soit $f \in \mathscr{F}([a, b], \R)$ une fonction monotone. Alors l'ensemble des points de discontinuité de $f$ est au plus dénombrable. 
\end{tcolorbox}

\begin{exercice}
    \emph{Exercice 1 TD VI} \\
    Soient $a < b$ deux réels et $f \in \mathscr{F}([a, b], \R)$ une fonction croissante. Pour tout $x \in ]a, b[$, on pose $f(x^-) \defeq \lim\limits_{t \to x^-} f(t)$, $f(x^+) \defeq \lim\limits_{t \to x^+} f(t)$ et $v_f(x) \defeq f(x^+)-f(x^-)$.
    \begin{enumerate}
        \item Soit $x \in ]a, b[$. Montrer que $v_f(x) \geqslant 0$ avec égalité si et seulement si $f$ est continue en $x$.
        \item Soit $p \in \Ne$ et $x_1 < \cdots < x_p$ des réels de $]a, b[$. Montrer que $\sum\limits_{j=1}^p v_f(x_j) \leqslant f(b)-f(a)$.
        \item En déduire que pour tout $\alpha > 0$, l'ensemble des points $x \in ]a, b[$ tels que $v_f(x) > \alpha$ est fini. 
        \item Montrer que l'ensemble des points de discontinuité de $f$ est au plus dénombrable. 
    \end{enumerate}
\end{exercice}

\begin{solution}
    \begin{enumerate}
        \item Soit $(x, y, z) \in ]a,b[^3$ tel que $y \leqslant x \leqslant z$. Par croissance de $f$ on a $f(y) \leqslant f(x) \leqslant f(z)$. La monotonie de la fonction $f$ assure qu'elle admet des limites à gauche et à droite en tout point. Donc par passage à la limite dans l'encadrement, 
        $$f(x^-) \leqslant f(x) \leqslant f(x^+).$$
        On en déduit que $v_f(x) \geqslant 0$ avec égalité si et seulement si $f(x^+)=f(x)=f(x^-)$ i.e. si et seulement si $f$ est continue en $x$. 
        \item \begin{align*}
            \sum_{i=1}^p v_f(x_i) &= \sum_{i=1}^p \left( f(x_i^+) - f(x_i^-)\right) \\
            &= f(x_p^+)-f(x_p^-) + \sum_{i=1}^{p-1} \left( f(x_i^+) - f(x_i^-)\right) \\
            &\leqslant f(b) - f(x_p^-) + \sum_{i=1}^{p-1} \left( f(x_{i+1}^-) - f(x_i^-)\right) \\
            \text{ par télescopage } &\leqslant f(b) - f(x_1^-) \\
            &\leqslant f(b) - f(a)
        \end{align*}
        \item Soit $\alpha > 0$. Raisonnons par l'absurde en supposant que l'ensemble des points $x \in ]a, b[$ tels que $v_f(x) > \alpha$ est infini. \\
        Soit $n \in \N$, d'après la question 2, 
        $$\underbrace{f(b)-f(a)}_{\in \R} \geqslant \sum_{i=0}^p v_f(x_i) \geqslant p \alpha \xrightarrow[p \to \infty]{} + \infty \quad \text{ car } \alpha > 0.$$
        On aboutit donc à une contradiction et l'ensemble des points $x \in ]a, b[$ tels que $v_f(x) > \alpha$ est fini. 
        \item Soit $\mathscr{D}$ l'ensemble des points de discontinuité. \\ 
        On pose $\mathscr{D}_{\alpha} \defeq \left\{ x \in [a,b], v_f(x) > \alpha \right\}$.
        $$\mathscr{D} = \bigcup_{\alpha > 0} \mathscr{D}_\alpha = \bigcup_{n \in \Ne} \mathscr{D}_\frac{1}{n}.$$
        Nous avons écrit l'ensemble $\mathscr{D}$ comme une union dénombrable d'ensembles finis donc $\mathscr{D}$ est au plus dénombrable.
    \end{enumerate}
\end{solution}

Voir aussi l'exercice 4.10 (p. 297) de \cite{oraux_x_ens_3} dont l'énoncé est :
\begin{exercice}
    Soit $A$ une partie dénombrable de $\R$. Montrer l'existence d'une fonction monotone $f: \R \to \R$ dont $A$ est l'ensemble des points de discontinuités.
\end{exercice}   

\section{Nombres algébriques}
\begin{exercice}
\emph{Exercice 2 TD VI} \\
Un nombre $z$ est \emph{algébrique} s'il existe $n \in \Ne$ et $(a_0, \dots, a_n) \in \Q^{n+1}$ tels que $a_n \not=0$ et 
$$\sum_{k=0}^n a_k z^k = 0.$$
Montrer que l'ensemble des nombres algébriques est dénombrable. 
\end{exercice}

\marginnote{Définition à revoir sur le corps des coefficients}

\begin{marginfigure}
    \resizebox{6.5cm}{!}{
\begin{forest}
[$\frac{1}{1}$ 
    [$\frac{1}{2}$ 
        [$\frac{1}{3}$ 
            [$\frac{1}{4}$
                []
                []
            ] 
            [$\frac{4}{3}$
                []
                []
            ]
        ] 
        [$\frac{3}{2}$ 
            [$\frac{3}{5}$
                []
                []
            ] 
            [$\frac{5}{2}$
                []
                []
            ] 
        ]   
    ]
    [$\frac{2}{1}$ 
        [$\frac{2}{3}$ 
            [$\frac{2}{5}$
                []
                []
            ] 
            [$\frac{5}{3}$
                []
                []
            ]
        ]
        [$\frac{3}{1}$ 
            [$\frac{3}{4}$
                []
                []
            ]
            [$\frac{4}{1}$
                []
                []
            ]
        ]
    ]
]
\end{forest}
}
    L'arbre de \textsc{Calkin}-\textsc{Wilf} est un arbre dont les sommets sont en bijection avec les nombres rationnels positifs.
\end{marginfigure}

\begin{solution}
Les rationnels sont dénombrables \dots Donc pour $n$ fixé, $\Q_n[X]$ est dénombrable en tant que produits finis d'ensembles dénombrables.
Donc $\bigcup\limits_{n \in \N} \Q_n[X]$ est dénombrable comme réunion dénombrable d'ensembles dénombrables. \\
Pour tout polynôme dans $\Q_n[X]$, le nombre de ses racines est fini et de cardinal inférieur à $n$. 
Donc les nombres algébriques sont dénombrables car on peut établir une application surjective de leur ensemble sur une réunion dénombrable d'ensembles finis. 
\end{solution}


\section{Fonction indicatrice d'\textsc{Euler}}
\begin{defi}{Fonction indicatrice d'\textsc{Euler}}
    La \emph{fonction indicatrice d'\textsc{Euler}} est une fonction arithmétique de la théorie des nombres, qui à tout entier naturel $n$ non nul associe le nombre d'entiers compris entre 1 et $n$ et premiers avec $n$. \\
    Autrement dit, 
    \begin{alignat*}{2}
        \varphi\ :\ \Ne\ &\longrightarrow\ \Ne\\
        n\ &\longmapsto\ \mathrm{Card} \Big( \ens[\big]{m \in \Ne \tq m \leqslant n \text{ et } m \text{ premier avec } n } \Big).
    \end{alignat*}
\end{defi}

\marginnote{faire un graphe de $\varphi$}

\begin{prop}{}
    La fonction indicatrice d'\textsc{Euler} peut s'écrire
    $$\varphi\ :\ n \longmapsto n \cdot \prod_{p \in \mathscr{P}_n} \left(1 - \frac{1}{p} \right)$$
    en notant $\mathscr{P}_n$ l'ensemble des nombres premiers divisant $n$.
\end{prop}

\begin{preuve}
    (Wikipedia) \\
    La valeur de l'indicatrice d'\textsc{Euler} s'obtient à partir de la décomposition en facteurs premiers de $n$. On note $n = \smashoperator{\prod\limits_{p \in \mathscr{P}_n}} p^{k_i}$. Alors, $\varphi(n) = $
\end{preuve}

\begin{exercice}
    \marginnote[0cm]{Source : RMS 132 3 p.32.}
    Montrer que pour tout $n \in \Ne, \varphi(n) \geqslant \frac{\sqrt{n}}{2}$.
\end{exercice}

\begin{exercice}
    \marginnote[0cm]{Source : \cite{acamanes} (\href{https://acamanes.github.io/psi/psi_doc/exos_e06.pdf}{Exercice 17. Chap. VI)}}
    On note $\varphi$ la fonction indicatrice d'\textsc{Euler}. Montrer que 
    $$\forall n \in \Ne \quad n = \sum_{d|n} \varphi(d).$$
\end{exercice}

Les points suivants ne peuvent être compris qu'avec la correction.
\begin{itemize}
    \item Q2): Résultat à retenir (bien que rappeler dans le DS5):\\
    $$\forall p \in J \subset \P,\ p \text{ divise } a \iff \prod_{p \in J} p \text{ divise } a$$
    \item Q3): un élément de $\Omega$ est premier avec $n$ si et seulement si il n'est divisible par aucun des diviseurs premiers de $n$. D'où
    $$\mathbb{P} \Big(\ens[\big]{ m \in \Omega \tq m \wedge n = 1} \Big) = \prod_{p \in \mathscr{P}_n} \left(1 - \frac{1}{p} \right).$$
    \item Q4): Calculer le cardinal de $B_j = \ens[\big]{ j\cdot d,\ j \in \llbracket 1, k \rrbracket \tq j \wedge k = 1}$.
    \item Q5): Montrer que $(B_d)_{d|n}$ forme un SCE de $\Omega$ et en déduire la formule de l'énoncé. $\displaystyle (\Omega = \bigcup_{d|n} B_d)$ \textcolor{green}{à compléter}
\end{itemize}

\section{\emph{Exercice 4. Chap. VII:}}
\begin{exercice}
    Lors d'une élection, 700 électeurs votent pour $A$ et 300 pour $B$. Quelle est la probabilité que, pendant le dépouillement, $A$ soit toujours strictement en tête?
\end{exercice}


\section{Distance en variation totale}
En mathématiques et plus particulièrement en théorie des probabilités et en statistique, la distance en variation totale (ou distance de variation totale ou encore distance de la variation totale) désigne une distance statistique définie sur l'ensemble des mesures de probabilité d'un espace probabilisable. 

\begin{exercice}
\marginnote[0cm]{\emph{Exercice 8. Chap. VII}}
Soit $\mathscr{E}$ l'espace des suites réelles $(p_n)_{n \in \N}$ telles que la série $\sum |p_n|$ converge, muni de la norme $\Norme{p} = \sum\limits_{n=0}^{+\infty} |p_n|$. Soit $\mathscr{P}$ le sous-ensemble de $\mathscr{E}$ formé des suites réelles positives $(p_n)_{n \in \N}$ telles que $\Norme{p}=1$.
\begin{enumerate}
    \item Montrer que $\mathscr{P}$ est borné et convexe. \\
    \item Pour $P, Q \in \mathscr{P}$, on pose $d(P, Q) \defeq \sup\limits_{A \subset \N} \left| \sum\limits_{n \in A} p_n - \sum\limits_{n \in A} q_n \right|$. Montrer que $d(P,Q) \in [0,1]$.
    \item Soit $(p,q) \in [0, 1]^2, P = (1-p, p, 0, \dots)$ et $Q = (1-q, q, 0, \dots)$. Déterminer $d(P, Q)$.
    \item Soient $n \in \N$ et $\lambda \in \Rp$. Montrer l'inégalité $\sum\limits_{k=n+1}^{+\infty} \leqslant \me^{\lambda} \frac{\lambda^{n+1}}{(n+1)!}$.
    \item Soient $X_\lambda$ et $X_\mu$ deux variables aléatoires suivant une loi de \textsc{Poisson} de paramètres respectifs $\lambda$ et $\mu$. Soit $P_\lambda = (\P(X_\lambda=n))_{n \in \N}$ et $P_\mu = (\P(X_\mu=n))_{n \in \N}$. Soit $n \in \Ne$. Montrer l'inégalité
    $$d(P_\lambda, P_\mu) \leqslant \max_{A \subset \llbracket 0, n \rrbracket} \left| \sum_{k \in A} \P(X_\lambda=k) - \sum_{k \in A} \P(X_\mu=k) \right|$$
    $$+ \frac{\lambda^{n+1}}{(n+1)!} + \frac{\mu^{n+1}}{(n+1)!}.$$
\end{enumerate}
\end{exercice}

\marginnote[-5cm]{
    \begin{kaobox}[frametitle=Ensemble convexe]
        Un ensemble $\mathscr{X}$ est dit \emph{convexe} lorsque
        $$\forall (x,y) \in \mathscr{X}^2 \enspace \forall t \in [0,1]$$
        $$tx + (1-t)y \in \mathscr{X}.$$
    \end{kaobox}
}

\begin{elem_sol}
    \begin{enumerate}
        \item Les éléments de $\mathscr{P}$ sont des suites positives...
        \item Bien justifier l'existence de la borne supérieure.
        \item Considérer une partie $B$ de $\N$ et distinguer quatre cas ($\{0\} \subset B$ et $\{1\} \not\subset B$; $\{0\} \not\subset B$ et $\{1\} \subset B$; $\{0, 1\} \subset B$; $ \{0, 1\} \not\subset B$). Le résultat est $d(P, Q) = |p-q|$.
        \item Passer par la formule de \textsc{Taylor} avec reste intégral.
        \item Soit $B$ une partie de $\N$. Écrire $B = \left(B \cap \llbracket0, n \rrbracket \right) \cup \{k \in B; k > n\}$. Sommer sur ces deux ensembles, séparer la valeur absolue et majorer à l'aide de la question précédente. (\textcolor{green}{à détailler})
    \end{enumerate}
\end{elem_sol}


\section{Majoration de la variance d'une v.a.d.r.}
\begin{exercice}    
    \emph{Exercice 5. Chap. VII}\\
    Soit $X$ une variable aléatoire discrète réelle à valeurs dans $[a, b]$. Montrer que $\V(X) \leqslant \frac{(b-a)^2}{4}$ et discuter le cas d'égalité.
\end{exercice}

\begin{solution}
    Deux méthodes sont introduites.\\
    \underline{Méthode 1:}
    \begin{itemize}
        \item Poser la fonction $f:t \mapsto \E \left[(X-t)^2 \right]$.
        \item $f$ atteint son minimum en $\E[X]$ et vaut $\V(X)$. 
        \item Comparer $\V(X)$ à $f \left( \frac{a + b}{2} \right)$.
        \item \url{https://stats.stackexchange.com/questions/45588/variance-of-a-bounded-random-variable}
    \end{itemize}
    
    \underline{Méthode 2:} (vue en cours)
    \begin{itemize}
        \item Justifier l'existence de l'espérance et de la variance\\
        $\sum \gamma^n \P(X=x_k)$ converge, avec $\gamma = \max \{ |a|, |b| \}$ \textcolor{red}{\emph{(ne pas oublier les valeurs absolues  car on ne connaît pas les signes de $a$ et de $b$)}}.
        \item On remarque que $\V(X) = \V \left( X - \frac{a+b}{2} \right)$
        \item Or comme $a \leqslant X \leqslant b$, $$\displaystyle \left| X - \frac{a+b}{2}\right| \leqslant \frac{b-a}{2} \text{ et } \displaystyle \left( X - \frac{a+b}{2} \right)^2 \leqslant \left(\frac{b-a}{2}\right)^2$$
        \item Donc par le théorème de \textsc{König}-\textsc{Huygens}, 
        \begin{align*}
            \V \left(X - \frac{a+b}{2} \right) &= \E \Bigg[ \underbrace{\left(X - \frac{a+b}{2} \right)^2}_{\leqslant \left( \frac{b-a}{2} \right)^2} \Bigg] - \underbrace{\E \left[X-\frac{a+b}{2} \right]^2}_{\geqslant 0} \\
            \V(X) &\leqslant \frac{(b-a)^2}{4} \text{ par croissance de l'espérance}
        \end{align*}
    \end{itemize}
    
\marginnote[-1cm]{
    \begin{kaobox}[frametitle=Théorème de \textsc{König}-\textsc{Huygens}]
        Soit $X$ une variable aléatoire admettant un moment d'ordre 2. Alors,
        $$\V(X) = \E \left[ X^2 \right] - \E[X]^2.$$
    \end{kaobox}
}
    
    \textbf{Cas d'égalité}\\
    D'après la majoration du premier terme et la minoration du deuxième terme de la relation ci-dessus, il y a égalité  si et seulement si
    
    \begin{align*}
        \V(X) = \frac{(b-a)^2}{4} &\Longleftrightarrow
        \begin{cases}
            \E \left[ \left(X - \frac{a+b}{2} \right)^2 \right] = \frac{(b-a)^2}{4}\\
            \text{et}\\
            \E \left[X-\frac{a+b}{2} \right]^2 = 0
        \end{cases}\\
        &\Longleftrightarrow
        \begin{cases}
            \P \left( X - \left(\frac{a+b}{2} \right)^2  \left(\frac{b-a}{2} \right)^2 \right)= 1\\
            \text{et}\\
            \E [X] = \frac{a+b}{2}
        \end{cases}\\
        &\Longleftrightarrow
        \begin{cases}
            \P \big( \{ X=b \} \cup \{X=a\} \big)= 1\\
            \text{et}\\
            \E [X] = \frac{a+b}{2}
        \end{cases}\\
        \V(X) = \frac{(b-a)^2}{4} &\Longleftrightarrow \P(X=a)=\P(X=b)=\frac{1}{2}
    \end{align*}
\end{solution}




%----------------------------------------------------------------------------------------

\backmatter % Denotes the end of the main document content
\setchapterstyle{plain} % Output plain chapters from this point onwards

\thispagestyle{empty}
\newgeometry{left=2cm,right=2cm, top=2cm}
\chapter{Mathématiciens}
\begin{multicols}{2}
\footnotesize
\begin{itemize}
    \input{mathematiciens.tex}
\end{itemize}
\end{multicols}

%----------------------------------------------------------------------------------------
%	BIBLIOGRAPHY
%----------------------------------------------------------------------------------------

% The bibliography needs to be compiled with biber using your LaTeX editor, or on the command line with 'biber main' from the template directory

% \defbibnote{bibnote}{Here are the references in citation order.\par\bigskip} % Prepend this text to the bibliography
\printbibliography[heading=bibintoc, title=Références] % Add the bibliography heading to the ToC, set the title of the bibliography and output the bibliography note

\printindex % Output the index

%----------------------------------------------------------------------------------------
%	BACK COVER
%----------------------------------------------------------------------------------------

% If you have a PDF/image file that you want to use as a back cover, uncomment the following lines

%\clearpage
%\thispagestyle{empty}
%\null%
%\clearpage
%\includepdf{cover-back.pdf}

%----------------------------------------------------------------------------------------

\end{document}

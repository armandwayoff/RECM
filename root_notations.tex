\DeclareMathOperator{\ch}{ch}

\newcommand{\e}{\mathrm{e}}
\renewcommand{\i}{\mathrm{i}}
\renewcommand{\j}{\mathrm{j}} % racine troisième de l'unité

\newcommand{\R}{\mathbb{R}}
\newcommand{\Rp}{\R_+}
\newcommand{\Rm}{\R_-}
\renewcommand{\Re}{\R^\star}
\newcommand{\Rpe}{\Re_+}
%%% Ajout du re par Alain %%%
\newcommand{\C}{\mathbb{C}}
\newcommand{\Ce}{\C^\star}
\newcommand{\K}{\mathbb{K}}
\newcommand{\Ke}{\K^\star}
\newcommand{\N}{\mathbb{N}}
\newcommand{\Ne}{\N^\star}
\newcommand{\Z}{\mathbb{Z}}
\newcommand{\Q}{\mathbb{Q}}
\newcommand{\A}{\mathbb{A}} % Nombres algébriques
\newcommand{\Premier}{\mathbb{P}} % Nombres premiers

\newcommand{\cotan}{\mathrm{cotan}}

\renewcommand{\d}{\, \mathrm{d}}

\newcommand{\norme}[1]{\Vert #1 \Vert}
%%% Ajout par Alain car definition manquante %%%
\newcommand{\Ninf}[1]{\Vert #1 \Vert_\infty}

% Complexes
% \DeclareMathOperator{\Reel}{\mathfrak{Re}}
% \DeclareMathOperator{\Imaginaire}{\mathfrak{Im}}
%%% Modifie par Alain car pb de compilation %%%
% \DeclareMathOperator{\Reel}{\mathscr{Re}}
\DeclareMathOperator{\Reel}{\mathrm{Re}}
\DeclareMathOperator{\Imaginaire}{\mathscr{Im}}


\newcommand{\ptnclegras}[1]{\textbf{#1}}
\newcommand{\ptnclecadre}[1]{\boxed{#1}}

% Algèbre
% \newcommand{\Gl}{\mathscr{G}\kern-0.16em\ell}
\newcommand{\Gl}{\mathrm{GL}}
\newcommand{\I}{\mathrm{I}}
\newcommand{\Id}{\mathrm{Id}}
\newcommand{\M}{\mathscr{M}}
\newcommand{\Endo}{\mathscr{L}}
\newcommand{\Ortho}{\mathrm{O}}
\newcommand{\Sym}{\mathscr{S}}

% https://tex.stackexchange.com/questions/48/how-can-i-specify-a-long-list-of-math-operators
\newcommand{\DeclareMyOperator}[1]{%
  \expandafter\DeclareMathOperator\csname #1\endcsname{#1}
}
\newcommand{\DeclareMathOperators}{\forcsvlist{\DeclareMyOperator}}

%%% Modifie par Alain : suppression de Tr, det dans la liste car conflit %%%
\DeclareMathOperators{Rg,Ker,Vect,Sp,com,Diag,Mat} % conflit entre physics et Tr

% Probas
\newcommand{\E}{\mathbf{E}}
\newcommand{\V}{\mathbf{V}}
\renewcommand{\P}{\mathbf{P}}
\newcommand{\indicatrice}[1]{\mathbf{1}_{#1}}


%%%%%%%%%%%%%%%%%%%%%%%%%%%%%%%%%%%%%%%%%%%%%%%%%%%%%%%%%%%%%%%%%%%%%%%%%%%%%%%%%%%%%%%%%%%%%%%%%%%
%https://tex.stackexchange.com/questions/4216/how-to-typeset-correctly
% :=
%\newcommand*{\defeq}{\mathrel{\vcenter{\baselineskip0.5ex \lineskiplimit0pt
%                     \hbox{\scriptsize.}\hbox{\scriptsize.}}}                 =}
% =:
%\newcommand*{\defeqright}{=\mathrel{\vcenter{\baselineskip0.5ex \lineskiplimit0pt
%                     \hbox{\scriptsize.}\hbox{\scriptsize.}}}%
%                      }
\newcommand{\defeq}{\overset{\mathrm{\tiny def}}{=}}
\newcommand{\defeqright}{\defeq}
%%%%%%%%%%%%%%%%%%%%%%%%%%%%%%%%%%%%%%%%%%%%%%%%%%%%%%%%%%%%%%%%%%%%%%%%%%%%%%%%%%%%%%%%%%%%%%%%%%%

% https://tex.stackexchange.com/questions/349747/equivalence-of-sequence-sim-with-some-text-under      
\newcommand{\isEquivTo}[1]{%
  \mathpalette\isEquivToInner{#1}%
}
\newcommand{\isEquivToInner}[2]{%
  \ifx#1\displaystyle
    \underset{#2}{\sim}
  \else
    \sim_{#2}
  \fi
}


%%%%%%%%%%%%%%%%%%%%%%%%%%%%%%%%%%%%%%%%%%%%%%%%%%%%%%%%%%%%%%%%%%%%%%%%
\newcommand{\interoo}[2]{\left]#1\,;#2\right[}
\newcommand{\interff}[2]{\left[#1\,;#2\right]}
\newcommand{\interof}[2]{\left]#1\,;#2\right]}
\newcommand{\interfo}[2]{\left[#1\,;#2\right[}
\newcommand{\interent}[2]{\llbracket #1\,; #2 \rrbracket}

%\newcommand{\intervalle}[4]{%
%\mathopen{#1}#2\,;#3\mathclose{#4}}
%\newcommand{\intervalleff}[2]{%
%\intervalle{[}{#1}{#2}{]}}
%\newcommand{\intervallefo}[2]{%
%\intervalle{[}{#1}{#2}{[}}
%\newcommand{\intervalleof}[2]{%
%\intervalle{]}{#1}{#2}{]}}
%\newcommand{\intervalleoo}[2]{%
%\intervalle{]}{#1}{#2}{[}}
%%%%%%%%%%%%%%%%%%%%%%%%%%%%%%%%%%%%%%%%%%%%%%%%%%%%%%%%%%%%%%%%%%%%%%%%


\newcommand{\Trsp}[1]{#1^\top}

\newcommand{\Inv}[1]{#1^{-1}}

\newcommand{\fact}[1]{#1\,!}

\newcommand{\chevron}[1]{\say{\,#1\,}}

\newcommand{\note}{\hbox{\scriptsize $\blacklozenge$\ }}

% A REVOIR %%%%%%%%%%%%%%%%%%%%%%%%%%%%%%%%%%%%%%%%%
\newcommand{\suite}[3]{(#1_#2)_{#3}}

% \newcommand{\somme}[2][x]{%#1_1+\cdots+#1_#2}

\newcommand{\convolution}[2]{#1 \ast #2}

\newcommand{\conjugue}[1]{\overline{#1}}

\newcommand{\module}[1]{\left| #1 \right|}

\DeclarePairedDelimiterX\ens[1]\lbrace\rbrace{\def\tq{\;\delimsize\vert\;}#1} % https://tex.stackexchange.com/questions/253077/how-do-you-create-a-set-in-latex

\newcommand{\fonctionligne}[3][f]{#1 \colon #2 \mapsto #3}
\newcommand{\fonctionens}[3][f]{#1 \colon #2 \rightarrow #3}

\newcommand{\fonction}[5][f]{
    #1 \colon \begin{array}{>{\displaystyle}r @{} >{{}}c<{{}} @{} >{\displaystyle}l} 
          #2 &\longrightarrow& #3 \\[\medskipamount]
          #4 &\longmapsto& #5 
         \end{array}
} 

%%%%%%%%%%%%%%%%%%%%%%%%%%%%%%%
%%%%%%%%%%%%%%%%%%%%%%%%%%%
%%%%%%%%%%%%%%%%%%%%%%%%
% \langle \rangle

\newcommand*{\limite}[4][]{\displaystyle\lim_{\substack{#2\to#3\\#1}}#4}

\newcommand{\faibletoile}[1]{
    \xrightharpoonup[#1]{}^\star
}

% Suites remarquables
\newcommand{\Leg}{\mathrm{L}}
\newcommand{\Lag}{\mathrm{L}}
\newcommand{\Hilb}{\mathrm{H}}
\newcommand{\Hermite}{\mathrm{H}}
\newcommand{\Bern}{\mathrm{B}}
\newcommand{\Tcheby}{\mathrm{T}}
\newcommand{\Wallis}{\mathrm{W}}
\newcommand{\Cauchy}{\mathrm{C}}
\newcommand{\Gram}{\mathrm{G}}
\newcommand{\Bernstein}{\mathrm{B}}
\newcommand{\Vandermonde}{\mathrm{V}}
\newcommand{\Harmonique}{\mathrm{H}}


\newcommand{\scnums}[1]{\ifmmode\mathsmaller{\newstylenums{#1}}\else\textsmaller{\newstylenums{#1}}\fi} % https://comp.text.tex.narkive.com/idxlcygk/numerals-in-small-caps-font

\newcommand{\definir}[1]{\emph{#1}}
\usepackage[T1]{fontenc} % Encodage de la police pour les caractères internationaux

\usepackage{anyfontsize} % Pour le titre de la page de couverture

% Pour faire des pages blanches
\usepackage{afterpage}

\usepackage{comment}

% Load the bibliography package
% \usepackage[natbib,refsection=chapter]{biblatex}
\usepackage{kaobiblio}

% Load mathematical packages for theorems and related environments
\usepackage[framed=true]{kaotheorems}

%%%% Ajoute par Alain pour pouvoir compiler le kaorefs %%%%
\usepackage[french]{babel}

% \usepackage[bitstream-charter]{mathdesign}
% \graphicspath{{examples/documentation/images/}{images/}} % Paths in which to look for images

% Load the package for hyperreferences
\usepackage{kaorefs}

\usepackage[splitindex]{imakeidx}

\usepackage[dvipsnames]{xcolor}
\usepackage{ragged2e}

\usepackage{subcaption}

\usepackage{pdfpages}

\usepackage{fontawesome} % pour les icônes

% Pour l'index des notations
\usepackage{longtable}

%%%%%%%%%%%%%%%%%%%%%%%%
% Pour le tableau des transformées de Laplace
\usepackage{adjustbox}
\usepackage{tabularray}

% Commande pour citer un théorème
\usepackage{ifthen}

% Pour créer des annexes par chapitre
% https://tex.stackexchange.com/questions/120716/appendix-after-each-chapter
\usepackage{appendix}
\usepackage{chngcntr}

% Start of subappendices environment
\AtBeginEnvironment{subappendices}{%
\newpage
% \addcontentsline{toc}{section}{Annexes}
\counterwithin{figure}{section}
\counterwithin{table}{section}
}

% End of subappendices environment
\AtEndEnvironment{subappendices}{%
\counterwithout{figure}{section}
\counterwithout{table}{section}
}
%%%%%%%%%%%%%%%%%%

\usepackage{tikz}
\tikzset{>=latex} % for LaTeX arrow head
\usepackage{pgfplots} % for the axis environment
\usepackage[outline]{contour} % halo around text
\contourlength{2pt}

\usetikzlibrary{positioning,
                calc, 
                backgrounds, % required for 'inner frame sep'
                automata, 
                decorations.pathreplacing, % pour les curly braces de la fig du déterminant
                decorations.markings,
                calligraphy, % pour les curly braces de la fig du déterminant
                arrows, % customizing arrows
                bending, % figure racines troisième de l'unité
                matrix, % critère de nilpotence par la trace
                patterns
                }

% to fill an area under function
\usepgfplotslibrary{fillbetween,
                    groupplots
                    }
\pgfplotsset{compat=newest} % TikZ coordinates <-> axes coordinates
% https://tex.stackexchange.com/questions/240642/add-vertical-line-of-equation-x-2-and-shade-a-region-in-graph-by-pgfplots

% plot aspect ratio
%\def\axisdefaultwidth{8cm}
%\def\axisdefaultheight{6cm}


%%% Ajoute par Alain
\usepackage{color}
\usepackage{todonotes}

% Pour la figure homothétie
\usepackage{tkz-euclide}

\usepackage{hyperref}
\usepackage{multicol}

% Pour le schéma de la preuve de la densité des matrices diagonalisables
\tikzset{
    arrowMe/.style={
        postaction=decorate,
        decoration={
            markings,
            mark=at position .45 with {\arrow[thick]{#1}}
        }
    }
}
                
% Pour la figure de projection orthogonale
%\usepackage[active,tightpage]{preview}

% Pour les longues équations (théorème d'intégration par parties généralisées) https://tex.stackexchange.com/questions/3782/how-can-i-split-an-equation-over-two-or-more-lines
%\usepackage{breqn}

% Pour l'arbre de Calkin-Wilf
\usepackage{forest}

\tikzset{node distance=3.5cm, % Minimum distance between two nodes. Change if necessary.
    every state/.style={ % Sets the properties for each state
    semithick,
    fill=red!20},
    initial text={},     % No label on start arrow
    double distance=1pt, % Adjust appearance of accept states
    every edge/.style={  % Sets the properties for each transition
    draw,
    ->,>=stealth',     % Makes edges directed with bold arrowheads
    auto,
    semithick}}
          
\tikzset{
    block/.style={
    draw, 
    rectangle, 
    minimum height=1.5cm, 
    minimum width=3cm, align=center
    }, 
    line/.style={->,>=latex'}
}
           
\usepackage{tikz-3dplot}

% Pour le diagramme quiver
\usepackage{tikz-cd}
\usepackage{quiver}

% Scale pour le diagramme sur le red des endos
% https://tex.stackexchange.com/questions/325297/how-to-scale-a-tikzcd-diagram
\tikzcdset{scale cd/.style={every label/.append style={scale=#1},
    cells={nodes={scale=#1}}}}

%%%% Commente par Alain car conflit, deja charge par koa ? %%%%
% \usepackage{fancyhdr}
\usepackage{amsfonts} 
\usepackage{amsmath}
\usepackage{amsthm}
\usepackage{amssymb}

% \usepackage{mathtools}
\usepackage{bm}
\usepackage{stmaryrd}
%\usepackage{mathrsfs}
\usepackage{euscript}
\usepackage{enumitem}
\usepackage{tasks}
\usepackage{xurl}
\usepackage{dsfont}  % Pour la fonction indicatrice \mathds{1}

\usepackage{cfr-lm} % https://tex.stackexchange.com/questions/301699/oldstyle-numbers-in-body-text-computer-modern

% Figures
\usepackage{physics}
\usepackage{mathdots}
\usepackage{cancel}
\usepackage{siunitx}
\usepackage{array}
\usepackage{multirow}
\usepackage{gensymb}
\usepackage{tabularx}
\usepackage{extarrows}

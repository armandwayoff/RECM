\usepackage[top=15mm, bottom=20mm, left=15mm, right=25mm]{geometry}

\usepackage[french]{babel}
\usepackage[T1]{fontenc} % Encodage de la police pour les caractères internationaux

\usepackage{amsmath} % pour \overset notamment
\usepackage{amsthm} % pour les environnements démonstration, solution, ...
\usepackage{amsfonts} % mathbb
\usepackage{amssymb} % exemple : \blacksquare

\usepackage{mathrsfs} % mathscr

\usepackage{stmaryrd} % ll/rrbracket

\usepackage{hyperref} % pour des liens cliquables

% \usepackage{cfr-lm} % https://tex.stackexchange.com/questions/301699/oldstyle-numbers-in-body-text-computer-modern
\usepackage{hfoldsty}

\usepackage{etoolbox} % pour créer des listes de macros (https://tex.stackexchange.com/questions/48/how-can-i-specify-a-long-list-of-math-operators)

\usepackage{mathtools} % pour \ens notamment

\usepackage{array} % pour la commande fonction[f]

\usepackage{xfrac} % pour \sfrac{}{}

\usepackage{physics} % pour les \dv{}{} ...

\usepackage[scr=rsfs,cal=cm,frak=euler,bb=ams]{mathalfa} % https://tex.stackexchange.com/questions/418352/my-pdf-reader-does-not-show-calligraphic-letter-produced-by-latex-correctly
% https://mirror.ibcp.fr/pub/CTAN/macros/latex/contrib/mathalpha/doc/mathalpha-doc.pdf

\usepackage[dvipsnames]{xcolor}

\usepackage{sectsty} % style sections, sous-sections, ...

\usepackage{enumitem} % pour les labels des listes

\usepackage{cancel} % pour les expressions barrées

\usepackage{relsize} % pour les chiffres capitaux

\usepackage{xurl} % voir conflit avec hyperref

\usepackage{fancyhdr} % pour le formatage des entêtes et bas de pages


% guillements
\usepackage[
    left = \flqq{},% 
    right = \frqq{},% 
    leftsub = \flq{},% 
    rightsub = \frq{} %
]{dirtytalk}
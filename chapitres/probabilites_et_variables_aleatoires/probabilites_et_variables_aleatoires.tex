\section{Identité de \textsc{Vandermonde}}
\begin{exercice}
    \marginnote[0cm]{\cite{exos_oraux} p.81}
    Soit $n \in \Ne$ et $x_1, \dots, x_n$ des complexes deux à deux distincts.
    \begin{enumerate}
        \item Montrer que l'application
        \begin{alignat*}{2}
            \varphi\ :\ \C_{n-1}[X]\ &\longrightarrow\ \C^n\\
            P\ &\longmapsto\ \left( P(x_1), \dots, P(x_n) \right)
        \end{alignat*}
        est un isomorphisme. Montrer que sa matrice dans les bases canoniques de départ et d'arrivée est 
        $$
        M_n(x_1, \dots, x_n) \defeq
        \begin{pmatrix}
            1 & x_1 & x_1^2 & \cdots & x_1^{n-1} \\
            1 & x_2 & x_2^2 & \cdots & x_2^{n-1} \\
            \vdots & \vdots & \vdots & & \vdots \\
            1 & x_n & x_n^2 & \cdots & x_n^{n-1}
        \end{pmatrix}.
        $$
        \item On note $\Lag_1(X), \dots, \Lag_n(X)$ les polynômes interpolteurs de \textsc{Lagrange} associés à $x_1, \dots, x_n$. Donner une relation entre les coefficients de $\Inv{M_n(x_1, \dots, x_n)}$ et ceux des polynômes $\Lag_i(X)$.
    \end{enumerate}
\end{exercice}

\section{Dénombrement des applications strictement croissantes}
Calcul du nombre d'applications strictement croissantes de $\llbracket 1, p \rrbracket$ dans $\llbracket 1, n \rrbracket$.

\begin{itemize}
    \item Réponse: $\displaystyle \binom{n}{p}$
\end{itemize}

\section{Dénombrement des applications croissantes}
Calcul du nombre d'applications croissantes de $\llbracket 1, p \rrbracket$ dans $\llbracket 1, n \rrbracket$.

\begin{itemize}
    \item Réponse: $\displaystyle \binom{n + p - 1}{p}$
    \item "Démonstration": représenter les éléments de l'ensemble de départ par des \emph{barres} qu'il faut placer entre les \emph{cases} de l'ensemble d'arrivée. 
\end{itemize}

\section{Dénombrement des surjections} \label{denombrement_surjections}
Calcul du nombre $S(p,n)$ de surjections de $\llbracket 1, p \rrbracket$ dans $\llbracket 1, n \rrbracket$. \\

\url{https://fr.wikipedia.org/wiki/Principe_d'inclusion-exclusion}

\begin{itemize}
    \item Étudier des cas particuliers
    \item Montrer que $n^p = \sum\limits_{k=0}^{n} \binom{n}{k} S(p,k)$
    \item En déduire que $S(p,n) = (-1)^n \sum\limits_{j=1}^{n} \binom{n}{j} j^p$.
\end{itemize}

\section{Loi d'un maximum/minimum}
\input{chapitres/probabilites_et_variables_aleatoires/loi_un_maximum_minimum}

\section{Lemmes de \textsc{Borel-Cantelli}}
\begin{tcolorbox}
    Si la somme des probabilités d'une suite $(A_n)_{n \in \N}$ d'événements d'un espace probabilisé $(\Omega, \mathscr{F}, \mathbb{P})$ est finie, alors la probabilité qu'une infinité d'entre eux se réalisent simultanément est nulle.
\end{tcolorbox}

\begin{itemize}
    \item \url{https://www.youtube.com/watch?v=Yw2qk42EZcM}
    \item \url{https://www.youtube.com/watch?v=2GqPQY-mBpk}
    \item \cite{intro_graph_alea} II) §5 page 34.
\end{itemize}

\section{Identité de \textsc{Wald}}
Soient $(X_n)_{n \in \Ne}$ une suite de variables aléatoires, \textbf{mutuellement indépendantes}, de même loi à valeurs dans $\N$, et $T$ une variable aléatoire à valeurs dans $\N$ indépendante des précédentes. La famille $(T, X_n)_{n \in \Ne}$ est une famille de variables aléatoires mutuellement indépendantes.\\
On note $G_X$ la fonction génératrice commune à toutes les $X_n$.\\
Pour $n \in \N$ et $\omega \in \Omega$, on pose $S_n(\omega) = \sum\limits_{k=1}^{n} X_k(\omega)$ et $S_0(\omega) = 0$, puis, $S(\omega) = S_{T(\omega)}(\omega)$.\\
Alors on a $\boxed{G_S = G_T \circ G_X}$ et $\boxed{\E[S] = \E[T] \E[X]}$.

\section{Chaîne de \textsc{Markov}} \label{chaîne_markov}
% \begin{tikzpicture}
    \node[state] (s1) {État 1};
    \node[state, below right of=s1] (s2) {État 2};
    \node[state, below left of=s1] (s3) {État 3};
    
    \draw (s1) edge[loop above] node {$p_{1,1}$} (s1);
    \draw (s1) edge[bend left] node {$p_{1,2}$} (s2);
    \draw (s1) edge[bend right, above left] node {$p_{1,3}$} (s3);
    
    \draw (s2) edge[bend left, above right] node {$p_{2,1}$} (s1);
    \draw (s2) edge[loop right] node {$p_{2,2}$} (s2);
    \draw (s2) edge[bend right] node {$p_{2,3}$} (s3);
    
    \draw (s3) edge[bend right] node {$p_{3,1}$} (s1);
    \draw (s3) edge[bend right] node {$p_{3,2}$} (s2);
    \draw (s3) edge[loop left] node {$p_{3,3}$} (s3);
\end{tikzpicture}


\section{Inégalités de concentration (et transformées de \textsc{Laplace})}
\begin{itemize}
    \item Inégalité de \textsc{Markov}
    \item Inégalité de \textsc{Bienaymé-Tchebychev}
    \item Loi faible des grands nombres
\end{itemize}

\section{Calcul de \texorpdfstring{$\E \left [ \left (\sum\limits_{i=1}^{n} X_i \right)^4 \right]$}{espérance de la somme puissance 4 de v.a.}}
\begin{exercice}
    On suppose que $(X_i)_{i \in \Ne}$ est une famille de variables aléatoires iid. Calculer $\E \left [ \left (\sum\limits_{i=1}^{n} X_i \right)^4 \right]$.
\end{exercice}

\begin{elem_sol}
    Écrire la somme comme une somme sur quatre indices, utiliser la linéarité de l'espérance i.e. $\sum\limits_{1 \leqslant i, j, k, l \leqslant n} \E[X_i X_j X_k X_l]$ et distinguer les cas suivants:
    \begin{itemize}
        \item $|\{i, j, k, l \}| = 4$ (tous les indices sont deux à deux distints). Par l'indépendance des v.a., $\E[X_i X_j X_k X_l] = \E[X_i]^4$.
        \item $|\{i, j, k, l \}| = 3$ (deux des indices sont égaux): $\E[X_i X_j X_k X_l] = \E[X_i^2]\E[X_k]\E[X_l]$.
        \item $|\{i, j, k, l \}| = 2$ (trois des indices sont égaux): $\E[X_i X_j X_k X_l] = \E[X_i^3]\E[X_l]$.
        \item $|\{i, j, k, l \}| = 1$ (tous les indices sont égaux): $\E[X_i X_j X_k X_l] = \E[X_i^4]$.
    \end{itemize}
\end{elem_sol}

\section{Exercice d'oral}
\cite{acamanes}
\begin{exercice}
    Soit $(\Omega, \mathscr{A}, \P)$ un espace probabilisé. 
    \begin{enumerate}
        \item Soit $B$ un ensemble non vide et $(A_{\beta})_{\beta \in B}$ une famille d'éléments deux à deux disjoints de $\mathscr{A}$ telle que pour tout $\beta \in B, \P(A_\beta) > 0$. Montrer que $B$ est au plus dénombrable. 
        \item Soit $X$ une variable aléatoire indépendante d'elle même. Montrer que $X$ est constante. 
    \end{enumerate}
\end{exercice} 

\marginnote{
\begin{methode}
    Écrire l'ensemble sous la forme d'une union finie ou dénombrable d'ensembles dénombrables ou finis.
\end{methode}
}

\begin{solution}
\begin{enumerate}
    \item Soit $I$ un ensemble non vide. 
    $$I = \bigcup_{n \in \Ne} \underbrace{\left \{ \beta \in B, \P(A_\beta) \geqslant \frac{1}{n} \right \}}_{\defeq I_n}.$$
    Soient $n, p \in \Ne$ et $(\beta_1, \dots, \beta_p) \in (I_n)^p$ deux à deux distincts. Alors
    \begin{align*}
        1 \geqslant \P \left( \bigsqcup_{i=1}^p A_{\beta_i} \right) &= \sum_{i=1}^p \P(A_{\beta_i}) \geqslant \sum_{i=1}^p \frac{1}{n}
    \end{align*}
    donc $p \leqslant n$. \\
    Ainsi, $I_n$ est fini et $|I_n| \leqslant n$ i.e. $I$ est au plus dénombrable. 
    \item \textcolor{red}{À vérifier} Soit $\omega \in X(\Omega)$. Par indépendance de $X$ avec elle-même,
    $$\P(X = \omega) = \P(\{X=\omega\} \cap \{X=\omega\})= \P(X=\omega)^2.$$
    On en déduit que $\P(X=\omega) \in \{ 0, 1 \}$. \\
    De plus, $\sum\limits_{\omega \in X(\Omega)} \P(X=\omega) = 1$ et donc il existe un unique $\omega_0 \in X(\Omega)$ tel que $X=\omega_0$ presque sûrement i.e. $X$ est presque sûrement constante. 
\end{enumerate}
\end{solution}

\section{Discontinuités des fonctions monotones}
\begin{tcolorbox}
    Soit $f \in \mathscr{F}([a, b], \R)$ une fonction monotone. Alors l'ensemble des points de discontinuité de $f$ est au plus dénombrable. 
\end{tcolorbox}

\begin{exercice}
    \emph{Exercice 1 TD VI} \\
    Soient $a < b$ deux réels et $f \in \mathscr{F}([a, b], \R)$ une fonction croissante. Pour tout $x \in ]a, b[$, on pose $v_f(x) = f(x^+)-f(x^-)$.
    \begin{enumerate}
        \item Soit $x \in ]a, b[$. Montrer que $v_f(x) \geqslant 0$ avec égalité si et seulement si $f$ est continue en $x$.
        \item Soit $p \in \Ne$ et $x_1 < \cdots < x_p$ des réels de $]a, b[$. Montrer que $\sum\limits_{j=1}^p v_f(x_j) \leqslant f(b)-f(a)$.
        \item En déduire que pour tout $\alpha > 0$, l'ensemble des points $x \in ]a, b[$ tels que $v_f(x) > \alpha$ est fini. 
        \item Montrer que l'ensemble des points de discontinuité de $f$ est au plus dénombrable. 
    \end{enumerate}
\end{exercice}

\begin{solution}
    \begin{enumerate}
        \item Soit $(x, y, z) \in ]a,b[^3$ tel que $y \leqslant x \leqslant z$. Par croissance de $f$ on a $f(y) \leqslant f(x) \leqslant f(z)$. La monotonie de $f$ assure qu'elle admet des limites à gauche et à droite en tout point:
        $$f(x^-) = \lim_{y \to x^-} f(y) \quad f(x^+) = \lim_{z \to x^+} f(z).$$
        Donc par passage à la limite dans l'encadrement, 
        $$f(x^-) \leqslant f(x) \leqslant f(x^+).$$
        On en déduit que $v_f(x) \geqslant 0$ avec égalité si et seulement si $f(x^+)=f(x)=f(x^-)$ i.e. si et seulement si $f$ est continue en $x$. 
        \item \begin{align*}
            \sum_{i=1}^p v_f(x) &= \sum_{i=1}^p \left( f(x_i^+) - f(x_i^-)\right) \\
            &= f(x_p^+)-f(x_p^-) + \sum_{i=1}^{p-1} \left( f(x_i^+) - f(x_i^-)\right) \\
            &\leqslant f(b) - f(x_p^-) + \sum_{i=1}^{p-1} \left( f(x_{i+1}^-) - f(x_i^-)\right) \\
            \text{ par télescopage } &\leqslant f(b) - f(x_1^-) \\
            &\leqslant f(b) - f(a)
        \end{align*}
        \item Soit $\alpha > 0$. Raisonnons par l'absurde en supposant que l'ensemble des points $x \in ]a, b[$ tels que $v_f(x) > \alpha$ est infini. \\
        Soit $n \in \N$. (à réécrire)
        $$\underbrace{f(b)-f(a)}_{\in \R} \geqslant \sum_{i=0}^p v_f(x_i) \geqslant p \alpha \xrightarrow[p \to \infty]{} + \infty \quad \text{ car } \alpha > 0.$$
        \item Soit $\mathscr{D}$ l'ensemble des points de discontinuité. On pose $\mathscr{D}_{\alpha} \defeq \left\{ x \in [a,b], v_f(x) > \alpha \right\}$.
        $$\mathscr{D} = \bigcup_{\alpha > 0} \mathscr{D}_\alpha = \bigcup_{n \in \Ne} \mathscr{D}_\frac{1}{n}.$$
        Nous avons écrit l'ensemble $\mathscr{D}$ comme une union dénombrable d'ensembles finis donc $\mathscr{D}$ est au plus dénombrable.
    \end{enumerate}
\end{solution}

Voir aussi l'exercice 4.10 (p. 297) de \cite{oraux_x_ens_3} dont l'énoncé est :
\begin{exercice}
    Soit $A$ une partie dénombrable de $\R$. Montrer l'existence d'une fonction monotone $f: \R \to \R$ dont $A$ est l'ensemble des points de discontinuités.
\end{exercice}   

\section{Nombres algébriques}
\begin{exercice}
\emph{Exercice 2 TD VI} \\
Un nombre $z$ est \emph{algébrique} s'il existe $n \in \Ne$ et $(a_0, \dots, a_n) \in \Q^{n+1}$ tels que $a_n \not=0$ et 
$$\sum_{k=0}^n a_k z^k = 0.$$
Montrer que l'ensemble des nombres algébriques est dénombrable. 
\end{exercice}

\marginnote{Définition à revoir sur le corps des coefficients}

\begin{marginfigure}
    \resizebox{5.5cm}{!}{
\begin{forest}
[$\frac{1}{1}$ 
    [$\frac{1}{2}$ 
        [$\frac{1}{3}$ 
            [$\frac{1}{4}$
                []
                []
            ] 
            [$\frac{4}{3}$
                []
                []
            ]
        ] 
        [$\frac{3}{2}$ 
            [$\frac{3}{5}$
                []
                []
            ] 
            [$\frac{5}{2}$
                []
                []
            ] 
        ]   
    ]
    [$\frac{2}{1}$ 
        [$\frac{2}{3}$ 
            [$\frac{2}{5}$
                []
                []
            ] 
            [$\frac{5}{3}$
                []
                []
            ]
        ]
        [$\frac{3}{1}$ 
            [$\frac{3}{4}$
                []
                []
            ]
            [$\frac{4}{1}$
                []
                []
            ]
        ]
    ]
]
\end{forest}
}
    L'arbre de \textsc{Calkin}-\textsc{Wilf} est un arbre dont les sommets sont en bijection avec les nombres rationnels positifs.
\end{marginfigure}

\begin{solution}
Les rationnels sont dénombrables \dots Donc pour $n$ fixé, $\Q_n[X]$ est dénombrable en tant que produits finis d'ensembles dénombrables.
Donc $\bigcup\limits_{n \in \N} \Q_n[X]$ est dénombrable comme réunion dénombrable d'ensembles dénombrables. \\
Pour tout polynôme dans $\Q_n[X]$, le nombre de ses racines est fini et de cardinal inférieur à $n$. 
Donc les nombres algébriques sont dénombrables car on peut établir une application surjective de leur ensemble sur une réunion dénombrable d'ensembles finis. 
\end{solution}


\section{Fonction indicatrice d'\textsc{Euler}}
\begin{defi}
    La \emph{fonction indicatrice d'\textsc{Euler}} est une fonction arithmétique de la théorie des nombres, qui à tout entier naturel $n$ non nul associe le nombre d'entiers compris entre 1 et $n$ et premiers avec $n$. \emph{(Wikipedia)} \\
    Autrement dit, 
    \begin{alignat*}{2}
        \varphi\ :\ \Ne\ &\longrightarrow\ \Ne\\
        n\ &\longmapsto\ \mathrm{Card}(\{m \in \Ne\ |\ m \leqslant n \text{ et } m \text{ premier avec } n \}).
    \end{alignat*}
\end{defi}

\marginnote{faire un graphe de $\varphi$}

\begin{tcolorbox}
    La fonction indicatrice d'\textsc{Euler} peut s'écrire
    $$\varphi\ :\ n \longmapsto n \cdot \prod_{p \in \mathscr{P}_n} \left(1 - \frac{1}{p} \right)$$
    en notant $\mathscr{P}_n$ l'ensemble des nombres premiers divisant $n$.
\end{tcolorbox}

\begin{preuve}
    (Wikipedia) \\
    La valeur de l'indicatrice d'\textsc{Euler} s'obtient à partir de la décomposition en facteurs premiers de $n$. On note $n = \prod\limits_{p \in \mathscr{P}_n} p^{k_i}$. Alors, $\varphi(n) = $
\end{preuve}

\begin{exercice}
    Issu de la RMS 132 3 p.32. \\
    Montrer que pour tout $n \in \Ne, \varphi(n) \geqslant \frac{\sqrt{n}}{2}$.
\end{exercice}

\begin{exercice}
\emph{Exercice 17. Chap. VI}. \\
    On note $\varphi$ la fonction indicatrice d'\textsc{Euler}. Montrer que 
    $$\forall n \in \Ne \quad n = \sum_{d|n} \varphi(d).$$
\end{exercice}

Les points suivants ne peuvent être compris qu'avec la correction.
\begin{itemize}
    \item Q2): Résultat à retenir (bien que rappeler dans le DS5):\\
    $$\forall p \in J \subset \P,\ p \text{ divise } a \Longleftrightarrow \prod_{p \in J} p \text{ divise } a$$
    \item Q3): un élément de $\Omega$ est premier avec $n$ si et seulement si il n'est divisible par aucun des diviseurs premiers de $n$. D'où
    $$\mathbb{P} \left(\{ m \in \Omega\ ;\ m \wedge n = 1 \} \right) = \prod_{p \in \mathscr{P}_n} \left(1 - \frac{1}{p} \right).$$
    \item Q4): Calculer le cardinal de $B_j = \{ j\cdot d,\ j \in \llbracket 1, k \rrbracket\ ;\ j \wedge k = 1 \}$.
    \item Q5): Montrer que $(B_d)_{d|n}$ forme un SCE de $\Omega$ et en déduire la formule de l'énoncé. $\displaystyle (\Omega = \bigcup_{d|n} B_d)$ \textcolor{green}{à compléter}
\end{itemize}

\section{\emph{Exercice 4. Chap. VII:}}
\begin{exercice}
    Lors d'une élection, 700 électeurs votent pour $A$ et 300 pour $B$. Quelle est la probabilité que, pendant le dépouillement, $A$ soit toujours strictement en tête?
\end{exercice}


\section{Distance en variation totale}
\begin{exercice}
\emph{Exercice 8. Chap. VII} \\
Soit $\mathscr{E}$ l'espace des suites réelles $(p_n)_{n \in \N}$ telles que la série $\sum |p_n|$ converge, muni de la norme $\Norme{p} = \sum\limits_{n=0}^{+\infty} |p_n|$. Soit $\mathscr{P}$ le sous-ensemble de $\mathscr{E}$ formé des suites réelles positives $(p_n)_{n \in \N}$ telles que $\Norme{p}=1$.
\begin{enumerate}
    \item Montrer que $\mathscr{P}$ est borné et convexe. \\
    \emph{On montrera que pour tout $(p, q) \in \mathscr{P}^2$ et $\lambda \in [0, 1], \lambda p + (1-\lambda)q \in \mathscr{P}$.}
    \item Pour $P, Q \in \mathscr{P}$, on pose $d(P, Q) = \sup\limits_{A \subset \N} \left| \sum\limits_{n \in A} p_n - \sum\limits_{n \in A} q_n \right|$. Montrer que $d(P,Q) \in [0,1]$.
    \item Soit $(p,q) \in [0, 1]^2, P = (1-p, p, 0, \dots)$ et $Q = (1-q, q, 0, \dots)$. Déterminer $d(P, Q)$.
    \item Soient $n \in \N$ et $\lambda \in \Rp$. Montrer l'inégalité $\sum\limits_{k=n+1}^{+\infty} \leqslant \me^{\lambda} \frac{\lambda^{n+1}}{(n+1)!}$.
    \item Soient $X_\lambda$ et $X_\mu$ deux variables aléatoires suivant une loi de \textsc{Poisson} de paramètres respectifs $\lambda$ et $\mu$. Soit $P_\lambda = (\P(X_\lambda=n))_{n \in \N}$ et $P_\mu = (\P(X_\mu=n))_{n \in \N}$. Soit $n \in \Ne$. Montrer l'inégalité
    $$d(P_\lambda, P_\mu) \leqslant \max_{A \subset \llbracket 0, n \rrbracket} \left| \sum_{k \in A} \P(X_\lambda=k) - \sum_{k \in A} \P(X_\mu=k) \right| + \frac{\lambda^{n+1}}{(n+1)!} + \frac{\mu^{n+1}}{(n+1)!}.$$
\end{enumerate}
\end{exercice}



\begin{enumerate}
    \item Les éléments de $\mathscr{P}$ sont des suites positives...
    \item Bien justifier l'existence de la borne supérieure.
    \item Considérer une partie $B$ de $\N$ et distinguer quatre cas ($\{0\} \subset B$ et $\{1\} \not\subset B$; $\{0\} \not\subset B$ et $\{1\} \subset B$; $\{0, 1\} \subset B$; $ \{0, 1\} \not\subset B$). Le résultat est $\mathrm{d}(P, Q) = |p-q|$.
    \item Passer par la formule de \textsc{Taylor} avec reste intégral.
    \item Soit $B$ une partie de $\N$. Écrire $B = \left(B \cap \llbracket0, n \rrbracket \right) \cup \{k \in B; k > n\}$. Sommer sur ces deux ensembles, séparer la valeur absolue et majorer à la l'aide de la question précédente. (\textcolor{green}{à détailler})
\end{enumerate}

\section{Majoration de la variance d'une v.a.d.r.}
\begin{exercice}    
    \emph{Exercice 5. Chap. VII}\\
    Soit $X$ une variable aléatoire discrète réelle à valeurs dans $[a, b]$. Montrer que $\V(X) \leqslant \frac{(b-a)^2}{4}$ et discuter le cas d'égalité.
\end{exercice}

\begin{solution}
    Deux méthodes sont introduites.\\
    \underline{Méthode 1:}
    \begin{itemize}
        \item Poser la fonction $f:t \mapsto \E \left[(X-t)^2 \right]$.
        \item $f$ atteint son minimum en $\E[X]$ et vaut $\V(X)$. 
        \item Comparer $\V(X)$ à $f \left( \frac{a + b}{2} \right)$.
        \item \url{https://stats.stackexchange.com/questions/45588/variance-of-a-bounded-random-variable}
    \end{itemize}
    
    \underline{Méthode 2:} (vue en cours)
    \begin{itemize}
        \item Justifier l'existence de l'espérance et de la variance\\
        $\sum \gamma^n \P(X=x_k)$ converge, avec $\boxed{\gamma = \max \{ |a|, |b| \}}$ \textcolor{red}{\emph{(ne pas oublier les valeurs absolues  car on ne connaît pas les signes de $a$ et de $b$)}}.
        \item On remarque que $\V(X) = \V \left( X - \frac{a+b}{2} \right)$
        \item Or comme $a \leqslant X \leqslant b$, $$\displaystyle \left| X - \frac{a+b}{2}\right| \leqslant \frac{b-a}{2} \text{ et } \displaystyle \left( X - \frac{a+b}{2} \right)^2 \leqslant \left(\frac{b-a}{2}\right)^2$$
        \item Donc par le théorème de \textsc{König-Huygens}, 
        \begin{align*}
            \V \left(X - \frac{a+b}{2} \right) &= \E \Bigg[ \underbrace{\left(X - \frac{a+b}{2} \right)^2}_{\leqslant \left( \frac{b-a}{2} \right)^2} \Bigg] - \underbrace{\E \left[X-\frac{a+b}{2} \right]^2}_{\geqslant 0} \\
            \V(X) &\leqslant \frac{(b-a)^2}{4} \text{ par croissance de l'espérance}
        \end{align*}
    \end{itemize}
    
    \textbf{Cas d'égalité}\\
    D'après la majoration du premier terme et la minoration du deuxième terme de la relation ci-dessus, il y a égalité  si et seulement si
    
    \begin{align*}
        \V(X) = \frac{(b-a)^2}{4} &\Longleftrightarrow
        \begin{cases}
            \E \left[ \left(X - \frac{a+b}{2} \right)^2 \right] = \frac{(b-a)^2}{4}\\
            \text{et}\\
            \E \left[X-\frac{a+b}{2} \right]^2 = 0
        \end{cases}\\
        &\Longleftrightarrow
        \begin{cases}
            \P \left( X - \left(\frac{a+b}{2} \right)^2  \left(\frac{b-a}{2} \right)^2 \right)= 1\\
            \text{et}\\
            \E [X] = \frac{a+b}{2}
        \end{cases}\\
        &\Longleftrightarrow
        \begin{cases}
            \P \left( \{ X=b \} \cup \{X=a\} \right)= 1\\
            \text{et}\\
            \E [X] = \frac{a+b}{2}
        \end{cases}\\
        \V(X) = \frac{(b-a)^2}{4} &\Longleftrightarrow \boxed{\P(X=a)=\P(X=b)=\frac{1}{2}}
    \end{align*}
\end{solution}



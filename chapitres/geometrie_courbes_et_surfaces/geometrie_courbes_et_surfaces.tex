\chapter{Géométrie, courbes et surfaces}
\labch{geometrie_courbes_et_surfaces}

\section{Orthoptique d'une parabole}

\begin{exercice}
    \marginnote[0cm]{Source : \cite{exos_oraux} p. 203}
    Soit $\mathscr{P}$ la parabole de foyer $F = (1, 1)$ et de directrice
    $$\mathscr{D}: x-y+1 = 0.$$
    \begin{enumerate}
        \item Donner une équation cartésienne de $\mathscr{P}$ dans le repère canonique de $\R^2$.
        \item En utilisant une équation réduire de $\mathscr{P}$, reconnaître la courbe $\mathscr{O}$ -- la courbe orthoptique -- des points d'où l'on peut mener deux tangentes à $\mathscr{P}$ qui soient perpendiculaires entre elles.
    \end{enumerate}
\end{exercice}  

\begin{itemize}
    \item Géométrie élémentaire dans l'espace
    \item Réduction, tracé de coniques/quadratiques
    \item Courbe orthoptique de l'ellipse (cercle de \textsc{Monge})
    \item Tracé du pentagone régulier à la règle et au compas
\end{itemize}

\section{Plan d'étude des courbes}

Soit 
\begin{alignat*}{2}
    \text{Soit } f\ :\ I\ &\longrightarrow\ \R^2\\
    t\ &\longmapsto\ \big( x(t), y(t) \big)
\end{alignat*}

\begin{itemize}
    \item Domaine de définition de $f$.
    \item Domaine d'étude: recherche des symétries
    \item Étude des variations de $x$ et de $y$ 
    \item Étude des points critiques
    \item Étude des branches infinies
    \item Recherche des points doubles (s'il y en a)
    \item Tracé
\end{itemize}

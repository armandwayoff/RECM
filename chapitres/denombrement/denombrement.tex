\chapter{Dénombrement}
\labch{denombrement}

\section{Identité de \textsc{Vandermonde}}
\begin{prop}
    $$\sum_{k = 0}^{p} \binom{n}{k} \binom{m}{p-k} = \binom{n + m}{p}.$$
\end{prop}

Les deux expressions correspondent à deux façons de dénombrer les parties à $p$ éléments de $E \cup F$, où $E$ et $F$ sont deux ensembles disjoints fixés, de cardinaux respectifs $m$ et $n$. 


\section{Dénombrement des applications strictement croissantes}
\begin{exercice}
    Calculer le nombre d'applications strictement croissantes de $\llbracket 1, p \rrbracket$ dans $\llbracket 1, n \rrbracket$.
\end{exercice}

\begin{elem_sol}
    Réponse: $\displaystyle \binom{n}{p}$
\end{elem_sol}

\section{Dénombrement des applications croissantes}
Calcul du nombre d'applications croissantes de $\llbracket 1, p \rrbracket$ dans $\llbracket 1, n \rrbracket$.

\begin{itemize}
    \item Réponse: $\displaystyle \binom{n + p - 1}{p}$
    \item "Démonstration": représenter les éléments de l'ensemble de départ par des \emph{barres} qu'il faut placer entre les \emph{cases} de l'ensemble d'arrivée. 
\end{itemize}

\section{Dénombrement des surjections} \label{denombrement_surjections}
\begin{exercice}
    Calculer le nombre $S(p,n)$ de surjections de $\llbracket 1, p \rrbracket$ dans $\llbracket 1, n \rrbracket$. \\
\end{exercice}


\begin{elem_sol}
    \href{https://fr.wikipedia.org/wiki/Principe_d'inclusion-exclusion}{Principe d'inclusion-exclusion -- \textsf{wikipedia.org}} \\
    Étudier des cas particuliers \\
    Montrer que $n^p = \sum\limits_{k=0}^{n} \binom{n}{k} S(p,k)$ \\
    En déduire que $S(p,n) = (-1)^n \sum\limits_{j=1}^{n} \binom{n}{j} j^p$.
\end{elem_sol}

\section{Discontinuités des fonctions monotones}
\begin{prop}{}
    Soit $f \in \mathscr{F}([a, b], \R)$ une fonction monotone. Alors l'ensemble des points de discontinuité de $f$ est au plus dénombrable. 
\end{prop}

\begin{exercice}
    \marginnote[0cm]{\emph{Exercice 1 TD VI}}
    Soient $a < b$ deux réels et $f \in \mathscr{F}([a, b], \R)$ une fonction croissante. Pour tout $x \in ]a, b[$, on pose $f(x^-) \defeq \lim\limits_{t \to x^-} f(t)$, $f(x^+) \defeq \lim\limits_{t \to x^+} f(t)$ et $v_f(x) \defeq f(x^+)-f(x^-)$.
    \begin{enumerate}
        \item Soit $x \in ]a, b[$. Montrer que $v_f(x) \geqslant 0$ avec égalité si et seulement si $f$ est continue en $x$.
        \item Soit $p \in \Ne$ et $x_1 < \cdots < x_p$ des réels de $]a, b[$. Montrer que $\sum\limits_{j=1}^p v_f(x_j) \leqslant f(b)-f(a)$.
        \item En déduire que pour tout $\alpha > 0$, l'ensemble des points $x \in ]a, b[$ tels que $v_f(x) > \alpha$ est fini. 
        \item Montrer que l'ensemble des points de discontinuité de $f$ est au plus dénombrable. 
    \end{enumerate}
\end{exercice}

\begin{solution}
    \begin{enumerate}
        \item Soit $(x, y, z) \in ]a,b[^3$ tel que $y \leqslant x \leqslant z$. Par croissance de $f$ on a $f(y) \leqslant f(x) \leqslant f(z)$. La monotonie de la fonction $f$ assure qu'elle admet des limites à gauche et à droite en tout point. Donc par passage à la limite dans l'encadrement, 
        $$f(x^-) \leqslant f(x) \leqslant f(x^+).$$
        On en déduit que $v_f(x) \geqslant 0$ avec égalité si et seulement si $f(x^+)=f(x)=f(x^-)$ i.e. si et seulement si $f$ est continue en $x$. 
        \item \begin{align*}
            \sum_{i=1}^p v_f(x_i) &= \sum_{i=1}^p \big( f(x_i^+) - f(x_i^-)\big) \\
            &= f(x_p^+)-f(x_p^-) + \sum_{i=1}^{p-1} \big( f(x_i^+) - f(x_i^-)\big) \\
            &\leqslant f(b) - f(x_p^-) + \sum_{i=1}^{p-1} \big( f(x_{i+1}^-) - f(x_i^-)\big) \\
            \text{ par télescopage } &\leqslant f(b) - f(x_1^-) \\
            &\leqslant f(b) - f(a)
        \end{align*}
        \item Soit $\alpha > 0$. Raisonnons par l'absurde en supposant que l'ensemble des points $x \in ]a, b[$ tels que $v_f(x) > \alpha$ est infini. \\
        Soit $n \in \N$, d'après la question 2, 
        $$\underbrace{f(b)-f(a)}_{\in \R} \geqslant \sum_{i=0}^p v_f(x_i) \geqslant p \alpha \xrightarrow[p \to \infty]{} + \infty \quad \text{ car } \alpha > 0.$$
        On aboutit donc à une contradiction et l'ensemble des points $x \in ]a, b[$ tels que $v_f(x) > \alpha$ est fini. 
        \item Soit $\mathscr{D}$ l'ensemble des points de discontinuité. \\ 
        On pose $\mathscr{D}_{\alpha} \defeq \left\{ x \in [a,b], v_f(x) > \alpha \right\}$.
        $$\mathscr{D} = \bigcup_{\alpha > 0} \mathscr{D}_\alpha = \bigcup_{n \in \Ne} \mathscr{D}_\frac{1}{n}.$$
        Nous avons écrit l'ensemble $\mathscr{D}$ comme une union dénombrable d'ensembles finis donc $\mathscr{D}$ est au plus dénombrable.
    \end{enumerate}
\end{solution}

Voir aussi l'exercice 4.10 (p. 297) de \cite{oraux_x_ens_3} dont l'énoncé est :
\begin{exercice}
    Soit $A$ une partie dénombrable de $\R$. Montrer l'existence d'une fonction monotone $f: \R \to \R$ dont $A$ est l'ensemble des points de discontinuités.
\end{exercice}   

\section{Nombres algébriques}
\begin{exercice}
\emph{Exercice 2 TD VI} \\
Un nombre $z$ est \emph{algébrique} s'il existe $n \in \Ne$ et $(a_0, \dots, a_n) \in \Q^{n+1}$ tels que $a_n \not=0$ et 
$$\sum_{k=0}^n a_k z^k = 0.$$
Montrer que l'ensemble des nombres algébriques est dénombrable. 
\end{exercice}

\marginnote{Définition à revoir sur le corps des coefficients}

\begin{marginfigure}
    \resizebox{5.5cm}{!}{
\begin{forest}
[$\frac{1}{1}$ 
    [$\frac{1}{2}$ 
        [$\frac{1}{3}$ 
            [$\frac{1}{4}$
                []
                []
            ] 
            [$\frac{4}{3}$
                []
                []
            ]
        ] 
        [$\frac{3}{2}$ 
            [$\frac{3}{5}$
                []
                []
            ] 
            [$\frac{5}{2}$
                []
                []
            ] 
        ]   
    ]
    [$\frac{2}{1}$ 
        [$\frac{2}{3}$ 
            [$\frac{2}{5}$
                []
                []
            ] 
            [$\frac{5}{3}$
                []
                []
            ]
        ]
        [$\frac{3}{1}$ 
            [$\frac{3}{4}$
                []
                []
            ]
            [$\frac{4}{1}$
                []
                []
            ]
        ]
    ]
]
\end{forest}
}
    \note L'arbre de \textsc{Calkin}-\textsc{Wilf} est un arbre dont les sommets sont en bijection avec les nombres rationnels positifs.
\end{marginfigure}

\begin{solution}
Les rationnels sont dénombrables \dots \note Donc pour $n$ fixé, $\Q_n[X]$ est dénombrable en tant que produits finis d'ensembles dénombrables.
Donc $\bigcup\limits_{n \in \N} \Q_n[X]$ est dénombrable comme réunion dénombrable d'ensembles dénombrables. \\
Pour tout polynôme dans $\Q_n[X]$, le nombre de ses racines est fini et de cardinal inférieur à $n$. 
Donc les nombres algébriques sont dénombrables car on peut établir une application surjective de leur ensemble sur une réunion dénombrable d'ensembles finis. 
\end{solution}
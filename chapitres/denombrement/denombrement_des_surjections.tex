\begin{exercice}
    Calculer le nombre $S(p,n)$ de surjections de $\llbracket 1, p \rrbracket$ dans $\llbracket 1, n \rrbracket$. \\
\end{exercice}


\begin{elem_sol}
    \href{https://fr.wikipedia.org/wiki/Principe_d'inclusion-exclusion}{Principe d'inclusion-exclusion -- \textsf{wikipedia.org}} \\
    Étudier des cas particuliers \\
    Montrer que $n^p = \sum\limits_{k=0}^{n} \binom{n}{k} S(p,k)$ \\
    En déduire que $S(p,n) = (-1)^n \sum\limits_{j=1}^{n} \binom{n}{j} j^p$.
\end{elem_sol}
\begin{prop}
    Soit $f \in \mathscr{F}([a, b], \R)$ une fonction monotone. Alors l'ensemble des points de discontinuité de $f$ est au plus dénombrable. 
\end{prop}

\begin{exercice}
    \marginnote[0cm]{\emph{Exercice 1 TD VI}}
    Soient $a < b$ deux réels et $f \in \mathscr{F}([a, b], \R)$ une fonction croissante. Pour tout $x \in ]a, b[$, on pose $f(x^-) \defeq \lim\limits_{t \to x^-} f(t)$, $f(x^+) \defeq \lim\limits_{t \to x^+} f(t)$ et $v_f(x) \defeq f(x^+)-f(x^-)$.
    \begin{enumerate}
        \item Soit $x \in ]a, b[$. Montrer que $v_f(x) \geqslant 0$ avec égalité si et seulement si $f$ est continue en $x$.
        \item Soit $p \in \Ne$ et $x_1 < \cdots < x_p$ des réels de $]a, b[$. Montrer que $\sum\limits_{j=1}^p v_f(x_j) \leqslant f(b)-f(a)$.
        \item En déduire que pour tout $\alpha > 0$, l'ensemble des points $x \in ]a, b[$ tels que $v_f(x) > \alpha$ est fini. 
        \item Montrer que l'ensemble des points de discontinuité de $f$ est au plus dénombrable. 
    \end{enumerate}
\end{exercice}

\begin{solution}
    \begin{enumerate}
        \item Soit $(x, y, z) \in ]a,b[^3$ tel que $y \leqslant x \leqslant z$. Par croissance de $f$ on a $f(y) \leqslant f(x) \leqslant f(z)$. La monotonie de la fonction $f$ assure qu'elle admet des limites à gauche et à droite en tout point. Donc par passage à la limite dans l'encadrement, 
        $$f(x^-) \leqslant f(x) \leqslant f(x^+).$$
        On en déduit que $v_f(x) \geqslant 0$ avec égalité si et seulement si $f(x^+)=f(x)=f(x^-)$ i.e. si et seulement si $f$ est continue en $x$. 
        \item \begin{align*}
            \sum_{i=1}^p v_f(x_i) &= \sum_{i=1}^p \left( f(x_i^+) - f(x_i^-)\right) \\
            &= f(x_p^+)-f(x_p^-) + \sum_{i=1}^{p-1} \left( f(x_i^+) - f(x_i^-)\right) \\
            &\leqslant f(b) - f(x_p^-) + \sum_{i=1}^{p-1} \left( f(x_{i+1}^-) - f(x_i^-)\right) \\
            \text{ par télescopage } &\leqslant f(b) - f(x_1^-) \\
            &\leqslant f(b) - f(a)
        \end{align*}
        \item Soit $\alpha > 0$. Raisonnons par l'absurde en supposant que l'ensemble des points $x \in ]a, b[$ tels que $v_f(x) > \alpha$ est infini. \\
        Soit $n \in \N$, d'après la question 2, 
        $$\underbrace{f(b)-f(a)}_{\in \R} \geqslant \sum_{i=0}^p v_f(x_i) \geqslant p \alpha \xrightarrow[p \to \infty]{} + \infty \quad \text{ car } \alpha > 0.$$
        On aboutit donc à une contradiction et l'ensemble des points $x \in ]a, b[$ tels que $v_f(x) > \alpha$ est fini. 
        \item Soit $\mathscr{D}$ l'ensemble des points de discontinuité. \\ 
        On pose $\mathscr{D}_{\alpha} \defeq \left\{ x \in [a,b], v_f(x) > \alpha \right\}$.
        $$\mathscr{D} = \bigcup_{\alpha > 0} \mathscr{D}_\alpha = \bigcup_{n \in \Ne} \mathscr{D}_\frac{1}{n}.$$
        Nous avons écrit l'ensemble $\mathscr{D}$ comme une union dénombrable d'ensembles finis donc $\mathscr{D}$ est au plus dénombrable.
    \end{enumerate}
\end{solution}

Voir aussi l'exercice 4.10 (p. 297) de \cite{oraux_x_ens_3} dont l'énoncé est :
\begin{exercice}
    Soit $A$ une partie dénombrable de $\R$. Montrer l'existence d'une fonction monotone $f: \R \to \R$ dont $A$ est l'ensemble des points de discontinuités.
\end{exercice}   
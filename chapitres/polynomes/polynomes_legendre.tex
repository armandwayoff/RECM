\begin{defi}{Polynômes de \textsc{Legendre}}
    On appelle famille des \emph{polynômes de \textsc{Legendre}} la suite de polynômes $\suite{\Leg}{n}{n \in \Ne}$ définie par
    $$\Leg_n(X) \defeq \frac{1}{2^n} \sum_{k=0}^{n} \binom{n}{k}^2 (X-1)^{n-k}(X+1)^{k}.$$
\end{defi}

\marginnote[0cm]{
    Voir aussi...
}

\begin{exercice}
    \marginnote[0cm]{Source : \cite{exos_oraux} p. 25} 
    \begin{enumerate}
        \item Montrer que $\Leg_n(X) = \frac{1}{2^n \fact{n}} \left[ \big(X^2-1\big)^n \right]^{(n)}$. En déduire la parité de $\Leg_n$ et l'égalité $\sum\limits_{k=0}^n \binom{n}{k}^2 = \binom{2n}{n}$.
        \item Soit $n \in \Ne$, montrer que $\Leg_n(X)$ est scindé à racines simples dans $\interoo{-1}{1}$. 
        \item Montrer que pour tout $n \in \N$, $$\big(X^2-1\big) \Leg''_n(X) + 2X \Leg'_n(X) = n(n+1) \Leg_n.$$
    \end{enumerate}
\end{exercice}  

\begin{elem_sol}
    Pour montrer que $\Leg_n(X)$ est scindé à racines simples dans $\interoo{-1}{1}$, raisonner par récurrence et penser à \textsc{Rolle}. 
\end{elem_sol}

\begin{marginfigure}[-11.5cm]
    \centering
	\input{illustrations/i_polynomes_legendre}
	\caption*{\centering Les premiers polynômes de \textsc{Legendre}}
	\small
	\begin{align*}
	    \color{blue} \Leg_0 = 1 \\
	    \color{red} \Leg_1 = x \\
	    \color{green} \Leg_2 = \frac{1}{2}\big(3x^2-1\big) \\
	    \color{purple} \Leg_3 = \frac{1}{2}\big(5x^3-3x\big) \\
	    \color{black} \Leg_4 = \frac{1}{8}\big(35x^4-30x^2+3\big) \\
	    \color{orange} \Leg_5 = \frac{1}{8}\big(63x^5-70x^3+15x\big)
	\end{align*}
\end{marginfigure}


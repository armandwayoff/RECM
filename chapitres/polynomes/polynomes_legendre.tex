\begin{defi}
    Pour tout $n \in \Ne$, on définit le polynôme $\Leg_n$ par:
    $$\Leg_n(X) = \frac{1}{2^n} \sum_{k=0}^{n} \binom{n}{k}^2 (X-1)^{n-k}(X+1)^{k},$$
    $$\Leg_n(X) = \frac{1}{2^n n!} \left( (X^2-1)^n \right)^{(n)}.$$
\end{defi}

Les polynômes de \textsc{Legendre} constituent l'exemple le plus simple d'une suite de polynômes orthogonaux. \\
La suite vient de \url{https://bibmath.net/dico/index.php?action=affiche&quoi=./l/legendrepoly.html}.
Les polynômes de \textsc{Legendre} sont orthogonaux pour le produit scalaire
$$\langle P, Q \rangle = \int_{-1}^1 P(t) Q(t) \d t$$
mais ne sont toutefois par orthonormaux car
$$\langle \Leg_n, \Leg_n \rangle = \frac{2}{2n+1}.$$

\begin{elem_sol}
    Pour montrer que $\Leg_n(X)$ est scindé à racines simples dans $]-1, 1[$, raisonner par récurrence et penser à \textsc{Rolle}. 
\end{elem_sol}

\begin{marginfigure}[-11.5cm]
    \centering
	\input{illustrations/i_polynomes_legendre}
	\caption*{\centering Les premiers polynômes de \textsc{Legendre}}
	{\small
	\color{blue} $$\Leg_0 = 1$$ 
	\color{red} $$\Leg_1 = x$$ 
	\color{green} $$\Leg_2 = \frac{1}{2}(3x^2-1)$$ 
	\color{purple} $$\Leg_3 = \frac{1}{2}(5x^3-3x)$$ 
	\color{black} $$\Leg_4 = \frac{1}{8}(35x^4-30x^2+3)$$ 
	\color{orange} $$\Leg_5 = \frac{1}{8}(63x^5-70x^3+15x)$$
	}
\end{marginfigure}


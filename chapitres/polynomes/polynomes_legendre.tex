\begin{tcolorbox}
    Pour tout $n \in \Ne$, on définit le polynôme $\Leg_n$ par:
    $$\Leg_n(X) = \frac{1}{2^n} \sum_{k=0}^{n} \binom{n}{k}^2 (X-1)^{n-k}(X+1)^{k}.$$
    Les polynômes de \textsc{Legendre} constituent l'exemple le plus simple d'une suite de polynômes orthogonaux.
\end{tcolorbox}

\begin{itemize}
    \item Pour montrer que $\Leg_n(X)$ est scindé à racines simples dans $]-1, 1[$, raisonner par récurrence et penser à \textsc{Rolle}. 
    \item $2^n n! \Leg_n(X) = \left[ (X^2-1)^n \right]^{(n)} \eqdef  \left[ U_n(X) \right ]^{(n)}$.
\end{itemize}

\begin{marginfigure}[-6cm]
	\input{illustrations/i_polynomes_legendre}
	\caption{Les premiers polynômes de \textsc{Legendre}}
	{\scriptsize
	\color{blue} $\Leg_0 = 1$ \\ 
	\color{red} $\Leg_1 = x$ \\
	\color{green} $\Leg_2 = \frac{1}{2}(3x^2-1)$ \\
	\color{purple} $\Leg_3 = \frac{1}{2}(5x^3-3x)$ \\
	\color{black} $\Leg_4 = \frac{1}{8}(35x^4-30x^2+3)$ \\
	\color{orange} $\Leg_5 = \frac{1}{8}(63x^5-70x^3+15x)$
	}
\end{marginfigure}


\begin{defi}
    Pour tout $n \in \Ne$, on définit le polynôme $\Leg_n$ par:
    $$\Leg_n(X) \defeq \frac{1}{2^n} \sum_{k=0}^{n} \binom{n}{k}^2 (X-1)^{n-k}(X+1)^{k},$$
\end{defi}

\marginnote[0cm]{
    Voir aussi...
}

\begin{exercice}
    \marginnote[0cm]{\cite{exos_oraux} p. 25} 
    \begin{enumerate}
        \item Montrer que $\Leg_n(X) = \frac{1}{2^n n!} \left( (X^2-1)^n \right)^{(n)}$. En déduire la parité de $\Leg_n$ et l'égalité $\sum\limits_{k=0}^n \binom{n}{k}^2 = \binom{2n}{n}$.
        \item Soit $n \in \Ne$, montrer que $\Leg_n(X)$ est scindé à racines simples dans $]-1, 1[$. 
        \item Montrer que pour tout $n \in \N$, $(X^2-1) \Leg''_n(X) + 2X \Leg'_n(X) = n(n+1) \Leg_n$.
    \end{enumerate}
\end{exercice}  

\begin{elem_sol}
    Pour montrer que $\Leg_n(X)$ est scindé à racines simples dans $]-1, 1[$, raisonner par récurrence et penser à \textsc{Rolle}. 
\end{elem_sol}

\begin{marginfigure}[-11.5cm]
    \centering
	\input{illustrations/i_polynomes_legendre}
	\caption*{\centering Les premiers polynômes de \textsc{Legendre}}
	\small
	\begin{align*}
	    \color{blue} \Leg_0 = 1 \\
	    \color{red} \Leg_1 = x \\
	    \color{green} \Leg_2 = \frac{1}{2}(3x^2-1) \\
	    \color{purple} \Leg_3 = \frac{1}{2}(5x^3-3x) \\
	    \color{black} \Leg_4 = \frac{1}{8}(35x^4-30x^2+3) \\
	    \color{orange} \Leg_5 = \frac{1}{8}(63x^5-70x^3+15x)
	\end{align*}
\end{marginfigure}


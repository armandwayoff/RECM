\marginnote[0cm]{Lire \cite{calcul_infinitesimal} p. 297}
% Note d'un thème (cf. bureau) \\

Il arrive fréquemment que le calcul exact d’une intégrale soit difficile, voire
impossible pour certaines fonctions et il est courant, dans ce cas, de chercher à
approcher la valeur de l’intégrale en utilisant des polynômes comme les polynômes de \textsc{Bernoulli} par exemple.

\begin{defi}{Polynômes de \textsc{Bernoulli}}
    Les \emph{polynômes de \textsc{Bernoulli}} forment l'unique suite de polynômes $(\Bern_n)_{n \in \N}$ telle que:
    \begin{align*}
        &\Bern_0 = 1, \\
        \forall n \in \N,\ &\Bern'_{n+1} = (n+1)\Bern_n, \\
        \forall n \in \Ne,\ &\int_{0}^{1} \Bern_n(x) \d x = 0.
    \end{align*}
\end{defi}

\begin{defi}{Nombres de \textsc{Bernoulli}}
    Pour tout $n \geqslant 0$, on pose $\mathrm{b}_n \defeq \Bern_n(0)$. La suite de réels $(\mathrm{b}_n)_{n \in \N}$ est appelée suite des \emph{nombres de \textsc{Bernoulli}}.
\end{defi}  

\begin{table}[H]
    \centering
    \begingroup
        \renewcommand{\arraystretch}{1.2}
        \begin{tabularx}{\textwidth}{ |c| *{10}{>{\centering\arraybackslash}X|}}
         \hline
         $n$ & $0$ & $1$ & $2$ & $3$ & $4$ & $5$ & $6$ & $7$ & $8$ \\ \hline
         $\mathrm{b}_n$ & $1$ & $-\frac{1}{2}$ & $\frac{1}{6}$ & $0$ & $-\frac{1}{30}$ & $0$ & $\frac{1}{42}$ & $0$ & $-\frac{1}{30}$ \\
         \hline
    \end{tabularx}
    \endgroup
    \caption{Valeurs des premiers nombres de \textsc{Bernoulli}}
\end{table}

\begin{exercice}
    \begin{enumerate}
        \item Déterminer le degré de $\Bern_n(X)$ pour $n \geqslant 0$. 
        \item Montrer que, pour tout $n \geqslant 2$, $\Bern_n(0) = \Bern_n(1)$.
        \item Soient $n \in \N$ et $x \in \R$. Montrer que 
        $$\Bern_n(x) = \sum_{k=0}^n \binom{n}{k} \mathrm{b}_{n-k} x^k.$$
        \item En déduire, pour $n \geqslant 1$, une expression de $\mathrm{b}_n$ en fonction de $\mathrm{b}_0, \dots, \mathrm{b}_{n-1}$.
        \item Montrer que la suite $(\mathrm{b}_n)_{n \in \N}$ est une suite de rationnels et que, pour $n \geqslant 0$, les polynômes $\Bern_n(X)$ sont à coefficients rationnels.
        \item Montrer que pour tout $n \geqslant 0$, $(-1)^n \Bern_n(1-X) = \Bern_n(X)$.
        \item En déduire que 
        $$
        \begin{cases}
            \forall n \geqslant 1, \mathrm{b}_{2n+1} = 0, \\
            \forall n \geqslant 0, \Bern_{2n+1}(\frac{1}{2}) = 0.
        \end{cases}
        $$
    \end{enumerate}    
\end{exercice}
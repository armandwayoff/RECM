Lire \cite{calcul_infinitesimal} page 297.
\begin{tcolorbox}
    Les polynômes de \textsc{Bernoulli} sont l'unique suite de polynômes $(\Bern_n)_{n \in \N}$ telle que:
    \begin{align*}
        &\Bern_0 = 1, \\
        \forall n \in \N,\ &\Bern'_{n+1} = (n+1)\Bern_n, \\
        \forall n \in \Ne,\ &\int_{0}^{1} \Bern_n(x) \d x = 0.
    \end{align*}
\end{tcolorbox}

\begin{itemize}
    \item Pour montrer que pour tout $n \in \N$, on a $\Bern_n(X)=(-1)^n \Bern_n(1-X)$, montrer que la suite $(\Bern_n^{\star})$ définie par $\Bern_n^{\star} = (-1)^n \Bern_n(1-X)$ vérifie les hypothèses de la suite des polynômes de \textsc{Bernoulli}. L'unicité de cette suite permet d'établir l'égalité. 
\end{itemize}
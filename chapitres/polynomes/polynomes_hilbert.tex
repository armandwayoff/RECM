\begin{defi}[Polynômes de \nom{Hilbert}]
    On appelle famille des \emph{polynômes de \nom{Hilbert}} la suite de polynômes $\suite{\Hilb}{n}{n \in \N}$ définie par
    $$\Hilb_0 \defeq 1,\ \forall n \in \Ne,\ \Hilb_n \defeq \frac{X(X-1)\cdots(X-n+1)}{\fact{n}}.$$
\end{defi}

Voir aussi \nameref{polynome_hilbert} dans la partie algèbre linéaire. 

\begin{exercice}
    \source{\cite{exos_oraux} p.27, \cite{fmaalouf}}
    \begin{questions}
        \item Montrer que pour tout $n \in \N$, $\Hilb_n(\Z) \subset \Z$. En déduire que le produit de $n$ entiers consécutifs dans $\Z$ est divisible par $\fact{n}$.
        \item Soient $n \in \Ne$ et $Q \in \R_n[X]$. Montrer que les deux assertions suivantes sont équivalentes:
        \begin{questions}
            \item $Q(\Z) \subset \Z$,
            \item $\forall m \in \interent{0}{n}, Q(m) \in \Z$.
            \item $\exists (\lambda_0, \dots, \lambda_n) \in \Z^{n+1} \text{ tel que } Q = \sum\limits_{k=0}^n \lambda_k \Hilb_k$.
        \end{questions}
    \end{questions}
\end{exercice}

\begin{solution}
\end{solution}
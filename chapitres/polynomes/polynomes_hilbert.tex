\begin{defi}{Polynômes de \textsc{Hilbert}}
    On appelle famille des polynômes de \textsc{Hilbert} la suite de polynômes $(\Hilb_n)$ définie par
    $$\Hilb_0 \defeq 1,\ \forall n \in \Ne,\ \Hilb_n \defeq \frac{X(X-1)\cdots(X-n+1)}{n!}.$$
\end{defi}

Voir aussi \nameref{polynome_hilbert} dans la partie algèbre linéaire. 

\begin{exercice}
    \marginnote[0cm]{\cite{exos_oraux} p.27}
    \begin{enumerate}
        \item Montrer que pour tout $(k, n) \in \N \times \Z$, $\Hilb_k(n) \in \Z$.
        \item Soient $n \in \Ne$ et $Q \in \R_n[X]$. Montrer que les deux assertions suivantes sont équivalentes:
        \begin{enumerate}[label=(\roman*)]
            \item $Q(\Z) \subset \Z$,
            \item $\forall m \in \llbracket 0, n \rrbracket, Q(m) \in \Z$.
        \end{enumerate}
    \end{enumerate}
\end{exercice}
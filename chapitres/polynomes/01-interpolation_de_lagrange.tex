%========
\section{Polynômes d'interpolation de \nom{Lagrange}}

Pour tous $i,\, j \in \N$, on définit le symbole de \nom{Kronecker} par $\delta_{i,j} = 1$ si $i = j$ et $\delta_{i,j} = 0$ sinon.

Dans la suite, $n$ désigne un entier naturel non nul et $(a_0, \ldots, a_n)$ une famille de nombres réels deux à deux distincts.

\begin{defi}[Polynôme d'interpolation de \nom{Lagrange}]
Pour tout $(b_0,\ldots,b_n) \in \K^{n+1}$, il existe un unique polynôme $P \in \mathbb{K}_n[X]$ tel que, pour tout $i \in \interent{0}{n}$, $P(a_i) = b_i$.
\end{defi}

\todoinline{Dessin de l'interpolation}

\begin{tikzpicture}
    \begin{axis}[width=6.5cm,
        axis lines=middle,
        inner axis line style={-latex},
        grid=major,
        xmin=-1.2, xmax=1.2,
        ymin=-1.1, ymax=1.1,
        % xlabel=$x$, xlabel style={right},
        % ylabel=$y$, ylabel style={above},
        %tick style={thick},
        %ticklabel style={font=\normalsize},
        xtick=\empty, 
        ytick=\empty,
        axis line style={-latex}
    ]
    
    \def\a{-1.1}
    \def\b{1.1}
    \def\colour{BrickRed}
    
    \addplot[red,thick,samples=100,domain=\a:\b] {
    (x+5/8)*(x-1/8)*(x-1/2)*(x-4/5)/((-7/8+5/8)*(-7/8-1/8)*(-7/8-1/2)*(-7/8-4/5))*(-1/2)
    + (x+7/8)*(x-1/8)*(x-1/2)*(x-4/5)/((-5/8+7/8)*(-5/8-1/8)*(-5/8-1/2)*(-5/8-4/5))*(-1/9)
    + (x+7/8)*(x+5/8)*(x-1/2)*(x-4/5)/((1/8+7/8)*(1/8+5/8)*(1/8-1/2)*(1/8-4/5))*(-1/8)
    + (x+7/8)*(x+5/8)*(x-1/8)*(x-4/5)/((1/2+7/8)*(1/2+5/8)*(1/2-1/8)*(1/2-4/5))*(1/2)
    + (x+7/8)*(x+5/8)*(x-1/8)*(x-1/2)/((4/5+7/8)*(4/5+5/8)*(4/5-1/8)*(4/5-1/2))*(2/9)
    };
    
    \addplot[\colour,mark=*] coordinates {(-7/8,-1/2)} node[left] {$M_0$};
    \addplot[\colour,mark=*] coordinates {(-5/8,-1/9)} node[below] {$M_1$};
    \addplot[\colour,mark=*] coordinates {(1/8,-1/8)} node[below] {\contour{white}{$M_2$}};
    \addplot[\colour,mark=*] coordinates {(1/2,1/2)} node[above] {\contour{white}{$M_3$}};
    \addplot[\colour,mark=*] coordinates {(4/5,2/9)} node[right] {$M_4$};
    
    \draw[blue, thick, dotted] (-7/8,-1/2) -- (-7/8, 0);
    \draw[blue, thick, dotted] (-7/8,-1/2) -- (0, -1/2);

    \draw[blue, thick, dotted] (-5/8,-1/9) -- (-5/8, 0);
    \draw[blue, thick, dotted] (-5/8,-1/9) -- (0, -1/9);

    \draw[blue, thick, dotted] (1/8,-1/8) -- (1/8, 0);
    \draw[blue, thick, dotted] (1/8,-1/8) -- (0,-1/8);

    \draw[blue, thick, dotted] (1/2,1/2) -- (1/2, 0) node[below] {$a_3$};
    \draw[blue, thick, dotted] (1/2,1/2) -- (0, 1/2) node[left] {$f_3$};
    
    \draw[blue, thick, dotted] (4/5,2/9) -- (4/5, 0);
    \draw[blue, thick, dotted] (4/5,2/9) -- (0, 2/9);
    
    \draw[black, thick] (-0.8,0.5) node[above] 
    {\footnotesize \contour{white}{{\parbox{2cm}{\centering Polynôme \\ interpolateur}}}} to [out=640,in=800] ($(-0.3,-1/4)$);
    \end{axis}
    
\end{tikzpicture}

%-----------
\subsection{Base de polynômes}

On propose ici plusieurs démonstrations de ce résultat.

\begin{exercice}
On considère l'application
\[
\begin{array}{lccc}
\phi : & \K[X] &\to& \K^{n+1}\\
&P&\mapsto&(P(a_0),\ldots,P(a_n))
\end{array}
\]
et on note $\phi_n$ la restriction de $\phi$ à $\K_n[X]$. On pose $P_a = \prod\limits_{i=0}^n (X - a_i)$.
\begin{questions}
\item Avec le théorème du rang.
\begin{questions}
\item Déterminer $\Ker \phi$.

\item Montrer que $\K_n[X]$ est un supplémentaire de $\Ker \phi$.

\item En déduire que $\phi_n$ réalise une bijection de $\K_n[X]$ sur $\K^{n+1}$.

\item Conclure.
\end{questions}

\item Avec le déterminant de \nom{Vandermonde}.
\begin{questions}
\item Déterminer la matrice de $\phi_n$ dans la base canonique.

\item Conclure.
\end{questions}

\item Avec une base bien choisie de $\K_n[X]$. Pour tout $i \in \interent{0}{n}$, on pose $L_i = \prod\limits_{j \neq i} \frac{X - a_j}{a_i - a_j}$.
\begin{questions}
\item Pour tout $(i, j) \in \interent{0}{n}^2$, déterminer $L_i(a_j)$.

\item Montrer que $(L_0,\ldots,L_n)$ est une famille libre.

\item En déduire que $(L_0,\ldots,L_n)$ est une base de $\K_n[X]$.

\item Conclure.
\end{questions}
\end{questions}
\end{exercice}

\begin{elemsolution}
\begin{reponses}
\item
\begin{reponses}
\item Soit $P \in \Ker \phi$. Alors, $P(a_0) = \cdots = P(a_n) = 0$. Comme les réels $a_0,\ldots,a_n$ sont deux à deux distincts, alors $\prod\limits_{i=0}^n (X - a_i)$ divise $P$. Ainsi, il existe un polynôme $Q$ tel que $P = Q P_a$.

La réciproque étant triviale,
\[
\Ker \phi = \left\{Q P_a,\, Q \in \K[X]\right\}.
\]

\item Soit $S \in \K[X]$. D'après le théorème de la division eucilidienne, il existe un unique couple $(Q, R) \in \K[X]^2$ tel que
\[
S = Q P_a + R
\text{ et }
\deg(R) < \deg(P_a).
\]

Comme $\deg(P_a) = n + 1$, alors $R \in \K_n[X]$ et tout polynôme de $\K[X]$ se décompose de manière unique comme somme d'un élément de $\Ker \phi$ et d'un élément de $\K_n[X]$. Ainsi,
\[
\K[X] = \Ker \phi \oplus \K_n[X].
\]

\item D'après le théorème du rang, $\phi_n$ réalise une bijection de tout supplémentaire de $\Ker \phi$ dans $\Im \phi$. Ainsi, $\phi$ réalise une bijection de $\K_n[X]$ dans $\Im \phi$.

Or, $\dim(\K_n[X]) = n + 1$. Ainsi, $\dim(\Im \phi) = n + 1$. Comme $\Im \phi$ est un sous-espace vectoriel de $\K^{n+1}$ (espace vectoriel de dimension $n + 1$), alors $\Im \phi = K^{n+1}$.

Ainsi, $\phi_n : \K_n[X] \to \K^{n+1}$, restriction de $\phi$ à $\K_n[X]$ est une bijection.

\item Comme $\phi_n$ est une bijection, le vecteur $(b_0,\ldots,b_n)$ possède un unique antécédent par $\phi_n$. Cet antécédent est le polynôme recherché.
\end{reponses}

\item
\begin{reponses}
\item En notant $V = (v_{i,j})_{0 \leq i, j \leq n}$ la matrice de $\phi_n$ dans la base canonique, alors $v_{i,j}$ est la $i$\ieme{} composante de $\phi_n(X^j)$. Ainsi, $v_{i,j} = (a_j)^i$.

La matrice $V$ est donc la matrice de \nom{Vandermonde} de la famille $(a_0,\ldots,a_n)$.

\todoinline{Ajouter un lien vers les déterminants de Vandermonde}

\item Comme les réels $a_0,\ldots,a_n$ sont deux à deux distincts, la matrice $V$ est inversible. Ainsi, l'application $\phi_n$ est inversible et le vecteur $(b_0,\ldots,b_n)$ possède un unique antécédent par $\phi_n$. Cet antécédent est le polynôme recherché.
\end{reponses}

\item
\begin{reponses}
\item On remarque que, pour tout $j \in \interent{0}{n}$, $L_i(a_j) = \delta_{i,j}$.

\item Soient $\lambda_0,\ldots,\lambda_n$ des réels tels que
\[
\sum_{i=0}^n \lambda_i L_i = 0.
\]
Alors, pour tout $j \in \interent{0}{n}$,
\begin{align*}
\sum_{i=0}^n \lambda_i L_i(a_j) &= 0\\
\sum_{i=0}^n \lambda_i \delta_{i,j} &= 0\\
\lambda_j = 0.
\end{align*}

\item La famille $(L_0,\ldots,L_n)$ est une famille libre de $(n + 1)$ vecteurs de $\K_n[X]$. Comme la dimension de $\K_n[X]$ est égale à $n + 1$, alors $(L_0,\ldots,L_n)$ est une base de $\K_n[X]$.

\item Soit $P \in \K_n[X]$. Comme $(L_0,\ldots,L_n)$ est une base de $\K_n[X]$, la décomposition
\[
P = \sum_{i=0}^n \lambda_i L_i
\]
est unique. De plus, pour tout $j \in \entiers{0}{n}$,
\[
\sum_{i=0}^n b_i L_i(a_j) = b_j.
\]
Ainsi, $P = \sum_{i=0}^n b_i L_i$ est l'unique polynôme qui satisfait les conditions de l'énoncé.
\end{reponses}
\end{reponses}
\end{elemsolution}

\begin{defi}[Base d'interpolation de \nom{Lagrange}]
Il existe une base $(L_0,\ldots,L_n)$ de $\mathbb{K}_n[X]$ telle que, pour tout $(i, j) \in \interent{0}{n}^2$, $L_i(a_j) = \delta_{i,j}$.
\end{defi}

\begin{demo}
Notons $\mathscr{C} = (e_1,\ldots,e_n)$ la base canonique de $\K^{n+1}$. Étant donné l'exercice précédent, l'existence de cette base peut être assurée par différents arguments :
\begin{itemize}
\item on peut poser $L_i = \phi_n^{-1}(e_i)$.

\item on peut poser $\mathrm{Mat}_{\mathscr{C}}(L_i) = V^{-1}n \textrm{Mat}_{\mathscr{C}}(e_i)$.

\item par l'expression explicite des polynômes donnée précédemment.
\end{itemize}
\end{demo}

\begin{prop}{}{}
Pour tout $i \in \interent{0}{n}$, notons $\phi_i$ l'application qui, à tout polynôme $P$ associe le scalaire $P(a_i)$ ($\phi_i$ est l'évaluation en $a_i$).
\begin{enumerate}
\item $\sum\limits_{i=0}^n L_i = 1$.

\item $(\phi_0,\ldots,\phi_n)$ est la famille des applications linéaires coordonnées associée à la base $(L_0,\ldots,L_n)$. En particulier, $(\phi_0,\ldots,\phi_n)$ est une base de $\K_n[X]^*$.
\end{enumerate}
\end{prop}

\begin{demo}
\begin{enumerate}
\item Pour tout $j \in \interent{0}{n}$,
\[
\sum_{i=0}^n L_i(a_j) = \sum_{i=0}^n \delta_{i,j} = 1.
\]
Ainsi, $1 - \sum_{i=0}^n L_i$ est un polynôme de $\K_n[X]$ possédant $(n + 1)$ racines distinctes. Il s'agit donc du polynôme nul.

\item D'après les démonstrations précédentes, pour tout polynôme $P$ de degré au plus $n$,
\begin{align*}
P
= \sum_{i=0}^n P(a_i) L_i
= \sum_{i=0}^n \phi_i(P) L_i.
\end{align*}
\end{enumerate}
\end{demo}

\begin{theo}
L'application linéaire $\pi$ suivante définit une projecteur :
\[
\begin{array}{lccc}
\pi : &\K[X]&\to\K[X]\\
&P&\mapsto&\sum_{i=0}^n P(a_i) L_i.
\end{array}
\]
\end{theo}

\begin{demo}
On remarque que $\pi$ est un endomorphisme de $\K[X]$. Ainsi, pour montrer que $\pi$ est un projecteur, il suffit de montrer que $\pi \circ \pi = \pi$.

D'après les propriétés des polynômes $L_0,\ldots,L_n$,
\[
\pi(L_i)
= \sum_{j=0}^n L_i(a_j) L_j
= \sum_{j=0}^n \delta_{i,j} L_j
= L_i.
\]

Soit $P \in \K[X]$. Alors,
\[
\pi(\pi(P))
= \sum_{i=0}^n P(a_i) \pi(L_i)
= \sum_{i=0}^n P(a_i) L_i
= \pi(P),
\]
et obtient le résultat annoncé
\end{demo}

\begin{remarque}
On montre sans difficulté que :
\[
\Ker \pi = \left\{ \prod_{i=0}^n (X - a_i) Q ; Q \in \K[X] \right\}.
\]

De plus, comme $(L_0,\ldots,L_n)$ est une base de $\K_n[X]$, alors
\begin{align*}
\mathrm{Im} \pi &= \mathrm{Vect}\{\pi(L_i),\, i \in \interent{0}{n}\} \\
&= \mathrm{Vect}\{L_i,\, i \in \interent{0}{n}\} \\
&= R_n[X].
\end{align*}
Ainsi,
\[
\mathrm{Im} \pi = \K_n[X].
\]

On retrouve ainsi la propriété de l'exercice précédent qui assure que :
\[
\K_n[X] \oplus \left\{ \prod_{i=0}^n (X - a_i) Q ; Q \in \K[X] \right\} = \K[X].
\]
\end{remarque}

%-----------
\subsection{Erreur d'approximation}

\begin{solution}
\begin{questions}
\item On montre sans difficulté que $\eg$ est un morphisme. \\
De plus, en notant $(\eg_0,\ldots,\eg_n)$ la base canonique de $\K^{n+1}$, alors pour tout entier naturel $i \in \interent{0}{n}$, $\eg(L_i) = \eg_i$. Ainsi, l'image par $\eg$ d'une base est une base et
{
$\delta$ \text{ est un isomorphisme}.
}

\item Soit $f$ une fonction à valeurs réelles. Comme $(f(a_i))_{i\in\interent{0}{n}} \in \K^{n+1}$ et que $\eg$ est surjective, d'après la question précédente, il existe un polynôme $P \in \K_n[X]$ tel que pour tout $i \in \interent{0}{n}$, $P(a_i) = f(a_i)$.

\item Soit $x \in [a,b]$. Si $x \in \{a_0,\ldots,a_n\}$, tout réel $\xi \in [-1, 1]$ convient. Sinon, soit $K \in \K$ tel que $\phi~:~[a,b] \to \K,\, x \mapsto f(x) - P(x) - K \prod_{i=0}^n (x - a_i) = 0$.\\
La fonction $\phi$ possède $n+2$ zéros, donc, comme $f$ est de classe $\mathscr{C}^{n+1}$, en appliquant plusieurs fois Rolles (à rédiger~!), il existe $\xi \in ]a,b[$ tel que $\phi^{(n+1)}(\xi) = 0$. Ainsi, $0 = f^{(n+1)}(\xi) - (n+1)! \cdot K$.\\
Comme $f^{(n+1)}$ est continue sur le segment $[a,b]$, elle admet une borne supérieure. D'où,
\[
|f(x) - P(x)| \leq \sup_{[a,b]} |f^{(n+1)}| \frac{\prod_{i=0}^n |x-a_i|}{(n+1)!}.
\]
{
On peut ainsi, pour des fonctions régulières, contrôler l'erreur d'approximation de la fonction par un polynôme. Cependant, tout ne se passe pas bien et il arrive que le polynôme d'interpolation de Lagrange { ne converge pas} (en un sens à préciser) vers la fonction $f$. On pourra consulter le sujet de Saint-Cyr 1993.
}

\item Le polynôme de Tchebychev $T_{n+1}$ est de degré $(n+1)$. De plus, pour tout $k \in \interent{0}{n}$, le réel $x_k = \cos \frac{(2k+1)\pi}{2(n+1)}$ est une racine de $T_{n+1}$. De plus, la fonction cosinus étant strictement monotone sur $[0,\pi]$, ces racines sont distinctes. Comme $T_{n+1}$ est de degré $n+1$, il possède au plus $n+1$ racines distinctes, donc ses racines sont exactement les $(x_k)_{k\in\interent{0}{n}}$ et $T_{n+1}$ est scindé.

\item On montre en utilisant les formules d'Euler et de Newton que le coefficient de plus haut degré de $T_{n+1}$ est $2^n$. Ainsi, $t_{n+1}$ est bien un polynôme unitaire.

\item Soit $Q$ un polynôme unitaire de degré $n+1$. Supposons par l'absurde que $\|Q\|_\infty < 2^{-n}$ et posons $R = Q - t_{n+1}$. Alors, pour tout $k \in \interent{0}{n+1}$, $t_{n+1}(\cos \frac{k\pi}{n+1}) = (-1)^k 2^{-n}$. Ainsi, le polynôme $R$ change $n+2$ fois de signe donc, d'après le théorème de Rolles, admet au moins $n+1$ racines distinctes. De plus, comme $Q$ et $t_{n+1}$ sont de degré $n+1$, alors $R$ est de degré au plus $n$. Ainsi, $R = 0$, soit $Q = t_{n+1}$ et on obtient une contradiction car $\|t_{n+1}\|_\infty = 2^{-n}$. D'où,
{
\[\|Q\|_\infty \geq 2^{-n}.\]
}

Soit $Q$ tel que $\|Q\|_\infty = 2^{-n}$. Posons $R = Q - t_{n+1}$. Notons, pour tout $k \in \interent{0}{n+1}$, $y_k = \cos \frac{k\pi}{n+1}$ et $L_R$ le polynôme d'interpolation de Lagrange de $R$ en les points $(y_k)_{k\in\interent{0}{n}}$. \\
D'une part, $R$ est de degré au plus $n$ car $Q$ et $t_{n+1}$ sont unitaires. Ainsi, $R = L_R$. Donc, le coefficient de degré $n+1$ de $L_R$ est nul, soit $\sum_{k=0}^n \frac{R(y_k)}{\prod_{i\neq k} (y_k-y_i)} = 0$.

Soit $k \in \interent{0}{n+1}$,
\begin{align*}
R(y_k) &= Q(y_k) - (-1)^k 2^{-n} \\
&= 2^{-n} \left[\frac{Q(y_k)}{2^{-n}} - (-1)^k\right],
\end{align*}
soit
\[
-2^{-n} (1 + (-1)^k) \leq R(y_k) \leq 2^{-n} (1 - (-1)^k)
\]
et $(-1)^k R(y_k) \geq 0$.

Or, $\prod_{i\neq k} (y_k-y_i)$ est du même signe que $(-1)^k$. Ainsi, d'après les deux points précédents,
\[
\forall\, k \in \interent{0}{n},\, R(y_k) = 0.
\]

Finalement, $R$ est le polynôme nul, soit
{
$Q = t_{n+1}$.
}

\item En notant $Q = \prod_{i=0}^n (X-a_i)$, $Q$ est un polynôme unitaire de degré $n+1$. Comme
\[
|f(x) - P_f(x)| \leq \|f^{(n+1)}\|_\infty \frac{\|Q\|_\infty}{(n+1)!},
\]
cette majoration est minimale lorsque $Q = t_{n+1}$. Ainsi, les racines des polynômes de Tchebychev permettent de minimiser l'erreur dans l'intrepolation de Lagrange.
% \indic{

Attention, cela ne signifie pas pour autant que le polynôme d'interpolation converge vers $f$ lorsque $n$ tend vers l'infini. On pourra étudier en particulier le phénomène de Runge.
% }
\end{questions}
\end{solution}
%========
\section{Polynômes d'interpolation de \nom{Lagrange}}

\todoinline{Parler de la constante de Lebesgue}

Pour tous $i,\, j \in \N$, on définit le symbole de \nom{Kronecker} par $\delta_{i,j} = 1$ si $i = j$ et $\delta_{i,j} = 0$ sinon.

Dans la suite, $n$ désigne un entier naturel non nul et $(a_0, \ldots, a_n)$ une famille de nombres réels deux à deux distincts. On note $P_a = \prod\limits_{i=0}^n (X - a_i)$.

\begin{defi}[Polynôme d'interpolation de \nom{Lagrange}]
Pour tout $(b_0,\ldots,b_n) \in \K^{n+1}$, il existe un unique polynôme $P \in \mathbb{K}_n[X]$ tel que, pour tout $i \in \interent{0}{n}$, $P(a_i) = b_i$.
\end{defi}

\todoinline{Dessin de l'interpolation}

\begin{tikzpicture}
    \begin{axis}[width=6.5cm,
        axis lines=middle,
        inner axis line style={-latex},
        grid=major,
        xmin=-1.2, xmax=1.2,
        ymin=-1.1, ymax=1.1,
        % xlabel=$x$, xlabel style={right},
        % ylabel=$y$, ylabel style={above},
        %tick style={thick},
        %ticklabel style={font=\normalsize},
        xtick=\empty, 
        ytick=\empty,
        axis line style={-latex}
    ]
    
    \def\a{-1.1}
    \def\b{1.1}
    \def\colour{BrickRed}
    
    \addplot[red,thick,samples=100,domain=\a:\b] {
    (x+5/8)*(x-1/8)*(x-1/2)*(x-4/5)/((-7/8+5/8)*(-7/8-1/8)*(-7/8-1/2)*(-7/8-4/5))*(-1/2)
    + (x+7/8)*(x-1/8)*(x-1/2)*(x-4/5)/((-5/8+7/8)*(-5/8-1/8)*(-5/8-1/2)*(-5/8-4/5))*(-1/9)
    + (x+7/8)*(x+5/8)*(x-1/2)*(x-4/5)/((1/8+7/8)*(1/8+5/8)*(1/8-1/2)*(1/8-4/5))*(-1/8)
    + (x+7/8)*(x+5/8)*(x-1/8)*(x-4/5)/((1/2+7/8)*(1/2+5/8)*(1/2-1/8)*(1/2-4/5))*(1/2)
    + (x+7/8)*(x+5/8)*(x-1/8)*(x-1/2)/((4/5+7/8)*(4/5+5/8)*(4/5-1/8)*(4/5-1/2))*(2/9)
    };
    
    \addplot[\colour,mark=*] coordinates {(-7/8,-1/2)} node[left] {$M_0$};
    \addplot[\colour,mark=*] coordinates {(-5/8,-1/9)} node[below] {$M_1$};
    \addplot[\colour,mark=*] coordinates {(1/8,-1/8)} node[below] {\contour{white}{$M_2$}};
    \addplot[\colour,mark=*] coordinates {(1/2,1/2)} node[above] {\contour{white}{$M_3$}};
    \addplot[\colour,mark=*] coordinates {(4/5,2/9)} node[right] {$M_4$};
    
    \draw[blue, thick, dotted] (-7/8,-1/2) -- (-7/8, 0);
    \draw[blue, thick, dotted] (-7/8,-1/2) -- (0, -1/2);

    \draw[blue, thick, dotted] (-5/8,-1/9) -- (-5/8, 0);
    \draw[blue, thick, dotted] (-5/8,-1/9) -- (0, -1/9);

    \draw[blue, thick, dotted] (1/8,-1/8) -- (1/8, 0);
    \draw[blue, thick, dotted] (1/8,-1/8) -- (0,-1/8);

    \draw[blue, thick, dotted] (1/2,1/2) -- (1/2, 0) node[below] {$a_3$};
    \draw[blue, thick, dotted] (1/2,1/2) -- (0, 1/2) node[left] {$f_3$};
    
    \draw[blue, thick, dotted] (4/5,2/9) -- (4/5, 0);
    \draw[blue, thick, dotted] (4/5,2/9) -- (0, 2/9);
    
    \draw[black, thick] (-0.8,0.5) node[above] 
    {\footnotesize \contour{white}{{\parbox{2cm}{\centering Polynôme \\ interpolateur}}}} to [out=640,in=800] ($(-0.3,-1/4)$);
    \end{axis}
    
\end{tikzpicture}

%-----------
\subsection{Base de polynômes}

On propose ici plusieurs démonstrations de ce résultat.

\begin{exercice}
On considère l'application
\[
\fonction[\phi]{\K[X]}{\K^{n+1}}{P}{\big(P(a_0),\ldots,P(a_n)\big)}
\]
et on note $\phi_n$ la restriction de $\phi$ à $\K_n[X]$.
\begin{questions}
\item Avec le théorème du rang.
\begin{questions}
\item Déterminer $\Ker \phi$.

\item Montrer que $\K_n[X]$ est un supplémentaire de $\Ker \phi$.

\item En déduire que $\phi_n$ réalise une bijection de $\K_n[X]$ sur $\K^{n+1}$.

\item Conclure.
\end{questions}

\item Avec le déterminant de \nom{Vandermonde}.
\begin{questions}
\item Déterminer la matrice de $\phi_n$ dans la base canonique.

\item Conclure.
\end{questions}

\item Avec une base bien choisie de $\K_n[X]$. Pour tout $i \in \interent{0}{n}$, on pose $L_i = \prod\limits_{j \neq i} \frac{X - a_j}{a_i - a_j}$.
\begin{questions}
\item Pour tout $(i, j) \in \interent{0}{n}^2$, déterminer $L_i(a_j)$.

\item Montrer que $(L_0,\ldots,L_n)$ est une famille libre.

\item En déduire que $(L_0,\ldots,L_n)$ est une base de $\K_n[X]$.

\item Conclure.
\end{questions}
\end{questions}
\end{exercice}

\begin{elemsolution}
\begin{reponses}
\item
\begin{reponses}
\item Soit $P \in \Ker \phi$. Alors, $P(a_0) = \cdots = P(a_n) = 0$. Comme les réels $a_0,\ldots,a_n$ sont deux à deux distincts, alors $\prod\limits_{i=0}^n (X - a_i)$ divise $P$. Ainsi, il existe un polynôme $Q$ tel que $P = Q P_a$.

La réciproque étant triviale,
\[
\Ker \phi = \left\{Q P_a,\, Q \in \K[X]\right\}.
\]

\item Soit $S \in \K[X]$. D'après le théorème de la division eucilidienne, il existe un unique couple $(Q, R) \in \K[X]^2$ tel que
\[
S = Q P_a + R
\text{ et }
\deg(R) < \deg(P_a).
\]

Comme $\deg(P_a) = n + 1$, alors $R \in \K_n[X]$ et tout polynôme de $\K[X]$ se décompose de manière unique comme somme d'un élément de $\Ker \phi$ et d'un élément de $\K_n[X]$. Ainsi,
\[
\K[X] = \Ker \phi \oplus \K_n[X].
\]

\item D'après le théorème du rang, $\phi_n$ réalise une bijection de tout supplémentaire de $\Ker \phi$ dans $\Im \phi$. Ainsi, $\phi$ réalise une bijection de $\K_n[X]$ dans $\Im \phi$.

Or, $\dim(\K_n[X]) = n + 1$. Ainsi, $\dim(\Im \phi) = n + 1$. Comme $\Im \phi$ est un sous-espace vectoriel de $\K^{n+1}$ (espace vectoriel de dimension $n + 1$), alors $\Im \phi = K^{n+1}$.

Ainsi, $\phi_n : \K_n[X] \to \K^{n+1}$, restriction de $\phi$ à $\K_n[X]$ est une bijection.

\item Comme $\phi_n$ est une bijection, le vecteur $(b_0,\ldots,b_n)$ possède un unique antécédent par $\phi_n$. Cet antécédent est le polynôme recherché.
\end{reponses}

\item
\begin{reponses}
\item En notant $V = (v_{i,j})_{0 \leq i, j \leq n}$ la matrice de $\phi_n$ dans la base canonique, alors $v_{i,j}$ est la $i$\ieme{} composante de $\phi_n(X^j)$. Ainsi, $v_{i,j} = (a_j)^i$.

La matrice $V$ est donc la matrice de \nom{Vandermonde} de la famille $(a_0,\ldots,a_n)$.

\todoinline{Ajouter un lien vers les déterminants de Vandermonde}

\item Comme les réels $a_0,\ldots,a_n$ sont deux à deux distincts, la matrice $V$ est inversible. Ainsi, l'application $\phi_n$ est inversible et le vecteur $(b_0,\ldots,b_n)$ possède un unique antécédent par $\phi_n$. Cet antécédent est le polynôme recherché.
\end{reponses}

\item
\begin{reponses}
\item On remarque que, pour tout $j \in \interent{0}{n}$, $L_i(a_j) = \delta_{i,j}$.

\item Soient $\lambda_0,\ldots,\lambda_n$ des réels tels que
\[
\sum_{i=0}^n \lambda_i L_i = 0.
\]
Alors, pour tout $j \in \interent{0}{n}$,
\begin{align*}
\sum_{i=0}^n \lambda_i L_i(a_j) &= 0\\
\sum_{i=0}^n \lambda_i \delta_{i,j} &= 0\\
\lambda_j = 0.
\end{align*}

\item La famille $(L_0,\ldots,L_n)$ est une famille libre de $(n + 1)$ vecteurs de $\K_n[X]$. Comme la dimension de $\K_n[X]$ est égale à $n + 1$, alors $(L_0,\ldots,L_n)$ est une base de $\K_n[X]$.

\item Soit $P \in \K_n[X]$. Comme $(L_0,\ldots,L_n)$ est une base de $\K_n[X]$, la décomposition
\[
P = \sum_{i=0}^n \lambda_i L_i
\]
est unique. De plus, pour tout $j \in \entiers{0}{n}$,
\[
\sum_{i=0}^n b_i L_i(a_j) = b_j.
\]
Ainsi, $P = \sum_{i=0}^n b_i L_i$ est l'unique polynôme qui satisfait les conditions de l'énoncé.
\end{reponses}
\end{reponses}
\end{elemsolution}

\begin{defi}[Base d'interpolation de \nom{Lagrange}]
Il existe une base $(L_0,\ldots,L_n)$ de $\mathbb{K}_n[X]$ telle que, pour tout $(i, j) \in \interent{0}{n}^2$, $L_i(a_j) = \delta_{i,j}$.
\end{defi}

\begin{demo}
Notons $\mathscr{C} = (e_1,\ldots,e_n)$ la base canonique de $\K^{n+1}$. Étant donné l'exercice précédent, l'existence de cette base peut être assurée par différents arguments :
\begin{itemize}
\item on peut poser $L_i = \phi_n^{-1}(e_i)$.

\item on peut poser $\mathrm{Mat}_{\mathscr{C}}(L_i) = V^{-1}n \textrm{Mat}_{\mathscr{C}}(e_i)$.

\item par l'expression explicite des polynômes donnée précédemment.
\end{itemize}
\end{demo}

\begin{prop}{}{}
Pour tout $i \in \interent{0}{n}$, notons $\phi_i$ l'application qui, à tout polynôme $P$ associe le scalaire $P(a_i)$ ($\phi_i$ est l'évaluation en $a_i$).
\begin{enumerate}
\item $\sum\limits_{i=0}^n L_i = 1$.

\item $(\phi_0,\ldots,\phi_n)$ est la famille des applications linéaires coordonnées associée à la base $(L_0,\ldots,L_n)$. En particulier, $(\phi_0,\ldots,\phi_n)$ est une base de $\K_n[X]^*$.
\end{enumerate}
\end{prop}

\begin{demo}
\begin{enumerate}
\item Pour tout $j \in \interent{0}{n}$,
\[
\sum_{i=0}^n L_i(a_j) = \sum_{i=0}^n \delta_{i,j} = 1.
\]
Ainsi, $1 - \sum_{i=0}^n L_i$ est un polynôme de $\K_n[X]$ possédant $(n + 1)$ racines distinctes. Il s'agit donc du polynôme nul.

\item D'après les démonstrations précédentes, pour tout polynôme $P$ de degré au plus $n$,
\begin{align*}
P
= \sum_{i=0}^n P(a_i) L_i
= \sum_{i=0}^n \phi_i(P) L_i.
\end{align*}
\end{enumerate}
\end{demo}

\begin{theo}
L'application linéaire $\pi$ suivante définit une projecteur :
\[
\begin{array}{lccc}
\pi : &\K[X]&\to\K[X]\\
&P&\mapsto&\sum_{i=0}^n P(a_i) L_i.
\end{array}
\]
\end{theo}

\begin{demo}
On remarque que $\pi$ est un endomorphisme de $\K[X]$. Ainsi, pour montrer que $\pi$ est un projecteur, il suffit de montrer que $\pi \circ \pi = \pi$.

D'après les propriétés des polynômes $L_0,\ldots,L_n$,
\[
\pi(L_i)
= \sum_{j=0}^n L_i(a_j) L_j
= \sum_{j=0}^n \delta_{i,j} L_j
= L_i.
\]

Soit $P \in \K[X]$. Alors,
\[
\pi(\pi(P))
= \sum_{i=0}^n P(a_i) \pi(L_i)
= \sum_{i=0}^n P(a_i) L_i
= \pi(P),
\]
et obtient le résultat annoncé
\end{demo}

\begin{remarque}
On montre sans difficulté que :
\[
\Ker \pi = \left\{ \prod_{i=0}^n (X - a_i) Q ; Q \in \K[X] \right\}.
\]

De plus, comme $(L_0,\ldots,L_n)$ est une base de $\K_n[X]$, alors
\begin{align*}
\mathrm{Im} \pi &= \mathrm{Vect}\{\pi(L_i),\, i \in \interent{0}{n}\} \\
&= \mathrm{Vect}\{L_i,\, i \in \interent{0}{n}\} \\
&= R_n[X].
\end{align*}
Ainsi,
\[
\mathrm{Im} \pi = \K_n[X].
\]

On retrouve ainsi la propriété de l'exercice précédent qui assure que :
\[
\K_n[X] \oplus \left\{ \prod_{i=0}^n (X - a_i) Q ; Q \in \K[X] \right\} = \K[X].
\]
\end{remarque}

%-----------
\subsection{Constante de \nom{Lebesgue}}

\begin{theo}
Soit $f$ une fonction continue sur $I$ et $P_n$ le polynôme d'interpolation de \nom{Lagrange} de $f$ aux poins $x_0,\ldots,x_n$. Pour tout polynôme $P \in \R_n[X]$,
\[
\forall\, x \in I,\, \module{f(x) - P_n(x)} \leq \left(1 + \sum_{k=0}^n \module{L_k(x)}\right) \norme{f - P}.
\]
\end{theo}

\begin{remarque}
La constante de \nom{Lebesgue} est le réel
\[
\lambda_n = \norme{\sum_{k=0}^n L_k(x)}_\infty.
\]
En général, il est difficile d'évaluer cette constante. Dans le cas de points d'interpolation équirépartis, on peut montrer $\lambda_n \sim_{n\to+\infty} \frac{2^{n+1}}{n \ln n}$. Dans le cas où les points d'interpolation sont les zéros des polynômes de \nom{Tchebychev}, une estimation de cette constante est également possible. Nous n'aborderons pas ces estimations ici.
\end{remarque}

\begin{elemsolution}
\cite{mines_2-mp-2001}
La continuité de $f$ assure que la fonction $f$ admet une borne supérieure.

En utilisant la propriété des polynômes d'interpolation de \nom{Lagrange},
\begin{align*}
\module{f(x) - P_n(x)}
&\leq \module{f(x)} + \module{P_n(x)}\\
&\leq \module{f(x)} + \sum_{k=0}^n \module{f(x_k)} \module{L_k(x)}\\
&\leq \left(1 + \sum_{k=0}^n \module{L_k(x)}\right) \norme{f}_\infty.
\end{align*}

Si $P$ est un polynôme de degré au plus $n$, alors il est égal à son polynôme d'interpolation de \nom{Lagrange}. On obtient donc le résultat annoncé en remplaçant $f$ par $f - P$ dans le calcul précédent.
\end{elemsolution}

%-----------
\subsection{Avec plus de régularité}

Dans cette partie, on suppose que $-1 \leq a_0 < a_1 < \cdots < a_n \leq n$.

\begin{theo}
Soit $f$ une fonction de classe $\mathscr{C}^{n+1}$ sur $\interff{-1}{1}$. Il existe un unique polynôme $P$ de degré inférieur ou égal à $n$ tel que
\[
\forall\, i \in \interent{0}{n},\, P(a_i) = f(a_i).
\]
De plus, il existe $\xi \in \interoo{-1}{1}$ tel que pour tout $x \in \interff{-1}{1}$,
\[
\module{f(x) - P(x)} = \frac{\module{f^{(n+1)}(x)}}{(n+1)!} \module{P_a(x)}.
\]
Enfin, la quantité $\norm{P_a}_\infty$ est minimale lorsque $a_0,\ldots,a_n$ sont les racines du polynôme de Tchebychev d'ordre~$n$.
\end{theo}

\begin{remarque}
On limite ici l'étude à l'intervalle $\interff{-1}{1}$. Cependant, une transformation affine permettrait ce généraliser ces résultats à un segment quelconque de $\R$.
\end{remarque}

\todoinline{Ajouter graphe avec une fonction approchée par des $(a_i)$ équidistants et des $(a_i)$ racines des polynômes de Tchebychev.}

\begin{exercice}
\begin{questions}
\item Montrer l'existence du polynôme $P$.
\end{questions}

Soit $x \in \interff{-1}{1}$. On pose $\phi : t \mapsto f(t) - P(t) - K \prod_{i=0}^n (t - a_i)$ où $K$ est un réel tel que $\phi(x) = 0$.

\begin{questions}[resume]
\item Montrer l'existence de $\xi$ lorsque $x \in \left\{a_0,\ldots,a_n\right\}$. On suppose dans la suite que $x$ n'appartient pas à cet ensemble.

\item Montrer que $\phi$ possède $n + 2$ zéros sur l'intervalle $\interff{-1}{1}$.

\item En déduire qu'il existe un réel $\xi \in \interoo{-1}{1}$ tel que $\phi^{n+1}(\xi) = 0$.

\item En déduire que, pour tout $x \in \interff{-1}{1}$,
\[
\module{f(x) - P(x)} = \frac{\module{f^{(n+1)}(\xi)}}{(n+1)!} P_a(x).
\]
\end{questions}

Soit $Q$ un polynôme unitaire de degré $n+1$. On note $T_{n+1}$ le polynôme de Tchebychev d'ordre $n + 1$ et, pour tout $k \in \interent{0}{n}$, $y_k = \cos \frac{k\pi}{n+1}$. On rappelle que le coefficient dominant de $T_{n+1}$ est $2^n$ et que, pour tout $k \in \interent{0}{n}$, $T_{n+1}(y_k) = (-1)^k$. On pose $t_{n+1} = 2^{-n} T_{n+1}$ et $R = Q - t_{n+1}$.
\begin{questions}[resume]
\item Montrer que $R$ est un polynôme de degré au plus $n$.

\item On suppose que $\norme{Q}_\infty \geq \frac{1}{2^n}$. Montrer que, pour tout $k \in \interent{0}{n}$, $(-1)^k R(y_k) \geq 0$. En déduire que $R = 0$.

\item En déduire que $\norme{Q}_\infty \leq 2^{-n}$ avec égalité si et seulement si $Q = t_{n+1}$.

\item Conclure.
\end{questions}
\end{exercice}

\begin{solution}
\begin{reponses}
\item Comme $(f(a_0),\ldots,f(a_n)) \in \R^{n+1}$, d'après la partie précédente, il existe un unique polynôme $P \in \R_n[X]$ tel que pour tout $i \in \interent{0}{n}$, $P(a_i) = f(a_i)$.

\item Si $x \in \{a_0,\ldots,a_n\}$, alors $f(x) - P(x) = 0$ et $P_a(x) = 0$. Ainsi, l'égalité attendue est satisfaite pour tout $x \in \interoo{-1}{1}$.

\item D'après la définition de $P$ et de $P_a$, pour tout $i \in \interent{0}{n}$, $\phi(a_i) = 0$. De plus, $K$ a été choisi tel que $\phi(x) = 0$. Comme $x, a_0,\ldots,a_n$ sont tous distincts, alors $\phi$ possède $(n + 2)$ zéros sur $\interff{-1}{1}$.

\item Comme $f$ est de classe $\mathscr{C}^{n+1}$ et que les polynômes $P$ et $P_a$ sont de classe $\mathscr{C}^\infty$, alors la fonction $\phi$ est de classe $\mathscr{C}^{n+1}$.

On applique successivement le théorème de \nom{Rolles} aux fonctions $\phi, \phi',\ldots, \phi^{(n)}$, entre chacun des $(n+2)$ zéros de $\phi$, puis entre les $(n+1)$ zéros de $\phi'$ et enfin entre les deux zéros de $\phi^{(n)}$. Ainsi, il existe $\xi \in \interoo{-1}{1}$ tel que $\phi^{(n+1)}(\xi) = 0$.

Comme $P$ est de degré $n$ et $P_a$ est unitaire de degré $n+1$, alors $\phi^{(n+1)} = f^{(n+1)} - K (n+1)!$.

Alors, $\phi(\xi) = f^{(n+1)}(\xi) - (n+1)! \cdot K = 0$ et
\[
f(x) - P(x) = \frac{f^{(n+1)}(\xi)}{(n+1)!} \times \prod_{i=0}^n (x-a_i).
\]

\item Comme $Q$ et $t_{n+1}$ sont de degrés $n+1$, alors $R$ est de degré au plus $n+1$. De plus, $Q$ et $t_{n+1}$ sont unitaires, donc $R$ est de degré au plus $n$.

\item Comme $T_{n+1} (y_k) = (-1)^k$, alors
\begin{align*}
(-1)^k 2^n R(y_k)
&= (-1)^k 2^n Q(y_k) + 1.
\end{align*}

Or, $\module{(-1)^k 2^n Q(y_k)} \leq 2^n \norme{Q}_\infty \geq 1$. Ainsi, $(-1)^k 2^n R(y_k) \geq 0$.

Ainsi, $R$ est un polynôme de degré $n$ qui change $n+1$ fois de signe, donc $R$ est identiquement nul.

\item D'après la question précédente,
\begin{itemize}
\item soit $\norme{Q}_\infty \geq 2^{-n}$ et alors $Q = t_{n+1}$, soit $\norme{Q}_\infty = 2^{-n}$ ;
\item soit $\norme{Q}_\infty < 2^{-n}$.
\end{itemize}
On obtient bien le résultat annoncé.

\item Le polynôme $P_a$ est un polynôme unitaire de degré $n+1$. Comme
\[
|f(x) - P(x)| \leq \|f^{(n+1)}\|_\infty \frac{\|P_a\|_\infty}{(n+1)!},
\]
cette majoration est minimale lorsque $P_a = t_{n+1}$. Ainsi, les racines des polynômes de Tchebychev permettent de minimiser l'erreur dans l'intrepolation de Lagrange.
\end{reponses}
\end{solution}

\begin{remarque}
La question précédente assure que, pour toute fonction $f$ de classe $\mathscr{C}^{n+1}$ sur $\interff{-1}{1}$, si $P$ est le polynôme d'interpolation de $f$ associé aux $a_0,\ldots,a_n$, alors
\[
\norme{f - P}_{\infty} \leq \frac{\norme{f}_\infty}{(n+1)!} \norme{P_a}_{\infty},
\]
et que le membre de droite est minimal lorsque $a_0,\ldots,a_n$ sont les racines du polynôme de Tchebychev d'ordre $n + 1$.

Cependant, ceci n'assure pas que le polynôme d'approximation converge vers la fonction $f$ lorsque le nombre $n$ de points d'approximations tend vers l'infini.
\end{remarque}

%-----------
\subsection{Phénomène de \nom{Runge}}

\todoinline{Intro qui explique ce qu'est le phénomène}

\begin{prop}{}{}
Soit $\alpha  0$. On considère, sur le segment $I = \interff{-1}{1}$, la fonction $f : x \mapsto \frac{1}{x^2 + \alpha^2}$. Pour tout $m$ entier naturel non nul et $k \in \entiers{0}{m-1}$, on pose $x_k = \frac{2 k + 1}{2 m}$ et $x_k' = - x_k$. On note $P_{2m}$ le polynôme d'interpolation de \nom{Lagrange} de $f$ aux points $(x_m',\ldots,x_1',x_1,\ldots,x_m)$. Alors, pour tout $\alpha$ suffisamment petit,
\[
\dim_{m\to+\infty} \norm{f - P_{2m}}_\infty = +\infty.
\]
\end{prop}

\begin{exercice}
On utilise les notations de la proposition précédente. De plus, on note $\omega_{2m} : x \mapsto \prod\limits_{k=0}^{m-1} (x^2 - x_k^2)$.
\begin{questions}
\item En étudiant les racines de $P_{2m}(X) - P_{2m}(-X)$, montrer que $P_{2m}$ est un polynôme pair.

\item En déduire que, pour tout $x \in I$, $f(x) - P_{2m}(x) = \frac{1}{x^2 + \alpha^2} \times \frac{\omega_{2m}(x)}{\omega_{2m}(\alpha \i)}$.

\item Montrer que $\omega_{2m}(1) \sim_{+\infty} \sqrt{2} \left(\frac{2}{\e}\right)^{2m}$.

\item On pose $f : t \mapsto \ln(\alpha^2 + t^2)$. Montrer que
\[
\module{\int_{\frac{k}{m}}^{\frac{k+1}{m}} f(t) \d t - \frac{1}{m} f\left(\frac{2 k + 1}{m}\right)} = O\left(\frac{1}{m^2}\right).
\]

\item En déduire qu'il existe deux constantes $C > 0$ et  $\beta_\alpha > 0$ telles que $\module{\omega_{2m}(\alpha \i)} \sim C \beta_\alpha^{2m}$.

\item Calculer $\lim\limits_{\alpha\to0} \beta_\alpha$.

\item Conclure.
\end{questions}
\end{exercice}

\begin{solution}
\begin{reponses}
\item Notons $\tilde{P}_{2m}(X) = P_{2m}(-X)$. On remarque que, pour tout $k \in \entiers{0}{m-1}$, $P_{2m}(x_k) = f(x_k) = f(-x_k) = \tilde{P}_{2m}(x_k)$ et $P_{2m}(x_k') = P_{2m}(x_k')$. Ainsi, le polynôme $P_{2m} - \tilde{P}_{2m}$ possède $2m$ racines distinctes. Comme il est de degré au plus $2 m- 1$, alors il s'agit du polynôme nul.

Ainsi, le polynôme $P_{2m}$ est un polynôme pair.

\item En mettant les fractions sous le même dénominarteur, il existe un polynôme $Q$ tel que
\[
f(x) - P_m(x) = \frac{Q(x)}{x^2 + \alpha^2}.
\]

Or, pour tout $k \in \entiers{0}{m-1}$, $f(x_k) = P_m(x_k)$ et $f(x_k') = P_m(x_k')$. Ainsi, les réels $x_m',\ldots,x_1',x_1,\ldots,x_m$ sont des racines de $Q$.

De plus, $P_{2m}$ est un polynôme de degré au plus $2 m - 1$ donc $\deg(P_{2m}) \leq 2 m - 2$.

Comme $Q = 1 - (X^2 + \alpha^2) P_{2m}$, alors $Q$ est un polynôme de degré au plus $2m$. Or, on a identifié $2m$ racines distinctes de $Q$. Ainsi, en notant $\lambda$ le coefficient dominant de $Q$,
\begin{align*}
Q(X)
&= \lambda \prod_{k=0}^{m-1} (X - x_k') \times \prod_{k=0}^{m-1} (X - x_k)\\
&= \lambda \prod_{k=0}^{m-1} (X + x_k) \times \prod_{k=0}^{m-1} (X - x_k)\\
&= \lambda \prod_{k=0}^{m-1} (X^2 - x_k^2)\\
&= \lambda \omega_{2m}(X).
\end{align*}

Alors, pour tout $x \in \interff{-1}{1}$,
\begin{align*}
f(x) - P_{2m}(x) &= \lambda \frac{\omega_{2m}(x)}{x^2 + \alpha^2}\\
\lambda \omega_{2m}(x) &= (x^2 + \alpha^2) (f(x) - P_{2m}(x))\\
&= 1 - (x^2 + \alpha^2) P_{2m}(x).
\end{align*}
Comme ces deux polynômes sont égaux en une infinité de points, alors ils sont égaux et, en particulier,
\[
\lambda \omega_{2m}(\alpha \i) = 1.
\]
On obtient ainsi le résultat annoncé.

\item D'après les définitions,
\begin{align*}
\omega_{2m}(1)
&= \prod_{k=0}^{m-1} \left(1 - \frac{2 k + 1}{2m}\right) \prod_{k=0}^{m-1} \left(1 + \frac{2 k + 1}{2 m}\right)\\
&= \prod_{k=0}^{m-1} \frac{2 (m - k) - 1)}{2 m} \prod_{k=0}^{m-1} \frac{2 (m + k) + 1)}{2 m}\\
&= \frac{\prod_{\ell=0}^{m-1} (2 \ell + 1) \prod_{\ell=m}^{2m-1} (2 \ell + 1)}{(2 m)^{2m}}\\
&= \frac{1 \times 3 \times 5 \times \cdots \times (4 m - 1)}{(2 m)^{2m}}\\
&= \frac{(4 m)!}{(2 m)! 2^{2m} (2 m)^{2 m}}.
\end{align*}
Il suffit d'appliquer la formule de Stirling pour obtenir l'équivalent annoncé.

\item En utilisant la formule de Taylor avec reste intégral, pour tout $t \in \interff{\frac{k}{m}}{\frac{k+1}{m}}$ en le point milieu $\frac{1}{1}\left(\frac{k}{m} + \frac{k+1}{m}\right) = \frac{2 k + 1}{2m}$,
\begin{align*}
f(t) - f\left(\frac{2k+1}{2m}\right)
&= \left(t - \frac{2 k + 1}{2 m}\right) f'\left(\frac{2k+1}{2m}\right) + \int_{\frac{2k+1}{2m}}^{t} \left(t - u\right) f''(u) \d u\\
\module{f(t) - f\left(\frac{2k+1}{2m}\right) - \left(t - \frac{2 k + 1}{2 m}\right) f'\left(\frac{2k+1}{2m}\right)} &\leq \frac{1}{2} \left(t - \frac{2 k + 1}{2m}\right)^2  \norme{f''}_{\infty,\interff{0}{1}} \\
\end{align*}
Ainsi, en utilisant l'inégalité triangulaire,
\begin{align*}
\module{\int_{\frac{k}{m}}^{\frac{k+1}{m}} f(t) \d t - \frac{1}{m} f\left(\frac{2 k +1}{2m}\right)} \leq \frac{1}{6} \frac{2}{(2m)^3} \norme{f''}_{\infty, \interff{0}{1}}.
\end{align*}
On obtient ainsi la domination annoncée.

\item En utilisant la définition,
\begin{align*}
\module{\omega_{2m}(\alpha \i)}
&= \prod_{k=0}^{m-1} (\alpha^2 + x_k^2).
\end{align*}

Ainsi,
\begin{align*}
\ln(\module{\omega_{2m}(\alpha \i)})
&= \sum_{k=0}^{m-1} \ln\left(\alpha^2 + \left(\frac{2 k + 1}{2m}\right)^2\right).
\end{align*}

Ainsi, en utilisant la question précédente et la relation de Chasles,
\[
\int_0^1 \ln(\alpha^2 + t^2) \d t - \frac{1}{m} \sum_{k=0}^{m-1} \ln\left(\alpha^2 + \left(\frac{2 k + 1}{2 m}\right)^2\right) = O\left(\frac{1}{m^2}\right).
\]

Ainsi, en notant $\beta_\alpha = \frac{1}{2} \int_0^1 \ln(\alpha^2 + t^2) \d t$,
\begin{align*}
\module{\omega_{2m}(\alpha \i)}
&= \beta_\alpha^{2m} \exp\left(2m \left(\frac{1}{2m} \ln\omega_{2m}(\alpha \i) - \beta_\alpha\right)\right).
\end{align*}

On obtient bien l'équivalent annoncé.

\todoinline{J'ai l'impression que $C = 1$. Il faudrait le vérifier\ldots}

\item D'après la question précédente,
\[
\ln(\beta_\alpha)
= \frac{1}{2} \int_0^1 \ln(\alpha^2 + t^2) \d t
= \frac{1}{2} \ln(1 + \alpha^2) - 1 + \alpha \arctan \frac{1}{\alpha}.
\]

Ainsi, $\lim\limits_{\alpha \to 0} \ln \beta_\alpha = -1$.

\item Ainsi, il existe un réel $\alpha_0$ tel que pour tout $0 < \alpha \leq \alpha_0$,
\[
\beta_\alpha < \frac{2}{\e}.
\]

Alors,
\[
\norme{f - P_{2m}}_\infty
\geq
\module{f(1) - P_{2m}(1)}
\sim \tilde{C} \left(\frac{2}{\e \beta_\alpha}\right)^{2m} \to +\infty.
\]
\end{reponses}
\end{solution}

\begin{remarque}
On pourra obtenir des résultats plus précis en étudiant les sujets \cite{st_cyr-1-commune-1993} ou \cite{x_ens-psi-1998}.
\end{remarque}
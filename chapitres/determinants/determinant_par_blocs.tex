\begin{exercice} 
À quelles conditions sur $A$, $B$, $C$ et $D$, des matrices carrées d'ordre $n$, a-t-on:
    $$
        \begin{vmatrix}
            A & B\\
            C & D\\
        \end{vmatrix} = \det(A \times D - B \times C) \text{ ?}
    $$ 
\end{exercice}
 
Rappel de cours:

\begin{prop}{}
    Soient $T_1, \dots, T_p$ des matrices carrées et 
    $$A \defeq
    \begin{pmatrix}
        T_1 & \star & \cdots & \star \\
        0 & T_2 & \ddots & \vdots \\
        \vdots & \ddots & \ddots & \star \\
        0 & \cdots & 0 & T_p
    \end{pmatrix}.
    $$
    Alors, 
    $$\det(A) = \prod_{i=1}^p \det(T_i).$$
\end{prop}

\begin{solution}
    D'après le rappel de cours, 
    $$\det(AD - BC) = \begin{vmatrix}
        AD - BC & \star \\
        0 & \I_n
    \end{vmatrix}.$$ 
    On \say{ remarque } que si les matrices $D$ et $C$ commutent et que si la matrice $D$ est inversible alors
    $$
    \begin{pmatrix}
        A & B \\
        C & D
    \end{pmatrix}
    \begin{pmatrix}
        D & 0 \\
        -C & \Inv{D}
    \end{pmatrix}
     = \begin{pmatrix}
         AD-BC & \star \\
         0 & \I_n
     \end{pmatrix}.
    $$
    ...
\end{solution}
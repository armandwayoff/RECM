\begin{defi}{Matrice tridiagonale}
    Soit $n \in \Ne$. Une matrice $M_n \defeq (m_{i,j}) \in \M_n(\K)$ est dite \emph{tridiagonale} si
    $$\forall (i,j) \in \llbracket 1, n \rrbracket^2,\ m_{i,j} = 0 \text{ si } |i-j| > 1.$$
\end{defi}

Pour mettre en avant la structure de la matrice $M_n$, on l'écrit sous la forme
$$
M_n = \begin{pmatrix}
a_1 & b_1 \\
c_1 & a_2 & b_2 \\
& c_2 & \ddots & \ddots \\
& & \ddots & \ddots & b_{n-1} \\
& & & c_{n-1} & a_n
\end{pmatrix}.
$$

% Code insipé de https://tex.stackexchange.com/questions/250788/block-matrices-with-latex
\newcommand{\mattrign}{
\left(\begin{gathered}
    \tikzpicture[every node/.style={anchor=south west}]
        \node[minimum width=1.5cm,minimum height=1cm] at (0.125,0.5) {\LARGE $M_{n-1}$};
        \node[minimum width=1.5cm,minimum height=0.5cm] at (0.75,0) {$c_{n-1}$};
        \node[minimum width=1cm,minimum height=0.5cm] at (2,0) {$a_n$};
        \node[minimum width=1cm,minimum height=1cm] at (2,0.25) {$b_{n-1}$};
        \draw (0,0.5) -- (2,0.5);
        \draw (2,0.5) -- (2,1.5);
        \draw[dashed] (2,0.5) -- (3,0.5);
        \draw[dashed] (2,0.5) -- (2,0);
    \endtikzpicture
    \end{gathered}\right)
}

\begin{remarque}
    Les matrices de la suite $(M_n)_{n \in \Ne}$ sont imbriquées les unes dans les autres de telle manière que pour $n > 1$,
    $$M_n = \mattrign.$$
\end{remarque}


\begin{prop}{}
    On note $D_n \defeq \det(M_n)$. La suite $(D_n)_{n \in \Ne}$ vérifie la relation de récurrence linéaire d'ordre deux 
    $$D_n = a_n D_{n-1} - b_{n-1}c_{n-1}D_{n-2}.$$
\end{prop}

\newcommand{\dettrign}{
\left|\begin{gathered}
    \tikzpicture[every node/.style={anchor=south west}]
        \node[minimum width=1.5cm,minimum height=1cm] at (0.125,0.5) {\LARGE $M_{n-1}$};
        \node[minimum width=1.5cm,minimum height=0.5cm] at (0.75,0) {$c_{n-1}$};
        \node[minimum width=1cm,minimum height=0.5cm] at (2,0) {$a_n$};
        \node[minimum width=1cm,minimum height=1cm] at (2,0.25) {$b_{n-1}$};
        \draw (0,0.5) -- (2,0.5);
        \draw (2,0.5) -- (2,1.5);
        \draw[dashed] (2,0.5) -- (3,0.5);
        \draw[dashed] (2,0.5) -- (2,0);
    \endtikzpicture
    \end{gathered}\right|
}

\newcommand{\dettrignmoinsun}{
\left|\begin{gathered}
    \tikzpicture[every node/.style={anchor=south west}]
        \node[minimum width=1.5cm,minimum height=1cm] at (0.125,0.5) {\LARGE $M_{n-2}$};
        \node[minimum width=0.5cm,minimum height=0.5cm] at (0,0) {$0$};
        \node[minimum width=0.5cm,minimum height=0.5cm] at (0.7,0) {$\cdots$};
        \node[minimum width=0.5cm,minimum height=0.5cm] at (1.5,0) {$0$};
        \node[minimum width=1cm,minimum height=0.5cm] at (2,0) {$c_{n-1}$};
        \node[minimum width=1cm,minimum height=1cm] at (2,0.25) {$b_{n-2}$};
        \draw (0,0.5) -- (2,0.5);
        \draw (2,0.5) -- (2,1.5);
        \draw[dashed] (2,0.5) -- (3,0.5);
        \draw[dashed] (2,0.5) -- (2,0);
    \endtikzpicture
    \end{gathered}\right|
}

\begin{preuve}
    \begin{align*}
        D_n &= \dettrign \\
            &= a_n \det(M_{n-1}) + (-1)^{n-1} b_{n-1} \dettrignmoinsun \\
            &= a_n D_{n-1} + (-1)^{n-1} b_{n-1} (-1)^{n-2} c_{n-1} \det(M_{n-2}) \\
        D_n &= a_n D_{n-1} - b_{n-1} c_{n-1} D_{n-2}
    \end{align*}
\end{preuve}

\begin{corol} \label{relation_det_toep_trig}
    Si les trois diagonales sont constantes, respectivement égales à $c, a$ et $b$ alors
    $$D_n = aD_{n-1} - bc D_{n-2}.$$
\end{corol}

\marginnote{
$$
\mathrm{Toep}_n (a,b,c) \defeq 
\begin{pmatrix}
a & b \\
c & a & b \\
& c & \ddots & \ddots \\
& & \ddots & \ddots & b \\
& & & c & a
\end{pmatrix}.
$$
}

Ces matrices sont des matrices de \textsc{Toeplitz} tridiagonales. Nous les noterons $\mathrm{Toep}_n (a,b,c)$ (notation non standart). 

\begin{prop}{}
    $$\Sp \big( \mathrm{Toep}_n(a,b,c) \big) = \left \{ a + 2 \sqrt{bc} \cos \left( \frac{k \pi}{n+1} \right) ,\ k \in \llbracket 1, n \rrbracket \right \}.$$
\end{prop}

\begin{preuve}
    \marginnote[0cm]{Deuxième réponse de \url{https://math.stackexchange.com/questions/955168/how-to-find-the-eigenvalues-of-tridiagonal-toeplitz-matrix}}
    Dans toute la suite on note $\Delta_n \defeq \det(\mathrm{Toep}_n(a,b,c))$. \\
    D'après le \nameref{relation_det_toep_trig}, (corollaire 3.1.)
    $$\Delta_n = a \Delta_{n-1} - bc \Delta_{n-2}$$
    avec $\Delta_0 = 1$ et $\Delta_1 = a$.
    On trouve 
    $$\Delta_n = \frac{1}{\sqrt{a^2 - 4bc}} \left[ \left( \frac{a + \sqrt{a^2 - 4bc}}{2} \right)^{n+1} - \left( \frac{a - \sqrt{a^2 - 4bc}}{2} \right)^{n+1} \right].$$
    Cette expression s'annule lorsque 
    $$\frac{a - \sqrt{a^2 - 4bc}}{a + \sqrt{a^2 - 4bc}} = \frac{1 - \sqrt{1 - \frac{4bc}{a^2}}}{1 + \sqrt{1 - \frac{4bc}{a^2}}}$$
    est une racine $(n+1)$-ème de l'unité que l'on note $\omega_k$. \\
    Alors
    $$\frac{4bc}{a^2} = 1 - \left( \frac{\omega_k - 1}{\omega_k + 1} \right)^2 = \frac{2 \omega_k}{(\omega_k + 1)^2}$$
    et 
    $$a = \pm \frac{\omega_k + 1}{2 \sqrt{\omega_k}}2 \sqrt{bc}.$$
    \say{ The coefficient in $\omega_k$ is real and can be written }
    $$\cos \left(\frac{k \pi}{n+1}\right).$$
    Enfin, en remplaçant $a$ par $a - \lambda$, on a montré que
    $$\lambda =a \pm 2 \sqrt{bc} \cos \left( \frac{k \pi}{n+1} \right).$$
\end{preuve}

\begin{methode}
    Le calcul du déterminant d'une matrice tridiagonale consiste à déterminer la relation de récurrence vérifiée par le déterminant puis à trouver l'expression du terme général à l'aide des formules sur les suites récurrentes linéaires d'ordre deux. 
\end{methode}

Mettons en pratique.

\begin{exercice}
Pour tout $x$ réel, déterminer le déterminant de taille $n$
    $$
        A_n(x) \defeq \begin{vmatrix}
            2x & 1 & & \\
            1 & 2x & \ddots\\
            & \ddots & \ddots & 1\\
            & & 1 & 2x
        \end{vmatrix}.
    $$   
\end{exercice}

\begin{solution}
    Soit $x \in \R$. D'après la remarque précédente, on obtient la relation de récurrence linéaire d'ordre 2
    $$A_n(x) = 2x A_{n-1}(x) - A_{n-2}(x)$$
    d'équation caractéristique 
    $$r^2 - 2xr + 1 = 0.$$
    Son déterminant a pour expression $4(x^2-1)$. \\
    À finir...
\end{solution}
 
\marginnote[-10cm]{
    \begin{kaobox}[frametitle=Relation de récurrence linéaire d'ordre 2]
    Cours Ch1 \cite{acamanes}. \\
    Soit $(a, b) \in \K^2$ tel que $b \not=0$. On considère les suites définies par le relation de récurrence
    $$u_{n+2} = a u_{n+1} + b u_n,\ \forall n \in \N.$$
    L'équation caractéristique $(\mathscr{E})$ associée est 
    $$r^2-ar-b=0.$$
    \begin{itemize}
        \item Si $(\mathscr{E})$ possède deux racines distinctes $r_1, r_2 \in \K^2$ alors il existe $(\lambda, \mu) \in \K^2$ tel que pour tout $n \in \N$,
        $$u_n = \lambda r_1^n + \mu r_2^n.$$
        \item Si $(\mathscr{E})$ possède une racine double $r_0 \K$ alors il existe $(\lambda, \mu) \in \K^2$ tel que pour tout $n \in \N$,
        $$u_n = (\lambda + n \mu) r_0^n.$$
        \item Si $(u_n)$ est une suite à valeurs réelles $(\mathscr{E})$ possède deux racines conjuguées distinctes $r_{1,2} = \rho \me^{\pm \mi \theta}$ alors il existe $(\lambda, \mu) \in \R^2$ tel que pour tout $n \in \N$,
        $$u_n = \rho^n \left( \lambda \cos(\theta n) + \mu \sin(\theta n) \right).$$
    \end{itemize}
    \end{kaobox}
}
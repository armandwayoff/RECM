\begin{defi}{Matrice de \textsc{Vandermonde}}
    Soit $(\alpha_1, \dots, \alpha_n)$ une famille de complexes. On définit la \emph{matrice de \textsc{Vandermonde}} de la famille $(\alpha_1, \dots, \alpha_n)$ par
    $$\Vandermonde(\alpha_1, \dots, \alpha_n) \defeq \begin{pmatrix}
    1 & \alpha_1 & \alpha_1^2 & \cdots & \alpha_1^{n-1} \\
    \vdots & \vdots & \vdots & \ddots & \vdots \\
    1 & \alpha_n & \alpha_n^2 & \cdots & \alpha_n^{n-1}
    \end{pmatrix}.$$
    On considère ici une matrice de \textsc{Vandermonde} carrée.
\end{defi}

\newcommand{\vandk}{
\left(\begin{gathered}
    \tikzpicture[every node/.style={anchor=south west}]
        \node[minimum width=2cm,minimum height=1.5cm] at (-0.16,0.625) {$\Vandermonde(\alpha_1, \dots, \alpha_{k-1})$};
        \node[minimum width=0.5cm,minimum height=0.5cm] at (0,0.05) {$1$};
        \node[minimum width=0.5cm,minimum height=0.5cm] at (0.4, 0) {$\alpha_{k}$};
        \node[minimum width=0.5cm,minimum height=0.5cm] at (0.95,0) {$\cdots$};
        \node[minimum width=0.5cm,minimum height=0.5cm] at (1.5,0) {$\alpha_{k}^{k-2}$};
        \node[minimum width=1cm,minimum height=0.5cm] at (2.5,0) {$\alpha_{k}^{k-1}$};
        \node[minimum width=1cm,minimum height=1cm] at (2.5,0.4) {$\alpha_{k-1}^{k-1}$};
        \node[minimum width=1cm,minimum height=1cm] at (2.5,1) {$\vdots$};
        \node[minimum width=1cm,minimum height=1cm] at (2.5,1.4) {$\alpha_1^{k-1}$};
        \draw (0,0.6) -- (2.5,0.6);
        \draw (2.5,0.6) -- (2.5,2.225);
        \draw[dashed] (2.5,0.6) -- (3.5,0.6);
        \draw[dashed] (2.5,0.6) -- (2.5,0);
    \endtikzpicture
    \end{gathered}\right)
}

\begin{remarque}
    Soit $(\alpha_1, \dots, \alpha_n)$ une famille de complexes. Les matrices de \textsc{Vandermonde} des familles $(\alpha_1, \dots, \alpha_k)$ pour $2 \leqslant k \leqslant n$ sont imbriquées les unes dans les autres de la manière suivante
    $$\Vandermonde(\alpha_1, \dots, \alpha_k) = \vandk.$$
\end{remarque}

\begin{prop}{Déterminant de \textsc{Vandermonde}} \labprop{determinant_vandermonde}
    $$\det \big( \Vandermonde(\alpha_1, \dots, \alpha_n) \big) = \prod_{1\leqslant i < j \leqslant n}(\alpha_j - \alpha_i).$$
\end{prop}

\newcommand{\detvandnplusun}{
\left|\begin{gathered}
    \tikzpicture[every node/.style={anchor=south west}]
        \node[minimum width=2cm,minimum height=1.5cm] at (0.05,0.625) {$\Vandermonde(\alpha_1, \dots, \alpha_{n})$};
        \node[minimum width=0.5cm,minimum height=0.5cm] at (0,0.05) {$1$};
        \node[minimum width=0.5cm,minimum height=0.5cm] at (0.3, 0) {$\alpha_{n+1}$};
        \node[minimum width=0.5cm,minimum height=0.5cm] at (1.05,0) {$\cdots$};
        \node[minimum width=0.5cm,minimum height=0.5cm] at (1.5,0) {$\alpha_{n+1}^{n-1}$};
        \node[minimum width=1cm,minimum height=0.5cm] at (2.5,0) {$\alpha_{n+1}^{n}$};
        \node[minimum width=1cm,minimum height=1cm] at (2.5,0.4) {$\alpha_{n}^{n}$};
        \node[minimum width=1cm,minimum height=1cm] at (2.5,1) {$\vdots$};
        \node[minimum width=1cm,minimum height=1cm] at (2.5,1.4) {$\alpha_1^{n}$};
        \draw (0,0.6) -- (2.5,0.6);
        \draw (2.5,0.6) -- (2.5,2.225);
        \draw[dashed] (2.5,0.6) -- (3.5,0.6);
        \draw[dashed] (2.5,0.6) -- (2.5,0);
    \endtikzpicture
    \end{gathered}\right|
}

\newcommand{\detvandpoly}{
\left|\begin{gathered}
    \tikzpicture[every node/.style={anchor=south west}]
        \node[minimum width=2cm,minimum height=1.5cm] at (-0.16,0.625) {$\Vandermonde(\alpha_1, \dots, \alpha_n)$};
        \node[minimum width=0.5cm,minimum height=0.5cm] at (0,0.05) {$1$};
        \node[minimum width=0.5cm,minimum height=0.5cm] at (0.4, 0) {$\alpha_{n+1}$};
        \node[minimum width=0.5cm,minimum height=0.5cm] at (0.95,0) {$\cdots$};
        \node[minimum width=0.5cm,minimum height=0.5cm] at (1.5,0) {$\alpha_{n+1}^{n-1}$};
        \node[minimum width=1cm,minimum height=0.5cm] at (2.5,0) {$P(\alpha_{n+1})$};
        \node[minimum width=1cm,minimum height=1cm] at (2.5,0.4) {$P(\alpha_n)$};
        \node[minimum width=1cm,minimum height=1cm] at (2.5,1) {$\vdots$};
        \node[minimum width=1cm,minimum height=1cm] at (2.5,1.4) {$P(\alpha_1)$};
        \draw (0,0.6) -- (2.5,0.6);
        \draw (2.5,0.6) -- (2.5,2.225);
        \draw[dashed] (2.5,0.6) -- (3.5,0.6);
        \draw[dashed] (2.5,0.6) -- (2.5,0);
    \endtikzpicture
    \end{gathered}\right|
}

\newcommand{\detvandzero}{
\left|\begin{gathered}
    \tikzpicture[every node/.style={anchor=south west}]
        \node[minimum width=2cm,minimum height=1.5cm] at (-0.16,0.625) {$\Vandermonde(\alpha_1, \dots, \alpha_n)$};
        \node[minimum width=0.5cm,minimum height=0.5cm] at (0,0.05) {$1$};
        \node[minimum width=0.5cm,minimum height=0.5cm] at (0.4, 0) {$\alpha_{n+1}$};
        \node[minimum width=0.5cm,minimum height=0.5cm] at (0.95,0) {$\cdots$};
        \node[minimum width=0.5cm,minimum height=0.5cm] at (1.5,0) {$\alpha_{n+1}^{n-1}$};
        \node[minimum width=1cm,minimum height=0.5cm] at (2.5,0) {$P(\alpha_{n+1})$};
        \node[minimum width=1cm,minimum height=1cm] at (2.5,0.4) {$0$};
        \node[minimum width=1cm,minimum height=1cm] at (2.5,1) {$\vdots$};
        \node[minimum width=1cm,minimum height=1cm] at (2.5,1.4) {$0$};
        \draw (0,0.6) -- (2.5,0.6);
        \draw (2.5,0.6) -- (2.5,2.225);
        \draw[dashed] (2.5,0.6) -- (3.5,0.6);
        \draw[dashed] (2.5,0.6) -- (2.5,0);
    \endtikzpicture
    \end{gathered}\right|
}

\begin{preuve}
    Nous allons raisonner par récurrence sur la taille de la matrice. Pour tout $n \in \Ne$ on pose
    $$\mathscr{P}_n: \text{\say{ Soit $(\alpha_1, \dots, \alpha_n) \in \C^n$, $\det \big(\Vandermonde(\alpha_1, \dots, \alpha_n)\big) = \prod\limits_{1\leqslant i < j \leqslant n}(\alpha_j - \alpha_i)$}}.$$
    \begin{itemize}
        \item[$\rhd$] L'initialisation pour $n = 1$ est triviale.
        \item[$\rhd$] Soit $n \in \Ne$. On suppose $\mathscr{P}_n$ vraie, montrons $\mathscr{P}_{n+1}$. \\ 
        Soit $(\alpha_1, \dots, \alpha_n, \alpha_{n+1})$ une famille de complexes. \\
        On pose $P(X) \defeq \prod\limits_{j=1}^n (X-\alpha_j) \defeq X^n + \sum\limits_{k=0}^{n-1} p_k X^k$. \\
        D'après les propriétés du déterminant, en ajoutant les $n$ premières colonnes respectivement multipliée par $p_k$ à la dernière, on obtient
        \begin{align*}
            \detvandnplusun &= \detvandpoly \\
            &= \detvandzero \\
            &= \det \big( \Vandermonde(\alpha_1, \dots, \alpha_n) \big) P(\alpha_{n+1}) \\
            \text{par hypothèse de récurrence } &= \prod\limits_{1\leqslant i < j \leqslant n}(\alpha_j - \alpha_i) \times \prod\limits_{j=1}^n (\alpha_{n+1}-\alpha_j) \\
            \det \big( \Vandermonde(\alpha_1, \dots, \alpha_{n+1}) \big) &= \prod_{1\leqslant i < j \leqslant n + 1}(\alpha_j - \alpha_i).
        \end{align*}
        La proprité est donc vraie au rang $n+1$ et on en déduit par récurrence qu'elle est vraie pour tout $n \in \Ne$.
    \end{itemize}
\end{preuve}

\begin{corol}
    La famille $(\alpha_1, \dots, \alpha_n)$ est libre si et seulement si le déterminant de sa matrice de \textsc{Vandermonde} est non nul.
\end{corol}

\begin{preuve} 
    D'après la \refprop{determinant_vandermonde}, la matrice de \textsc{Vandermonde} de la famille $(\alpha_1, \dots, \alpha_n)$ est inversible si et seulement si les $\alpha_i$ sont distincts deux à deux. 
\end{preuve}

\begin{exercice}
    \marginnote[0cm]{\cite{maths-france} Planche no 35. Déterminants}
    Pour $a \in \C$, on pose pour tout $x \in \R$, $f_a(x) \defeq \me^{ax}$. Soit $(a_1, \dots, a_n)$, $n$ complexes deux à deux distincts. Montrer que la famille de fonctions $(f_{a_k})_{1 \leqslant k \leqslant n}$ est libre.
\end{exercice}

\begin{solution}
    \item Soit $(\lambda_k)_{1 \leqslant k \leqslant n} \in \C^n$ tel que $\sum\limits_{k=1}^n \lambda_k f_{a_k} = 0$. Alors, 
    \begin{align*}
        & \forall p \in \llbracket 0, n-1 \rrbracket, \sum_{k=1}^n \lambda_k f^{(p)}_{a_k} = 0, \\
        \text{soit } & \forall p \in \llbracket 0, n-1 \rrbracket, \forall x \in \R, \sum_{k=1}^n \lambda_k a_k^p \me^{a_k x} = 0, \\
        \text{pour $x=0$ } & \forall p \in \llbracket 0, n-1 \rrbracket, \sum_{k=1}^n \lambda_k a_k^p = 0.
    \end{align*}
    Les $n$ dernières égalités écrites constituent un système $(\mathscr{S})$ de $n$ équations linéaires à $n$ inconnues $\lambda_1, \dots, \lambda_n$. Le déterminant de ce système est $\Vandermonde(a_1, \dots, a_n)$ et ce déterminant est non nul car les $a_k$ sont deux à deux distincts. Par suite, $(\mathscr{S})$ est un système de \textsc{Cramer} homogène. Le système $(\mathscr{S})$ admet donc l'unique solution $(\lambda_1, \dots, \lambda_n) = (0, \dots, 0)$ et on a ainsi montré la liberté de la famille $(f_{a_k})_{1 \leqslant k \leqslant n}$.
\end{solution}

\subsection{Inverse de la matrice de {\textsc{Vandermonde}}}

\begin{exercice}
    \marginnote[0cm]{\cite{exos_oraux} p.81}
    Soient $n \in \Ne$ et $(x_1, \dots, x_n)$ des complexes deux à deux distincts.
    \begin{enumerate}
        \item Montrer que l'application
        \begin{alignat*}{2}
            \varphi\ :\ \C_{n-1}[X]\ &\longrightarrow\ \C^n\\
            P\ &\longmapsto\ \big( P(x_1), \dots, P(x_n) \big)
        \end{alignat*}
        est un isomorphisme. Montrer que sa matrice dans les bases canoniques de départ et d'arrivée est 
        $$
        M_n(x_1, \dots, x_n) \defeq
        \begin{pmatrix}
            1 & x_1 & x_1^2 & \cdots & x_1^{n-1} \\
            1 & x_2 & x_2^2 & \cdots & x_2^{n-1} \\
            \vdots & \vdots & \vdots & & \vdots \\
            1 & x_n & x_n^2 & \cdots & x_n^{n-1}
        \end{pmatrix}.
        $$
        \item On note $\Lag_1(X), \dots, \Lag_n(X)$ les polynômes interpolateurs de \textsc{Lagrange} associés à $x_1, \dots, x_n$. Donner une relation entre les coefficients de $\Inv{M_n(x_1, \dots, x_n)}$ et ceux des polynômes $\Lag_i(X)$.
    \end{enumerate}
\end{exercice}

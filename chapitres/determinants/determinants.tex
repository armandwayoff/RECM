\chapter{Déterminants}
\labch{determinants}

\textsl{L'utilisation des matrices et des déterminants trouve son origine dans l'étude systématique des systèmes linéaires menée à partir du \textsc{xvii}$^\me$ siècle. Alors que \textsc{Leibniz} et \textsc{Mac Laurin} avaient déjà introduit les notations à indices et résolu les systèmes à deux ou trois inconnues \textsc{Cramer}, en 1754, comprend que les solutions d'un systèmes linéaire s'expriment comme le quotient de deux expressions polynomiales multilinéaires des coefficients du système. Ces expressions représentent des déterminants mais ces derniers, étudiés notamment par \textsc{Vandermonde} et \textsc{Laplace} ne sont définis alors que par récurrence sur la taille (autrement dit par le développement par rapport à une rangée). On doit également à \textsc{Laplace} l'interprétation du déterminant en termes de volume. Par le suite, au début du \textsc{xix}$^\me$ siècle \textsc{Gauss}, dans ses recherches sur les formes quadratiques, représente les changements de base dans $\R^3$ à l'aide de tableaux de nombres (les matrices) et introduit le produit de deux de ces tableaux pour obtenir la composée de deux changements de bases. Cela devait suggérer en 1812 à \textsc{Cauchy} la règle générale du produit de deux déterminants; il lui revient d'imposer la terminologie moderne.}

\begin{marginfigure}[-8.2cm]
    \caption*{\centering Interprétation géométrique du déterminant en dimension $2$}
   \begin{tikzpicture}[
    label/.style={black},
    vector/.style={ultra thick,-latex}
  ]

  \def\xmin{-1} \def\xmax{5}
  \def\ymin{-1} \def\ymax{5}
  \def\deltax{0.4} \def\deltay{0.3}
  \def\gridscale{3}
  
  \def\spacing{0.15} 

  \def\colTrigHB{red} \def\colTrigGD{blue} \def\colRect{green}
  \def\opac{0.3}
  \def\colx{violet} \def\coly{cyan}

  \begin{scope}
    \coordinate (origin) at (0,0);
    
    % repère droit
    % \draw [very thick,->] (\xmin,0) -- (\xmax,0);
    % \draw [very thick,->] (0,\ymin) -- (0,\ymax);
    \clip [draw] (\xmin,\ymin) rectangle (\xmax,\ymax);
    
    % quadrillage droit
    \draw[style=help lines] (\xmin-\xmax,\ymin-\ymax) grid[step=\deltay/\deltax] (-\xmin+\xmax,-\ymin+\ymax);
    
    % triangle haut
    \filldraw [\colTrigHB!50, fill opacity=\opac] (\deltax * \gridscale, \gridscale) -- (\gridscale + \deltax * \gridscale, \gridscale + \deltay * \gridscale) -- (\deltax * \gridscale, \gridscale + \deltay * \gridscale) -- cycle;
    
    % triangle bas
    \filldraw [\colTrigHB!50, fill opacity=\opac] (origin) -- (\gridscale, \deltay * \gridscale) -- (\gridscale, 0) -- cycle;
    
    % triangle gauche
    \filldraw [\colTrigGD!50, fill opacity=\opac] (origin) -- (0, \gridscale) -- (\deltax * \gridscale, \gridscale) -- cycle;
    
    % triangle droit
    \filldraw [\colTrigGD!50, fill opacity=\opac] (\gridscale, \deltay * \gridscale) -- (\gridscale + \deltax * \gridscale, \gridscale + \deltay * \gridscale) -- (\gridscale + \deltax * \gridscale, \deltay * \gridscale) -- cycle;
    
    % rectangle haut gauche
    \filldraw[\colRect!50,fill opacity=\opac] (\deltax * \gridscale, \gridscale) rectangle (0, \gridscale + \deltay * \gridscale);
    
    % rectangle bas droite
    \filldraw[\colRect!50,fill opacity=\opac] (\gridscale, 0) rectangle (\gridscale + \deltax * \gridscale, \deltay * \gridscale);
    
    % curly brackets
    \draw [thick, decorate,
    decoration = {calligraphic brace, mirror, raise=1pt, amplitude=5pt}] (origin) --  (\gridscale, 0) node[pos=0.5pt,below=5pt,black]{$\textcolor{\colx}{a}$};
    
    \draw [thick, decorate,
    decoration = {calligraphic brace, mirror, raise=1pt, amplitude=5pt}] (\gridscale, 0) -- (\gridscale + \deltax * \gridscale, 0) node[pos=0.5pt,below=5pt,black]{$\textcolor{\coly}{c}$};
    
    \draw [thick, decorate,
    decoration = {calligraphic brace, mirror, raise=1pt, amplitude=5pt}] (\gridscale + \deltax * \gridscale, 0) --  (\gridscale + \deltax * \gridscale, \deltay * \gridscale) node[pos=0.5pt,right=5pt,black]{$\textcolor{\colx}{b}$};
    
    \draw [thick, decorate,
    decoration = {calligraphic brace, mirror, raise=1pt, amplitude=5pt}] (\gridscale + \deltax * \gridscale, \deltay * \gridscale) -- (\gridscale + \deltax * \gridscale, \gridscale + \deltay * \gridscale) node[pos=0.5pt,right=5pt,black]{$\textcolor{\coly}{d}$};
    
    % aires
    \node at (\gridscale + \deltax * \gridscale / 2, \deltay * \gridscale / 2) {$\textcolor{black}{bc}$};
    \node at (\gridscale + \deltax * \gridscale / 1.5, \gridscale / 2 + \deltay * \gridscale / 2) {$\displaystyle \frac{dc}{2}$};
    \node at (\gridscale / 1.5, \deltay * \gridscale / 3) {$ab/2$};
    \node at (\gridscale / 2 + \deltax * \gridscale / 2, \gridscale / 2 + \deltay * \gridscale / 2) {\Huge $\mathcal{A}$};


    \pgftransformcm{1}{\deltay}{\deltax}{1}{\pgfpoint{0}{0}}

    % quadrillage oblique
    \draw[style=help lines,dashed] (\xmin-\xmax,\ymin-\ymax) grid[step=\gridscale] (-\xmin+\xmax,-\ymin+\ymax);

    % noeuds
    \foreach \x in {\xmin,...,\xmax}{
        \foreach \y in {\ymin,...,\ymax}{
            \node[draw,circle,inner sep=1pt,fill] at (\gridscale*\x,\gridscale*\y) {};
          }
      }
      
    \filldraw[fill=yellow,fill opacity=\opac] (origin) rectangle (\gridscale,\gridscale);

    \draw [vector, \colx] (origin) -- (\gridscale,0) node [label,right=0] {$\textcolor{\colx}{\vec{x}}$};
    \draw [vector, \coly] (origin) -- (0,\gridscale) node [label,above=0] {$\textcolor{\coly}{\vec{y}}$};
    
    % je rajoute le point de l'origine
    \node[draw,circle,inner sep=1pt,fill] at (origin) {};
    \end{scope}
\end{tikzpicture}
\end{marginfigure}

\marginnote[-1cm]{
    $$
    \textcolor{violet}{\vec{x} = 
    \begin{pmatrix} 
        a \\ 
        b 
    \end{pmatrix}}
    \text{ et }
    \textcolor{cyan}{\vec{y} = 
    \begin{pmatrix} 
        c \\ 
        d 
    \end{pmatrix}}
    $$
    \begin{align*}
        \mathcal{A} &= (a + c)(b + d) - 2 \left( \frac{ab}{2} + bc + \frac{dc}{2} \right) \\
        &= ad - bc \\
        \mathcal{A} &= 
        \begin{vmatrix}
            \textcolor{violet}{a} & \textcolor{cyan}{c} \\
            \textcolor{violet}{b} & \textcolor{cyan}{d}
        \end{vmatrix}
    \end{align*}
}

\newpage

\section{Déterminant tridiagonal}
\begin{defi}
    Soit $n \in \Ne$. Une matrice $M_n \defeq (m_{i,j}) \in \M_n(\K)$ est dite \emph{tridiagonale} si
    $$\forall (i,j) \in \llbracket 1, n \rrbracket^2,\ m_{i,j} = 0 \text{ si } |i-j| > 1.$$
\end{defi}

Pour mettre en avant la structure de la matrice $M_n$, on l'écrit sous la forme
$$
M_n = \begin{pmatrix}
a_1 & b_1 \\
c_1 & a_2 & b_2 \\
& c_2 & \ddots & \ddots \\
& & \ddots & \ddots & b_{n-1} \\
& & & c_{n-1} & a_n
\end{pmatrix}.
$$

% Code insipé de https://tex.stackexchange.com/questions/250788/block-matrices-with-latex
\newcommand{\mattrign}{
\left(\begin{gathered}
    \tikzpicture[every node/.style={anchor=south west}]
        \node[minimum width=1.5cm,minimum height=1cm] at (0.125,0.5) {\LARGE $M_{n-1}$};
        \node[minimum width=1.5cm,minimum height=0.5cm] at (0.75,0) {$c_{n-1}$};
        \node[minimum width=1cm,minimum height=0.5cm] at (2,0) {$a_n$};
        \node[minimum width=1cm,minimum height=1cm] at (2,0.25) {$b_{n-1}$};
        \draw (0,0.5) -- (2,0.5);
        \draw (2,0.5) -- (2,1.5);
        \draw[dashed] (2,0.5) -- (3,0.5);
        \draw[dashed] (2,0.5) -- (2,0);
    \endtikzpicture
    \end{gathered}\right)
}

\begin{remarque}
    Les matrices de la suite $(M_n)_{n \in \Ne}$ sont imbriquées les unes dans les autres de telle manière que pour $n > 1$,
    $$M_n = \mattrign.$$
\end{remarque}


\begin{prop}
    On note $D_n \defeq \det(M_n)$. La suite $(D_n)_{n \in \Ne}$ vérifie la relation de récurrence linéaire d'ordre deux 
    $$D_n = a_n D_{n-1} - b_{n-1}c_{n-1}D_{n-2}.$$
\end{prop}

\newcommand{\dettrign}{
\left|\begin{gathered}
    \tikzpicture[every node/.style={anchor=south west}]
        \node[minimum width=1.5cm,minimum height=1cm] at (0.125,0.5) {\LARGE $M_{n-1}$};
        \node[minimum width=1.5cm,minimum height=0.5cm] at (0.75,0) {$c_{n-1}$};
        \node[minimum width=1cm,minimum height=0.5cm] at (2,0) {$a_n$};
        \node[minimum width=1cm,minimum height=1cm] at (2,0.25) {$b_{n-1}$};
        \draw (0,0.5) -- (2,0.5);
        \draw (2,0.5) -- (2,1.5);
        \draw[dashed] (2,0.5) -- (3,0.5);
        \draw[dashed] (2,0.5) -- (2,0);
    \endtikzpicture
    \end{gathered}\right|
}

\newcommand{\dettrignmoinsun}{
\left|\begin{gathered}
    \tikzpicture[every node/.style={anchor=south west}]
        \node[minimum width=1.5cm,minimum height=1cm] at (0.125,0.5) {\LARGE $M_{n-2}$};
        \node[minimum width=0.5cm,minimum height=0.5cm] at (0,0) {$0$};
        \node[minimum width=0.5cm,minimum height=0.5cm] at (0.7,0) {$\cdots$};
        \node[minimum width=0.5cm,minimum height=0.5cm] at (1.5,0) {$0$};
        \node[minimum width=1cm,minimum height=0.5cm] at (2,0) {$c_{n-1}$};
        \node[minimum width=1cm,minimum height=1cm] at (2,0.25) {$b_{n-2}$};
        \draw (0,0.5) -- (2,0.5);
        \draw (2,0.5) -- (2,1.5);
        \draw[dashed] (2,0.5) -- (3,0.5);
        \draw[dashed] (2,0.5) -- (2,0);
    \endtikzpicture
    \end{gathered}\right|
}

\begin{preuve}
    \begin{align*}
        D_n &= \dettrign \\
            &= a_n \det(M_{n-1}) + (-1)^{n-1} b_{n-1} \dettrignmoinsun \\
            &= a_n D_{n-1} + (-1)^{n-1} b_{n-1} (-1)^{n-2} c_{n-1} \det(M_{n-2}) \\
        D_n &= a_n D_{n-1} - b_{n-1} c_{n-1} D_{n-2}
    \end{align*}
\end{preuve}

\begin{corol} \label{relation_det_toep_trig}
    Si les trois diagonales sont constantes, respectivement égales à $c, a$ et $b$ alors
    $$D_n = aD_{n-1} - bc D_{n-2}.$$
\end{corol}

\marginnote{
$$
\mathrm{Toep}_n (a,b,c) \defeq 
\begin{pmatrix}
a & b \\
c & a & b \\
& c & \ddots & \ddots \\
& & \ddots & \ddots & b \\
& & & c & a
\end{pmatrix}.
$$
}

Ces matrices sont des matrices de \textsc{Toeplitz} tridiagonales. Nous les noterons $\mathrm{Toep}_n (a,b,c)$ (notation non standart). 

\begin{prop}
    $$\Sp(\mathrm{Toep}_n(a,b,c)) = \left \{ a + 2 \sqrt{bc} \cos \left( \frac{k \pi}{n+1} \right) ,\ k \in \llbracket 1, n \rrbracket \right \}.$$
\end{prop}

\begin{preuve}
    \marginnote[0cm]{Deuxième réponse de \url{https://math.stackexchange.com/questions/955168/how-to-find-the-eigenvalues-of-tridiagonal-toeplitz-matrix}}
    Dans toute la suite on note $\Delta_n \defeq \det(\mathrm{Toep}_n(a,b,c))$. \\
    D'après le \nameref{relation_det_toep_trig}, (corollaire 3.1.)
    $$\Delta_n = a \Delta_{n-1} - bc \Delta_{n-2}$$
    avec $\Delta_0 = 1$ et $\Delta_1 = a$.
    On trouve 
    $$\Delta_n = \frac{1}{\sqrt{a^2 - 4bc}} \left[ \left( \frac{a + \sqrt{a^2 - 4bc}}{2} \right)^{n+1} - \left( \frac{a - \sqrt{a^2 - 4bc}}{2} \right)^{n+1} \right].$$
    Cette expression s'annule lorsque 
    $$\frac{a - \sqrt{a^2 - 4bc}}{a + \sqrt{a^2 - 4bc}} = \frac{1 - \sqrt{1 - \frac{4bc}{a^2}}}{1 + \sqrt{1 - \frac{4bc}{a^2}}}$$
    est une racine $(n+1)$-ème de l'unité que l'on note $\omega_k$. \\
    Alors
    $$\frac{4bc}{a^2} = 1 - \left( \frac{\omega_k - 1}{\omega_k + 1} \right)^2 = \frac{2 \omega_k}{(\omega_k + 1)^2}$$
    et 
    $$a = \pm \frac{\omega_k + 1}{2 \sqrt{\omega_k}}2 \sqrt{bc}.$$
    \say{ The coefficient in $\omega_k$ is real and can be written }
    $$\cos \left(\frac{k \pi}{n+1}\right).$$
    Enfin, en remplaçant $a$ par $a - \lambda$, on a montré que
    $$\lambda =a \pm 2 \sqrt{bc} \cos \left( \frac{k \pi}{n+1} \right).$$
\end{preuve}

\begin{methode}
    Le calcul du déterminant d'une matrice tridiagonale consiste à déterminer la relation de récurrence vérifiée par le déterminant puis à trouver l'expression du terme général à l'aide des formules sur les suites récurrentes linéaires d'ordre deux. 
\end{methode}

Mettons en pratique.

\begin{exercice}
Pour tout $x$ réel, déterminer le déterminant de taille $n$
    $$
        A_n(x) \defeq \begin{vmatrix}
            2x & 1 & & \\
            1 & 2x & \ddots\\
            & \ddots & \ddots & 1\\
            & & 1 & 2x
        \end{vmatrix}.
    $$   
\end{exercice}

\begin{solution}
    Soit $x \in \R$. D'après la remarque précédente, on obtient la relation de récurrence linéaire d'ordre 2
    $$A_n(x) = 2x A_{n-1}(x) - A_{n-2}(x)$$
    d'équation caractéristique 
    $$r^2 - 2xr + 1 = 0.$$
    Son déterminant a pour expression $4(x^2-1)$. \\
    À finir...
\end{solution}
 
\marginnote[-10cm]{
    \begin{kaobox}[frametitle=Relation de récurrence linéaire d'ordre 2]
    Cours Ch1 \cite{acamanes}. \\
    Soit $(a, b) \in \K^2$ tel que $b \not=0$. On considère les suites définies par le relation de récurrence
    $$u_{n+2} = a u_{n+1} + b u_n,\ \forall n \in \N.$$
    L'équation caractéristique $(\mathscr{E})$ associée est 
    $$r^2-ar-b=0.$$
    \begin{itemize}
        \item Si $(\mathscr{E})$ possède deux racines distinctes $r_1, r_2 \in \K^2$ alors il existe $(\lambda, \mu) \in \K^2$ tel que pour tout $n \in \N$,
        $$u_n = \lambda r_1^n + \mu r_2^n.$$
        \item Si $(\mathscr{E})$ possède une racine double $r_0 \K$ alors il existe $(\lambda, \mu) \in \K^2$ tel que pour tout $n \in \N$,
        $$u_n = (\lambda + n \mu) r_0^n.$$
        \item Si $(u_n)$ est une suite à valeurs réelles $(\mathscr{E})$ possède deux racines conjuguées distinctes $r_{1,2} = \rho \me^{\pm \mi \theta}$ alors il existe $(\lambda, \mu) \in \R^2$ tel que pour tout $n \in \N$,
        $$u_n = \rho^n \left( \lambda \cos(\theta n) + \mu \sin(\theta n) \right).$$
    \end{itemize}
    \end{kaobox}
}

\section{Déterminant des \texorpdfstring{$|a_i - a_j|$}{|a_i - a_j|}}
\begin{exercice}
    \marginnote[0cm]{Source : \cite{exos_oraux} p. 71}
    Soient $n \geqslant 2$ et $(a_0, \dots, a_n) \in \R^{n+1}$. Calculer le déterminant de la matrice dont l'élément ligne $i$, colonne $j$ est $|a_{i-1} - a_{j-1}|$.
\end{exercice}

\section{Déterminant de \textsc{Cauchy}}
\begin{defi}
    Le déterminant de \textsc{Cauchy} est un déterminant de taille $n$ et de terme général $\frac{1}{a_i+b_j}$, où les complexes $(a_1, \dots, a_n)$ et $(b_1, \dots, b_n)$ sont tels que pour tout $(i, j)$, $a_i+b_j \not= 0$.
    $$\Cauchy_n \defeq \begin{vmatrix}
        \frac{1}{a_1+b_1} & \frac{1}{a_1+b_2} & \cdots & \frac{1}{a_1+b_n} \\
        \frac{1}{a_2+b_1} & \frac{1}{a_2+b_2} & \cdots & \frac{1}{a_2+b_n} \\
        \vdots & \vdots & & \vdots \\
        \frac{1}{a_n+b_1} & \frac{1}{a_n+b_2} & \cdots & \frac{1}{a_n+b_n}
    \end{vmatrix}.$$
\end{defi}

\begin{prop}
    Le déterminant de \textsc{Cauchy} a pour expression
    $$\Cauchy_n = \frac{\prod\limits_{i<j}(a_j-a_i) \prod\limits_{i<j}(b_j-b_i)}{\prod\limits_{i,j}(a_i+b_j)}.$$
\end{prop}
        
\begin{preuve}
    On raisonne par récurrence sur $n$. \\
    Premièrement, faisons apparaître une ligne de 1 en multipliant toutes les colonnes par $a_n+b_j$. \\
    Par $n$-linéarité du déterminant,
    $$\Cauchy_n = \frac{1}{\prod\limits_{i=1}^{n}(a_n + b_i)} \begin{vmatrix}
        \frac{a_n+b_1}{a_1+b_1} & \cdots & \frac{a_n+b_n}{a_1+b_n} \\
        \vdots & & \vdots \\
        \frac{a_n+b_1}{a_{n-1}+b_1} & \cdots & \frac{a_n+b_n}{a_{n-1}+b_n} \\
        1 & \cdots & 1
    \end{vmatrix}.$$
    
    Ensuite, pour faire apparaître des 0 à la fin des $(n-1)$ premières colonnes, soustrayons la dernière colonne à toutes les autres. 
    
    $$
        \forall (i,j) \in \llbracket 1, n-1 \rrbracket^2, \quad \frac{a_n+b_i}{a_i+b_j} - \frac{a_n+b_n}{a_i+b_n} = \frac{(a_n - a_i)(b_n-b_j)}{(a_i + b_j)(a_i + b_n)}
    $$
    $$\Cauchy_n = \frac{1}{\prod\limits_{i=1}^{n}(a_n + b_i)} \begin{vmatrix}
        \frac{\textcolor{red}{(a_n - a_1)}\textcolor{blue}{(b_n-b_1)}}{(a_1 + b_1) \textcolor{green}{(a_1 + b_n)}} & \cdots & \frac{\textcolor{red}{(a_n - a_1)}\textcolor{blue}{(b_n-b_{n-1})}}{(a_1 + b_{n-1})\textcolor{green}{(a_1 + b_n)}} & \frac{a_n+b_n}{a_1+b_n} \\
        \vdots & & \vdots & \vdots \\
        \frac{\textcolor{red}{(a_n - a_{n-1})}\textcolor{blue}{(b_n-b_1)}}{(a_{n-1} + b_1)\textcolor{green}{(a_{n-1} + b_n)}} & \cdots & \frac{\textcolor{red}{(a_n - a_{n-1})}\textcolor{blue}{(b_n-b_{n-1})}}{(a_{n-1} + b_{n-1})\textcolor{green}{(a_{n-1} + b_n)}} & \frac{a_n+b_n}{a_{n-1}+b_n} \\
        0 & \cdots & 0 & 1
    \end{vmatrix}.$$
        
    Factorisons par les termes communs aux lignes et aux colonnes.
        
    $$\Cauchy_n =  \frac{\textcolor{red}{\prod\limits_{i=1}^{n-1} (a_n - a_i)} \textcolor{blue}{\prod\limits_{i = 1}^{n-1}(b_n - b_i)}}{\prod\limits_{i=1}^{n}(a_n + b_i) \textcolor{green}{\prod\limits_{i = 1}^{n - 1} (a_i + b_n)}} \begin{vmatrix}
        \frac{1}{a_1+b_1} & \cdots & \frac{1}{a_1+b_{n-1}} & \bigstar \\
        \vdots & & \vdots & \vdots \\
        \frac{1}{a_{n-1}+b_1} & \cdots & \frac{1}{a_{n-1}+b_{n-1}} & \bigstar \\
        0 & \cdots & 0 & 1
    \end{vmatrix}.$$
        
    Finalement, en développant par rapport à la dernière ligne on trouve la relation de récurrence:
    $$\Cauchy_n =  \frac{1}{a_n + b_n} \Bigg( \prod\limits_{i = 1}^{n-1}\frac{(a_n - a_i)(b_n - b_i)}{(a_n + b_i)(a_i + b_n)} \Bigg) \Cauchy_{n-1}$$
    
    Comme $\Cauchy_1 = \frac{1}{a_1+b_1}$, on obtient par récurrence
    $$\Cauchy_n = \frac{\prod\limits_{i<j}(a_j-a_i) \prod\limits_{i<j}(b_j-b_i)}{\prod\limits_{i,j}(a_i+b_j)}.$$
\end{preuve}

\begin{remarque}
    On peut aussi écrire
    $$\Cauchy_n = \frac{\Vandermonde(a_1, \dots, a_n) \Vandermonde(b_1, \dots, b_n)}{\prod\limits_{i,j}(a_i+b_j)}$$
    en notant $\Vandermonde(\alpha_1, \dots, \alpha_n)$ le déterminant de la matrice de \textsc{Vandermonde} de la famille $(\alpha_1, \dots, \alpha_n)$.
\end{remarque}

\begin{methode}
    La méthode pour calculer ce genre de déterminants \say{ compliqués } est toujours la même: 
    \begin{enumerate}
        \item Raisonner par récurrence sur la taille de la matrice.
        \item Faire des opérations élémentaires sur les lignes/colonnes pour faire apparaître une ligne/colonne composée de 1. 
        \item Soustraire les lignes/colonnes de manière à obtenir une ligne/colonne presque composée uniquement de 0 sauf pour un seul coefficient. 
        \item Développer par rapport à cette dernière et obtenir une relation de récurrence sur le déterminant. 
    \end{enumerate}
\end{methode}

\begin{defi} \label{matrice_hilbert}
    Une matrice de \textsc{Hilbert} est une matrice carrée de terme général
    $$\Hilb_{i,j} \defeq {\frac {1}{i+j-1}}.$$
\end{defi}

\begin{remarque}
    Le déterminant d'une matrice de \textsc{Hilbert} est un cas particulier du déterminant de \textsc{Cauchy}.
\end{remarque}

\begin{remarque}
    \cite{exos_oraux} p. 82 \\
    Les matrices de \textsc{Hilbert} sont souvent utilisées pour tester des ordinateurs ou leurs programmes censés inverser des matrices. Le fait que leur déterminant soit rapidement très petit rend l'inversion de la matrice numériquement difficile et c'est ce qui fait la qualité du test. 
\end{remarque}


\section{Déterminant par blocs}
\begin{exercice} 
À quelles conditions sur $A$, $B$, $C$ et $D$, des matrices carrées d'ordre $n$, a-t-on:
    $$
        \begin{vmatrix}
            A & B\\
            C & D\\
        \end{vmatrix} = \det(A \times D - B \times C) \text{ ?}
    $$ 
\end{exercice}
 
Rappel de cours:

\begin{prop}{}
    Soient $T_1, \dots, T_p$ des matrices carrées et 
    $$A \defeq
    \begin{pmatrix}
        T_1 & \star & \cdots & \star \\
        0 & T_2 & \ddots & \vdots \\
        \vdots & \ddots & \ddots & \star \\
        0 & \cdots & 0 & T_p
    \end{pmatrix}.
    $$
    Alors, 
    $$\det(A) = \prod_{i=1}^p \et(T_i).$$
\end{prop}

\begin{solution}
    D'après le rappel de cours, $\det(AD - BC) = \begin{vmatrix}
        AD - BC & \star \\
        0 & \I_n
    \end{vmatrix}$.  
    On \say{ remarque } que si les matrices $D$ et $C$ commutent et que si la matrice $D$ est inversible alors
    $$
    \begin{pmatrix}
        A & B \\
        C & D
    \end{pmatrix}
    \begin{pmatrix}
        D & 0 \\
        -C & \Inv{D}
    \end{pmatrix}
     = \begin{pmatrix}
         AD-BC & \star \\
         0 & \I_n
     \end{pmatrix}.
    $$
    ...
\end{solution}

\section{Dérivée du déterminant}
\begin{exercice}
    \marginnote[0cm]{\cite{maths-france} Planche no 4. Révision algèbre linéaire.
Déterminants}
    \begin{enumerate}
        \item Soient $a_{i,j}$, $1 \leqslant i, j \leqslant n$, $n^2$ fonctions dérivables sur $\R$ à valeurs dans $\C$. Soit $d \defeq \det(a_{i,j})_{1 \leqslant i, j \leqslant n}$. Montrer que $d$ est dérivable sur $\R$ et calculer $d'$.
        \item Application: calculer
        $
        d_n(x) \defeq
        \begin{vmatrix}
            x+1 & 1 & \cdots & 1 \\
            1 & \ddots & \ddots & \vdots \\
            \vdots & \ddots & \ddots & 1 \\
            1 & \cdots & 1 & x+1
        \end{vmatrix}
        $.
    \end{enumerate}
\end{exercice}

\section{Déterminant de \textsc{Hürwitz}}
\begin{exercice}
    \marginnote[0cm]{\cite{fmaalouf}}
    Calculer le déterminant de la matrice $A \in \M_n(\R)$ de la forme
    $$
    \begin{pmatrix}
        r_1 & a & \cdots & a \\
        b & r_2 & \ddots & \vdots \\
        \vdots & \ddots & \ddots & a \\
        b & \cdots & b & r_n
    \end{pmatrix}.
    $$
\end{exercice}

\section{Matrice circulante modulo \texorpdfstring{$p$}{p}}

\begin{exercice}
    \marginnote[0cm]{\cite{fmaalouf}}
    Soit $p$ premier et $(a_0, \dots, a_{p-1}) \in \Z^p$. Montrer que
    $$
    \begin{vmatrix}
        a_0 & a_1 & a_2 & \cdots & a_{p-1} \\
        a_{p-1} & a_0 & a_1 & \cdots & a_{p-2} \\
        a_{p-2} & a_{p-1} & a_0 & \cdots & a_{p-3} \\
        \vdots & \vdots & \vdots & \ddots & \vdots \\
        a_1 & a_2 & a_3 & \cdots & a_0
    \end{vmatrix}
    \equiv a_0 + \cdots + a_{p-1}\ [p].
    $$
\end{exercice}
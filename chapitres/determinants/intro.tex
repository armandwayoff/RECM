\textsl{L'utilisation des matrices et des déterminants trouve son origine dans l'étude systématique des systèmes linéaires menée à partir du \textsc{xvii}$^\e$ siècle. Alors que \textsc{Leibniz} et \textsc{Mac Laurin} avaient déjà introduit les notations à indices et résolu les systèmes à deux ou trois inconnues \textsc{Cramer}, en 1754, comprend que les solutions d'un systèmes linéaire s'expriment comme le quotient de deux expressions polynomiales multilinéaires des coefficients du système. Ces expressions représentent des déterminants mais ces derniers, étudiés notamment par \textsc{Vandermonde} et \textsc{Laplace} ne sont définis alors que par récurrence sur la taille (autrement dit par le développement par rapport à une rangée). On doit également à \textsc{Laplace} l'interprétation du déterminant en termes de volume. Par le suite, au début du \textsc{xix}$^\e$ siècle \textsc{Gauss}, dans ses recherches sur les formes quadratiques, représente les changements de base dans $\R^3$ à l'aide de tableaux de nombres (les matrices) et introduit le produit de deux de ces tableaux pour obtenir la composée de deux changements de bases. Cela devait suggérer en 1812 à \textsc{Cauchy} la règle générale du produit de deux déterminants; il lui revient d'imposer la terminologie moderne.} \\

\cite{objectif_agregation} p. 184.

\begin{marginfigure}[-7.6cm]
    \caption*{\centering Interprétation géométrique du déterminant en dimension $2$}
   \begin{tikzpicture}[
    label/.style={black},
    vector/.style={ultra thick,-latex}
  ]

  \def\xmin{-1} \def\xmax{5}
  \def\ymin{-1} \def\ymax{5}
  \def\deltax{0.4} \def\deltay{0.3}
  \def\gridscale{3}
  
  \def\spacing{0.15} 

  \def\colTrigHB{red} \def\colTrigGD{blue} \def\colRect{green}
  \def\opac{0.3}
  \def\colx{violet} \def\coly{cyan}

  \begin{scope}
    \coordinate (origin) at (0,0);
    
    % repère droit
    % \draw [very thick,->] (\xmin,0) -- (\xmax,0);
    % \draw [very thick,->] (0,\ymin) -- (0,\ymax);
    \clip [draw] (\xmin,\ymin) rectangle (\xmax,\ymax);
    
    % quadrillage droit
    \draw[style=help lines] (\xmin-\xmax,\ymin-\ymax) grid[step=\deltay/\deltax] (-\xmin+\xmax,-\ymin+\ymax);
    
    % triangle haut
    \filldraw [\colTrigHB!50, fill opacity=\opac] (\deltax * \gridscale, \gridscale) -- (\gridscale + \deltax * \gridscale, \gridscale + \deltay * \gridscale) -- (\deltax * \gridscale, \gridscale + \deltay * \gridscale) -- cycle;
    
    % triangle bas
    \filldraw [\colTrigHB!50, fill opacity=\opac] (origin) -- (\gridscale, \deltay * \gridscale) -- (\gridscale, 0) -- cycle;
    
    % triangle gauche
    \filldraw [\colTrigGD!50, fill opacity=\opac] (origin) -- (0, \gridscale) -- (\deltax * \gridscale, \gridscale) -- cycle;
    
    % triangle droit
    \filldraw [\colTrigGD!50, fill opacity=\opac] (\gridscale, \deltay * \gridscale) -- (\gridscale + \deltax * \gridscale, \gridscale + \deltay * \gridscale) -- (\gridscale + \deltax * \gridscale, \deltay * \gridscale) -- cycle;
    
    % rectangle haut gauche
    \filldraw[\colRect!50,fill opacity=\opac] (\deltax * \gridscale, \gridscale) rectangle (0, \gridscale + \deltay * \gridscale);
    
    % rectangle bas droite
    \filldraw[\colRect!50,fill opacity=\opac] (\gridscale, 0) rectangle (\gridscale + \deltax * \gridscale, \deltay * \gridscale);
    
    % curly brackets
    \draw [thick, decorate,
    decoration = {calligraphic brace, mirror, raise=1pt, amplitude=5pt}] (origin) --  (\gridscale, 0) node[pos=0.5pt,below=5pt,black]{$\textcolor{\colx}{a}$};
    
    \draw [thick, decorate,
    decoration = {calligraphic brace, mirror, raise=1pt, amplitude=5pt}] (\gridscale, 0) -- (\gridscale + \deltax * \gridscale, 0) node[pos=0.5pt,below=5pt,black]{$\textcolor{\coly}{c}$};
    
    \draw [thick, decorate,
    decoration = {calligraphic brace, mirror, raise=1pt, amplitude=5pt}] (\gridscale + \deltax * \gridscale, 0) --  (\gridscale + \deltax * \gridscale, \deltay * \gridscale) node[pos=0.5pt,right=5pt,black]{$\textcolor{\colx}{b}$};
    
    \draw [thick, decorate,
    decoration = {calligraphic brace, mirror, raise=1pt, amplitude=5pt}] (\gridscale + \deltax * \gridscale, \deltay * \gridscale) -- (\gridscale + \deltax * \gridscale, \gridscale + \deltay * \gridscale) node[pos=0.5pt,right=5pt,black]{$\textcolor{\coly}{d}$};
    
    % aires
    \node at (\gridscale + \deltax * \gridscale / 2, \deltay * \gridscale / 2) {$\textcolor{black}{bc}$};
    \node at (\gridscale + \deltax * \gridscale / 1.5, \gridscale / 2 + \deltay * \gridscale / 2) {$\displaystyle \frac{dc}{2}$};
    \node at (\gridscale / 1.5, \deltay * \gridscale / 3) {$ab/2$};
    \node at (\gridscale / 2 + \deltax * \gridscale / 2, \gridscale / 2 + \deltay * \gridscale / 2) {\Huge $\mathcal{A}$};


    \pgftransformcm{1}{\deltay}{\deltax}{1}{\pgfpoint{0}{0}}

    % quadrillage oblique
    \draw[style=help lines,dashed] (\xmin-\xmax,\ymin-\ymax) grid[step=\gridscale] (-\xmin+\xmax,-\ymin+\ymax);

    % noeuds
    \foreach \x in {\xmin,...,\xmax}{
        \foreach \y in {\ymin,...,\ymax}{
            \node[draw,circle,inner sep=1pt,fill] at (\gridscale*\x,\gridscale*\y) {};
          }
      }
      
    \filldraw[fill=yellow,fill opacity=\opac] (origin) rectangle (\gridscale,\gridscale);

    \draw [vector, \colx] (origin) -- (\gridscale,0) node [label,right=0] {$\textcolor{\colx}{\vec{x}}$};
    \draw [vector, \coly] (origin) -- (0,\gridscale) node [label,above=0] {$\textcolor{\coly}{\vec{y}}$};
    
    % je rajoute le point de l'origine
    \node[draw,circle,inner sep=1pt,fill] at (origin) {};
    \end{scope}
\end{tikzpicture}
\end{marginfigure}

\marginnote[-1cm]{
    $$
    \textcolor{violet}{\vec{x} = 
    \begin{pmatrix} 
        a \\ 
        b 
    \end{pmatrix}}
    \text{ et }
    \textcolor{cyan}{\vec{y} = 
    \begin{pmatrix} 
        c \\ 
        d 
    \end{pmatrix}}
    $$
    \begin{align*}
        \mathcal{A} &= (a + c)(b + d) - 2 \left( \frac{ab}{2} + bc + \frac{dc}{2} \right) \\
        &= ad - bc \\
        \mathcal{A} &= 
        \begin{vmatrix}
            \textcolor{violet}{a} & \textcolor{cyan}{c} \\
            \textcolor{violet}{b} & \textcolor{cyan}{d}
        \end{vmatrix}
    \end{align*}
}
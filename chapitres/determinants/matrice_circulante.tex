\begin{tcolorbox}
Une \href{https://fr.wikipedia.org/wiki/Matrice_circulante}{\emph{matrice circulante}} est une matrice carrée dans laquelle on passe d'une ligne à la suivante par permutation circulaire des coefficients:
$$
\mathrm{C}(c_0, \dots, c_{n-1})=
\begin{pmatrix}
c_0 & c_1 & c_2 & \cdots & c_{n-1} \\
c_{n-1} & c_0 & c_1 & \cdots & c_{n-2} \\
c_{n-2} & c_{n-1} & c_0 & \cdots & c_{n-3} \\
\vdots & \vdots & \vdots & \ddots & \vdots \\
c_1 & c_2 & c_3 & \cdots & c_0
\end{pmatrix}.
$$
\end{tcolorbox}

\begin{remarque}
    Une matrice circulante est un cas particulier de \href{https://fr.wikipedia.org/wiki/Matrice_de_Toeplitz}{matrice de \textsc{Toeplitz}}.
\end{remarque}

\begin{prop}
On pose $\omega = \me^{\mi \frac{2 \pi}{n}}$.
    $$\Sp(\mathrm{C}(c_0, \dots, c_{n-1})) = \left \{ \sum_{j=0}^{n-1} c_j \omega^j,\ k \in \llbracket 1, n-1 \rrbracket \right \}.$$
\end{prop}

\begin{preuve}
    Pour alléger les notations, on pose $\mathrm{C} := \mathrm{C}(c_0, \dots, c_{n-1})$. \\
    On pose 
    $$
    \mathrm{J} = 
    \begin{pmatrix}
    0 & 1 & 0 & \cdots & 0 \\
    \vdots  &   & \ddots & \\
    0 & & & & 1 \\
    1 & 0 & \cdots & \cdots & 0
    \end{pmatrix}
    $$
    
    (faire une remarque sur la structure de la matrice J et que J$^n = \I_n$) 
    
    $$\mathrm{C} = \sum_{k=0}^{n-1} c_k \mathrm{J}^k = \mathrm{P}_{\mathrm{C}}(\mathrm{J}).$$
    
    En développant par rapport à la première colonne, 
    \begin{align*}
        \chi_{\mathrm{J}}(X) &= 
        \begin{vmatrix}
            X & -1 & 0 & \cdots & 0 \\
            \vdots  &   & \ddots & \\
            0 & & & & -1 \\
            -1 & 0 & \cdots & \cdots & X
        \end{vmatrix} \\
        &= X \times X^{n-1} + (-1) \times (-1)^{n+1} \times (-1)^{n-1} \\
        \chi_{\mathrm{J}}(X) &= X^n-1. 
    \end{align*}
    Le polynôme caractéristique de $\mathrm{J}$ est scindé à racines simples sur $\C$ donc $\mathrm{J}$ est diagonalisable et en posant $\omega = \me^{\mi \frac{2 \pi}{n}}$, 
    $$\Sp(\mathrm{J}) = \mathbb{U}_n = \left \{ \omega^k,\ k \in \llbracket 0, n-1 \rrbracket \right \}.$$
    Ainsi, $\mathrm{J}$ est semblable à la matrice 
    $$
    \begin{pmatrix}
    1 & & & & \\
    & \omega & & & \\
    & & \omega^2 & & \\
    & & & \ddots & \\
    & & & & \omega^{n-1} \\
    \end{pmatrix}
    $$
    et donc comme $\mathrm{C} = \mathrm{P}_{\mathrm{C}}(\mathrm{J})$, la matrice $\mathrm{C}$ est semblable à  la matrice
    $$
    \begin{pmatrix}
    \mathrm{P}(1) & & & & \\
    & \mathrm{P}(\omega) & & & \\
    & & \mathrm{P}(\omega^2) & & \\
    & & & \ddots & \\
    & & & & \mathrm{P}(\omega^{n-1}) \\
    \end{pmatrix}.
    $$
    Ainsi, $\Sp(\mathrm{C}) = \left \{ \mathrm{P}(\omega^k),\ k \in \llbracket 0, n-1 \rrbracket \right \}$.
\end{preuve}

\begin{corol}
    $$\det(\mathrm{C}(c_0, \dots, c_{n-1})) = \prod_{j=0}^{n-1} \left( \sum_{k=0}^{n-1} c_k \exp \left( \mi \frac{2kj \pi}{n} \right) \right).$$
\end{corol}

\begin{preuve}
    Le déterminant d'une matrice est égal au produit de ses valeurs propres.
\end{preuve}

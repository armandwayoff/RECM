\begin{tcolorbox}
Une \href{https://fr.wikipedia.org/wiki/Matrice_circulante}{\emph{matrice circulante}} est une matrice carrée dans laquelle on passe d'une ligne à la suivante par permutation circulaire des coefficients. 
$$
    C = \begin{pmatrix}
        c_0 & c_1 & c_2 & \cdots & c_{n-1} \\
        c_{n-1} & c_0 & c_1 & \cdots & c_{n-2} \\
        c_{n-2} & c_{n-1} & c_0 & \cdots & c_{n-3} \\
        \vdots & \vdots & \vdots & \ddots & \vdots \\
        c_1 & c_2 & c_3 & \cdots & c_0
    \end{pmatrix}.
$$
Une matrice circulante est un cas particulier de \href{https://fr.wikipedia.org/wiki/Matrice_de_Toeplitz}{matrice de \textsc{Toeplitz}}.  
\end{tcolorbox}
\begin{itemize}
    \item Poser la matrice:
    $$
    \boxed{
        J=\begin{pmatrix}
            0 & \dots & 0 & 1\\
            1 & & (0) & 0\\
            & \ddots & & \vdots\\
            (0) & & 1 & 0
         \end{pmatrix}
        }
    $$
    La matrice circulante est un polynôme en $J$. 
    $$C = \sum_{k = 0}^{n-1} c_k J^k.$$
    L'ensemble des matrices circulantes est que l'algèbre commutative des polynômes en $J$.
    \item Déterminer le polynôme caractéristique de $J$ et en déduire ses valeurs propres. On remarque que $J^n = \I_n$. Les valeurs propres de $J$ sont les racines $n$-ièmes de l'unité (cf. \textsc{Ensam} 24 de \cite{exos_oraux} p. 112).
    \item $J$ est diagonalisable...
    $$\det(A) = \prod_{j=0}^{n-1} \left( \sum_{k=0}^{n-1} \alpha_k \exp \left( \mi \frac{2kj \pi}{n} \right) \right).$$
\end{itemize}
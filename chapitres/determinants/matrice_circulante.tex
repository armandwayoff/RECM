\begin{defi}{Matrice circulante}
Une \emph{matrice circulante} est une matrice carrée dans laquelle on passe d'une ligne à la suivante par permutation circulaire des coefficients:
$$
\mathrm{C}(c_0, \dots, c_{n-1}) \defeq
\begin{pmatrix}
c_0 & c_1 & c_2 & \cdots & c_{n-1} \\
c_{n-1} & c_0 & c_1 & \cdots & c_{n-2} \\
c_{n-2} & c_{n-1} & c_0 & \cdots & c_{n-3} \\
\vdots & \vdots & \vdots & \ddots & \vdots \\
c_1 & c_2 & c_3 & \cdots & c_0
\end{pmatrix}.
$$
\end{defi}

\begin{remarque}
    Une matrice circulante est un cas particulier de matrice de \textsc{Toeplitz}.
\end{remarque}

\begin{prop}
    $$\Sp(\mathrm{C}(c_0, \dots, c_{n-1})) = \left \{ \sum_{k=0}^{n-1} c_k \exp \left( \mi \frac{2kj \pi}{n} \right),\ k \in \llbracket 1, n-1 \rrbracket \right \}.$$
\end{prop}

\begin{preuve}
    Pour alléger les notations, on pose $\mathrm{C} \defeq \mathrm{C}(c_0, \dots, c_{n-1})$. \\
    Nous cherchons à déterminer le spectre de la matrice $\mathrm{C}$ ce qui ne semble pas évident au premier abord (on se rend rapidement compte que le calcul direct du polynôme caractéristique n'est pas une bonne approche). \\
    Notre démarche va consister à exprimer la matrice $\mathrm{C}$ comme le polynôme d'une matrice simple dont on peut calculer le spectre. On en déduira ensuite le spectre de la matrice $\mathrm{C}$. \\
    On pose 
    $$
    \mathrm{J}_n \defeq \mathrm{C}(0, 1, 0, \dots 0) = 
    \begin{pmatrix}
    0 & 1 & 0 & \cdots & 0 \\
    0 & 0 & \ddots & & \vdots \\
    \vdots & & \ddots & \ddots & 0 \\
    0 & & & 0 & 1 \\
    1 & 0 & \cdots & 0 & 0
    \end{pmatrix} \in \M_n(\R).
    $$
    
\marginnote[0cm]{
    \note Exemple; les puissances successives de $\mathrm{J}_3:$
    $$\mathrm{J_3} = 
    \begin{pmatrix}
        0 & 1 & 0 \\
        0 & 0 & 1 \\
        1 & 0 & 0
    \end{pmatrix}
    $$
    $$\mathrm{J_3^2} = 
    \begin{pmatrix}
        0 & 0 & 1 \\
        1 & 0 & 0 \\
        0 & 1 & 0
    \end{pmatrix}
    $$
    $$\mathrm{J_3^3} = 
    \begin{pmatrix}
        1 & 0 & 0 \\
        0 & 1 & 0 \\
        0 & 0 & 1
    \end{pmatrix}
    $$
}
    On remarque que la matrice $\mathrm{C}$ s'écrit comme une combinaison linéaire des puissances successives de la matrice $\mathrm{J}_n$ \note. Plus précisément, 
    $$\mathrm{C} = \sum_{k=0}^{n-1} c_k \mathrm{J}_n^k \defeq \mathrm{P}_{\mathrm{C}}(\mathrm{J}_n).$$
    Calculons le polynôme caractéristique de la matrice $\mathrm{J}_n$. Par définition, $\chi_{\mathrm{J}_n} = \det(X \I_n - \mathrm{J}_n)$ soit 
    $$
        \chi_{\mathrm{J}_n}(X) = 
        \begin{vmatrix}
        X & -1 & 0 & \cdots & 0 \\
        0 & X & \ddots & & \vdots \\
        \vdots & & \ddots & \ddots & 0 \\
        0 & & & X & -1 \\
        -1 & 0 & \cdots & 0 & X
        \end{vmatrix}.
    $$
    En développant par rapport à la première colonne on trouve
    \begin{align*}
        \chi_{\mathrm{J}_n}(X) &= X \times X^{n-1} + (-1) \times (-1)^{n+1} \times (-1)^{n-1} \\
        \chi_{\mathrm{J}_n}(X) &= X^n-1.
    \end{align*}
    
    Le polynôme caractéristique de $\mathrm{J}_n$ est scindé à racines simples sur $\C$ donc $\mathrm{J}_n$ est diagonalisable et en posant $\omega \defeq \me^{\mi \frac{2 \pi}{n}}$, 
    $$\Sp(\mathrm{J}_n) = \mathbb{U}_n = \left \{ \omega^k,\ k \in \llbracket 0, n-1 \rrbracket \right \}.$$
    Ainsi, $\mathrm{J}_n$ est semblable à la matrice $\Diag(1, \omega, \omega^2, \dots, \omega^{n-1})$ et donc comme $\mathrm{C} = \mathrm{P}_{\mathrm{C}}(\mathrm{J}_n)$, par linéarité, la matrice $\mathrm{C}$ est semblable à  la matrice $\Diag \left(\mathrm{P}_\mathrm{C}(1), \mathrm{P}_\mathrm{C}(\omega), \mathrm{P}_\mathrm{C}(\omega^2), \dots,\mathrm{P}_\mathrm{C}(\omega^{n-1}) \right)$. \\
    Ainsi, $\Sp(\mathrm{C}) = \left \{ \mathrm{P}_\mathrm{C}(\omega^k),\ k \in \llbracket 0, n-1 \rrbracket \right \}$.
\end{preuve}    

\marginnote[-3cm]{
    $$A = P D \Inv{P}$$
    $$A^k = P D^k \Inv{P}$$
    $$Q(X) \defeq \sum_{k=0}^n q_k X^k$$
    $$Q(A) = P Q(D) \Inv{P}$$
}

\begin{corol}
    $$\det(\mathrm{C}(c_0, \dots, c_{n-1})) = \prod_{j=0}^{n-1} \left( \sum_{k=0}^{n-1} c_k \exp \left( \mi \frac{2kj \pi}{n} \right) \right).$$
\end{corol}

\begin{preuve}
    Le déterminant d'une matrice est égal au produit de ses valeurs propres.
\end{preuve}

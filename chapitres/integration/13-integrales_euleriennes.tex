\section{Intégrales eulériennes}\label{secinteuleriennes}

\marginnote[0mm]{Source : \href{https://lescoursdemathsdepjh.monsite-orange.fr/file/7bcfcf82249b1046f185e9a3495845cd.pdf}{Fonction eulériennes -- Pierre-Jean \textsc{Hormière}}}
De premières tentatives pour définir la factorielle de valeurs non entières remontent à \textsc{Stirling} et Daniel \textsc{Bernoulli}. Dans une lettre à Christian \textsc{Goldbach} du 13 octobre 1729, \textsc{Euler} découvre (ou invente ?) une fonction de variable réelle prolongeant de manière naturelle la fonction $n!$. D'abord introduite comme limite de produits, cette fonction fut plus tard présentée sous forme intégrale et reliée à des fonctions voisines.

Les fonctions eulériennes sont les plus importantes \say{ fonctions spéciales } de l'analyse classique, réelle et complexe. \textsc{Legendre} les a nommées, classifiées et étudiées. Elles ont aussi été étudiées par \textsc{Gauss}, \textsc{Binet}, \textsc{Plana}, \textsc{Malmsten}, \textsc{Raabe}, \textsc{Weierstrass}, \textsc{Hankel}, H. \textsc{Bohr}, \textsc{Mollerup}, \textsc{Artin}, \ldots

\todoinline{Dans la phrase suivante, tu te rappelles à quoi tu pensais pour les nombreuses façons ? Il me semble que le prolongement analytique est unique, non ? Mes cours d'analyse complexe sont un peu loin...}
\todoarmand{Malheureusement, la source dont vient le texte n'est plus accessible, je vais poursuivre mes recherches.}
\todoarmand{
\url{https://people.math.osu.edu/gautam.42/S20/DavisGammaFunction.pdf} : un texte passionnant qui raconte en détail toute l'aventure intellectuelle autour de la fonction Gamma et qui explique (p. 18-20) la manière dont on peut construire des pseudo fonctions Gamma qui vérifient l'essentiel des propriétés d'un prolongement de la fonction $n!$.}

Il y a bien des façons de prolonger la fonction $n!$ au domaine réel, même en se limitant aux fonctions continues. Une idée naturelle est de partir de la formule $\displaystyle n! = \int_{0}^{+ \infty} t^n \e^{-t} \d t$. Cette forme intégrale de la factorielle suggère de considérer la fonction $\displaystyle F(x)=\int_{0}^{+ \infty} t^x \e^{-t} \d t$. Cette fonction, définie sur $\interoo{-1}{+\infty}$, prolonge intelligemment la factorielle, en ce sens qu'elle possède des propriétés nombreuses et cohérentes. Par commodité, on considère plutôt $\displaystyle \Gamma(x) = \int_{0}^{+ \infty} t^{x-1} \e^{-t} \d t$.
\todoarmand{Introduction à réécrire à partir de la ressource ci-dessus.}

\subsection{Fonction Gamma d'\textsc{Euler}}

%\begin{marginfigure}[3cm]
%    \begin{tikzpicture}[]

\begin{axis}[
xmin = -4.9, xmax = 5.1, 
%ymin = -3.5, ymax = 3.5,  
restrict y to domain=-6:6,
axis lines = middle,
axis line style={-latex},  
xlabel={$x$}, 
ylabel={$\Gamma(x)$},
%enlarge x limits={upper={val=0.2}},
enlarge y limits=0.05,
x label style={at={(ticklabel* cs:1.00)}, inner sep=5pt, anchor=north},
y label style={at={(ticklabel* cs:1.00)}, inner sep=2pt, anchor=south east},
]

\addplot[color=red, samples=222, smooth, 
domain = 0:5] gnuplot{gamma(x)};

\foreach[evaluate={\N=\n-1}] \n in {0,...,-5}{%
\addplot[color=red, samples=555, smooth,  
domain = \n:\N] gnuplot{gamma(x)};
%
\addplot [domain=-6:6, samples=2, densely dashed, thin] (\N, x);
}%
\end{axis}
\end{tikzpicture}
%    \caption*{\centering Graphe de la fonction Gamma}
%\end{marginfigure}

\begin{defi}[Fonction Gamma d'\textsc{Euler}]
    La \emph{fonction Gamma d'\textsc{Euler}} est définie par: 
    $$\Gamma(x) \defeq \int_{0}^{+\infty} t^{x-1} \e^{-t} \d t.$$
\end{defi}

\begin{remarque}
    À un changement de variable près, la fonction $\Gamma$ est la \nameref{transformee_laplace} de la fonction $t \mapsto t^x$. 
\end{remarque} 

\begin{theo}
\begin{itemize}
\item La fonction $\Gamma$ est définie sur $\Rpe$.

\item Pour tout $x > 0$, $\Gamma(x+1) = x\Gamma(x)$.

En particulier, pour tout $n \in \N$, $\Gamma(n+1) = n!$. 
\end{itemize}
\end{theo}

\begin{demo}
\begin{itemize}
\item On pose $\fonctionligne[f_x]{t}{t^{x - 1} \e^{-t}}$.

La fonction $f_x$ est continue sur $\R_+^\star$. Comme $f_x$ est à valeurs positives, son intégrabilité est équivalente à la convergence de l'intégrale.

Comme $f_x(t) = o_{+\infty}\mathopen{}\left(\frac{1}{t^2}\right)$, d'après les théorèmes de comparaison aux intégrales de \textsc{Riemann}, $f_x$ est intégrable sur $\interfo{1}{+\infty}$.

Comme $f_x(t) \sim_0 \frac{1}{t^{1-x}}$, alors $f_x$ est intégrable sur $\interof{0}{1}$ si et seulement si $x > 0$.

Finalement, $\Gamma$ est définie sur $\R_+^\star$.

\item Les fonctions $t \mapsto \e^{-t}$ et $t \mapsto t^x$ sont de classe $\mathscr{C}^1$ sur $\R_+^\star$. D'après les croissances comparées, $\lim_{t\to0} t^x \e^{-t} = \lim_{t\to+\infty} t^x \e^{-t} = 0$. D'après la formule d'intégration par parties généralisée,
\[
\int_0^{+\infty} t^x \e^{-t} \d t = x \int_0^{+\infty} t^{x-1} \e^{-t} \d t
\]
et $\Gamma(x+1) = x \Gamma(x)$.

\item On remarque que $\Gamma(1) = 1$ et on montre par récurrence que $\Gamma(n+1) = n!$.
\end{itemize}
\end{demo}

% La démonstration est donnée dans \ref{prolongementFonctionGamma}.

% Cette fonction, introduite en 1729 par le mathématicien suisse, prolonge la fonction factorielle à l'ensemble des réels strictement positifs.

\todoinline{Si on rappelle l'IPP ici, peut être faut-il le faire avec d'autres théorèmes ?}
\todoarmand{Oui, on pourrait faire une compilation des théorèmes utilisés sur l'ensemble des chapitres sur l'intégration.}

\marginnote[-5cm]{
    \begin{theo}[Théorème (Intégration par parties généralisées)]
        Source : \cite{acamanes}\\
        Soient $f$ et $g$ deux fonctions de classe $\mathscr{C}^1$ sur $I$. Si la fonction $fg$ a une limite finie en $a$ et en $b$, alors les intégrales
        $$\int_a^b f'(t)g(t) \d t \text{ et } \int_a^b f(t) g'(t) \d t$$
        sont de même nature. Si ces quantités sont convergentes, en notant
        \begin{align*}
            [f(t)g(t)]_a^b \\
            = \lim_{x \to b^-} \big(f(x)g(x)\big) - \lim_{x \to a^+} \big(f(x)g(x)\big),
        \end{align*}
        on obtient la relation
        $$\int_a^b f'(t) g(t) \d t$$
        $$ = \left[f(t)g(t)\right]_a^b - \int_a^b f(t) g'(t) \d t.$$
    \end{theo}
}

\begin{theo}[Régularité]
La fonction $\Gamma$ est de classe $\mathscr{C}^\infty$ sur $\Rpe$. De plus,
\[
\forall k \in \N,\quad \forall x \in \R_+^\star,\quad \Gamma^{(k)}(x) = \int_{0}^{+\infty} (\ln t)^k t^{x-1} \e^{-t} \d t.
\]
\end{theo}

\begin{demo}
\begin{itemize}
\item \textbf{Étude de la continuité.} Soit $0 < a < A$. On pose $\fonctionligne[f]{(x, t)}{t^{x-1} \e^{-t}}$.
\begin{itemize}
\item Soit $t \in \Rpe$. Alors, $f(\,\cdot\,, t)$ est continue sur $\interff{a}{A}$.

\item Soit $x \in \interff{a}{A}$. Alors, $f(x, \cdot)$ est continue sur $\Rpe$.

\item Pour tout $(x, t) \in \interff{a}{A} \times \Rpe$,
\[
\abs{f(x, t)} \leqslant \phi(t) = 
\begin{cases}
\frac{\e^{-t}}{t^{1-a}} &\text{ si } t \in \interof{0}{1},\\
t^{A-1} \e^{-t} &\text{ si } t \in \interfo{1}{+\infty}
\end{cases}
\]
La fonction $\phi$ est continue sur $\Rpe$. De plus, $\phi(t) \sim_0 \frac{1}{t^{1 - a}}$ et $\phi(t) = o_{+\infty}\mathopen{}\left(\frac{1}{t^2}\right)$, donc $\phi$ est intégrable sur $\Rpe$.
\end{itemize}
Finalement, d'après le théorème de continuité sous le signe intégral, $\Gamma$ est continue sur $\interff{a}{A}$ pour tous $0 < a < A$ et $\Gamma$ est donc continue sur $\Rpe$.

\item On applique le théorème de dérivation sous le signe intégral sur l'intervalle $\interff{a}{A}$:
\begin{itemize}
\item la fonction $\fonctionligne[f(\,\cdot\,, t)]{t}{t^{x - 1} \e^{-t}}$ est de classe $\mathscr{C}^k$ sur $\interff{a}{A}$.

\item la dérivée partielle $\partial_1^j f(x, t) = \ln(t)^j t^{x - 1} \e^{-t}$ et $t \mapsto \partial_1^j f(x, t)$ est continue sur $\Rpe$.

\item pour tout $x \in \interff{a}{A}$,
\[
\abs{\partial_1^j f(x, t)} \leqslant
\begin{cases}
\abs{\ln(t)}^j \e^{-t} t^{a-1} &\text{ si } t < 1 \\
\abs{\ln(t)}^j \e^{-t} t^{A-1} &\text{ si } t \geqslant 1
\end{cases}
\]

En notant $\phi_j$ cette fonction,
\begin{itemize}
\item $\phi_j$ est continue sur $\Rpe$.
\item $\phi_j(t) = o_{+\infty}\mathopen{}\left(\frac{1}{t^2}\right)$ et $\phi_j(t) = o_0(t^{x/2-1})$. Ainsi, $\phi_j$ est intégrable sur $\Rp$.
\end{itemize}
\end{itemize}
Finalement, $\Gamma$ est de classe $\mathscr{C}^\infty$ et
\[
\Gamma^{(j)}(x) = \int_0^{+\infty} (\ln t)^j t^{x-1} \e^{-t} \d t.
\]
\end{itemize}
\end{demo}

% \begin{elem_preuve}
    % Utiliser une domination locale sur un segment $[a, A] \subset \R_+^\star$ par la fonction:
    % $$\varphi_k:t \mapsto 
    % \begin{cases}
        % |\ln t |^k \e^{-t} t^{a-1} & \text{si } t \in ]0, 1], \\
        % |\ln t |^k \e^{-t} t^{A-1} & \text{si } t > 1.
    % \end{cases}
    % $$
% \end{elem_preuve}

\begin{exercice}
\marginnote[0cm]{Source : \cite{fmaalouf}}
\begin{enumerate}
\item Montrer que $\Gamma(x) \isEquivTo{0^+} \frac{1}{x}$.

\item Montrer que $\Gamma$ est convexe.
% et étudier ses variations.
\end{enumerate}
\end{exercice}

\begin{demo}
\begin{enumerate}
\item Comme $\Gamma$ est continue en $1$, 
\[
\lim_{x\to0} \Gamma(x + 1) = \Gamma(1) = \int_0^{+\infty} \e^{-t} \d t = 1.
\]

De plus, pour tout $x > 0$, $\Gamma(x + 1) = x \Gamma(x)$. Ainsi,
\[
\Gamma(x) \sim_0 \frac{1}{x}.
\]

\item D'après l'étude de régularité précédente, pour tout $x > 0$,
\[
\Gamma''(x) = \int_0^{+\infty} (\ln t)^2 t^{x-1} \e^{-t} \d t.
\]
Ainsi, $\Gamma'' > 0$ et $\Gamma$ est convexe.
\end{enumerate}
\end{demo}

\todoinline{On peut montrer que $\Gamma$ est l'unique fonction telle que $\ln \Gamma$ est convexe, $\Gamma(1) = 1$ et $\Gamma(x + 1) = x \Gamma(x)$. On le met ici ? (Source : Rudin)\\
On a également $\gamma = -\Gamma'(1)$. On doit pouvoir trouver une preuve faisable.}

\marginnote[0cm]{Patrice Lassère, p. 210}
\begin{exercice}
    Montrer que 
    \[
    \Gamma'(1) = \lim_{x \to \infty} \int_0^n \left(1 - \frac{t}{n} \right)^n \ln(t) \d t
    \]
    pour en déduire que 
    \[
    \Gamma'(1) = - \gamma \defeq \lim_{n \to \infty} \left(1 + \frac{1}{2} + \cdots + \frac{1}{n} - \ln(n) \right).
    \]
    La constante $\gamma$ est la constante d'\textsc{Euler}. On pourra effectuer le changement de variable $u = 1 - \frac{t}{n}$.
\end{exercice}

%-----------
\subsection{Fonction $\Gamma$ et intégrale de \textsc{Gauss}}
\marginnote[0cm]{\href{sec:intGauss}{\faLink~Intégrale de \textsc{Gauss}}}

\begin{theo}
$\Gamma\mathopen{}\left(\frac{1}{2}\right) = \sqrt{\pi}$
et pour tout $n \in \Ne$,
$\Gamma\mathopen{}\left(n + \frac{1}{2}\right) = \frac{(2n)!}{4^n n!} \sqrt{\pi}$.
\end{theo}
% \begin{exercice}
    \marginnote[0cm]{\cite{fmaalouf}}
    % Caculer $\Gamma \left( \frac{1}{2} \right)$, puis $\Gamma \left( n + \frac{1}{2} \right)$ pour tout $n \in \Ne$.
% \end{exercice}

\todoinline{Le marginnote suivant ne compile pas chez moi}
% \marginnote[0cm]{Source : \href{https://fr.wikipedia.org/wiki/Intégrale_de_Gauss#Calcul_de_l'intégrale_de_Gauss}{Intégrale de Gauss -- \textsf{wikipedia.org}}}

\begin{demo}
On utilise le changement de variable $\mathscr{C}^1$ et bijectif $u \mapsto u^2$. Alors,
\begin{align*}
\Gamma\mathopen{}\left( \frac{1}{2} \right)
&= \int_0^{+ \infty} t^{\frac{1}{2}-1} \e^{-t} \d t\\
&= 2 \int_0^{+ \infty} \e^{-u^2} \d u \\
&= \int_{- \infty}^{+ \infty} \e^{-u^2} \d u\\
&= \sqrt{\pi},
\end{align*}
d'après les résultats sur l'intégrale de \textsc{Gauss}.

Le second point s'obtient en utilisant l'égalité
\[
\Gamma\left(n + \frac{1}{2}\right) = \frac{n + 2}{2} \Gamma\left(n - 1 + \frac{1}{2}\right).
\]
\end{demo}

%-----------
\subsection{Fonction bêta}

% \todoinline{J'ai changé les $I_{p,q}$ en $B(p, q)$ pour coller à la définition classique de la fonction Bêta}

\begin{defi}[Fonction bêta]
Pour tout $(p,q) \in \N^2$, on note
$$B(p, q) \defeq \int_{0}^{1} t^p (1-t)^q \d t.$$
\end{defi}

\begin{theo}[Expression factiorelle de la fonction bêta]
Pour tout $(p,q) \in \N^2$,
    $$B(p, q) = \frac{p! q!}{(p + q + 1)!}.$$
\end{theo}

\marginnote[0cm]{
Sources : 
\begin{itemize}
    \item \href{https://fr.wikipedia.org/wiki/Intégrale_d'Euler}{Intégrale d'\textsc{Euler} -- \textsf{wikipedia.org}}
    \item \cite{calcul_infinitesimal} Chapitre IV, 3 Intégrales eulériennes, page 125.
\end{itemize}
}

\begin{exercice}
\begin{enumerate}
\item Déterminer $B(p, 0)$.

\item Pour tout $(p, q) \in \N \times \Ne$, montrer que
\[
B(p, q) = \frac{q}{p + 1} B(p+1, q-1).
\]

\item Conclure.
\end{enumerate}
\end{exercice}

\begin{demo}
\begin{enumerate}
\item On a immédiatement $B(p, 0) = \int_0^1 t^p \d t = \frac{1}{p + 1}$.

\item On pose $\fonctionligne[u]{t}{\frac{1}{p+1} t^{p+1}}$ et $\fonctionligne[v]{t}{(1-t)^q}$, toutes deux de classe $\mathscr{C}^1$ sur $\interff{0}{1}$. Alors, d'après la formule d'intégration par parties, 
\begin{align*}
B(p,q) &= \left[ \frac{1}{p+1}t^{p+1} \times (1-t)^q \right]_0^1 + \frac{q}{p+1} \int_0^1 t^{p+1} (1-t)^{q-1} \d t \\
B(p,q) &= \frac{q}{p+1} B(p+1, q-1).
\end{align*}

On en déduit que 
\begin{align*}
B(p, q)
&= \frac{q}{p+1} \times \frac{q-1}{p+2} \times \cdots \times \frac{1}{p+q} B(p+q, 0) \\
&= \frac{p! q!}{(p + q + 1)!}.
\end{align*}
\end{enumerate}
\end{demo}

\marginnote[0cm]{\todoinline{D'après Oral Centrale 2016}}
\begin{exercice}%
Soit $p \in \Ne$. On dispose de $p$ urnes numérotées de $1$ à $p$. Chaque urne contient $p$ boules et, pour tout $i \in \interent{1}{p}$, l'urne numéro $i$ contient $i$ boules noires et $p - i$ boules blanches. On effectue l'expérience suivante : choisir au hasard une urne puis effectuer des tirages avec remise dans l'urne choisie. On note, pour $n \in \Ne$, $A_n$ l'événement \textit{on a effectué $2 n$ tirages et obtenu le même nombre de boules blanches que de noires}.

\begin{enumerate}
\item Exprimer $\mathbf{P}(A_n)$ sous forme d'une somme.

\item On note $b_{n,p}$ la probabilité que la prochaine boule tirée soit blanche sachant que $A_n$ est réalisé. Exprimer~$b_{n,p}$.
\item Calculer $\lim\limits_{p\to+\infty} b_{n,p}$.
\end{enumerate}
\end{exercice}

\begin{demo}
\begin{enumerate}
\item En utilisant la formule des probabilités totales,
\begin{align*}
\mathbf{P}(A_n) &= \sum_{i=1}^p \mathbf{P}(A_n | U_i) \mathbf{P}(U_i) \\
&= \sum_{i=1}^n \binom{2n}{n} \left(\frac{p-i}{p}\right)^n \left(\frac{i}{p}\right)^n.
\end{align*}

\item En utilisant le même raisonnement que précédemment,
\[
b_{n,p} = \frac{\frac{1}{p} \sum\limits_{i=1}^p \left(1 - \frac{i}{p}\right)^{n+1} \left(\frac{i}{p}\right)^n}{\frac{1}{p} \sum\limits_{i=1}^p \left(1 - \frac{i}{p}\right)^{n} \left(\frac{i}{p}\right)^n} \cdot \frac{\binom{2n+1}{n+1}}{\binom{2n}{n}}.
\]

\item En utilisant les limites des sommes de \textsc{Riemann},
\[
\lim\limits_{p\to+\infty} b_{n,p} = \frac{2n+1}{n+1} \cdot \frac{\int_0^1 (1 - x)^{n+1} x^n \d x}{\int_0^1 (1 - x)^n x^n \d x}.
\]
Cette limite vaut donc
\[
\frac{2n+1}{n+1} \cdot \frac{n! (n+1)!}{(2n+2)!} \cdot \frac{(2n+1)!}{n! n!}  = \frac{2n+1}{2n+2}.
\]
\end{enumerate}
\end{demo}

%-----------
\subsection{Des fonctions Bêta et des sommes}


\begin{exercice}
    \marginnote[0cm]{Source : Clémentine \textsc{Portal} (PCSI1, Collège Stanislas) feuille d'exo n°7, exo 11}
Soit $(p, q) \in \N^2$.

Déterminer une expression sans signe somme de $\sum\limits_{k=0}^q \binom{q}{k} \frac{(-1)^k}{p+k+1}$.
\end{exercice}

\begin{demo}
En utilisant la formule du binôme de \textsc{Newton} puis la linéarité de l'intégrale,
\begin{align*}
B(p, q)
&= \int_0^1 t^p (1 - t)^q \d t\\
&= \int_0^1 t^p \left(\sum_{k=0}^q \binom{q}{k} (-1)^k t^k\right) \d t\\
&= \sum_{k=0}^q \binom{q}{k} (-1)^k \left(\int_0^1 t^{p + k} \d t\right)\\
&= \sum_{k=0}^q \binom{q}{k} \frac{(-1)^k}{p + k + 1}.
\end{align*}
Ainsi,
\[
\sum_{k=0}^q \binom{q}{k} \frac{(-1)^k}{p + k + 1}
= \frac{p! q!}{(p + q + 1)!}
= \frac{\binom{p+q+1}{p}^{-1}}{p + q + 1}.
\]
\end{demo}

%---------------

\marginnote[0cm]{\todoinline{Source : Oral Mines 2018}}
\begin{exercice}
Montrer que la série $\sum B(n,n)$ converge et déterminer sa somme.
\end{exercice}

\begin{solution}
D'après la définition, $0 \leqslant B(n, n) \leqslant \frac{1}{4^n}$. Ainsi, $\sum B(n, n)$ converge.

Comme $t^p (1 - t)^q \geqslant 0$ pour $t \in \interff{0}{1}$, d'après le théorème d'interversion série / intégrale,
\begin{align*}
\sum_{n=0}^{+\infty} B(n, n)
&= \int_0^1 \left(\sum_{n=0}^{+\infty} t^n (1 - t)^n\right) \d t \\
&= \int_0^1 \frac{\d t}{1 - t(1 - t)}\\
&= \frac{4}{3} \int_0^1 \frac{\d t}{\left[\frac{2}{\sqrt{3}}(t - 1/2)\right]^2 + 1} \\
&= \frac{2}{\sqrt{3}} \left[\arctan\left(\frac{2}{\sqrt{3}} (t-1/2)\right)\right]_0^1 \\
&= \frac{4}{\sqrt{3}} \arctan\frac{1}{\sqrt{3}} \\
&= \frac{2\pi}{3\sqrt{3}}
\end{align*}
\end{solution}

% --------------

\marginnote[0cm]{\todoinline{Source : D'après un oral Centrale 2016}}
\begin{exercice}%
Soit $p \geqslant 2$ et $q \in \N$. On pose $S(p) = \sum\limits_{n=p}^{+\infty} \binom{n}{p}^{-1}$.
\begin{enumerate}
\item Montrer l'existence de $S(p)$.

\item Montrer que $S_N(p) = \sum\limits_{n=p}^{N+p} \binom{n}{p}^{-1} = \sum\limits_{n=0}^N (n + p + 1) B(n, p)$.

\item En déduire que $S(p) = \frac{p}{p-1}$.
\end{enumerate}
\end{exercice}

\begin{demo}
\begin{enumerate}
\item D'après la définition des coefficients binomiaux,
\[
\binom{n}{p}^{-1} = \frac{p (p-1) \cdots 2 \cdot 1}{n (n-1) \cdots (n-p+2) (n-p+1)} = o\mathopen{}\left(\frac{1}{n^2}\right).
\]
Ainsi, d'après les théorèmes de comparaison des séries à termes positifs, $S(p)$ est bien définie.

\item D'après le théorème précédent,
\[
B(p, q) = \frac{p! q!}{(p + q + 1) (p + q)!} = \frac{\binom{p+q}{p}^{-1}}{p + q + 1}.
\]

Ainsi, si $p \geqslant 1$,
\begin{align*}
S_N(p)
&= \sum_{n=0}^{N} \binom{n+p}{p}^{-1}
= \sum_{n=0}^{N} (n+p+1) B(n, p) \\
&= \int_0^1 \sum_{n=0}^N (n+1) t^n (1 -t)^p \d t + \int_0^1 \sum_{n=0}^N p\, t^n (1 - t)^p \d t.
\end{align*}

De plus,
\[
\lim_{N\to+\infty} \sum_{n=0}^N (n + 1) t^n (1 - t)^p = (1 - t)^{p-2}
\quad \text{et} \quad 
\lim_{N\to+\infty} \sum_{n=0}^N t^n (1 - t)^p = (1 - t)^{p-1}.
\]
Comme ces suites sont croissantes et que les fonctions limites sont intégrables sur $\interff{0}{1}$, d'après le théorème de convergence dominée,
\begin{align*}
S(p)
&= \int_0^1 (1 -t)^{p-2} \d t + \int_0^1 p (1 - t)^{p-1} \d t 
= \frac{1}{p-1} + p \frac{1}{p}
= \frac{p}{p-1}.
\end{align*}

% \gras{Remarque.} On peut obtenir directement :
% \[
% \binom{n}{p}^{-1} = \binom{n-1}{p-1}^{-1} - \frac{n-p}{n-p+1} \binom{n}{p-1}^{-1}.
% \]
% Ainsi,
% \begin{align*}
% S(p) &= \sum_{n=0}^{+\infty} \binom{n+p}{n}^{-1} = 1 + \sum_{n=1}^{+\infty} \binom{n+p-1}{n-1}^{-1} - \frac{p}{p+1} \sum_{n=1}^{+\infty} \binom{n+p}{p-1}^{-1} \\
% &= 1 + S(p) - \frac{p}{p+1} S(p+1).
% \end{align*}
\end{enumerate}
\end{demo}

\subsection{Intégrales eulériennes et intégrales de \textsc{Wallis}}

Après le changement de variable $\sin x = \sqrt{t}$ dans l'intégrale de \textsc{Wallis} $\Wallis_n$, on obtient
\[
\Wallis_n = \frac{1}{2} \int_0^1 t^{\frac{n-1}{2}} (1-t)^{-\frac{1}{2}} \d t
\]
soit 
\[
\Wallis_n = \frac{1}{2} B\mathopen{}\left( \frac{n-1}{2}, -\frac{1}{2} \right) = \frac{1}{2} \frac{\Gamma\mathopen{}\left(\frac{n+1}{2}\right) \sqrt{\pi}}{\Gamma\mathopen{}\left(\frac{n}{2}+1\right)}
\]
\url{https://claude-gimenes.fr/mathematiques/calcul-integral/-i-integrales-euleriennes}
\section{Intégrales eulériennes}\label{secinteuleriennes}

\marginnote[0mm]{Source : \href{https://lescoursdemathsdepjh.monsite-orange.fr/file/7bcfcf82249b1046f185e9a3495845cd.pdf}{Fonction eulériennes -- Pierre-Jean \textsc{Hormière}}}
De premières tentatives pour définir la factorielle de valeurs non entières remontent à \textsc{Stirling} et Daniel \textsc{Bernoulli}. Dans une lettre à Christian \textsc{Goldbach} du 13 octobre 1729, \textsc{Euler} découvre (ou invente ?) une fonction de variable réelle prolongeant de manière naturelle la fonction $n!$. D'abord introduite comme limite de produits, cette fonction fut plus tard présentée sous forme intégrale et reliée à des fonctions voisines. \\
Les fonctions eulériennes sont les plus importantes \say{ fonctions spéciales } de l'analyse classique, réelle et complexe. \textsc{Legendre} les a nommées, classifiées et étudiées. Elles ont aussi été étudiées par \textsc{Gauss}, \textsc{Binet}, \textsc{Plana}, \textsc{Malmsten}, \textsc{Raabe}, \textsc{Weierstrass}, \textsc{Hankel}, H. \textsc{Bohr}, \textsc{Mollerup}, \textsc{Artin} \dots \\
Il y a bien des façons de prolonger la fonction $n!$ au domaine réel, même en se limitant aux fonctions continues. Une idée naturelle est de partir de la formule $\displaystyle n! = \int_{0}^{+ \infty} t^n \e^{-t} \d t$. Cette forme intégrale de la factorielle suggère de considérer la fonction $\displaystyle F(x)=\int_{0}^{+ \infty} t^x \e^{-t} \d t$. Cette fonction, définie sur $]-1, +\infty[$, prolonge intelligemment la factorielle, en ce sens qu'elle possède des propriétés nombreuses et cohérentes. Par commodité, on considère plutôt $\displaystyle \Gamma(x) = \int_{0}^{+ \infty} t^{x-1} \e^{-t} \d t$.

\subsection{Fonction Gamma d'\textsc{Euler}}

%\begin{marginfigure}[3cm]
%    \begin{tikzpicture}[]

\begin{axis}[
xmin = -4.9, xmax = 5.1, 
%ymin = -3.5, ymax = 3.5,  
restrict y to domain=-6:6,
axis lines = middle,
axis line style={-latex},  
xlabel={$x$}, 
ylabel={$\Gamma(x)$},
%enlarge x limits={upper={val=0.2}},
enlarge y limits=0.05,
x label style={at={(ticklabel* cs:1.00)}, inner sep=5pt, anchor=north},
y label style={at={(ticklabel* cs:1.00)}, inner sep=2pt, anchor=south east},
]

\addplot[color=red, samples=222, smooth, 
domain = 0:5] gnuplot{gamma(x)};

\foreach[evaluate={\N=\n-1}] \n in {0,...,-5}{%
\addplot[color=red, samples=555, smooth,  
domain = \n:\N] gnuplot{gamma(x)};
%
\addplot [domain=-6:6, samples=2, densely dashed, thin] (\N, x);
}%
\end{axis}
\end{tikzpicture}
%    \caption*{\centering Graphe de la fonction Gamma}
%\end{marginfigure}

\begin{defi}{Fonction Gamma d'\textsc{Euler}}
    La \emph{fonction Gamma d'\textsc{Euler}} est définie par: 
    $$\Gamma(x) \defeq \int_{0}^{+\infty} t^{x-1} \e^{-t} \d t.$$
\end{defi}

\begin{remarque}
    À un changement de variable près, la fonction $\Gamma$ est la \nameref{transformee_laplace} de la fonction $t \mapsto t^x$. 
\end{remarque} 

\begin{prop}{}
    \begin{itemize}
        \item La fonction $\Gamma$ est définie si et seulement si $x>0$.
        \item Pour tout $x > 0$, $\Gamma(x+1) = x\Gamma(x)$. \\
        En particulier, pour tout $n \in \N$, $\Gamma(n+1) = n!$. 
    \end{itemize}
\end{prop}

La démonstration est donnée dans \ref{prolongementFonctionGamma}.

Cette fonction, introduite en 1729 par le mathématicien suisse, prolonge la fonction factorielle à l'ensemble des réels strictement positifs.

\begin{exercice}
    \marginnote[0cm]{\cite{fmaalouf}}
    Caculer $\Gamma \left( \frac{1}{2} \right)$, puis $\Gamma \left( n + \frac{1}{2} \right)$ pour tout $n \in \Ne$.
\end{exercice}

\marginnote[-5cm]{
    \begin{theo}{Théorème (Intégration par parties généralisées)}
        Source : \cite{acamanes}\\
        Soient $f$ et $g$ deux fonctions de classe $\mathscr{C}^1$ sur $I$. Si la fonction $fg$ a une limite finie en $a$ et en $b$, alors les intégrales
        $$\int_a^b f'(t)g(t) \d t \text{ et } \int_a^b f(t) g'(t) \d t$$
        sont de même nature. Si ces quantités sont convergentes, en notant
        \begin{align*}
            [f(t)g(t)]_a^b \\
            = \lim_{x \to b^-} \big(f(x)g(x)\big) - \lim_{x \to a^+} \big(f(x)g(x)\big),
        \end{align*}
        on obtient la relation
        $$\int_a^b f'(t) g(t) \d t$$
        $$ = \left[f(t)g(t)\right]_a^b - \int_a^b f(t) g'(t) \d t.$$
    \end{theo}
}

\begin{exercice}
    \marginnote[0cm]{Source : \cite{fmaalouf}}
    Montrer que la fonction $\Gamma$ est continue sur $\Rpe$.
\end{exercice}

\begin{prop}{Dérivées successives de la fonction Gamma}
    $$\forall k \in \N,\ \forall x \in \R_+^\star,\ \Gamma^{(k)}(x) = \int_{0}^{+\infty} (\ln t)^k t^{x-1} \e^{-t} \d t$$
\end{prop}

\begin{elem_preuve}
    Utiliser une domination locale sur un segment $[a, A] \subset \R_+^\star$ par la fonction:
    $$\varphi_k:t \mapsto 
    \begin{cases}
        |\ln t |^k \e^{-t} t^{a-1} & \text{si } t \in ]0, 1], \\
        |\ln t |^k \e^{-t} t^{A-1} & \text{si } t > 1.
    \end{cases}
    $$
\end{elem_preuve}

\begin{exercice}
    \marginnote[0cm]{Source : \cite{fmaalouf}}
    Montrer que $\Gamma(x) \isEquivTo{0^+} \frac{1}{x}$. \\
    Montrer que $\Gamma$ est convexe et étudier ses variations.
\end{exercice}

\subsection{L'intégrale de \textsc{Gauss} comme valeur particulière de la fonction Gamma}

%%% Commente par Alain car pb de compilation %%%
% \marginnote[0cm]{Source : \href{https://fr.wikipedia.org/wiki/Intégrale_de_Gauss#Calcul_de_l'intégrale_de_Gauss}{Intégrale de Gauss -- \textsf{wikipedia.org}}}

$$\Gamma \left( \frac{1}{2} \right) = \int_0^{+ \infty} t^{\frac{1}{2}-1} \e^{-t} \d t = 2 \int_0^{+ \infty} \e^{-u^2} \d u = \int_{- \infty}^{+ \infty} e^{-u^2} \d u.$$

\subsection{Fonction bêta}

\todoinline{Retrouver l'exercice de Maths II Centrale qui somme des quantités qui resemblent à ceci !}

\todoarmand{Centrale Maths 2 MP 2015}



%--------------

\begin{exercice}%
{RMS 833}%
{16}%
{Centrale}%
Pour tout entier naturel $n$, on note $u_n = \sum_{k=0}^n \frac{1}{\binom{n}{k}}$.
\begin{itemize}
\item Écrire une fonction \verb?binomial(n, k)? qui renvoie $\binom{n}{k}$. Tracer, pour $n \in \ens{5, 8, 9}$, les points $\left(\binom{n}{k}\right)_{2\leq k \leq n-2}$.

\item Montrer que, pour tout $2 \leq k \leq n-2$, $\binom{n}{k} \geq \binom{n}{2}$.

\item Pour tout $n \in \N$, on note $A_n$ le point de coordonnées $(n, u_n)$. Afficher les $31$ premiers termes $A_0,\ldots,A_{30}$. Conjecturer le comportement asymptotique de $(u_n)$.

\item Démontrer rigoureusement la convergence de $(u_n)$.

Soit $p \geq 2$ et $q \in \N$. On pose $S(p) = \sum_{n=p}^{+\infty} \frac{1}{\binom{n}{p}}$.

\item Montrer l'existence de $S(p)$.

\item On note $S_N = \sum_{n=p}^N \binom{n}{p}^{-1}$. Tracer $(p-1) S_{200}(p)$ en fonction de $p$ pour $p \in \entiers{2}{50}$.

\item Exprimer $I(p, q) = \int_0^1 t^p (1 - t)^q \d t$ en fonction d'un coefficient binomial.

\item En déduire que $S(p) = \frac{p}{p-1}$.
\end{itemize}
\end{exercice}

\begin{preuve}
\begin{itemize}
\item

\item D'après la définition des coefficients binomiaux,
\[
\frac{\binom{n}{k}}{\binom{n}{2}} = \frac{n (n-1) \cdots (n-k+1) 2}{k (k-1) \cdots 3 \cdot 2 \cdot n (n-1)} \geq 1.
\]

\item D'après la question précédente,
\[
2 \leq u_{n+1} \leq 2 + \sum_{k=2}^{n-2} \frac{1}{\binom{n}{2}} \leq 2 + \frac{2(n-3)}{n (n-2)}.
\]
Ainsi, $\lim_{n\to+\infty} u_n = 2$.

\item D'après la définition,
\[
\binom{n}{p}^{-1} = \frac{p (p-1) \cdots 2 \cdot 1}{n (n-1) \cdots (n-p+2) (n-p+1)} = o\left(\frac{1}{n^2}\right).
\]
Ainsi, d'après les théorèmes de comparaison des séries à termes positifs, $S(p)$ est bien définie.

\item À l'aide d'une intégration par parties, $I_{p,q} = \frac{q}{p+1} I_{p+1,q-1}$. De plus, $I_{p,0} = \frac{1}{p+1}$. Ainsi,
\[
I_{p,q} = \frac{q (q-1) \cdots 1}{(p+1) \cdots (p+q)} I_{p+q,0} = \frac{1}{p+q+1} \binom{p+q}{p}^{-1}.
\]

\item D'après le calcul précédent, si $p \geq 1$,
\begin{align*}
S_N(p) &= \sum_{n=0}^{N} \binom{n+p}{p}^{-1} = \sum_{n=0}^{N} (n+p+1) I_{n,p} \\
&= \int_0^1 \sum_{n=0}^N (n+1) t^n (1 -t)^p \d t + \int_0^1 \sum_{n=0}^N p t^n (1 - t)^p \d t \\
&\to \int_0^1 (1 -t)^{p-2} \d t + \int_0^1 p (1 - t)^{p-1} \d t 
= \frac{1}{p-1} + p \frac{1}{p} = \frac{p}{p-1},
\end{align*}
où le théorème de convergence dominée a pu être appliqué car toutes les sommes partielles sont croissantes et leur limite est intégrable car continue.

% \gras{Remarque.} On peut obtenir directement :
% \[
% \binom{n}{p}^{-1} = \binom{n-1}{p-1}^{-1} - \frac{n-p}{n-p+1} \binom{n}{p-1}^{-1}.
% \]
% Ainsi,
% \begin{align*}
% S(p) &= \sum_{n=0}^{+\infty} \binom{n+p}{n}^{-1} = 1 + \sum_{n=1}^{+\infty} \binom{n+p-1}{n-1}^{-1} - \frac{p}{p+1} \sum_{n=1}^{+\infty} \binom{n+p}{p-1}^{-1} \\
% &= 1 + S(p) - \frac{p}{p+1} S(p+1).
% \end{align*}
\end{itemize}
\end{preuve}


\begin{defi}{Fonction bêta}
    Pour tout $(p,q) \in \N^2$, on note
    $$I_{p,q} \defeq \int_{0}^{1} t^p (1-t)^q \d t.$$
\end{defi}

\begin{prop}{Expression factiorelle de la fonction bêta}
    Pour tout $(p,q) \in \N^2$,
    $$I_{p,q} = \frac{p! q!}{(p + q + 1)!}.$$
\end{prop}

\marginnote[0cm]{
Sources : 
\begin{itemize}
    \item \href{https://fr.wikipedia.org/wiki/Intégrale_d'Euler}{Intégrale d'\textsc{Euler} -- \textsf{wikipedia.org}}
    \item \cite{calcul_infinitesimal} Chapitre IV, 3 Intégrales eulériennes, page 125.
\end{itemize}
}

\begin{preuve}
    Soit $(p,q) \in \N^2$. Nous allons déterminer une relation entre $I_{p,q}$ et $I_{p+1, q-1}$ en faisant une intégration par parties. \\
    On pose $u:t\mapsto \frac{1}{p+1} t^{p+1}$ et $v:t\mapsto (1-t)^q$, toutes deux de classe $\mathscr{C}^1$ sur $[0, 1]$. Alors, 
    \begin{align*}
        I_{p,q} &= \left[ \frac{1}{p+1}t^{p+1} \times (1-t)^q \right]_0^1 + \frac{q}{p+1} \int_0^1 t^{p+1} (1-t)^{q-1} \d t \\
        I_{p,q} &= \frac{q}{p+1} I_{p+1, q-1}.
    \end{align*}
    On en déduit que 
    \begin{align*}
        I_{p,q} &= \frac{q}{p+1} \times \frac{q-1}{p+2} \times \cdots \times \frac{1}{p+q} I_{p+q,0} \\
        I_{p,q} &= \frac{p! q!}{(p + q + 1)!}.
    \end{align*}
\end{preuve}

\todoinline{Je crois pas que ce soit la même chose que le renforcement de Polya qui converge vers une loi Beta, mais il faut regarder plus en détail pour être sûr (si on garde).}

\todoarmand{Je suis partant pour garder cet exercice}

\begin{exercice}%
{RMS 2017 1071 \& 1073}%
{Centrale 16'}%
\begin{enumerate}
\item Soit $(p, q) \in (\N^\ast)^2$. Calculer l'intégrale $\int_0^1 x^p (1 - x)^q \d x$.

\item Soit $p \in \N^\ast$. On dispose de $p$ urnes numérotées de $1$ à $p$. Chaque urne contient $p$ boules et pour tout $i \in \entiers{1}{p}$, l'urne numéro $i$ contient $i$ boules noires et $p - i$ boules blanches. On effectue l'expérience suivante : choisir au hasard une urne puis effectuer des tirages avec remise dans l'urne choisie. On note, pour $n \in \N^\ast$, $A_n$ l'événement \textit{on a effectué $2 n$ tirages et obtenu le même nombre de boules blanches que de noires}.
\begin{itemize}
\item Exprimer $\mathbf{P}(A_n)$ sous forme d'une somme.

\item On note $b_{n,p}$ la probabilité que la prochaine boule tirée soit blanche sachant que $A_n$ est réalisé. Exprimer $b_{n,p}$.
\item Calculer $\lim\limits_{p\to+\infty} b_{n,p}$.
\end{itemize}
\end{enumerate}
\end{exercice}

\begin{preuve}
\begin{itemize}
\item En effectuant une intégration par parties, dès que $q > 0$, alors $I_{p,q} = \frac{q}{p+1} I_{p+1, q-1}$. Ainsi, par récurrence,
\[
I_{p,q} = \prod_{k=0}^{q-1} \frac{q - k}{p+1+k} I_{p+q, 0} = \frac{q! p!}{(p+q+1)!}.
\]

\item
\begin{itemize}
\item En utilisant la formule des probabilités totales,
\begin{align*}
\mathbf{P}(A_n) &= \sum_{i=1}^p \mathbf{P}(A_n | U_i) \mathbf{P}(U_i) \\
&= \sum_{i=1}^n \binom{2n}{n} \left(\frac{p-i}{p}\right)^n \left(\frac{i}{p}\right)^n.
\end{align*}

\item En utilisant le même raisonnement que précédemment,
\[
b_{n,p} = \frac{\frac{1}{p} \sum_{i=1}^p \left(1 - \frac{i}{p}\right)^{n+1} \left(\frac{i}{p}\right)^n}{\frac{1}{p} \sum_{i=1}^p \left(1 - \frac{i}{p}\right)^{n} \left(\frac{i}{p}\right)^n} \cdot \frac{\binom{2n+1}{n+1}}{\binom{2n}{n}}.
\]

\item En utilisant les limites des sommes de Riemann,
\[
\lim\limits_{p\to+\infty} b_{n,p} = \frac{2n+1}{n+1} \cdot \frac{\int_0^1 (1 - x)^{n+1} x^n \d x}{\int_0^1 (1 - x)^n x^n \d x}.
\]
Cette limite vaut donc
\[
\frac{2n+1}{n+1} \cdot \frac{n! (n+1)!}{(2n+2)!} \cdot \frac{(2n+1)!}{n! n!}  = \frac{2n+1}{2n+2}.
\]
\end{itemize}
\end{itemize}
\end{preuve}

\begin{exercice}
    \marginnote[0cm]{Souce : \cite{acamanes}}
    Déterminer la nature de la série $\sum I_{n,n}$ et, le cas échéant, calculer sa somme. 
\end{exercice}

\begin{exercice}
    \marginnote[0cm]{Source : Clémentine \textsc{Portal} (PCSI1, Collège Stanislas) feuille d'exo n°7, exo 11}
    Déterminer une expression simplifiée de $\sum\limits_{k=0}^q \binom{q}{k} \frac{(-1)^k}{p+k+1}$, pour tout $(p, q) \in \N^2$.
\end{exercice}

\begin{preuve}
\end{preuve}

\begin{prop}{}
    \marginnote[0cm]{Source : \cite{exos_oraux} p.280}
    Soit $f$ continue par morceaux sur le segment $[a,b]$, avec $a < b$,
    $$\lim_{n \to +\infty} \int_{a}^{b} f(t) | \sin (nt) | \d t = \frac{2}{\pi} \int_{a}^{b} f(t) \d t.$$
\end{prop}

\todoinline{Relire et compléter la correction. Trouver une application}

\todoarmand{Pour une application du lemme de Riemann-Lebesgue (qui n'est pas tout à fait l'énoncé d'au-dessus) : exercices 4 et 5 de \url{https://www.louboutin.org/LeSite20182019/mathematiques/Exercices/Ex11_1819.pdf}}

\begin{elem_preuve}
    \begin{enumerate}
        \item On va montrer ce résultat dans le cas où \ptnclegras{$f$ est constante sur $[a, b]$} (véritable difficulté du problème) 
        \item On va ensuite montrer ce résultat dans le cas où \ptnclegras{$f$ est une fonction en escalier} en appliquant le résultat précédent sur chacun des intervalles de la subdivision de $[a, b]$.
        \item Finalement on va montrer le \ptnclegras{cas général} en encadrant $f$ par deux fonctions en escalier (méthode de l'intégrale de $\textsc{Riemann}$).
    \end{enumerate}
\end{elem_preuve}
\begin{preuve}
    \begin{enumerate}
        \item On pose $f = \lambda$. On va étudier la limite de l'intégrale
        $$I_n = \int_{a}^{b} \lambda | \sin(nt) | \d t = \frac{\lambda}{n} \int_{na}^{nb} | \sin(u) | \d u.$$
        L'idée est alors de découper l'intervalle $[a, b]$ en trois intervalles: des \textcolor{YellowGreen}{extrémités} où l'intégrale tendra vers $0$ puis un intervalle \textcolor{Salmon}{central} de longueur $k_n \pi$ qui sera simple à traiter. \\
        \textcolor{green}{A mieux rédiger...} \\
        On pose (qui existent pour $n \geqslant \frac{\pi}{b-a}$)
        $$c_n = \min( \pi \mathbb{Z} \cap [na, nb]) \text{ et } d_n = \max( \pi \mathbb{Z} \cap [na, nb]).$$
        $c_n \sim na$ et $d_n \sim nb$. 
        \item Aucune difficulté.
        \item Il existe deux fonctions en escalier $\varphi$ et $\psi$ telles que $\varphi \leqslant f \leqslant \psi$ et $\int_{a}^{b} (\psi - \varphi) \leqslant \varepsilon$.
        \begin{align*}
            & \left | \int_{a}^{b} f(t) | \sin (nt) | \d t - \frac{2}{\pi} \int_{a}^{b} f(t) \d t \right| \\
            \leqslant & \left | \int_{a}^{b} [f(t) - \varphi(t) ] | \sin(t) | \d t \right| + \left | \int_{a}^{b} \varphi(t) |\sin(t)| \d t - \frac{2}{\pi} \int_{a}^{b} \varphi(t) \d t \right| + \left| \frac{2}{\pi} \int_{a}^{b} [f(t) - \varphi(t)] \d t \right|
        \end{align*}
    \end{enumerate}
\end{preuve}

\section{Intégration des relations de comparaisons}

\begin{prop}{}
    Soit $f: \Rp \rightarrow \C$ une fonction continue par morceaux et $g, h:\Rp \rightarrow \Rp$ deux fonctions continues par morceaux, strictement positives. On suppose que $f = o_{+\infty}(g)$ et $f \sim_{+\infty} h$.\\
    \begin{itemize}
        \item Si $g$ et $h$ ne sont pas intégrables sur $\Rp$,
        $$\int_{0}^{x} f = o_{+\infty} \left(\int_{0}^{x} g \right) \text{ et } \int_{0}^{x} f \sim_{+\infty} \int_{0}^{x} h.$$
        \item Si $g$ et $h$ sont intégrables sur $\Rp$,
        $$\int_{x}^{+\infty} f = o_{+\infty} \left(\int_{x}^{+\infty} g \right) \text{ et } \int_{x}^{+\infty} f \sim_{+\infty} \int_{x}^{+\infty} h.$$
    \end{itemize}
\end{prop} 

La démonstration est analogue à celle de la \nameref{sommation_relations_comparaison}

\todoinline{J'ai mis dans "documents" le sujet Centrale PC 2003 - Il fait à la fois des relations de comparaisons, de l'intégrale de Bertrand à la fin et une intégrale fonction des bornes. C'est peut être une bonne idée !}
%--------
\section{Comparaisons Séries / Intégrales}
% \section{Intégration des relations de comparaisons}

\todoinline{Le sujet de centrale est très prometteur au début puis utilise des résultats beaucoup plus faibles que ceux démontrés en préliminaires. Je propose de déplacer ce résultat dans la partie Séries numériques}

% Source : Centrale - PC - Maths 1 - 2003
\begin{prop}{}
Il existe deux constantes $\gamma$ et $\delta$ telles que
\[
n! = \delta n^{n + \frac{1}{2}} \e^{-n} \left(1 + \frac{1}{12 n} + o\left(\frac{1}{n}\right)\right).
\]
\end{prop}

\begin{itemize}
\item On suppose que $f$ et $g$ sont positives et vérifient $f(x) \sim_b g(x)$. Si $\int_a^b f(t) \d t$ converge, alors $\int_x^b f(t) \d t \sim_b \int_x^b g(t) \d t$.

Soit $\epsilon > 0$. La fonction $g$ étant strictement positive et comme $f \sim g$, il existe $\beta \in [a, b[$ tel que
\[
\forall\, t \in [\beta, b[,\, 0 < \abs{f(t) - g(t)} \leq \epsilon g(t).
\]
Ainsi, comme $g$ est intégrable sur $[\beta, b[$, alors $f - g$ l'est également. D'après la croissance de l'intégrale et l'inégalité triangulaire,
\[
\forall\, x \in [\beta, b[,\, 0 \leq \abs{\int_x^b f(t) \d t - \int_x^b g(t) \d t} \leq \epsilon \int_x^b g(t) \d t.
\]
Ainsi, $\int_x^b f(t) \d t \sim \int_x^b g(t) \d t$.

\item Si $a_n \sim b_n$ sont positifs et les termes généraux de séries convergentes, alors $\sum_{k=n+1}^{+\infty} a_k \sim \sum_{k=n+1}^{+\infty} b_k$.

Il suffit d'appliquer le résultat précédent aux fonctions en escalier
\[
f(t) = \sum_{k=0}^{+\infty} a_k \indicatrice{[k,k+1[}
\text{ et }
g(t) = \sum_{k=0}^{+\infty} b_k \indicatrice{[k, k+1[}.
\]

\item On pose $b_n = 1 + \left(n - \frac{1}{2}\right) \ln\left(1 - \frac{1}{n}\right)$ pour $n \geq 2$ et $b_0 = 0$, $b_1 = 1$.

On remarque que $b_n \sim -\frac{1}{12 n^2}$. Comme $\sum a_n$ et $\sum b_n$ convergent, alors
\[
\sum_{k=n+1}^{+\infty} a_n \sim \sum_{k=n+1}^{+\infty} b_n.
\]

\item En utilisant les comparaisons séries / intégrales,
\[
\sum_{k=n+1}^{+\infty} \frac{1}{k^2} \sim \int_n^{+\infty} \frac{\d t}{t^2} = \frac{1}{n}.
\]
\todoinline{À justifier}


Ainsi, il existe un réel $\ell$ tel que
\[
\sum_{k=0}^n b_k = \ln\left(\frac{n! \e^n}{n^{n+\frac{1}{2}}}\right) = \ell + \frac{1}{12 n} + o\left(\frac{1}{n}\right).
\]

En utilisant la fonction expontentielle,
\[
\frac{n! \e^n}{n^{n+\frac{1}{2}}} = \e^{\ell} \times \e^{\frac{1}{12 n} + o\left(\frac{1}{n}\right)}.
\]

On obtient le résultat attendu en utilisant un développement limité de la fonction exponentielle.
\end{itemize}



\begin{prop}{}
    Soit $f: \Rp \rightarrow \C$ une fonction continue par morceaux et $g, h:\Rp \rightarrow \Rp$ deux fonctions continues par morceaux, strictement positives. On suppose que $f = o_{+\infty}(g)$ et $f \sim_{+\infty} h$.\\
    \begin{itemize}
        \item Si $g$ et $h$ ne sont pas intégrables sur $\Rp$,
        $$\int_{0}^{x} f = o_{+\infty} \left(\int_{0}^{x} g \right) \text{ et } \int_{0}^{x} f \sim_{+\infty} \int_{0}^{x} h.$$
        \item Si $g$ et $h$ sont intégrables sur $\Rp$,
        $$\int_{x}^{+\infty} f = o_{+\infty} \left(\int_{x}^{+\infty} g \right) \text{ et } \int_{x}^{+\infty} f \sim_{+\infty} \int_{x}^{+\infty} h.$$
    \end{itemize}
\end{prop} 

La démonstration est analogue à celle de la \nameref{sommation_relations_comparaison}


\todoinline{J'ai relu. Sans illustrations et sans application, est-ce qu'on le laisse ? Ou alors on trouve une application, mais je n'en ai pas sous le coude à l'instant !}

\begin{theo}{Intégrales de \textsc{Bertrand}}
    Soient $(\alpha, \beta) \in  \R^2$ et 
    $$f:t \mapsto \frac{1}{t^{\alpha} \ln^{\beta} (t)}.$$
    Alors,
    $$\int_{2}^{+ \infty} f(t) \d t \text{ converge si et seulement si }
    \begin{cases}
    \alpha > 1 \\
    \text{ou}\\
    \alpha = 1 \text{ et } \beta > 1
    \end{cases}.
    $$
\end{theo}

% \todoinline{On écrit plutôt $\int_2^{+\infty} f(t) dt$ et $\int_{[2,+\infty[} f$.}

\begin{preuve}
    Distinguons trois cas selon les valeurs prises par $\alpha$:
    \begin{enumerate}
        \item[$\rhd$] \textbf{Cas où $\alpha > 1$.} Soit $\gamma \in ]1, \alpha[$. Par croissances comparées,
        $$\displaystyle \frac{1}{t^{\alpha} \ln^{\beta} (t)} = o_{+ \infty} \left( \frac{1}{t^{\gamma}} \right).$$
        Or, d'après la convergence des intégrales de \textsc{Riemann}, la fonction $t \mapsto \frac{1}{t^\gamma}$ est intégrable sur $[2, +\infty[$ car $\gamma > 1$. Ainsi, en appliquant les théorèmes de comparaison, $\int_2^{+ \infty} f$ converge.

        \item[$\rhd$] \textbf{Cas où $\alpha < 1$.} Soit $\gamma \in ]\alpha, 1[$. Par croissances comparées,
        $$t^{\gamma} f(t) \xrightarrow[t \to + \infty]{} + \infty$$
        donc à partir d'un certain rang, $f(t) \geqslant \frac{1}{t^{\gamma}} > 0$. Or, d'après la convergence des intégrales de \textsc{Riemann}, la fonction $t \mapsto \frac{1}{t^\gamma}$ n'est intégrable pas sur $[2, +\infty[$ car $\gamma < 1$. Ainsi, en appliquant les théorèmes de comparaison (les intégrandes sont positives), $\int_2^{+ \infty} f$ diverge.
        
        \item[$\rhd$] \textbf{Cas où $\alpha = 1$.} Revenons aux intégrales partielles: soit $X > 2$,
        $$\int_{2}^{X} \frac{1}{t \ln^{\beta} (t)} \d t = 
        \begin{cases}
            \left[ \frac{\ln ^{1-\beta} (t)}{1-\beta} \right]_2 ^X & \text{si } \beta \not = 1, \\
            \left[\ln (\ln t) \right]_2 ^X & \text{si } \beta = 1.
        \end{cases}
        $$
        On en déduit que l'intégrale de la fonction $t \mapsto \frac{1}{t \ln^{\beta} (t)}$ converge sur $[2, + \infty[$ si et seulement si $\beta > 1$.
    \end{enumerate}
\end{preuve}

\todoinline{J'ai mis dans "documents" le sujet Centrale PC 2003 - Il fait à la fois des relations de comparaisons, de l'intégrale de Bertrand à la fin et une intégrale fonction des bornes. C'est peut être une bonne idée !}

\todoarmand{Effectivement, c'est une bonne idée. Ça permettrait de supprimer l'exercice sur l'intégrale de Bertrand et de l'intégrer avec celui-ci}
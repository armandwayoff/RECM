\section{Intégrales dépendant des bornes}

%-----------
\subsection{Une modification du logarithme intégral}
\todoarmand{Ajouter une note sur sa signification en théorie des nombres pour compter les nombres premiers.}

\begin{marginfigure}[-2cm]
    \centering
    %% Creator: Matplotlib, PGF backend
%%
%% To include the figure in your LaTeX document, write
%%   \input{<filename>.pgf}
%%
%% Make sure the required packages are loaded in your preamble
%%   \usepackage{pgf}
%%
%% Also ensure that all the required font packages are loaded; for instance,
%% the lmodern package is sometimes necessary when using math font.
%%   \usepackage{lmodern}
%%
%% Figures using additional raster images can only be included by \input if
%% they are in the same directory as the main LaTeX file. For loading figures
%% from other directories you can use the `import` package
%%   \usepackage{import}
%%
%% and then include the figures with
%%   \import{<path to file>}{<filename>.pgf}
%%
%% Matplotlib used the following preamble
%%   
%%   \usepackage{fontspec}
%%   \setmainfont{DejaVuSerif.ttf}[Path=\detokenize{/home/wayoff/.pyenv/versions/3.8.10/lib/python3.8/site-packages/matplotlib/mpl-data/fonts/ttf/}]
%%   \setsansfont{DejaVuSans.ttf}[Path=\detokenize{/home/wayoff/.pyenv/versions/3.8.10/lib/python3.8/site-packages/matplotlib/mpl-data/fonts/ttf/}]
%%   \setmonofont{DejaVuSansMono.ttf}[Path=\detokenize{/home/wayoff/.pyenv/versions/3.8.10/lib/python3.8/site-packages/matplotlib/mpl-data/fonts/ttf/}]
%%   \makeatletter\@ifpackageloaded{underscore}{}{\usepackage[strings]{underscore}}\makeatother
%%
\begingroup%
\makeatletter%
\begin{pgfpicture}%
\pgfpathrectangle{\pgfpointorigin}{\pgfqpoint{3.000000in}{2.000000in}}%
\pgfusepath{use as bounding box, clip}%
\begin{pgfscope}%
\pgfsetbuttcap%
\pgfsetmiterjoin%
\definecolor{currentfill}{rgb}{1.000000,1.000000,1.000000}%
\pgfsetfillcolor{currentfill}%
\pgfsetlinewidth{0.000000pt}%
\definecolor{currentstroke}{rgb}{1.000000,1.000000,1.000000}%
\pgfsetstrokecolor{currentstroke}%
\pgfsetdash{}{0pt}%
\pgfpathmoveto{\pgfqpoint{0.000000in}{0.000000in}}%
\pgfpathlineto{\pgfqpoint{3.000000in}{0.000000in}}%
\pgfpathlineto{\pgfqpoint{3.000000in}{2.000000in}}%
\pgfpathlineto{\pgfqpoint{0.000000in}{2.000000in}}%
\pgfpathlineto{\pgfqpoint{0.000000in}{0.000000in}}%
\pgfpathclose%
\pgfusepath{fill}%
\end{pgfscope}%
\begin{pgfscope}%
\pgfsetbuttcap%
\pgfsetmiterjoin%
\definecolor{currentfill}{rgb}{1.000000,1.000000,1.000000}%
\pgfsetfillcolor{currentfill}%
\pgfsetlinewidth{0.000000pt}%
\definecolor{currentstroke}{rgb}{0.000000,0.000000,0.000000}%
\pgfsetstrokecolor{currentstroke}%
\pgfsetstrokeopacity{0.000000}%
\pgfsetdash{}{0pt}%
\pgfpathmoveto{\pgfqpoint{0.619136in}{0.576079in}}%
\pgfpathlineto{\pgfqpoint{2.850000in}{0.576079in}}%
\pgfpathlineto{\pgfqpoint{2.850000in}{1.850000in}}%
\pgfpathlineto{\pgfqpoint{0.619136in}{1.850000in}}%
\pgfpathlineto{\pgfqpoint{0.619136in}{0.576079in}}%
\pgfpathclose%
\pgfusepath{fill}%
\end{pgfscope}%
\begin{pgfscope}%
\pgfpathrectangle{\pgfqpoint{0.619136in}{0.576079in}}{\pgfqpoint{2.230864in}{1.273921in}}%
\pgfusepath{clip}%
\pgfsetrectcap%
\pgfsetroundjoin%
\pgfsetlinewidth{0.803000pt}%
\definecolor{currentstroke}{rgb}{0.690196,0.690196,0.690196}%
\pgfsetstrokecolor{currentstroke}%
\pgfsetdash{}{0pt}%
\pgfpathmoveto{\pgfqpoint{0.718507in}{0.576079in}}%
\pgfpathlineto{\pgfqpoint{0.718507in}{1.850000in}}%
\pgfusepath{stroke}%
\end{pgfscope}%
\begin{pgfscope}%
\pgfsetbuttcap%
\pgfsetroundjoin%
\definecolor{currentfill}{rgb}{0.000000,0.000000,0.000000}%
\pgfsetfillcolor{currentfill}%
\pgfsetlinewidth{0.803000pt}%
\definecolor{currentstroke}{rgb}{0.000000,0.000000,0.000000}%
\pgfsetstrokecolor{currentstroke}%
\pgfsetdash{}{0pt}%
\pgfsys@defobject{currentmarker}{\pgfqpoint{0.000000in}{-0.048611in}}{\pgfqpoint{0.000000in}{0.000000in}}{%
\pgfpathmoveto{\pgfqpoint{0.000000in}{0.000000in}}%
\pgfpathlineto{\pgfqpoint{0.000000in}{-0.048611in}}%
\pgfusepath{stroke,fill}%
}%
\begin{pgfscope}%
\pgfsys@transformshift{0.718507in}{0.576079in}%
\pgfsys@useobject{currentmarker}{}%
\end{pgfscope}%
\end{pgfscope}%
\begin{pgfscope}%
\definecolor{textcolor}{rgb}{0.000000,0.000000,0.000000}%
\pgfsetstrokecolor{textcolor}%
\pgfsetfillcolor{textcolor}%
\pgftext[x=0.718507in,y=0.478857in,,top]{\color{textcolor}\sffamily\fontsize{10.000000}{12.000000}\selectfont \(\displaystyle 0\)}%
\end{pgfscope}%
\begin{pgfscope}%
\pgfpathrectangle{\pgfqpoint{0.619136in}{0.576079in}}{\pgfqpoint{2.230864in}{1.273921in}}%
\pgfusepath{clip}%
\pgfsetrectcap%
\pgfsetroundjoin%
\pgfsetlinewidth{0.803000pt}%
\definecolor{currentstroke}{rgb}{0.690196,0.690196,0.690196}%
\pgfsetstrokecolor{currentstroke}%
\pgfsetdash{}{0pt}%
\pgfpathmoveto{\pgfqpoint{1.226538in}{0.576079in}}%
\pgfpathlineto{\pgfqpoint{1.226538in}{1.850000in}}%
\pgfusepath{stroke}%
\end{pgfscope}%
\begin{pgfscope}%
\pgfsetbuttcap%
\pgfsetroundjoin%
\definecolor{currentfill}{rgb}{0.000000,0.000000,0.000000}%
\pgfsetfillcolor{currentfill}%
\pgfsetlinewidth{0.803000pt}%
\definecolor{currentstroke}{rgb}{0.000000,0.000000,0.000000}%
\pgfsetstrokecolor{currentstroke}%
\pgfsetdash{}{0pt}%
\pgfsys@defobject{currentmarker}{\pgfqpoint{0.000000in}{-0.048611in}}{\pgfqpoint{0.000000in}{0.000000in}}{%
\pgfpathmoveto{\pgfqpoint{0.000000in}{0.000000in}}%
\pgfpathlineto{\pgfqpoint{0.000000in}{-0.048611in}}%
\pgfusepath{stroke,fill}%
}%
\begin{pgfscope}%
\pgfsys@transformshift{1.226538in}{0.576079in}%
\pgfsys@useobject{currentmarker}{}%
\end{pgfscope}%
\end{pgfscope}%
\begin{pgfscope}%
\definecolor{textcolor}{rgb}{0.000000,0.000000,0.000000}%
\pgfsetstrokecolor{textcolor}%
\pgfsetfillcolor{textcolor}%
\pgftext[x=1.226538in,y=0.478857in,,top]{\color{textcolor}\sffamily\fontsize{10.000000}{12.000000}\selectfont \(\displaystyle 1/4\)}%
\end{pgfscope}%
\begin{pgfscope}%
\pgfpathrectangle{\pgfqpoint{0.619136in}{0.576079in}}{\pgfqpoint{2.230864in}{1.273921in}}%
\pgfusepath{clip}%
\pgfsetrectcap%
\pgfsetroundjoin%
\pgfsetlinewidth{0.803000pt}%
\definecolor{currentstroke}{rgb}{0.690196,0.690196,0.690196}%
\pgfsetstrokecolor{currentstroke}%
\pgfsetdash{}{0pt}%
\pgfpathmoveto{\pgfqpoint{1.734568in}{0.576079in}}%
\pgfpathlineto{\pgfqpoint{1.734568in}{1.850000in}}%
\pgfusepath{stroke}%
\end{pgfscope}%
\begin{pgfscope}%
\pgfsetbuttcap%
\pgfsetroundjoin%
\definecolor{currentfill}{rgb}{0.000000,0.000000,0.000000}%
\pgfsetfillcolor{currentfill}%
\pgfsetlinewidth{0.803000pt}%
\definecolor{currentstroke}{rgb}{0.000000,0.000000,0.000000}%
\pgfsetstrokecolor{currentstroke}%
\pgfsetdash{}{0pt}%
\pgfsys@defobject{currentmarker}{\pgfqpoint{0.000000in}{-0.048611in}}{\pgfqpoint{0.000000in}{0.000000in}}{%
\pgfpathmoveto{\pgfqpoint{0.000000in}{0.000000in}}%
\pgfpathlineto{\pgfqpoint{0.000000in}{-0.048611in}}%
\pgfusepath{stroke,fill}%
}%
\begin{pgfscope}%
\pgfsys@transformshift{1.734568in}{0.576079in}%
\pgfsys@useobject{currentmarker}{}%
\end{pgfscope}%
\end{pgfscope}%
\begin{pgfscope}%
\definecolor{textcolor}{rgb}{0.000000,0.000000,0.000000}%
\pgfsetstrokecolor{textcolor}%
\pgfsetfillcolor{textcolor}%
\pgftext[x=1.734568in,y=0.478857in,,top]{\color{textcolor}\sffamily\fontsize{10.000000}{12.000000}\selectfont \(\displaystyle 1/2\)}%
\end{pgfscope}%
\begin{pgfscope}%
\pgfpathrectangle{\pgfqpoint{0.619136in}{0.576079in}}{\pgfqpoint{2.230864in}{1.273921in}}%
\pgfusepath{clip}%
\pgfsetrectcap%
\pgfsetroundjoin%
\pgfsetlinewidth{0.803000pt}%
\definecolor{currentstroke}{rgb}{0.690196,0.690196,0.690196}%
\pgfsetstrokecolor{currentstroke}%
\pgfsetdash{}{0pt}%
\pgfpathmoveto{\pgfqpoint{2.242599in}{0.576079in}}%
\pgfpathlineto{\pgfqpoint{2.242599in}{1.850000in}}%
\pgfusepath{stroke}%
\end{pgfscope}%
\begin{pgfscope}%
\pgfsetbuttcap%
\pgfsetroundjoin%
\definecolor{currentfill}{rgb}{0.000000,0.000000,0.000000}%
\pgfsetfillcolor{currentfill}%
\pgfsetlinewidth{0.803000pt}%
\definecolor{currentstroke}{rgb}{0.000000,0.000000,0.000000}%
\pgfsetstrokecolor{currentstroke}%
\pgfsetdash{}{0pt}%
\pgfsys@defobject{currentmarker}{\pgfqpoint{0.000000in}{-0.048611in}}{\pgfqpoint{0.000000in}{0.000000in}}{%
\pgfpathmoveto{\pgfqpoint{0.000000in}{0.000000in}}%
\pgfpathlineto{\pgfqpoint{0.000000in}{-0.048611in}}%
\pgfusepath{stroke,fill}%
}%
\begin{pgfscope}%
\pgfsys@transformshift{2.242599in}{0.576079in}%
\pgfsys@useobject{currentmarker}{}%
\end{pgfscope}%
\end{pgfscope}%
\begin{pgfscope}%
\definecolor{textcolor}{rgb}{0.000000,0.000000,0.000000}%
\pgfsetstrokecolor{textcolor}%
\pgfsetfillcolor{textcolor}%
\pgftext[x=2.242599in,y=0.478857in,,top]{\color{textcolor}\sffamily\fontsize{10.000000}{12.000000}\selectfont \(\displaystyle 3/4\)}%
\end{pgfscope}%
\begin{pgfscope}%
\pgfpathrectangle{\pgfqpoint{0.619136in}{0.576079in}}{\pgfqpoint{2.230864in}{1.273921in}}%
\pgfusepath{clip}%
\pgfsetrectcap%
\pgfsetroundjoin%
\pgfsetlinewidth{0.803000pt}%
\definecolor{currentstroke}{rgb}{0.690196,0.690196,0.690196}%
\pgfsetstrokecolor{currentstroke}%
\pgfsetdash{}{0pt}%
\pgfpathmoveto{\pgfqpoint{2.750629in}{0.576079in}}%
\pgfpathlineto{\pgfqpoint{2.750629in}{1.850000in}}%
\pgfusepath{stroke}%
\end{pgfscope}%
\begin{pgfscope}%
\pgfsetbuttcap%
\pgfsetroundjoin%
\definecolor{currentfill}{rgb}{0.000000,0.000000,0.000000}%
\pgfsetfillcolor{currentfill}%
\pgfsetlinewidth{0.803000pt}%
\definecolor{currentstroke}{rgb}{0.000000,0.000000,0.000000}%
\pgfsetstrokecolor{currentstroke}%
\pgfsetdash{}{0pt}%
\pgfsys@defobject{currentmarker}{\pgfqpoint{0.000000in}{-0.048611in}}{\pgfqpoint{0.000000in}{0.000000in}}{%
\pgfpathmoveto{\pgfqpoint{0.000000in}{0.000000in}}%
\pgfpathlineto{\pgfqpoint{0.000000in}{-0.048611in}}%
\pgfusepath{stroke,fill}%
}%
\begin{pgfscope}%
\pgfsys@transformshift{2.750629in}{0.576079in}%
\pgfsys@useobject{currentmarker}{}%
\end{pgfscope}%
\end{pgfscope}%
\begin{pgfscope}%
\definecolor{textcolor}{rgb}{0.000000,0.000000,0.000000}%
\pgfsetstrokecolor{textcolor}%
\pgfsetfillcolor{textcolor}%
\pgftext[x=2.750629in,y=0.478857in,,top]{\color{textcolor}\sffamily\fontsize{10.000000}{12.000000}\selectfont \(\displaystyle 1\)}%
\end{pgfscope}%
\begin{pgfscope}%
\definecolor{textcolor}{rgb}{0.000000,0.000000,0.000000}%
\pgfsetstrokecolor{textcolor}%
\pgfsetfillcolor{textcolor}%
\pgftext[x=1.734568in,y=0.284413in,,top]{\color{textcolor}\sffamily\fontsize{10.000000}{12.000000}\selectfont \(\displaystyle x\)}%
\end{pgfscope}%
\begin{pgfscope}%
\pgfpathrectangle{\pgfqpoint{0.619136in}{0.576079in}}{\pgfqpoint{2.230864in}{1.273921in}}%
\pgfusepath{clip}%
\pgfsetrectcap%
\pgfsetroundjoin%
\pgfsetlinewidth{0.803000pt}%
\definecolor{currentstroke}{rgb}{0.690196,0.690196,0.690196}%
\pgfsetstrokecolor{currentstroke}%
\pgfsetdash{}{0pt}%
\pgfpathmoveto{\pgfqpoint{0.619136in}{0.633771in}}%
\pgfpathlineto{\pgfqpoint{2.850000in}{0.633771in}}%
\pgfusepath{stroke}%
\end{pgfscope}%
\begin{pgfscope}%
\pgfsetbuttcap%
\pgfsetroundjoin%
\definecolor{currentfill}{rgb}{0.000000,0.000000,0.000000}%
\pgfsetfillcolor{currentfill}%
\pgfsetlinewidth{0.803000pt}%
\definecolor{currentstroke}{rgb}{0.000000,0.000000,0.000000}%
\pgfsetstrokecolor{currentstroke}%
\pgfsetdash{}{0pt}%
\pgfsys@defobject{currentmarker}{\pgfqpoint{-0.048611in}{0.000000in}}{\pgfqpoint{-0.000000in}{0.000000in}}{%
\pgfpathmoveto{\pgfqpoint{-0.000000in}{0.000000in}}%
\pgfpathlineto{\pgfqpoint{-0.048611in}{0.000000in}}%
\pgfusepath{stroke,fill}%
}%
\begin{pgfscope}%
\pgfsys@transformshift{0.619136in}{0.633771in}%
\pgfsys@useobject{currentmarker}{}%
\end{pgfscope}%
\end{pgfscope}%
\begin{pgfscope}%
\definecolor{textcolor}{rgb}{0.000000,0.000000,0.000000}%
\pgfsetstrokecolor{textcolor}%
\pgfsetfillcolor{textcolor}%
\pgftext[x=0.452469in, y=0.581009in, left, base]{\color{textcolor}\sffamily\fontsize{10.000000}{12.000000}\selectfont \(\displaystyle 0\)}%
\end{pgfscope}%
\begin{pgfscope}%
\pgfpathrectangle{\pgfqpoint{0.619136in}{0.576079in}}{\pgfqpoint{2.230864in}{1.273921in}}%
\pgfusepath{clip}%
\pgfsetrectcap%
\pgfsetroundjoin%
\pgfsetlinewidth{0.803000pt}%
\definecolor{currentstroke}{rgb}{0.690196,0.690196,0.690196}%
\pgfsetstrokecolor{currentstroke}%
\pgfsetdash{}{0pt}%
\pgfpathmoveto{\pgfqpoint{0.619136in}{0.968475in}}%
\pgfpathlineto{\pgfqpoint{2.850000in}{0.968475in}}%
\pgfusepath{stroke}%
\end{pgfscope}%
\begin{pgfscope}%
\pgfsetbuttcap%
\pgfsetroundjoin%
\definecolor{currentfill}{rgb}{0.000000,0.000000,0.000000}%
\pgfsetfillcolor{currentfill}%
\pgfsetlinewidth{0.803000pt}%
\definecolor{currentstroke}{rgb}{0.000000,0.000000,0.000000}%
\pgfsetstrokecolor{currentstroke}%
\pgfsetdash{}{0pt}%
\pgfsys@defobject{currentmarker}{\pgfqpoint{-0.048611in}{0.000000in}}{\pgfqpoint{-0.000000in}{0.000000in}}{%
\pgfpathmoveto{\pgfqpoint{-0.000000in}{0.000000in}}%
\pgfpathlineto{\pgfqpoint{-0.048611in}{0.000000in}}%
\pgfusepath{stroke,fill}%
}%
\begin{pgfscope}%
\pgfsys@transformshift{0.619136in}{0.968475in}%
\pgfsys@useobject{currentmarker}{}%
\end{pgfscope}%
\end{pgfscope}%
\begin{pgfscope}%
\definecolor{textcolor}{rgb}{0.000000,0.000000,0.000000}%
\pgfsetstrokecolor{textcolor}%
\pgfsetfillcolor{textcolor}%
\pgftext[x=0.344444in, y=0.915713in, left, base]{\color{textcolor}\sffamily\fontsize{10.000000}{12.000000}\selectfont \(\displaystyle 0{,}2\)}%
\end{pgfscope}%
\begin{pgfscope}%
\pgfpathrectangle{\pgfqpoint{0.619136in}{0.576079in}}{\pgfqpoint{2.230864in}{1.273921in}}%
\pgfusepath{clip}%
\pgfsetrectcap%
\pgfsetroundjoin%
\pgfsetlinewidth{0.803000pt}%
\definecolor{currentstroke}{rgb}{0.690196,0.690196,0.690196}%
\pgfsetstrokecolor{currentstroke}%
\pgfsetdash{}{0pt}%
\pgfpathmoveto{\pgfqpoint{0.619136in}{1.303179in}}%
\pgfpathlineto{\pgfqpoint{2.850000in}{1.303179in}}%
\pgfusepath{stroke}%
\end{pgfscope}%
\begin{pgfscope}%
\pgfsetbuttcap%
\pgfsetroundjoin%
\definecolor{currentfill}{rgb}{0.000000,0.000000,0.000000}%
\pgfsetfillcolor{currentfill}%
\pgfsetlinewidth{0.803000pt}%
\definecolor{currentstroke}{rgb}{0.000000,0.000000,0.000000}%
\pgfsetstrokecolor{currentstroke}%
\pgfsetdash{}{0pt}%
\pgfsys@defobject{currentmarker}{\pgfqpoint{-0.048611in}{0.000000in}}{\pgfqpoint{-0.000000in}{0.000000in}}{%
\pgfpathmoveto{\pgfqpoint{-0.000000in}{0.000000in}}%
\pgfpathlineto{\pgfqpoint{-0.048611in}{0.000000in}}%
\pgfusepath{stroke,fill}%
}%
\begin{pgfscope}%
\pgfsys@transformshift{0.619136in}{1.303179in}%
\pgfsys@useobject{currentmarker}{}%
\end{pgfscope}%
\end{pgfscope}%
\begin{pgfscope}%
\definecolor{textcolor}{rgb}{0.000000,0.000000,0.000000}%
\pgfsetstrokecolor{textcolor}%
\pgfsetfillcolor{textcolor}%
\pgftext[x=0.344444in, y=1.250418in, left, base]{\color{textcolor}\sffamily\fontsize{10.000000}{12.000000}\selectfont \(\displaystyle 0{,}4\)}%
\end{pgfscope}%
\begin{pgfscope}%
\pgfpathrectangle{\pgfqpoint{0.619136in}{0.576079in}}{\pgfqpoint{2.230864in}{1.273921in}}%
\pgfusepath{clip}%
\pgfsetrectcap%
\pgfsetroundjoin%
\pgfsetlinewidth{0.803000pt}%
\definecolor{currentstroke}{rgb}{0.690196,0.690196,0.690196}%
\pgfsetstrokecolor{currentstroke}%
\pgfsetdash{}{0pt}%
\pgfpathmoveto{\pgfqpoint{0.619136in}{1.637884in}}%
\pgfpathlineto{\pgfqpoint{2.850000in}{1.637884in}}%
\pgfusepath{stroke}%
\end{pgfscope}%
\begin{pgfscope}%
\pgfsetbuttcap%
\pgfsetroundjoin%
\definecolor{currentfill}{rgb}{0.000000,0.000000,0.000000}%
\pgfsetfillcolor{currentfill}%
\pgfsetlinewidth{0.803000pt}%
\definecolor{currentstroke}{rgb}{0.000000,0.000000,0.000000}%
\pgfsetstrokecolor{currentstroke}%
\pgfsetdash{}{0pt}%
\pgfsys@defobject{currentmarker}{\pgfqpoint{-0.048611in}{0.000000in}}{\pgfqpoint{-0.000000in}{0.000000in}}{%
\pgfpathmoveto{\pgfqpoint{-0.000000in}{0.000000in}}%
\pgfpathlineto{\pgfqpoint{-0.048611in}{0.000000in}}%
\pgfusepath{stroke,fill}%
}%
\begin{pgfscope}%
\pgfsys@transformshift{0.619136in}{1.637884in}%
\pgfsys@useobject{currentmarker}{}%
\end{pgfscope}%
\end{pgfscope}%
\begin{pgfscope}%
\definecolor{textcolor}{rgb}{0.000000,0.000000,0.000000}%
\pgfsetstrokecolor{textcolor}%
\pgfsetfillcolor{textcolor}%
\pgftext[x=0.344444in, y=1.585122in, left, base]{\color{textcolor}\sffamily\fontsize{10.000000}{12.000000}\selectfont \(\displaystyle 0{,}6\)}%
\end{pgfscope}%
\begin{pgfscope}%
\definecolor{textcolor}{rgb}{0.000000,0.000000,0.000000}%
\pgfsetstrokecolor{textcolor}%
\pgfsetfillcolor{textcolor}%
\pgftext[x=0.288889in,y=1.213040in,,bottom,rotate=90.000000]{\color{textcolor}\sffamily\fontsize{10.000000}{12.000000}\selectfont \(\displaystyle f(x)\)}%
\end{pgfscope}%
\begin{pgfscope}%
\pgfpathrectangle{\pgfqpoint{0.619136in}{0.576079in}}{\pgfqpoint{2.230864in}{1.273921in}}%
\pgfusepath{clip}%
\pgfsetrectcap%
\pgfsetroundjoin%
\pgfsetlinewidth{1.505625pt}%
\definecolor{currentstroke}{rgb}{0.000000,0.000000,1.000000}%
\pgfsetstrokecolor{currentstroke}%
\pgfsetdash{}{0pt}%
\pgfpathmoveto{\pgfqpoint{0.720539in}{0.633985in}}%
\pgfpathlineto{\pgfqpoint{0.748989in}{0.638700in}}%
\pgfpathlineto{\pgfqpoint{0.793696in}{0.648483in}}%
\pgfpathlineto{\pgfqpoint{0.850595in}{0.663398in}}%
\pgfpathlineto{\pgfqpoint{0.911559in}{0.681670in}}%
\pgfpathlineto{\pgfqpoint{0.980651in}{0.704735in}}%
\pgfpathlineto{\pgfqpoint{1.053807in}{0.731528in}}%
\pgfpathlineto{\pgfqpoint{1.131028in}{0.762174in}}%
\pgfpathlineto{\pgfqpoint{1.212313in}{0.796828in}}%
\pgfpathlineto{\pgfqpoint{1.297662in}{0.835665in}}%
\pgfpathlineto{\pgfqpoint{1.387075in}{0.878871in}}%
\pgfpathlineto{\pgfqpoint{1.476489in}{0.924511in}}%
\pgfpathlineto{\pgfqpoint{1.569966in}{0.974701in}}%
\pgfpathlineto{\pgfqpoint{1.667508in}{1.029651in}}%
\pgfpathlineto{\pgfqpoint{1.765050in}{1.087129in}}%
\pgfpathlineto{\pgfqpoint{1.866656in}{1.149592in}}%
\pgfpathlineto{\pgfqpoint{1.968262in}{1.214610in}}%
\pgfpathlineto{\pgfqpoint{2.073933in}{1.284855in}}%
\pgfpathlineto{\pgfqpoint{2.183667in}{1.360555in}}%
\pgfpathlineto{\pgfqpoint{2.293402in}{1.438986in}}%
\pgfpathlineto{\pgfqpoint{2.407201in}{1.523132in}}%
\pgfpathlineto{\pgfqpoint{2.520999in}{1.610072in}}%
\pgfpathlineto{\pgfqpoint{2.638863in}{1.702997in}}%
\pgfpathlineto{\pgfqpoint{2.748597in}{1.792095in}}%
\pgfpathlineto{\pgfqpoint{2.748597in}{1.792095in}}%
\pgfusepath{stroke}%
\end{pgfscope}%
\begin{pgfscope}%
\pgfsetrectcap%
\pgfsetmiterjoin%
\pgfsetlinewidth{0.803000pt}%
\definecolor{currentstroke}{rgb}{0.000000,0.000000,0.000000}%
\pgfsetstrokecolor{currentstroke}%
\pgfsetdash{}{0pt}%
\pgfpathmoveto{\pgfqpoint{0.619136in}{0.576079in}}%
\pgfpathlineto{\pgfqpoint{0.619136in}{1.850000in}}%
\pgfusepath{stroke}%
\end{pgfscope}%
\begin{pgfscope}%
\pgfsetrectcap%
\pgfsetmiterjoin%
\pgfsetlinewidth{0.803000pt}%
\definecolor{currentstroke}{rgb}{0.000000,0.000000,0.000000}%
\pgfsetstrokecolor{currentstroke}%
\pgfsetdash{}{0pt}%
\pgfpathmoveto{\pgfqpoint{2.850000in}{0.576079in}}%
\pgfpathlineto{\pgfqpoint{2.850000in}{1.850000in}}%
\pgfusepath{stroke}%
\end{pgfscope}%
\begin{pgfscope}%
\pgfsetrectcap%
\pgfsetmiterjoin%
\pgfsetlinewidth{0.803000pt}%
\definecolor{currentstroke}{rgb}{0.000000,0.000000,0.000000}%
\pgfsetstrokecolor{currentstroke}%
\pgfsetdash{}{0pt}%
\pgfpathmoveto{\pgfqpoint{0.619136in}{0.576079in}}%
\pgfpathlineto{\pgfqpoint{2.850000in}{0.576079in}}%
\pgfusepath{stroke}%
\end{pgfscope}%
\begin{pgfscope}%
\pgfsetrectcap%
\pgfsetmiterjoin%
\pgfsetlinewidth{0.803000pt}%
\definecolor{currentstroke}{rgb}{0.000000,0.000000,0.000000}%
\pgfsetstrokecolor{currentstroke}%
\pgfsetdash{}{0pt}%
\pgfpathmoveto{\pgfqpoint{0.619136in}{1.850000in}}%
\pgfpathlineto{\pgfqpoint{2.850000in}{1.850000in}}%
\pgfusepath{stroke}%
\end{pgfscope}%
\end{pgfpicture}%
\makeatother%
\endgroup%

    \caption{Graphe de la fonction $f \colon x \mapsto \int_x^{x^2} \frac{\d t}{\ln(t)}$ sur l'intervalle $\interoo{0}{1}$}
\end{marginfigure}

\begin{prop}
Pour tout $x \in \interff{0}{1}$, on définit $f(x) = \int_x^{x^2} \frac{\d t}{\ln(t)}$.
\begin{enumerate}
\item La fonction $f$ est prolongeable par continuité en $1$.

\item La fonction $f$ est dérivable sur $\interff{0}{1}$.
\end{enumerate}
\end{prop}

\begin{remarque}
La fonction $x \mapsto \int_0^x \frac{1}{\ln(t)} \d t$ est la fonction \textsl{logarithme intégral}.
\end{remarque}

\begin{exercice}
\begin{questions}
\item En utilisant la concavité du logarithme, montrer que
\[
\forall\, t \in \big]x^2\,;1\big],\quad
\frac{\ln(x^2)}{x^2 - 1} (t - 1) \leqslant \ln(t).
\]

\item En déduire que
\[
\frac{(x + 1) (x - 1)}{2 \ln(x)} \ln\module{x + 1} \leqslant f(x) \leqslant \ln\module{x + 1}.
\]

\item Montrer que $f$ est prolongeable en par continuité en $0$.

\item Calculer $f'$ sur $\interof{0}{1}$ puis montrer que $f$ est dérivable en $0$.
\end{questions}
\end{exercice}


\begin{elemsolution}
\begin{reponses}
\item Soit $t \in \interfo{x^2}{1}$. Comme la fonction logarithme est concave, sa courbe représentative se situe en-dessous de sa tangente en $1$ et
\[
\ln(t) - \ln(1) \leqslant t - 1.
\]

\medskip

Toujours par concavité de la fonction logarithme, la corde reliant les points de coordonnées $\big(x^2, \ln\big(x^2\big)\big)$ et $(1, 0)$ se situe en-dessous de la courbe représentant le logarithme, soit :
\[
\frac{\ln(x^2) - \ln(1)}{x^2 - 1} (t - 1) \leqslant \ln(t).
\]

\item En utilisant l'encadrement précédent, pour tout $t \in \interoo{x^2}{1}$,
\begin{align*}
\frac{1}{t - 1} \leqslant \frac{1}{\ln(t)} &\leqslant \frac{x^2 - 1}{2 \ln(x)} \times \frac{1}{t - 1}\\
\frac{x^2 - 1}{2 \ln(x)} \int_x^{x^2} \frac{\d t}{t - 1} &\leqslant f(x) \leqslant \int_x^{x^2} \frac{\d t}{t - 1}\\
\frac{x^2 - 1}{2 \ln(x)} \ln \module{\frac{x^2 - 1}{x - 1}} &\leqslant f(x) \leqslant \ln\module{\frac{x^2 - 1}{x - 1}}\\
\frac{(x + 1) (x - 1)}{2 \ln(x)} \ln\module{x + 1} &\leqslant f(x) \leqslant \ln\module{x + 1}.
\end{align*}

Ainsi, d'après le \theoremeutilise{théorème d'encadrement}{theo:encadrement}, $\lim\limits_{x\to 1} f(x) = \ln(2)$.

\item En utilisant la dérivation par rapport aux bornes,
\begin{align*}
f'(x)
= 2 x \frac{1}{\ln(x^2)} - x \frac{1}{\ln(x)} 
= \frac{x - 1}{\ln(x)}.
\end{align*}

La fonction $f$ est prolongeable par continuité sur $\interof{0}{1}$ et dérivable sur $\interoo{0}{1}$. De plus, $\lim\limits_{x\to 1} f'(x) = 1$. D'après le \theoremeutilise{théorème de prolongement dérivable}{theo:prolongementderivable}, $f$ est dérivable en $1$ et $f'(1) = 1$.
\end{reponses}
\end{elemsolution}

%-----------
\subsection{Un développement asymptotique}

\todoarmand{Lien avec l'exponentielle intégrale}

\begin{prop}
On pose
\[
\fonctionligne[F]{x}{\int_x^{+\infty} \frac{\e^{-t}}{t} \d t}.
\]
Alors,
\[
F(x) = \frac{\e^{-x}}{x} + \frac{\e^{-x}}{x^2} + o_{+\infty}\mathopen{}\bigg(\frac{\e^{-x}}{x^2}\bigg)
\quad \text{et} \quad
F(x) \sim_0 -\ln(x).
\]
\end{prop}

\begin{exercice}
\begin{questions}
\item Déterminer l'ensemble de définition de $F$. Étudier brièvement le comportement de la fonction $F$ et tracer sa courbe représentative.

\item Montrer que $F(x) \sim_{+\infty} \frac{\e^{-x}}{x}$.

\item Montrer que $F(x) = \frac{\e^{-x}}{x} + \frac{\e^{-x}}{x^2} + o_{+\infty}\mathopen{}\left(\frac{\e^{-x}}{x^2}\right)$.
       
\item Montrer que $F(x) \sim_0 -\ln(x)$.
\end{questions}
\end{exercice}

\begin{marginfigure}[-8cm]
    \centering
    \begin{tikzpicture}
\begin{axis}[
    xtick={10,15},
    xticklabels={$10$, $15$},
    ytick={0,0.000005},
    yticklabels={$0$, $5 \cdot 10^{-6}$},
    scaled y ticks=false,
    xlabel=$x$,
    width=7cm,
    axis lines=middle,
    axis line style=thick,
    axis line style={-latex},
    % xlabel style={at={(axis description cs:1.05,0)}, anchor=north},
    % ylabel style={at={(axis description cs:0,1.05)}, anchor=south},
    xmin=9.5,
    xmax=20,
    ymin=0,
    ymax=0.0000055,
    restrict x to domain=10:20,
    legend style={
        draw=none,
        fill=none,
        font=\footnotesize,
        legend image code/.code={\node[anchor=west] {#1};}
    },
    every axis x label/.style={at={(current axis.right of origin)},anchor=north},
]

% Plot from the data file
\draw[color=myred, thick, domain=10:18, samples=100] plot (\x,{exp(-\x)/\x + exp(-\x)/(\x^2)});
\addlegendentry{\textcolor{myred}{$\displaystyle x \mapsto \frac{\e^{-x}}{x} + \frac{\e^{-x}}{x^2}$}}

\addplot[myblue, thick, smooth] table[x=x, y=y] {chapitres/integration/illustrations/i_16-un_developpement_asymptotique.dat};

\end{axis}
\end{tikzpicture}
    \caption{Représentation graphique de la fonction $F$ et des premiers termes de son développement asymptotique en $+\infty$}
\end{marginfigure}
\begin{marginfigure}[-1cm]
    \centering
    \begin{tikzpicture}
\begin{axis}[
    xtick={0.001,0.5},
    xticklabels={$10^{-3}$,$0{,}5$},
    ytick={10},
    yticklabels={$10$},
    scaled y ticks=false,
    xlabel=$x$,
    width=7cm,
    axis lines=middle,
    axis line style=thick,
    axis line style={-latex},
    % xlabel style={at={(axis description cs:1.05,0)}, anchor=north},
    % ylabel style={at={(axis description cs:0,1.05)}, anchor=south},
    xmin=0,
    xmax=0.55,
    ymin=0,
    ymax=11,
    restrict x to domain=0.0001:0.5,
    legend style={
        draw=none,
        fill=none,
        font=\footnotesize,
        legend image code/.code={\node[anchor=west] {#1};}
    },
    every axis x label/.style={at={(current axis.right of origin)},anchor=north},
]

% Plot from the data file
\draw[color=mygreen, thick, domain=0.0001:0.5, samples=100] plot (\x,{-ln(\x)});
\addlegendentry{\textcolor{mygreen}{$x \mapsto -\ln(x)$}}

\addplot[myblue, thick, smooth] table[x=x, y=y] {chapitres/integration/illustrations/i_16-un_developpement_asymptotique.dat};
% \addlegendentry{\textcolor{myblue}{$\displaystyle x \mapsto \int_x^{+\infty} \frac{\e^{-t}}{t} \d t$}}
\end{axis}
\end{tikzpicture}

    \caption{Représentation graphique de la fonction $F$ proche de $0$}
\end{marginfigure}

\begin{elemsolution}
\begin{reponses}
\item La fonction $\fonctionligne[f]{t}{\frac{\e^{-t}}{t}}$ est positive et continue sur $\Re$.

\begin{itemize}
\item D'après le \theoremeutilise{théorème des croissances comparées}{theo:croissancescomparees}, $\frac{\e^{-t}}{t} = o_{+\infty}\mathopen{}\left(\frac{1}{t^2}\right)$. Ainsi, d'après le \theoremeutilise{théorème de comparaison des intégrales de fonctions à valeurs positives}{theo:comparaisonintegralesfonctionsvaleurspositives}, $\int_1^{+\infty} \frac{\e^{-t}}{t} \d t$ converge.

\item Comme $f(t) \sim_0 \frac{1}{t}$, d'après le théorème de comparaison des intégrales de fonctions à valeurs positives, $\int_0^1 \frac{\e^{-t}}{t} \d t$ diverge.
\end{itemize}

Ainsi, le domaine de définition de $F$ est $\interoo{0}{+ \infty}$.

\item Les fonctions $\fonctionligne[u]{t}{-\e^{-t}}$ et $\fonctionligne[v]{t}{\frac{1}{t}}$ sont de classe $\Cont^1$ sur $\interfo{x}{+\infty}$ et $\lim\limits_{t\to+\infty} u(t) v(t) = 0$. Ainsi, d'après le \theoremeutilise{théorème d'intégration par parties}{theo:ippgeneralisees}, pour $x > 1$, 
\begin{align*}
F(x)
&= \int_x^{+ \infty} \frac{\e^{-t}}{t} \d t 
= \frac{\e^{-x}}{x} - \int_x^{+\infty} \frac{\e^{-t}}{t^2} \d t.
\end{align*}

De plus,
\begin{align*}
\int_x^{+\infty} \frac{\e^{-t}}{t^2} \d t
&\leqslant \e^{-x} \int_x^{+\infty} \frac{1}{t^2} \d t
\leqslant \frac{\e^{-x}}{x^2}.
\end{align*}

Ainsi,
$$\frac{\e^{-t}}{t^2} = o_{+\infty}\mathopen{}\bigg( \frac{\e^{-t}}{t} \bigg).$$

D'où,
\[
\int_x^{+\infty} \frac{\e^{-t}}{t} \d t \sim_{+\infty} \frac{\e^{-x}}{x}.
\]

\begin{remarque}
On aurait également pu utiliser les théorèmes d'intégration des relations de comparaison car $\e^{-x}/x = o(\e^{-x}/x^2)$.
\end{remarque}

\item On effectue une nouvelle intégration par parties avec $\fonctionligne[u]{t}{\e^{-t}}$ et $\fonctionligne[v]{t}{\frac{1}{t^2}}$. On obtient ainsi
\begin{align*}
F(x)
&= \frac{\e^{-x}}{x} + \frac{\e^{-x}}{x^2} - \int_x^{+\infty} \frac{2 \e^{-t}}{t^3} \d t.
\end{align*}

On montre alors comme précédemment que l'intégrable est négligeable, en $+\infty$, devant~$\frac{\e^{-x}}{x^2}$.

\item En utilisant la relation de \nom{Chasles},
\begin{align*}
F(x)
&= \int_x^1 \frac{\e^{-t}}{t} \d t + \int_1^{+\infty} \frac{\e^{-t}}{t} \d t.
\end{align*}

De plus, la fonction exponentielle étant convexe, pour tout $t > 0$,
\begin{align*}
1 - t &\leqslant \e^{-t} \leqslant 1\\
\frac{1}{t} - 1 &\leqslant \frac{\e^{-t}}{t} \leqslant \frac{1}{t}\\
-\ln(x) - (1 - x) &\leqslant \int_x^1 \frac{\e^{-t}}{t} \d t \leqslant -\ln(x).
\end{align*}

D'après le \theoremeutilise{théorème d'encadrement}{theo:encadrement}, $\int_x^1 \frac{\e^{-t}}{t} \d t \sim_0 -\ln(x)$ et
\[
F(x) \sim_0 -\ln(x).
\]
\end{reponses}
\end{elemsolution}
\section{Intégrale à paramètre dans les bornes}

\begin{itemize}
    \item On pose $f:x \mapsto \int_{x}^{x^2} \frac{\d t}{\ln (t)}$.
    $$\displaystyle f(x) = \int_{x}^{x^2} \frac{t}{t \ln (t)} \d t \leqslant x^2 \int_{x}^{x^2} \frac{\d t}{t \ln(t)} = x^2 \bigl[ \ln (\ln (t)) \bigr]_x^{x^2} = x^2 \ln(2).$$
\end{itemize}

\section{Exercice oral}
\begin{exercice}
   On pose
   $$F:x \mapsto \int_x^{+\infty} \frac{\e^{-t}}{t} \d t.$$
   \begin{enumerate}
       \item Déterminer l'ensemble de définition de $F$. Étudier brièvement le comportement de la fonction $F$ et tracer sa courbe représentative.
       \item Déterminer un  équivalent de $F$ en $+\infty$.
       \item Montrer que $F(1) - F(x) - \ln x$ converge vers un réel. (pas sûr de cette question).
   \end{enumerate}
\end{exercice}

\begin{enumerate}
   \item $D = ]0, + \infty[$.
   \item Intégration par parties ou comparaison série / intégrale: $F(x) \sim_{+\infty} \frac{\e^{x}}{x}$. \\
   \url{https://www.bibmath.net/ressources/index.php?action=affiche&quoi=bde/analyse/integration/integralesimpropres&type=fexo} exercice 38. (à réécrire)\\
   On remarque d'abord que $\int_{1}^{+\infty} \frac{\e^{-t}}{t} \d t$ converge: en effet, la fonction $t \mapsto \frac{\e^{-t}}{t}$ est continue et positive sur $[1, + \infty[$ et $\lim\limits_{t \to +\infty} t^2 \frac{\e^{-t}}{t} = 0$. On intègre ensuite par parties, en intégrant $t \mapsto \e^{-t}$ et en dérivant $t \mapsto \frac{1}{t}$. On obtient, pour $x > 1$, 
   $$\int_x^{+ \infty} \frac{\e^{-t}}{t} \d t = \left[ -\frac{\e^{-t}}{t} \right]_x^{+\infty} - \int_x^{+\infty} \frac{\e^{-t}}{t^2} \d t$$
   $$\int_x^{+ \infty} \frac{\e^{-t}}{t} \d t= \frac{\e^{-x}}{x} - \int_x^{+\infty} \frac{\e^{-t}}{t^2} \d t.$$
   Or, au voisinage de $+ \infty$, 
   $$\frac{\e^{-t}}{t^2} = o\left( \frac{\e^{-t}}{t} \right).$$
   Par intégration des relations de comparaison (les fonctions sont positives et intégrables), on trouve
   $$\int_x^{+\infty} \frac{\e^{-t}}{t^2} \d t = o_{+\infty} \left( \int_x^{+\infty} \frac{\e^{-t}}{t} \d t \right).$$
   On en déduit que
   $$\int_x^{+\infty} \frac{\e^{-t}}{t} \d t \sim_{+\infty} \frac{\e^{-x}}{x}.$$
   \item À faire.
\end{enumerate}

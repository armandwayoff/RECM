\section{Intégrale à paramètre dans les bornes}

\section{Autour du logarithme intégral}

\todoinline{Ajouter une représentation graphique de $f$.}

\begin{prop}
Pour tout $x \in ]0, 1[$, on définit $f(x) = \int_x^{x^2} \frac{\d t}{\ln(t)}$.
\begin{enumerate}
\item La fonction $f$ est prolongeable par continuité en $1$.

\item La fonction $f$ est dérivable sur $[0, 1]$.
\end{enumerate}
\end{prop}

\begin{remarque}
La fonction $x \mapsto \int_0^x \frac{1}{\ln(t)} \d t$ est la fonction logarithme intégral.
\end{remarque}

\begin{elem_sol}
\begin{enumerate}
\item Soit $t \in ]x^2, 1]$. Comme la fonction logarithme est concave, sa courbe représentative se situe en-dessous de sa tangente en $1$ et
\[
\ln(t) - \ln(1) \leq t - 1.
\]

\medskip

Toujours par concavité de la fonction logarithme, la corde reliant les points de coordonnées $(x^2, \ln(x^2))$ et $(1, 0)$ se situe en-dessous de la courbe représentant le logarithme, soit :
\[
\frac{\ln(x^2) - \ln(1)}{x^2 - 1} (t - 1) \leq \ln(t).
\]

\item En utilisant l'encadrement précédent, pour tout $t \in ]x^2, 1[$,
\begin{align*}
\frac{1}{t - 1} \leq \frac{1}{\ln(t)} &\leq \frac{x^2 - 1}{2 \ln(x)} \times \frac{1}{t - 1}\\
\frac{x^2 - 1}{2 \ln(x)} \int_x^{x^2} \frac{\d t}{t - 1} &\leq f(x) \leq \int_x^{x^2} \frac{\d t}{t - 1}\\
\frac{x^2 - 1}{2 \ln(x)} \ln \abs{\frac{x^2 - 1}{x - 1}} &\leq f(x) \leq \ln\abs{\frac{x^2 - 1}{x - 1}}\\
\frac{(x + 1) (x - 1)}{2 \ln(x)} \ln\abs{x + 1} &\leq f(x) \leq \ln\abs{x + 1}.
\end{align*}

Ainsi, d'après le théorème d'encadrement, $\lim_{x\to 1} f(x) = \ln(2)$.

\item En utilisant la dérivation par rapport aux bornes,
\begin{align*}
f'(x)
&= 2 x \frac{1}{\ln(x^2)} - x \frac{1}{\ln(x)}\\
&= \frac{x - 1}{\ln(x)}.
\end{align*}

La fonction $f$ est prolongeable par continuité sur $]0, 1]$ et dérivable sur $]0, 1[$. De plus, $\lim_{x\to 1} f'(x) = 1$. D'après le théorème de prolongement dérivable, $f$ est dérivable en $1$ et $f'(1) = 1$.
\end{enumerate}
\end{elem_sol}

%========
\section{Un développement asymptotique}

\todoinline{Ajouter une représentation graphique de $F$ et des premiers termes du développement ?}

\begin{exercice}
   On pose
   $$F:x \mapsto \int_x^{+\infty} \frac{\e^{-t}}{t} \d t.$$
\begin{enumerate}
\item Déterminer l'ensemble de définition de $F$. Étudier brièvement le comportement de la fonction $F$ et tracer sa courbe représentative.

\item Montrer que $F(x) \sim_{+\infty} \frac{\e^{-x}}{x}$.

\item Montrer que $F(x) = \frac{\e^{-x}}{x} + \frac{\e^{-x}}{x^2} + o_{+\infty}\left(\frac{\e^{-x}}{x^2}\right)$.
       
\item Montrer que $F(x) \sim_0 \ln(x)$.
\end{enumerate}
\end{exercice}

\begin{elem_sol}
\begin{enumerate}
\item   La fonction $f : t \mapsto \frac{\e^{-t}}{t}$ est positive et continue sur $\R^*$.

\begin{itemize}
\item D'après le théorème des croissances comparées, $\frac{\e^{-t}}{t} = o_{+\infty}\left(\frac{1}{t^2}\right)$. Ainsi, d'après le théorème de comparaison des intégrales de fonctions à valeurs positives, $\int_1^{+\infty} \frac{\e^{-t}}{t} \d t$ converge.

\item Comme $f(t) \sim_0 \frac{1}{t}$, d'après le théorème de comparaison des intégrales de fonctions à valeurs positives, $\int_0^1 \frac{\e^{-t}}{t} \d t$ diverge.
\end{itemize}

Ainsi, le domaine de définition de $F$ est $]0, + \infty[$.

Les fonctions $u : t \mapsto -\e^{-t}$ et $v : t \mapsto \frac{1}{t}$ sont de classe $\mathscr{C}^1$ sur $\R_+^*$ et $\lim_{t\to+\infty} u(t) v(t) = 0$. Ainsi, d'après le théorème d'intégration par parties, pour $x > 1$, 
\begin{align*}
F(x)
&= \int_x^{+ \infty} \frac{\e^{-t}}{t} \d t 
= \frac{\e^{-x}}{x} - \int_x^{+\infty} \frac{\e^{-t}}{t^2} \d t.
\end{align*}

De plus,
\begin{align*}
\int_x^{+\infty} \frac{\e^{-t}}{t^2} \d t
&\leq \e^{-x} \int_x^{+\infty} \frac{1}{t^2} \d t
\leq \frac{\e^{-x}}{x^2}.
\end{align*}

Ainsi,
$$\frac{\e^{-t}}{t^2} = o_{+\infty}\left( \frac{\e^{-t}}{t} \right).$$

D'où,
\[
\int_x^{+\infty} \frac{\e^{-t}}{t} \d t \sim_{+\infty} \frac{\e^{-x}}{x}.
\]

\begin{remarque}
On aurait également pu utiliser les théorèmes d'intégration des relations de comparaison car $\e^{-x}/x = o(\e^{-x}/x^2)$.
\end{remarque}

\item On effectue une nouvelle intégration par parties avec $u : t \mapsto \e^{-t}$ et $v : t \mapsto \frac{1}{t^2}$. On obtient ainsi
\begin{align*}
F(x)
&= \frac{\e^{-x}}{x} + \frac{\e^{-x}}{x^2} - \int_x^{+\infty} \frac{2 \e^{-t}}{t^3} \d t.
\end{align*}

On montre alors comme précédemment que l'intégrable est négligeable, en $+\infty$, devant $\frac{\e^{-x}}{x^2}$.

\item En utilisant la relation de Chasles,
\begin{align*}
F(x)
&= \int_x^1 \frac{\e^{-t}}{t} \d t + \int_1^{+\infty} \frac{\e^{-t}}{t} \d t.
\end{align*}

De plus, la fonction exponentielle étant convexe, pour tout $t > 0$,
\begin{align*}
1 - t \leq \e^{-t} &\leq 1\\
\frac{1}{t} - 1 &\leq \frac{\e^{-t}}{t} \leq \frac{1}{t}\\
\ln(x) - (1 - x) &\leq \int_x^1 \frac{\e^{-t}}{t} \d t \leq \ln(x).
\end{align*}

D'après le théorème d'encadrement, $\int_x^1 \frac{\e^{-t}}{t} \d t \sim_0 \ln(x)$ et
\[
F(x) \sim_0 \ln(x).
\]
\end{enumerate}
\end{elem_sol}
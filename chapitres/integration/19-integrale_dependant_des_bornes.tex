\section{Intégrales dépendant des bornes}

%-----------
\subsection{Une modification du logarithme intégral}
\todoarmand{Ajouter une note sur sa signification en théorie des nombres pour compter les nombres premiers.}

\begin{marginfigure}[-2cm]
    \centering
    %% Creator: Matplotlib, PGF backend
%%
%% To include the figure in your LaTeX document, write
%%   \input{<filename>.pgf}
%%
%% Make sure the required packages are loaded in your preamble
%%   \usepackage{pgf}
%%
%% Also ensure that all the required font packages are loaded; for instance,
%% the lmodern package is sometimes necessary when using math font.
%%   \usepackage{lmodern}
%%
%% Figures using additional raster images can only be included by \input if
%% they are in the same directory as the main LaTeX file. For loading figures
%% from other directories you can use the `import` package
%%   \usepackage{import}
%%
%% and then include the figures with
%%   \import{<path to file>}{<filename>.pgf}
%%
%% Matplotlib used the following preamble
%%   
%%   \usepackage{fontspec}
%%   \setmainfont{DejaVuSerif.ttf}[Path=\detokenize{/home/wayoff/.pyenv/versions/3.8.10/lib/python3.8/site-packages/matplotlib/mpl-data/fonts/ttf/}]
%%   \setsansfont{DejaVuSans.ttf}[Path=\detokenize{/home/wayoff/.pyenv/versions/3.8.10/lib/python3.8/site-packages/matplotlib/mpl-data/fonts/ttf/}]
%%   \setmonofont{DejaVuSansMono.ttf}[Path=\detokenize{/home/wayoff/.pyenv/versions/3.8.10/lib/python3.8/site-packages/matplotlib/mpl-data/fonts/ttf/}]
%%   \makeatletter\@ifpackageloaded{underscore}{}{\usepackage[strings]{underscore}}\makeatother
%%
\begingroup%
\makeatletter%
\begin{pgfpicture}%
\pgfpathrectangle{\pgfpointorigin}{\pgfqpoint{3.000000in}{2.000000in}}%
\pgfusepath{use as bounding box, clip}%
\begin{pgfscope}%
\pgfsetbuttcap%
\pgfsetmiterjoin%
\definecolor{currentfill}{rgb}{1.000000,1.000000,1.000000}%
\pgfsetfillcolor{currentfill}%
\pgfsetlinewidth{0.000000pt}%
\definecolor{currentstroke}{rgb}{1.000000,1.000000,1.000000}%
\pgfsetstrokecolor{currentstroke}%
\pgfsetdash{}{0pt}%
\pgfpathmoveto{\pgfqpoint{0.000000in}{0.000000in}}%
\pgfpathlineto{\pgfqpoint{3.000000in}{0.000000in}}%
\pgfpathlineto{\pgfqpoint{3.000000in}{2.000000in}}%
\pgfpathlineto{\pgfqpoint{0.000000in}{2.000000in}}%
\pgfpathlineto{\pgfqpoint{0.000000in}{0.000000in}}%
\pgfpathclose%
\pgfusepath{fill}%
\end{pgfscope}%
\begin{pgfscope}%
\pgfsetbuttcap%
\pgfsetmiterjoin%
\definecolor{currentfill}{rgb}{1.000000,1.000000,1.000000}%
\pgfsetfillcolor{currentfill}%
\pgfsetlinewidth{0.000000pt}%
\definecolor{currentstroke}{rgb}{0.000000,0.000000,0.000000}%
\pgfsetstrokecolor{currentstroke}%
\pgfsetstrokeopacity{0.000000}%
\pgfsetdash{}{0pt}%
\pgfpathmoveto{\pgfqpoint{0.619136in}{0.576079in}}%
\pgfpathlineto{\pgfqpoint{2.850000in}{0.576079in}}%
\pgfpathlineto{\pgfqpoint{2.850000in}{1.850000in}}%
\pgfpathlineto{\pgfqpoint{0.619136in}{1.850000in}}%
\pgfpathlineto{\pgfqpoint{0.619136in}{0.576079in}}%
\pgfpathclose%
\pgfusepath{fill}%
\end{pgfscope}%
\begin{pgfscope}%
\pgfpathrectangle{\pgfqpoint{0.619136in}{0.576079in}}{\pgfqpoint{2.230864in}{1.273921in}}%
\pgfusepath{clip}%
\pgfsetrectcap%
\pgfsetroundjoin%
\pgfsetlinewidth{0.803000pt}%
\definecolor{currentstroke}{rgb}{0.690196,0.690196,0.690196}%
\pgfsetstrokecolor{currentstroke}%
\pgfsetdash{}{0pt}%
\pgfpathmoveto{\pgfqpoint{0.718507in}{0.576079in}}%
\pgfpathlineto{\pgfqpoint{0.718507in}{1.850000in}}%
\pgfusepath{stroke}%
\end{pgfscope}%
\begin{pgfscope}%
\pgfsetbuttcap%
\pgfsetroundjoin%
\definecolor{currentfill}{rgb}{0.000000,0.000000,0.000000}%
\pgfsetfillcolor{currentfill}%
\pgfsetlinewidth{0.803000pt}%
\definecolor{currentstroke}{rgb}{0.000000,0.000000,0.000000}%
\pgfsetstrokecolor{currentstroke}%
\pgfsetdash{}{0pt}%
\pgfsys@defobject{currentmarker}{\pgfqpoint{0.000000in}{-0.048611in}}{\pgfqpoint{0.000000in}{0.000000in}}{%
\pgfpathmoveto{\pgfqpoint{0.000000in}{0.000000in}}%
\pgfpathlineto{\pgfqpoint{0.000000in}{-0.048611in}}%
\pgfusepath{stroke,fill}%
}%
\begin{pgfscope}%
\pgfsys@transformshift{0.718507in}{0.576079in}%
\pgfsys@useobject{currentmarker}{}%
\end{pgfscope}%
\end{pgfscope}%
\begin{pgfscope}%
\definecolor{textcolor}{rgb}{0.000000,0.000000,0.000000}%
\pgfsetstrokecolor{textcolor}%
\pgfsetfillcolor{textcolor}%
\pgftext[x=0.718507in,y=0.478857in,,top]{\color{textcolor}\sffamily\fontsize{10.000000}{12.000000}\selectfont \(\displaystyle 0\)}%
\end{pgfscope}%
\begin{pgfscope}%
\pgfpathrectangle{\pgfqpoint{0.619136in}{0.576079in}}{\pgfqpoint{2.230864in}{1.273921in}}%
\pgfusepath{clip}%
\pgfsetrectcap%
\pgfsetroundjoin%
\pgfsetlinewidth{0.803000pt}%
\definecolor{currentstroke}{rgb}{0.690196,0.690196,0.690196}%
\pgfsetstrokecolor{currentstroke}%
\pgfsetdash{}{0pt}%
\pgfpathmoveto{\pgfqpoint{1.226538in}{0.576079in}}%
\pgfpathlineto{\pgfqpoint{1.226538in}{1.850000in}}%
\pgfusepath{stroke}%
\end{pgfscope}%
\begin{pgfscope}%
\pgfsetbuttcap%
\pgfsetroundjoin%
\definecolor{currentfill}{rgb}{0.000000,0.000000,0.000000}%
\pgfsetfillcolor{currentfill}%
\pgfsetlinewidth{0.803000pt}%
\definecolor{currentstroke}{rgb}{0.000000,0.000000,0.000000}%
\pgfsetstrokecolor{currentstroke}%
\pgfsetdash{}{0pt}%
\pgfsys@defobject{currentmarker}{\pgfqpoint{0.000000in}{-0.048611in}}{\pgfqpoint{0.000000in}{0.000000in}}{%
\pgfpathmoveto{\pgfqpoint{0.000000in}{0.000000in}}%
\pgfpathlineto{\pgfqpoint{0.000000in}{-0.048611in}}%
\pgfusepath{stroke,fill}%
}%
\begin{pgfscope}%
\pgfsys@transformshift{1.226538in}{0.576079in}%
\pgfsys@useobject{currentmarker}{}%
\end{pgfscope}%
\end{pgfscope}%
\begin{pgfscope}%
\definecolor{textcolor}{rgb}{0.000000,0.000000,0.000000}%
\pgfsetstrokecolor{textcolor}%
\pgfsetfillcolor{textcolor}%
\pgftext[x=1.226538in,y=0.478857in,,top]{\color{textcolor}\sffamily\fontsize{10.000000}{12.000000}\selectfont \(\displaystyle 1/4\)}%
\end{pgfscope}%
\begin{pgfscope}%
\pgfpathrectangle{\pgfqpoint{0.619136in}{0.576079in}}{\pgfqpoint{2.230864in}{1.273921in}}%
\pgfusepath{clip}%
\pgfsetrectcap%
\pgfsetroundjoin%
\pgfsetlinewidth{0.803000pt}%
\definecolor{currentstroke}{rgb}{0.690196,0.690196,0.690196}%
\pgfsetstrokecolor{currentstroke}%
\pgfsetdash{}{0pt}%
\pgfpathmoveto{\pgfqpoint{1.734568in}{0.576079in}}%
\pgfpathlineto{\pgfqpoint{1.734568in}{1.850000in}}%
\pgfusepath{stroke}%
\end{pgfscope}%
\begin{pgfscope}%
\pgfsetbuttcap%
\pgfsetroundjoin%
\definecolor{currentfill}{rgb}{0.000000,0.000000,0.000000}%
\pgfsetfillcolor{currentfill}%
\pgfsetlinewidth{0.803000pt}%
\definecolor{currentstroke}{rgb}{0.000000,0.000000,0.000000}%
\pgfsetstrokecolor{currentstroke}%
\pgfsetdash{}{0pt}%
\pgfsys@defobject{currentmarker}{\pgfqpoint{0.000000in}{-0.048611in}}{\pgfqpoint{0.000000in}{0.000000in}}{%
\pgfpathmoveto{\pgfqpoint{0.000000in}{0.000000in}}%
\pgfpathlineto{\pgfqpoint{0.000000in}{-0.048611in}}%
\pgfusepath{stroke,fill}%
}%
\begin{pgfscope}%
\pgfsys@transformshift{1.734568in}{0.576079in}%
\pgfsys@useobject{currentmarker}{}%
\end{pgfscope}%
\end{pgfscope}%
\begin{pgfscope}%
\definecolor{textcolor}{rgb}{0.000000,0.000000,0.000000}%
\pgfsetstrokecolor{textcolor}%
\pgfsetfillcolor{textcolor}%
\pgftext[x=1.734568in,y=0.478857in,,top]{\color{textcolor}\sffamily\fontsize{10.000000}{12.000000}\selectfont \(\displaystyle 1/2\)}%
\end{pgfscope}%
\begin{pgfscope}%
\pgfpathrectangle{\pgfqpoint{0.619136in}{0.576079in}}{\pgfqpoint{2.230864in}{1.273921in}}%
\pgfusepath{clip}%
\pgfsetrectcap%
\pgfsetroundjoin%
\pgfsetlinewidth{0.803000pt}%
\definecolor{currentstroke}{rgb}{0.690196,0.690196,0.690196}%
\pgfsetstrokecolor{currentstroke}%
\pgfsetdash{}{0pt}%
\pgfpathmoveto{\pgfqpoint{2.242599in}{0.576079in}}%
\pgfpathlineto{\pgfqpoint{2.242599in}{1.850000in}}%
\pgfusepath{stroke}%
\end{pgfscope}%
\begin{pgfscope}%
\pgfsetbuttcap%
\pgfsetroundjoin%
\definecolor{currentfill}{rgb}{0.000000,0.000000,0.000000}%
\pgfsetfillcolor{currentfill}%
\pgfsetlinewidth{0.803000pt}%
\definecolor{currentstroke}{rgb}{0.000000,0.000000,0.000000}%
\pgfsetstrokecolor{currentstroke}%
\pgfsetdash{}{0pt}%
\pgfsys@defobject{currentmarker}{\pgfqpoint{0.000000in}{-0.048611in}}{\pgfqpoint{0.000000in}{0.000000in}}{%
\pgfpathmoveto{\pgfqpoint{0.000000in}{0.000000in}}%
\pgfpathlineto{\pgfqpoint{0.000000in}{-0.048611in}}%
\pgfusepath{stroke,fill}%
}%
\begin{pgfscope}%
\pgfsys@transformshift{2.242599in}{0.576079in}%
\pgfsys@useobject{currentmarker}{}%
\end{pgfscope}%
\end{pgfscope}%
\begin{pgfscope}%
\definecolor{textcolor}{rgb}{0.000000,0.000000,0.000000}%
\pgfsetstrokecolor{textcolor}%
\pgfsetfillcolor{textcolor}%
\pgftext[x=2.242599in,y=0.478857in,,top]{\color{textcolor}\sffamily\fontsize{10.000000}{12.000000}\selectfont \(\displaystyle 3/4\)}%
\end{pgfscope}%
\begin{pgfscope}%
\pgfpathrectangle{\pgfqpoint{0.619136in}{0.576079in}}{\pgfqpoint{2.230864in}{1.273921in}}%
\pgfusepath{clip}%
\pgfsetrectcap%
\pgfsetroundjoin%
\pgfsetlinewidth{0.803000pt}%
\definecolor{currentstroke}{rgb}{0.690196,0.690196,0.690196}%
\pgfsetstrokecolor{currentstroke}%
\pgfsetdash{}{0pt}%
\pgfpathmoveto{\pgfqpoint{2.750629in}{0.576079in}}%
\pgfpathlineto{\pgfqpoint{2.750629in}{1.850000in}}%
\pgfusepath{stroke}%
\end{pgfscope}%
\begin{pgfscope}%
\pgfsetbuttcap%
\pgfsetroundjoin%
\definecolor{currentfill}{rgb}{0.000000,0.000000,0.000000}%
\pgfsetfillcolor{currentfill}%
\pgfsetlinewidth{0.803000pt}%
\definecolor{currentstroke}{rgb}{0.000000,0.000000,0.000000}%
\pgfsetstrokecolor{currentstroke}%
\pgfsetdash{}{0pt}%
\pgfsys@defobject{currentmarker}{\pgfqpoint{0.000000in}{-0.048611in}}{\pgfqpoint{0.000000in}{0.000000in}}{%
\pgfpathmoveto{\pgfqpoint{0.000000in}{0.000000in}}%
\pgfpathlineto{\pgfqpoint{0.000000in}{-0.048611in}}%
\pgfusepath{stroke,fill}%
}%
\begin{pgfscope}%
\pgfsys@transformshift{2.750629in}{0.576079in}%
\pgfsys@useobject{currentmarker}{}%
\end{pgfscope}%
\end{pgfscope}%
\begin{pgfscope}%
\definecolor{textcolor}{rgb}{0.000000,0.000000,0.000000}%
\pgfsetstrokecolor{textcolor}%
\pgfsetfillcolor{textcolor}%
\pgftext[x=2.750629in,y=0.478857in,,top]{\color{textcolor}\sffamily\fontsize{10.000000}{12.000000}\selectfont \(\displaystyle 1\)}%
\end{pgfscope}%
\begin{pgfscope}%
\definecolor{textcolor}{rgb}{0.000000,0.000000,0.000000}%
\pgfsetstrokecolor{textcolor}%
\pgfsetfillcolor{textcolor}%
\pgftext[x=1.734568in,y=0.284413in,,top]{\color{textcolor}\sffamily\fontsize{10.000000}{12.000000}\selectfont \(\displaystyle x\)}%
\end{pgfscope}%
\begin{pgfscope}%
\pgfpathrectangle{\pgfqpoint{0.619136in}{0.576079in}}{\pgfqpoint{2.230864in}{1.273921in}}%
\pgfusepath{clip}%
\pgfsetrectcap%
\pgfsetroundjoin%
\pgfsetlinewidth{0.803000pt}%
\definecolor{currentstroke}{rgb}{0.690196,0.690196,0.690196}%
\pgfsetstrokecolor{currentstroke}%
\pgfsetdash{}{0pt}%
\pgfpathmoveto{\pgfqpoint{0.619136in}{0.633771in}}%
\pgfpathlineto{\pgfqpoint{2.850000in}{0.633771in}}%
\pgfusepath{stroke}%
\end{pgfscope}%
\begin{pgfscope}%
\pgfsetbuttcap%
\pgfsetroundjoin%
\definecolor{currentfill}{rgb}{0.000000,0.000000,0.000000}%
\pgfsetfillcolor{currentfill}%
\pgfsetlinewidth{0.803000pt}%
\definecolor{currentstroke}{rgb}{0.000000,0.000000,0.000000}%
\pgfsetstrokecolor{currentstroke}%
\pgfsetdash{}{0pt}%
\pgfsys@defobject{currentmarker}{\pgfqpoint{-0.048611in}{0.000000in}}{\pgfqpoint{-0.000000in}{0.000000in}}{%
\pgfpathmoveto{\pgfqpoint{-0.000000in}{0.000000in}}%
\pgfpathlineto{\pgfqpoint{-0.048611in}{0.000000in}}%
\pgfusepath{stroke,fill}%
}%
\begin{pgfscope}%
\pgfsys@transformshift{0.619136in}{0.633771in}%
\pgfsys@useobject{currentmarker}{}%
\end{pgfscope}%
\end{pgfscope}%
\begin{pgfscope}%
\definecolor{textcolor}{rgb}{0.000000,0.000000,0.000000}%
\pgfsetstrokecolor{textcolor}%
\pgfsetfillcolor{textcolor}%
\pgftext[x=0.452469in, y=0.581009in, left, base]{\color{textcolor}\sffamily\fontsize{10.000000}{12.000000}\selectfont \(\displaystyle 0\)}%
\end{pgfscope}%
\begin{pgfscope}%
\pgfpathrectangle{\pgfqpoint{0.619136in}{0.576079in}}{\pgfqpoint{2.230864in}{1.273921in}}%
\pgfusepath{clip}%
\pgfsetrectcap%
\pgfsetroundjoin%
\pgfsetlinewidth{0.803000pt}%
\definecolor{currentstroke}{rgb}{0.690196,0.690196,0.690196}%
\pgfsetstrokecolor{currentstroke}%
\pgfsetdash{}{0pt}%
\pgfpathmoveto{\pgfqpoint{0.619136in}{0.968475in}}%
\pgfpathlineto{\pgfqpoint{2.850000in}{0.968475in}}%
\pgfusepath{stroke}%
\end{pgfscope}%
\begin{pgfscope}%
\pgfsetbuttcap%
\pgfsetroundjoin%
\definecolor{currentfill}{rgb}{0.000000,0.000000,0.000000}%
\pgfsetfillcolor{currentfill}%
\pgfsetlinewidth{0.803000pt}%
\definecolor{currentstroke}{rgb}{0.000000,0.000000,0.000000}%
\pgfsetstrokecolor{currentstroke}%
\pgfsetdash{}{0pt}%
\pgfsys@defobject{currentmarker}{\pgfqpoint{-0.048611in}{0.000000in}}{\pgfqpoint{-0.000000in}{0.000000in}}{%
\pgfpathmoveto{\pgfqpoint{-0.000000in}{0.000000in}}%
\pgfpathlineto{\pgfqpoint{-0.048611in}{0.000000in}}%
\pgfusepath{stroke,fill}%
}%
\begin{pgfscope}%
\pgfsys@transformshift{0.619136in}{0.968475in}%
\pgfsys@useobject{currentmarker}{}%
\end{pgfscope}%
\end{pgfscope}%
\begin{pgfscope}%
\definecolor{textcolor}{rgb}{0.000000,0.000000,0.000000}%
\pgfsetstrokecolor{textcolor}%
\pgfsetfillcolor{textcolor}%
\pgftext[x=0.344444in, y=0.915713in, left, base]{\color{textcolor}\sffamily\fontsize{10.000000}{12.000000}\selectfont \(\displaystyle 0{,}2\)}%
\end{pgfscope}%
\begin{pgfscope}%
\pgfpathrectangle{\pgfqpoint{0.619136in}{0.576079in}}{\pgfqpoint{2.230864in}{1.273921in}}%
\pgfusepath{clip}%
\pgfsetrectcap%
\pgfsetroundjoin%
\pgfsetlinewidth{0.803000pt}%
\definecolor{currentstroke}{rgb}{0.690196,0.690196,0.690196}%
\pgfsetstrokecolor{currentstroke}%
\pgfsetdash{}{0pt}%
\pgfpathmoveto{\pgfqpoint{0.619136in}{1.303179in}}%
\pgfpathlineto{\pgfqpoint{2.850000in}{1.303179in}}%
\pgfusepath{stroke}%
\end{pgfscope}%
\begin{pgfscope}%
\pgfsetbuttcap%
\pgfsetroundjoin%
\definecolor{currentfill}{rgb}{0.000000,0.000000,0.000000}%
\pgfsetfillcolor{currentfill}%
\pgfsetlinewidth{0.803000pt}%
\definecolor{currentstroke}{rgb}{0.000000,0.000000,0.000000}%
\pgfsetstrokecolor{currentstroke}%
\pgfsetdash{}{0pt}%
\pgfsys@defobject{currentmarker}{\pgfqpoint{-0.048611in}{0.000000in}}{\pgfqpoint{-0.000000in}{0.000000in}}{%
\pgfpathmoveto{\pgfqpoint{-0.000000in}{0.000000in}}%
\pgfpathlineto{\pgfqpoint{-0.048611in}{0.000000in}}%
\pgfusepath{stroke,fill}%
}%
\begin{pgfscope}%
\pgfsys@transformshift{0.619136in}{1.303179in}%
\pgfsys@useobject{currentmarker}{}%
\end{pgfscope}%
\end{pgfscope}%
\begin{pgfscope}%
\definecolor{textcolor}{rgb}{0.000000,0.000000,0.000000}%
\pgfsetstrokecolor{textcolor}%
\pgfsetfillcolor{textcolor}%
\pgftext[x=0.344444in, y=1.250418in, left, base]{\color{textcolor}\sffamily\fontsize{10.000000}{12.000000}\selectfont \(\displaystyle 0{,}4\)}%
\end{pgfscope}%
\begin{pgfscope}%
\pgfpathrectangle{\pgfqpoint{0.619136in}{0.576079in}}{\pgfqpoint{2.230864in}{1.273921in}}%
\pgfusepath{clip}%
\pgfsetrectcap%
\pgfsetroundjoin%
\pgfsetlinewidth{0.803000pt}%
\definecolor{currentstroke}{rgb}{0.690196,0.690196,0.690196}%
\pgfsetstrokecolor{currentstroke}%
\pgfsetdash{}{0pt}%
\pgfpathmoveto{\pgfqpoint{0.619136in}{1.637884in}}%
\pgfpathlineto{\pgfqpoint{2.850000in}{1.637884in}}%
\pgfusepath{stroke}%
\end{pgfscope}%
\begin{pgfscope}%
\pgfsetbuttcap%
\pgfsetroundjoin%
\definecolor{currentfill}{rgb}{0.000000,0.000000,0.000000}%
\pgfsetfillcolor{currentfill}%
\pgfsetlinewidth{0.803000pt}%
\definecolor{currentstroke}{rgb}{0.000000,0.000000,0.000000}%
\pgfsetstrokecolor{currentstroke}%
\pgfsetdash{}{0pt}%
\pgfsys@defobject{currentmarker}{\pgfqpoint{-0.048611in}{0.000000in}}{\pgfqpoint{-0.000000in}{0.000000in}}{%
\pgfpathmoveto{\pgfqpoint{-0.000000in}{0.000000in}}%
\pgfpathlineto{\pgfqpoint{-0.048611in}{0.000000in}}%
\pgfusepath{stroke,fill}%
}%
\begin{pgfscope}%
\pgfsys@transformshift{0.619136in}{1.637884in}%
\pgfsys@useobject{currentmarker}{}%
\end{pgfscope}%
\end{pgfscope}%
\begin{pgfscope}%
\definecolor{textcolor}{rgb}{0.000000,0.000000,0.000000}%
\pgfsetstrokecolor{textcolor}%
\pgfsetfillcolor{textcolor}%
\pgftext[x=0.344444in, y=1.585122in, left, base]{\color{textcolor}\sffamily\fontsize{10.000000}{12.000000}\selectfont \(\displaystyle 0{,}6\)}%
\end{pgfscope}%
\begin{pgfscope}%
\definecolor{textcolor}{rgb}{0.000000,0.000000,0.000000}%
\pgfsetstrokecolor{textcolor}%
\pgfsetfillcolor{textcolor}%
\pgftext[x=0.288889in,y=1.213040in,,bottom,rotate=90.000000]{\color{textcolor}\sffamily\fontsize{10.000000}{12.000000}\selectfont \(\displaystyle f(x)\)}%
\end{pgfscope}%
\begin{pgfscope}%
\pgfpathrectangle{\pgfqpoint{0.619136in}{0.576079in}}{\pgfqpoint{2.230864in}{1.273921in}}%
\pgfusepath{clip}%
\pgfsetrectcap%
\pgfsetroundjoin%
\pgfsetlinewidth{1.505625pt}%
\definecolor{currentstroke}{rgb}{0.000000,0.000000,1.000000}%
\pgfsetstrokecolor{currentstroke}%
\pgfsetdash{}{0pt}%
\pgfpathmoveto{\pgfqpoint{0.720539in}{0.633985in}}%
\pgfpathlineto{\pgfqpoint{0.748989in}{0.638700in}}%
\pgfpathlineto{\pgfqpoint{0.793696in}{0.648483in}}%
\pgfpathlineto{\pgfqpoint{0.850595in}{0.663398in}}%
\pgfpathlineto{\pgfqpoint{0.911559in}{0.681670in}}%
\pgfpathlineto{\pgfqpoint{0.980651in}{0.704735in}}%
\pgfpathlineto{\pgfqpoint{1.053807in}{0.731528in}}%
\pgfpathlineto{\pgfqpoint{1.131028in}{0.762174in}}%
\pgfpathlineto{\pgfqpoint{1.212313in}{0.796828in}}%
\pgfpathlineto{\pgfqpoint{1.297662in}{0.835665in}}%
\pgfpathlineto{\pgfqpoint{1.387075in}{0.878871in}}%
\pgfpathlineto{\pgfqpoint{1.476489in}{0.924511in}}%
\pgfpathlineto{\pgfqpoint{1.569966in}{0.974701in}}%
\pgfpathlineto{\pgfqpoint{1.667508in}{1.029651in}}%
\pgfpathlineto{\pgfqpoint{1.765050in}{1.087129in}}%
\pgfpathlineto{\pgfqpoint{1.866656in}{1.149592in}}%
\pgfpathlineto{\pgfqpoint{1.968262in}{1.214610in}}%
\pgfpathlineto{\pgfqpoint{2.073933in}{1.284855in}}%
\pgfpathlineto{\pgfqpoint{2.183667in}{1.360555in}}%
\pgfpathlineto{\pgfqpoint{2.293402in}{1.438986in}}%
\pgfpathlineto{\pgfqpoint{2.407201in}{1.523132in}}%
\pgfpathlineto{\pgfqpoint{2.520999in}{1.610072in}}%
\pgfpathlineto{\pgfqpoint{2.638863in}{1.702997in}}%
\pgfpathlineto{\pgfqpoint{2.748597in}{1.792095in}}%
\pgfpathlineto{\pgfqpoint{2.748597in}{1.792095in}}%
\pgfusepath{stroke}%
\end{pgfscope}%
\begin{pgfscope}%
\pgfsetrectcap%
\pgfsetmiterjoin%
\pgfsetlinewidth{0.803000pt}%
\definecolor{currentstroke}{rgb}{0.000000,0.000000,0.000000}%
\pgfsetstrokecolor{currentstroke}%
\pgfsetdash{}{0pt}%
\pgfpathmoveto{\pgfqpoint{0.619136in}{0.576079in}}%
\pgfpathlineto{\pgfqpoint{0.619136in}{1.850000in}}%
\pgfusepath{stroke}%
\end{pgfscope}%
\begin{pgfscope}%
\pgfsetrectcap%
\pgfsetmiterjoin%
\pgfsetlinewidth{0.803000pt}%
\definecolor{currentstroke}{rgb}{0.000000,0.000000,0.000000}%
\pgfsetstrokecolor{currentstroke}%
\pgfsetdash{}{0pt}%
\pgfpathmoveto{\pgfqpoint{2.850000in}{0.576079in}}%
\pgfpathlineto{\pgfqpoint{2.850000in}{1.850000in}}%
\pgfusepath{stroke}%
\end{pgfscope}%
\begin{pgfscope}%
\pgfsetrectcap%
\pgfsetmiterjoin%
\pgfsetlinewidth{0.803000pt}%
\definecolor{currentstroke}{rgb}{0.000000,0.000000,0.000000}%
\pgfsetstrokecolor{currentstroke}%
\pgfsetdash{}{0pt}%
\pgfpathmoveto{\pgfqpoint{0.619136in}{0.576079in}}%
\pgfpathlineto{\pgfqpoint{2.850000in}{0.576079in}}%
\pgfusepath{stroke}%
\end{pgfscope}%
\begin{pgfscope}%
\pgfsetrectcap%
\pgfsetmiterjoin%
\pgfsetlinewidth{0.803000pt}%
\definecolor{currentstroke}{rgb}{0.000000,0.000000,0.000000}%
\pgfsetstrokecolor{currentstroke}%
\pgfsetdash{}{0pt}%
\pgfpathmoveto{\pgfqpoint{0.619136in}{1.850000in}}%
\pgfpathlineto{\pgfqpoint{2.850000in}{1.850000in}}%
\pgfusepath{stroke}%
\end{pgfscope}%
\end{pgfpicture}%
\makeatother%
\endgroup%

    \caption{Graphe de la fonction $f \colon x \mapsto \int_x^{x^2} \frac{\d t}{\ln(t)}$ sur l'intervalle $\interoo{0}{1}$}
\end{marginfigure}

\begin{prop}
Pour tout $x \in \interff{0}{1}$, on définit $f(x) = \int_x^{x^2} \frac{\d t}{\ln(t)}$.
\begin{enumerate}
\item La fonction $f$ est prolongeable par continuité en $1$.

\item La fonction $f$ est dérivable sur $\interff{0}{1}$.
\end{enumerate}
\end{prop}

\begin{remarque}
La fonction $x \mapsto \int_0^x \frac{1}{\ln(t)} \d t$ est la fonction \textsl{logarithme intégral}.
\end{remarque}

\begin{exercice}
\begin{questions}
\item En utilisant la concavité du logarithme, montrer que
\[
\forall\, t \in \interof{x^2}{1},\,
\frac{\ln(x^2)}{x^2 - 1} (t - 1) \leq \ln(t).
\]

\item En déduire que
\[
\frac{(x + 1) (x - 1)}{2 \ln(x)} \ln\module{x + 1} \leq f(x) \leq \ln\module{x + 1}.
\]

\item Montrer que $f$ est prolongeable en par continuité en $0$.

\item Calculer $f'$ sur $]0, 1]$ puis montrer que $f$ est dérivable en $0$.
\end{questions}
\end{exercice}


\begin{elemsolution}
\begin{reponses}
\item Soit $t \in \interfo{x^2}{1}$. Comme la fonction logarithme est concave, sa courbe représentative se situe en-dessous de sa tangente en $1$ et
\[
\ln(t) - \ln(1) \leqslant t - 1.
\]

\medskip

Toujours par concavité de la fonction logarithme, la corde reliant les points de coordonnées $\big(x^2, \ln(x^2)\big)$ et $(1, 0)$ se situe en-dessous de la courbe représentant le logarithme, soit :
\[
\frac{\ln(x^2) - \ln(1)}{x^2 - 1} (t - 1) \leqslant \ln(t).
\]

\item En utilisant l'encadrement précédent, pour tout $t \in \interoo{x^2}{1}$,
\begin{align*}
\frac{1}{t - 1} \leqslant \frac{1}{\ln(t)} &\leqslant \frac{x^2 - 1}{2 \ln(x)} \times \frac{1}{t - 1}\\
\frac{x^2 - 1}{2 \ln(x)} \int_x^{x^2} \frac{\d t}{t - 1} &\leqslant f(x) \leqslant \int_x^{x^2} \frac{\d t}{t - 1}\\
\frac{x^2 - 1}{2 \ln(x)} \ln \abs{\frac{x^2 - 1}{x - 1}} &\leqslant f(x) \leqslant \ln\abs{\frac{x^2 - 1}{x - 1}}\\
\frac{(x + 1) (x - 1)}{2 \ln(x)} \ln\abs{x + 1} &\leqslant f(x) \leqslant \ln\abs{x + 1}.
\end{align*}

Ainsi, d'après le théorème d'encadrement, $\lim\limits_{x\to 1} f(x) = \ln(2)$.

\item En utilisant la dérivation par rapport aux bornes,
\begin{align*}
f'(x)
&= 2 x \frac{1}{\ln(x^2)} - x \frac{1}{\ln(x)}\\
&= \frac{x - 1}{\ln(x)}.
\end{align*}

La fonction $f$ est prolongeable par continuité sur $]0, 1]$ et dérivable sur $]0, 1[$. De plus, $\lim\limits_{x\to 1} f'(x) = 1$. D'après le théorème de prolongement dérivable, $f$ est dérivable en $1$ et $f'(1) = 1$.
\end{reponses}
\end{elemsolution}

%-----------
\subsection{Un développement asymptotique}

\begin{prop}
On pose
$$F:x \mapsto \int_x^{+\infty} \frac{\e^{-t}}{t} \d t.$$
Alors,
\[
F(x) = \frac{\e^{-x}}{x} + \frac{\e^{-x}}{x^2} + o_{+\infty}\left(\frac{\e^{-x}}{x^2}\right)
\text{ et }
F(x) \sim_0 -\ln(x).
\]
\end{prop}

\begin{exercice}
\begin{questions}
\item Déterminer l'ensemble de définition de $F$. Étudier brièvement le comportement de la fonction $F$ et tracer sa courbe représentative.

\item Montrer que $F(x) \sim_{+\infty} \frac{\e^{-x}}{x}$.

\item Montrer que $F(x) = \frac{\e^{-x}}{x} + \frac{\e^{-x}}{x^2} + o_{+\infty}\left(\frac{\e^{-x}}{x^2}\right)$.
       
\item Montrer que $F(x) \sim_0 -\ln(x)$.
\end{questions}
\end{exercice}

\begin{marginfigure}[-10cm]
    \centering
    %% Creator: Matplotlib, PGF backend
%%
%% To include the figure in your LaTeX document, write
%%   \input{<filename>.pgf}
%%
%% Make sure the required packages are loaded in your preamble
%%   \usepackage{pgf}
%%
%% Also ensure that all the required font packages are loaded; for instance,
%% the lmodern package is sometimes necessary when using math font.
%%   \usepackage{lmodern}
%%
%% Figures using additional raster images can only be included by \input if
%% they are in the same directory as the main LaTeX file. For loading figures
%% from other directories you can use the `import` package
%%   \usepackage{import}
%%
%% and then include the figures with
%%   \import{<path to file>}{<filename>.pgf}
%%
%% Matplotlib used the following preamble
%%   
%%   \usepackage{fontspec}
%%   \setmainfont{DejaVuSerif.ttf}[Path=\detokenize{/home/wayoff/.pyenv/versions/3.8.10/lib/python3.8/site-packages/matplotlib/mpl-data/fonts/ttf/}]
%%   \setsansfont{DejaVuSans.ttf}[Path=\detokenize{/home/wayoff/.pyenv/versions/3.8.10/lib/python3.8/site-packages/matplotlib/mpl-data/fonts/ttf/}]
%%   \setmonofont{DejaVuSansMono.ttf}[Path=\detokenize{/home/wayoff/.pyenv/versions/3.8.10/lib/python3.8/site-packages/matplotlib/mpl-data/fonts/ttf/}]
%%   \makeatletter\@ifpackageloaded{underscore}{}{\usepackage[strings]{underscore}}\makeatother
%%
\begingroup%
\makeatletter%
\begin{pgfpicture}%
\pgfpathrectangle{\pgfpointorigin}{\pgfqpoint{3.000000in}{3.000000in}}%
\pgfusepath{use as bounding box, clip}%
\begin{pgfscope}%
\pgfsetbuttcap%
\pgfsetmiterjoin%
\definecolor{currentfill}{rgb}{1.000000,1.000000,1.000000}%
\pgfsetfillcolor{currentfill}%
\pgfsetlinewidth{0.000000pt}%
\definecolor{currentstroke}{rgb}{1.000000,1.000000,1.000000}%
\pgfsetstrokecolor{currentstroke}%
\pgfsetdash{}{0pt}%
\pgfpathmoveto{\pgfqpoint{0.000000in}{0.000000in}}%
\pgfpathlineto{\pgfqpoint{3.000000in}{0.000000in}}%
\pgfpathlineto{\pgfqpoint{3.000000in}{3.000000in}}%
\pgfpathlineto{\pgfqpoint{0.000000in}{3.000000in}}%
\pgfpathlineto{\pgfqpoint{0.000000in}{0.000000in}}%
\pgfpathclose%
\pgfusepath{fill}%
\end{pgfscope}%
\begin{pgfscope}%
\pgfsetbuttcap%
\pgfsetmiterjoin%
\definecolor{currentfill}{rgb}{1.000000,1.000000,1.000000}%
\pgfsetfillcolor{currentfill}%
\pgfsetlinewidth{0.000000pt}%
\definecolor{currentstroke}{rgb}{0.000000,0.000000,0.000000}%
\pgfsetstrokecolor{currentstroke}%
\pgfsetstrokeopacity{0.000000}%
\pgfsetdash{}{0pt}%
\pgfpathmoveto{\pgfqpoint{0.316667in}{0.339968in}}%
\pgfpathlineto{\pgfqpoint{2.850000in}{0.339968in}}%
\pgfpathlineto{\pgfqpoint{2.850000in}{2.850000in}}%
\pgfpathlineto{\pgfqpoint{0.316667in}{2.850000in}}%
\pgfpathlineto{\pgfqpoint{0.316667in}{0.339968in}}%
\pgfpathclose%
\pgfusepath{fill}%
\end{pgfscope}%
\begin{pgfscope}%
\definecolor{textcolor}{rgb}{0.000000,0.000000,0.000000}%
\pgfsetstrokecolor{textcolor}%
\pgfsetfillcolor{textcolor}%
\pgftext[x=1.583333in,y=0.284413in,,top]{\color{textcolor}\sffamily\fontsize{10.000000}{12.000000}\selectfont \(\displaystyle x\)}%
\end{pgfscope}%
\begin{pgfscope}%
\pgfpathrectangle{\pgfqpoint{0.316667in}{0.339968in}}{\pgfqpoint{2.533333in}{2.510032in}}%
\pgfusepath{clip}%
\pgfsetrectcap%
\pgfsetroundjoin%
\pgfsetlinewidth{0.803000pt}%
\definecolor{currentstroke}{rgb}{0.690196,0.690196,0.690196}%
\pgfsetstrokecolor{currentstroke}%
\pgfsetdash{}{0pt}%
\pgfpathmoveto{\pgfqpoint{0.316667in}{0.453693in}}%
\pgfpathlineto{\pgfqpoint{2.850000in}{0.453693in}}%
\pgfusepath{stroke}%
\end{pgfscope}%
\begin{pgfscope}%
\pgfsetbuttcap%
\pgfsetroundjoin%
\definecolor{currentfill}{rgb}{0.000000,0.000000,0.000000}%
\pgfsetfillcolor{currentfill}%
\pgfsetlinewidth{0.803000pt}%
\definecolor{currentstroke}{rgb}{0.000000,0.000000,0.000000}%
\pgfsetstrokecolor{currentstroke}%
\pgfsetdash{}{0pt}%
\pgfsys@defobject{currentmarker}{\pgfqpoint{-0.048611in}{0.000000in}}{\pgfqpoint{-0.000000in}{0.000000in}}{%
\pgfpathmoveto{\pgfqpoint{-0.000000in}{0.000000in}}%
\pgfpathlineto{\pgfqpoint{-0.048611in}{0.000000in}}%
\pgfusepath{stroke,fill}%
}%
\begin{pgfscope}%
\pgfsys@transformshift{0.316667in}{0.453693in}%
\pgfsys@useobject{currentmarker}{}%
\end{pgfscope}%
\end{pgfscope}%
\begin{pgfscope}%
\definecolor{textcolor}{rgb}{0.000000,0.000000,0.000000}%
\pgfsetstrokecolor{textcolor}%
\pgfsetfillcolor{textcolor}%
\pgftext[x=0.150000in, y=0.400932in, left, base]{\color{textcolor}\sffamily\fontsize{10.000000}{12.000000}\selectfont \(\displaystyle 0\)}%
\end{pgfscope}%
\begin{pgfscope}%
\pgfpathrectangle{\pgfqpoint{0.316667in}{0.339968in}}{\pgfqpoint{2.533333in}{2.510032in}}%
\pgfusepath{clip}%
\pgfsetrectcap%
\pgfsetroundjoin%
\pgfsetlinewidth{1.505625pt}%
\definecolor{currentstroke}{rgb}{0.000000,0.000000,1.000000}%
\pgfsetstrokecolor{currentstroke}%
\pgfsetdash{}{0pt}%
\pgfpathmoveto{\pgfqpoint{0.431818in}{2.353395in}}%
\pgfpathlineto{\pgfqpoint{0.454964in}{2.193744in}}%
\pgfpathlineto{\pgfqpoint{0.478111in}{2.047597in}}%
\pgfpathlineto{\pgfqpoint{0.501257in}{1.913804in}}%
\pgfpathlineto{\pgfqpoint{0.524403in}{1.791312in}}%
\pgfpathlineto{\pgfqpoint{0.547549in}{1.679160in}}%
\pgfpathlineto{\pgfqpoint{0.570695in}{1.576470in}}%
\pgfpathlineto{\pgfqpoint{0.593841in}{1.482437in}}%
\pgfpathlineto{\pgfqpoint{0.616987in}{1.396326in}}%
\pgfpathlineto{\pgfqpoint{0.640133in}{1.317466in}}%
\pgfpathlineto{\pgfqpoint{0.663279in}{1.245242in}}%
\pgfpathlineto{\pgfqpoint{0.686425in}{1.179085in}}%
\pgfpathlineto{\pgfqpoint{0.709571in}{1.118495in}}%
\pgfpathlineto{\pgfqpoint{0.732717in}{1.062994in}}%
\pgfpathlineto{\pgfqpoint{0.755863in}{1.012153in}}%
\pgfpathlineto{\pgfqpoint{0.779009in}{0.965577in}}%
\pgfpathlineto{\pgfqpoint{0.802155in}{0.922906in}}%
\pgfpathlineto{\pgfqpoint{0.825301in}{0.883812in}}%
\pgfpathlineto{\pgfqpoint{0.848447in}{0.847992in}}%
\pgfpathlineto{\pgfqpoint{0.871593in}{0.815170in}}%
\pgfpathlineto{\pgfqpoint{0.894739in}{0.785095in}}%
\pgfpathlineto{\pgfqpoint{0.917885in}{0.757535in}}%
\pgfpathlineto{\pgfqpoint{0.952604in}{0.720452in}}%
\pgfpathlineto{\pgfqpoint{0.987323in}{0.687917in}}%
\pgfpathlineto{\pgfqpoint{1.022042in}{0.659368in}}%
\pgfpathlineto{\pgfqpoint{1.056761in}{0.634315in}}%
\pgfpathlineto{\pgfqpoint{1.091480in}{0.612327in}}%
\pgfpathlineto{\pgfqpoint{1.126199in}{0.593028in}}%
\pgfpathlineto{\pgfqpoint{1.160918in}{0.576087in}}%
\pgfpathlineto{\pgfqpoint{1.195637in}{0.561214in}}%
\pgfpathlineto{\pgfqpoint{1.241930in}{0.544168in}}%
\pgfpathlineto{\pgfqpoint{1.288222in}{0.529834in}}%
\pgfpathlineto{\pgfqpoint{1.334514in}{0.517781in}}%
\pgfpathlineto{\pgfqpoint{1.392379in}{0.505370in}}%
\pgfpathlineto{\pgfqpoint{1.450244in}{0.495371in}}%
\pgfpathlineto{\pgfqpoint{1.519682in}{0.485900in}}%
\pgfpathlineto{\pgfqpoint{1.600693in}{0.477543in}}%
\pgfpathlineto{\pgfqpoint{1.693277in}{0.470620in}}%
\pgfpathlineto{\pgfqpoint{1.797434in}{0.465208in}}%
\pgfpathlineto{\pgfqpoint{1.924737in}{0.460890in}}%
\pgfpathlineto{\pgfqpoint{2.098333in}{0.457489in}}%
\pgfpathlineto{\pgfqpoint{2.341366in}{0.455246in}}%
\pgfpathlineto{\pgfqpoint{2.734848in}{0.454061in}}%
\pgfpathlineto{\pgfqpoint{2.734848in}{0.454061in}}%
\pgfusepath{stroke}%
\end{pgfscope}%
\begin{pgfscope}%
\pgfpathrectangle{\pgfqpoint{0.316667in}{0.339968in}}{\pgfqpoint{2.533333in}{2.510032in}}%
\pgfusepath{clip}%
\pgfsetrectcap%
\pgfsetroundjoin%
\pgfsetlinewidth{1.505625pt}%
\definecolor{currentstroke}{rgb}{1.000000,0.000000,0.000000}%
\pgfsetstrokecolor{currentstroke}%
\pgfsetdash{}{0pt}%
\pgfpathmoveto{\pgfqpoint{0.431818in}{2.735908in}}%
\pgfpathlineto{\pgfqpoint{0.454964in}{2.541284in}}%
\pgfpathlineto{\pgfqpoint{0.478111in}{2.363401in}}%
\pgfpathlineto{\pgfqpoint{0.501257in}{2.200802in}}%
\pgfpathlineto{\pgfqpoint{0.524403in}{2.052164in}}%
\pgfpathlineto{\pgfqpoint{0.547549in}{1.916274in}}%
\pgfpathlineto{\pgfqpoint{0.570695in}{1.792030in}}%
\pgfpathlineto{\pgfqpoint{0.593841in}{1.678425in}}%
\pgfpathlineto{\pgfqpoint{0.616987in}{1.574539in}}%
\pgfpathlineto{\pgfqpoint{0.640133in}{1.479533in}}%
\pgfpathlineto{\pgfqpoint{0.663279in}{1.392641in}}%
\pgfpathlineto{\pgfqpoint{0.686425in}{1.313165in}}%
\pgfpathlineto{\pgfqpoint{0.709571in}{1.240465in}}%
\pgfpathlineto{\pgfqpoint{0.732717in}{1.173960in}}%
\pgfpathlineto{\pgfqpoint{0.755863in}{1.113118in}}%
\pgfpathlineto{\pgfqpoint{0.779009in}{1.057451in}}%
\pgfpathlineto{\pgfqpoint{0.802155in}{1.006516in}}%
\pgfpathlineto{\pgfqpoint{0.825301in}{0.959908in}}%
\pgfpathlineto{\pgfqpoint{0.848447in}{0.917256in}}%
\pgfpathlineto{\pgfqpoint{0.871593in}{0.878222in}}%
\pgfpathlineto{\pgfqpoint{0.894739in}{0.842497in}}%
\pgfpathlineto{\pgfqpoint{0.917885in}{0.809798in}}%
\pgfpathlineto{\pgfqpoint{0.941031in}{0.779867in}}%
\pgfpathlineto{\pgfqpoint{0.964177in}{0.752467in}}%
\pgfpathlineto{\pgfqpoint{0.998896in}{0.715648in}}%
\pgfpathlineto{\pgfqpoint{1.033615in}{0.683394in}}%
\pgfpathlineto{\pgfqpoint{1.068334in}{0.655133in}}%
\pgfpathlineto{\pgfqpoint{1.103053in}{0.630370in}}%
\pgfpathlineto{\pgfqpoint{1.137772in}{0.608668in}}%
\pgfpathlineto{\pgfqpoint{1.172491in}{0.589646in}}%
\pgfpathlineto{\pgfqpoint{1.207210in}{0.572971in}}%
\pgfpathlineto{\pgfqpoint{1.241930in}{0.558352in}}%
\pgfpathlineto{\pgfqpoint{1.288222in}{0.541623in}}%
\pgfpathlineto{\pgfqpoint{1.334514in}{0.527580in}}%
\pgfpathlineto{\pgfqpoint{1.380806in}{0.515791in}}%
\pgfpathlineto{\pgfqpoint{1.438671in}{0.503675in}}%
\pgfpathlineto{\pgfqpoint{1.496536in}{0.493933in}}%
\pgfpathlineto{\pgfqpoint{1.565974in}{0.484725in}}%
\pgfpathlineto{\pgfqpoint{1.646985in}{0.476620in}}%
\pgfpathlineto{\pgfqpoint{1.739569in}{0.469924in}}%
\pgfpathlineto{\pgfqpoint{1.843726in}{0.464705in}}%
\pgfpathlineto{\pgfqpoint{1.971029in}{0.460553in}}%
\pgfpathlineto{\pgfqpoint{2.144625in}{0.457297in}}%
\pgfpathlineto{\pgfqpoint{2.387658in}{0.455160in}}%
\pgfpathlineto{\pgfqpoint{2.734848in}{0.454102in}}%
\pgfpathlineto{\pgfqpoint{2.734848in}{0.454102in}}%
\pgfusepath{stroke}%
\end{pgfscope}%
\begin{pgfscope}%
\pgfsetrectcap%
\pgfsetmiterjoin%
\pgfsetlinewidth{0.803000pt}%
\definecolor{currentstroke}{rgb}{0.000000,0.000000,0.000000}%
\pgfsetstrokecolor{currentstroke}%
\pgfsetdash{}{0pt}%
\pgfpathmoveto{\pgfqpoint{0.316667in}{0.339968in}}%
\pgfpathlineto{\pgfqpoint{0.316667in}{2.850000in}}%
\pgfusepath{stroke}%
\end{pgfscope}%
\begin{pgfscope}%
\pgfsetrectcap%
\pgfsetmiterjoin%
\pgfsetlinewidth{0.803000pt}%
\definecolor{currentstroke}{rgb}{0.000000,0.000000,0.000000}%
\pgfsetstrokecolor{currentstroke}%
\pgfsetdash{}{0pt}%
\pgfpathmoveto{\pgfqpoint{2.850000in}{0.339968in}}%
\pgfpathlineto{\pgfqpoint{2.850000in}{2.850000in}}%
\pgfusepath{stroke}%
\end{pgfscope}%
\begin{pgfscope}%
\pgfsetrectcap%
\pgfsetmiterjoin%
\pgfsetlinewidth{0.803000pt}%
\definecolor{currentstroke}{rgb}{0.000000,0.000000,0.000000}%
\pgfsetstrokecolor{currentstroke}%
\pgfsetdash{}{0pt}%
\pgfpathmoveto{\pgfqpoint{0.316667in}{0.339968in}}%
\pgfpathlineto{\pgfqpoint{2.850000in}{0.339968in}}%
\pgfusepath{stroke}%
\end{pgfscope}%
\begin{pgfscope}%
\pgfsetrectcap%
\pgfsetmiterjoin%
\pgfsetlinewidth{0.803000pt}%
\definecolor{currentstroke}{rgb}{0.000000,0.000000,0.000000}%
\pgfsetstrokecolor{currentstroke}%
\pgfsetdash{}{0pt}%
\pgfpathmoveto{\pgfqpoint{0.316667in}{2.850000in}}%
\pgfpathlineto{\pgfqpoint{2.850000in}{2.850000in}}%
\pgfusepath{stroke}%
\end{pgfscope}%
\begin{pgfscope}%
\pgfsetbuttcap%
\pgfsetmiterjoin%
\definecolor{currentfill}{rgb}{1.000000,1.000000,1.000000}%
\pgfsetfillcolor{currentfill}%
\pgfsetfillopacity{0.800000}%
\pgfsetlinewidth{1.003750pt}%
\definecolor{currentstroke}{rgb}{0.800000,0.800000,0.800000}%
\pgfsetstrokecolor{currentstroke}%
\pgfsetstrokeopacity{0.800000}%
\pgfsetdash{}{0pt}%
\pgfpathmoveto{\pgfqpoint{1.048729in}{1.902568in}}%
\pgfpathlineto{\pgfqpoint{2.752778in}{1.902568in}}%
\pgfpathquadraticcurveto{\pgfqpoint{2.780556in}{1.902568in}}{\pgfqpoint{2.780556in}{1.930345in}}%
\pgfpathlineto{\pgfqpoint{2.780556in}{2.752778in}}%
\pgfpathquadraticcurveto{\pgfqpoint{2.780556in}{2.780556in}}{\pgfqpoint{2.752778in}{2.780556in}}%
\pgfpathlineto{\pgfqpoint{1.048729in}{2.780556in}}%
\pgfpathquadraticcurveto{\pgfqpoint{1.020952in}{2.780556in}}{\pgfqpoint{1.020952in}{2.752778in}}%
\pgfpathlineto{\pgfqpoint{1.020952in}{1.930345in}}%
\pgfpathquadraticcurveto{\pgfqpoint{1.020952in}{1.902568in}}{\pgfqpoint{1.048729in}{1.902568in}}%
\pgfpathlineto{\pgfqpoint{1.048729in}{1.902568in}}%
\pgfpathclose%
\pgfusepath{stroke,fill}%
\end{pgfscope}%
\begin{pgfscope}%
\pgfsetrectcap%
\pgfsetroundjoin%
\pgfsetlinewidth{1.505625pt}%
\definecolor{currentstroke}{rgb}{0.000000,0.000000,1.000000}%
\pgfsetstrokecolor{currentstroke}%
\pgfsetdash{}{0pt}%
\pgfpathmoveto{\pgfqpoint{1.076507in}{2.535803in}}%
\pgfpathlineto{\pgfqpoint{1.215396in}{2.535803in}}%
\pgfpathlineto{\pgfqpoint{1.354285in}{2.535803in}}%
\pgfusepath{stroke}%
\end{pgfscope}%
\begin{pgfscope}%
\definecolor{textcolor}{rgb}{0.000000,0.000000,0.000000}%
\pgfsetstrokecolor{textcolor}%
\pgfsetfillcolor{textcolor}%
\pgftext[x=1.465396in,y=2.487192in,left,base]{\color{textcolor}\sffamily\fontsize{10.000000}{12.000000}\selectfont \(\displaystyle F(x) = \int_x^{+\infty} \frac{\mathrm{e}^{-t}}{t} \, \mathrm{d} t\)}%
\end{pgfscope}%
\begin{pgfscope}%
\pgfsetrectcap%
\pgfsetroundjoin%
\pgfsetlinewidth{1.505625pt}%
\definecolor{currentstroke}{rgb}{1.000000,0.000000,0.000000}%
\pgfsetstrokecolor{currentstroke}%
\pgfsetdash{}{0pt}%
\pgfpathmoveto{\pgfqpoint{1.076507in}{2.102005in}}%
\pgfpathlineto{\pgfqpoint{1.215396in}{2.102005in}}%
\pgfpathlineto{\pgfqpoint{1.354285in}{2.102005in}}%
\pgfusepath{stroke}%
\end{pgfscope}%
\begin{pgfscope}%
\definecolor{textcolor}{rgb}{0.000000,0.000000,0.000000}%
\pgfsetstrokecolor{textcolor}%
\pgfsetfillcolor{textcolor}%
\pgftext[x=1.465396in,y=2.053394in,left,base]{\color{textcolor}\sffamily\fontsize{10.000000}{12.000000}\selectfont \(\displaystyle x \mapsto \frac{\mathrm{e}^{-x}}{x} + \frac{\mathrm{e}^{-x}}{x^2}\)}%
\end{pgfscope}%
\end{pgfpicture}%
\makeatother%
\endgroup%

    \caption{Représentation graphique de la fonction $F$ et des premiers termes de son développement asymptotique en $+\infty$}
\end{marginfigure}
\begin{marginfigure}[0cm]
    \centering
    %% Creator: Matplotlib, PGF backend
%%
%% To include the figure in your LaTeX document, write
%%   \input{<filename>.pgf}
%%
%% Make sure the required packages are loaded in your preamble
%%   \usepackage{pgf}
%%
%% Also ensure that all the required font packages are loaded; for instance,
%% the lmodern package is sometimes necessary when using math font.
%%   \usepackage{lmodern}
%%
%% Figures using additional raster images can only be included by \input if
%% they are in the same directory as the main LaTeX file. For loading figures
%% from other directories you can use the `import` package
%%   \usepackage{import}
%%
%% and then include the figures with
%%   \import{<path to file>}{<filename>.pgf}
%%
%% Matplotlib used the following preamble
%%   
%%   \usepackage{fontspec}
%%   \setmainfont{DejaVuSerif.ttf}[Path=\detokenize{/home/wayoff/.pyenv/versions/3.8.10/lib/python3.8/site-packages/matplotlib/mpl-data/fonts/ttf/}]
%%   \setsansfont{DejaVuSans.ttf}[Path=\detokenize{/home/wayoff/.pyenv/versions/3.8.10/lib/python3.8/site-packages/matplotlib/mpl-data/fonts/ttf/}]
%%   \setmonofont{DejaVuSansMono.ttf}[Path=\detokenize{/home/wayoff/.pyenv/versions/3.8.10/lib/python3.8/site-packages/matplotlib/mpl-data/fonts/ttf/}]
%%   \makeatletter\@ifpackageloaded{underscore}{}{\usepackage[strings]{underscore}}\makeatother
%%
\begingroup%
\makeatletter%
\begin{pgfpicture}%
\pgfpathrectangle{\pgfpointorigin}{\pgfqpoint{3.000000in}{3.000000in}}%
\pgfusepath{use as bounding box, clip}%
\begin{pgfscope}%
\pgfsetbuttcap%
\pgfsetmiterjoin%
\definecolor{currentfill}{rgb}{1.000000,1.000000,1.000000}%
\pgfsetfillcolor{currentfill}%
\pgfsetlinewidth{0.000000pt}%
\definecolor{currentstroke}{rgb}{1.000000,1.000000,1.000000}%
\pgfsetstrokecolor{currentstroke}%
\pgfsetdash{}{0pt}%
\pgfpathmoveto{\pgfqpoint{0.000000in}{0.000000in}}%
\pgfpathlineto{\pgfqpoint{3.000000in}{0.000000in}}%
\pgfpathlineto{\pgfqpoint{3.000000in}{3.000000in}}%
\pgfpathlineto{\pgfqpoint{0.000000in}{3.000000in}}%
\pgfpathlineto{\pgfqpoint{0.000000in}{0.000000in}}%
\pgfpathclose%
\pgfusepath{fill}%
\end{pgfscope}%
\begin{pgfscope}%
\pgfsetbuttcap%
\pgfsetmiterjoin%
\definecolor{currentfill}{rgb}{1.000000,1.000000,1.000000}%
\pgfsetfillcolor{currentfill}%
\pgfsetlinewidth{0.000000pt}%
\definecolor{currentstroke}{rgb}{0.000000,0.000000,0.000000}%
\pgfsetstrokecolor{currentstroke}%
\pgfsetstrokeopacity{0.000000}%
\pgfsetdash{}{0pt}%
\pgfpathmoveto{\pgfqpoint{0.150000in}{0.571603in}}%
\pgfpathlineto{\pgfqpoint{2.850000in}{0.571603in}}%
\pgfpathlineto{\pgfqpoint{2.850000in}{2.850000in}}%
\pgfpathlineto{\pgfqpoint{0.150000in}{2.850000in}}%
\pgfpathlineto{\pgfqpoint{0.150000in}{0.571603in}}%
\pgfpathclose%
\pgfusepath{fill}%
\end{pgfscope}%
\begin{pgfscope}%
\pgfpathrectangle{\pgfqpoint{0.150000in}{0.571603in}}{\pgfqpoint{2.700000in}{2.278397in}}%
\pgfusepath{clip}%
\pgfsetrectcap%
\pgfsetroundjoin%
\pgfsetlinewidth{0.803000pt}%
\definecolor{currentstroke}{rgb}{0.690196,0.690196,0.690196}%
\pgfsetstrokecolor{currentstroke}%
\pgfsetdash{}{0pt}%
\pgfpathmoveto{\pgfqpoint{0.272236in}{0.571603in}}%
\pgfpathlineto{\pgfqpoint{0.272236in}{2.850000in}}%
\pgfusepath{stroke}%
\end{pgfscope}%
\begin{pgfscope}%
\pgfsetbuttcap%
\pgfsetroundjoin%
\definecolor{currentfill}{rgb}{0.000000,0.000000,0.000000}%
\pgfsetfillcolor{currentfill}%
\pgfsetlinewidth{0.803000pt}%
\definecolor{currentstroke}{rgb}{0.000000,0.000000,0.000000}%
\pgfsetstrokecolor{currentstroke}%
\pgfsetdash{}{0pt}%
\pgfsys@defobject{currentmarker}{\pgfqpoint{0.000000in}{-0.048611in}}{\pgfqpoint{0.000000in}{0.000000in}}{%
\pgfpathmoveto{\pgfqpoint{0.000000in}{0.000000in}}%
\pgfpathlineto{\pgfqpoint{0.000000in}{-0.048611in}}%
\pgfusepath{stroke,fill}%
}%
\begin{pgfscope}%
\pgfsys@transformshift{0.272236in}{0.571603in}%
\pgfsys@useobject{currentmarker}{}%
\end{pgfscope}%
\end{pgfscope}%
\begin{pgfscope}%
\definecolor{textcolor}{rgb}{0.000000,0.000000,0.000000}%
\pgfsetstrokecolor{textcolor}%
\pgfsetfillcolor{textcolor}%
\pgftext[x=0.272236in,y=0.474381in,,top]{\color{textcolor}\sffamily\fontsize{10.000000}{12.000000}\selectfont \(\displaystyle 0\)}%
\end{pgfscope}%
\begin{pgfscope}%
\definecolor{textcolor}{rgb}{0.000000,0.000000,0.000000}%
\pgfsetstrokecolor{textcolor}%
\pgfsetfillcolor{textcolor}%
\pgftext[x=1.500000in,y=0.284413in,,top]{\color{textcolor}\sffamily\fontsize{10.000000}{12.000000}\selectfont \(\displaystyle x\)}%
\end{pgfscope}%
\begin{pgfscope}%
\pgfpathrectangle{\pgfqpoint{0.150000in}{0.571603in}}{\pgfqpoint{2.700000in}{2.278397in}}%
\pgfusepath{clip}%
\pgfsetrectcap%
\pgfsetroundjoin%
\pgfsetlinewidth{1.505625pt}%
\definecolor{currentstroke}{rgb}{0.000000,0.000000,1.000000}%
\pgfsetstrokecolor{currentstroke}%
\pgfsetdash{}{0pt}%
\pgfpathmoveto{\pgfqpoint{0.272727in}{2.608253in}}%
\pgfpathlineto{\pgfqpoint{0.280936in}{1.920357in}}%
\pgfpathlineto{\pgfqpoint{0.289146in}{1.761645in}}%
\pgfpathlineto{\pgfqpoint{0.297355in}{1.667290in}}%
\pgfpathlineto{\pgfqpoint{0.305564in}{1.599980in}}%
\pgfpathlineto{\pgfqpoint{0.313773in}{1.547656in}}%
\pgfpathlineto{\pgfqpoint{0.321982in}{1.504872in}}%
\pgfpathlineto{\pgfqpoint{0.338401in}{1.437378in}}%
\pgfpathlineto{\pgfqpoint{0.354819in}{1.385099in}}%
\pgfpathlineto{\pgfqpoint{0.371237in}{1.342475in}}%
\pgfpathlineto{\pgfqpoint{0.387656in}{1.306528in}}%
\pgfpathlineto{\pgfqpoint{0.404074in}{1.275473in}}%
\pgfpathlineto{\pgfqpoint{0.420493in}{1.248160in}}%
\pgfpathlineto{\pgfqpoint{0.445120in}{1.212545in}}%
\pgfpathlineto{\pgfqpoint{0.469748in}{1.181836in}}%
\pgfpathlineto{\pgfqpoint{0.494375in}{1.154877in}}%
\pgfpathlineto{\pgfqpoint{0.519003in}{1.130875in}}%
\pgfpathlineto{\pgfqpoint{0.551839in}{1.102521in}}%
\pgfpathlineto{\pgfqpoint{0.584676in}{1.077488in}}%
\pgfpathlineto{\pgfqpoint{0.617513in}{1.055109in}}%
\pgfpathlineto{\pgfqpoint{0.658559in}{1.030143in}}%
\pgfpathlineto{\pgfqpoint{0.699605in}{1.007887in}}%
\pgfpathlineto{\pgfqpoint{0.748860in}{0.984063in}}%
\pgfpathlineto{\pgfqpoint{0.798115in}{0.962800in}}%
\pgfpathlineto{\pgfqpoint{0.855579in}{0.940619in}}%
\pgfpathlineto{\pgfqpoint{0.913043in}{0.920757in}}%
\pgfpathlineto{\pgfqpoint{0.978717in}{0.900388in}}%
\pgfpathlineto{\pgfqpoint{1.052600in}{0.879917in}}%
\pgfpathlineto{\pgfqpoint{1.134691in}{0.859653in}}%
\pgfpathlineto{\pgfqpoint{1.224992in}{0.839832in}}%
\pgfpathlineto{\pgfqpoint{1.323503in}{0.820618in}}%
\pgfpathlineto{\pgfqpoint{1.430222in}{0.802128in}}%
\pgfpathlineto{\pgfqpoint{1.545151in}{0.784436in}}%
\pgfpathlineto{\pgfqpoint{1.676497in}{0.766529in}}%
\pgfpathlineto{\pgfqpoint{1.816054in}{0.749722in}}%
\pgfpathlineto{\pgfqpoint{1.972028in}{0.733154in}}%
\pgfpathlineto{\pgfqpoint{2.144421in}{0.717072in}}%
\pgfpathlineto{\pgfqpoint{2.341441in}{0.701028in}}%
\pgfpathlineto{\pgfqpoint{2.554880in}{0.685930in}}%
\pgfpathlineto{\pgfqpoint{2.727273in}{0.675167in}}%
\pgfpathlineto{\pgfqpoint{2.727273in}{0.675167in}}%
\pgfusepath{stroke}%
\end{pgfscope}%
\begin{pgfscope}%
\pgfpathrectangle{\pgfqpoint{0.150000in}{0.571603in}}{\pgfqpoint{2.700000in}{2.278397in}}%
\pgfusepath{clip}%
\pgfsetrectcap%
\pgfsetroundjoin%
\pgfsetlinewidth{1.505625pt}%
\definecolor{currentstroke}{rgb}{0.000000,0.501961,0.000000}%
\pgfsetstrokecolor{currentstroke}%
\pgfsetdash{}{0pt}%
\pgfpathmoveto{\pgfqpoint{0.272727in}{2.746437in}}%
\pgfpathlineto{\pgfqpoint{0.280936in}{2.058140in}}%
\pgfpathlineto{\pgfqpoint{0.289146in}{1.899028in}}%
\pgfpathlineto{\pgfqpoint{0.297355in}{1.804273in}}%
\pgfpathlineto{\pgfqpoint{0.305564in}{1.736564in}}%
\pgfpathlineto{\pgfqpoint{0.313773in}{1.683842in}}%
\pgfpathlineto{\pgfqpoint{0.321982in}{1.640660in}}%
\pgfpathlineto{\pgfqpoint{0.338401in}{1.572369in}}%
\pgfpathlineto{\pgfqpoint{0.354819in}{1.519296in}}%
\pgfpathlineto{\pgfqpoint{0.371237in}{1.475879in}}%
\pgfpathlineto{\pgfqpoint{0.387656in}{1.439139in}}%
\pgfpathlineto{\pgfqpoint{0.404074in}{1.407294in}}%
\pgfpathlineto{\pgfqpoint{0.420493in}{1.379191in}}%
\pgfpathlineto{\pgfqpoint{0.445120in}{1.342395in}}%
\pgfpathlineto{\pgfqpoint{0.469748in}{1.310508in}}%
\pgfpathlineto{\pgfqpoint{0.494375in}{1.282372in}}%
\pgfpathlineto{\pgfqpoint{0.519003in}{1.257198in}}%
\pgfpathlineto{\pgfqpoint{0.551839in}{1.227285in}}%
\pgfpathlineto{\pgfqpoint{0.584676in}{1.200698in}}%
\pgfpathlineto{\pgfqpoint{0.617513in}{1.176770in}}%
\pgfpathlineto{\pgfqpoint{0.658559in}{1.149875in}}%
\pgfpathlineto{\pgfqpoint{0.699605in}{1.125698in}}%
\pgfpathlineto{\pgfqpoint{0.748860in}{1.099580in}}%
\pgfpathlineto{\pgfqpoint{0.798115in}{1.076033in}}%
\pgfpathlineto{\pgfqpoint{0.855579in}{1.051202in}}%
\pgfpathlineto{\pgfqpoint{0.913043in}{1.028706in}}%
\pgfpathlineto{\pgfqpoint{0.978717in}{1.005345in}}%
\pgfpathlineto{\pgfqpoint{1.052600in}{0.981530in}}%
\pgfpathlineto{\pgfqpoint{1.134691in}{0.957580in}}%
\pgfpathlineto{\pgfqpoint{1.224992in}{0.933738in}}%
\pgfpathlineto{\pgfqpoint{1.323503in}{0.910179in}}%
\pgfpathlineto{\pgfqpoint{1.430222in}{0.887029in}}%
\pgfpathlineto{\pgfqpoint{1.545151in}{0.864372in}}%
\pgfpathlineto{\pgfqpoint{1.676497in}{0.840858in}}%
\pgfpathlineto{\pgfqpoint{1.816054in}{0.818173in}}%
\pgfpathlineto{\pgfqpoint{1.972028in}{0.795127in}}%
\pgfpathlineto{\pgfqpoint{2.144421in}{0.771998in}}%
\pgfpathlineto{\pgfqpoint{2.333232in}{0.748992in}}%
\pgfpathlineto{\pgfqpoint{2.538462in}{0.726263in}}%
\pgfpathlineto{\pgfqpoint{2.727273in}{0.707101in}}%
\pgfpathlineto{\pgfqpoint{2.727273in}{0.707101in}}%
\pgfusepath{stroke}%
\end{pgfscope}%
\begin{pgfscope}%
\pgfsetrectcap%
\pgfsetmiterjoin%
\pgfsetlinewidth{0.803000pt}%
\definecolor{currentstroke}{rgb}{0.000000,0.000000,0.000000}%
\pgfsetstrokecolor{currentstroke}%
\pgfsetdash{}{0pt}%
\pgfpathmoveto{\pgfqpoint{0.150000in}{0.571603in}}%
\pgfpathlineto{\pgfqpoint{0.150000in}{2.850000in}}%
\pgfusepath{stroke}%
\end{pgfscope}%
\begin{pgfscope}%
\pgfsetrectcap%
\pgfsetmiterjoin%
\pgfsetlinewidth{0.803000pt}%
\definecolor{currentstroke}{rgb}{0.000000,0.000000,0.000000}%
\pgfsetstrokecolor{currentstroke}%
\pgfsetdash{}{0pt}%
\pgfpathmoveto{\pgfqpoint{2.850000in}{0.571603in}}%
\pgfpathlineto{\pgfqpoint{2.850000in}{2.850000in}}%
\pgfusepath{stroke}%
\end{pgfscope}%
\begin{pgfscope}%
\pgfsetrectcap%
\pgfsetmiterjoin%
\pgfsetlinewidth{0.803000pt}%
\definecolor{currentstroke}{rgb}{0.000000,0.000000,0.000000}%
\pgfsetstrokecolor{currentstroke}%
\pgfsetdash{}{0pt}%
\pgfpathmoveto{\pgfqpoint{0.150000in}{0.571603in}}%
\pgfpathlineto{\pgfqpoint{2.850000in}{0.571603in}}%
\pgfusepath{stroke}%
\end{pgfscope}%
\begin{pgfscope}%
\pgfsetrectcap%
\pgfsetmiterjoin%
\pgfsetlinewidth{0.803000pt}%
\definecolor{currentstroke}{rgb}{0.000000,0.000000,0.000000}%
\pgfsetstrokecolor{currentstroke}%
\pgfsetdash{}{0pt}%
\pgfpathmoveto{\pgfqpoint{0.150000in}{2.850000in}}%
\pgfpathlineto{\pgfqpoint{2.850000in}{2.850000in}}%
\pgfusepath{stroke}%
\end{pgfscope}%
\begin{pgfscope}%
\pgfsetbuttcap%
\pgfsetmiterjoin%
\definecolor{currentfill}{rgb}{1.000000,1.000000,1.000000}%
\pgfsetfillcolor{currentfill}%
\pgfsetfillopacity{0.800000}%
\pgfsetlinewidth{1.003750pt}%
\definecolor{currentstroke}{rgb}{0.800000,0.800000,0.800000}%
\pgfsetstrokecolor{currentstroke}%
\pgfsetstrokeopacity{0.800000}%
\pgfsetdash{}{0pt}%
\pgfpathmoveto{\pgfqpoint{1.048729in}{2.095402in}}%
\pgfpathlineto{\pgfqpoint{2.752778in}{2.095402in}}%
\pgfpathquadraticcurveto{\pgfqpoint{2.780556in}{2.095402in}}{\pgfqpoint{2.780556in}{2.123179in}}%
\pgfpathlineto{\pgfqpoint{2.780556in}{2.752778in}}%
\pgfpathquadraticcurveto{\pgfqpoint{2.780556in}{2.780556in}}{\pgfqpoint{2.752778in}{2.780556in}}%
\pgfpathlineto{\pgfqpoint{1.048729in}{2.780556in}}%
\pgfpathquadraticcurveto{\pgfqpoint{1.020952in}{2.780556in}}{\pgfqpoint{1.020952in}{2.752778in}}%
\pgfpathlineto{\pgfqpoint{1.020952in}{2.123179in}}%
\pgfpathquadraticcurveto{\pgfqpoint{1.020952in}{2.095402in}}{\pgfqpoint{1.048729in}{2.095402in}}%
\pgfpathlineto{\pgfqpoint{1.048729in}{2.095402in}}%
\pgfpathclose%
\pgfusepath{stroke,fill}%
\end{pgfscope}%
\begin{pgfscope}%
\pgfsetrectcap%
\pgfsetroundjoin%
\pgfsetlinewidth{1.505625pt}%
\definecolor{currentstroke}{rgb}{0.000000,0.000000,1.000000}%
\pgfsetstrokecolor{currentstroke}%
\pgfsetdash{}{0pt}%
\pgfpathmoveto{\pgfqpoint{1.076507in}{2.535803in}}%
\pgfpathlineto{\pgfqpoint{1.215396in}{2.535803in}}%
\pgfpathlineto{\pgfqpoint{1.354285in}{2.535803in}}%
\pgfusepath{stroke}%
\end{pgfscope}%
\begin{pgfscope}%
\definecolor{textcolor}{rgb}{0.000000,0.000000,0.000000}%
\pgfsetstrokecolor{textcolor}%
\pgfsetfillcolor{textcolor}%
\pgftext[x=1.465396in,y=2.487192in,left,base]{\color{textcolor}\sffamily\fontsize{10.000000}{12.000000}\selectfont \(\displaystyle F(x) = \int_x^{+\infty} \frac{\mathrm{e}^{-t}}{t} \, \mathrm{d} t\)}%
\end{pgfscope}%
\begin{pgfscope}%
\pgfsetrectcap%
\pgfsetroundjoin%
\pgfsetlinewidth{1.505625pt}%
\definecolor{currentstroke}{rgb}{0.000000,0.501961,0.000000}%
\pgfsetstrokecolor{currentstroke}%
\pgfsetdash{}{0pt}%
\pgfpathmoveto{\pgfqpoint{1.076507in}{2.234291in}}%
\pgfpathlineto{\pgfqpoint{1.215396in}{2.234291in}}%
\pgfpathlineto{\pgfqpoint{1.354285in}{2.234291in}}%
\pgfusepath{stroke}%
\end{pgfscope}%
\begin{pgfscope}%
\definecolor{textcolor}{rgb}{0.000000,0.000000,0.000000}%
\pgfsetstrokecolor{textcolor}%
\pgfsetfillcolor{textcolor}%
\pgftext[x=1.465396in,y=2.185679in,left,base]{\color{textcolor}\sffamily\fontsize{10.000000}{12.000000}\selectfont \(\displaystyle x \mapsto -\ln(x)\)}%
\end{pgfscope}%
\end{pgfpicture}%
\makeatother%
\endgroup%

    \caption{Représentation graphique de la fonction $F$ proche de $0$}
\end{marginfigure}

\begin{elemsolution}
\begin{reponses}
\item La fonction $f : t \mapsto \frac{\e^{-t}}{t}$ est positive et continue sur $\R^*$.

\begin{itemize}
\item D'après le théorème des croissances comparées, $\frac{\e^{-t}}{t} = o_{+\infty}\left(\frac{1}{t^2}\right)$. Ainsi, d'après le théorème de comparaison des intégrales de fonctions à valeurs positives, $\int_1^{+\infty} \frac{\e^{-t}}{t} \d t$ converge.

\item Comme $f(t) \sim_0 \frac{1}{t}$, d'après le théorème de comparaison des intégrales de fonctions à valeurs positives, $\int_0^1 \frac{\e^{-t}}{t} \d t$ diverge.
\end{itemize}

Ainsi, le domaine de définition de $F$ est $]0, + \infty[$.

\item Les fonctions $u : t \mapsto -\e^{-t}$ et $v : t \mapsto \frac{1}{t}$ sont de classe $\mathscr{C}^1$ sur $\interfo{x}{+\infty}$ et $\lim\limits_{t\to+\infty} u(t) v(t) = 0$. Ainsi, d'après le théorème d'intégration par parties, pour $x > 1$, 
\begin{align*}
F(x)
&= \int_x^{+ \infty} \frac{\e^{-t}}{t} \d t 
= \frac{\e^{-x}}{x} - \int_x^{+\infty} \frac{\e^{-t}}{t^2} \d t.
\end{align*}

De plus,
\begin{align*}
\int_x^{+\infty} \frac{\e^{-t}}{t^2} \d t
&\leq \e^{-x} \int_x^{+\infty} \frac{1}{t^2} \d t
\leq \frac{\e^{-x}}{x^2}.
\end{align*}

Ainsi,
$$\frac{\e^{-t}}{t^2} = o_{+\infty}\left( \frac{\e^{-t}}{t} \right).$$

D'où,
\[
\int_x^{+\infty} \frac{\e^{-t}}{t} \d t \sim_{+\infty} \frac{\e^{-x}}{x}.
\]

\begin{remarque}
On aurait également pu utiliser les théorèmes d'intégration des relations de comparaison car $\e^{-x}/x = o(\e^{-x}/x^2)$.
\end{remarque}

\item On effectue une nouvelle intégration par parties avec $u : t \mapsto \e^{-t}$ et $v : t \mapsto \frac{1}{t^2}$. On obtient ainsi
\begin{align*}
F(x)
&= \frac{\e^{-x}}{x} + \frac{\e^{-x}}{x^2} - \int_x^{+\infty} \frac{2 \e^{-t}}{t^3} \d t.
\end{align*}

On montre alors comme précédemment que l'intégrable est négligeable, en $+\infty$, devant $\frac{\e^{-x}}{x^2}$.

\item En utilisant la relation de \nom{Chasles},
\begin{align*}
F(x)
&= \int_x^1 \frac{\e^{-t}}{t} \d t + \int_1^{+\infty} \frac{\e^{-t}}{t} \d t.
\end{align*}

De plus, la fonction exponentielle étant convexe, pour tout $t > 0$,
\begin{align*}
1 - t \leq \e^{-t} &\leq 1\\
\frac{1}{t} - 1 &\leq \frac{\e^{-t}}{t} \leq \frac{1}{t}\\
-\ln(x) - (1 - x) &\leq \int_x^1 \frac{\e^{-t}}{t} \d t \leq -\ln(x).
\end{align*}

D'après le théorème d'encadrement, $\int_x^1 \frac{\e^{-t}}{t} \d t \sim_0 -\ln(x)$ et
\[
F(x) \sim_0 -\ln(x).
\]
\end{reponses}
\end{elemsolution}
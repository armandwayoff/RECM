\section{Version intégrale du lemme de \textsc{Cesàro}}

\todoarmand{Voir aussi la section \ref{variante_cesaro}.}

\todoinline{Je mettrais ceci dans un thème Cesaro, dans une partie analyse élémentaire en mettant la version avec les suites, la version $2^{-n} \sum_{k=0}^n \binom{n}{k} u_k \to \ell$ et la version $f(x + 1) - f(x) \to \ell$.}

\begin{itemize}
    \item Pour les applications : \url{http://luls55.free.fr/fichiers/Analyse1/SUITES/11Cesaro.pdf} (II) : déterminer $\lim\limits_{n \to +\infty} w_n$ où $$w_n = \prod_{k=1}^n \left(1 + \frac{2}{k}\right)^{k/n}$$
    \item \url{http://poiret.aurelien.free.fr/Musculation/Analyse/Theoremes%20Tauberiens/Theoremes%20Tauberiens.pdf}
    \item \url{https://math-os.com/histoire-lemme-cesaro/}
\end{itemize}


\begin{theo}
Soit $f$ une fonction continue sur $\Rp$ et $\ell \in \overline{\R}$ tels que $\lim\limits_{+\infty} f = \ell$. Alors 
\[
\lim_{x \to + \infty} \frac{1}{x} \int_0^x f(t) \d t = \ell.
\]
\end{theo}

\begin{exercice}
Soit $f$ une fonction continue sur $\Rp$ telle que $\lim\limits_{x\to+\infty} f(x) = \ell$. On suppose que $\ell \in \R$. Soit $\varepsilon > 0$.
\begin{questions}
\item Montrer qu'il existe $x_0 \in \Rp$ tel que
\[
\forall x \geqslant x_0,\quad \module{f(x) - \ell} \leq \varepsilon.
\]
\todoarmand{Est-ce qu'on garde la Q1) ?}

\item En déduire qu'il existe une constante $K$ telle que pour tout $x \geqslant x_0$,
\[
\module{\frac{1}{x} \int_0^x f(t) \d t - \ell} \leqslant \frac{K}{x} + \varepsilon.
\]

\item Conclure.

\item Démontrer le résultat lorsque $\ell = +\infty$.
\end{questions}
\end{exercice}

\begin{solution}
La démonstration est directement adaptée de celle de la version discrète. 
\begin{reponses}
\item Comme la fonction $f$ converge vers $\ell$ en $+ \infty$, il existe $x_0 \in \Rp$ tel que pour tout $x \geqslant x_0$, $|f(x) - \ell| \leqslant \varepsilon$. \\

\item Comme la fonction $f$ est continue sur le segment $\interff{0}{x_0}$, elle est bornée sur ce segment. Ainsi, en utilisant la relation de \textsc{Chasles},
\begin{align*}
\left| \frac{1}{x} \int_0^x f(t) \d t - \ell \right| &= \left| \frac{1}{x} \int_0^x \mathopen{}\big(f(t) - \ell\big) \d t \right| \\
\text{par l'inégalité triangulaire} &\leqslant \frac{1}{x} \int_0^x |f(t) - \ell| \d t \\
&\leqslant \frac{1}{x} \Bigg( \underbrace{\int_{0}^{x_0} |f(t) - \ell| \d t}_{\defeq K} + \int_{x_0}^{x} \underbrace{|f(t) - \ell|}_{\leqslant \varepsilon} \d t \Bigg) \\
&\leqslant \frac{K}{x} + \varepsilon
\end{align*}

\item Comme $\lim\limits_{x \to \infty} \frac{K}{x} = 0$, il existe $x_1 \in \Rp$ tel que pour tout $x \geqslant x_1$, $ \big| \frac{K}{x} \big| \leqslant \varepsilon$.

Ainsi pour tout $x \geqslant \max \{ x_0, x_1 \}$, 
$$\left| \frac{1}{x} \int_0^x f(t) \d t - \ell \right| \leqslant 2 \varepsilon.$$
On en déduit le résultat. 

\item On suppose que $\ell = +\infty$. Soit $M \in \Rp$. Comme $f$ tend vers $+\infty$ en $+\infty$, il existe $x_0 \in \Rp$ tel que
\[
\forall x \geqslant x_0,\quad f(x) \geqslant M.
\]

Comme $f$ est continue sur le segment $\interff{0}{x_0}$, elle est bornée par une constante $K$ sur ce segment. \\
Ainsi, en utilisant la relation de \textsc{Chasles} et l'inégalité triangulaire, pour tout $x \geqslant x_0$,
\begin{align*}
\frac{1}{x} \int_0^x f(t) \d t
&= \frac{1}{x} \int_0^{x_0} f(t) \d t + \frac{1}{x} \int_{x_0}^x f(t) \d t\\
&\geqslant -\frac{K}{x} + M (x - x_0).
\end{align*}

Comme $\lim\limits_{x\to+\infty} \frac{K}{x} = 0$ et $\lim\limits_{x\to+\infty} M (x - x_0) = +\infty$, il existe un réel $x_1$ tel que
\[
\forall x \geqslant x_1,\quad \frac{K}{x} \geqslant -\frac{1}{2} \quad \text{et} \quad M (x - x_0) \geqslant M + \frac{1}{2}.
\]
Ainsi, pour tout $x \geqslant \max\{x_0, x_1\}$,
\begin{align*}
\frac{1}{x} \int_0^x f(t) \d t
&\geqslant M.
\end{align*}
Finalement, 
\[
\lim\limits_{x\to+\infty} \frac{1}{x}\int_0^x f(t) \d t = +\infty.
\]
\end{reponses}
\end{solution}
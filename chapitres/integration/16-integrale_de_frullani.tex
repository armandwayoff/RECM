\section{Intégrale de \nom{Frullani}}

\todoinline{Pourrait-on ajouter une illustration ici ? Faudrait voir ce que cela fait de tracer ces fonctions...}

\todoarmand{Oui, une illustration serait appréciable. J'ai fait quelques graphiques mais je n'ai encore rien trouvé de convaincant.}

\todoarmand{
Exercices 51, 52, 53 de \url{http://vonbuhren.free.fr/Prepa/Colles/integration_intervalle_quelconque.pdf}}

\todoinline{Je ne mettrais pas de référence ici car c'est très classique.}

\begin{theo}\label{theo:frullani}
Soit $\fonctionligne[f]{\interoo{0}{+\infty}}{\R}$ une fonction continue et $\ell$ et $L$ deux réels tels que
\[
\lim_{x\to0} f(x) = \ell
\quad \text{et} \quad 
\lim_{x\to+\infty} f(x) = L.
\]
Alors, l'intégrale $I = \int_0^{+\infty} \frac{f(a t) - f(b t)}{t} \d t$ converge et a pour valeur
\[
\int_0^{+\infty} \frac{f(a t) - f(b t)}{t} \d t = (\ell - L) \ln\frac{b}{a}.
\]
\end{theo}

\begin{exercice}
Soient $f$ une fonction satisfaisant les hypothèses du théorème \ref{theo:frullani} et $0 < \varepsilon \leqslant M$.
\begin{questions}
\item Montrer que
\[
\int_\varepsilon^M \frac{f(a t) - f(b t)}{t} \d t
= \int_{a\varepsilon}^{b\varepsilon} \frac{f(t)}{t} \d t - \int_{a M}^{b M} \frac{f(t)}{t} \d t.
\]

\item Montrer que
\[
\lim_{\varepsilon \to 0} \int_{a\varepsilon}^{b\varepsilon} \frac{f(t) - \ell}{t} \d t = 0.
\]

\item Conclure.
\end{questions}
\end{exercice}

\begin{solution}
\begin{reponses}
\item La fonction $f$ étant continue sur $\interff{\varepsilon}{M}$, les intégrales sont bien définies et on utilise les changements de variables $u \mapsto \frac{u}{a}$ et $u \mapsto \frac{u}{b}$ puis la relation de \nom{Chasles}:
\begin{align*}
\int_\varepsilon^M \frac{f(a t) - f(b t)}{t} \d t
&= \int_\varepsilon^M \frac{f(a t)}{t} \d t - \int_\varepsilon^M \frac{f(b t)}{t} \d t\\
&= \int_{a\varepsilon}^{a M} \frac{f(u)}{\frac{u}{a}} \frac{1}{a} \d u - \int_{b\varepsilon}^{b M} \frac{f(u)}{\frac{u}{b}} \frac{1}{b} \d u\\
&= \int_{a\varepsilon}^{a M} \frac{f(u)}{u} \d u - \int_{b\varepsilon}^{b M} \frac{f(u)}{u} \d u \\
&= \int_{a\varepsilon}^{b\varepsilon} \frac{f(u)}{u} \d u + \cancel{\int_{b\varepsilon}^{aM} \frac{f(u)}{u} \d u} - \cancel{\int_{b\varepsilon}^{aM} \frac{f(u)}{u} \d u} - \int_{aM}^{bM} \frac{f(u)}{u} \d u.
\end{align*}

\item Soit $\eta > 0$. Comme $\lim\limits_{x \to 0} f(x) = \ell$, il existe $\delta > 0$ tel que pour tout $x \in \interff{0}{\delta}$,
\[
\module{f(x) - \ell} < \eta.
\]
Soit $\varepsilon < \frac{\delta}{b}$. Alors, $\interff{a\varepsilon}{b\varepsilon} \subset \interff{0}{\delta}$ et par l'inégalité triangulaire
\begin{align*}
\module{\int_{a\varepsilon}^{b\varepsilon} \frac{f(t) - \ell}{t} \d t}
&\leqslant \int_{a\varepsilon}^{b\varepsilon} \frac{\module{f(t) - \ell}}{t} \d t\\
&\leqslant \eta \int_{a\varepsilon}^{b\varepsilon} \frac{1}{t} \d t\\
&\leqslant \eta \ln\frac{b}{a}.
\end{align*}

Ainsi, $\lim\limits_{\varepsilon\to0} \module{\int_{a\varepsilon}^{b\varepsilon} \frac{f(t) - \ell}{t} \d t} = 0$ soit $\lim\limits_{\varepsilon \to 0} \int_{a\varepsilon}^{b\varepsilon} \frac{f(t)}{t} \d t = \ell \ln \frac{b}{a}$.

\item On montre de manière analogue que
\[
\lim_{M\to+\infty} \int_{aM}^{bM} \frac{f(t) - L}{t} \d t = 0.
\]
Ainsi, l'intégrale $\int_0^{+\infty} \frac{f(a t) - f(b t)}{t} \d t$ converge et
\[
\int_0^{+\infty} \frac{f(a t) - f(b t)}{t} \d t
= \ell \ln\frac{b}{a} - L \ln\frac{b}{a}.
\]
\end{reponses}
\end{solution}

\begin{exercice}
Soient $a$ et $b$ deux réels tels que $0<a<b$. Montrer que les intégrales suivantes sont convergentes et calculer leur valeur.
\begin{questions}
\item $I = \displaystyle \int_0^{+\infty} \frac{\e^{-ax} - \e^{-bx}}{x} \d x$.
\item $J = \displaystyle \int_0^{+\infty} \frac{\arctan(bx) - \arctan(ax)}{x} \d x$.
\item $K = \displaystyle \int_0^{+\infty} \frac{\tanh(3x) - \tanh(x)}{x} \d x$.\hspace*{\fill} \textsl{(Hélène de Troie, Oraux Mines 2016)}
\item $L = \displaystyle \int_0^1 \frac{t - 1}{\ln(t)} \d t$.
\end{questions}
\end{exercice}

\begin{solution}
\begin{reponses}
\item Comme $\lim\limits_{x\to+\infty} \e^{-x} = 0$ et $\lim\limits_{x\to0} \e^{-x} = 1$, alors $I$ converge et
\[
\int_0^{+\infty} \frac{\e^{-a x} - \e^{-b x}}{x} = \ln\frac{b}{a}.
\]

\item Comme $\lim\limits_{x\to+\infty} \arctan(x) = \frac{\pi}{2}$ et $\lim\limits_{x\to0} \arctan(x) = 0$, alors $J$ converge et
\[
\int_0^{+\infty} \frac{\arctan(b x) - \arctan(a x)}{x} \d x = -\frac{\pi}{2} \ln\frac{b}{a}
= \frac{\pi}{2} \ln\frac{a}{b}.
\]

\item Comme $\lim\limits_{x\to+\infty} \tanh(x) = 1$ et $\lim\limits_{x\to 0} \tanh(x) = 0$, alors $K$ converge et
\[
\int_0^{+\infty} \frac{\tanh(3 x) - \tanh(x)}{x} \d x
= -\ln\frac{1}{3}
= \ln(3).
\]

\item Le changement de variable $u \mapsto \e^{-u}$ est $\mathscr{C}^1$ et bijectif. Ainsi, l'intégrale $L$ converge si et seulement si $\int_0^{+\infty} \frac{\e^{-u} - 1}{-u} \e^{-u} \d u$ converge. Cette intégrale s'écrit également $\int_0^{+\infty} \frac{\e^{-u} - \e^{-2u}}{u} \d u$.

En utilisant un raisonnement analogue à celui effectué pour l'intégrale $I$, l'intégrale $L$ converge et
\[
\int_0^1 \frac{t - 1}{\ln(t)} \d t
= \ln(2).
\]
\end{reponses}
\end{solution}

% \begin{exercice}
    % % % % Soit $\fonctionens{\interoo{0}{+\infty}}{\R}$ une fonction continue admettant une limite finie $L$ en $+\infty$ et une limite finie $\ell$ en $0$. On considère un couple $(a,b) \in \R^2$ vérifiant $0<a<b$ et les intégrales
    % \[
    % % I = \int_0^{+\infty} \frac{f(at) - f(bt)}{t} \d t \quad \text{et} \quad J = \int_0^1 \frac{t-1}{\ln(t)} \d t.
    % \]
    % \begin{enumerate}
        % % \item Montrer que l'intégrale $I$ est convergente et calculer sa valeur. 
        % % \item En déduire que l'intégrale $J$ converge et calculer sa valeur. 
    % \end{enumerate}
% \end{exercice}

% \begin{exercice}
    % % % Soient $\fonctionens{\interof{0}{+\infty}}{\R}$ une fonction continue, un couple $(a, b) \in \R^2$ avec $0<a<b$ et les intégrales
    % \[
    % % I = \int_1^{+\infty} \frac{f(t)}{t} \d t \quad \text{et} \quad J = \int_0^{+\infty} \frac{f(at)-f(bt)}{t} \d t.
    % \]
    % On suppose que l'intégrale $I$ est convergente. 
    % \begin{enumerate}
        % \item Montrer que pour tout $x \in \Rpe$, on a 
        % \[
        % % \int_x^{+\infty} \frac{f(at) - f(bt)}{t} \d t = \int_{ax}^{bx} \frac{f(t)}{t} \d t.
        % \]
        % % \item En déduire que l'intégrale $J$ converge et calculer sa valeur.
    % \end{enumerate}
    % \end{exercice}

% \todoinline{Ajout en vrac d'exercices ci-dessous}



% %---------------

% \begin{exercice}
% {Mines}
% {16}
% {RMS 728}
% % Montrer l'existence puis calculer la valeur de $\int_0^{+\infty} \frac{\tanh(3x) - \tanh(x)}{x} \d x$.
% \end{exercice}

% \begin{preuve}
% % La fonction $f$ est continue sur $\R_+^\ast$, prolongeable par continuité par $2$ en $0$. De plus, en $+\infty$,
% \[
% % \tanh(3 x) - \tanh(x) = \frac{1 - \e^{-6x}}{1 + \e^{-6 x}} - \frac{1 - \e^{-2x}}{1 + \e^{-2x}} = 2 \e^{-2x} + o(\e^{-2x}),
% \]
% % et $f$ est un $o(1/x^2)$. Ainsi, $f$ est intégrable sur $\R_+^\ast$.

% À l'aide d'une formule de changement de variables,
% \[
% % \int_0^M \frac{\tanh(3 x) - \tanh(x)}{x} \d x = \int_{M}^{3M} \frac{\tanh(x)}{x} \d x
% \]
% % Or, quand $M$ est grand, $\abs{\tanh(x) - 1} \leq \eta$ et $\int_M^{3M} \frac{\tanh(x) - 1}{x} \d x \leq \eta \ln(3)$.

% Finalement, $\int_0^{+\infty} \frac{\tanh(3x) - \tanh(x)}{x} \d x = \ln(3)$.
% \end{preuve}
\section{Aire et fonction réciproque}

\begin{prop}
Soit $f$ de classe $\Cont^1$ sur $\interff{a}{b}$ telle que $f'$ soit strictement positive sur $\interff{a}{b}$. Alors,
\[
\int_{a}^{b} f(t) \d t + \int_{f(a)}^{f(b)} f^{-1}(t) \d t = b f(b) - a f(a).
\]
\end{prop}

\begin{elemsolution}
    \begin{enumerate}
    \item Comme $f' > 0$, alors $f$ est strictement croissante. Comme $f$ est continue, d'après le \theoremeutilise{théorème de la bijection monotone}{theo:bijectionmonotone}, $f$ réalise une bijection de $\interff{a}{b}$ sur $\interff{f(a)}{f(b)}$.
    
    \item On utilise le changement de variable $\fonctionens[\phi]{\interff{a}{b}}{\interff{f(a)}{f(b)}}, u \mapsto f(u)$. Alors, $\phi$ est de classe $\Cont^1$ et
    \begin{align*}
    \int_{f(a)}^{f(b)} f^{-1}(t) \d t
    &= \int_a^b f^{-1}\mathopen{}\big(f(u)\big) f'(u) \d u\\
    &= \int_a^b u f'(u) \d u\\
    &= \big[u f(u)\big]_a^b - \int_a^b f(u) \d u,
    \end{align*}
    où on a réalisé une intégration par parties.
    \end{enumerate}

    Finalement,
    \[
    \int_a^b f(t) \d t + \int_{f(a)}^{f(b)} f^{-1}(t) \d t = b f(b) - a f(a).
    \]
\end{elemsolution}

\begin{figure}[H]
    \centering
    \begin{tikzpicture}[scale=0.65]

    \def\u{0.5}
    \def\v{0.2}
    
    \def\xa{5}
    \def\ya{1.0}
    \def\xb{6.5}
    \def\yb{2}
    \def\xc{8}
    \def\yc{4}

\begin{scope}[local bounding box=struct, scale=1]

    \path (\xa, \ya)    coordinate (A)
        (\xb, \yb)    coordinate (B)
        (\xc, \yc)  coordinate (C)
        (\ya, \xa)    coordinate (A')
        (\yb, \xb)    coordinate (B')
        (\yc, \xc)  coordinate (C');

    \begin{scope}
        \clip (A') ..controls +(0.05*\u, 0.05*\u) and ( $(B') + (-5*\v, -0.5*\u)$ )..
        (B') ..controls +(5*\v, 0.5*\u) and ( $(C') + (-2*\v, -2*\v)$ )..
        (C') |- cycle;
        %\clip (A') ..controls +(0.05*\u, 0.05*\u) and ( $(B') + (-0.5*\v, -0.5*\u)$ )..
        %(B') ..controls +(5*\v, 0.5*\u) and ( $(C') + (-2*\v, -2*\v)$ )..
        %(C') |- cycle;
        \foreach \x in {\ya00, \ya01,...,\yc.000}   
            \draw[mylightblue, opacity=0.5] (\x,0) -- ++(0,8);
    \end{scope}
    
    \fill[mylightblue] (\ya,0) rectangle ++(\yc-\ya,\xa);
 
    \begin{scope}
        \clip (A) ..controls +(0.05*\u, 0.05*\u) and ( $(B) + (-0.5*\u, -5*\v)$ )..
        (B) ..controls +(0.5*\u, 5*\v) and ( $(C) + (-2*\v, -2*\v)$ )..
            (C) |- cycle;
        \foreach \x in {\xa.000, \xa.001,...,\xc.000}   
            \draw[mylightred, opacity=0.5] (\x,-\ya) -- ++(0,10);
    \end{scope}

    \draw[thick, myblue] 
        (A') ..controls +(0.05*\u, 0.05*\u) and ( $(B') + (-5*\v, -0.5*\u)$ )..
        (B') ..controls +(5*\v, 0.5*\u) and ( $(C') + (-2*\v, -2*\v)$ )..
        (C') node[right] {$\mathcal{C}_{f^{-1}}$};
        
    \draw[thick, myred] 
        (A) ..controls +(0.05*\u, 0.05*\u) and ( $(B) + (-0.5*\u, -5*\v)$ )..
        (B) ..controls +(0.5*\u, 5*\v) and ( $(C) + (-2*\v, -2*\v)$ )..
            (C) node[above] {$\mathcal{C}_f$};


    \fill[mylightred] (\xa,0) rectangle ++(\xc-\xa,\ya);

    \draw[gray] (-.5, -.5) -- (\xc + 0.5, \xc + 0.5) node[pos=0.95, above, sloped] {\footnotesize $y = x$};

    
    \draw[dashed] (\xa, 0) node[black,below]{\footnotesize$a$} -- (A);
    \draw[dashed] (0, \ya) node[black,left]{\footnotesize$f(a)$} -- (A);
    
    \draw[dashed] (\xc, 0) node[black,below]{\footnotesize$b$} -- (C);
    \draw[dashed] (0, \yc) node[black,left]{\footnotesize$f(b)$} -- (C);

    \draw[dashed] (0, \xa) node[black,left]{\footnotesize$a$} -- (A');
    \draw[dashed] (\ya, 0) node[black,below]{\footnotesize$f(a)$} -- (A');
    
    \draw[dashed] (0, \xc) node[black,left]{\footnotesize$b$} -- (C');
    \draw[dashed] (\yc, 0) node[black,below]{\footnotesize$f(b)$} -- (C');

    % \draw[->, black] (5.5,2) node[above] 
    \node at (6.8,0.7)
    {\footnotesize \contour{mylightred}{\color{myred}$\displaystyle \int_a^b f(t)\, \mathrm{d} t$}}; % to [out=-90,in=180] ($(7,1.5)$);

    % \draw[->, black] (3,-0.5) node[below] 
    % {\footnotesize \contour{white}{$\displaystyle \int_{f(a)}^{f(b)} f^{-1}(t)\, \mathrm{d} t$}} to [out=70,in=-70] ($(3,1.5)$);
    \node at ((2.5, 5.8) [rotate=45] {\footnotesize \contour{mylightblue}{\color{myblue}$\displaystyle \int_{f(a)}^{f(b)} f^{-1}(t)\, \mathrm{d} t$}};

    \draw[thick, -latex] (-.5, 0) -- (\xc + 0.5, 0) node[above] {$x$};
    \draw[thick, -latex] (0, -.5) -- (0, \xc + 0.5) node[left] {$y$};

\end{scope}

\draw[thick, ->] (\xc + 1, 2) -- (11, 2);

\begin{scope}[shift={(12,0))}
,baseline=(struct.south)
]
    \path (\xa, \ya)    coordinate (A)
        (\xb, \yb)    coordinate (B)
        (\xc, \yc)  coordinate (C)
        (\ya, \xa)    coordinate (A')
        (\yb, \xb)    coordinate (B')
        (\yc, \xc)  coordinate (C');

    \fill[mylightblue] (0, \ya) rectangle (\xc, \yc);

    \begin{scope}
        \clip (A) ..controls +(0.05*\u, 0.05*\u) and ( $(B) + (-0.5*\u, -5*\v)$ )..
        (B) ..controls +(0.5*\u, 5*\v) and ( $(C) + (-2*\v, -2*\v)$ )..
            (C) |- cycle;
        \foreach \x in {\xa.000, \xa.001,...,\xc.000}   
            \draw[mylightred, opacity=0.5] (\x,-\ya) -- ++(0,10);
    \end{scope}

    \draw[dashed] 
        (A) ..controls +(0.05*\u, 0.05*\u) and ( $(B) + (-0.5*\u, -5*\v)$ )..
        (B) ..controls +(0.5*\u, 5*\v) and ( $(C) + (-2*\v, -2*\v)$ )..
            (C);


    \fill[mylightred] (\xa,0) rectangle ++(\xc-\xa,\ya);

    \fill[mylightgreen] (0, 0) rectangle (A);
    \node at (\xa/2, \ya/2) {\footnotesize \color{mygreen}$a \times f(a)$};

    \draw[dashed] (\xa, 0) node[black,below]{\footnotesize$a$} -- (A);
    \draw[dashed] (0, \ya) node[black,left]{\footnotesize$f(a)$} -- (A);
    
    \draw[dashed] (\xc, 0) node[black,below]{\footnotesize$b$} -- (C);
    \draw[dashed] (0, \yc) node[black,left]{\footnotesize$f(b)$} -- (C);

    % \draw[->, black] (5.5,2) node[above] 
    \node at (6.8,0.7)
    {\footnotesize \contour{mylightred}{\color{myred}$\displaystyle \int_a^b f(t)\, \mathrm{d} t$}}; % to [out=-90,in=180] ($(7,1.5)$);

    \node at ((4.8, 2.5) {\footnotesize \contour{mylightblue}{\color{myblue}$\displaystyle \int_{f(a)}^{f(b)} f^{-1}(t)\, \mathrm{d} t$}};
    
    \draw[gray] (-.5, -.5) -- (\xc + 0.5, \xc + 0.5) node[pos=0.95, above, sloped] {\footnotesize $y = x$};
    
    \draw[thick, -latex] (-.5, 0) -- (\xc + 0.5, 0) node[above] {$x$};
    \draw[thick, -latex] (0, -.5) -- (0, \xc + 0.5) node[left] {$y$};

    \draw[ultra thick, myorange] (0, 0) rectangle (\xc, \yc) node[above left] {\footnotesize \color{myorange} $b \times f(b)$};
\end{scope}

\end{tikzpicture}
    \caption{Illustration géométrique de la formule d'intégration par parties}
    \label{fig:i_01-une_propriete_geometrique_de_l_integrale}
\end{figure}

\begin{comment}
\begin{remarque}
La représentation graphique \ref{fig:i_01-une_propriete_geometrique_de_l_integrale} permet d'illustrer géométriquement la formule d'intégration par parties.

En effet, si $f$, $g$ sont de classe $\mathcal{C}^1$ et strictement monotones, alors elles sont bijectives. On a de plus \mbox{$\big(f \circ g^{-1}\big)^{-1} = g \circ f^{-1}$}. Ainsi, en appliquant la relation précédente à la fonction $f \circ g^{-1}$, on obtient
\begin{align*}
\int_{g(a)}^{g(b)} \big(f \circ g^{-1}\big)(u) \d u + \int_{(f\circ g^{-1})(g(a))}^{(f\circ g^{-1})(g(b))} \big(f \circ g^{-1}\big)^{-1}(u) \d u &= \begin{multlined}[t] 
g(b) \big(f \circ g^{-1}\big)\big(g(b)\big) \\
- g(a) \big(f \circ g^{-1}\big)\big(g(a)\big)
\end{multlined} \\
\int_{g(a)}^{g(b)} \big(f \circ g^{-1}\big)(u) \d u + \int_{f(a)}^{f(b)} \big(g \circ f^{-1}\big)(u) \d u
&= g(b) f(b) - g(a) f(a).
\end{align*}

Ainsi, en effectuant le changement de variable $t \mapsto g(t)$ (resp. $t \mapsto f(t)$) dans la première (resp. seconde) intégrale, on retrouve la formule d'intégration par parties :
\[
\int_a^b f(t) g'(t) \,\mathrm{d}t + \int_a^b g(t) f'(t) \,\mathrm{d}t = f(b) g(b) - f(a) g(a).
\]
\end{remarque}
\end{comment}
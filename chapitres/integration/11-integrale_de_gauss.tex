\section{Intégrale de \textsc{Gauss}}

\begin{prop}{}
    $$\int_{0}^{+\infty} \e^{-x^2} \d x = \frac{\sqrt{\pi}}{2}$$
\end{prop}

\begin{exercice}
    \marginnote[0cm]{Source : \cite{maths-france} Planche no 13. Suites et séries d’intégrales}
    \begin{enumerate}
        \item \textbf{Première méthode:} \say{ à la main }. \\ 
        Pour $n \in \Ne$, on pose
        $$
        f_n(x) \defeq
        \begin{cases}
            \left(1 - \frac{x}{n} \right)^n &\text{si } x \in [0, n] \\
            0 &\text{si } x \geqslant n
        \end{cases}.
        $$
        Pour tout réel positif $x$, on pose $f(x) = \e^{-x^2}$.
        \begin{enumerate}
            \item Montrer que pour tout réel positif $x$, 
            $$|f(x) - f_n(x)| \leqslant \frac{1}{n \e}.$$
            \item À l'aide de la suite $(f_n)_{n \in \Ne}$, calculer l'intégrale de \textsc{Gauss}.
        \end{enumerate}
        \item \textbf{Deuxième méthode:} \say{ avec le théorème de convergence dominée }. \\
        Pour $n \in \Ne$, on pose
        $$
        f_n(x) \defeq
        \begin{cases}
            \left(1 - \frac{x^2}{n} \right)^n &\text{si } x \in [0, \sqrt{n}] \\
            0 &\text{si } x > \sqrt{n}
        \end{cases}.
        $$
        \begin{enumerate}
            \item Montrer que la suite $(f_n)_{n \in \Ne}$ converge simplement sur $\Rp$ vers la fonction $f:x \mapsto \e^{-x^2}$.
            \item À l'aide de la convergence dominée, calculer l'intégrale de \textsc{Gauss}.
        \end{enumerate}
    \end{enumerate}
\end{exercice}

\begin{solution}
\end{solution}

\subsection{Calcul de l'intégrale de \textsc{Gauss} avec celle de \textsc{Wallis}}
\marginnote[0cm]{Source : \href{https://fr.wikipedia.org/wiki/Intégrale_de_Wallis}{Intégrale de \textsc{Wallis} -- \textsf{wikipedia.org}}}
On peut aisément utiliser les intégrales de \textsc{Wallis} pour calculer l'intégrale de \text{Gauss}. \\
On utilise pour cela l'encadrement suivant, issu de la construction de la fonction exponentielle par la méthode d'\textsc{Euler}: pour tout entier $n > 0$ et tout réel $u \in ]-n, n[$, 
$$\left(1 + \frac{u}{n} \right)^n \leqslant \e^u \leqslant \left( 1 - \frac{u}{n} \right)^{-n}.$$
Posant alors $u = -x^2$, on obtient:
$$\int_0^{\sqrt{n}} \left( 1 - \frac{x^2}{n} \right)^n \d x \leqslant \int_0^{\sqrt{n}} \e^{-x^2} \d x \leqslant \int_0^{\sqrt{n}} \left( 1 + \frac{x^2}{n} \right)^{-n} \d x.$$
Or les intégrales d'encadrement sont liées aux intégrales de \textsc{Wallis}. Pour celle de gauche, il suffit de poser $x = \sqrt{n} \sin t$ ($t$ variant de $0$ à $\pi/2$). Quant à celle de droite, on peut poser $x = \sqrt{n} \tan t$ ($t$ variant de $0$ à $\pi/4$) puis majorer par l'intégrale de $0$ à $\pi/2$. On obtient ainsi:
$$\sqrt{n} \Wallis_{2n+1} \leqslant \int_0^{\sqrt{n}} \e^{-x^2} \d x \leqslant \sqrt{n} \Wallis_{2n-2}.$$
Par le théorème des gendarmes, on déduit alors de l'équivalent de $\Wallis_n$ que
$$\int_0^{+ \infty} \e^{-x^2} \d x = \frac{\sqrt{\pi}}{2}.$$

\todoinline{Pour avoir une application, on peut regarder la remarque dans le chapitre sur les polynômes d'Hermite ou alors calculer la transformée de Fourier d'une gaussienne.\\
Dans le document bestiaire.pdf, on peut trouver :\\
* une preuve par étude d'intégrale à paramètre en page 178\\
* une preuve avec Wallis en p. 223 mais je l'ai déjà quelque part rédigée}

\todoarmand{
* Je ne vois pas de quelle remarque vous parlez \\
* On peut calculer la transformée de Fourier d'une gaussienne, ça pourrait aussi être l'occasion de dire un mot sur ce qu'est la TF. \\
* J'aime bien la preuve avec Wallis, il faudrait alors mettre cette section après celle sur Wallis. Il faut voir comment l'intégrer (c'est le cas de le dire) avec les autres exercices sur Wallis et Stirling pour ne pas trop se répéter.
}
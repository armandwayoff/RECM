\section{Intégrale de \textsc{Gauss}}

\begin{prop}{}
    $$\int_{0}^{+\infty} \e^{-x^2} \d x = \frac{\sqrt{\pi}}{2}$$
\end{prop}

\begin{exercice}
    \marginnote[0cm]{Source : \cite{maths-france} Planche no 13. Suites et séries d’intégrales}
    \begin{enumerate}
        \item \textbf{Première méthode:} \say{ à la main }. \\ 
        Pour $n \in \Ne$, on pose
        $$
        f_n(x) \defeq
        \begin{cases}
            \left(1 - \frac{x}{n} \right)^n &\text{si } x \in [0, n] \\
            0 &\text{si } x \geqslant n
        \end{cases}.
        $$
        Pour tout réel positif $x$, on pose $f(x) = \e^{-x^2}$.
        \begin{enumerate}
            \item Montrer que pour tout réel positif $x$, 
            $$|f(x) - f_n(x)| \leqslant \frac{1}{n \e}.$$
            \item À l'aide de la suite $(f_n)_{n \in \Ne}$, calculer l'intégrale de \textsc{Gauss}.
        \end{enumerate}
        \item \textbf{Deuxième méthode:} \say{ avec le théorème de convergence dominée }. \\
        Pour $n \in \Ne$, on pose
        $$
        f_n(x) \defeq
        \begin{cases}
            \left(1 - \frac{x^2}{n} \right)^n &\text{si } x \in [0, \sqrt{n}] \\
            0 &\text{si } x > \sqrt{n}
        \end{cases}.
        $$
        \begin{enumerate}
            \item Montrer que la suite $(f_n)_{n \in \Ne}$ converge simplement sur $\Rp$ vers la fonction $f:x \mapsto \e^{-x^2}$.
            \item À l'aide de la convergence dominée, calculer l'intégrale de \textsc{Gauss}.
        \end{enumerate}
    \end{enumerate}
\end{exercice}

\begin{preuve}
\end{preuve}

\subsection{Calcul de l'intégrale de \textsc{Gauss} avec celle de \textsc{Wallis}}
\marginnote[0cm]{Source : \href{https://fr.wikipedia.org/wiki/Intégrale_de_Wallis}{Intégrale de \textsc{Wallis} -- \textsf{wikipedia.org}}}
On peut aisément utiliser les intégrales de \textsc{Wallis} pour calculer l'intégrale de \text{Gauss}. \\
On utilise pour cela l'encadrement suivant, issu de la construction de la fonction exponentielle par la méthode d'\textsc{Euler}: pour tout entier $n > 0$ et tout réel $u \in ]-n, n[$, 
$$\left(1 + \frac{u}{n} \right)^n \leqslant \e^u \leqslant \left( 1 - \frac{u}{n} \right)^{-n}.$$
Posant alors $u = -x^2$, on obtient:
$$\int_0^{\sqrt{n}} \left( 1 - \frac{x^2}{n} \right)^n \d x \leqslant \int_0^{\sqrt{n}} \e^{-x^2} \d x \leqslant \int_0^{\sqrt{n}} \left( 1 + \frac{x^2}{n} \right)^{-n} \d x.$$
Or les intégrales d'encadrement sont liées aux intégrales de \textsc{Wallis}. Pour celle de gauche, il suffit de poser $x = \sqrt{n} \sin t$ ($t$ variant de $0$ à $\pi/2$). Quant à celle de droite, on peut poser $x = \sqrt{n} \tan t$ ($t$ variant de $0$ à $\pi/4$) puis majorer par l'intégrale de $0$ à $\pi/2$. On obtient ainsi:
$$\sqrt{n} \Wallis_{2n+1} \leqslant \int_0^{\sqrt{n}} \e^{-x^2} \d x \leqslant \sqrt{n} \Wallis_{2n-2}.$$
Par le théorème des gendarmes, on déduit alors de l'équivalent de $\Wallis_n$ que
$$\int_0^{+ \infty} \e^{-x^2} \d x = \frac{\sqrt{\pi}}{2}.$$



%---------------

\begin{exercice}
\begin{itemize}
\item Montrer que
\[
\int_0^{\sqrt{n}} \left(1 - \frac{t^2}{n}\right)^n \d t \leq \int_0^{\sqrt{n}} \e^{-t^2} \d t \leq \int_0^{+\infty} \frac{\d t}{\left(1 + \frac{t^2}{n}\right)^n}.
\]

\item En déduire que $\int_0^{\sqrt{n}} \e^{-t^2} \d t \sim \sqrt{n} \int_0^{\pi/2} \cos^{2n+1}(\theta) \d tg$.
{En utilisant les intégrales de {Wallis}, on montre que $\int_0^{+\infty} \e^{-t^2} \d t = \frac{\sqrt{\pi}}{2}$.}
\end{itemize}
\end{exercice}

\begin{preuve}
\begin{itemize}
\item Rappelons que $\ln(1 + x) \leq x$. Ainsi, $\left(1 - \frac{t^2}{n}\right)^n \leq \e^{-t^2}$. De même, $-\ln(1+t^2/n) \geq -t^2/n$ et $\e^{-\frac{t^2}{n}} \geq \left(1 + \frac{t^2}{n}\right)^{-n}$.

On obtient ainsi le résultat en intégrant entre $0$ et $\sqrt{n}$. De plus, $\left(1 + \frac{t^2}{n}\right)^{-n} = O(1/t^2)$ donc l'intégrale est convergente.

\item On pose $\phi : t \mapsto \sqrt{n} \sin(t)$ dans la première intégrale et $\psi : t \mapsto \sqrt{n} \tan(t)$ dans la seconde. On obtient ainsi l'encadrement
\begin{align*}
\sqrt{n} \int_0^{\pi/2} \cos^{2n+1}(t) \d t \leq \int_0^{\sqrt{n}} \e^{-t^2} \d t &\leq \sqrt{n} \int_0^{\pi/2} \cos^{2n-2}(t) \d t\\
& \leq \sqrt{n} \int_0^{\pi/2} \cos^{2n-3}(t) \d t.
\end{align*}
En notant $I_n = \int_0^{\pi/2} \cos^{2n+1}(t) \d t$, comme la suite $(I_n)$ est décroissante. De plus, en utilisant une intégration par parties, on obtient une relation de récurrence puis $I_{n+1} \sim I_n$.

{On obtiendra ce résultat plus simplement en utilisant le théorème de convergence dominée.}
\end{itemize}
\end{preuve}



%---------------

\begin{exercice}
{X-ENS}
{16}%
Soient $f$ et $g$ les fonctions définies pour tout $x \in \R_+$ par $f(x) = \int_0^1 \frac{e^{-(t^2+1) x^2}}{1 + t^2} \d t$ et $g(x) = \int_0^x e^{-t^2} \d t$.
\begin{itemize}
\item Calculer $f(0)$ puis $\lim_{x\to+\infty} f(x)$.

\item Montrer que $f$ est de classe $\mathscr{C}^1$ sur $\R_+$ et que, pour tout $x \in \R_+$, $-2 g'(x) g(x) = f'(x)$.

\item En déduire $I = \int_0^{+\infty} e^{-t^2} \d t$.

Soit $h$ une fonction continue par morceaux, décroissante sur $\R_+$ telle que $\int_0^{+\infty} h(t) \d t$ soit convergente et non nulle.

\item Montrer que $h$ est à valeurs positives.

Pour tout réel positif $t$ non nul, on pose $S(t) = \sum_{n=1}^{+\infty} h(n t)$.
\item Montrer que $S$ existe.

\item Déterminer un équivalent de $S(t)$ lorsque $t$ tend vers $0^+$.

\item Déterminer un équivalent de $\sum_{n=1}^{+\infty} x^{n^2}$ lors que $x$ tend vers $1^-$.
\end{itemize}
\end{exercice}

\begin{preuve}
\begin{itemize}
\item D'après la définition, $f(0) = \int_0^1 \frac{1}{1 + t^2} \d t = \arctan(1) = \frac{\pi}{4}$.

On remarque que l'intégrande est majorée par $t \mapsto \frac{1}{1+ t^2}$ qui est intégrable donc en appliquant le théorème de convergence dominée,
\[
\lim_{x\to+\infty} f(x) = 0.
\]

\item Sur $[a, b]$, on majore la dérivée par $b^2$ qui est intégrable sur $[0, 1]$. Ainsi, d'après le théorème de dérivation sous le signe intégral,
\begin{align*}
f'(x) &= - \int_0^1 2 x e^{-(t^2+1) x^2} \d t \\
&= - 2 x e^{-x^2} \int_0^1 e^{-(t x)^2} \d t \\
&= - 2 x e^{-x^2} \int_0^x e^{-t^2} \frac{\d t}{x} \\
&= - 2 g'(x) g(x).
\end{align*}

\item D'après la question précédente,
\[
f(x) - f(0) = - (g(x)^2 - g(0)^2).
\]
Ainsi,
\[
\frac{\pi}{4} - f(x) = g(x)^2.
\]
La fonction $g$ étant à valeurs positives, $I = \frac{\sqrt{\pi}}{2}$.

\item $h$ est décroissante sur $\R_+$. Elle admet donc une limite en $+\infty$. Si cette limite (dans $\bar{\R}$) est égale à $\ell < 0$, alors $h(t) \leq \frac{\ell}{2}$ pour $t$ assez grand et $\int_0^{+\infty} h(t) \d t$ diverge. On raisonne de même pour $\ell > 0$. Ainsi, $h$ tend vers $0$ en $+\infty$ et $h$ est à valeurs positives.

\medskip

{2ème méthode (si $h$ est continue).} En utilisant le théorème des accroissements finis, $H(n+1) - H(n) = h(c_n)$ et le membre de gauche tend vers $0$ donc $\ell = 0$.

\item D'après la décroissance de $h$,
\begin{align*}
\sum_{n=1}^N h(n t) &= \sum_{n=1}^N (n t - (n-1)t) \frac{1}{t} h(n t) \\
&\leq \frac{1}{t} \sum_{n=1}^N \int_{(n-1)t}^{nt} h(u) \d u \\
&\leq \frac{1}{t} \int_0^{Nt} h(u) \d u.
\end{align*}
Ainsi, comme $\int_0^{+\infty} h(t) \d t$ converge, alors d'après les théorèmes sur les séries à termes positifs, $S$ converge.

\item D'après la question précédente, en utilisant une minoration,
\[
J - c t \leq t S(t) \leq J.
\]
Ainsi, comme $J \neq 0$, alors
\[
S(t) \sim_{t\to0} \frac{J}{t}.
\]

\item En posant $h(t) = e^{-t^2}$, la fonction $h$ est bien continue, décroissante et d'intégrale sur $\R_+$ convergente. Ainsi, d'après la question précédente, pour $x \in ]0, 1[$,
\[
S(\sqrt{-\ln(x)}) = \sum_{n=1}^{+\infty} x^{n^2}.
\]
Ainsi, d'après la question précédente,
\[
\sum_{n=1}^{+\infty} x^{n^2} \sim_1 \frac{\sqrt{\pi}}{2 \sqrt{\abs{\ln(x)}}}.
\]
\end{itemize}
\end{preuve}

\todoinline{Pour avoir une application, on peut regarder la remarque dans le chapitre sur les polynômes d'Hermite ou alors calculer la transformée de Fourier d'une gaussienne.\\
Dans le document bestiaire.pdf, on peut trouver :\\
* une preuve par étude d'intégrale à paramètre en page 178\\
* une preuve avec Wallis en p. 223 mais je l'ai déjà quelque part rédigée}

\todoarmand{
* Je ne vois pas de quelle remarque vous parlez \\
* On peut calculer la transformée de Fourier d'une gaussienne, ça pourrait aussi être l'occasion de dire un mot sur ce qu'est la TF. \\
* J'aime bien la preuve avec Wallis, il faudrait alors mettre cette section après celle sur Wallis. Il faut voir comment l'intégrer (c'est le cas de le dire) avec les autres exercices sur Wallis et Stirling pour ne pas trop se répéter.
}



%---------------

\begin{exercice}
Polynômes d'{Hermite}
{RMS 2017 154 - Autres écoles}
{TPE}
{16}
Soit $f : x \mapsto \e^{-x^2}$. On rappelle que $\int_{-\infty}^{+\infty} f(x) \d x = \sqrt{\pi}$.
\begin{itemize}
\item Montrer qu'il existe un polynôme $P_n$ tel que $f^{(n)}(x) = f(x) P_n(x)$. Préciser le degré, la parité et le coefficient dominant de $P_n$.

\item Montrer l'existence puis calculer $\int_{-\infty}^{+\infty} f(x) P_n(x) P_m(x) \d x$.
\end{itemize}
\end{exercice}

\begin{preuve}
\begin{itemize}
\item On raisonne par récurrence en remarquant que $P_0 = 1$ et $P_{n+1} = P_n' - 2 X P_n$. Ainsi, le degré de $P_n$ est $n$ et son coefficient dominant $(-2)^n$ et $P_n$ a même parité que $n$, i.e. $P_n(-X) = (-1)^n P_n(X)$.

\item Soient $0 < m \leq n$. Les intégrales sont bien définies car ce sont des $o(1/x^2)$ en $\pm\infty$. Comme les crochets tendent vers $0$, en utilisant les fonctions $t \mapsto f^{(n-1)}(t)$ et $t \mapsto P_m(t)$ qui sont de classe $\mathscr{C}^1$, on remarque que
\[
\int_\R f P_n f P_m = \int_\R f^{(n)} P_m = -\int_\R f^{(n-1)} P_m'.
\]
Ainsi, en itérant ce procédé, comme $m \leq n$,
\[
\int_\R f P_n P_m = (-1)^m \int_\R f^{(n-m)} P_m^{(m)}
= (-1)^m (-2)^m \int_\R f^{(n-m)}.
\]
Ainsi,
\begin{itemize}
\item Si $n - m = 0$, i.e. $m = n$, alors $\int_\R f P_n^2 = m! 2^m \sqrt{\pi}$.
\item Si $n - m \neq 0$, alors $\int_\R f P_n P_m = \left[f^{(n-m-1)}\right]_{-\infty}^{+\infty} = 0$.
\end{itemize}
\end{itemize}
\end{preuve}



%---------------

\begin{exercice}

Transformée de~{Fourier} d'une gaussienne

On pose $f(x) = \int_{-\infty}^{+\infty} \e^{-t^2 + i t x} \d t$.
\begin{itemize}
\item Montrer que la fonction $f$  est définie sur $\R$.
\item Montrer que $f$ est dérivable sur $\R$ et que $f'(x) = \int_{-\infty}^{+\infty} i t \e^{-t^2 + i t x} \d t$.
\item Montrer que $f$ vérifie l'équation différentielle $2 y' + x y = 0$.
{On pourra utiliser une intégration par parties.}
\item En déduire la valeur de $f$ connaissant $f(0) = \sqrt{\pi}$.
\end{itemize}
\end{exercice}

\begin{preuve}
\begin{itemize}
\item $\abs{\phi(x, t)} \leq \e^{-t^2}$ est une fonction intégrable.

\item $\abs{\frac{\partial \phi}{\partial x}(x, t)} \leq t \e^{-t^2}$ est intégrable sur $\R$ et $f'(x) = \int_\R i t \e^{-t^2 + i t x} \d t$.

\item En utilisant une intégration par parties généralisée,
\begin{align*}
f'(x) &= \int_\R t \e^{-t^2} i \e^{i t x} \d t \\
&= \left[-\frac{1}{2} \e^{-t^2} i \e^{i t x}\right]_{-\infty}^{+\infty} - \int_\R \frac{x}{2} \e^{-t^2} \e^{i t x} \d t \\
f'(x) &= -\frac{x}{2} f(x).
\end{align*}

\item $f(x) = \lg \e^{-x^2/4} = \sqrt{\pi} \e^{-x^2/4}$.
\end{itemize}
\end{preuve}
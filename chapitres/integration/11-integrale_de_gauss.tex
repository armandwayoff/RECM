\section{Intégrale de \textsc{Gauss}}

\begin{marginfigure}[0cm]
    \centering
    % Author: Izaak Neutelings (August, 2017)


\tikzset{>=latex} % for LaTeX arrow head
\contourlength{1.2pt}
\usetikzlibrary{positioning,calc}
\usetikzlibrary{backgrounds}% required for 'inner frame sep'
%\usepackage{adjustbox} % add whitespace (trim)

% define gaussian pdf and cdf
\pgfmathdeclarefunction{gauss}{3}{%
  \pgfmathparse{1/(#3*sqrt(2*pi))*exp(-((#1-#2)^2)/(2*#3^2))}%
}
\colorlet{mydarkblue}{blue!30!black}

% to fill an area under function
\usepgfplotslibrary{fillbetween}
\usetikzlibrary{patterns}
\pgfplotsset{compat=1.12} % TikZ coordinates <-> axes coordinates
% https://tex.stackexchange.com/questions/240642/add-vertical-line-of-equation-x-2-and-shade-a-region-in-graph-by-pgfplots

% plot aspect ratio
%\def\axisdefaultwidth{8cm}
%\def\axisdefaultheight{6cm}

% number of sample points
\def\N{50}

\begin{tikzpicture}[scale=1]
  \message{Cumulative probability^^J}
  
  \def\B{0};
  \def\Bs{6.0};
  \def\xmax{\B+3.5*\Bs};
  \def\ymin{{-0.1*gauss(\B,\B,\Bs)}};
  \def\h{0.07*gauss(\B,\B,\Bs)};
  \def\a{\B-0.8*\Bs};
  
  \begin{axis}[every axis plot post/.append style={
               mark=none,
               domain={-1*(\xmax)}:{1*(\xmax)},samples=\N,smooth},
               % xmin={-1*(\xmax)}, 
               xmax={1.06*(\xmax)},
               ymin=\ymin, ymax={1.3*gauss(\B,\B,\Bs)},
               axis lines=middle,
               axis line style=thick,
               axis line style={-latex},
               ticks=none,
               xlabel=$x$,
               every axis x label/.style={at={(current axis.right of origin)},anchor=north},
               y=700pt,
               clip=false
              ]
    
    % PLOTS
    \addplot[blue,thick,name path=B] {gauss(x,\B,\Bs)};
    % FILL
    \path[name path=xaxis]
      (0,0) -- (\pgfkeysvalueof{/pgfplots/xmax},0);
    \addplot[blue!25, opacity=0.9] fill between[of=xaxis and B];
    % LINES
    \node[blue,above left] at ({-0.7*(\B+\Bs)},{1.2*gauss(\B+\Bs,\B,\Bs)}) {$x \mapsto \mathrm{e}^{-x^2}$};
    \node[blue!60!black] at ({0},{0.6*gauss(0.85*(\a),\B,\Bs)}) {$\sqrt{\pi}$};
    
  \end{axis}
\end{tikzpicture}
    \caption{Intégrale de \textsc{Gauss}}
\end{marginfigure}

\begin{theo}{}
On montre que
\[
\int_0^{+\infty} \e^{-x^2} \d x = \lim_{n\to+\infty} \int_0^{\sqrt{n}} \left(1 - \frac{x^2}{n}\right)^n \d x.
\]

Ainsi,
\begin{equation}\label{eqIntGauss}
        \int_{0}^{+\infty} \e^{-x^2} \d x = \frac{\sqrt{\pi}}{2}        
    \end{equation}
\end{theo}

%-----------
\subsection{Calcul de l'intégrale}

\begin{marginfigure}[5cm]
    \centering
    %% Creator: Matplotlib, PGF backend
%%
%% To include the figure in your LaTeX document, write
%%   \input{<filename>.pgf}
%%
%% Make sure the required packages are loaded in your preamble
%%   \usepackage{pgf}
%%
%% Also ensure that all the required font packages are loaded; for instance,
%% the lmodern package is sometimes necessary when using math font.
%%   \usepackage{lmodern}
%%
%% Figures using additional raster images can only be included by \input if
%% they are in the same directory as the main LaTeX file. For loading figures
%% from other directories you can use the `import` package
%%   \usepackage{import}
%%
%% and then include the figures with
%%   \import{<path to file>}{<filename>.pgf}
%%
%% Matplotlib used the following preamble
%%   
%%   \usepackage{fontspec}
%%   \setmainfont{DejaVuSerif.ttf}[Path=\detokenize{/home/wayoff/.pyenv/versions/3.8.10/lib/python3.8/site-packages/matplotlib/mpl-data/fonts/ttf/}]
%%   \setsansfont{DejaVuSans.ttf}[Path=\detokenize{/home/wayoff/.pyenv/versions/3.8.10/lib/python3.8/site-packages/matplotlib/mpl-data/fonts/ttf/}]
%%   \setmonofont{DejaVuSansMono.ttf}[Path=\detokenize{/home/wayoff/.pyenv/versions/3.8.10/lib/python3.8/site-packages/matplotlib/mpl-data/fonts/ttf/}]
%%   \makeatletter\@ifpackageloaded{underscore}{}{\usepackage[strings]{underscore}}\makeatother
%%
\begingroup%
\makeatletter%
\begin{pgfpicture}%
\pgfpathrectangle{\pgfpointorigin}{\pgfqpoint{3.000000in}{2.500000in}}%
\pgfusepath{use as bounding box, clip}%
\begin{pgfscope}%
\pgfsetbuttcap%
\pgfsetmiterjoin%
\definecolor{currentfill}{rgb}{1.000000,1.000000,1.000000}%
\pgfsetfillcolor{currentfill}%
\pgfsetlinewidth{0.000000pt}%
\definecolor{currentstroke}{rgb}{1.000000,1.000000,1.000000}%
\pgfsetstrokecolor{currentstroke}%
\pgfsetdash{}{0pt}%
\pgfpathmoveto{\pgfqpoint{0.000000in}{0.000000in}}%
\pgfpathlineto{\pgfqpoint{3.000000in}{0.000000in}}%
\pgfpathlineto{\pgfqpoint{3.000000in}{2.500000in}}%
\pgfpathlineto{\pgfqpoint{0.000000in}{2.500000in}}%
\pgfpathlineto{\pgfqpoint{0.000000in}{0.000000in}}%
\pgfpathclose%
\pgfusepath{fill}%
\end{pgfscope}%
\begin{pgfscope}%
\pgfsetbuttcap%
\pgfsetmiterjoin%
\definecolor{currentfill}{rgb}{1.000000,1.000000,1.000000}%
\pgfsetfillcolor{currentfill}%
\pgfsetlinewidth{0.000000pt}%
\definecolor{currentstroke}{rgb}{0.000000,0.000000,0.000000}%
\pgfsetstrokecolor{currentstroke}%
\pgfsetstrokeopacity{0.000000}%
\pgfsetdash{}{0pt}%
\pgfpathmoveto{\pgfqpoint{0.316667in}{0.571603in}}%
\pgfpathlineto{\pgfqpoint{2.850000in}{0.571603in}}%
\pgfpathlineto{\pgfqpoint{2.850000in}{2.350000in}}%
\pgfpathlineto{\pgfqpoint{0.316667in}{2.350000in}}%
\pgfpathlineto{\pgfqpoint{0.316667in}{0.571603in}}%
\pgfpathclose%
\pgfusepath{fill}%
\end{pgfscope}%
\begin{pgfscope}%
\pgfpathrectangle{\pgfqpoint{0.316667in}{0.571603in}}{\pgfqpoint{2.533333in}{1.778397in}}%
\pgfusepath{clip}%
\pgfsetrectcap%
\pgfsetroundjoin%
\pgfsetlinewidth{0.803000pt}%
\definecolor{currentstroke}{rgb}{0.690196,0.690196,0.690196}%
\pgfsetstrokecolor{currentstroke}%
\pgfsetdash{}{0pt}%
\pgfpathmoveto{\pgfqpoint{0.431818in}{0.571603in}}%
\pgfpathlineto{\pgfqpoint{0.431818in}{2.350000in}}%
\pgfusepath{stroke}%
\end{pgfscope}%
\begin{pgfscope}%
\pgfsetbuttcap%
\pgfsetroundjoin%
\definecolor{currentfill}{rgb}{0.000000,0.000000,0.000000}%
\pgfsetfillcolor{currentfill}%
\pgfsetlinewidth{0.803000pt}%
\definecolor{currentstroke}{rgb}{0.000000,0.000000,0.000000}%
\pgfsetstrokecolor{currentstroke}%
\pgfsetdash{}{0pt}%
\pgfsys@defobject{currentmarker}{\pgfqpoint{0.000000in}{-0.048611in}}{\pgfqpoint{0.000000in}{0.000000in}}{%
\pgfpathmoveto{\pgfqpoint{0.000000in}{0.000000in}}%
\pgfpathlineto{\pgfqpoint{0.000000in}{-0.048611in}}%
\pgfusepath{stroke,fill}%
}%
\begin{pgfscope}%
\pgfsys@transformshift{0.431818in}{0.571603in}%
\pgfsys@useobject{currentmarker}{}%
\end{pgfscope}%
\end{pgfscope}%
\begin{pgfscope}%
\definecolor{textcolor}{rgb}{0.000000,0.000000,0.000000}%
\pgfsetstrokecolor{textcolor}%
\pgfsetfillcolor{textcolor}%
\pgftext[x=0.431818in,y=0.474381in,,top]{\color{textcolor}\sffamily\fontsize{10.000000}{12.000000}\selectfont \(\displaystyle 0\)}%
\end{pgfscope}%
\begin{pgfscope}%
\definecolor{textcolor}{rgb}{0.000000,0.000000,0.000000}%
\pgfsetstrokecolor{textcolor}%
\pgfsetfillcolor{textcolor}%
\pgftext[x=1.583333in,y=0.284413in,,top]{\color{textcolor}\sffamily\fontsize{10.000000}{12.000000}\selectfont \(\displaystyle t\)}%
\end{pgfscope}%
\begin{pgfscope}%
\pgfpathrectangle{\pgfqpoint{0.316667in}{0.571603in}}{\pgfqpoint{2.533333in}{1.778397in}}%
\pgfusepath{clip}%
\pgfsetrectcap%
\pgfsetroundjoin%
\pgfsetlinewidth{0.803000pt}%
\definecolor{currentstroke}{rgb}{0.690196,0.690196,0.690196}%
\pgfsetstrokecolor{currentstroke}%
\pgfsetdash{}{0pt}%
\pgfpathmoveto{\pgfqpoint{0.316667in}{0.571603in}}%
\pgfpathlineto{\pgfqpoint{2.850000in}{0.571603in}}%
\pgfusepath{stroke}%
\end{pgfscope}%
\begin{pgfscope}%
\pgfsetbuttcap%
\pgfsetroundjoin%
\definecolor{currentfill}{rgb}{0.000000,0.000000,0.000000}%
\pgfsetfillcolor{currentfill}%
\pgfsetlinewidth{0.803000pt}%
\definecolor{currentstroke}{rgb}{0.000000,0.000000,0.000000}%
\pgfsetstrokecolor{currentstroke}%
\pgfsetdash{}{0pt}%
\pgfsys@defobject{currentmarker}{\pgfqpoint{-0.048611in}{0.000000in}}{\pgfqpoint{-0.000000in}{0.000000in}}{%
\pgfpathmoveto{\pgfqpoint{-0.000000in}{0.000000in}}%
\pgfpathlineto{\pgfqpoint{-0.048611in}{0.000000in}}%
\pgfusepath{stroke,fill}%
}%
\begin{pgfscope}%
\pgfsys@transformshift{0.316667in}{0.571603in}%
\pgfsys@useobject{currentmarker}{}%
\end{pgfscope}%
\end{pgfscope}%
\begin{pgfscope}%
\definecolor{textcolor}{rgb}{0.000000,0.000000,0.000000}%
\pgfsetstrokecolor{textcolor}%
\pgfsetfillcolor{textcolor}%
\pgftext[x=0.150000in, y=0.518842in, left, base]{\color{textcolor}\sffamily\fontsize{10.000000}{12.000000}\selectfont \(\displaystyle 0\)}%
\end{pgfscope}%
\begin{pgfscope}%
\pgfpathrectangle{\pgfqpoint{0.316667in}{0.571603in}}{\pgfqpoint{2.533333in}{1.778397in}}%
\pgfusepath{clip}%
\pgfsetrectcap%
\pgfsetroundjoin%
\pgfsetlinewidth{0.803000pt}%
\definecolor{currentstroke}{rgb}{0.690196,0.690196,0.690196}%
\pgfsetstrokecolor{currentstroke}%
\pgfsetdash{}{0pt}%
\pgfpathmoveto{\pgfqpoint{0.316667in}{2.265314in}}%
\pgfpathlineto{\pgfqpoint{2.850000in}{2.265314in}}%
\pgfusepath{stroke}%
\end{pgfscope}%
\begin{pgfscope}%
\pgfsetbuttcap%
\pgfsetroundjoin%
\definecolor{currentfill}{rgb}{0.000000,0.000000,0.000000}%
\pgfsetfillcolor{currentfill}%
\pgfsetlinewidth{0.803000pt}%
\definecolor{currentstroke}{rgb}{0.000000,0.000000,0.000000}%
\pgfsetstrokecolor{currentstroke}%
\pgfsetdash{}{0pt}%
\pgfsys@defobject{currentmarker}{\pgfqpoint{-0.048611in}{0.000000in}}{\pgfqpoint{-0.000000in}{0.000000in}}{%
\pgfpathmoveto{\pgfqpoint{-0.000000in}{0.000000in}}%
\pgfpathlineto{\pgfqpoint{-0.048611in}{0.000000in}}%
\pgfusepath{stroke,fill}%
}%
\begin{pgfscope}%
\pgfsys@transformshift{0.316667in}{2.265314in}%
\pgfsys@useobject{currentmarker}{}%
\end{pgfscope}%
\end{pgfscope}%
\begin{pgfscope}%
\definecolor{textcolor}{rgb}{0.000000,0.000000,0.000000}%
\pgfsetstrokecolor{textcolor}%
\pgfsetfillcolor{textcolor}%
\pgftext[x=0.150000in, y=2.212553in, left, base]{\color{textcolor}\sffamily\fontsize{10.000000}{12.000000}\selectfont \(\displaystyle 1\)}%
\end{pgfscope}%
\begin{pgfscope}%
\pgfpathrectangle{\pgfqpoint{0.316667in}{0.571603in}}{\pgfqpoint{2.533333in}{1.778397in}}%
\pgfusepath{clip}%
\pgfsetrectcap%
\pgfsetroundjoin%
\pgfsetlinewidth{1.505625pt}%
\definecolor{currentstroke}{rgb}{0.579608,0.770196,0.873725}%
\pgfsetstrokecolor{currentstroke}%
\pgfsetdash{}{0pt}%
\pgfpathmoveto{\pgfqpoint{0.431818in}{2.265314in}}%
\pgfpathlineto{\pgfqpoint{0.454872in}{2.264254in}}%
\pgfpathlineto{\pgfqpoint{0.477925in}{2.261072in}}%
\pgfpathlineto{\pgfqpoint{0.500979in}{2.255768in}}%
\pgfpathlineto{\pgfqpoint{0.524032in}{2.248343in}}%
\pgfpathlineto{\pgfqpoint{0.549391in}{2.237726in}}%
\pgfpathlineto{\pgfqpoint{0.574749in}{2.224542in}}%
\pgfpathlineto{\pgfqpoint{0.600108in}{2.208790in}}%
\pgfpathlineto{\pgfqpoint{0.625467in}{2.190472in}}%
\pgfpathlineto{\pgfqpoint{0.653131in}{2.167561in}}%
\pgfpathlineto{\pgfqpoint{0.680795in}{2.141596in}}%
\pgfpathlineto{\pgfqpoint{0.708459in}{2.112575in}}%
\pgfpathlineto{\pgfqpoint{0.738428in}{2.077689in}}%
\pgfpathlineto{\pgfqpoint{0.768397in}{2.039218in}}%
\pgfpathlineto{\pgfqpoint{0.798367in}{1.997161in}}%
\pgfpathlineto{\pgfqpoint{0.830641in}{1.947861in}}%
\pgfpathlineto{\pgfqpoint{0.862916in}{1.894402in}}%
\pgfpathlineto{\pgfqpoint{0.895191in}{1.836785in}}%
\pgfpathlineto{\pgfqpoint{0.929771in}{1.770439in}}%
\pgfpathlineto{\pgfqpoint{0.964351in}{1.699320in}}%
\pgfpathlineto{\pgfqpoint{1.001236in}{1.618198in}}%
\pgfpathlineto{\pgfqpoint{1.038122in}{1.531646in}}%
\pgfpathlineto{\pgfqpoint{1.077312in}{1.433734in}}%
\pgfpathlineto{\pgfqpoint{1.116503in}{1.329691in}}%
\pgfpathlineto{\pgfqpoint{1.155694in}{1.219517in}}%
\pgfpathlineto{\pgfqpoint{1.197190in}{1.096180in}}%
\pgfpathlineto{\pgfqpoint{1.238686in}{0.965969in}}%
\pgfpathlineto{\pgfqpoint{1.282487in}{0.821069in}}%
\pgfpathlineto{\pgfqpoint{1.326289in}{0.668510in}}%
\pgfpathlineto{\pgfqpoint{1.353953in}{0.571603in}}%
\pgfpathlineto{\pgfqpoint{2.734848in}{0.571603in}}%
\pgfpathlineto{\pgfqpoint{2.734848in}{0.571603in}}%
\pgfusepath{stroke}%
\end{pgfscope}%
\begin{pgfscope}%
\pgfpathrectangle{\pgfqpoint{0.316667in}{0.571603in}}{\pgfqpoint{2.533333in}{1.778397in}}%
\pgfusepath{clip}%
\pgfsetrectcap%
\pgfsetroundjoin%
\pgfsetlinewidth{1.505625pt}%
\definecolor{currentstroke}{rgb}{0.290980,0.594510,0.789020}%
\pgfsetstrokecolor{currentstroke}%
\pgfsetdash{}{0pt}%
\pgfpathmoveto{\pgfqpoint{0.431818in}{2.265314in}}%
\pgfpathlineto{\pgfqpoint{0.454872in}{2.264254in}}%
\pgfpathlineto{\pgfqpoint{0.477925in}{2.261074in}}%
\pgfpathlineto{\pgfqpoint{0.500979in}{2.255782in}}%
\pgfpathlineto{\pgfqpoint{0.524032in}{2.248386in}}%
\pgfpathlineto{\pgfqpoint{0.549391in}{2.237838in}}%
\pgfpathlineto{\pgfqpoint{0.574749in}{2.224787in}}%
\pgfpathlineto{\pgfqpoint{0.600108in}{2.209262in}}%
\pgfpathlineto{\pgfqpoint{0.627772in}{2.189546in}}%
\pgfpathlineto{\pgfqpoint{0.655436in}{2.166984in}}%
\pgfpathlineto{\pgfqpoint{0.683100in}{2.141638in}}%
\pgfpathlineto{\pgfqpoint{0.713069in}{2.111120in}}%
\pgfpathlineto{\pgfqpoint{0.743039in}{2.077520in}}%
\pgfpathlineto{\pgfqpoint{0.775313in}{2.038016in}}%
\pgfpathlineto{\pgfqpoint{0.809893in}{1.992044in}}%
\pgfpathlineto{\pgfqpoint{0.846779in}{1.939084in}}%
\pgfpathlineto{\pgfqpoint{0.885970in}{1.878683in}}%
\pgfpathlineto{\pgfqpoint{0.927466in}{1.810494in}}%
\pgfpathlineto{\pgfqpoint{0.973572in}{1.730195in}}%
\pgfpathlineto{\pgfqpoint{1.024290in}{1.637185in}}%
\pgfpathlineto{\pgfqpoint{1.084228in}{1.522337in}}%
\pgfpathlineto{\pgfqpoint{1.167220in}{1.357911in}}%
\pgfpathlineto{\pgfqpoint{1.296319in}{1.102119in}}%
\pgfpathlineto{\pgfqpoint{1.351647in}{0.997576in}}%
\pgfpathlineto{\pgfqpoint{1.397754in}{0.914996in}}%
\pgfpathlineto{\pgfqpoint{1.436945in}{0.849085in}}%
\pgfpathlineto{\pgfqpoint{1.471525in}{0.794903in}}%
\pgfpathlineto{\pgfqpoint{1.503799in}{0.748245in}}%
\pgfpathlineto{\pgfqpoint{1.533769in}{0.708748in}}%
\pgfpathlineto{\pgfqpoint{1.561433in}{0.675928in}}%
\pgfpathlineto{\pgfqpoint{1.586791in}{0.649205in}}%
\pgfpathlineto{\pgfqpoint{1.609845in}{0.627929in}}%
\pgfpathlineto{\pgfqpoint{1.632898in}{0.609736in}}%
\pgfpathlineto{\pgfqpoint{1.653646in}{0.596168in}}%
\pgfpathlineto{\pgfqpoint{1.674394in}{0.585415in}}%
\pgfpathlineto{\pgfqpoint{1.692837in}{0.578347in}}%
\pgfpathlineto{\pgfqpoint{1.711280in}{0.573737in}}%
\pgfpathlineto{\pgfqpoint{1.727417in}{0.571809in}}%
\pgfpathlineto{\pgfqpoint{1.752776in}{0.571603in}}%
\pgfpathlineto{\pgfqpoint{2.734848in}{0.571603in}}%
\pgfpathlineto{\pgfqpoint{2.734848in}{0.571603in}}%
\pgfusepath{stroke}%
\end{pgfscope}%
\begin{pgfscope}%
\pgfpathrectangle{\pgfqpoint{0.316667in}{0.571603in}}{\pgfqpoint{2.533333in}{1.778397in}}%
\pgfusepath{clip}%
\pgfsetrectcap%
\pgfsetroundjoin%
\pgfsetlinewidth{1.505625pt}%
\definecolor{currentstroke}{rgb}{0.090196,0.392941,0.670588}%
\pgfsetstrokecolor{currentstroke}%
\pgfsetdash{}{0pt}%
\pgfpathmoveto{\pgfqpoint{0.431818in}{2.265314in}}%
\pgfpathlineto{\pgfqpoint{0.454872in}{2.264254in}}%
\pgfpathlineto{\pgfqpoint{0.477925in}{2.261075in}}%
\pgfpathlineto{\pgfqpoint{0.500979in}{2.255786in}}%
\pgfpathlineto{\pgfqpoint{0.526337in}{2.247547in}}%
\pgfpathlineto{\pgfqpoint{0.551696in}{2.236795in}}%
\pgfpathlineto{\pgfqpoint{0.577055in}{2.223564in}}%
\pgfpathlineto{\pgfqpoint{0.602413in}{2.207893in}}%
\pgfpathlineto{\pgfqpoint{0.630077in}{2.188071in}}%
\pgfpathlineto{\pgfqpoint{0.657741in}{2.165474in}}%
\pgfpathlineto{\pgfqpoint{0.687711in}{2.137959in}}%
\pgfpathlineto{\pgfqpoint{0.717680in}{2.107402in}}%
\pgfpathlineto{\pgfqpoint{0.749955in}{2.071241in}}%
\pgfpathlineto{\pgfqpoint{0.784535in}{2.028953in}}%
\pgfpathlineto{\pgfqpoint{0.821420in}{1.980074in}}%
\pgfpathlineto{\pgfqpoint{0.860611in}{1.924222in}}%
\pgfpathlineto{\pgfqpoint{0.902107in}{1.861135in}}%
\pgfpathlineto{\pgfqpoint{0.948214in}{1.786902in}}%
\pgfpathlineto{\pgfqpoint{1.001236in}{1.697114in}}%
\pgfpathlineto{\pgfqpoint{1.065786in}{1.583137in}}%
\pgfpathlineto{\pgfqpoint{1.176442in}{1.382217in}}%
\pgfpathlineto{\pgfqpoint{1.266350in}{1.220906in}}%
\pgfpathlineto{\pgfqpoint{1.326289in}{1.117757in}}%
\pgfpathlineto{\pgfqpoint{1.377006in}{1.034788in}}%
\pgfpathlineto{\pgfqpoint{1.420807in}{0.967140in}}%
\pgfpathlineto{\pgfqpoint{1.462303in}{0.907044in}}%
\pgfpathlineto{\pgfqpoint{1.501494in}{0.854269in}}%
\pgfpathlineto{\pgfqpoint{1.538379in}{0.808434in}}%
\pgfpathlineto{\pgfqpoint{1.572959in}{0.769053in}}%
\pgfpathlineto{\pgfqpoint{1.605234in}{0.735568in}}%
\pgfpathlineto{\pgfqpoint{1.637509in}{0.705334in}}%
\pgfpathlineto{\pgfqpoint{1.667478in}{0.680219in}}%
\pgfpathlineto{\pgfqpoint{1.697448in}{0.657969in}}%
\pgfpathlineto{\pgfqpoint{1.727417in}{0.638569in}}%
\pgfpathlineto{\pgfqpoint{1.757386in}{0.621972in}}%
\pgfpathlineto{\pgfqpoint{1.787356in}{0.608094in}}%
\pgfpathlineto{\pgfqpoint{1.817325in}{0.596815in}}%
\pgfpathlineto{\pgfqpoint{1.847294in}{0.587975in}}%
\pgfpathlineto{\pgfqpoint{1.877264in}{0.581373in}}%
\pgfpathlineto{\pgfqpoint{1.909538in}{0.576482in}}%
\pgfpathlineto{\pgfqpoint{1.946424in}{0.573243in}}%
\pgfpathlineto{\pgfqpoint{1.990225in}{0.571769in}}%
\pgfpathlineto{\pgfqpoint{2.091660in}{0.571603in}}%
\pgfpathlineto{\pgfqpoint{2.734848in}{0.571603in}}%
\pgfpathlineto{\pgfqpoint{2.734848in}{0.571603in}}%
\pgfusepath{stroke}%
\end{pgfscope}%
\begin{pgfscope}%
\pgfpathrectangle{\pgfqpoint{0.316667in}{0.571603in}}{\pgfqpoint{2.533333in}{1.778397in}}%
\pgfusepath{clip}%
\pgfsetrectcap%
\pgfsetroundjoin%
\pgfsetlinewidth{1.505625pt}%
\definecolor{currentstroke}{rgb}{0.031373,0.188235,0.419608}%
\pgfsetstrokecolor{currentstroke}%
\pgfsetdash{}{0pt}%
\pgfpathmoveto{\pgfqpoint{0.431818in}{2.265314in}}%
\pgfpathlineto{\pgfqpoint{0.454872in}{2.264254in}}%
\pgfpathlineto{\pgfqpoint{0.477925in}{2.261076in}}%
\pgfpathlineto{\pgfqpoint{0.500979in}{2.255790in}}%
\pgfpathlineto{\pgfqpoint{0.526337in}{2.247559in}}%
\pgfpathlineto{\pgfqpoint{0.551696in}{2.236827in}}%
\pgfpathlineto{\pgfqpoint{0.577055in}{2.223632in}}%
\pgfpathlineto{\pgfqpoint{0.602413in}{2.208022in}}%
\pgfpathlineto{\pgfqpoint{0.630077in}{2.188306in}}%
\pgfpathlineto{\pgfqpoint{0.657741in}{2.165867in}}%
\pgfpathlineto{\pgfqpoint{0.687711in}{2.138599in}}%
\pgfpathlineto{\pgfqpoint{0.717680in}{2.108385in}}%
\pgfpathlineto{\pgfqpoint{0.749955in}{2.072726in}}%
\pgfpathlineto{\pgfqpoint{0.784535in}{2.031157in}}%
\pgfpathlineto{\pgfqpoint{0.821420in}{1.983284in}}%
\pgfpathlineto{\pgfqpoint{0.860611in}{1.928811in}}%
\pgfpathlineto{\pgfqpoint{0.904412in}{1.864080in}}%
\pgfpathlineto{\pgfqpoint{0.955130in}{1.784894in}}%
\pgfpathlineto{\pgfqpoint{1.015068in}{1.686857in}}%
\pgfpathlineto{\pgfqpoint{1.098060in}{1.546368in}}%
\pgfpathlineto{\pgfqpoint{1.238686in}{1.308064in}}%
\pgfpathlineto{\pgfqpoint{1.300930in}{1.207158in}}%
\pgfpathlineto{\pgfqpoint{1.353953in}{1.125210in}}%
\pgfpathlineto{\pgfqpoint{1.402365in}{1.054386in}}%
\pgfpathlineto{\pgfqpoint{1.446166in}{0.994074in}}%
\pgfpathlineto{\pgfqpoint{1.487662in}{0.940557in}}%
\pgfpathlineto{\pgfqpoint{1.526853in}{0.893457in}}%
\pgfpathlineto{\pgfqpoint{1.566043in}{0.849846in}}%
\pgfpathlineto{\pgfqpoint{1.602929in}{0.812071in}}%
\pgfpathlineto{\pgfqpoint{1.639814in}{0.777507in}}%
\pgfpathlineto{\pgfqpoint{1.674394in}{0.748020in}}%
\pgfpathlineto{\pgfqpoint{1.708974in}{0.721325in}}%
\pgfpathlineto{\pgfqpoint{1.743554in}{0.697367in}}%
\pgfpathlineto{\pgfqpoint{1.778134in}{0.676064in}}%
\pgfpathlineto{\pgfqpoint{1.812714in}{0.657311in}}%
\pgfpathlineto{\pgfqpoint{1.847294in}{0.640983in}}%
\pgfpathlineto{\pgfqpoint{1.884180in}{0.626074in}}%
\pgfpathlineto{\pgfqpoint{1.921065in}{0.613564in}}%
\pgfpathlineto{\pgfqpoint{1.960256in}{0.602662in}}%
\pgfpathlineto{\pgfqpoint{2.001752in}{0.593512in}}%
\pgfpathlineto{\pgfqpoint{2.045553in}{0.586168in}}%
\pgfpathlineto{\pgfqpoint{2.093965in}{0.580360in}}%
\pgfpathlineto{\pgfqpoint{2.149293in}{0.576061in}}%
\pgfpathlineto{\pgfqpoint{2.216148in}{0.573246in}}%
\pgfpathlineto{\pgfqpoint{2.306056in}{0.571859in}}%
\pgfpathlineto{\pgfqpoint{2.518147in}{0.571603in}}%
\pgfpathlineto{\pgfqpoint{2.734848in}{0.571603in}}%
\pgfpathlineto{\pgfqpoint{2.734848in}{0.571603in}}%
\pgfusepath{stroke}%
\end{pgfscope}%
\begin{pgfscope}%
\pgfpathrectangle{\pgfqpoint{0.316667in}{0.571603in}}{\pgfqpoint{2.533333in}{1.778397in}}%
\pgfusepath{clip}%
\pgfsetrectcap%
\pgfsetroundjoin%
\pgfsetlinewidth{1.505625pt}%
\definecolor{currentstroke}{rgb}{0.031373,0.188235,0.419608}%
\pgfsetstrokecolor{currentstroke}%
\pgfsetdash{}{0pt}%
\pgfpathmoveto{\pgfqpoint{0.431818in}{2.265314in}}%
\pgfpathlineto{\pgfqpoint{0.454872in}{2.264254in}}%
\pgfpathlineto{\pgfqpoint{0.477925in}{2.261076in}}%
\pgfpathlineto{\pgfqpoint{0.500979in}{2.255792in}}%
\pgfpathlineto{\pgfqpoint{0.526337in}{2.247568in}}%
\pgfpathlineto{\pgfqpoint{0.551696in}{2.236851in}}%
\pgfpathlineto{\pgfqpoint{0.577055in}{2.223683in}}%
\pgfpathlineto{\pgfqpoint{0.602413in}{2.208119in}}%
\pgfpathlineto{\pgfqpoint{0.630077in}{2.188481in}}%
\pgfpathlineto{\pgfqpoint{0.657741in}{2.166159in}}%
\pgfpathlineto{\pgfqpoint{0.687711in}{2.139073in}}%
\pgfpathlineto{\pgfqpoint{0.719985in}{2.106692in}}%
\pgfpathlineto{\pgfqpoint{0.752260in}{2.071185in}}%
\pgfpathlineto{\pgfqpoint{0.786840in}{2.029925in}}%
\pgfpathlineto{\pgfqpoint{0.826031in}{1.979508in}}%
\pgfpathlineto{\pgfqpoint{0.867527in}{1.922378in}}%
\pgfpathlineto{\pgfqpoint{0.913634in}{1.855061in}}%
\pgfpathlineto{\pgfqpoint{0.966656in}{1.773663in}}%
\pgfpathlineto{\pgfqpoint{1.035816in}{1.663072in}}%
\pgfpathlineto{\pgfqpoint{1.264045in}{1.294528in}}%
\pgfpathlineto{\pgfqpoint{1.321678in}{1.207567in}}%
\pgfpathlineto{\pgfqpoint{1.372395in}{1.134872in}}%
\pgfpathlineto{\pgfqpoint{1.420807in}{1.069358in}}%
\pgfpathlineto{\pgfqpoint{1.464609in}{1.013670in}}%
\pgfpathlineto{\pgfqpoint{1.508410in}{0.961606in}}%
\pgfpathlineto{\pgfqpoint{1.549906in}{0.915756in}}%
\pgfpathlineto{\pgfqpoint{1.589097in}{0.875621in}}%
\pgfpathlineto{\pgfqpoint{1.628288in}{0.838585in}}%
\pgfpathlineto{\pgfqpoint{1.667478in}{0.804629in}}%
\pgfpathlineto{\pgfqpoint{1.706669in}{0.773704in}}%
\pgfpathlineto{\pgfqpoint{1.745860in}{0.745730in}}%
\pgfpathlineto{\pgfqpoint{1.785050in}{0.720601in}}%
\pgfpathlineto{\pgfqpoint{1.824241in}{0.698189in}}%
\pgfpathlineto{\pgfqpoint{1.863432in}{0.678348in}}%
\pgfpathlineto{\pgfqpoint{1.902622in}{0.660916in}}%
\pgfpathlineto{\pgfqpoint{1.944118in}{0.644894in}}%
\pgfpathlineto{\pgfqpoint{1.987920in}{0.630468in}}%
\pgfpathlineto{\pgfqpoint{2.031721in}{0.618347in}}%
\pgfpathlineto{\pgfqpoint{2.080133in}{0.607325in}}%
\pgfpathlineto{\pgfqpoint{2.130851in}{0.598102in}}%
\pgfpathlineto{\pgfqpoint{2.186179in}{0.590329in}}%
\pgfpathlineto{\pgfqpoint{2.248423in}{0.583912in}}%
\pgfpathlineto{\pgfqpoint{2.319888in}{0.578892in}}%
\pgfpathlineto{\pgfqpoint{2.405186in}{0.575249in}}%
\pgfpathlineto{\pgfqpoint{2.515842in}{0.572906in}}%
\pgfpathlineto{\pgfqpoint{2.686436in}{0.571785in}}%
\pgfpathlineto{\pgfqpoint{2.734848in}{0.571696in}}%
\pgfpathlineto{\pgfqpoint{2.734848in}{0.571696in}}%
\pgfusepath{stroke}%
\end{pgfscope}%
\begin{pgfscope}%
\pgfpathrectangle{\pgfqpoint{0.316667in}{0.571603in}}{\pgfqpoint{2.533333in}{1.778397in}}%
\pgfusepath{clip}%
\pgfsetrectcap%
\pgfsetroundjoin%
\pgfsetlinewidth{1.505625pt}%
\definecolor{currentstroke}{rgb}{1.000000,0.000000,0.000000}%
\pgfsetstrokecolor{currentstroke}%
\pgfsetdash{}{0pt}%
\pgfpathmoveto{\pgfqpoint{0.431818in}{2.265314in}}%
\pgfpathlineto{\pgfqpoint{0.454872in}{2.264254in}}%
\pgfpathlineto{\pgfqpoint{0.477925in}{2.261077in}}%
\pgfpathlineto{\pgfqpoint{0.500979in}{2.255795in}}%
\pgfpathlineto{\pgfqpoint{0.526337in}{2.247578in}}%
\pgfpathlineto{\pgfqpoint{0.551696in}{2.236875in}}%
\pgfpathlineto{\pgfqpoint{0.577055in}{2.223735in}}%
\pgfpathlineto{\pgfqpoint{0.602413in}{2.208216in}}%
\pgfpathlineto{\pgfqpoint{0.630077in}{2.188655in}}%
\pgfpathlineto{\pgfqpoint{0.657741in}{2.166449in}}%
\pgfpathlineto{\pgfqpoint{0.687711in}{2.139542in}}%
\pgfpathlineto{\pgfqpoint{0.719985in}{2.107432in}}%
\pgfpathlineto{\pgfqpoint{0.754565in}{2.069672in}}%
\pgfpathlineto{\pgfqpoint{0.791451in}{2.025893in}}%
\pgfpathlineto{\pgfqpoint{0.830641in}{1.975836in}}%
\pgfpathlineto{\pgfqpoint{0.874443in}{1.916155in}}%
\pgfpathlineto{\pgfqpoint{0.925160in}{1.843010in}}%
\pgfpathlineto{\pgfqpoint{0.985099in}{1.752414in}}%
\pgfpathlineto{\pgfqpoint{1.072702in}{1.615468in}}%
\pgfpathlineto{\pgfqpoint{1.204106in}{1.410314in}}%
\pgfpathlineto{\pgfqpoint{1.270961in}{1.310323in}}%
\pgfpathlineto{\pgfqpoint{1.326289in}{1.231374in}}%
\pgfpathlineto{\pgfqpoint{1.377006in}{1.162681in}}%
\pgfpathlineto{\pgfqpoint{1.425418in}{1.100792in}}%
\pgfpathlineto{\pgfqpoint{1.471525in}{1.045446in}}%
\pgfpathlineto{\pgfqpoint{1.515326in}{0.996263in}}%
\pgfpathlineto{\pgfqpoint{1.559127in}{0.950467in}}%
\pgfpathlineto{\pgfqpoint{1.600623in}{0.910232in}}%
\pgfpathlineto{\pgfqpoint{1.642120in}{0.873043in}}%
\pgfpathlineto{\pgfqpoint{1.683616in}{0.838852in}}%
\pgfpathlineto{\pgfqpoint{1.725112in}{0.807580in}}%
\pgfpathlineto{\pgfqpoint{1.766608in}{0.779123in}}%
\pgfpathlineto{\pgfqpoint{1.808104in}{0.753358in}}%
\pgfpathlineto{\pgfqpoint{1.849600in}{0.730148in}}%
\pgfpathlineto{\pgfqpoint{1.893401in}{0.708253in}}%
\pgfpathlineto{\pgfqpoint{1.937202in}{0.688850in}}%
\pgfpathlineto{\pgfqpoint{1.983309in}{0.670909in}}%
\pgfpathlineto{\pgfqpoint{2.031721in}{0.654569in}}%
\pgfpathlineto{\pgfqpoint{2.082439in}{0.639920in}}%
\pgfpathlineto{\pgfqpoint{2.135461in}{0.627004in}}%
\pgfpathlineto{\pgfqpoint{2.190789in}{0.615808in}}%
\pgfpathlineto{\pgfqpoint{2.250728in}{0.605937in}}%
\pgfpathlineto{\pgfqpoint{2.317583in}{0.597246in}}%
\pgfpathlineto{\pgfqpoint{2.391354in}{0.589960in}}%
\pgfpathlineto{\pgfqpoint{2.476651in}{0.583878in}}%
\pgfpathlineto{\pgfqpoint{2.575780in}{0.579128in}}%
\pgfpathlineto{\pgfqpoint{2.697963in}{0.575591in}}%
\pgfpathlineto{\pgfqpoint{2.734848in}{0.574873in}}%
\pgfpathlineto{\pgfqpoint{2.734848in}{0.574873in}}%
\pgfusepath{stroke}%
\end{pgfscope}%
\begin{pgfscope}%
\pgfsetrectcap%
\pgfsetmiterjoin%
\pgfsetlinewidth{0.803000pt}%
\definecolor{currentstroke}{rgb}{0.000000,0.000000,0.000000}%
\pgfsetstrokecolor{currentstroke}%
\pgfsetdash{}{0pt}%
\pgfpathmoveto{\pgfqpoint{0.316667in}{0.571603in}}%
\pgfpathlineto{\pgfqpoint{0.316667in}{2.350000in}}%
\pgfusepath{stroke}%
\end{pgfscope}%
\begin{pgfscope}%
\pgfsetrectcap%
\pgfsetmiterjoin%
\pgfsetlinewidth{0.803000pt}%
\definecolor{currentstroke}{rgb}{0.000000,0.000000,0.000000}%
\pgfsetstrokecolor{currentstroke}%
\pgfsetdash{}{0pt}%
\pgfpathmoveto{\pgfqpoint{2.850000in}{0.571603in}}%
\pgfpathlineto{\pgfqpoint{2.850000in}{2.350000in}}%
\pgfusepath{stroke}%
\end{pgfscope}%
\begin{pgfscope}%
\pgfsetrectcap%
\pgfsetmiterjoin%
\pgfsetlinewidth{0.803000pt}%
\definecolor{currentstroke}{rgb}{0.000000,0.000000,0.000000}%
\pgfsetstrokecolor{currentstroke}%
\pgfsetdash{}{0pt}%
\pgfpathmoveto{\pgfqpoint{0.316667in}{0.571603in}}%
\pgfpathlineto{\pgfqpoint{2.850000in}{0.571603in}}%
\pgfusepath{stroke}%
\end{pgfscope}%
\begin{pgfscope}%
\pgfsetrectcap%
\pgfsetmiterjoin%
\pgfsetlinewidth{0.803000pt}%
\definecolor{currentstroke}{rgb}{0.000000,0.000000,0.000000}%
\pgfsetstrokecolor{currentstroke}%
\pgfsetdash{}{0pt}%
\pgfpathmoveto{\pgfqpoint{0.316667in}{2.350000in}}%
\pgfpathlineto{\pgfqpoint{2.850000in}{2.350000in}}%
\pgfusepath{stroke}%
\end{pgfscope}%
\begin{pgfscope}%
\pgfsetbuttcap%
\pgfsetmiterjoin%
\definecolor{currentfill}{rgb}{1.000000,1.000000,1.000000}%
\pgfsetfillcolor{currentfill}%
\pgfsetfillopacity{0.800000}%
\pgfsetlinewidth{1.003750pt}%
\definecolor{currentstroke}{rgb}{0.800000,0.800000,0.800000}%
\pgfsetstrokecolor{currentstroke}%
\pgfsetstrokeopacity{0.800000}%
\pgfsetdash{}{0pt}%
\pgfpathmoveto{\pgfqpoint{1.770025in}{0.977861in}}%
\pgfpathlineto{\pgfqpoint{2.752778in}{0.977861in}}%
\pgfpathquadraticcurveto{\pgfqpoint{2.780556in}{0.977861in}}{\pgfqpoint{2.780556in}{1.005639in}}%
\pgfpathlineto{\pgfqpoint{2.780556in}{2.252778in}}%
\pgfpathquadraticcurveto{\pgfqpoint{2.780556in}{2.280556in}}{\pgfqpoint{2.752778in}{2.280556in}}%
\pgfpathlineto{\pgfqpoint{1.770025in}{2.280556in}}%
\pgfpathquadraticcurveto{\pgfqpoint{1.742247in}{2.280556in}}{\pgfqpoint{1.742247in}{2.252778in}}%
\pgfpathlineto{\pgfqpoint{1.742247in}{1.005639in}}%
\pgfpathquadraticcurveto{\pgfqpoint{1.742247in}{0.977861in}}{\pgfqpoint{1.770025in}{0.977861in}}%
\pgfpathlineto{\pgfqpoint{1.770025in}{0.977861in}}%
\pgfpathclose%
\pgfusepath{stroke,fill}%
\end{pgfscope}%
\begin{pgfscope}%
\pgfsetrectcap%
\pgfsetroundjoin%
\pgfsetlinewidth{1.505625pt}%
\definecolor{currentstroke}{rgb}{0.579608,0.770196,0.873725}%
\pgfsetstrokecolor{currentstroke}%
\pgfsetdash{}{0pt}%
\pgfpathmoveto{\pgfqpoint{1.797803in}{2.168088in}}%
\pgfpathlineto{\pgfqpoint{1.936692in}{2.168088in}}%
\pgfpathlineto{\pgfqpoint{2.075580in}{2.168088in}}%
\pgfusepath{stroke}%
\end{pgfscope}%
\begin{pgfscope}%
\definecolor{textcolor}{rgb}{0.000000,0.000000,0.000000}%
\pgfsetstrokecolor{textcolor}%
\pgfsetfillcolor{textcolor}%
\pgftext[x=2.186692in,y=2.119477in,left,base]{\color{textcolor}\sffamily\fontsize{10.000000}{12.000000}\selectfont \(\displaystyle n=1\)}%
\end{pgfscope}%
\begin{pgfscope}%
\pgfsetrectcap%
\pgfsetroundjoin%
\pgfsetlinewidth{1.505625pt}%
\definecolor{currentstroke}{rgb}{0.290980,0.594510,0.789020}%
\pgfsetstrokecolor{currentstroke}%
\pgfsetdash{}{0pt}%
\pgfpathmoveto{\pgfqpoint{1.797803in}{1.964231in}}%
\pgfpathlineto{\pgfqpoint{1.936692in}{1.964231in}}%
\pgfpathlineto{\pgfqpoint{2.075580in}{1.964231in}}%
\pgfusepath{stroke}%
\end{pgfscope}%
\begin{pgfscope}%
\definecolor{textcolor}{rgb}{0.000000,0.000000,0.000000}%
\pgfsetstrokecolor{textcolor}%
\pgfsetfillcolor{textcolor}%
\pgftext[x=2.186692in,y=1.915620in,left,base]{\color{textcolor}\sffamily\fontsize{10.000000}{12.000000}\selectfont \(\displaystyle n=2\)}%
\end{pgfscope}%
\begin{pgfscope}%
\pgfsetrectcap%
\pgfsetroundjoin%
\pgfsetlinewidth{1.505625pt}%
\definecolor{currentstroke}{rgb}{0.090196,0.392941,0.670588}%
\pgfsetstrokecolor{currentstroke}%
\pgfsetdash{}{0pt}%
\pgfpathmoveto{\pgfqpoint{1.797803in}{1.760374in}}%
\pgfpathlineto{\pgfqpoint{1.936692in}{1.760374in}}%
\pgfpathlineto{\pgfqpoint{2.075580in}{1.760374in}}%
\pgfusepath{stroke}%
\end{pgfscope}%
\begin{pgfscope}%
\definecolor{textcolor}{rgb}{0.000000,0.000000,0.000000}%
\pgfsetstrokecolor{textcolor}%
\pgfsetfillcolor{textcolor}%
\pgftext[x=2.186692in,y=1.711763in,left,base]{\color{textcolor}\sffamily\fontsize{10.000000}{12.000000}\selectfont \(\displaystyle n=3\)}%
\end{pgfscope}%
\begin{pgfscope}%
\pgfsetrectcap%
\pgfsetroundjoin%
\pgfsetlinewidth{1.505625pt}%
\definecolor{currentstroke}{rgb}{0.031373,0.188235,0.419608}%
\pgfsetstrokecolor{currentstroke}%
\pgfsetdash{}{0pt}%
\pgfpathmoveto{\pgfqpoint{1.797803in}{1.556516in}}%
\pgfpathlineto{\pgfqpoint{1.936692in}{1.556516in}}%
\pgfpathlineto{\pgfqpoint{2.075580in}{1.556516in}}%
\pgfusepath{stroke}%
\end{pgfscope}%
\begin{pgfscope}%
\definecolor{textcolor}{rgb}{0.000000,0.000000,0.000000}%
\pgfsetstrokecolor{textcolor}%
\pgfsetfillcolor{textcolor}%
\pgftext[x=2.186692in,y=1.507905in,left,base]{\color{textcolor}\sffamily\fontsize{10.000000}{12.000000}\selectfont \(\displaystyle n=5\)}%
\end{pgfscope}%
\begin{pgfscope}%
\pgfsetrectcap%
\pgfsetroundjoin%
\pgfsetlinewidth{1.505625pt}%
\definecolor{currentstroke}{rgb}{0.031373,0.188235,0.419608}%
\pgfsetstrokecolor{currentstroke}%
\pgfsetdash{}{0pt}%
\pgfpathmoveto{\pgfqpoint{1.797803in}{1.352659in}}%
\pgfpathlineto{\pgfqpoint{1.936692in}{1.352659in}}%
\pgfpathlineto{\pgfqpoint{2.075580in}{1.352659in}}%
\pgfusepath{stroke}%
\end{pgfscope}%
\begin{pgfscope}%
\definecolor{textcolor}{rgb}{0.000000,0.000000,0.000000}%
\pgfsetstrokecolor{textcolor}%
\pgfsetfillcolor{textcolor}%
\pgftext[x=2.186692in,y=1.304048in,left,base]{\color{textcolor}\sffamily\fontsize{10.000000}{12.000000}\selectfont \(\displaystyle n=10\)}%
\end{pgfscope}%
\begin{pgfscope}%
\pgfsetrectcap%
\pgfsetroundjoin%
\pgfsetlinewidth{1.505625pt}%
\definecolor{currentstroke}{rgb}{1.000000,0.000000,0.000000}%
\pgfsetstrokecolor{currentstroke}%
\pgfsetdash{}{0pt}%
\pgfpathmoveto{\pgfqpoint{1.797803in}{1.110918in}}%
\pgfpathlineto{\pgfqpoint{1.936692in}{1.110918in}}%
\pgfpathlineto{\pgfqpoint{2.075580in}{1.110918in}}%
\pgfusepath{stroke}%
\end{pgfscope}%
\begin{pgfscope}%
\definecolor{textcolor}{rgb}{0.000000,0.000000,0.000000}%
\pgfsetstrokecolor{textcolor}%
\pgfsetfillcolor{textcolor}%
\pgftext[x=2.186692in,y=1.062307in,left,base]{\color{textcolor}\sffamily\fontsize{10.000000}{12.000000}\selectfont \(\displaystyle t \mapsto \mathrm{e}^{-t^2}\)}%
\end{pgfscope}%
\end{pgfpicture}%
\makeatother%
\endgroup%

    \caption{Illustration de la convergence simple de la suite $(f_n)_{n \in \N}$ vers $f$}
\end{marginfigure}

\begin{exercice}
\marginnote[0cm]{Source : \cite{maths-france} Planche no 13. Suites et séries d’intégrales}
On note $I = \int_0^{+\infty} \e^{-x^2} \d x$ et on pose $\fonctionligne[f]{x}{\e^{-x^2}}$.

\begin{enumerate}
\item Montrer que $I$ est une intégrale convergente.
\end{enumerate}
On propose ensuite de déterminer la valeur de $I$.
\begin{enumerate}[resume]
\item \textbf{Première méthode: \say{ à la main }.} \\ 
Pour $n \in \Ne$, on pose
$$
g_n(x) \defeq
\begin{cases}
\left(1 - \frac{x}{n} \right)^n &\text{si } x \in \interff{0}{n} \\
0 &\text{si } x \geqslant n
\end{cases}.
$$

On pose $\fonctionligne[g]{x}{\e^{-x}}$ et $h_n = g - g_n$·
\begin{enumerate}
\item Montrer que $h_n$ atteint son maximum sur $\interff{0}{n}$. On notera $x_n$ l'abscisse d'un point en laquelle $h_n$ atteint ce maximum.

\item Montrer que $h_n$ est à valeurs positives.

\item Montrer que $h_n(x_n) = \frac{x_n}{n} \e^{-x_n}$.

\item En étudiant la fonction $u \mapsto u \e^{-u}$, en déduire que
\[
\module{h_n} \leqslant \frac{1}{n \e}.
\]

\item On pose $I_n= \int_0^{+\infty} g_n(x^2) \d x$. Montrer que $\module{I_n - I} \leqslant \frac{1}{\e \sqrt{n}} + \int_{\sqrt{n}}^{+\infty} \e^{-x^2} \d x$.

\item En déduire le résultat attendu.
\end{enumerate}

\item \textbf{Deuxième méthode: avec le théorème de convergence dominée et les intégrales de \textsc{Wallis.}} \\
On pose $f_n(x) = \left(1 - \frac{x^2}{n}\right)^n \indicatrice{\interfo{0}{\sqrt{n}}}(x)$.
\begin{enumerate}
\item Montrer que la suite $(f_n)_{n\in\N}$ converge simplement vers $f$.

\item À l'aide du théorème de convergence dominée, en déduire que 
\[
\lim\limits_{n\to+\infty} \int_{\R_+} f_n = \int_0^{+\infty} \e^{-x^2} \d x.
\]
\end{enumerate}

\item On rappelle que, d'après les résultats sur les intégrales de \textsc{Wallis},
\[
\lim_{n\to+\infty} \sqrt{n} \int_0^{\frac{\pi}{2}} \cos(t)^{2n+1} \d t = \frac{\sqrt{\pi}}{2}.
\]

En déduire la valeur de $I$.
\end{enumerate}
% Pour $n \in \Ne$, on pose
        % $$
        % f_n(x) \defeq
        % \begin{cases}
            % \left(1 - \frac{x^2}{n} \right)^n &\text{si } x \in [0, \sqrt{n}] \\
            % 0 &\text{si } x > \sqrt{n}
        % \end{cases}.
        % $$
        % \begin{enumerate}
            % \item Montrer que la suite $(f_n)_{n \in \Ne}$ converge simplement sur $\Rp$ vers la fonction $f:x \mapsto \e^{-x^2}$.
            % \item À l'aide de la convergence dominée, calculer l'intégrale de \textsc{Gauss}.
        % \end{enumerate}
\end{exercice}

\begin{preuve}
\begin{enumerate}
\item La fonction $x \mapsto \e^{-x^2}$ est continue sur $\interfo{0}{+\infty}$. D'après le théorème des croissances comparées, $\e^{-x^2} = o_{+\infty}\mathopen{}\left(\frac{1}{x^2}\right)$. Ainsi, d'après le théorème de comparaison aux intégrales de \textsc{Riemann}, la fonction $x \mapsto \e^{-x^2}$ est intégrable sur $\interfo{0}{+\infty}$.

\item
\begin{enumerate}
\item Pour tout $x \geqslant n$, $h_n(x) = \e^{-x} \leqslant \e^{-n} = h_n(n)$. De plus, la restriction de la fonction $h_n$ au segment $\interff{0}{n}$ est continue sur ce segment, donc elle y est bornée et atteint ses bornes. On note $x_n$ l'abscisse d'un point où $h_n$ atteint ce maximum.

\item En utilisant l'inégalité de convexité du logarithme, pour $x \in \interff{0}{n}$,
\[
g_n(x)
= \exp\mathopen{}\left(n \ln\mathopen{}\left(1 - \frac{x}{n}\right)\right)
\leqslant \e^{-x} = g(x).
\]
Ainsi, $h_n$ est à valeurs positives.

\item Pour tout $x \in \interff{0}{n}$,
\[
h_n'(x) = -\e^{-x} + \left(1 - \frac{x}{n}\right)^{n-1}.
\]
Ainsi, $h_n'(0) = 0$ et $h_n'(n) = - \e^{-n} < 0$.

Comme $h_n'(n) < 0$, alors $h_n$ est décroissante sur un voisinage de $n$. De plus, $h_n$ est positive et $h_n(0) = 0$. Ainsi, $x_n \in \interoo{0}{n}$. Comme $h_n$ est dérivable et atteint son maximum sur l'ouvert $\interoo{0}{n}$, alors $h_n'(x_n) = 0$, soit
\[
\e^{-x_n} = \left(1 - \frac{x_n}{n}\right)^{n-1}.
\]

Alors,
\begin{align*}
h_n(x_n)
&= \e^{-x_n} - \left(1 - \frac{x_n}{n}\right)^n
= \e^{-x_n} - \left(1 - \frac{x_n}{n}\right) \e^{-x_n}\\
&= \frac{x_n \e^{-x_n}}{n}.
\end{align*}

\item La fonction $u \mapsto u \e^{-u}$ est dérivable et sa dérivée vaut $u \mapsto \e^{-u} (1 - u)$. Elle atteint donc son maximum en $1$ et la valeur de ce maximum est $\e^{-1}$.

D'après la question précédente, pour tout $x \in \Rp$,
\begin{align*}
\module{h_n(x)}
\leqslant h_n(x_n)
\leqslant \frac{1}{n \e}.
\end{align*}

\item D'après la question précédente, pour tout $x$ réel positif,
\[
\module{\e^{-x^2} - \left(1 - \frac{x^2}{n}\right)^n} \indicatrice{\interff{0}{n}}(x^2)
\leqslant \frac{1}{n \e}.
\]

Ainsi,
\begin{align*}
\module{I_n - I}
&\leqslant \int_0^{+\infty} \module{h_n(x^2)} \d x\\
&\leqslant \int_0^{\sqrt{n}} \module{h_n(x^2)} \d x + \int_{\sqrt{n}}^{+\infty} \module{h_n(x^2)} \d x\\
&\leqslant \int_0^{\sqrt{n}} \frac{1}{n\e} \d x + \int_{\sqrt{n}}^{+\infty} \e^{-x^2} \d x\\
&\leqslant \frac{1}{\sqrt{n} \e} + \int_{\sqrt{n}}^{+\infty} \e^{-x^2} \d x.
\end{align*}

\item Comme la fonction $x \mapsto \e^{-x^2}$ est intégrable,
\[
\lim_{n\to+\infty} \int_{\sqrt{n}}^{+\infty} \e^{-x^2} \d x = 0.
\]

Ainsi, d'après le théorème d'encadrement,
\[
\lim_{n\to+\infty} I_n
= \lim_{n\to+\infty} \int_0^{\sqrt{n}} \left(1 - \frac{x^2}{n}\right)^n \d x
= \int_0^{+\infty} \e^{-x^2} \d x.
\]
\todoarmand{Je trouve que le terme du milieu porte à confusion, la première égalité est une définition tandis que la deuxième est un résultat.}
\end{enumerate}

\item
\begin{enumerate}
\item Soit $x \in \Rp$ tel que $n \geqslant x^2$. Alors, $x \in \interff{0}{\sqrt{n}}$ et 
\[
f_n(x)
= \left(1 - \frac{x^2}{n}\right)^n
= \exp\mathopen{}\left(n \ln\mathopen{}\left(1 - \frac{x^2}{n}\right)\right).
\]

D'après les équivalents classiques, $\ln\mathopen{}\left(1 - \frac{x^2}{n}\right) \sim -\frac{x^2}{n}$.

Ainsi, d'après la continuité de la fonction exponentielle en $x^2$,
\[
\lim_{n\to+\infty} f_n(x) = \e^{-x^2}.
\]

\item On vérifie les hypothèses du théorème:
\begin{itemize}
\item pour tout $n \in \N$, la fonction $f_n$ est continue par morceaux sur $\Rp$, % $t \mapsto \left(1 - \frac{x^2}{n}\right)^n \indicatrice{\interfo{0}{\sqrt{n}}}(t) \in \mathscr{C}^-(\R_+)$.

\item d'après la question précédente, la suite $(f_n)_{n\in\N}$ converge simplement vers $f$, qui est continue par morceaux, 

% \item la fonction $f \in \mathscr{C}^-(\R_+, \R_+)$.

\item en utilisant l'inégalité de convexité du logarithme, pour $x \in \interff{0}{\sqrt{n}}$,
\[
\abs{f_n(x)}
= \exp\mathopen{}\left(n \ln\left(1 - \frac{x^2}{n}\right)\right)
\leqslant \e^{-x^2}.
\]
Ainsi, pour tout $x \in \R_+$, $\abs{f_n(x)} \leqslant f(x)$, qui est bien intégrable.
\end{itemize}

D'après le théorème de convergence dominée,
\[
\lim_{n\to+\infty} \int_{\R_+} f_n = \int_0^{+\infty} \e^{-x^2} \d x.
\]
\end{enumerate}

\item En effectuant le changement de variable $\fonction[\phi]{\interff{0}{\pi/2}}{\interff{0}{\sqrt{n}}}{t}{\sqrt{n} \sin(t)}$ qui est bien de classe $\mathscr{C}^1$ et strictement croissant,
\begin{align*}
\int_0^{\sqrt{n}} \left(1 - \frac{x^2}{n}\right)^n \d x &= \int_0^{\pi/2} \big(1 - \sin(t)^2\big)^n \sqrt{n} \cos(t) \d t \\
&= \sqrt{n} \int_0^{\pi/2} \cos^{2n+1}(t) \d t \xrightarrow[n \to \infty]{} \frac{\sqrt{\pi}}{2}.
\end{align*}
Finalement, $I = \frac{\sqrt{\pi}}{2}$.
\end{enumerate}
\end{preuve}

% \todoinline{J'ajouterais ici cette remarque et je supprimerais la partie sur les intégrales de Wallis}
\marginnote[0cm]{Source : \href{https://fr.wikipedia.org/wiki/Intégrale_de_Wallis}{Intégrale de \textsc{Wallis} -- \textsf{wikipedia.org}}}
\begin{remarque}
La convexité de la fonction exponentielle permet même de montrer que, pour tout entier $n \in \Ne$ et tout réel $u \in \interoo{-n}{n}$, 
\[
\left(1 + \frac{u}{n} \right)^n \leqslant \e^u \leqslant \left( 1 - \frac{u}{n} \right)^{-n},
\]
puis que
\[
\int_0^{\sqrt{n}} \left( 1 - \frac{x^2}{n} \right)^n \d x \leqslant \int_0^{\sqrt{n}} \e^{-x^2} \d x \leqslant \int_0^{\sqrt{n}} \left( 1 + \frac{x^2}{n} \right)^{-n} \d x.
\]
En utilisant un changement de variable puis les intégrales de \textsc{Wallis}, on obtient alors l'encadement 
\[
\sqrt{n}\, \Wallis_{2n+1} \leqslant \int_0^{\sqrt{n}} \e^{-x^2} \d x \leqslant \sqrt{n}\, \Wallis_{2n-2}.
\]
\end{remarque}

\begin{remarque}
En raison de la parité de la fonction $x \mapsto \e^{-x^2}$, $\int_\R \e^{-x^2} \d x = \sqrt{\pi}$.
\end{remarque}

\todoinline{Ajouter une remarque pour son utilisation en probas ? Avec un dessin d'une planche de Galton ?}

\subsection{Moments de la gaussienne}

\todoinline{Je prends $\sigma = \frac{1}{\sqrt{2}}$ pour plus coller à la section précédente et simplifier les notations. On peut bien sûr revenir à un $\sigma$ qcq si tu préfères. J'ai rédigé la dérivation sous le signe intégral}

\begin{theo}{} Pour tout $n$ entier naturel,
    \[
    \int_\R x^{2n} \e^{-x^2} \d x
    = \frac{\sqrt{\pi}}{2^n} \prod_{k=1}^n (2k - 1)
    = \frac{\sqrt{\pi}}{4^n} \times n! \binom{2n}{n}.
    \]
\end{theo}

\marginnote[0cm]{D'après \url{https://djalil.chafai.net/blog/2024/04/28/two-details-about-gaussians/}}
\begin{exercice}
\begin{enumerate}
\item Pour tout $\beta > 0$, déterminer la valeur de $\int_\R \e^{-\beta x^2} \d x$.

\item Calculer la dérivée $n$-ième de la fonction $\fonctionligne[f]{\beta}{\sqrt{\frac{\pi}{\beta}}}$.

\item En déduire le résultat annoncé.
\end{enumerate}
\end{exercice}

\begin{preuve}
\begin{enumerate}
\item Le changement de variable affine $u = \sqrt{\beta}\, x$ dans l'intégrale permet d'obtenir l'intégrale de \textsc{Gauss}. Ainsi,
\[
\int_\R \e^{-\beta x^2} \d x
= \int_\R \e^{-u^2} \times \frac{1}{\sqrt{\beta}} \d u
= \sqrt{\frac{\pi}{\beta}}.
\]

\item Comme $f(\beta) = \sqrt{\pi}\, \beta^{-1/2}$, une récurrence permet d'obtenir :
\begin{align*}
f_n'(\beta)
&= \sqrt{\pi} (-1)^n \beta^{-\frac{2n+1}{2}} \prod_{k=1}^n \left(\frac{1}{2} + k - 1\right)\\
&= (-1)^n \sqrt{\frac{\pi}{\beta^{2n+1}}} \prod_{k=1}^n \frac{2k-1}{2}.
\end{align*}

\item Posons $\fonctionligne[g]{(\beta, x)}{\e^{-\beta x^2}}$ et $I(\beta) = \int_{-\infty}^{+\infty} g(\beta, x) \d x$.
\begin{enumerate}
\item Pour tout $x \in \R$, la fonction $g(\,\cdot\,, x)$ est de classe $\mathscr{C}^\infty$ sur $\Rpe$.

\item Pour tout $n \in \N$ et $x \in \R$,
\[
\frac{\partial^n g}{\partial \beta^n}(\beta, x) = (-1)^n x^{2n} \e^{-\beta x^2}\]
donc $\frac{\partial^n g}{\partial \beta^n}(\beta, \,\cdot\,)$ est continue sur $\R$.

\item Pour tout $a > 0$, pour tout $\beta \in \interfo{a}{+\infty}$ et pour tout $x \in \R$,
\[
\module{\frac{\partial^n g}{\partial \beta^n}(\beta, x)} \leqslant x^{2n} \e^{-a x^2}
\]
qui est une fonction intégrable.
\end{enumerate}

En utilisant le théorème de dérivation sous le signe intégral, la fonction $I(\,\cdot\,)$ est de classe $\mathscr{C}^\infty$ et pour tout $n$ entier naturel,
\[
I^{(n)}(\beta)
= (-1)^n \int_\R x^{2n} \e^{-\beta x^2} \d x.
\]

Finalement, on obtient l'égalité :
\[
\int_\R x^{2n} \e^{-\beta x^2} \d x
= \sqrt{\frac{\pi}{\beta^{2n+1}}} \prod_{k=1}^n \frac{2k-1}{2}.
\]

Évaluer en $\beta = 1$ permet d'obtenir le résultat annoncé.
\end{enumerate}
\end{preuve}


% \begin{prop}{}
    % \[
    % \int_\R x^{2n} \frac{1}{\sqrt{2 \pi \sigma^2}} \e^{-\frac{x^2}{2 \sigma^2}} \d x = \sigma^{2n} \prod_{k=1}^n (2k - 1) = \sigma^{2n} (2n-1) !!
    % \]
% \end{prop}
% \begin{preuve}
    % La dérivée $n$-ième par rapport à $\beta$ de l'intégrale suivante
    % \[
    % \int_\R \e^{-\beta x^2} \d x = \sqrt{\frac{\pi}{\beta}},
    % \]
    % que l'on obtient aisément par un changement de variable linéaire dans \ref{eqIntGauss}, s'écrit
    % \[
    % (-1)^n \int_\R x^{2n} \e^{-\beta x^2} \d x = \sqrt{\pi} (-1)^n \beta^{-\frac{2n+1}{2}} \prod_{k=1}^n \left( \frac{1}{2} + k - 1 \right) = (-1)^n \sqrt{\frac{\pi}{\beta^{2n+1}}} \prod_{k=1}^n \left( \frac{1}{2} + k - 1 \right),
    % \]
    % et en prenant $\beta = \frac{1}{2 \sigma^2}$, on obtient
    % \[
    % \int_\R x^{2n} \e^{-\frac{x^2}{2 \sigma^2}} \d x = \sqrt{\pi (2 \sigma^2)^{2n+1}} \prod_{k=1}^n \left( \frac{2k-1}{2} \right)
    % \]
    % soit 
    % \[
    % \int_\R x^{2n} \frac{1}{\sqrt{2 \pi \sigma^2}} \e^{-\frac{x^2}{2 \sigma^2}} \d x = \sigma^{2n} \prod_{k=1}^n (2k - 1) = \sigma^{2n} (2n-1) !!.
    % \]
% \end{preuve}
% Cette dernière notation se nomme la \emph{double factorielle}.

\begin{comment}
\subsection{Calcul de l'intégrale de \textsc{Gauss} avec celle de \textsc{Wallis}}
\todoinline{À supprimer suite à la remarque précédente ?}
\marginnote[0cm]{Source : \href{https://fr.wikipedia.org/wiki/Intégrale_de_Wallis}{Intégrale de \textsc{Wallis} -- \textsf{wikipedia.org}}}
On peut aisément utiliser les intégrales de \textsc{Wallis} pour calculer l'intégrale de \text{Gauss}. \\
On utilise pour cela l'encadrement suivant, issu de la construction de la fonction exponentielle par la méthode d'\textsc{Euler}: pour tout entier $n > 0$ et tout réel $u \in ]-n, n[$, 
$$\left(1 + \frac{u}{n} \right)^n \leqslant \e^u \leqslant \left( 1 - \frac{u}{n} \right)^{-n}.$$
Posant alors $u = -x^2$, on obtient:
$$\int_0^{\sqrt{n}} \left( 1 - \frac{x^2}{n} \right)^n \d x \leqslant \int_0^{\sqrt{n}} \e^{-x^2} \d x \leqslant \int_0^{\sqrt{n}} \left( 1 + \frac{x^2}{n} \right)^{-n} \d x.$$
Or les intégrales d'encadrement sont liées aux intégrales de \textsc{Wallis}. Pour celle de gauche, il suffit de poser $x = \sqrt{n} \sin t$ ($t$ variant de $0$ à $\pi/2$). Quant à celle de droite, on peut poser $x = \sqrt{n} \tan t$ ($t$ variant de $0$ à $\pi/4$) puis majorer par l'intégrale de $0$ à $\pi/2$. On obtient ainsi:
$$\sqrt{n} \Wallis_{2n+1} \leqslant \int_0^{\sqrt{n}} \e^{-x^2} \d x \leqslant \sqrt{n} \Wallis_{2n-2}.$$
Par le théorème des gendarmes, on déduit alors de l'équivalent de $\Wallis_n$ que
$$\int_0^{+ \infty} \e^{-x^2} \d x = \frac{\sqrt{\pi}}{2}.$$



%---------------

\begin{exercice}
\begin{enumerate}
\item Montrer que
\[
\int_0^{\sqrt{n}} \left(1 - \frac{t^2}{n}\right)^n \d t \leq \int_0^{\sqrt{n}} \e^{-t^2} \d t \leq \int_0^{+\infty} \frac{\d t}{\left(1 + \frac{t^2}{n}\right)^n}.
\]

\item En déduire que $\int_0^{\sqrt{n}} \e^{-t^2} \d t \sim \sqrt{n} \int_0^{\pi/2} \cos^{2n+1}(\theta) \d \theta$.
{En utilisant les intégrales de {Wallis}, on montre que $\int_0^{+\infty} \e^{-t^2} \d t = \frac{\sqrt{\pi}}{2}$.}
\end{enumerate}
\end{exercice}

\begin{preuve}
\begin{enumerate}
\item Rappelons que $\ln(1 + x) \leq x$. Ainsi, $\left(1 - \frac{t^2}{n}\right)^n \leq \e^{-t^2}$. De même, $-\ln(1+t^2/n) \geq -t^2/n$ et $\e^{-\frac{t^2}{n}} \geq \left(1 + \frac{t^2}{n}\right)^{-n}$.

On obtient ainsi le résultat en intégrant entre $0$ et $\sqrt{n}$. De plus, $\left(1 + \frac{t^2}{n}\right)^{-n} = O(1/t^2)$ donc l'intégrale est convergente.

\item On pose $\varphi : t \mapsto \sqrt{n} \sin(t)$ dans la première intégrale et $\psi : t \mapsto \sqrt{n} \tan(t)$ dans la seconde. On obtient ainsi l'encadrement
\begin{align*}
\sqrt{n} \int_0^{\pi/2} \cos^{2n+1}(t) \d t \leq \int_0^{\sqrt{n}} \e^{-t^2} \d t &\leq \sqrt{n} \int_0^{\pi/2} \cos^{2n-2}(t) \d t\\
& \leq \sqrt{n} \int_0^{\pi/2} \cos^{2n-3}(t) \d t.
\end{align*}
En notant $I_n = \int_0^{\pi/2} \cos^{2n+1}(t) \d t$, comme la suite $(I_n)$ est décroissante. De plus, en utilisant une intégration par parties, on obtient une relation de récurrence puis $I_{n+1} \sim I_n$.

{On obtiendra ce résultat plus simplement en utilisant le théorème de convergence dominée.}
\end{enumerate}
\end{preuve}



% \todoinline{Pour avoir une application, on peut regarder la remarque dans le chapitre sur les polynômes d'Hermite ou alors calculer la transformée de Fourier d'une gaussienne.\\
% Dans le document bestiaire.pdf, on peut trouver :\\
% * une preuve par étude d'intégrale à paramètre en page 178\\
% * une preuve avec Wallis en p. 223 mais je l'ai déjà quelque part rédigée}

% \todoarmand{
% * Je ne vois pas de quelle remarque vous parlez \\
% * On peut calculer la transformée de Fourier d'une gaussienne, ça pourrait aussi être l'occasion de dire un mot sur ce qu'est la TF. \\
% * J'aime bien la preuve avec Wallis, il faudrait alors mettre cette section après celle sur Wallis. Il faut voir comment l'intégrer (c'est le cas de le dire) avec les autres exercices sur Wallis et Stirling pour ne pas trop se répéter.
% }
\end{comment}

%-----------
\subsection{Transformée de Fourier d'une gaussienne}

\todoarmand{On pourrait traiter le cas un peu plus général de la TF de $x \mapsto \e^{-a x^2}$ et ajouter une illustration}

\begin{theo}{}
Pour tout $x$ réel,
\[
\int_{-\infty}^{+\infty} \e^{\i t x} \e^{-t^2} \d t
= \sqrt{\pi}\, \e^{-\frac{x^2}{4}}.
\]
\end{theo}

\begin{exercice}
On pose $f(x) = \int_{-\infty}^{+\infty} \e^{-t^2 + \i t x} \d t$.
\begin{enumerate}
\item Montrer que la fonction $f$  est définie sur $\R$.

\item Montrer que $f$ est dérivable sur $\R$ et que $f'(x) = \int_{-\infty}^{+\infty} \i t \e^{-t^2 + \i t x} \d t$.

\item Montrer que $f$ vérifie l'équation différentielle $2 y' + x y = 0$.
{On pourra utiliser une intégration par parties.}

\item En déduire la valeur de $f$.
\end{enumerate}
\end{exercice}

\begin{preuve} On pose $\fonctionligne[g]{(x, t)}{\e^{-t^2 + \i t x}}$.
\begin{enumerate}
\item Pour tout $(x, t) \in \R^2$,  $\abs{g(x, t)} \leqslant \e^{-t^2}$. Ainsi, la fonction $g(x, \,\cdot\,)$ est intégrable pour tout $x$ réel.

\item Les hypothèses de régularité sont aisément vérifiables. De plus,
\[
\module{\frac{\partial g}{\partial x}(x, t)} = t \e^{-t^2}
\]
qui est intégrable sur $\R$.

Ainsi, en appliquant le théorème de dérivation sous le signe intégral, la fonction $f$ est dérivable et
\[
f'(x) = \int_\R \i t \e^{-t^2 + \i t x} \d t.
\]

\item En utilisant une intégration par parties généralisée dont le crochet converge,
\begin{align*}
f'(x) &= \int_\R t \e^{-t^2} \i \e^{\i t x} \d t \\
&= \left[-\frac{1}{2} \e^{-t^2} \i \e^{\i t x}\right]_{-\infty}^{+\infty} - \int_\R \frac{x}{2} \e^{-t^2} \e^{\i t x} \d t \\
f'(x) &= -\frac{x}{2} f(x).
\end{align*}

\item L'ensemble des solutions de l'équation du différentielle du premier ordre $2 y' + x y = 0$ est
\[
\left\{x \mapsto \lambda \e^{-\frac{x^2}{4}},\, \lambda \in \R\right\}.
\]

D'après le calcul de l'intégrale de \textsc{Gauss}, $f(0) = \sqrt{\pi}$. Ainsi,
\[
f(x) = \sqrt{\pi}\, \e^{-\frac{x^2}{4}}.
\]
\end{enumerate}
\end{preuve}

\subsection{À trier}

\begin{exercice}
\begin{enumerate}
    \item Calculer $\int_{-\infty}^{+\infty} \e^{-x^2} \d x$. \\
    \emph{Indication :} On pourra d'abord calculer $\int_{R^2} \e^{-(x^2 + y^2)} \d x \d y$ en passant en coordonnées polaires. 
    \item \emph{Calcul de l'aire de la sphère unité de $\R^n$.} Soit $\mathscr{S}_{n-1} = \ens[\big]{(x_1, \dots, x_n) \in \R^n \tq \sum\limits_{i=1}^n x_i{}^2 = 1}$ la sphère unité de $\R^n$. On note $\mathscr{A}_{n-1}$ son aire. Calculer
    \[
    \int_{\R^n} \exp\left({-\sum\limits_{i=1}^n x_i{}^2}\right) \d x_1 \dots \d x_n
    \]
    en fonction de $\mathscr{A}_{n-1}$. En déduire l'expression de $\mathscr{A}_{n-1}$ en fonction de la fonction $\Gamma$ :
    \[
    \Gamma(s) \defeq \int_0^{+\infty} x^{s-1} \e^{-x} \d x.
    \]
    \item \emph{Calcul du volume de la boule unité de $\R^n$.} Soit $\mathscr{B}_n = \ens[\big]{(x_1, \dots, x_n) \in \R^n \tq \sum\limits_{i=1}^n x_i{}^2 \leqslant 1}$ la boule fermée de rayon $1$ dans $\R^n$. On note $\mathscr{V}_n$ son volume. Montrer que $\mathscr{V}_n = \frac{\mathscr{A}_{n-1}}{n}$. En déduire que:
    \[
    \mathscr{V}_n = \frac{\pi^{n/2}}{\Gamma \left( \frac{n}{2} + 1 \right)}.
    \]
    \item \emph{Application :} Que vaut l'aire de la sphère de rayon $R$ de $\R^2$ ? $\R^3$ ? Que vaut le volume de la boule de rayon $R$ de $\R$ ? $\R^2$ ? $\R^3$?
\end{enumerate}
\end{exercice}

\todoinline{On met cet exercice avec les polynômes orthogonaux ?}

%---------------

\begin{exercice}
Polynômes d'{Hermite}
{RMS 2017 154 - Autres écoles}
{TPE}
{16}
Soit $f : x \mapsto \e^{-x^2}$. On rappelle que $\int_{-\infty}^{+\infty} f(x) \d x = \sqrt{\pi}$.
\begin{enumerate}
\item Montrer qu'il existe un polynôme $P_n$ tel que $f^{(n)}(x) = f(x) P_n(x)$. Préciser le degré, la parité et le coefficient dominant de $P_n$.

\item Montrer l'existence puis calculer $\int_{-\infty}^{+\infty} f(x) P_n(x) P_m(x) \d x$.
\end{enumerate}
\end{exercice}

\begin{preuve}
\begin{enumerate}
\item On raisonne par récurrence en remarquant que $P_0 = 1$ et $P_{n+1} = P_n' - 2 X P_n$. Ainsi, le degré de $P_n$ est $n$ et son coefficient dominant $(-2)^n$ et $P_n$ a même parité que $n$, i.e. $P_n(-X) = (-1)^n P_n(X)$.

\item Soient $0 < m \leq n$. Les intégrales sont bien définies car ce sont des $o(1/x^2)$ en $\pm\infty$. Comme les crochets tendent vers $0$, en utilisant les fonctions $t \mapsto f^{(n-1)}(t)$ et $t \mapsto P_m(t)$ qui sont de classe $\mathscr{C}^1$, on remarque que
\[
\int_\R f P_n f P_m = \int_\R f^{(n)} P_m = -\int_\R f^{(n-1)} P_m'.
\]
Ainsi, en itérant ce procédé, comme $m \leq n$,
\[
\int_\R f P_n P_m = (-1)^m \int_\R f^{(n-m)} P_m^{(m)}
= (-1)^m (-2)^m \int_\R f^{(n-m)}.
\]
Ainsi,
\begin{itemize}
\item Si $n - m = 0$, i.e. $m = n$, alors $\int_\R f P_n^2 = m! 2^m \sqrt{\pi}$.
\item Si $n - m \neq 0$, alors $\int_\R f P_n P_m = \left[f^{(n-m-1)}\right]_{-\infty}^{+\infty} = 0$.
\end{itemize}
\end{enumerate}
\end{preuve}


%%%%%%%%%%%%%%%%%%%%%

\todoinline{Je supprimerais l'exercice suivant}

%---------------

\begin{exercice}
{X-ENS}
{16}%
Soient $f$ et $g$ les fonctions définies pour tout $x \in \R_+$ par $f(x) = \int_0^1 \frac{e^{-(t^2+1) x^2}}{1 + t^2} \d t$ et $g(x) = \int_0^x e^{-t^2} \d t$.
\begin{enumerate}
\item Calculer $f(0)$ puis $\lim_{x\to+\infty} f(x)$.

\item Montrer que $f$ est de classe $\mathscr{C}^1$ sur $\R_+$ et que, pour tout $x \in \R_+$, $-2 g'(x) g(x) = f'(x)$.

\item En déduire $I = \int_0^{+\infty} e^{-t^2} \d t$.

Soit $h$ une fonction continue par morceaux, décroissante sur $\R_+$ telle que $\int_0^{+\infty} h(t) \d t$ soit convergente et non nulle.

\item Montrer que $h$ est à valeurs positives.

Pour tout réel positif $t$ non nul, on pose $S(t) = \sum_{n=1}^{+\infty} h(n t)$.
\item Montrer que $S$ existe.

\item Déterminer un équivalent de $S(t)$ lorsque $t$ tend vers $0^+$.

\item Déterminer un équivalent de $\sum_{n=1}^{+\infty} x^{n^2}$ lors que $x$ tend vers $1^-$.
\end{enumerate}
\end{exercice}

\begin{preuve}
\begin{enumerate}
\item D'après la définition, $f(0) = \int_0^1 \frac{1}{1 + t^2} \d t = \arctan(1) = \frac{\pi}{4}$.

On remarque que l'intégrande est majorée par $t \mapsto \frac{1}{1+ t^2}$ qui est intégrable donc en appliquant le théorème de convergence dominée,
\[
\lim_{x\to+\infty} f(x) = 0.
\]

\item Sur $[a, b]$, on majore la dérivée par $b^2$ qui est intégrable sur $[0, 1]$. Ainsi, d'après le théorème de dérivation sous le signe intégral,
\begin{align*}
f'(x) &= - \int_0^1 2 x e^{-(t^2+1) x^2} \d t \\
&= - 2 x e^{-x^2} \int_0^1 e^{-(t x)^2} \d t \\
&= - 2 x e^{-x^2} \int_0^x e^{-t^2} \frac{\d t}{x} \\
&= - 2 g'(x) g(x).
\end{align*}

\item D'après la question précédente,
\[
f(x) - f(0) = - (g(x)^2 - g(0)^2).
\]
Ainsi,
\[
\frac{\pi}{4} - f(x) = g(x)^2.
\]
La fonction $g$ étant à valeurs positives, $I = \frac{\sqrt{\pi}}{2}$.

\item $h$ est décroissante sur $\R_+$. Elle admet donc une limite en $+\infty$. Si cette limite (dans $\bar{\R}$) est égale à $\ell < 0$, alors $h(t) \leq \frac{\ell}{2}$ pour $t$ assez grand et $\int_0^{+\infty} h(t) \d t$ diverge. On raisonne de même pour $\ell > 0$. Ainsi, $h$ tend vers $0$ en $+\infty$ et $h$ est à valeurs positives.

\medskip

{2ème méthode (si $h$ est continue).} En utilisant le théorème des accroissements finis, $H(n+1) - H(n) = h(c_n)$ et le membre de gauche tend vers $0$ donc $\ell = 0$.

\item D'après la décroissance de $h$,
\begin{align*}
\sum_{n=1}^N h(n t) &= \sum_{n=1}^N (n t - (n-1)t) \frac{1}{t} h(n t) \\
&\leq \frac{1}{t} \sum_{n=1}^N \int_{(n-1)t}^{nt} h(u) \d u \\
&\leq \frac{1}{t} \int_0^{Nt} h(u) \d u.
\end{align*}
Ainsi, comme $\int_0^{+\infty} h(t) \d t$ converge, alors d'après les théorèmes sur les séries à termes positifs, $S$ converge.

\item D'après la question précédente, en utilisant une minoration,
\[
J - c t \leq t S(t) \leq J.
\]
Ainsi, comme $J \neq 0$, alors
\[
S(t) \sim_{t\to0} \frac{J}{t}.
\]

\item En posant $h(t) = e^{-t^2}$, la fonction $h$ est bien continue, décroissante et d'intégrale sur $\R_+$ convergente. Ainsi, d'après la question précédente, pour $x \in ]0, 1[$,
\[
S(\sqrt{-\ln(x)}) = \sum_{n=1}^{+\infty} x^{n^2}.
\]
Ainsi, d'après la question précédente,
\[
\sum_{n=1}^{+\infty} x^{n^2} \sim_1 \frac{\sqrt{\pi}}{2 \sqrt{\abs{\ln(x)}}}.
\]
\end{enumerate}
\end{preuve}

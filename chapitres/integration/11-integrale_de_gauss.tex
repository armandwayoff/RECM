\section{Intégrale de \textsc{Gauss}}

\begin{prop}{}
    $$\int_{0}^{+\infty} \e^{-x^2} \d x = \frac{\sqrt{\pi}}{2}$$
\end{prop}

\begin{exercice}
    \marginnote[0cm]{Source : \cite{maths-france} Planche no 13. Suites et séries d’intégrales}
    \begin{enumerate}
        \item \textbf{Première méthode:} \say{ à la main }. \\ 
        Pour $n \in \Ne$, on pose
        $$
        f_n(x) \defeq
        \begin{cases}
            \left(1 - \frac{x}{n} \right)^n &\text{si } x \in [0, n] \\
            0 &\text{si } x \geqslant n
        \end{cases}.
        $$
        Pour tout réel positif $x$, on pose $f(x) = \e^{-x^2}$.
        \begin{enumerate}
            \item Montrer que pour tout réel positif $x$, 
            $$|f(x) - f_n(x)| \leqslant \frac{1}{n \e}.$$
            \item À l'aide de la suite $(f_n)_{n \in \Ne}$, calculer l'intégrale de \textsc{Gauss}.
        \end{enumerate}
        \item \textbf{Deuxième méthode:} \say{ avec le théorème de convergence dominée }. \\
        Pour $n \in \Ne$, on pose
        $$
        f_n(x) \defeq
        \begin{cases}
            \left(1 - \frac{x^2}{n} \right)^n &\text{si } x \in [0, \sqrt{n}] \\
            0 &\text{si } x > \sqrt{n}
        \end{cases}.
        $$
        \begin{enumerate}
            \item Montrer que la suite $(f_n)_{n \in \Ne}$ converge simplement sur $\Rp$ vers la fonction $f:x \mapsto \e^{-x^2}$.
            \item À l'aide de la convergence dominée, calculer l'intégrale de \textsc{Gauss}.
        \end{enumerate}
    \end{enumerate}
\end{exercice}

\begin{solution}
\end{solution}

\todoinline{Pour avoir une application, on peut regarder la remarque dans le chapitre sur les polynômes d'Hermite ou alors calculer la transformée de Fourier d'une gaussienne.\\
Faire un lien aussi avec la partie intégrales de Wallis qui propose une autre démonstration. Il y a un document de Lasserre qui traîne sur internet et qui recense une série de preuves. Je vais essayer de le retrouver pour voir s'il n'y a pas une autre idée intéressante.}
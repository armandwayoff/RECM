\todoinline{Pour moi il faut faire le lien quelque part avec la formule de Stirling.}

\begin{defi}{Intégrale de \textsc{Wallis}}
    $$\Wallis_n \defeq \int_{0}^{\frac{\pi}{2}} \sin(x)^n \d x = \int_{0}^{\frac{\pi}{2}} \cos(x)^n \d x$$
\end{defi}

\begin{prop}{} \labprop{prop_wallis}
    $$\Wallis_{2p} = \frac{\binom{2p}{p}}{2^{2p}}\frac{\pi}{2} \text{ et } \Wallis_{2p+1} = \frac{2^{2p} (p!)^2}{(2p+1)!}$$
    $$\Wallis_{n+1} \sim \Wallis_n \qquad \Wallis_n \sim \sqrt{\frac{\pi}{2n}}$$
    $$\Wallis_n \Wallis_{n+1} = \frac{\pi}{2(n+1)}$$
\end{prop}

\begin{preuve}
    Calculons $\Wallis_{n+2}$ en effectuant une intégration par parties. On pose $u(t) \defeq - \cos(t)$ et $v(t) \defeq \sin(t)^{n+1}$, toutes deux de classe $\mathscr{C}^1$ sur $\left[0, \frac{\pi}{2} \right]$. 
    \begin{align*}
        \Wallis_{n+2} &= \underbrace{\left[ -\cos(t) \sin(t)^{n+1} \right]_0^{\pi/2}}_{=0} + (n+1) \int_0^{\pi/2} \cos(t)^2 \sin(t)^n \d t \\
        &= (n+1) \int_0^{\pi/2} \big(1 - \sin(t)^2 \big) \sin(t)^n \d t \\
        &= (n+1) \Wallis_n - (n+1) \Wallis_{n+2} \\
        \text{soit } (n+2) \Wallis_{n+2} &= (n+1) \Wallis_n.
\end{align*}
Soit $p \in \N$. D'après la relation précédente, 
\begin{figure*}[h!]
\begin{multicols}{2}
\begin{align*}
    \Wallis_{2p} &= \frac{2p-1}{2p} \Wallis_{2p-2} \\
    &= \frac{2p-1}{2p} \times \frac{2p-3}{2p-2} \times \cdots \times \frac{1}{2} \times \underbrace{\Wallis_0}_{=\pi/2} \\
    &= \frac{\prod\limits_{k=1}^p (2k+1)}{\prod\limits_{k=1}^{p+1} (2k)} \frac{\pi}{2} \\
    &= \frac{\left[\prod\limits_{k=1}^p (2k+1) \right] \times \left[ \prod\limits_{k=1}^{p+1} (2k) \right]}{\left[\prod\limits_{k=1}^{p+1} (2k) \right]^2} \frac{\pi}{2} \\
    \Wallis_{2p} &= \frac{(2p)!}{2^{2p}(p!)^2} \frac{\pi}{2}.
\end{align*}
\begin{align*}
    \Wallis_{2p+1} &= \frac{2p}{2p+1} \Wallis_{2p-1} \\
    &= \frac{2p}{2p+1} \times \frac{2p-2}{2p-1} \times \cdots \times \frac{2}{3} \times \underbrace{\Wallis_1}_{=1} \\
    &= \frac{\prod\limits_{k=1}^p (2k)}{\prod\limits_{k=0}^p (2k+1)} \\
    &= \frac{\left[ \prod\limits_{k=1}^p (2k) \right]^2}{\left[ \prod\limits_{k=0}^p (2k+1) \right] \left[ \prod\limits_{k=1}^p (2k) \right]} \\
    \Wallis_{2p+1} &= \frac{2^{2p}(p!)^2}{(2p+1)!}.
\end{align*}
\end{multicols}
\end{figure*}
\end{preuve}

\subsection{Séries génératrices}

\todoinline{C'est rigolo ! Est-ce qu'on pourrait retrouver l'expression de $W_n$ à l'aide d'un produit de DSE ? Il y a aussi une application ici : https://math-os.com/coefficient-binomial-central/}

\textcolor{red}{Ajouter un texte d'introduction}

\begin{prop}{}
    \marginnote[0cm]{Source : \href{https://fr.wikipedia.org/wiki/Intégrale_de_Wallis}{Intégrale de \textsc{Wallis} -- \textsf{wikipedia.org}}}
    La série génératrice des termes pairs est 
    $$\sum_{p=0}^\infty \Wallis_{2p} x^{2p} = \frac{\pi}{2} \frac{1}{\sqrt{1-x^2}}.$$
    La série génératrice des termes impairs est 
    $$\sum_{p=0}^\infty \Wallis_{2p+1} x^{2p+1} = \frac{\arcsin x}{\sqrt{1-x^2}}.$$
\end{prop}

\begin{exercice}
    Soit $x \in ]0,1[$. Calculer $\sum\limits_{n=0}^\infty (-1)^n \Wallis_n$ puis $\sum\limits_{n=0}^\infty \Wallis_n x^n$.
\end{exercice}

\begin{solution}
    \marginnote[0cm]{Source : \href{http://exo7.emath.fr/ficpdf/fic00126.pdf}{Exercices de Jean-Louis \textsc{Rouget} (fic00126) -- \textsf{http://exo7.emath.fr}}}
    D'après \vrefprop{prop_wallis}, $\Wallis_n \sim \sqrt{\frac{\pi}{2n}}$ et la règle de \textsc{d'Alembert} fournit $R = 1$. Soit $x \in ]-1, 1[$. \\
    Pour tout $t \in \left[ 0, \frac{\pi}{2} \right]$ et tout entier naturel $n$, $|x^n \cos^n t| \leqslant |x|^n$. Comme la série numérique de terme général $|x|^n$ converge, la série de fonctions de terme général $t \mapsto x^n \cos^n t$ est normalement convergente et donc uniformément convergente sur le segment $\left[ 0, \frac{\pi}{2} \right]$. D'après le théorème d'intégration terme à terme sur un segment, 
    \begin{align*}
        \sum_{n=0}^{+ \infty} \Wallis_n x^n &= \sum_{n=0}^{+ \infty} \left[ x^n \int_0^{\pi/2} \cos^n t \d t \right] = \int_0^{\pi/2} \left( \sum_{n=0}^{+\infty} x^n \cos^n t \right) \d t \\
        &=\int_0^{\pi/2} \frac{1}{1 - x \cos t} \d t \\
        &= \int_0^1 \frac{1}{1 - x \frac{1-u^2}{1+u^2}} \frac{2}{1 + u^2} \d u \quad \text{en posant } u = \tan \frac{t}{2} \\
        &= 2 \int_0^1 \frac{1}{(1+x)u^2 + (1-x)} \d u \\
        &= 2 \times \frac{1}{1+x} \times \frac{1}{\sqrt{\frac{1-x}{1+x}}} \left[ \arctan \left( \frac{u}{\sqrt{\frac{1-x}{1+x}}} \right) \right]_0^1 \\
        \sum_{n=0}^{+ \infty} \Wallis_n x^n &= \frac{2}{\sqrt{1-x^2}} \arctan \sqrt{\frac{x+1}{x-1}}.
    \end{align*}
\end{solution}

\subsection{Calcul de l'intégrale de \textsc{Gauss}}
\marginnote[0cm]{Source : \href{https://fr.wikipedia.org/wiki/Intégrale_de_Wallis}{Intégrale de \textsc{Wallis} -- \textsf{wikipedia.org}}}
On peut aisément utiliser les intégrales de \textsc{Wallis} pour calculer l'intégrale de \text{Gauss}. \\
On utilise pour cela l'encadrement suivant, issu de la construction de la fonction exponentielle par la méthode d'\textsc{Euler}: pour tout entier $n > 0$ et tout réel $u \in ]-n, n[$, 
$$\left(1 + \frac{u}{n} \right)^n \leqslant \e^u \leqslant \left( 1 - \frac{u}{n} \right)^{-n}.$$
Posant alors $u = -x^2$, on obtient:
$$\int_0^{\sqrt{n}} \left( 1 - \frac{x^2}{n} \right)^n \d x \leqslant \int_0^{\sqrt{n}} \e^{-x^2} \d x \leqslant \int_0^{\sqrt{n}} \left( 1 + \frac{x^2}{n} \right)^{-n} \d x.$$
Or les intégrales d'encadrement sont liées aux intégrales de \textsc{Wallis}. Pour celle de gauche, il suffit de poser $x = \sqrt{n} \sin t$ ($t$ variant de $0$ à $\pi/2$). Quant à celle de droite, on peut poser $x = \sqrt{n} \tan t$ ($t$ variant de $0$ à $\pi/4$) puis majorer par l'intégrale de $0$ à $\pi/2$. On obtient ainsi:
$$\sqrt{n} \Wallis_{2n+1} \leqslant \int_0^{\sqrt{n}} \e^{-x^2} \d x \leqslant \sqrt{n} \Wallis_{2n-2}.$$
Par le théorème des gendarmes, on déduit alors de l'équivalent de $\Wallis_n$ que
$$\int_0^{+ \infty} \e^{-x^2} \d x = \frac{\sqrt{\pi}}{2}.$$

\subsection{Volume d'une boule en dimension \texorpdfstring{$n$}{n}}

\begin{exercice}
    \marginnote[0cm]{Source : \cite{fmaalouf}}
    Pour $n \in \Ne$ et $R \in \Rpe$ on désigne par $V_n(R)$ le volume de la boule de $\R^n$ de centre $O$ et de rayon $R$, 
    $$V_n(R) \defeq \idotsint_{x_1^2 + \cdots + x_n^2 \leqslant R^2} \d x_1 \cdots \d x_n.$$
    Montrer que pour tout $p \in \Ne$, 
    $$V_{2p}(R) = \frac{\pi^p R^{2p}}{p!}.$$
\end{exercice}

\todoinline{Ajouter une preuve.}

\todoarmand{Les exercices 4 et 5 de \url{http://exo7.emath.fr/ficpdf/fic00143.pdf} donnent les liens entre calcul de la surface de la sphère unité de $\R^n$, son volume, la fonction Gamma et les intégrales de Wallis.}

\subsection{\textsc{Grain de raisin}: Produit de \textsc{Wallis}}

\begin{prop}{Produit de \textsc{Wallis}}
    $$\prod_{n=1}^{\infty} \frac{4n^2}{4n^2-1} = \frac{\pi}{2}$$
\end{prop}

\begin{preuve}
    Puisque $\Wallis_{2n} \sim \Wallis_{2n+1}$, 
    $$\lim_{n \to +\infty} \frac{\Wallis_{2n+1}}{\Wallis_{2n} / \frac{\pi}{2}} = \frac{\pi}{2}.$$
    Or d'après le calcul des intégrales de \textsc{Wallis}:
    $$\frac{\Wallis_{2n+1}}{\Wallis_{2n} / \frac{\pi}{2}} = \frac{\frac{2^{2p}(p!)^2}{(2p+1)!}}{\frac{(2p)!}{2^{2p}(p!)^2}} = \prod_{k=1}^n \frac{4k^2}{4k^2-1}.$$
\end{preuve}
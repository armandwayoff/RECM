Voir \nameref{preuve_stirling} aussi la page \url{https://fr.wikipedia.org/wiki/Intégrale_de_Wallis} est très complète. 

\begin{defi}{Intégrale de \textsc{Wallis}}
    $$\Wallis_n \defeq \int_{0}^{\frac{\pi}{2}} \sin^n x \d x = \int_{0}^{\frac{\pi}{2}} \cos^n x \d x$$
\end{defi}

\begin{prop}{}
    $$\Wallis_{2p} = \frac{\binom{2p}{p}}{2^{2p}}\frac{\pi}{2} \text{ et } \Wallis_{2p+1} = \frac{2^{2p} (p!)^2}{(2p+1)!}$$
    $$\Wallis_{n+1} \sim \Wallis_n \qquad \Wallis_n \sim \sqrt{\frac{\pi}{2n}}$$
    $$\Wallis_n \Wallis_{n=1} = \frac{\pi}{2(n+1)}$$
\end{prop}

\underline{Produit de Wallis (1665):}
$$\prod_{n=1}^{\infty} \frac{4n^2}{4n^2-1} = \frac{\pi}{2}$$

\begin{exercice}
    Soit $x \in ]0,1[$. Calculer $\sum\limits_{n=0}^\infty (-1)^n \Wallis_n$ puis $\sum\limits_{n=0}^\infty x^n \Wallis_n$.
\end{exercice}
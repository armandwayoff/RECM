%---------------
\begin{exercice}%
Fonction Zêta de Riemann%

Rudin Ex.16 p.131

Pour tout réel $s > 1$, on pose $\zeta(s) = \sum_{n=1}^\infty \frac{1}{n^s}$.
\begin{itemize}
\item Montrer que $\zeta(s) = s \int_1^\infty \frac{\lfloor x\rfloor}{x^{s + 1}} \d x$.
% \indic{Indication : Comparer l'intégrale sur le segment $[1,N]$ et la somme partielle.}

\item Montrer que $\zeta(s) = \frac{s}{s - 1} - s \int_1^\infty \frac{x - \lfloor x\rfloor}{x^{s+1}} \d x$.

\item Montrer que cette dernière intégrale converge pour tout $s > 0$.
\end{itemize}
\end{exercice}


\begin{solution}
\begin{itemize}
\item En utilisant la relation de Chasles,
\begin{align*}
s \int_1^N \frac{\lfloor x\rfloor}{x^{s+1}} \d x &= \sum_{n=1}^{N-1} s \int_n^{n+1} \frac{\lfloor x\rfloor}{x^{s+1}} \d x \\
&= \sum_{n=1}^{N-1} n \left[-\frac{1}{(n+1)^s} + \frac{1}{n^s}\right] \\
&= \sum_{n=1}^{N-1} \frac{1}{n^{s-1}} - \sum_{n=1}^{N-1} \frac{1}{(n+1)^{s-1}} + \sum_{n=1}^{N-1} \frac{1}{(n+1)^s} \\
&= 1 - \frac{1}{N^{s-1}} + \sum_{n=2}^N \frac{1}{n^s} \\
&\to \zeta(s)
\end{align*}

\item En utilisant la relation précédente,
\begin{align*}
s \int_1^M \frac{x - \lfloor x\rfloor}{x^{s+1}} \d x &= s \int_1^M x^{-s} \d x - s \int_1^M \frac{\lfloor x\rfloor}{x^{s+1}} \d x \\
&= \frac{s}{s-1} \left[1 - \frac{1}{M^{s-1}}\right] - s \int_1^M \frac{\lfloor x\rfloor}{x^{s+1}} \d x \\
&\to \frac{s}{s-1} - \zeta(s)
\end{align*}
On obtient ainsi la convergence de l'intégrale.

\item D'après la définition de la fonction partie entière,
\[
\abs{\frac{x-\lfloor x\rfloor}{x^{s+1}}} \leq \frac{1}{x^{s+1}}
\]
Ainsi, cette intégrale est convergente dès que $s > 0$. Cette formule permet d'étendre la définition de la fonction $\zeta$ sur $\R_+^*$.
\end{itemize}
\end{solution}
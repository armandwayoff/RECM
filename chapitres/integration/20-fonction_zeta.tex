\section{Prolongement de la zêta de \nom{Riemann} par une intégrale}

\begin{comment}
\subsection{Fonction Gamma d'\textsc{Euler}}\label{prolongementFonctionGamma}

On rappelle la définition de la fonction Gamma d'\textsc{Euler} (cf. \ref{secinteuleriennes}).
\begin{defi}[Fonction Gamma d'\textsc{Euler}]
    La \emph{fonction Gamma d'\textsc{Euler}} est définie par: 
    $$\Gamma(x) \defeq \int_{0}^{+\infty} t^{x-1} \e^{-t} \d t.$$
\end{defi}

\begin{prop}
    \begin{itemize}
        \item La fonction $\Gamma$ est définie si et seulement si $x>0$.
        \item Pour tout $x > 0$, $\Gamma(x+1) = x\Gamma(x)$. \\
        En particulier, pour tout $n \in \N$, $\Gamma(n+1) = n!$. 
    \end{itemize}
\end{prop}

\begin{demo}
    \begin{itemize}
        \item La fonction $f_x:t \mapsto t^{x-1} \e^{-t}$ est continue sur $]0, + \infty[$ comme produit de fonctions qui y sont continues. La fonction $f_x$ est donc intégrable sur tout segment de $]0, +\infty[$. Il reste à étudier son intégrabilité en $0$ et en $+ \infty$:
        \begin{itemize}
            \item[$\triangleright$] En $+\infty:$ par croissances comparées, $f_x(t) = o_{+\infty} \left(\frac{1}{t^2} \right)$. D'après le théorème de comparaison des fonctions à termes positifs, $f_x$ est intégrable au voisinage de $+\infty$.
            \item[$\triangleright$] En $0$: $f_x(t) \sim_0 t^{x-1}$ qui est intégrable d'après \textsc{Riemann} si et seulement si $1-x < 1$ i.e. si et seulement si $x > 0$.
        \end{itemize}

        \item Soit $x > 0$. Calculons $\Gamma(x+1)$ en effectuant une intégration par parties. Posons $u:t \mapsto \e^{-t}$ et $v:t \mapsto t^x$, toutes deux de classe $\mathscr{C}^1$ sur $\Rp$. Vérifions la convergence du \emph{crochet}:
        \begin{align*}
            \text{par croissances comparées} \quad & \lim_{t \to +\infty} u(t) v(t) = 0, \\
            \text{ comme } x > 0 \quad & \lim_{t \to 0} u(t) v(t) = 0.
        \end{align*}
        Ainsi, d'après le théorème d'intégration par parties généralisées, 
        $$\int_{0}^{+\infty} t^{x-1} \e^{-t} \d t = \underbrace{0}_{\mathclap{\text{crochet}}} - \int_{0}^{+\infty} xt^{x-1} (-\e^{-t}) \d t.$$
        soit 
        $$\Gamma(x+1) = x \Gamma(x).$$
        En particulier, $\Gamma(1) = 1$ et pour tout $n \in \Ne, \Gamma(n+1) = n \Gamma(n)$. Donc
        $$\forall n \in \Ne, \Gamma(n+1) = n!$$
    \end{itemize}
\end{demo}
\end{comment}

% \subsection{Fonction zêta de \textsc{Riemann}}

\label{subsec:fonctionZeta}

\begin{theo}
\marginpar[0cm]{\cite{rudin2006}, Ex.16 p.131}
Pour tout $s > 1$,
\[
\zeta(s)
= \sum_{n=1}^{+\infty} \frac{1}{n^s}
= \frac{s}{s - 1} - s \int_1^\infty \frac{x - \lfloor x\rfloor}{x^{s+1}} \d x,
\]
où $I(s) = \frac{s}{s - 1} - s \int_1^\infty \frac{x - \lfloor x\rfloor}{x^{s+1}} \d x$ est définie pour tout $s > 0$.

La fonction $I$ prolonge ainsi $\zeta$ sur le segment $\interof{0}{1}$.
\end{theo}

%---------------
\begin{exercice}
\begin{questions}
\item Montrer que pour tout $N \in \Ne$,
\[
s \int_1^N \frac{\lfloor x\rfloor}{x^{s+1}} \d x
= 1 - \frac{1}{N^{s-1}} + \sum_{n=2}^N \frac{1}{n^s}.
\]

\item En déduire que que, pour tout $s > 1$, $\zeta(s) = s \int_1^\infty \frac{\lfloor x\rfloor}{x^{s + 1}} \d x$.
% \indic{Indication : Comparer l'intégrale sur le segment $[1,N]$ et la somme partielle.}

\item Montrer que, pour tout $s > 1$, $\zeta(s) = \frac{s}{s - 1} - s \int_1^\infty \frac{x - \lfloor x\rfloor}{x^{s+1}} \d x$.

\item Montrer que cette dernière intégrale converge pour tout $s > 0$.
\end{questions}
\end{exercice}

\begin{solution}
\begin{reponses}
\item On utilise la relation de \nom{Chasles} pour découper l'intervalle en morceaux sur lesquels la fonction partie entière est constante :
\begin{align*}
s \int_1^N \frac{\lfloor x\rfloor}{x^{s+1}} \d x &= \sum_{n=1}^{N-1} s \int_n^{n+1} \frac{\lfloor x\rfloor}{x^{s+1}} \d x\\
&= \sum_{n=1}^{N-1} n \left[-\frac{1}{(n+1)^s} + \frac{1}{n^s}\right] \\
&= \sum_{n=1}^{N-1} \left(\frac{1}{n^{s-1}} - \frac{n + 1 - 1}{(n+1)^{s-1}}\right) \\
&= \sum_{n=1}^{N-1} \frac{1}{n^{s-1}} - \sum_{n=1}^{N-1} \frac{1}{(n+1)^{s-1}} + \sum_{n=1}^{N-1} \frac{1}{(n+1)^s} \\
&= 1 - \frac{1}{N^{s-1}} + \sum_{n=2}^N \frac{1}{n^s}.
\end{align*}

\item Dans la question précédente, le membre de droite converge vers $\zeta(s)$ lorsque $N \to +\infty$. Ainsi,
\[
\lim_{N\to+\infty} s \int_1^N \frac{\lfloor x\rfloor}{x^{s + 1}} \d x = \zeta(s).
\]

Comme $x \mapsto \frac{\lfloor x\rfloor}{x^{s + 1}}$ est à valeurs positives, alors $y \mapsto \displaystyle\int_1^y \frac{\lfloor x\rfloor}{x^{s + 1}} \d x$ est croissante. D'après le point précédent, on en déduit la convergence de l'intégrale et
\[
\int_1^{+\infty} \frac{\lfloor x\rfloor}{x^{s + 1}} \d x
= \lim_{N\to+\infty} \int_1^N \frac{\lfloor x\rfloor}{x^{s + 1}} \d x.
\]

\item En utilisant la relation précédente,
\begin{align*}
s \int_1^M \frac{x - \lfloor x\rfloor}{x^{s+1}} \d x &= s \int_1^M x^{-s} \d x - s \int_1^M \frac{\lfloor x\rfloor}{x^{s+1}} \d x \\
&= \frac{s}{s-1} \left[1 - \frac{1}{M^{s-1}}\right] - s \int_1^M \frac{\lfloor x\rfloor}{x^{s+1}} \d x.
\end{align*}
Comme le membre de droite converge vers $\frac{s}{s-1} - \zeta(s)$, alors l'intégrale converge et on obtient la relation demandée.

\item D'après la définition de la fonction partie entière,
\[
\abs{\frac{x-\lfloor x\rfloor}{x^{s+1}}} \leq \frac{1}{x^{s+1}}
\]
Ainsi, cette intégrale est convergente dès que $s > 0$.
\end{reponses}
\end{solution}
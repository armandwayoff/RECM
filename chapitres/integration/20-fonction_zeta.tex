\section{Fonction Zeta de Riemann}

\todoinline{Pourrait être mis dans une section : prolongement de fonctions par des intégrales ; au même titre que $\Gamma$ prolonge la factorielle. Il faudrait peut être mettre un mot sur l'unicité du prolongement analytique.}

\begin{prop}
Pour tout $s > 1$, on rappelle que $\zeta(s) = \sum_{n=1}^{+\infty} \frac{1}{n^s}$. Pour tout $s > 0$, posons $I(s) = \frac{s}{s - 1} - s \int_1^\infty \frac{x - \lfloor x\rfloor}{x^{s+1}} \d x$. Alors,
\[
\forall\, s > 1,\, \zeta(s) = I(s).
\]
La fonction $I$ prolonge ainsi $\zeta$ sur le segment $]0, 1]$.
\end{prop}

%---------------
\begin{exercice}%
Fonction Zêta de Riemann%

Source : Rudin Ex.16 p.131

Pour tout réel $s > 1$, on pose $\zeta(s) = \sum_{n=1}^\infty \frac{1}{n^s}$.
\begin{itemize}
\item Montrer que $\zeta(s) = s \int_1^\infty \frac{\lfloor x\rfloor}{x^{s + 1}} \d x$.
% \indic{Indication : Comparer l'intégrale sur le segment $[1,N]$ et la somme partielle.}

\item Montrer que $\zeta(s) = \frac{s}{s - 1} - s \int_1^\infty \frac{x - \lfloor x\rfloor}{x^{s+1}} \d x$.

\item Montrer que cette dernière intégrale converge pour tout $s > 0$.
\end{itemize}
\end{exercice}

\begin{solution}
\begin{itemize}
\item Si $M$ est un réel, alors
\begin{align*}
s \int_1^M \frac{\lfloor x\rfloor}{x^{s + 1}} \d x
&= s \int_1^{\lfloor M\rfloor} \frac{\lfloor x\rfloor}{x^{s + 1}} \d x + s \int_{\lfloor M\rfloor}^M \frac{\lfloor x\rfloor}{x^{s + 1}} \d x.
\end{align*}

Or, le second membre vaut
\[
s \lfloor M \rfloor \int_{\lfloor M\rfloor}^M \frac{1}{x^{s + 1}} \d x
= \frac{1}{M^s} - \frac{1}{\lfloor M \rfloor^s}.
\]

Il tend donc vers $0$ lorsque $M \to +\infty$. Pour étudier la convergence de l'intégrale on peut donc se limiter à l'étudier pour une suite d'entiers.

\item On utilise la relation de Chasles pour découper l'intervalle en morceaux sur lesquels la fonction partie entière est constante :
\begin{align*}
s \int_1^N \frac{\lfloor x\rfloor}{x^{s+1}} \d x &= \sum_{n=1}^{N-1} s \int_n^{n+1} \frac{\lfloor x\rfloor}{x^{s+1}} \d x\\
&= \sum_{n=1}^{N-1} n \left[-\frac{1}{(n+1)^s} + \frac{1}{n^s}\right] \\
&= \sum_{n=1}^{N-1} \left(\frac{1}{n^{s-1}} - \frac{n + 1 - 1}{(n+1)^{s-1}}\right) \\
&= \sum_{n=1}^{N-1} \frac{1}{n^{s-1}} - \sum_{n=1}^{N-1} \frac{1}{(n+1)^{s-1}} + \sum_{n=1}^{N-1} \frac{1}{(n+1)^s} \\
&= 1 - \frac{1}{N^{s-1}} + \sum_{n=2}^N \frac{1}{n^s}.
\end{align*}
Comme le membre de droite converge vers $\zeta(s)$ lorsque $N \to +\infty$, alors l'intégrale est convergente et on obtient la relation indiquée.

\item En utilisant la relation précédente,
\begin{align*}
s \int_1^M \frac{x - \lfloor x\rfloor}{x^{s+1}} \d x &= s \int_1^M x^{-s} \d x - s \int_1^M \frac{\lfloor x\rfloor}{x^{s+1}} \d x \\
&= \frac{s}{s-1} \left[1 - \frac{1}{M^{s-1}}\right] - s \int_1^M \frac{\lfloor x\rfloor}{x^{s+1}} \d x.
\end{align*}
Comme le membre de droite converge vers $\frac{s}{s-1} - \zeta(s)$, alors l'intégrale converge et on obtient la relation demandée.

\item D'après la définition de la fonction partie entière,
\[
\abs{\frac{x-\lfloor x\rfloor}{x^{s+1}}} \leq \frac{1}{x^{s+1}}
\]
Ainsi, cette intégrale est convergente dès que $s > 0$.
\end{itemize}
\end{solution}
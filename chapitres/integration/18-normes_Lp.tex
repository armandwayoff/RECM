\section{Normes $L^p$}
\ref{normes_lp_et_inegalites}

\todoinline{On peut aussi montrer l'inclusion des $L^p$ lorsqu'on intègre sur un segment, c'est ensuite utile dans la théorie de Lebesgue, dans le cadre des probabilités. Il y a une réciproque pour savoir quand $L^p \subset L^q$. J'ai des notes là-dessus, mais faut voir si c'est faisable niveau prépa.}


\begin{definition}
Soient $p \in \N$ et $I$ une intervalle de $\R$. Une fonction est de classe $L^p$ sur $I$ si $f^p$ est intégrable. On note alors
\[
\norm{f}_{L^p} = \left(\int_I \module{f}^p\right)^{1/p}.
\]
On note $L^p(I)$ l'ensemble des fonctions de classes $L^p$ sur $I$.
\end{definition}

\todoinline{mettre une remarque pour indiquer que, dans le cadre plus général de la théorie de l'intégrale de Lebesgue, on ne demande plus le caractère continu ?}

%-----------
\subsection{Limite $p \to +\infty$}

\todoinline{Je pense qu'il faut mettre une illustration avec des graphes de $f(t)^n$ et de l'aire associée où on verrait bien que le max prend le dessus sur les autres points.}


% \subsubsection{Étude de la suite de terme général \texorpdfstring{$\left( \frac{1}{b-a} \int_{a}^{b} f(x)^n \d x \right)^{1/n}$}{égal à une intégrale}}

\begin{theo}{}
Soit $f$ une fonction continue sur le segment $[a, b]$. Alors,
\[
\lim_{p\to+\infty} \norm{f}_{L^p} = \norm{f}_\infty.
\]
\end{theo}

\todoinline{Ajouter $\norm{\cdot}_\infty$ à un index des notations.}

% \begin{exercice}
    % \marginnote[0cm]{Source : \cite{acamanes} \href{https://acamanes.github.io/psi/psi_doc/exos_e01.pdf}{(Exercice 9. TD I)}}
    % Soit $f$ une fonction supposée continue et positive sur $\interff{a}{b}$. Étudier la suite de terme général 
    % $$u_n \defeq \left( \frac{1}{b-a} \int_{a}^{b} f(x)^n \d x \right)^{1/n}.$$
 % \end{exercice}


\begin{exercice}
Soit $f$ une fonction continue sur le segment $[a, b]$.
\begin{enumerate}
\item Montrer qu'il existe un réel $c_p$ tel que $\norm{f}_{L^p} \leq c_p \norm{f}_\infty$.

\item Soit $\epsilon > 0$. Montrer qu'il existe un segment $[c, d] \subset [a, b]$ tel que
\[
\forall\, x \in [c, d],\, f(x) \geq \norm{f}_{L^\infty} - \epsilon.
\]

\item En déduire qu'il existe un réel $d_p$ tel que
\[
d_p (\norm{f}_\infty - \epsilon) \leq \norm{f}_{L^p}.
\]

\item Conclure.
\end{enumerate}
\end{exercice}

\begin{preuve}
la fonction $f$ est continue sur un segment donc est en particuler, elle y est bornée. Ainsi, $\norm{f}_\infty$ est bien défini.

\begin{enumerate}
\item Comme, pour tout $t \in [a, b]$, $\module{f(t)} \leq \norm{f}_\infty$, la croissance de l'intégrale assure que
\[
\int_a^b \module{f(t)}^p \d t \leq \int_a^b (\norm{f}_\infty)^p \d t.
\]
On obtient ainsi la majoration demandée, avec $c_p = (b - a)^{1/p}$.

\item Comme $f$ est continue, elle atteint ses bornes sur $[a, b]$ et il existe un réel $x_0$ tel que $f(x_0) = \norm{f}_\infty$.

Toujours comme $f$ est continue, il existe un réel $\eta$ tel uqe pour tout $x \in [a, b]$,
\[
\module{x - x_0} \leq \eta \Rightarrow \module{f(x) - f(x_0)} \leq \epsilon.
\]

En particulier, pour $x \in [x_0-\eta, x_0+\eta] \cap [a, b]$,
\[
f(x) \geq f(x_0) - \epsilon = \norm{f}_\infty - \epsilon.
\]
On note $[c, d] = [x_0-\eta, x_0+\eta] \cap [a, b]$.

\item En utilisant la croissance de l'intégrale, on obtient ainsi
\[
(d - c)^{1/p} (\norm{f}_\infty - \epsilon) \leq \norm{f}_{L^p}.
\]

\item D'après les questions précédentes, pour tout $p \in \N$ et pour tout $\epsilon > 0$,
\[
(d - c)^{1/p} (\norm{f}_\infty - \epsilon) \leq \norm{f}_{L^p} \leq (b - a)^{1/p} \norm{f}_\infty.
\]

Comme cette propriété est vraie pour tout $\epsilon > 0$, alors

\[
(d - c)^{1/p} \norm{f}_\infty \leq \norm{f}_{L^p} \leq (b - a)^{1/p} \norm{f}_\infty.
\]

Enfin, comme $\lim_{p\to+\infty} (d - c)^{1/p} = \lim_{p\to+\infty} (b - a)^{1/p} = 1$, le théorème d'encadrement permet de conclure.

% \item \textbf{Minoration:} soit $\varepsilon > 0$, soit $x_0$ tel que $f(x_0) = M$. Comme $f$ est continue en $x_0$, il existe $[c, d] \subset [a, b]$ tel que $x_0 \in [c, d]$ et pour tout $x \in [c, d]$, $f(x) \geqslant M - \varepsilon$ \emph{(un dessin permet de bien comprendre la stratégie)}.\\
        % On peut ensuite montrer que $u_n \geqslant \left(\frac{d-c}{b-a} \right)^{1/n}(M-\varepsilon) \xrightarrow[n \to + \infty]{} M-\varepsilon$.
        % \item Finalement, $u_n \displaystyle \longrightarrow M = \max_{[ a, b ]} f = \Ninf{f}$.
    % \end{itemize}
\end{enumerate}
\end{preuve}

\todoinline{Je ne parviens pas à compiler la figure suivante}


% \begin{figure}
    % \centering
    
% \begin{tikzpicture}

% \tikzset{>=latex} % for LaTeX arrow head

% % \def\tick#1#2{\draw[thick] (#1)++(#2:0.12) --++ (#2-180:0.24)}
% \def\N{100} % number of samples

  % \def\xmax{3.5}
  % \def\ymax{3.5}
  % \def\xeps{3}
  % \def\yeps{1.8}
  % \coordinate (O) at (0,0);
  % \coordinate (C) at (0.927575,0);
  % \coordinate (A) at (0.5,0);
  % \coordinate (B) at (2.796001,0);
  % \coordinate (D) at (2.0700,0);
  % \coordinate (X0) at (1.63694,0);
  % \def\yMAX{2.587020}

  % % AXIS
  % \draw[->,thick]
    % (-0.1*\ymax,0) -- (\xmax,0) node[below] {$x$};
  % \draw[->,thick]
  % % (0,-0.1*\ymax) -- (0,\ymax) node[below left] {$\textcolor{red}{f}(x)$};
  % \langle{P}\rangle
  
  % % PLOT
  % % \draw[xline,red,samples=\N,smooth,variable=\x,domain=-0.1:0.94*\xmax,thick]
    % plot(\x,{3/(1+(0.36*\x*\x-1)^2) -\x/4});
    
  % \draw[dashed,thin] (X0) --++ (0,\yMAX);
  % \draw[dashed,thin] (C) --++ (0,\yeps);
  % \draw[dashed,thin] (D) --++ (0,\yeps);
  
  % % \draw[thin] (0,\yMAX) -- (\xmax,\yMAX) node[left] at (0,\yMAX) {$M$};
  % \draw[thin] (0,\yeps) -- (\xmax,\yeps);
  
  % \draw[<->] (\xeps,\yeps) -- (\xeps,\yMAX)
    % node[midway,scale=0.9] {\contour{white}{$\varepsilon$}};
    
  % \tick{X0}{90} node[below] {$x_0$};
  % \tick{A}{90} node[below] {$a$};
  % \tick{B}{90} node[below] {$b$};
  % \tick{C}{90} node[below] {$c$};
  % \tick{D}{90} node[below] {$d$};
  
% \end{tikzpicture}

    % \caption{Illustration à finir}
% \end{figure}

\subsection{Limite $p \to 0$}

\begin{theo}{}
Soit $f \in \mathscr{C}([0,1],\R^*)$. Alors,
\[
\lim_{p\to0} \norm{f}_{L^p} = \exp\left\{\int_0^1 \ln(f(x)) \d x\right\}.
\]
\end{theo}

%---------------

\begin{exercice}%
Soit $f \in \mathscr{C}([0,1],\R^\ast)$. On suppose que $f$ est à valeurs positives et on pose $I(y) = \int_0^1 f(x)^y \d x$.
\begin{enumerate}
\item Soit $g$ une fonction définie sur un voisinage de $0$ à valeurs dans $\R_+^\ast$, dérivable en $0$, vérifiant $g(0) = 1$. Déterminer $\lim_{y\to0} g(y)^{1/y}$.

\item Montrer que $I$ est une fonction dérivable sur $\R_+$.

\item En déduire que
\[
\lim_{y\to0} \left(\int_0^1 f(x)^y \d x\right)^{1/y} = \exp\left\{\int_0^1 \ln(f(x)) \d x\right\}.
\]
\end{enumerate}
\end{exercice}


\begin{preuve}
\begin{enumerate}
\item Comme $g$ est dérivable, d'après le théorème de Taylor-Young, $g(y) = 1 + y g'(0) + o(y)$. Ainsi,
\begin{align*}
g(y)^{1/y} &= \exp\left\{\frac{1}{y} \ln(g(0) + y g'(0) + o(y))\right\} \\
&= \exp\left\{g'(0) + o(1)\right\} \\
&\to \e^{g'(0)}.
\end{align*}

\item En posant $F: (x, y) \mapsto f(x)^y$, alors
\[
\abs{\frac{\partial F}{\partial y}(x, y)} = \abs{\ln(f(x)) f(x)^y}.
\]

La fonction $\ln \circ f$ est continue sur le segment $[0, 1]$ donc elle est bornée par une constante $M$. Ainsi,
\begin{align*}
\module{F(x, y)} &\leq \e^{a M},\\
\module{\frac{\partial F}{\partial y}(x, y)} &\leq M \cdot \e^{a M}.
\end{align*}

D'après le théorème de dérivation sous le signe intégral, la fonction $I$ est dérivable sur $[0, 1[$ et
\begin{align*}
I'(y) &= \int_0^1 \ln(f(x)) f(x)^y \d x \\
I'(0) &= \int_0^1 \ln(f(x)) \d x
\end{align*}

\item Finalement,
\[
\lim_{y\to0} \left(\int_0^1 f(x)^y \d x\right)^{1/y} = \exp\left\{\int_0^1 \ln(f(x)) \d x\right\}.
\]
\end{enumerate}
\end{preuve}

\url{https://math.stackexchange.com/questions/2351581/convergence-question-about-lp-norm-when-p-tends-to-zero}


%-----------
\subsection{Cas $p \geq 1$}

Lorsque l’exposant $p$ satisfait $0 < p < 1$, on constate qu’une inégalité triangulaire ne peut pas être satisfaite, ce qui justifie, pour bénéficier d’une structure naturelle
d’espace vectoriel, de se restreindre à supposer $1 < p < +\infty$.

\begin{theo}{Inégalité de Minkowski}
Soit $p \geq 1$. Alors, si $f$ et $g$ sont deux fonctions appartenant à $L^p(I)$, alors
\[
\norm{f + g}_{L^p} \leq \norm{f}_{L^p} + \norm{g}_{L^p}.
\]
Ainsi, $L^p(I)$ est un espace vectoriel et $\norm{\cdot}_{L^p}$ est une norme.
\end{theo}

\begin{remarque}
Les cas $p = 1$ et $p = +\infty$ sont aisément vérifiables. On se limite par la suite au cas $1 < p < +\infty$.
\end{remarque}

Cette inégalité repose sur le théorème suivant :
\begin{theo}{Inégalité de Hölder}
Soient $1 < p < +\infty$, $q$ tel que $\frac{1}{p} + \frac{1}{q} = 1$ et $f$ et $g$ deux fonctions à valeurs strictement positives. Si $f$ appartient à $L^p(I)$ et $g \in L^p(I)$, alors $f g$ appartient à $L^1(I)$ et
\[
\int_I f g \leq \norm{f}_{L^p} \norm{g}_{L^q}.
\]
\end{theo}

\begin{remarque}
Lorsque $p = 2$, alors $q = 2$ et on retrouve l'inégalité de Cauchy-Schwarz.
\end{remarque}

\begin{exercice}
Les cas où $\norm{f}_{L^p} = 0$ ou $\norm{g}_{L^q} = 0$ sont triviaux. On se limite donc au cas où ces quantités sont non nulles et on pose
\[
F = \frac{f}{\norm{f}_{L^p}}
\text{ et }
G = \frac{g}{\norm{g}_{L^q}}.
\]
Soit $x \in I$.
\begin{enumerate}
\item Montrer qu'il existe deux réels $s$ et $t$ tels que $F(x) = \e^{t/p}$ et $G(x) = \e^{s/q}$.

\item Montrer que $\e^{\frac{s}{p} + \frac{t}{q}} \leq \frac{1}{p} \e^s + \frac{1}{q} \e^t$.

\item En déduire l'inégalité de Hölder.
\end{enumerate}
\end{exercice}

\begin{preuve}
\begin{enumerate}
\item Pour tout $x$ réel, $F(x)$ et $G(x)$ sont deux réels strictement positifs, on pose $t = p \ln(F(x))$ et $s = q \ln(G(x))$.

\item Rappelons que $\frac{1}{p} + \frac{1}{q} = 1$. Pour tous $s$ et $t$ réels, comme la fonction exponentielle est convexe, d'après l'inégalité de Jensen,
\[
\e^{\frac{s}{p} + \frac{t}{q}} \leq \frac{1}{p} \e^s + \frac{1}{q} \e^t.
\]

\item Alors,
\begin{align*}
F(x) G(x) &\leq \frac{1}{p} F(x)^p + \frac{1}{q} G(x)^q
\end{align*}
Comme $F^p$ et $G^q$ sont intégrables, alors $F G$ est intégrable et
\[
\int_I F G \leq \frac{1}{p} + \frac{1}{q} = 1.
\]
La linéarité de l'intégrale permet alors de conclure.
\end{enumerate}
\end{preuve}

\begin{exercice}
Soit $f$ et $g$ deux fonctions de classe $L^p$ sur $I$.
\begin{enumerate}
\item Montrer que $\module{f + g}^p \leq 2^p \module{f}^p + 2^p \module{g}^p$.

\item En déduire que $(f + g) \in L^p(I)$.

\item En décomposant $(f + g)^p = f (f + g)^{p-1} + g (f + g)^{p-1}$, en déduire l'inégalité de Minkowski.
\end{enumerate}
\end{exercice}

\begin{preuve}
\begin{enumerate}
\item Pour tout $x \in I$,
\begin{itemize}
\item soit $\module{f(x)} \leq \module{g(x)}$ et $\module{f(x) + g(x)}^p \leq 2^p \module{g(x)}^p$,
\item soit $\module{g(x)} \leq \module{f(x)}$ et $\module{f(x) + g(x)}^p \leq 2^p \module{f(x)}^p$.
\end{itemize}
Ainsi, dans tous les cas,
\[
\module{f + g}^p \leq 2^p \module{f}^p + 2^p \module{g}^p.
\]

\item D'après la question précédente et le théorème de comparaison des intégrales, $f + g \in L^p(I)$.

\item En utilisant la décomposition proposée, on remarque que
\begin{align*}
\left(\module{f + g}^{p-1}\right)^q
&= \module{f + g}^{q(p-1)}
= \module{f + g}^p.
\end{align*}
Ainsi, $\module{f + g}^{p-1} \in L^q(I)$.

Ainsi, en appliquant deux fois l'inégalité d'Hölder,
\begin{align*}
\norme{f + g}_{L^p}^p
&= \norme{|f| \cdot |f + g|^{p-1}}_{L^1}
+ \norme{|g| \cdot |f + g|^{p-1}}_{L^1}\\
&\leq \norm{f}_{L^p}^p \norm{f + g}_{L^p}^{p/q}
+ \norm{g}_{L^p}^p \norm{f + g}_{L^p}{p/q}\\
&\leq \norm{f + g}_{L^p}^{p/q} \left(\norm{f}_{L^p} + \norm{f}_{L^q}\right).
\end{align*}

On conclut en simplifiant par $\norm{f + g}_{L^p}^{p/q}$.
\end{enumerate}
\end{preuve}

Exercice 2 de \url{https://www.imo.universite-paris-saclay.fr/~joel.merker/Enseignement/Integration/l-p-espaces.pdf}
% \begin{exercice}
    % On considère les espaces $L^p(\R^d)$ pour  $0 < p < +\infty$. Montrer que si l'on a 
    % \[
    % \norm{f + g}_{L^p} \leqslant  \norm{f}_{L^p} + \norm{g}_{L^p}
    % \]
    % pour toutes fonctions $f, g \in L^p(\R^d )$, alors nécessairement $p \geqslant 1$.
% \end{exercice}

%-----------
\subsection{Inclusions entre les $L^p$}

\begin{theo}{}
Si $I$ est un intervalle borné, non note $|I|$ sa longueur. Pour tout $p < q$, $L^q(I) \subset L^p(I)$ et $\|f\|_p \leq \module{I}^{\frac{1}{p} - \frac{1}{q}} \norm{f}_{L^q}$.
    % Si $\Omega$ est de mesure $\module{\Omega}$ finie, alors pour $p<q$, $L^q(\Omega) \subset L^p(\Omega)$ et pour tout $f \in L^q(\Omega)$, $\norm{f}_{L^p} \leqslant \module{\Omega}^{\frac{1}{p} - \frac{1}{q}} \norm{f}_{L^q}$.
\end{theo}

\begin{preuve}
Comme $I$ est borné, alors la fonction constante égale à $1$ appartient à tous les espaces $L^q$. Ainsi, d'après l'inégalité de Hölder,
\begin{align*}
\norm{f}_{L^p}^p
&\leq \left(\int_I \module{f}^q\right)^{p/q} \left(\int_I 1\right)^{1-p/q}\\
&\leq \norm{f}_{L^q}^p \cdot \module{I}^{1-p/q}.
\end{align*}
\end{preuve}

\begin{remarque}
En utilisant des intégrales de Riemann, on montre que ces inclusions sont fausses dès que $I$ n'est pas borné.
\end{remarque}
\todoinline{Expliciter la remarque précédente ?}
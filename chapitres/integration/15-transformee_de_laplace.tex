\section{Transformée de \textsc{Laplace}} 
\label{transformee_laplace}

\todoinline{Rechercher l'exercice corrigé sur les théorèmes des valeurs initiales / finales.}

La transformation de \textsc{Laplace} généralise la transformation de \textsc{Fourier} qui est également utilisée pour résoudre les équations différentielles : contrairement à cette dernière, elle tient compte des conditions initiales et peut ainsi être utilisée en théorie des vibrations mécaniques ou en électricité dans l'étude des régimes forcés sans négliger le régime transitoire. De manière générale, ses propriétés vis-à-vis de la dérivation permettent un traitement plus simple de certaines équations différentielles, et elle est de ce fait très utilisée en automatique. \\
Dans ce type d'analyse, la transformation de \textsc{Laplace} est souvent interprétée comme un passage du domaine temps, dans lequel les entrées et sorties sont des fonctions du temps, dans le domaine des fréquences, dans lequel les mêmes entrées et sorties sont des fonctions de la \say{ fréquence } (complexe) $p$. Ainsi; il est possible d'analyser simplement l'effet du système sur l'entrée pour donner la sortie en matière d'opérations algébriques simples (cf. théorie des fonctions de transfert en électronique ou en mécanique). 

\begin{defi}{Transformée de \textsc{Laplace}}
    Pour tout fonction $f \in \mathscr{C}(\Rp, \R)$, on note, lorsqu'elle converge, 
    $$\mathscr{L}(f)(p) \defeq \int_{0}^{+ \infty} \e^{-pt} f(t) \d t.$$
    La fonction $\mathscr{L}(f)$ est la \emph{transformée de \textsc{Laplace} de f}.
\end{defi}

\marginnote[0cm]{Sources : \cite{exos_oraux} + \cite{acamanes} (Exercice cerise Ch. 12)}
\underline{Démonstration du théorème de la valeur finale:}
\begin{itemize}
    \item Généralisation classique du théorème des bornes $\leadsto$ $f$ est bornée
    \item Changement de variable: $\varphi: u \mapsto \frac{u}{p}$
    \item Caractérisation séquentielle de la limite
    \item Théorème de convergence dominée
\end{itemize}

\todoinline{Ajout d'exercices en vrac ci-dessous}

\begin{exercice}
{Dunod - p.734 \& 736}
Pour toute fonction $f \in \mathscr{C}(\R_+,\R)$, on note, lorsqu'elle converge, $\mathscr{L}(f)(p) = \int_0^{+\infty} \e^{-p t} f(t) \d t$. La fonction $\mathscr{L}(f)$ est la transformée de~{Laplace} de $f$.
\begin{enumerate}
\item Soient $\lg \in \C$ et $n \in \N$. Pour chacune des fonctions suivantes, déterminer leur transformée de~{Laplace} en précisant son domaine de définition :
\begin{enumerate}
\item $t \mapsto 1$.
\item $t \mapsto \e^{\lg t}$.
\item $t \mapsto t^n$.
\end{enumerate}

\item On suppose que $f$ est bornée. Montrer que $\mathscr{L}(f)$ est définie et de classe $\mathscr{C}^\infty$ sur $\R_+^*$.

\item {Théorème de la valeur finale.} On suppose qu'il existe un réel $\ell$ non nul tel que $\lim_{+\infty} f(x) = \ell$. Déterminer un équivalent de $\mathscr{L}(f)$ en $0$.

\medskip
On suppose $f$ continue uniquement sur $\R_+^*$ et qu'il existe $p_0 > 0$ tel que, pour pour tout $p > p_0$, $t \mapsto \e^{-p t} f(t)$ est intégrable sur $\R_+$. 

\item Montrer que $\mathscr{L}(f)$ est définie et continue sur $]p_0,+\infty[$.

\item {Théorème de la valeur initiale.} On note $\ell = \lim_{t\to0^+} f(t)$. Déterminer la limite de $p \mapsto p \mathscr{L}(f)(p)$ en $+\infty$.
\end{enumerate}
\end{exercice}

\begin{preuve}
\begin{enumerate}
\begin{enumerate}
\item $t \mapsto \e^{-pt}$ est intégrable sur $\R_+$ si et seulement si $p > 0$. Alors, $\mathscr{L}(f)(p) = \int_0^{+\infty} \e^{-pt} \d t = \frac{1}{p}$.

\item $t \mapsto \e^{-(\lg-p)t}$ est intégrable sur $\R_+$ si et seulement si $p > \Re{\lg}$. Alors, $\mathscr{L}(f)(p) = \int_0^{+\infty} \e^{-(p-\lg)t} \d t = \frac{1}{p-\lg}$.

\item Soit $n \in \N^*$ et $f_n : t \mapsto t^n \e^{-pt}$.
\begin{itemize}
\item $f_n$ est continue sur $\R_+^*$.
\item Si $p > 0$, alors $f(t) = o_{+\infty}(1/t^2)$ et $f$ est intégrable sur $\R_+^*$.

Si $p \leq 0$, alors $\lim_{+\infty} f = +\infty$ et $f$ n'est pas intégrable sur $\R_+^*$.
\end{itemize}
Ainsi, $f_n$ est intégrable si et seulement si $p > 0$. Une récurrence classique avec des intégrations par parties permet de montrer que
\[
\mathscr{L}(f)(p) = \frac{n!}{p^{n+1}}.
\]

{Remarque.} À un changement de variable près, on retrouve la fonction $\Gamma$ d'Euler.
\end{enumerate}

\item La fonction $f$ est bornée par une constante $M$. On utilise le théorème de dérivation sous le signe intégral. Notons $g : (p, t) \mapsto f(t) \e^{-pt}$.
\begin{itemize}
\item Pour tout $t \in \R_+^*$, $g(\cdot, t) : p \mapsto \e^{-p t} f(t)$ est de classe $\mathscr{C}^\infty$ sur $\R_+^*$ et
\[
\forall\, j \in \N,\, \frac{\partial^j g}{\partial p^j} g(p, t) = (-1)^j t^j f(t) \e^{-pt}.
\]

\item Pour tout $p > 0$, $g(p, \cdot) : t \mapsto (-1)^j t^j \e^{-p t} f(t)$ est continue sur $[0,+\infty[$.

\item Soit $\tilde{p} > 0$. Alors, pour tout $p \geq \tilde{p}$,
\[
\abs{(-1)^j t^j f(t) \e^{-p t}} \leq M t^j \e^{- \tilde{p} t} M.
\]
De plus, $\phi_j : t \mapsto M t^j \e^{-\tilde{p} t}$ est continue sur $\R_+$, et est négligeable devant $1/t^2$ en $+\infty$. Ainsi, $\phi_j$ est intégrable.
\end{itemize}
Ainsi, $\mathscr{L}(f)$ est de classe $\mathscr{C}^\infty$ sur $\R_+^*$.

\item Comme $f$ est continue sur $\R_+$ et possède une limite en $+\infty$, une généralisation classique du théorème des bornes assure que $f$ est bornée sur $\R_+$. Ainsi, d'après la question précédente, $\mathscr{L}(f)$ est bien définie sur $\R_+^*$.

\smallskip


On effectue le changement de variable affine $\phi : u \mapsto \frac{u}{p}$. Alors,
\[
p \mathscr{L}(f)(p) = \int_0^{+\infty} \e^{-u} f\left(\frac{u}{p}\right) \d u.
\]
Soit $(p_n)$ une suite de réels strictement positifs qui tend vers $0$ et $g_n : u \mapsto \e^{-u} f(u/p_n)$.
\begin{itemize}
\item Pour tout $n \in \N$, $g_n$ est continue sur $\R_+$.

\item Soit $u > 0$. D'après les hypothèses sur la fonction $f$, $\lim_{n\to+\infty} \e^{-u} f(u/p_n) = \e^{-u} \ell$. Ainsi, $(g_n)$ converge simplement sur $\R_+^*$ vers $u \mapsto \e^{-u} \ell$.

\item Comme $f$ est bornée par une constante $M$, $\abs{g_n(u)} \leq M \e^{-u}$ et $u \mapsto \e^{-u}$ est intégrable sur $\R_+^*$.
\end{itemize}
Ainsi, d'après le théorème de convergence dominée,
\[
\lim_{n\to+\infty} p_n \mathscr{L}(f)(p_n) = \ell \int_0^{+\infty} \e^{-u} \d u.
\]
En utilisant la caractérisation séquentielle de la limite,
\[
\lim_{p\to 0} p \mathscr{L}(f)(p) = \ell \text{ soit } \mathscr{L}(f)(p) \sim_0 \frac{\ell}{p}.
\]

\item Soit $g : (p, u) \mapsto f(u) \e^{-p u}$.
\begin{itemize}
\item $\forall\, t \in \R_+^*,\, g(\cdot, u)$ est continue sur $]p_0, +\infty[$.

\item $\forall\, p > p_0,\, g(p, \cdot)$ est continue sur $\R_+^*$.

\item Soit $\tilde{p} > p_0$. Pour tout $p \geq \tilde{p}$ et $t \in \R_+^*$,
\[
\abs{\e^{-p t} f(t)} \leq \e^{- \tilde{p} t} \abs{f(t)}.
\]
De plus, $t \mapsto f(t) \e^{-\tilde{p} t}$ est intégrable sur $\R_+^*$ par hypothèse.
\end{itemize}
D'après le théorème de continuité sous le signe intégral, $\mathscr{L}(f)$ est continue sur $]p_0, +\infty[$.

\item Si $f$ est bornée, on peut appliquer la même méthode que pour le théorème de la valeur finale.

\smallskip

Sinon, on raisonne à coup de $\eg$. Soit $\eg > 0$. Comme $\lim_{0^+} f = \ell$, il existe $h > 0$ tel que
\[
\forall\, t \in ]0, h],\, \abs{f(t) - \ell} \leq \eg.
\]
Alors, pour $p > p_0 + 1$ et $\tilde{p} = p_0 + \frac{1}{2}$,
\begin{align*}
\abs{p \mathscr{L}(f)(p) - \ell} &\leq \int_0^{+\infty} p \abs{f(t) - \ell} \e^{-p t} \d t \\
&\leq \int_0^h p \abs{f(t) - \ell} \e^{-pt} \d t + \int_h^{+\infty} p \abs{f(t) - \ell} \e^{-pt} \d t\\
&\leq \eg \left(1 - \underbrace{\e^{-p h}}_{\geq 0}\right)
+ p \int_h^{+\infty} \abs{f(t)} \e^{-p t} \d t
+ p \int_h^{+\infty} \abs{\ell} \e^{- p t} \d t \\
&\leq \eg
+ \int_h^{+\infty} \abs{f(t)} \e^{-(p-\tilde{p})t - \tilde{p} t} \d t + \abs{\ell} \e^{-p h}\\
&\leq \eg + p \underbrace{\e^{-(p - \tilde{p}) h}}_{\to 0} \underbrace{\abs{\int_0^{+\infty} \abs{f(t)} \e^{-\tilde{p} t} \d t}}_{cte} + \abs{f(0)} \e^{-ph} \\
&\leq 3 \eg,
\end{align*}
pour $p$ assez grand. Ainsi, d'après la définition de la limite,
\[
\lim_{p\to +\infty} p \mathscr{L}(p) = \ell.
\]
\end{enumerate}
\end{preuve}

%---------------

\begin{exercice}
{ESCP 2013 - QC}%
Soit $f~:~\R_+ \to \R$ une fonction continue telle que $\int_0^{+\infty} f(t) \d t$ converge. Montrer que, pour tout réel $x$ positif, $\int_0^{+\infty} \e^{-x t} f(t) \d t$ converge.
\end{exercice}

\begin{preuve}
Lorsque $x = 0$, l'intégrale converge par hypothèse.

Pour $x > 0$. La fonction $F$ admet une limite en $+\infty$ et elle est continue. Ainsi, $F$ est bornée. Ainsi, comme $F$ est bornée et $x > 0$, alors $F(t) \e^{-x t} \to 0$ lorsque $x \to +\infty$.

Enfin, à l'aide d'une intégration par parties, comme la fonction $F~:~x \mapsto \int_0^x f(t) \d t$ est de classe $\mathscr{C}^1$ sur $\R_+$ de dérivée égale à $f$, l'intégrale $\int_0^{+\infty} \e^{-x t} f(t) \d t$ est de même nature que $\int_0^{+\infty} \e^{-x t} F(t) \d t$. Enfin, comme $\abs{\e^{-x t} F(t)} \leq M \e^{-x t}$, alors $\int_0^{+\infty} \e^{-x t} F(t) \d t$ converge.
\end{preuve}

\section{Théorème de \nom{Fubini}}

\url{https://www.bibmath.net/dico/index.php?action=affiche&quoi=./f/fubini.html} pour quelques éléments historiques sur les intégrales multiples. 

\marginnote[0pt]{Ce théorème a été démontré par le mathématicien italien Guido \nom{Fubini} (1879--1943) en 1907.}
\begin{theo}[\nom{Fubini}]
    Soit $\fonctionligne[f]{\interff{a}{b} \times \interff{c}{d}}{\K}$ une application continue. Alors,
    \[
    \int_{a}^{b} \mathopen{}\bigg( \int_{c}^{d} f(x,y) \d y \bigg) \d x = \int_{c}^{d} \mathopen{}\bigg( \int_{a}^{b} f(x,y) \d x \bigg) \d y.
    \]
\end{theo}

\begin{marginfigure}[0cm]
    \centering
    % define colors 
\colorlet{filltop}{mylightblue}
\colorlet{curvecolor}{myblue}
\colorlet{meshcolor}{myblue}

 \begin{tikzpicture}[
 %3d view={135}{60},
 x={(-0.6cm,-0.35cm)},z={(0,0.2cm)},y={(1cm,-.25cm)}, 
 %  x={(-0.6cm,-0.5cm)},z={(0,0.2cm)},y={(1cm,-.3cm)}, 
 scale=1.5, line cap=round, line join=round
 ]
% \shorthandoff{:;!?} % https://groups.google.com/g/fr.comp.text.tex/c/K2CEGtgU3YQ
 %===================================
 %  surface 1: z=xy with domain boundary x=1, y=x, y=0
 % surface 2:  z=x^2+y^2 with domain boundary y=x, y=1, x=0
 %===================================
%  \tikzset{
%     declare function={
%         f(\u,\v)=(-(\u/1)^2+(\v/0.8)^2)/3+6.5;
%         }
%     }
\pgfmathdeclarefunction{f}{2}{%
    \pgfmathparse{(-(#1/1)^2+(#2/0.8)^2)/3+6.5}%
}

    \def\l{2}
 
    \def\w{1}
    \def\offset{1/4}
    \def\eps{1/30}
    \def\rad{0.05}

    \def\a{0.2}
    \def\b{1.9}
    \def\c{0.3}
    \def\d{2.1}


    \def\N{10}
    \pgfmathsetmacro{\step}{(\b-\a)/(\N+2)}
    \pgfmathsetmacro{\y}{\c+6*\step}
    \def\ythick{1*\step}

    \def\opacoupe{0.3}

% set coordinates 
     \def \mxmin{0}\def \xdash{0} \def\mxmax{2.5}
     \def \mymin{0}\def \ydash{0} \def\mymax{2}
     \def \mzmin{0}\def \zdash{0} \def\mzmax{2}

%%%%%%%%%%%%%%%%%%%%%%%%%%%%%%%%%%%%
%%%%%%%%%%%%%%%%%%%%%%%%%%%%%%%%%%%%
    \coordinate (O) at (\l*1/2,\l*1/2,0);
    \coordinate (X) at (\l*1/4,\l*1/4,0);
    \coordinate (Y) at ({\l*(1-1/4)},{\l*(1-1/4)},0);    
    
    \draw[line width=\w pt,black, dashed] (\a,\c,0) -- (\b,\c,0) -- (\b,\d,0) -- (\a,\d,0) -- cycle;

    \draw[arrows = {-latex[slant=-0.85]}] (0,0,0)--(\l*1.1,0,0) node[left]{$x$};
    \draw[arrows = {-latex[slant=0.75]}] (0,0,0)--(0,\l*1.2,0) node[right]{$y$};
    \draw[-latex] (0,0,0) -- (0,0,7) node[pos = 1.1] {$z$};
     
%%%%%%%%%%%%%%%%%%%%%%%%%%%%%%%%%%%%
%%%%%%%%%%%%%%%%%%%%%%%%%%%%%%%%%%%%

   \draw[dashed] (\a,-\eps,0)[above left] node {$a$} --(\a,\c,0);
   \draw[dashed] (\b,-\eps,0)[above left] node {$b$} --(\b,\c,0);
   \draw[dashed] (-\eps,\c,0)[above right] node {$c$}--(\a,\c,0);
   \draw[dashed] (-\eps,\d,0)[above right] node {$d$}--(\a,\d,0);

   \draw[dashed] (\y,-\eps,0)[above] node {$x$}--(\y,\c,0);


% help lines
   \draw[thick,dashed] (\a,\c,0)--(\a,\c,{f(\a,\c)});
   \draw[thick,dashed] (\b,\c,0)--(\b,\c,{f(\b,\c)});
   \draw[thick,dashed] (\a,\d,0)--(\a,\d,{f(\a,\d)});
% coupe

\draw[thick,draw=myred,fill=mylightred,opacity=\opacoupe] 
   (\y,\c,0) --
   plot[domain=0:{f(\y,\c)},samples=50,smooth] (\y,\c,\x)
   --
   plot[domain=\c:\d,samples=50,smooth] ({\y},{\x},{f(\y,\x)})
   --
   plot[domain={f(\y,\d)}:0,samples=50,smooth] (\y,\d,\x)
   --
   cycle;
\draw[thick,draw=myred,fill=mylightred,opacity=\opacoupe] 
   (\y,\c,0) -- (\y+\ythick,\c,0) -- (\y+\ythick,\d,0) -- (\y,\d,0) -- cycle;
\draw[thick,draw=myred,fill=mylightred,opacity=\opacoupe] 
   (\y,\c,0) -- (\y+\ythick,\c,0) -- (\y+\ythick,\c,{f(\y+\ythick,\c)}) -- (\y,\c,{f(\y,\c)}) -- cycle;
\draw[thick,draw=myred,fill=mylightred,opacity=\opacoupe] 
   (\y+\ythick,\c,0) --
   plot[domain=0:{f(\y+\ythick,\c)},samples=50,smooth] (\y+\ythick,\c,\x)
   --
   plot[domain=\c:\d,samples=50,smooth] ({\y+\ythick},{\x},{f(\y+\ythick,\x)})
   --
   plot[domain={f(\y+\ythick,\d)}:0,samples=50,smooth] (\y+\ythick,\d,\x)
   --
   cycle;
\draw[thick,draw=myred,fill=mylightred,opacity=\opacoupe] 
   (\y,\d,0) -- (\y+\ythick,\d,0) -- (\y+\ythick,\d,{f(\y+\ythick,\d)}) -- (\y,\d,{f(\y,\d)}) -- cycle;

\draw[thick,draw=myred,fill=mylightred,opacity=\opacoupe] 
   (\y,\c,{f(\y,\c)})--
   plot[domain=\c:\d,samples=50,smooth] ({\y},{\x},{f(\y,\x)})
   --
   plot[variable=\Y,domain=\y:\y+\ythick,samples=50,smooth] ({\Y},\d,{f(\Y,\d)}) 
   --
   plot[domain=\d:\c,samples=50,smooth] (\y+\ythick,\x,{f(\y+\ythick,\x)}) 
   --
   plot[variable=\Y,domain=\y+\ythick:\y,samples=50,smooth] (\Y,\c,{f(\Y,\c)}) 
   --
   cycle;

   
\draw[thick,dashed] (\b,\d,0)--(\b,\d,{f(\b,\d)});

%  surface 1
  \draw[thick,draw=curvecolor,fill=filltop,opacity=0.4] 
   (\a,\c,{f(\a,\c)})--
   plot[domain=\a:\b,samples=50,smooth] ({\x},{\c},{f(\x,\c)})
   --
   plot[variable=\y,domain=\c:\d,samples=50,smooth] (\b,{\y},{f(\b,\y)}) 
   --
   plot[domain=\b:\a,samples=50,smooth] (\x,\d,{f(\x,\d)}) 
   --
   plot[variable=\y,domain=\d:\c,samples=50,smooth] (\a,\y,{f(\a,\y)}) 
   --
   cycle;
   
% surface 1: mesh lines  
\foreach \k [evaluate=\k as \x using \a + \k * \step] in {-1,...,\N} {
    \draw[meshcolor] plot[variable=\y,domain=\c:\d,samples=50,smooth] (\a+\x+0.1,\y,{f(\a+\x+0.1,\y)});
}
\foreach \k [evaluate=\k as \i using \c + \k * \step] in {-2,...,\N} {
    \draw[meshcolor] plot[domain=\a:\b,samples=50,smooth] (\x,\c+\i+0.1,{f(\x,\c+\i+0.1)});
}

 %======================
 \end{tikzpicture}
    \caption{Découpage selon l'axe des abscisses}
\end{marginfigure}
\begin{marginfigure}[5cm]
    \centering
    
% define colors 
\colorlet{fillbottom}{yellow!60}
\colorlet{filltop}{blue!20}
\colorlet{curvecolor}{cyan}
\colorlet{meshcolor}{cyan}

 \begin{tikzpicture}[
 %3d view={135}{60},
 x={(-0.6cm,-0.35cm)},z={(0,0.2cm)},y={(1cm,-.25cm)}, 
 %  x={(-0.6cm,-0.5cm)},z={(0,0.2cm)},y={(1cm,-.3cm)}, 
 scale=1.5, line cap=round, line join=round
 ]
% \shorthandoff{:;!?} % https://groups.google.com/g/fr.comp.text.tex/c/K2CEGtgU3YQ
 %===================================
 %  surface 1: z=xy with domain boundary x=1, y=x, y=0
 % surface 2:  z=x^2+y^2 with domain boundary y=x, y=1, x=0
 %===================================
%  \tikzset{
%     declare function={
%         f(\u,\v)=(-(\u/1)^2+(\v/0.8)^2)/3+6.5;
%         }
%     }
\pgfmathdeclarefunction{f}{2}{%
    \pgfmathparse{(-(#1/1)^2+(#2/0.8)^2)/3+6.5}%
}

    \def\l{2}
 
    \def\w{1}
    \def\offset{1/4}
    \def\eps{1/30}
    \def\rad{0.05}

    \def\a{0.2}
    \def\b{1.9}
    \def\c{0.3}
    \def\d{2.1}


    \def\N{10}
    \pgfmathsetmacro{\step}{(\b-\a)/(\N+2)}
    \pgfmathsetmacro{\y}{\c+6*\step}
    \def\ythick{1*\step}

    \def\opacoupe{0.3}

% set coordinates 
     \def \mxmin{0}\def \xdash{0} \def\mxmax{2.5}
     \def \mymin{0}\def \ydash{0} \def\mymax{2}
     \def \mzmin{0}\def \zdash{0} \def\mzmax{2}

%%%%%%%%%%%%%%%%%%%%%%%%%%%%%%%%%%%%
%%%%%%%%%%%%%%%%%%%%%%%%%%%%%%%%%%%%
    \coordinate (O) at (\l*1/2,\l*1/2,0);
    \coordinate (X) at (\l*1/4,\l*1/4,0);
    \coordinate (Y) at ({\l*(1-1/4)},{\l*(1-1/4)},0);    
    
    \draw[line width=\w pt,black, dashed] (\a,\c,0) -- (\b,\c,0) -- (\b,\d,0) -- (\a,\d,0) -- cycle;

    \draw[arrows = {-latex[slant=-0.85]}] (0,0,0)--(\l*1.1,0,0) node[left]{$x$};
    \draw[arrows = {-latex[slant=0.75]}] (0,0,0)--(0,\l*1.2,0) node[right]{$y$};
    \draw[-latex] (0,0,0) -- (0,0,7) node[pos = 1.1] {$z$};
     
%%%%%%%%%%%%%%%%%%%%%%%%%%%%%%%%%%%%
%%%%%%%%%%%%%%%%%%%%%%%%%%%%%%%%%%%%

   \draw[dashed] (\a,-\eps,0)[above left] node {$a$} --(\a,\c,0);
   \draw[dashed] (\b,-\eps,0)[above left] node {$b$} --(\b,\c,0);
   \draw[dashed] (-\eps,\c,0)[above right] node {$c$}--(\a,\c,0);
   \draw[dashed] (-\eps,\d,0)[above right] node {$d$}--(\a,\d,0);

   \draw[dashed] (-\eps,\y,0)[above right] node {$y$}--(\a,\y,0);


% help lines
   \draw[thick,dashed] (\a,\c,0)--(\a,\c,{f(\a,\c)});
   \draw[thick,dashed] (\b,\c,0)--(\b,\c,{f(\b,\c)});
   \draw[thick,dashed] (\a,\d,0)--(\a,\d,{f(\a,\d)});
% coupe

\draw[thick,draw=red,fill=red!50!white,opacity=\opacoupe] 
   (\a,\y,0) --
   plot[domain=0:{f(\a,\y)},samples=50,smooth] (\a,\y,\x)
   --
   plot[domain=\a:\b,samples=50,smooth] ({\x},{\y},{f(\x,\y)})
   --
   plot[domain={f(\b,\y)}:0,samples=50,smooth] (\b,\y,\x)
   --
   cycle;
\draw[thick,draw=red,fill=red!50!white,opacity=\opacoupe] 
   (\a,\y,0) -- (\a,\y+\ythick,0) -- (\b,\y+\ythick,0) -- (\b,\y,0) -- cycle;
\draw[thick,draw=red,fill=red!50!white,opacity=\opacoupe] 
   (\a,\y,0) -- (\a,\y+\ythick,0) -- (\a,\y+\ythick,{f(\a,\y+\ythick)}) -- (\a,\y,{f(\a,\y)}) -- cycle;
\draw[thick,draw=red,fill=red!50!white,opacity=\opacoupe] 
   (\a,\y+\ythick,0) --
   plot[domain=0:{f(\a,\y+\ythick)},samples=50,smooth] (\a,\y+\ythick,\x)
   --
   plot[domain=\a:\b,samples=50,smooth] ({\x},{\y+\ythick},{f(\x,\y+\ythick)})
   --
   plot[domain={f(\b,\y+\ythick)}:0,samples=50,smooth] (\b,\y+\ythick,\x)
   --
   cycle;
\draw[thick,draw=red,fill=red!50!white,opacity=\opacoupe] 
   (\b,\y,0) -- (\b,\y+\ythick,0) -- (\b,\y+\ythick,{f(\b,\y+\ythick)}) -- (\b,\y,{f(\b,\y)}) -- cycle;

\draw[thick,draw=red,fill=red!50!white,opacity=\opacoupe] 
   (\a,\y,{f(\a,\y)})--
   plot[domain=\a:\b,samples=50,smooth] ({\x},{\y},{f(\x,\y)})
   --
   plot[variable=\Y,domain=\y:\y+\ythick,samples=50,smooth] (\b,{\Y},{f(\b,\Y)}) 
   --
   plot[domain=\b:\a,samples=50,smooth] (\x,\y+\ythick,{f(\x,\y+\ythick)}) 
   --
   plot[variable=\Y,domain=\y+\ythick:\y,samples=50,smooth] (\a,\Y,{f(\a,\Y)}) 
   --
   cycle;

   
\draw[thick,dashed] (\b,\d,0)--(\b,\d,{f(\b,\d)});

%  surface 1
  \draw[thick,draw=curvecolor,fill=filltop,opacity=0.4] 
   (\a,\c,{f(\a,\c)})--
   plot[domain=\a:\b,samples=50,smooth] ({\x},{\c},{f(\x,\c)})
   --
   plot[variable=\y,domain=\c:\d,samples=50,smooth] (\b,{\y},{f(\b,\y)}) 
   --
   plot[domain=\b:\a,samples=50,smooth] (\x,\d,{f(\x,\d)}) 
   --
   plot[variable=\y,domain=\d:\c,samples=50,smooth] (\a,\y,{f(\a,\y)}) 
   --
   cycle;

\begin{comment}
% surface 1: mesh lines  
\foreach \k [evaluate=\k as \x using \a + \k * \step] in {-1,...,\N} {
    \draw[meshcolor] plot[variable=\y,domain=\c:\d,samples=50,smooth] (\a+\x+0.1,\y,{f(\a+\x+0.1,\y)});
}
\foreach \k [evaluate=\k as \i using \c + \k * \step] in {-2,...,\N} {
    \draw[meshcolor] plot[domain=\a:\b,samples=50,smooth] (\x,\c+\i+0.1,{f(\x,\c+\i+0.1)});
}
\end{comment}

    % Define your color for shading
    \definecolor{curvecolor}{rgb}{0.1, 0.6, 0.8}
    \definecolor{filltop}{rgb}{0.3, 0.7, 1.0}

    % Draw the shaded surface with lighting effects
    \shade[thick, draw=curvecolor, left color=filltop!80!white, right color=filltop!20!black, opacity=0.4, smooth]
        (\a,\c,{f(\a,\c)}) --
        plot[domain=\a:\b, samples=50, smooth] ({\x},{\c},{f(\x,\c)}) --
        plot[variable=\y, domain=\c:\d, samples=50, smooth] (\b,{\y},{f(\b,\y)}) --
        plot[domain=\b:\a, samples=50, smooth] (\x,\d,{f(\x,\d)}) --
        plot[variable=\y, domain=\d:\c, samples=50, smooth] (\a,\y,{f(\a,\y)}) --
        cycle;

 %======================
 \end{tikzpicture}


\begin{comment}
\begin{tikzpicture}[
  x=(215:2em/sqrt 2), y=(0:2em), z=(90:2em),
  declare function={f(\x,\y)=((\x-3)^2+(-\y+3)^3)/8+3;}, 
  very thick, line join=round]
\draw [-stealth, black!75] (0,0,0) -- (5,0,0) node [below left] {$x$};
\draw [-stealth, black!75] (0,0,0) -- (0,5,0) node [below right] {$y$};
\draw [-stealth, black!75] (0,0,0) -- (0,0,5) node [right] {$z$};
\foreach \x in {1,...,4}
  \foreach \y [evaluate={\j=\x+.5; \i=\y+.5; \k=f(\j,\i);}] in {1,...,4}{
    \path [fill=black!50, draw=white] (\x, \y+1, 0) -- (\x+1, \y+1, 0) -- 
      (\x+1, \y+1, \k) -- (\x, \y+1, \k) -- cycle;
    \path [fill=black!25, draw=white] (\x+1, \y, 0) -- (\x+1, \y+1, 0) -- 
      (\x+1, \y+1, \k) -- (\x+1, \y, \k) -- cycle;
    \path [fill=black!10, draw=white] (\x, \y, \k)  -- (\x+1, \y, \k) -- 
      (\x+1, \y+1, \k) -- (\x, \y+1, \k) -- cycle;
  }
 \foreach \x in {1,...,4}
   \foreach \y in {1,...,4}{
 \draw [black, fill=black, fill opacity=0.125, 
    domain=0:1, samples=10, variable=\t] 
    plot (\x+\t, \y, {f(\x+\t,\y)}) -- 
    plot (\x+1, \y+\t, {f(\x+1,\y+\t)}) -- 
    plot (\x+1-\t, \y+1, {f(\x+1-\t,\y+1)}) --
    plot (\x, \y+1-\t, {f(\x,\y+1-\t)}) -- cycle;
  }
\end{tikzpicture}
\end{comment}
    \caption{Découpage selon l'axe des ordonnées}
\end{marginfigure}

\begin{exercice}
    Pour tout $(x, t) \in \interff{a}{b} \times \interff{c}{d}$ on pose 
    $$\varphi(x, t) \defeq \int_{a}^{x} f(u, t) \d u.$$
    \begin{questions}
    \item Montrer que pour tout $x \in \interff{a}{b}$, l'application $t \mapsto \varphi(x, t)$ est continue sur $\interff{c}{d}$.
    \end{questions}

    On pose alors, pour tout $x  \in \interff{a}{b}$,
    $$\psi(x) \defeq \int_{c}^{d} \varphi(x, t) \d t.$$
    \begin{questions}[resume]
        \item Montrer que $\psi$ est de classe $\mathscr{C}^1$ sur $\interff{a}{b}$ et donner une expression de $\psi'$.
        \item En déduire que pour tout $x \in \interff{a}{b}$,
        \[
        \int_{a}^{x} \mathopen{}\bigg( \int_{c}^{d} f(u,t) \d t \bigg) \d u = \int_{c}^{d} \mathopen{}\bigg( \int_{a}^{x} f(u,t) \d u \bigg) \d t.
        \]
    \end{questions}
\end{exercice}

% \todoarmand{Je propose les deux figures ci-contre (qui sont à peaufiner). Est-ce qu'on garde cette version de tranches épaisses à la "Riemann" ou bien on fait plutôt une version comme dans \url{https://math.hawaii.edu/~kcorea/courses/spring_2023/244/static/15.1-scan.pdf} p.7}

\marginnote[2cm]{Source : correction du sujet Mines Maths 2 PSI 2021 par Doc Solus.} 
\begin{solution}
\begin{reponses}
\item Les hypothèses de régularité du théorème de continuité des intégrales à paramètre sont immédiates.
    
Pour vérifier l'hypothèse de domination, on constate que la fonction $f$ est continue sur $\interff{a}{b} \times \interff{c}{d}$ qui est une partie fermée bornée de $\R^2$. Ainsi, d'après le théorème des bornes, la fonction $f$ est bornée sur $\interff{a}{b} \times \interff{c}{d}$ par une constante $M \in \Rp$. Les fonctions constantes sont intégrables car l'intégrale s'effectue ici sur un segment.
        
\item On applique le théorème de dérivation des intégrales à paramètre à la fonction $x \mapsto \int_{c}^{d} \varphi(x, t) \d t$:
        \begin{itemize}
            \item Pour tout $t \in \interff{c}{d}$, la fonction $x \mapsto \varphi(x, t)$ est de classe $\mathscr{C}^1$ sur $\interff{a}{b}$ car c'est la primitive s'annulant en $a$ de la fonction continue $x \mapsto f(x, t)$. 
            \item Sa dérivée partielle s'écrit $\frac{\partial \varphi}{\partial x}(x, t) = f(x, t)$.
            \item La domination se fait par la même constante $M$ que précédemment. 
            \end{itemize}
            Ainsi,
            \[
            \forall x \in \interff{a}{b} \quad \psi'(x) = \int_c^d \frac{\partial \varphi}{\partial x}(x, t) \d t = \int_{c}^{d} f(x, t) \d t.
            \]
        \item Soit $x \in \interff{a}{b}$. D'une part,
        $$\psi(x) = \int_{c}^{d} \mathopen{}\bigg( \int_{a}^{x} f(u,t) \d u \bigg) \d t.$$
        D'autre part, d'après la question précédente et le théorème fondamental de l'analyse, 
        \begin{align*}
            \int_{a}^{x} \mathopen{}\bigg( \int_{c}^{d} f(u,t) \d t \bigg) \d u &= \int_{a}^{x} \psi'(u) \d u  = \psi(x) - \psi(a) \\
            \text{Or, } \psi(a) &= \int_{c}^{d} \varphi(a, t) \d t \\
            \text{et } \forall t \in \interff{c}{d}, \quad \varphi(a, t) &= \int_{a}^{a} f(u, t) \d u = 0
        \end{align*}
        d'où $\psi(a) = 0$ et le résultat. \\
        En particulier, pour $x = b$ on obtient le résultat final.
    \end{reponses}
\end{solution}    

\todoarmand{Une application possible : le produit de convolution. J'ai pour l'instant simplement repris le théorème de l'un de mes cours de première année.}

\todoinline{Le pb est que la démo précédente de Fubini ne fonctionne que sur un segment. J'aime bien aussi le produit de convolution, je crée un exercice dessus.}

\begin{comment}
\begin{theo}
    Soient $u$ et $v$ deux fonctions de $L^1(\R^d)$. \textcolor{red}{Pour presque tout} $x \in \R^d$, on peut définir
    \begin{equation}\label{defconvolution}
    (u \ast v)(x) = \int_{\R^d} u(x-y) v(y) \d y = \int_{\R^d} u(y) v(x-y) \d y.
    \end{equation}
    La fonction $u \ast v$ ainsi définie est appelée le \emph{produit de convolution} de $u$ et $v$. Elle appartient à $L^1(\R^d)$ et 
    \[
    \norm{u \ast v}_{L^1} \leqslant \norm{u}_{L^1} \norm{v}_{L^1}.
    \]
\end{theo}
\begin{demo}
    L'existence des intégrales \ref{defconvolution} n'est pas évidente : à $x$ fixé, il s'agit d'intégrer le produit de deux fonctions intégrables. Mais un tel produit n'est pas intégrable en général. Dans ce cas particulier, nous allons le déduire tu théorème de \textsc{Fubini}. On a
    \[
    \int_{\R^d \times \R^d} \module{u(x-y) v(y)} \d x \d y = \int_{\R^d} \module{v(y)} \left( \int_{\R^d} \module{u(x - y)} \d x \right) \d y.
    \]
    En effectuant le changement de variable $z = x - y$ (à $y$ fixé), il vient
    \[
    \int_{\R^d} \module{u(x-y)} \d x = \int_{\R^d} \module{u(z)} \d z = \norm{u}_{L^1}.
    \]
    Ainsi l'égalité précédente s'écrit
    \[
    \int_{\R^d \times \R^d} \module{u(x-y) v(y)} \d x \d y = \norm{u}_{L^1} \int_{\R^d} \module{v(y)} \d y = \norm{u}_{L^1} \norm{v}_{L^1}.
    \]
    Il s'ensuit que la fonction $(x,y) \mapsto u(x-y) v(y)$ appartient à $L^1(\R^d \times \R^d)$. Le théorème de \textsc{Fubini} nous permet donc de dire que pour presque tout $x$, la fonction $y \mapsto u(x-y) v(y)$ appartient à $L^1(\R^d)$, ce qui donne un sens à la première intégrale dans \ref{defconvolution}, la seconde s'en déduisant par le changement de variable $z = x - y$ (à $x$ fixé). Il reste à constater que 
    \[
    \norm{u \ast v}_{L^1} = \int_{\R^d} \module{\int_{\R^d} u(x-y)v(y) \d y} \d x \leqslant \int_{\R^d \times \R^d} \module{u(x-y) v(y)} \d x \d y,
    \]
    où l'on vient de voir que la dernière intégrale n'est autre que $\norm{u}_{L^1} \norm{v}_{L^1}$.
\end{demo}

\todoarmand{On pourrait ensuite proposer quelques exercices sur le calcul du produit de convolution de fonctions classiques. C'est aussi l'occasion de faire de jolies illustrations.}
\end{comment}

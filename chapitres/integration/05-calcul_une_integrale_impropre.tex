\section{Calcul d'une intégrale impropre}

\todoinline{Semble s'appeler l'intégrale d'Euler - \url{https://fr.wikipedia.org/wiki/Table_d'intégrales}}

% \begin{elem_sol}
    % $=-\pi \ln(2)$
% \end{elem_sol}

\begin{exercice}
\cite{Oraux - CCP-PSI-2016}
    Soient $I = \int_0^{\pi/2} \ln\sin(t)) \d t$ et $J = \int_0^{\pi/2} \ln(\cos(t)) \d t$.
    \begin{enumerate}
        \item Montrer que $I$ et $J$ sont convergentes et que $I = J$.
        \item Calculer $I + J$ et en déduire $I$ et $J$.
    \end{enumerate}
\end{exercice}

% \todoinline{En mettre un peu plus sur la démo ? J'ai la version suivante à relire et changer les dt (CCP-PSI-2016) :}

\begin{elem_sol}
\begin{enumerate}
\item La fonction $t \mapsto \ln(\sin(t))$ est continue sur $]0,\pi/2]$. De plus,
\[
\ln(\sin(t)) = \ln(t + o(t)) = \ln(t) + \ln(1 + o(1)) = o(\ln(t)).
\]
Ainsi, $t \mapsto \ln(\sin(t))$ est intégrable en $0$.

La formule de changement de variable, avec $\phi : u \mapsto \pi/2 - u$ assure la convergence de $J$ ainsi que l'égalité $I = J$.

\item Comme ces intégrales sont bien définies, en utilisant la relation de Chasles et la symétrie dans la dernière égalité,
\[
I + J = \int_0^{\pi/2} \ln\left(\frac{\sin(2t)}{2}\right) \mathrm{d} t = \frac{1}{2} \int_0^\pi \ln(\sin(t)) \mathrm{d} t - \frac{\pi}{2} \ln(2) = I - \frac{\pi}{2} \ln(2).
\]
Ainsi, $I = J = -\frac{\pi}{2} \ln(2)$.
\end{enumerate}
\end{elem_sol}

    
\begin{exercice}
    \marginnote[0cm]{Source : \cite{exos_oraux} p. 268}
    Soit $f : \Rp \to \R$ une fonction continue, décroissante et intégrable. Montrer que $x f(x) \xrightarrow[x \to +\infty]{} 0$.
\end{exercice}

\begin{elem_sol}
\todoarmand{Correction : exercice 3 de \url{http://ddmaths.free.fr/section115.html}}
\begin{itemize}
\item Montrer que $f$ tend vers 0 en utilisant sa décroissance et son intégrabilité. $f$ est donc à valeurs positives. 
\item Encadrer $x f(x)$ en écrivant $x=2 \cdot \frac{x}{2}$.
\end{itemize}
\end{elem_sol}

\todoinline{Concernant les fonctions décroissantes, on pourrait ajouter un exercice qui doit ressembler à (je ne l'ai jamais écrit...) : Soit $f$ décroissante, continue par morceaux et intégrable sur $]0, 1]$. Alors, $\lim\limits_{n\to+\infty} \sum\limits_{k=0}^n f\left(\frac{k}{n}\right) = \int_0^1 f(t) \d t$. Il se prête à une illustration graphique ! Je mets dans un dossier un article qui contient ce résultat. J'ai un exercice d'oral qui utilise ce résultat. À chercher ?}
    
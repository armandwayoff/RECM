 \begin{exercice}
    \marginnote[0cm]{Source : \cite{acamanes} \href{https://acamanes.github.io/psi/psi_doc/exos_e01.pdf}{(Exercice 9. TD I)}}
    Soit $f$ une fonction supposée continue et positive sur $\interff{a}{b}$. Étudier la suite de terme général 
    $$u_n \defeq \left( \frac{1}{b-a} \int_{a}^{b} f(x)^n \d x \right)^{1/n}.$$
 \end{exercice}

\begin{solution}
    \begin{itemize}
        \item La démarche consiste à encadrer le terme $u_n$. 
        \item \textbf{Majoration:} la fonction $f$ est continue et positive sur un segment donc est en particuler, elle y est bornée par un réel positif $M$. Montrons que $u_n \leqslant M$ pour tout $n \in \N$. \emph{ne pas oublier l’argument de la continuité lors du passage à l’intégrale}
        \item \textbf{Minoration:} soit $\varepsilon > 0$, soit $x_0$ tel que $f(x_0) = M$. Comme $f$ est continue en $x_0$, il existe $[c, d] \subset [a, b]$ tel que $x_0 \in [c, d]$ et pour tout $x \in [c, d]$, $f(x) \geqslant M - \varepsilon$ \emph{(un dessin permet de bien comprendre la stratégie)}.\\
        On peut ensuite montrer que $u_n \geqslant \left(\frac{d-c}{b-a} \right)^{1/n}(M-\varepsilon) \xrightarrow[n \to + \infty]{} M-\varepsilon$.
        \item Finalement, $u_n \displaystyle \longrightarrow M = \max_{[ a, b ]} f = \Ninf{f}$.
    \end{itemize}
\end{solution}

\begin{figure}
    \centering


\begin{tikzpicture}

\tikzset{>=latex} % for LaTeX arrow head

\def\tick#1#2{\draw[thick] (#1)++(#2:0.12) --++ (#2-180:0.24)}
\def\N{100} % number of samples

  \def\xmax{3.5}
  \def\ymax{3.5}
  \def\xeps{3}
  \def\yeps{1.8}
  \coordinate (O) at (0,0);
  \coordinate (C) at (0.927575,0);
  \coordinate (A) at (0.5,0);
  \coordinate (B) at (2.796001,0);
  \coordinate (D) at (2.0700,0);
  \coordinate (X0) at (1.63694,0);
  \def\yMAX{2.587020}

  % AXIS
  \draw[->,thick]
    (-0.1*\ymax,0) -- (\xmax,0) node[below] {$x$};
  \draw[->,thick]
    (0,-0.1*\ymax) -- (0,\ymax) node[below left] {$\textcolor{red}{f}(x)$}; %\langle{P}\rangle
  
  % PLOT
  \draw[xline,red,samples=\N,smooth,variable=\x,domain=-0.1:0.94*\xmax,thick]
    plot(\x,{3/(1+(0.36*\x*\x-1)^2) -\x/4});
    
  \draw[dashed,thin] (X0) --++ (0,\yMAX);
  \draw[dashed,thin] (C) --++ (0,\yeps);
  \draw[dashed,thin] (D) --++ (0,\yeps);
  
  \draw[thin] (0,\yMAX) -- (\xmax,\yMAX) node[left] at (0,\yMAX) {$M$};
  \draw[thin] (0,\yeps) -- (\xmax,\yeps);
  
  \draw[<->] (\xeps,\yeps) -- (\xeps,\yMAX)
    node[midway,scale=0.9] {\contour{white}{$\varepsilon$}};
    
  \tick{X0}{90} node[below] {$x_0$};
  \tick{A}{90} node[below] {$a$};
  \tick{B}{90} node[below] {$b$};
  \tick{C}{90} node[below] {$c$};
  \tick{D}{90} node[below] {$d$};
  
\end{tikzpicture}

    \caption{Illustration à finir}
\end{figure}

\subsection{Inclusions entre les $L^p(\Omega)$}

\begin{theo}{}
    Si $\Omega$ est de mesure $\module{\Omega}$ finie, alors pour $p<q$, $L^q(\Omega) \subset L^p(\Omega)$ et pour tout $f \in L^q(\Omega)$, $\norm{f}_{L^p} \leqslant \module{\Omega}^{\frac{1}{p} - \frac{1}{q}} \norm{f}_{L^q}$.
\end{theo}

\subsection{Cas $0 < p < 1$}

Lorsque l’exposant $p$ satisfait $0 < p < 1$, on constate qu’une inégalité triangulaire ne peut pas être satisfaite, ce qui justifie, pour bénéficier d’une structure naturelle
d’espace vectoriel, de se restreindre à supposer $1 < p < +\infty$. 

Exercice 2 de \url{https://www.imo.universite-paris-saclay.fr/~joel.merker/Enseignement/Integration/l-p-espaces.pdf}
\begin{exercice}
    On considère les espaces $L^p(\R^d)$ pour  $0 < p < +\infty$. Montrer que si l'on a 
    \[
    \norm{f + g}_{L^p} \leqslant  \norm{f}_{L^p} + \norm{g}_{L^p}
    \]
    pour toutes fonctions $f, g \in L^p(\R^d )$, alors nécessairement $p \geqslant 1$.
\end{exercice}

\subsection{Cas $p \to 0$}

\url{https://math.stackexchange.com/questions/2351581/convergence-question-about-lp-norm-when-p-tends-to-zero}

%========
\section{Inégalités intégrales}

%-----------
\subsection{En utilisant l'inégalité de \nom{Cauchy}--\nom{Schwarz}}

%---------------

\begin{prop}%
\source{Oral : ENSAM 2016}
Soit $f$ une fonction de classe $\Cont^2$ sur $\Rp$ et à valeurs réelles telle que $f(0) = 0$. On suppose que $f$ et $f''$ sont de carrés intégrables. Montrer que $f'$ est de carré intégrable et que
\[
\left(\int_0^{+\infty} \big(f'(t)\big)^2\d t \right)^2
\leqslant
\left(\int_0^{+\infty} f^2(t) \d t\right)
\left(\int_0^{+\infty} \big(f''(t)\big)^2 \d t\right).
\]
\end{prop}

\begin{exercice}
Soit $M \geqslant 0$.
\begin{questions}
\item Exprimer $\int_0^M (f'(t))^2 \d t$ en fonction de $\int_0^M f(t) f''(t) \d t$.

\item En déduire que $f'$ est de carré intégrable.

\item Montrer que $\lim\limits_{M\to+\infty} f(M) f'(M) = 0$.

\item Montrer que $\Big(\int_0^M f(t) f''(t) \d t\Big)^2 \leqslant \Big( \int_0^M f(t)^2 \d t \Big) \Big( \int_0^M f''(t)^2 \d t \Big)$.

\item En déduire l'inégalité demandée.
\end{questions}
\end{exercice}

\begin{solution}
\begin{reponses}
\item Comme la fonction $f'$ est de classe $\Cont^1$ sur $\interff{0}{M}$, d'après la formule d'intégration par parties,
\[
\int_0^M f(t) f''(t) \d t = \big(f(M) f'(M) - f(0) f'(0)\big) - \int_0^M \big(f'(t)\big)^2 \d t.
\]

\item Supposons par l'absurde que $f'$ ne soit pas de carré intégrable. Comme $M~\mapsto~\int_0^M \big(f'(t)\big)^2 \d t$ est croissante, alors $\lim\limits_{M\to+\infty} \int_0^M \big(f'(t)\big)^2 \d t = +\infty$. D'après la formule d'intégration par parties, comme $f$ est de carré intégrable, $\lim\limits_{M\to+\infty} f(M) f'(M) = +\infty$.

Alors, il existe un réel $M_0$ à partir duquel la fonction $f(M) f(M')$ est plus grande que~$2$. Ainsi, pour $M \geqslant M_0$,
\[
f(M)^2 - f(M_0)^2 = 2 \int_{M_0}^M f(t) f'(t) \d t \geqslant 2 (M - M_0),
\]
et $f$ n'est pas de carré intégrable.

Finalement, $f'$ est bien de carré intégrable.

\item En reprenant la formule d'intégration par parties, comme $M \mapsto \int_0^M f(t) f''(t) \d t$ et $M \mapsto \int_0^M \big(f'(t)\big)^2 \d t$ admettent une limite à l'infini, alors $M \mapsto f(M) f'(M)$ admet également une limite.

De plus, l'argument précédent assure que cette limite doit être nulle.

\item En utilisant l'\theoremeutilise{inégalité de \nom{Cauchy}--\nom{Schwarz}}{theo:inegalitecs} sur $\interff{0}{M}$,
\[
\left(\int_0^M f(t) f''(t) \d t\right)^2 \leqslant \left(\int_0^M f(t)^2 \d t\right) \left( \int_0^M \big(f''(t)\big)^2 \d t \right).
\]

\item Comme $f(0) = 0$, alors
\[
\int_0^{+\infty} f(t) f''(t) \d t = - \int_0^{+\infty} \big(f'(t)\big)^2 \d t.
\]
On conclut en passant à la limite dans l'inégalité de \nom{Cauchy}--\nom{Schwarz}.
\end{reponses}
\end{solution}

\begin{remarque}
Le cas d'égalité correspond au cas d'égalité dans l'inégalité de \nom{Cauchy}--\nom{Schwarz}. Ici,
\begin{itemize}
\item soit $f'' = 0$ et il existe $(a, b) \in \R^2$ tel que $\fonctionligne[f]{t}{a t + b}$. Comme $f$ est de carré intégrable, alors $f \equiv 0$.

\item soit il existe $\lambda \in \R$ tel que $f = \lambda f''$. Alors,
\[
f \in \ens[\big]{t \mapsto A \cos(\omega t + \phi),\, t \mapsto A \cosh(\omega t + \phi) \tq (\omega,\, \phi) \in \R^2}.
\]
La seule fonction de carré intégrable de cet ensemble est la fonction nulle.
\end{itemize}
Ainsi, l'égalité est atteinte uniquement par la fonction nulle.
\end{remarque}

%---------------

\begin{prop}
\source{Oral : Mines 2017}
Soit $f \in \Cont^1(\Rp,\R)$ telle que $f$ et $f'$ soient de carré intégrable sur $\Rp$. Alors,
\[
\abs{f(0)}^4 \leqslant 2 \left(\int_0^{+\infty} f^2(t) \d t\right) \left(\int_0^{+\infty} \big(f'(t)\big)^2 \d t \right).
\]
\end{prop}

\begin{exercice}
Soit $M \geqslant 0$.
\begin{questions}
\item Montrer que $\int_0^M f(t) f'(t) \d t = \frac{f(M)^2 - f(0)^2}{2}$.

\item En déduire que $\lim\limits_{M\to+\infty} f(M) = 0$.

\item Conclure.
\end{questions}
\end{exercice}

\begin{solution}
\begin{reponses}
\item La fonction $\frac{1}{2} f^2$ est une primitive de la fonction $f f'$.

\item Comme $f$ et $f'$ sont de carré intégrable, alors $M \mapsto \int_0^M f(t) f'(t) \d t$ converge. Ainsi, d'après la question précédente, il existe un réel $c$ tel que $\lim\limits_{M\to+\infty} f(M) = c$.

Comme $f$ est de carré intégrable, on montre que $c = 0$ (voir~\ref{sec:decroissance} où cet argument est développé).

\item En passant à la limite dans l'égalité précédente,
\[
2 \int_0^{+\infty} f(t) f'(t) \d t = f(0)^2.
\]

De plus, en utilisant l'inégalité de \nom{Cauchy}--\nom{Schwarz}, comme $f$ et $f'$ sont de carré intégrable,
\[
|f(0)|^2 \leq 2 \int_0^{+\infty} f(t)^2 \d t \cdot \int_0^{+\infty} (f'(t))^2 \d t.
\]
\end{reponses}
\end{solution}

%--------------
\subsubsection{Inégalité de \nom{Poincaré}}

\begin{prop}[Inégalité de \nom{Poincaré}]
Soit $f : [0, 1] \to \R$ une fonction de classe $\mathscr{C}^1$ telle que \mbox{$f(0) = f(1) = 0$}. Alors,
\[
\int_0^1 f(x)^2 \d x \leq \frac{1}{\pi^2} \int_0^1 f'(x)^2 \d x.
\]
\end{prop}

\begin{exercice}
Soit $f$ une fonction de classe $\mathscr{C}^1$ telle que $f(0) = f(1) = 0$.
\begin{questions}
\item Déterminer des équivalents de $x \mapsto \cotan(\pi x)$ en $x = 0$ puis en $x = 1$.
\end{questions}

On pose
\[
I = \int_0^1 f(x) f'(x) \cotan(\pi x) \d x.
\]

\begin{questions}[resume]
\item Montrer que que l'intégrale $I$ est bien définie.

\item Soit $0 < a < b < 1$. Appliquer une intégration par parties à l'intégrale
\[
\int_a^b f(x) f'(x) \cotan(\pi x) \d x.
\]

\item En déduire que l'on a
\[
2 \pi I = \pi^2 \int_0^1 f(x)^2 (1 + \cotan(\pi x)^2) \d x.
\]
\end{questions}

On considère l'intégrale $J$ définie par
\[
J = \int_0^1 \left(f'(x) - \pi f(x) \cotan(\pi x)\right)^2 \d x.
\]

\begin{questions}[resume]
\item Montrer que $J$ est bien définie.

\item Conclure.
\end{questions}
\end{exercice}

\begin{solution}
\begin{reponses}
\item En utilisant les équivalents classiques en $0$, $\cotan(\pi x) = \frac{\cos(\pi x)}{\sin(\pi x)} \sim_0 \frac{1}{\pi x}$.


En effectuant le changement de variable $u = 1 - x$,
\begin{align*}
\cotan(\pi x)
&= \cotan(\pi - \pi u)
% = \frac{\cos(\pi - \pi u)}{\sin(\pi - \pi u)}\\
% &= \frac{-\cos(-\pi u)}{\sin(\pi u)}\\
% &= -\frac{\cos(\pi u)}{\sin(\pi u)}\\
= - \cotan(\pi u)
\underset{u\to0}{\sim} - \frac{1}{\pi u}
\underset{x\to1}{\sim} -\frac{1}{\pi (1 - x)}.
\end{align*}

\item Posons $u : x \mapsto f(x) f'(x) \cotan(\pi x)$.
\begin{itemize}
\item La fonction $u$ est continue sur $]0, 1[$.

\item D'après la question précédente,
$f(x) f'(x) \cotan(\pi x) \sim_0 \frac{f(x) f'(x)}{\pi x}$.

Comme $f$ est dérivable en $0$ et $f(0) = 0$, alors $\lim\limits_{x\to 0} u(x) = \frac{f'(0)^2}{\pi}$ et $u$ est prolongeable par continuité en $0$.

\item D'après la question précédente,
$f(x) f'(x) \cotan(\pi x) \sim_1 \frac{f(x) f'(x)}{\pi (x - 1)}$.

Comme $f$ est dérivable en $1$ et $f(1) = 0$, alors $\lim\limits_{x\to 1} u(x) = \frac{f'(1)^2}{\pi}$ et $u$ est prolongeable par continuité en $1$.
\end{itemize}

Finalement, la fonction $u$ est prolongeable par continuité sur $[0, 1]$ et l'intégrale $\int_0^1 u(x) \d x$ est donc convergente.

\item Posons $u(x) = \frac{f(x)^2}{2}$ et $v(x) = \cotan(\pi x)$. Alors, $u'(x) = f'(x) f(x)$ et $v'(x) = -\pi (1 + \cotan^2(\pi x))$. Les fonctions $u$ et $v$ sont donc de classe $\mathscr{C}^1$ sur $[a, b]$. En utilisant la formule d'intégration par parties,
\begin{align*}
\int_a^b f(x) f'(x) \cotan(\pi x) \d x
&= \left[\frac{f(x)^2}{2} \cotan(\pi x)\right]_a^b + \cdots\\
&\qquad\cdots + \pi \int_a^b \frac{f(x)^2}{2} (1 + \cotan^2(\pi x)) \d x\\
&= \frac{f(a)^2}{2} \cotan(\pi a) - \frac{f(b)^2}{2} \cotan(\pi b) + \cdots\\
&\qquad\cdots + \pi \int_a^b \frac{f(x)^2}{2} (1 + \cotan(\pi x)^2) \d x.
\end{align*}

\item D'une part,
$
f(a)^2 \cotan(\pi a)
\sim_0 \frac{f(a)^2}{\pi a}
\to \frac{f'(0) f(0)}{\pi}
= 0$.


D'autre part, comme l'intégrale $I$ converge,
$\lim\limits_{a\to 0}
\int_a^b f(x) f'(x) \cotan(\pi x) \d x
\to \int_0^b f(x) f'(x) \cotan(\pi x) \d x$.

Ainsi, d'après les \theoremeutilise{théorèmes d'addition des limites}{theo:additionlimites}, la fonction \mbox{$a \mapsto \int_a^b f(x)^2 (1 + \cotan(\pi x)^2) \d x$} admet une limite en $0$ et
\begin{align*}
\int_0^b f(x) f'(x) \cotan(\pi x) \d x
&= - \frac{f(b)^2}{2} \cotan(\pi b) + \frac{\pi}{2} \int_0^1 f(x)^2 (1 + \cotan(\pi x)^2) \d x.
\end{align*}

\medskip

Un raisonnement analogue montre que
\[
\int_0^1 f(x) f'(x) \cotan(\pi x) \d x = \frac{\pi}{2} \int_0^1 f(x)^2 (1 + \cotan(\pi x)^2) \d x.
\]

Finalement, on obtient bien
\[
2 \pi I = \pi^2 \int_0^1 f(x)^2 (1 + \cotan(\pi x)^2) \d x.
\]

\item Nous avons déjà montré que $\lim\limits_{x\to0} \pi f(x) \cotan(\pi x) = f'(0)$ et $\lim\limits_{x\to1} \pi f(x) \cotan(\pi x) = f'(1)$.

Ainsi, $x \mapsto f'(x) - \pi f(x) \cotan(\pi x)$ est prolongeable par continuité par $0$ en $0$ et en $1$.

Comme cette fonction est par ailleurs continue sur $]0, 1[$, l'intégrale $J$ est bien convergente.

\item Comme $J$ est l'intégrale d'une fonction positive, alors $J \geq 0$. De plus, en utilisant la linéarité des intégrales convergentes,
\begin{align*}
0 \leq J &= \int_0^1 \left[f'(x)^2 - 2 \pi f(x) f'(x) \cotan(\pi x) + \pi^2 f(x)^2 \cotan(\pi x)^2\right] \d x\\
&= \int_0^1 f'(x)^2 \d x - 2 \pi I + \pi^2 \int_0^1 f(x)^2 \cotan(\pi x)^2 \d x\\
&= \int_0^1 f'(x)^2 \d x - \pi^2 \int_0^1 f(x)^2 (1 + \cotan(\pi x)^2) \d x + \cdots\\
&\qquad\qquad\qquad\qquad\qquad  \cdots + \pi^2 \int_0^1 f(x)^2 (1 + \cotan(\pi x)^2) \d x\\
&= \int_0^1 f'(x)^2 \d x - \pi^2 \int_0^1 f(x)^2 \d x.
\end{align*}

Finalement,
\[
\int_0^1 f(x)^2 \d x \leq \frac{1}{\pi^2} \int_0^1 f'(x)^2 \d x.
\]
\end{reponses}
\end{solution}
\section{Lemme de \textsc{Lebesgue}}

\todoinline{J'ai commenté le marginnote suivant, ça compilait pas avec ! LateX me dit qu'il n'a pas assez de mémoire ! Je ne sais pas ce qui se passe. Une idée ?}
\todoarmand{Étonnant, pour moi ça ne pose pas de problème à la compilation. Je le laisse en commentaire en attendant}
\begin{exercice}
% \marginnote[0cm]{Source : \cite{maths-france} Planche no 37. Intégration sur un segment}
Soit $a < b$.
    \begin{enumerate}
        \item On suppose que $f$ est une fonction de classe $\mathscr{C}^1$ sur $[a, b]$. Montrer que
        $$\lim_{ \lambda \to +\infty} \int_a^b \sin(\lambda t) f(t) \d t = 0.$$
        \item Redémontrer le même résultat en supposant simplement que la fonction $f$ est continue par morceaux sur~$[a, b]$.
    \end{enumerate}
\end{exercice}

\todoinline{Remarque en passant. Pourrait-on tracer les fonctions $t \mapsto \sin(\lambda t) f(t)$ et les aires associées pour comprendre (?) pourquoi elles tendent vers 0 ? Je n'ai aucune intuition ici mais on pourrait essayer, non ?}

\todoarmand{C'est un cas particulier du théorème de Riemann-Lebesgue dans la théorie de Fourier : (où $\mathcal{F}$ est la TF)
Si $f \in L^1\big(\R^d\big)$, alors
\[\mathcal{F} f \in \mathscr{C}^0\big(\R^d\big) \quad \text{et} \quad \lim\limits_{|\xi| \to \infty} \mathcal{F} f(\xi) = 0.
\]
}

\begin{solution}
    \begin{enumerate}
        \item Puisque la fonction $f$ est de classe $\mathscr{C}^1$ sur $[a, b]$, on peut effectuer une intégration par parties qui fournit pour $\lambda > 0$:
        $$\left| \int_a^b f(t) \sin(\lambda t) \d t \right| = \left| \frac{1}{\lambda} \left( -\big[ \cos(\lambda t) f(t) \big]_a^b + \int_a^b f'(t) \cos(\lambda t) \d t  \right) \right| \leqslant \frac{1}{\lambda} \left( |f(a)| + |f(b)| + \int_a^b |f'(t)| \d t \right).$$
        Cette dernière expression tend vers $0$ quand $\lambda$ tend vers $+ \infty$, et donc $\int_a^b f(t) \sin(\lambda t) \d t$ tend vers $0$ quand $\lambda$ tend vers $+\infty$.
        \item Si la fonction $f$ est simplement supposée continue par morceaux, on ne peut donc plus effectuer une intégration par parties. \\
        Le résultat est clair si $f = 1$, car pour $\lambda > 0$, $\left| \int_a^b \sin(\lambda t) \d t \right| = \left| \frac{\cos(\lambda a) - \cos(\lambda b)}{\lambda} \right| \leqslant \frac{2}{\lambda}$. \\
        Le résultat s'étend aux fonctions constantes par linéarité de l'intégrale puis aux fonctions constantes par morceaux par additivité par rapport à l'intervalle d'intégration, c'est-à-dire aux fonctions en escaliers. Pour toute fonction $g$ continue par morceaux sur $[a, b]$, on note $\|g\|_{\infty} = \sup_{[a, b]} |g|$.\\
        Soit alors $f$ une fonction continue par morceaux sur $[a, b]$. \\
        \todoinline{Là on admet un théorème d'approximation non trivial et hors programme en PCSI. Il faut voir ce qu'on indique en introduction du chapitre ?}
        Soit $\varepsilon > 0$. Il existe une fonction en escalier $\varphi$ telle que $\|f - \varphi\|_\infty \leq \varepsilon$. De plus, d'après le point précédent, il existe un réel $\lambda_0$ strictement positif tel que pour tout $\lambda > \lambda_0$,
        \[
        \left|\int_a^b \sin(\lg t) \varphi(t) \d t\right| \leq \varepsilon.
        \]
        Finalement, d'après l'inégalité triangulaire, pour tout $\lambda > \lambda_0$,
        \begin{align*}
        \left|\int_a^b f(t) \sin(\lambda t) \d t\right|
        &\leq         \left|\int_a^b (f(t) - \varphi(t)) \sin(\lambda t) \d t\right| + \left|\int_a^b \varphi(t) \sin(\lambda t) \d t\right|\\
        &\leq \norme{f - \varphi}_\infty (b - a) + \varepsilon\\
        &\leq \varepsilon (1 + b - a).
        \end{align*}
Finalement, $\lim\limits_{\lambda \to +\infty} \int_a^b f(t) \sin(\lambda t) \d t = 0$.
        \end{enumerate}
\end{solution}


% \todoinline{
% La variante proposée ci-dessous me semble difficile.
%
% Le calcul de $\sum \frac{1}{n^2}$ est classique et pourrait être directement généralisé avec St Cyr 1995 - Je mets une version dans le dossier /chapitres/integration/documents. En plus on ferait un peu de polynômes de Bernoulli.
%
% Le calcul de l'intégrale de Dirichlet est top.
% }

% \todoarmand{Cela me convient, nous pouvons supprimer la variante. On pourrait la remplacer par le  mais en renvoyant vers un exercice du chapitre polynômes.}

\todoinline{Insérer III de St Cyr 1995 et indiquer clairement dans la partie polynômes de Bernoulli ce dont on aura besoin ici}

\subsection{Séries de \textsc{Riemann} et nombres de \textsc{Bernoulli}}

\begin{enumerate}
    \item Montrer que pour $N$ entier naturel non nul:
    \[
    \forall t \in \interoo{0}{1}, \quad 1 + 2 \sum_{k=1}^N \cos(2k \pi t) = \frac{\sin\big((2N+1) \pi t \big)}{\sin(\pi t)}.
    \]
    Pour tout entier naturel $n$ strictement positif, on pose:
    \[
    \forall t \in \interoo{0}{1}, \quad \varphi_n(t) = \frac{B_n(t) - B_n(0)}{\sin(\pi t)}.
    \]
    \item Montrer que pour tout entier $n \geqslant 2$, la fonction $\varphi_n$ est prolongeable par continuité à $\interff{0}{1}$ et que le prolongement est de classe $\mathscr{C}^1$.
    \item Montrer (en utilisant éventuellement une intégration par parties) que pour toute fonction $f$ de classe $\mathscr{C}^1$ sur $\interff{0}{1}$, 
    \[
    \lim_{x \to + \infty} \int_0^1 f(t) \sin(x t) \d t = 0.
    \]
    \item Pour $k$ et $n$ entiers strictement positifs, on défnit:
    \[
    I_{n, k} = \int_0^1 B_n(t) \cos(2 k \pi t) \d t.
    \]
    Trouver une relation entre $I_{n,k}$ et $I_{n-2, k}$ et en déduire selon la parité de $n$, l'expression de $I_{n,k}$ en fonction de $n$ et de $k$. 
    \item 
    \begin{enumerate}
        \item En utilisant les questions précédentes, trouver, pour $N$ entier naturel, une expression de 
        \[
        \int_0^1 \varphi_{2m}(t) \sin \big((2N+1) \pi t \big) \d t
        \]
        en fonction de $m$, $N$ et $B_{2m}(0)$.
        \item En déduire la valeur de $\sum\limits_{k=1}^\infty \frac{1}{k^{2m}}$ en fonction de $m$ et de $B_{2m}(0)$.
        \item Donner les valeurs de $\sum\limits_{k=1}^\infty \frac{1}{k^2}$ et de $\sum\limits_{k=1}^\infty \frac{1}{k^4}$.
    \end{enumerate}
    \item Montrer, pour tout $m$ entier naturel non nul, la majoration:
    \[
    \sum_{k=1}^\infty \frac{1}{k^{2m}} \leqslant 2
    \]
    et en déduire la majoration $\module{B_{2m}(0)} \leqslant \frac{4}{(4 \pi^2)^m}$.
\end{enumerate}

\begin{solution}
    
\begin{enumerate}
    \item Pour tout $t \in \interoo{0}{1}$, $\e^{2 \i k \pi t} \not= 1$. Ainsi, d'après la somme des termes d'une suite géométrique, 
    \begin{align*}
        \sum_{k=0}^N \e^{2 \i k \pi t} &= \frac{\e^{2 \i (N+1) \pi} - 1}{\e^{2 \i \pi} - 1} \\
        &= \e^{\i N \pi} \frac{\sin(N+1) \pi t}{\sin(\pi t)}. \\
        \sum_{k=0}^N \cos(2 k \pi t) &= \cos(N \pi t) \frac{\sin \big((N+1) \pi t \big)}{\sin(\pi t)}\text{, en prenant les parties réelles,} \\
        1 + 2 \sum_{k=1}^N \cos(2 k \pi t) &= 2 \frac{\cos(N \pi t) \sin \big((N+1) \pi t\big)}{\sin(\pi t)} - 1 \\
        &= \frac{\sin\big((2N+1) \pi t \big) + \sin( \pi t)}{\sin(\pi t)} - 1 \\
        &= \frac{\sin\big((2N+1) \pi t \big)}{\sin(\pi t)}.
    \end{align*}
    \item La fonction $B_n$ étant polynomiale, elle est de classe $\mathscr{C}^1$ en $0$ et, en utilisant la formule de \textsc{Taylor}--\textsc{Young},
    \begin{align*}
        \varphi_n(t) &= \frac{B'_n(0)t + \frac{B''_n(0)}{2}t^2 + o(t^2)}{\pi t \big(1 + o(t) \big)} \\
        &= \frac{B'_n(0)}{\pi} + \frac{B''_n(0)}{2 \pi}t + o(t).
    \end{align*}
    Ainsi, $\lim\limits_0 \varphi_n = \frac{B'_n(0)}{\pi}$ et $\varphi_n$ est une fonction prolongeable par continuité en $0$. De plus, $\lim\limits_{t \to 0} \frac{\varphi_n(t) - \frac{B'_n(0)}{\pi}}{t} = \frac{B''_n(0)}{2 \pi}$. \\
    De plus, pour tout $t$ non nul, $\varphi'_n(t) = \frac{B'_n(t) \sin(\pi t) - \big(B_n(t) - B_n(0) \big) \pi \cos(\pi t)}{\sin(\pi t)^2}$. Ainsi, en effectuant un développement limité à l'ordre $2$ du numérateur, alors $\lim\limits_{t \to 0} \varphi'_n(t) = \frac{1}{2 \pi} B''_n(0)$. \\
    Ainsi, $\varphi_n$ est prolongeable en une fonction de classe $\mathscr{C}^1$ sur $\interfo{0}{1}$. \\
    Enfin, $\varphi_n(1-t) = (-1)^n \frac{B_n(t) - B_n(1)}{\sin(\pi t)}$. Comme, pour tout $n \geqslant 2$, $B_n(0) = B_n(1)$, alors la fonction $\varphi_n$ est bien prolongeable en une fonction de classe $\mathscr{C}^1$ sur $\interff{0}{1}$.
    \item Soit $f$ une fonction continue sur $\interff{0}{1}$. Comme les fonctions $t \mapsto f(t)$ et $t \mapsto \sin(xt)$ sont de classe $\mathscr{C}^1$ sur $\interff{0}{1}$, d'après la formule d'intégration par parties, 
    \begin{align*}
        \int_0^1 f(t) \sin(xt) \d t &= \frac{f(1) \cos(x) - f(0)}{x} - \frac{1}{x} \int_0^1 f'(t) \cos(xt) \d t \\
        \module{\int_0^1 f(t) \sin(x t) \d t} &\leqslant \frac{\module{f(1)} + \module{f(0)}}{\module{x}} + \frac{1}{\module{x}} \int_0^1 \module{f'(t)} \d t.
    \end{align*}
    Ainsi, d'après le théorème d'encadrement, 
    \[
    \lim_{x \to \infty} \int_0^1 f(t) \sin(xt) \d t = 0.
    \]
    \item On utilise deux intégrations par parties successives pour obtenir
    \[
    I_{n,k} = \frac{1}{4k^2 \pi^2} \big(B_{n-1}(1) - B_{n-1}(0) - I_{n-2, k} \big).
    \]
    Ainsi, $I_{0,k} = 0$, $I_{1,k} = 0$, $I_{2,k} = \frac{1}{4 \pi^2}$ et, pour tout $n \geqslant 3, I_{n,k} = - \frac{1}{4 k^2 \pi^2}I_{n-2, k}$. Ainsi, 
    \[
    \forall p > 0, \quad I_{2p, k} = \frac{(-1)^{p-1}}{(2 k \pi)^{2p}} \quad \text{et} \quad I_{2p+1,k} = 0.
    \]
    \item 
    \begin{enumerate}
        \item D'après la définition de $\varphi_{2m}$,
        \begin{align*}
            \int_0^1 \varphi_{2m}(t) \sin \big((2N+1) \pi t \big) \d t &= \int_0^1 \big(B_{2m}(t) - B_{2m}(0) \big) \frac{\sin\big((2N+1) \pi t \big)}{\sin(\pi t)} \d t \\
            &= \int_0^1 \big( B_{2m}(t) - B_{2m}(0) \big) \d t + \cdots \\
            &\cdots + 2 \sum_{k=1}^N \int_0^1 \big(B_{2m}(t) - B_{2m}(0) \big) \cos(2 k \pi t) \d t \\
            &= - B_{2m}(0) + 2 \sum_{k=1}^N \frac{(-1)^{m-1}}{(2 k \pi)^{2m}}.
        \end{align*}
        \item Comme la fonction $\varphi_{2m}$ est de classe $\mathscr{C}^1$ sur $\interff{0}{1}$, alors en utilisant les questions précédentes, 
        \[
        \sum_{k=1}^{+\infty} \frac{1}{k^{2m}} = (-1)^{m-1} 2^{2m-1} \pi^{2m} B_{2m}(0).
        \]
        \item D'après la question précédente, 
        \begin{align*}
            \sum_{k=1}^{+\infty} \frac{1}{k^2} &= 2 \pi^2 B_2(0) = \frac{\pi^2}{6} \\
            \sum_{k=1}^{+\infty} \frac{1}{k^4} &= -2^3 \pi^4 B_4(0) = \frac{\pi^4}{90}.
        \end{align*}
    \end{enumerate}
    \item Comme, pour $k \geqslant 2$, $k^{2m} \geqslant 4^m$, alors
    \[
    \sum_{k=1}^{+\infty} \frac{1}{k^{2m}} \leqslant 1 + \sum_{k=2}^{+\infty} \frac{1}{4^m} = 1 + \frac{1}{16} \cdot \frac{4}{3} < 2.
    \]
    Ainsi, 
    \[
    \module{B_{2m}(0)} \leqslant \frac{4}{(4\pi^2)^m}.
    \]
\end{enumerate}

\end{solution}

\todoinline{À voir si on laisse ici ou si on met directement dans la partie Dirichlet.}

\begin{exercice}
\begin{enumerate}
    \item Justifier l'existence de l'intégrale $I = \int_0^{+\infty} \frac{\sin t}{t} \d t$. 
    \item Calculer $I_n = \int_0^\pi \frac{\sin \left(n + \frac{1}{2}\right)t}{2 \sin \frac{t}{2}} \d t$. \\
    \emph{Indication :} Calculer $I_{n+1} - I_n$. 
    \item Montrer que la fonction définie sur $\interof{0}{\pi}$ par $f(x) = \frac{1}{x} - \frac{1}{2 \sin \frac{x}{2}}$ peut être prolongée à $\interff{0}{\pi}$ en une fonction de classe $\mathscr{C}^1$. 
    \item Soit $f$ une fonction de classe $\mathscr{C}^1$ sur l'intervalle $\interff{a}{b}$. Montrer que 
    \[
    \lim_{\lambda \to +\infty} \int_a^b \sin(\lambda t) f(t) \d t = 0.
    \]
    \item En déduire la valeur de l'intégrale $I$.
\end{enumerate}
\end{exercice}
\section{Lemme de \textsc{Lebesgue}}\label{sec:lemmeLebesgue}

\todoinline{Parvenir à décommenter le marginnote suivant. D'après mes recherches sur internet, il faudrait "externaliser" la compilation des figures en pdflatex qui demandent trop de ressources.}
\todoarmand{Je n'ai pas de problème de compilation avec le marginnote.}

\marginnote[0cm]{Source : \cite{maths-france} Planche no 37. Intégration sur un segment}
\begin{marginfigure}[-2cm]
    \includegraphics[width=0.9\textwidth]{illustrations/integration-02_lebesgue.png}
    \caption{Illustration des aires compensées dans le cadre du lemme de \textsc{Lebesgue} \todoarmand{titre à modifier}}
\end{marginfigure}
\begin{lemme}\label{lemmeLebesgue} Soit $a < b$.
\begin{enumerate}
\item On suppose que $f$ est une fonction de classe $\mathscr{C}^1$ sur $\interff{a}{b}$. Alors,
\[
\lim_{ \lambda \to +\infty} \int_a^b \sin(\lambda t) f(t) \d t = 0.
\]
\item Redémontrer le même résultat en supposant simplement que la fonction $f$ est continue par morceaux sur~$\interff{a}{b}$.
\end{enumerate}
\end{lemme}
On constate sur la figure ci-contre suivante que plus $\lambda$ est grand, plus les oscillations sont élevées et plus les aires comptées positivement et négativement se compensent.
\begin{solution}
    \begin{enumerate}
        \item Puisque la fonction $f$ est de classe $\mathscr{C}^1$ sur $\interff{a}{b}$, on peut effectuer une intégration par parties qui fournit pour $\lambda > 0$:
        \[
        \begin{multlined}[t]
        \left| \int_a^b f(t) \sin(\lambda t) \d t \right| = \left| \frac{1}{\lambda} \left( -\big[ \cos(\lambda t) f(t) \big]_a^b + \int_a^b f'(t) \cos(\lambda t) \d t  \right) \right| \\
        \leqslant \frac{1}{\lambda} \left( |f(a)| + |f(b)| + \int_a^b |f'(t)| \d t \right),
        \end{multlined}
        \]
        par inégalité triangulaire et majoration de la fonction cosinus par $1$. \\
        Cette dernière expression tend vers $0$ lorsque $\lambda$ tend vers $+ \infty$, et donc l'intégrale $\int_a^b f(t) \sin(\lambda t) \d t$ tend bien vers $0$ lorsque $\lambda$ tend vers $+\infty$.
        \item Si la fonction $f$ est simplement supposée continue par morceaux, on ne peut plus effectuer une intégration par parties. \\
        Le résultat est clair si $f = 1$, car pour $\lambda > 0$, $\left| \int_a^b \sin(\lambda t) \d t \right| = \left| \frac{\cos(\lambda a) - \cos(\lambda b)}{\lambda} \right| \leqslant \frac{2}{\lambda}$. \\
        Le résultat s'étend aux fonctions constantes par linéarité de l'intégrale puis aux fonctions constantes par morceaux par additivité par rapport à l'intervalle d'intégration, c'est-à-dire aux fonctions en escaliers. 
        \todoarmand{J'ai commenté la définition de la norme $L^\infty$ puisqu'elle est donnée dans l'index des notations.}
        % Pour toute fonction $g$ continue par morceaux sur $[a, b]$, on note $\|g\|_{\infty} = \sup_{[a, b]} |g|$.\\
        Considérons donc $f$ une fonction continue par morceaux sur $\interff{a}{b}$. \\
        \todoinline{Là on admet un théorème d'approximation non trivial et hors programme en PCSI. Il faut voir ce qu'on indique en introduction du chapitre ?}
        \todoarmand{On pourrait donner le théorème au début du chapitre, en précisant les filières pour lesquelles il est hors programme.}
        Soit $\varepsilon > 0$, il existe une fonction en escalier $\varphi$ telle que $\|f - \varphi\|_\infty \leqslant \varepsilon$. De plus, d'après le point précédent, il existe un réel $\lambda_0$ strictement positif tel que pour tout $\lambda > \lambda_0$,
        \[
        \left| \varphi(t) \int_a^b \sin(\lambda t) \d t\right| \leqslant \varepsilon.
        \]
        Finalement, d'après l'inégalité triangulaire, pour tout $\lambda > \lambda_0$,
        \begin{align*}
        \left|\int_a^b f(t) \sin(\lambda t) \d t\right|
        &\leqslant         \left|\int_a^b \big(f(t) - \varphi(t)\big) \sin(\lambda t) \d t\right| + \left|\int_a^b \varphi(t) \sin(\lambda t) \d t\right|\\
        &\leqslant \norme{f - \varphi}_\infty (b - a) + \varepsilon\\
        &\leqslant \varepsilon (1 + b - a).
        \end{align*}
Finalement, $\lim\limits_{\lambda \to +\infty} \int_a^b f(t) \sin(\lambda t) \d t = 0$.
        \end{enumerate}
\end{solution}

% \todoinline{
% La variante proposée ci-dessous me semble difficile.
%
% Le calcul de $\sum \frac{1}{n^2}$ est classique et pourrait être directement généralisé avec St Cyr 1995 - Je mets une version dans le dossier /chapitres/integration/documents. En plus on ferait un peu de polynômes de Bernoulli.
%
% Le calcul de l'intégrale de Dirichlet est top.
% }

% \todoarmand{Cela me convient, nous pouvons supprimer la variante. On pourrait la remplacer par le  mais en renvoyant vers un exercice du chapitre polynômes.}

\bigskip

Nous allons utiliser le \hyperref[lemmeLebesgue]{lemme de \textsc{Lebesgue}} pour calculer certaines valeurs de la fonction $\zeta$ de \textsc{Riemann}. La fonction zêta de \textsc{Riemann} est également abordée dans la sous-section \ref{subsec:fonctionZeta}. Nous verrons ultérieurement une autre utilisation au calcul de la valeur de l'\hyperref[sec:intDirichlet]{intégrale de \textsc{Dirichlet}}.

\todoinline{Partie re-rédigée, à relire !}

\todoinline{Dans la partie sur Bernoulli, il faudra :
la symétrie, la dérivée et les valeurs de $B_2(0)$ et $B_4(0)$}

\todoarmand{La symétrie et la dérivée sont faites dans l'exercice et j'ai ajouté un tableau avec les premières valeurs de nombres de Bernoulli}

On note $(B_n)$ la suite des \hyperref[sec:polynomes_de_bernoulli]{polynômes de \textsc{Bernoulli}} et, pour tout $x > 1$, on introduit la fonction $\zeta(x) = \sum\limits_{k=1}^{\infty} \frac{1}{k^x}$.

\todoarmand{Il y a bien un lien vers le thème sur les polynômes de Bernoulli}

\begin{theo}
Pour tout entier $m \geqslant 1$, $\zeta(2m) = (-1)^{m-1} (2 \pi)^{2m} \frac{B_{2m}(0)}{2}$.
\end{theo}

L'exercice suivant, issu du concours de l'\textsc{esm} Saint-Cyr 1995, donne une démonstration de ce résultat. 
\begin{exercice}
\begin{enumerate}
    \item Pour $k$ et $n$ entiers strictement positifs \todoarmand{ou bien Pour tout $k,n \in \Ne$}, on définit les intégrales
    \[
    I_{n,k} = \int_0^1 B_n(t) \cos(2 k \pi t) \d t.
    \]
    Trouver une relation entre $I_{n,k}$ et $I_{n-2, k}$ et en déduire selon la parité de $n$, l'expression de $I_{n,k}$ en fonction de $n$ et de $k$.
\end{enumerate}
    Pour tout entier naturel $n$ strictement positif, on pose:
    \[
    \forall t \in \interoo{0}{1}, \quad \varphi_n(t) = \frac{B_n(t) - B_n(0)}{\sin(\pi t)}.
    \]
\begin{enumerate}[resume]
    \item Montrer que pour tout entier $n \geqslant 2$, la fonction $\varphi_n$ est prolongeable par continuité à $\interff{0}{1}$ et que le prolongement est de classe $\mathscr{C}^1$.
    \item Montrer que pour $N$ entier naturel non nul:
    \[
    \forall t \in \interoo{0}{1}, \quad 1 + 2 \sum_{k=1}^N \cos(2k \pi t) = \frac{\sin\mathopen{}\big((2N+1) \pi t \big)}{\sin(\pi t)}.
    \]
    % \item Montrer (en utilisant éventuellement une intégration par parties) que pour toute fonction $f$ de classe $\mathscr{C}^1$ sur $\interff{0}{1}$, 
    % \[
    % \lim_{x \to +\infty} \int_0^1 f(t) \sin(x t) \d t = 0.
    % \]
    \item \begin{enumerate}
        \item En utilisant les questions précédentes, trouver, pour $N$ un entier naturel, une expression de l'intégrale
        \[
        \int_0^1 \varphi_{2m}(t) \sin\mathopen{}\big( (2N+1) \pi t \big) \d t
        \]
        en fonction de $m$, $N$ et de $B_{2m}(0)$.
        \item En déduire la valeur de $\sum\limits_{k=1}^\infty \frac{1}{k^{2m}}$ en fonction de $m$ et de $B_{2m}(0)$.
        \end{enumerate}
    %     \item Donner les valeurs de $\sum\limits_{k=1}^\infty \frac{1}{k^2}$ et de $\sum\limits_{k=1}^\infty \frac{1}{k^4}$.
    % \end{enumerate}
    % \item Montrer, pour tout $m$ entier naturel non nul, la majoration:
    % \[
    % \sum_{k=1}^\infty \frac{1}{k^{2m}} \leqslant 2
    % \]
    % et en déduire la majoration $\module{B_{2m}(0)} \leqslant \frac{4}{(4 \pi^2)^m}$.
\end{enumerate}
\end{exercice}

\todoarmand{Faire un encadrement rappel avec à nouveau un lien vers le thème Bernoulli}
On rappelle que 
\begin{align*}
B_n(1 - x) &= (-1)^n B_n(x),\\
B_n'(x) &= n B_{n-1}(x).
\end{align*}

\begin{elemsolution}
\begin{enumerate}
\item Pour tout entier $p > 0$,
\[
I_{2p, k} = \frac{(-1)^{p-1}}{(2 k \pi)^{2p}} \quad \text{et} \quad I_{2p+1,k} = 0.
\]

En effet, en utilisant deux intégrations par parties successives,
\[
I_{n,k} = \frac{1}{4k^2 \pi^2} \big(B_{n-1}(1) - B_{n-1}(0) - I_{n-2, k} \big).
\]
De plus, $I_{0,k} = 0$, $I_{1,k} = 0$, $I_{2,k} = \frac{1}{4 \pi^2}$. Donc,
\[
\forall n \geqslant 3,\quad I_{n,k} = - \frac{1}{4 k^2 \pi^2}I_{n-2, k}.
\]
On obtient ainsi le résultat annoncé.
\item Pour tout $n \geqslant 2$, la fonction $\varphi_n$ est prolongeable par continuité à $\interff{0}{1}$ et le prolongement est de classe $\mathscr{C}^1$. En effet,

\begin{itemize}
\item D'après les quotients de fonctions de classe $\mathscr{C}^1$ dont le dénominateur ne s'annule pas, la fonction $\varphi_n$ est de classe $\mathscr{C}^1$ sur $\interoo{0}{1}$.

\item La fonction $B_n$ étant polynomiale, elle est de classe $\mathscr{C}^1$ en $0$ et, en utilisant la formule de \textsc{Taylor}--\textsc{Young},
\begin{align*}
\varphi_n(t) &= \frac{B'_n(0)t + \frac{B''_n(0)}{2}t^2 + o(t^2)}{\pi t \big(1 + o(t) \big)} \\
&= \frac{B'_n(0)}{\pi} + \frac{B''_n(0)}{2 \pi} \, t + o(t).
\end{align*}

Ainsi, $\lim\limits_{t \to 0} \varphi_n(t) = \frac{B'_n(0)}{\pi}$ et $\varphi_n$ est une fonction prolongeable par continuité en $0$.

\item
% De plus, $\lim\limits_{t \to 0} \frac{\varphi_n(t) - \frac{B'_n(0)}{\pi}}{t} = \frac{B''_n(0)}{2 \pi}$.
%
De plus, pour tout $t$ non nul, $\displaystyle \varphi'_n(t) = \frac{B'_n(t) \sin(\pi t) - \big(B_n(t) - B_n(0) \big) \pi \cos(\pi t)}{\sin(\pi t)^2}$. Ainsi, en effectuant un développement limité à l'ordre $2$ du numérateur, on obtient $\lim\limits_{t \to 0} \varphi'_n(t) = \frac{1}{2 \pi} B''_n(0)$.
\marginnote[-7pt]{\theoremeutilise{\hyperref[thm:prolongementDesDerivees]{Théorème de prolongement des dérivées}} \todoarmand{Mettre en évidence les théorèmes / prop. utilisés ?}}D'après le théorème de prolongement des dérivées, $\varphi_n$ est prolongeable en une fonction de classe $\mathscr{C}^1$ sur $\interfo{0}{1}$.

Enfin, $\varphi_n(1-t) = (-1)^n \frac{B_n(t) - B_n(1)}{\sin(\pi t)}$. Comme pour tout $n \geqslant 2$, $B_n(0) = B_n(1)$, la fonction $\varphi_n$ est bien prolongeable en une fonction de classe $\mathscr{C}^1$ sur $\interff{0}{1}$.
\end{itemize}
\end{enumerate}

Pour tout $N$ entier naturel non nul et $t \in \interoo{0}{1}$, on pose :
\[
D_n(t) = 1 + 2 \sum_{k=1}^N \cos(2k \pi t) = \frac{\sin\mathopen{}\big((2N+1) \pi t \big)}{\sin(\pi t)}.
\]

\begin{enumerate}[resume]
\item Cette quantité, appelée \textsl{noyau de \textsc{Dirichlet}}\marginnote{\todoarmand{\href{https://www.bibmath.net/dico/index.php?action=affiche&quoi=./n/noyautrigo.html}{Noyaux trigonométriques}, le noyau de \textsc{Dirichlet} intervient aussi \href{https://fr.wikipedia.org/wiki/Noyau_de_Dirichlet}{en optique}.}}, s'exprime simplement à l'aide de la fonction sinus :
\[
D_n(t) = \frac{\sin\mathopen{}\big((2N+1) \pi t \big)}{\sin(\pi t)}.
\]
En effet, pour tout $t \in \interoo{0}{1}$, $\e^{2 \i k \pi t} \not= 1$. Ainsi, d'après la somme des termes d'une suite géométrique, 
    \begin{align*}
        \sum_{k=0}^N \e^{2 \i k \pi t} &= \frac{\e^{2 \i (N+1) \pi} - 1}{\e^{2 \i \pi} - 1} \\
        &= \e^{\i N \pi} \frac{\sin(N+1) \pi t}{\sin(\pi t)}. \\
        \intertext{En prenant les parties réelles,}
        \sum_{k=0}^N \cos(2 k \pi t) &= \cos(N \pi t) \frac{\sin\mathopen{}\big((N+1) \pi t \big)}{\sin(\pi t)} \\
        1 + 2 \sum_{k=1}^N \cos(2 k \pi t) &= 2 \frac{\cos(N \pi t) \sin\mathopen{}\big((N+1) \pi t\big)}{\sin(\pi t)} - 1 \\
        &= \frac{\sin\mathopen{}\big((2N+1) \pi t \big) + \sin( \pi t)}{\sin(\pi t)} - 1 \\
        &= \frac{\sin\mathopen{}\big((2N+1) \pi t \big)}{\sin(\pi t)}.
    \end{align*}

\item 
\begin{enumerate}
    \item Ainsi,
\[
\int_0^1 \varphi_{2m}(t) \sin\mathopen{}\big((2N+1) \pi t \big) \d t
= - B_{2m}(0) + 2 \sum_{k=1}^N \frac{(-1)^{m-1}}{(2 k \pi)^{2m}}.
\]

En effet, d'après la définition de $\varphi_{2m}$,
\begin{align*}
\int_0^1 \varphi_{2m}(t) \sin\mathopen{}\big((2N+1) \pi t \big) \d t &= \int_0^1 \big(B_{2m}(t) - B_{2m}(0) \big) \frac{\sin\mathopen{}\big((2N+1) \pi t \big)}{\sin(\pi t)} \d t \\
&= \int_0^1 \big( B_{2m}(t) - B_{2m}(0) \big) \d t + \cdots \\
&\quad \cdots + 2 \sum_{k=1}^N \int_0^1 \big(B_{2m}(t) - B_{2m}(0) \big) \cos(2 k \pi t) \d t \\
&= - B_{2m}(0) + 2 \sum_{k=1}^N \frac{(-1)^{m-1}}{(2 k \pi)^{2m}}.
\end{align*}
\item D'après le lemme de \textsc{Lebesgue},
\[
\lim_{N\to+\infty} \int_0^1 \varphi_{2m}(t) \sin\mathopen{}\big((2N+1) \pi t \big) \d t = 0
\]
et on obtient bien le résultat annoncé.
\end{enumerate}
\end{enumerate}
\end{elemsolution}

\begin{remarque}
En utilisant les valeurs remarquables des premiers polynômes de \textsc{Bernoulli}, on obtient
\begin{align*}
\sum_{k=1}^{+\infty} \frac{1}{k^2} &= 2 \pi^2 B_2(0) = \frac{\pi^2}{6} \\
\sum_{k=1}^{+\infty} \frac{1}{k^4} &= -2^3 \pi^4 B_4(0) = \frac{\pi^4}{90}.
\end{align*}
\end{remarque}

\section{Lemme de \textsc{Lebesgue}}

\todoinline{J'ai commenté le marginnote suivant, ça compilait pas avec ! LateX me dit qu'il n'a pas assez de mémoire ! Je ne sais pas ce qui se passe. Une idée ?}
\todoarmand{Étonnant, pour moi ça ne pose pas de problème à la compilation. Je le laisse en commentaire en attendant}
\begin{exercice}
% \marginnote[0cm]{Source : \cite{maths-france} Planche no 37. Intégration sur un segment}
Soit $a < b$.
    \begin{enumerate}
        \item On suppose que $f$ est une fonction de classe $\mathscr{C}^1$ sur $[a, b]$. Montrer que
        $$\lim_{ \lambda \to +\infty} \int_a^b \sin(\lambda t) f(t) \d t = 0.$$
        \item Redémontrer le même résultat en supposant simplement que la fonction $f$ est continue par morceaux sur~$[a, b]$.
    \end{enumerate}
\end{exercice}

\todoinline{Remarque en passant. Pourrait-on tracer les fonctions $t \mapsto \sin(\lambda t) f(t)$ et les aires associées pour comprendre (?) pourquoi elles tendent vers 0 ? Je n'ai aucune intuition ici mais on pourrait essayer, non ?}

\todoarmand{Bonne idée ! Je vais faire des expériences avec Geogebra pour essayer de comprendre ce qu'il se passe}
\todoarmand{Ça me fait penser à la modulation en amplitude des ondes. Les fonctions $f$ et $-f$ définissent une enveloppe dans laquelle la fonction $t \mapsto \sin(\lambda t) f(t)$ oscille et $\lambda$ joue le rôle de pulsation.}

\begin{solution}
    \begin{enumerate}
        \item Puisque la fonction $f$ est de classe $\mathscr{C}^1$ sur $[a, b]$, on peut effectuer une intégration par parties qui fournit pour $\lambda > 0$:
        $$\left| \int_a^b f(t) \sin(\lambda t) \d t \right| = \left| \frac{1}{\lambda} \left( -\big[ \cos(\lambda t) f(t) \big]_a^b + \int_a^b f'(t) \cos(\lambda t) \d t  \right) \right| \leqslant \frac{1}{\lambda} \left( |f(a)| + |f(b)| + \int_a^b |f'(t)| \d t \right).$$
        Cette dernière expression tend vers $0$ quand $\lambda$ tend vers $+ \infty$, et donc $\int_a^b f(t) \sin(\lambda t) \d t$ tend vers $0$ quand $\lambda$ tend vers $+\infty$.
        \item Si la fonction $f$ est simplement supposée continue par morceaux, on ne peut donc plus effectuer une intégration par parties. \\
        Le résultat est clair si $f = 1$, car pour $\lambda > 0$, $\left| \int_a^b \sin(\lambda t) \d t \right| = \left| \frac{\cos(\lambda a) - \cos(\lambda b)}{\lambda} \right| \leqslant \frac{2}{\lambda}$. \\
        Le résultat s'étend aux fonctions constantes par linéarité de l'intégrale puis aux fonctions constantes par morceaux par additivité par rapport à l'intervalle d'intégration, c'est-à-dire aux fonctions en escaliers. Pour toute fonction $g$ continue par morceaux sur $[a, b]$, on note $\|g\|_{\infty} = \sup_{[a, b]} |g|$.\\
        Soit alors $f$ une fonction continue par morceaux sur $[a, b]$. \\
        \todoinline{Là on admet un théorème d'approximation non trivial et hors programme en PCSI. Il faut voir ce qu'on indique en introduction du chapitre ?}
        Soit $\varepsilon > 0$. Il existe une fonction en escalier $\varphi$ telle que $\|f - \varphi\|_\infty \leq \varepsilon$. De plus, d'après le point précédent, il existe un réel $\lambda_0$ strictement positif tel que pour tout $\lambda > \lambda_0$,
        \[
        \left|\int_a^b \sin(\lg t) \varphi(t) \d t\right| \leq \varepsilon.
        \]
        Finalement, d'après l'inégalité triangulaire, pour tout $\lambda > \lambda_0$,
        \begin{align*}
        \left|\int_a^b f(t) \sin(\lambda t) \d t\right|
        &\leq         \left|\int_a^b (f(t) - \varphi(t)) \sin(\lambda t) \d t\right| + \left|\int_a^b \varphi(t) \sin(\lambda t) \d t\right|\\
        &\leq \norme{f - \varphi}_\infty (b - a) + \varepsilon\\
        &\leq \varepsilon (1 + b - a).
        \end{align*}
Finalement, $\lim\limits_{\lambda \to +\infty} \int_a^b f(t) \sin(\lambda t) \d t = 0$.
        \end{enumerate}
\end{solution}


% \todoinline{
% La variante proposée ci-dessous me semble difficile.
%
% Le calcul de $\sum \frac{1}{n^2}$ est classique et pourrait être directement généralisé avec St Cyr 1995 - Je mets une version dans le dossier /chapitres/integration/documents. En plus on ferait un peu de polynômes de Bernoulli.
%
% Le calcul de l'intégrale de Dirichlet est top.
% }

% \todoarmand{Cela me convient, nous pouvons supprimer la variante. On pourrait la remplacer par le  mais en renvoyant vers un exercice du chapitre polynômes.}

\todoinline{Insérer III de St Cyr 1995 et indiquer clairement dans la partie polynômes de Bernoulli ce dont on aura besoin ici}

\subsection{Séries de \textsc{Riemann} et nombres de \textsc{Bernoulli}}

\begin{enumerate}
    \item Montrer que pour $N$ entier naturel non nul:
    \[
    \forall t \in \interoo{0}{1}, \quad 1 + 2 \sum_{k=1}^N \cos(2k \pi t) = \frac{\sin\big((2N+1) \pi t \big)}{\sin(\pi t)}.
    \]
    Pour tout entier naturel $n$ strictement positif, on pose:
    \[
    \forall t \in \interoo{0}{1}, \quad \varphi_n(t) = \frac{B_n(t) - B_n(0)}{\sin(\pi t)}.
    \]
    \item Montrer que pour tout entier $n \geqslant 2$, la fonction $\varphi_n$ est prolongeable par continuité à $\interff{0}{1}$ et que le prolongement est de classe $\mathscr{C}^1$.
    \item Montrer (en utilisant éventuellement une intégration par parties) que pour toute fonction $f$ de classe $\mathscr{C}^1$ sur $\interff{0}{1}$, 
    \[
    \lim_{x \to + \infty} \int_0^1 f(t) \sin(x t) \d t = 0.
    \]
    \item Pour $k$ et $n$ entiers strictement positifs, on défnit:
    \[
    I_{n, k} = \int_0^1 B_n(t) \cos(2 k \pi t) \d t.
    \]
    Trouver une relation entre $I_{n,k}$ et $I_{n-2, k}$ et en déduire selon la parité de $n$, l'expression de $I_{n,k}$ en fonction de $n$ et de $k$. 
    \item 
    \begin{enumerate}
        \item En utilisant les questions précédentes, trouver, pour $N$ entier naturel, une expression de 
        \[
        \int_0^1 \varphi_{2m}(t) \sin \big((2N+1) \pi t \big) \d t
        \]
        en fonction de $m$, $N$ et $B_{2m}(0)$.
        \item En déduire la valeur de $\sum\limits_{k=1}^\infty \frac{1}{k^{2m}}$ en fonction de $m$ et de $B_{2m}(0)$.
        \item Donner les valeurs de $\sum\limits_{k=1}^\infty \frac{1}{k^2}$ et de $\sum\limits_{k=1}^\infty \frac{1}{k^4}$.
    \end{enumerate}
    \item Montrer, pour tout $m$ entier naturel non nul, la majoration:
    \[
    \sum_{k=1}^\infty \frac{1}{k^{2m}} \leqslant 2
    \]
    et en déduire la majoration $\module{B_{2m}(0)} \leqslant \frac{4}{(4 \pi^2)^m}$.
\end{enumerate}

\todoinline{À voir si on laisse ici ou si on met directement dans la partie Dirichlet.}

\begin{exercice}
\begin{enumerate}
    \item Justifier l'existence de l'intégrale $I = \int_0^{+\infty} \frac{\sin t}{t} \d t$. 
    \item Calculer $I_n = \int_0^\pi \frac{\sin \left(n + \frac{1}{2}\right)t}{2 \sin \frac{t}{2}} \d t$. \\
    \emph{Indication :} Calculer $I_{n+1} - I_n$. 
    \item Montrer que la fonction définie sur $\interof{0}{\pi}$ par $f(x) = \frac{1}{x} - \frac{1}{2 \sin \frac{x}{2}}$ peut être prolongée à $\interff{0}{\pi}$ en une fonction de classe $\mathscr{C}^1$. 
    \item Soit $f$ une fonction de classe $\mathscr{C}^1$ sur l'intervalle $\interff{a}{b}$. Montrer que 
    \[
    \lim_{\lambda \to +\infty} \int_a^b \sin(\lambda t) f(t) \d t = 0.
    \]
    \item En déduire la valeur de l'intégrale $I$.
\end{enumerate}
\end{exercice}
% \section{Théorèmes utilisés}

\printindex[theoremesutilises]

\begin{theo}[Approximation par des fonctions en escalier]
\label{theo:approximationescalier}
\end{theo}

\begin{theo}[Limite monotone]\label{theo:limitemonotone}
\end{theo}

\begin{theo}[Bijection monotone]
\label{theo:bijectionmonotone}
\end{theo}

\begin{theo}[Encadrement]\label{theo:encadrement}
\end{theo}

\begin{theo}[Convergence des sommes de \nom{Riemann} pour des fonctions intégrables sur un intervalle ouvert]\label{theo:convergencesommeriemann}
\end{theo}

\begin{theo}[Comparaison des fonctions de signe constant]\label{theo:comparaisonfonctionssigneconstant}
\end{theo}

\begin{theo}[Intégration terme à terme]\label{theo:integrationtermeaterme}
\end{theo}

\begin{theo}[Convergence dominée]\label{theo:convergencedominee}
\end{theo}

\begin{theo}[Séries alternées]\label{theo:seriesalternees}
\end{theo}

\begin{theo}[Prolongement des dérivées]\label{theo:prolongementDesDerivees}
\end{theo}

\begin{theo}[Formule de \nom{Taylor} avec reste intégal]\label{theo:taylorresteintegral}
\end{theo}

\begin{theo}[Prolongement dérivable]\label{theo:prolongementderivable}
\end{theo}

\begin{theo}[Intégration par parties généralisée, Source : \cite{acamanes}]\label{theo:ippgeneralisees}
    Soient $f$ et $g$ deux fonctions de classe $\mathscr{C}^1$ sur $I$. Si la fonction $fg$ a une limite finie en $a$ et en $b$, alors les intégrales
    \[
    \int_a^b f'(t)g(t) \d t \text{ et } \int_a^b f(t) g'(t) \d t
    \]
    sont de même nature. Si ces quantités sont convergentes, en notant
    \[
        [f(t)g(t)]_a^b = \lim_{x \to b^-} \big(f(x)g(x)\big) - \lim_{x \to a^+} \big(f(x)g(x)\big),
    \]
    on obtient la relation
    \[
    \int_a^b f'(t) g(t) \d t = \left[f(t)g(t)\right]_a^b - \int_a^b f(t) g'(t) \d t.
    \]
\end{theo}
%========
\section{Produit de convolution} 

\todoinline{Pour ajouter les ptés relatives aux transformées de Laplace et de Fourier, il faut le théorème de Fubini. On peut choisir de l'admettre.}

\begin{marginfigure}[0cm]
    \centering
    \begin{tikzpicture}

\begin{scope}[local bounding box=diagramme, scale=0.9, anchor=center, baseline]
    \node[myblue] (x) at (0, 2) {$x(t)$};
    \node [draw, rectangle, thick, color=myblue] (h) at (2, 2) {$h(t)$};
    \node[myblue] (y) at (5, 2) {$y(t) = h(t) * x(t)$};
    \draw[->, thick, myblue] (x) -- (h);
    \draw[->, thick, myblue] (h) -- (y);
    
    \node (laplace1) at (0, 1) {$\mathcal{L}$};
    \node (laplace2) at (2, 1) {$\mathcal{L}$};
    \node (inverse_laplace) at (5, 1) {$\mathcal{L}^{-1}$};
    \draw[thick] (x) -- (laplace1);
    \draw[thick] (h) -- (laplace2);
    \draw[<-, thick] (y) -- (inverse_laplace);
    
    \node[myred] (X) at (0, 0) {$X(p)$};
    \node [draw, rectangle, thick, color=myred] (H) at (2, 0) {$H(p)$};
    \node[myred] (Y) at (5, 0) {$Y(p) = H(p) \cdot X(p)$};
    \draw[->, thick, myred] (X) -- (H);
    \draw[->, thick, myred] (H) -- (Y);
    
    \draw[->, thick] (laplace1) -- (X);
    \draw[->, thick] (laplace2) -- (H);
    \draw[thick] (inverse_laplace) -- (Y);
\end{scope}
\node at (diagramme.north) [above] {Domaine temporel};
\node at (diagramme.south) [below] {Domaine fréquentiel};
    
\end{tikzpicture}
    \caption{Source : \url{https://commons.wikimedia.org/wiki/File:Time_and_frequency_domains.svg?uselang=fr}}
\end{marginfigure}

\begin{defi}[Produit de convolution]
Soient $f$ et $g$ deux fonctions continues par morceaux sur $\R$. Pour tout $x \in \R$, on définit, lorsque l'intégrale converge,
\begin{equation}\label{defconvolution}
(f \ast g)(x) \defeq \int_{\R} f(x-y) \, g(y) \d y.
\end{equation}
La fonction $f \ast g$ ainsi définie est le \definir{produit de convolution} de $f$ et $g$.
\end{defi}

Ce thème est inspiré du sujet Maths 1 - Centrale Supélec 2012~\cite{cs_1_2012}.

%-----------
\subsection{Propriétés du produit de convolution}

\begin{theo}
Le produit de convolution est bien défini lorsque :
\begin{enumerate}
\item la fonction $f$ est intégrable et la fonction $g$ est continue et bornée.

\item les fonctions $f$ et $g$ sont de carré intégrable.
\end{enumerate}
\end{theo}

\begin{demo}
\begin{enumerate}
\item On remarque que $\module{f(x - \,\cdot\,) \, g(\,\cdot\,)} \leqslant \norm{g}_\infty \module{f(x - \,\cdot\,)}$. Ainsi, la fonction $y \mapsto \module{f(x - \,\cdot\,) \, g(\,\cdot\,)}$ est dominée par une fonction intégrable, donc est intégrable.

\item On remarque que $\module{f(x - \,\cdot\,) \, g(\,\cdot\,)} \leqslant \frac{f(x - \,\cdot\,)^2 + g(\,\cdot\,)^2}{2}$. Ainsi, la fonction $y \mapsto \module{f(x - \,\cdot\,) \, g(\,\cdot\,)}$ est dominée par une fonction intégrable, donc est intégrable.
\end{enumerate}
\end{demo}

\begin{theo}
Si le produit de convolution $f \ast g$ est bien défini, alors $f \ast g = g \ast f$.
\end{theo}

\begin{demo}
Il suffit d'effectuer le changement de variable affine $u = x - y$.
\end{demo}


\begin{comment}
\todoinline{Ajouter un support compact pour avoir Fubini ?}

\begin{theo}{}
\[
\mathcal{L}(f \ast g) = \mathcal{L}(f) \mathcal{L}(g)
\]
\end{theo}

\end{comment}

%-----------
\subsection{Approximation de l'unité}

\todoarmand{Ajouter du contexte}

Pour tout entier naturel $n$, on note
\[
\fonctionligne[h_n]{t}{\frac{\big(1 - t^2\big)^n}{\lambda_n} \indicatrice{\interff{-1}{1}}},
\quad
\text{où}
\quad \lambda_n = \displaystyle\int_{-1}^1 \big(1 - t^2\big)^n \d t.
\]

\todoarmand{
Au passage, $\lambda_n = \frac{\sqrt{\pi}\Gamma(n+1)}{\Gamma\mathopen{}\left(n + \frac{3}{2}\right)}$
}

\begin{theo}
Pour tout $n$ entier naturel,
\begin{enumerate}
\item la fonction $h_n$ est positive sur $\R$ ;
\item $\displaystyle \int_{-\infty}^{+\infty} h_n(t) \d t = 1$ ;
\item $\forall \varepsilon > 0$, $\displaystyle\lim_{n\to+\infty} \int_{-\infty}^{-\varepsilon} h_n(t) \d t = 0$ et $\displaystyle\lim_{n\to+\infty} \int_{\varepsilon}^{+\infty} h_n(t) \d t = 0$.
\end{enumerate}
La suite $(h_n)_{n \in \N}$ est une \definir{approximation de l'unité}. \\
\noindent De plus, pour toute fonction $f$ continue bornée sur $\R$, $(f \ast h_n)_{n\in\N}$ converge simplement vers $f$ sur $\R$.
\end{theo}

\begin{marginfigure}[0cm]
    \centering
    \begin{tikzpicture}
    \begin{axis}[
        xmin=-1.25, xmax=1.25,
        ymin=-0.05, ymax=2.85,
        axis lines=middle,
        axis line style=thick,
        axis line style={-latex},
        legend cell align={left}, 
        legend style={
            draw=none,
            fill=none,
            font=\footnotesize,
            legend image code/.code={\node[anchor=west] {#1};}
        },
        xtick=\empty,
        xtick={-1, 1},
        xticklabels={$-1$, $1$},
        ytick={1},
        yticklabels={$1$},
        xlabel=$x$,
        ylabel=$h_n(x)$,
        every axis x label/.style={at={(current axis.right of origin)},anchor=north},
    ];

    \addplot[domain=-1:1, samples=200, color=myblue!25!mycyan, smooth, thick, line join=round,line cap=round] {(1 - x^2)^1/(4/3)};
    \addplot[domain=-1.25:-1, samples=200, color=myblue!25!mycyan, smooth, thick] {0};
    \addplot[domain=1:1.15, samples=200, color=myblue!25!mycyan, smooth, thick] {0};
    \addlegendentry{\textcolor{myblue!20!mycyan}{$n = 1$}}

    \addplot[domain=-1:1, samples=200, color=myblue!50!mycyan, smooth, thick, line join=round,line cap=round] {(1 - x^2)^2/(32/30)};
    \addlegendentry{\textcolor{myblue!40!mycyan}{$n = 2$}}

    \addplot[domain=-1:1, samples=200, color=myblue!75!mycyan, smooth, thick, line join=round,line cap=round] {(1 - x^2)^5/0.73881};
    \addlegendentry{\textcolor{myblue!80!mycyan}{$n = 5$}}

    \addplot[domain=-1:1, samples=200, color=myblue!100!mycyan, smooth, thick, line join=round,line cap=round] {(1 - x^2)^20/0.38909};
    \addlegendentry{\textcolor{myblue!100!mycyan}{$n = 20$}}
    \end{axis}
\end{tikzpicture}
    \caption{}
\end{marginfigure}

\begin{exercice}
Soit $f$ une fonction continue bornée et $n \in \N$.
\begin{questions}
\item Montrer que $h_n$ est positive sur $\R$ et que $\displaystyle\int_{-\infty}^{+\infty} h_n(t) \d t = 1$.
\item Montrer que $\lambda_n \geqslant \frac{1}{n + 1}$.
\end{questions}
Soit $\varepsilon > 0$.
\begin{questions}[resume]
\item Si $\module{\varepsilon} \geqslant 1$, calculer les intégrales $\displaystyle \int_{-\infty}^{-\varepsilon} h_n(t) \d t$ et $\displaystyle \int_\varepsilon^{+\infty} h_n(t) \d t$.
\item Si $\module{\varepsilon} < 1$, montrer que $\displaystyle \int_\varepsilon^1 h_n(t) \d t \leqslant (n + 1) \big(1 - \varepsilon^2\big)^n$.
\item En déduire que la suite $(h_n)_{n \in \N}$ est une approximation de l'unité.
\end{questions}
Soit $x \in \R$ et $\varepsilon > 0$.
\begin{questions}[resume]
\item Montrer qu'il existe $\delta > 0$ tel que
\[
\module{t} \leqslant \delta \implies \module{f(x - t) - f(x)} \leqslant \varepsilon.
\]
\item Montrer qu'il existe un réel $n_0$ tel que pour tout $n \geqslant n_0$,
\[
\int_{-\infty}^{-\delta} h_n(t) \d t \leqslant \varepsilon
\quad \text{et} \quad
\int_{\delta}^{+\infty} h_n(t) \d t \leqslant \varepsilon.
\]
\item En déduire que la suite $(f \ast h_n)_{n \in \N}$ converge simplement vers $f$ sur $\R$.
\end{questions}
\end{exercice}


\begin{solution}
\begin{reponses}
\item Ces deux premiers points sont des conséquences de la définition de $h_n$.

\item Si $t \in \interff{-1}{1}$, alors $t^2 \leqslant t$. Ainsi, d'après la définition,
\[
\lambda_n
= \int_{-1}^1 \big(1 - t^2\big)^n \d t
\geqslant \int_{-1}^1 (1 - t)^n \d t
= \frac{2^{n+1}}{n + 1} \geqslant \frac{1}{n+1}.
\]


\item Comme $h_n$ est nulle en dehors de $\interff{-1}{1}$, alors 
\[
\int_\varepsilon^{+\infty} h_n(t) \d t
= \int_{-\infty}^{-\varepsilon} h_n(t) \d t
= 0.
\]

\item Si $0 \leqslant \varepsilon \leqslant t < 1$, alors $\varepsilon^2 \leqslant t^2$. Ainsi,
\[
\int_\varepsilon^1 h_n(t) \d t
\leqslant
\int_\varepsilon^1 \frac{\big(1 - \varepsilon^2\big)^n}{\lambda_n} \d t
\leqslant
(n + 1) \big(1 - \varepsilon^2\big)^n.
\]
De manière analogue, si $-1 \leqslant \varepsilon \leqslant 0$, alors
\[
\int_{-1}^{-\varepsilon} h_n(t) \d t \leqslant (n + 1) \big(1 - \varepsilon^2\big)^n.
\]

\item Finalement, d'après le \theoremeutilise{théorème des croissances comparées}{theo:croissancescomparees},
\[
\forall \varepsilon > 0,\quad
\displaystyle\lim_{n\to+\infty} \int_{-\infty}^{-\varepsilon} h_n(t) \d t
= \displaystyle\lim_{n\to+\infty} \int_{\varepsilon}^{+\infty} h_n(t) \d t
= 0.
\]

\item Comme $f$ est continue en $x$, il existe un réel $\delta$ tel que pour tout $t \in \interff{-\delta}{\delta}$,
\[
\module{f(x - t) - f(x)} \leqslant \varepsilon.
\]

\item Il suffit de traduire avec des quantificateurs la limite démontrée dans les questions précédentes.

\item On utilise la relation de \nom{Chasles} ainsi que les questions précédentes, pour $n \geqslant n_0$, :
\begin{align*}
\module{(f \ast h_n)(x) - f(x)}
&= \module{\int_{-\infty}^{+\infty} f(x - t) h_n(t) \d t - f(x) \int_{-\infty}^{+\infty} h_n(t) \d t}\\
&\leqslant
\int_{-\infty}^{-\delta} \module{f(x - t) - f(t)} h_n(t) \d t
+ \int_{-\delta}^{\delta} \module{f(x - t) - f(t)} h_n(t) \d t
+ \int_{\delta}^{+\infty} \module{f(x - t) - f(t)} h_n(t) \d t\\
&\leqslant 2 \norm{f}_{\infty} \mathopen{}\left(\int_{-\infty}^{-\delta} h_n(t) \d t + \int_{\delta}^{+\infty} h_n(t) \d t\right) + \varepsilon \int_{-\delta}^{\delta} h_n(t) \d t\\
&\leqslant 4 \norm{f}_{\infty} \varepsilon + \varepsilon \int_{-\delta}^{\delta} h_n(t) \d t\\
&\leqslant 4 \norm{f}_{\infty} \varepsilon + \varepsilon \int_{-\infty}^{+\infty} h_n(t) \d t\\
&\leqslant \big(4 \norm{f}_\infty + 1\big) \varepsilon.
\end{align*}
On obtient ainsi la convergence simple annoncée.
\end{reponses}
\end{solution}
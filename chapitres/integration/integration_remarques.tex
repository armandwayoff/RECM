\setchapterpreamble[u]{\margintoc}
\chapter{Intégration}
\labch{integration}

\todoinline{
Note pour plus tard : dans la partie dérivation, il y a les contrôles des dérivées étudiées dans le problème Centrale Supélec - PC 1 - 2001
}

\todoinline{
Remarques générales :\\
* J'aurais bien mis un exercice sur les normes $L^p$ sur un segment. À voir comment l'articuler avec les $\ell^p$ et ne pas faire de redite sur la convexité.\\

* Pour les $\ell^p(\K^n)$, lorsque $n = 2$, on peut illustrer la forme des différentes boules. J'ai une animation sur ce sujet qui traîne. \\

* Pour les $L^p$, sur un segment, j'y mettrais la section 7.8 qui étudie le cas $p \to +\infty$ et justifie la notation $\norme[\infty]{\cdot}$. Lorsque $p < 1$, on perd le caractère d'espace vectoriel mais on peut regarder ce qui se passe lorsque $p \to 0$. J'ai un exo là-dessus avec l'utilisation d'un théorème de domination.\\

* On peut aussi montrer l'inclusion des $L^p$ lorsqu'on intègre sur un segment, c'est ensuite utile dans la théorie de Lebesgue, dans le cadre des probabilités. Il y a une réciproque pour savoir quand $L^p \subset L^q$. J'ai des notes là-dessus, mais faut voir si c'est faisable niveau prépa.\\

* Prévoir un truc (un lien hypertexte pour le pdf, une façon de numéroter qui relie les exercices) qui permet de relier des exercices entre eux, par ex. Hölder dans un chapitre convexité et les $L^p$ ?\\

* Prévoir de mettre des références (livres, articles ?) par chapitre ? Par section ?\\

* Prévoir une annexe avec les théorèmes utilisés ?\\

% * Est-ce que ça vaut le coup de mettre les méthodes d'approximation des intégrales sur un segment ? Moi je les aime bien, ça fait des jolis dessins, mais sans application en info...\\

Niveau présentation, je pense qu'il faut choisir entre des propriétés énoncées puis démontrées ou des exercices. Je pencherais pour des petites propriétés démontrées pas à pas, comme le serait un problème et, éventuellement, à la fin, un exercice d'application (non corrigé ?)\\

Il y a les exercices sur les intégrales qui permettent de prolonger les fonctions sur des espaces plus grands. Je pense à $\Gamma$ qui prolonge la factorielle sur $]0, +\infty[$ ou encore la formule intégrale pour la fonction $\zeta$ (j'ai un exercice sur ce sujet - PSI - 20/21)
}
\todoarmand{C'est un  thème intéressant}
\todoinline{
L'intégrale de Poisson utilise les sommes de Riemann et les nombres complexes ! On la met ?
}
\todoarmand{Oui, très bonne idée !}
\todoinline{
Si on peut rajouter une partie, peut être que l'intégrale de Frullani pourrait être intéressante, elle apparaît régulièrement dans des exercices :
https://fr.wikipedia.org/wiki/IntC3%A9grale_de_Frullani
}
\todoarmand{J'ai trouvé ce livre de ECS \url{https://excerpts.numilog.com/books/9782340005853.pdf} qui propose 17 thèmes classiques qui recouvrent l'ensemble de leur programme}
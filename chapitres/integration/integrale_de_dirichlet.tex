\textcolor{red}{A revoir}
\begin{prop}{}
    L'intégrale de \textsc{Dirichlet} (1829) est l'intégrale de la fonction sinus cardinal sur la demi-droite des réels positifs
    $$\int_{0}^{+\infty} \frac{\sin x}{x} \d x = \frac{\pi}{2}.$$
\end{prop}

\begin{marginfigure}[0cm]
    \begin{tikzpicture}
    
\begin{axis}[
    % width=7.5cm,
    % grid=both,
    xmin=-11,
    xmax=11,
    ymin=-0.25,
    ymax=1.15,
    % ylabel=$\mathrm{sinc}(x)$,
    axis lines=middle,
    axis line style=thick,
    axis line style={-latex},
    xticklabels={},
    xtick={-3*3.141592, -2*3.141592, -3.141592, 3.141592, 2*3.141592, 3*3.141592},
    ytick={0, 1},
    xlabel=$x$,
    every axis x label/.style={at={(current axis.right of origin)},anchor=south},
]
              
  \addplot[domain=-15:15, blue, samples=200, name path=B] plot[thick] {sin(deg(x))/x};

  \path[name path=xaxis]
      (-15,0) -- (\pgfkeysvalueof{/pgfplots/xmax},0);
    \addplot[blue!25, opacity=0.9] fill between[of=xaxis and B];
  
  \node[blue,above left] at (-2,0.5) {$\displaystyle x \mapsto \frac{\sin(x)}{x}$};
\end{axis}
\begin{axis}[
    % width=7.5cm,
    % grid=both,
    xmin=-11,
    xmax=11,
    ymin=-0.25,
    ymax=1.15,
    % ylabel=$\mathrm{sinc}(x)$,
    axis lines=middle,
    % axis line style=thick,
    % axis line style={-latex},
    axis line style={draw=none},
    xticklabels={\contour{white}{$-3\pi$}, \contour{white}{$-2\pi$}, \contour{white}{$-\pi$}, \contour{white}{$\pi$}, \contour{white}{$2\pi$}, \contour{white}{$3\pi$}},
    xtick={-3*3.141592, -2*3.141592, -3.141592, 3.141592, 2*3.141592, 3*3.141592},
    ytick={0, 1},
]
\end{axis}
\end{tikzpicture}
\end{marginfigure}

\begin{preuve}
    \begin{enumerate}
        \item Montrer que $\int_{0}^{1} \frac{\sin (t)}{t} \d t$ est convergente. 
        \item Deux méthodes.
        \begin{itemize}
            \item Montrer que la série de terme général $\int_{n \pi}^{(n+1) \pi} \frac{\sin(t)}{t} \d t$ est convergente. En déduire que $\int_{1}^{+ \infty} \frac{\sin(t)}{t} \d t$ converge. 
            \begin{enumerate}
                \item Montrer que $u_n = \int_{n \pi}^{(n+1) \pi} \frac{\sin(t)}{t} \d t$ est le terme général d'une série alternée. Donc $\sum u_n$ converge.\\
                Attention: on ne peut pas en déduire directement que $\sum\limits_{n=0}^{+ \infty} \int_{n \pi}^{(n+1) \pi} \frac{\sin(t)}{t} \d t = \int_{1}^{+ \infty} \frac{\sin(t)}{t} \d t$ car on n'a pas encore démontrer la convergence du deuxième membre (c.f. relation de \textsc{Chasles}).\\
                \item Il faut montrer la convergence de $\int_{\pi}^{x} \frac{\sin (t)}{t} \d t$. \textcolor{green}{A compléter.}
            \end{enumerate}
            \item On peut aussi procéder par intégration par parties en posant
            $$
            \begin{drcases}                
                u(t) = \frac{1}{t}\\
                v(t) = - \cos(t)
            \end{drcases}
            \mathscr{C}^1 \text{ sur } [1, +\infty].
            $$
            Bien présicer que $u(t)v(t)=-\frac{\cos(t)}{t}$ admet une limite finie en 1 et en $+ \infty$.\\
            \begin{remarque}
                L'intégration par parties préserve la régularité de l'intégrale mais ne préserve pas l'intégrabilité.
            \end{remarque}
        \end{itemize}
        \item On en déduit immédiatement que $\int_{0}^{+ \infty} \frac{\sin (t)}{t} \d t$ converge.
        \item De plus, on peut montrer que cette intégrale est semi-convergente (i.e. elle n'est pas intégrable sur $\Rp$). Pour cela, montrer que pour tout entier naturel $n$, $\int_{n \pi}^{(n+1) \pi} \frac{\sin(t)}{t} \d t \geqslant \frac{2}{(n+1) \pi}$. 
    \end{enumerate}
\end{preuve}

\subsection{Intégrabilité du sinus cardinal sur  \texorpdfstring{$\Rpe$}{R+*}}

\begin{prop}{}
    La fonction sinus cardinal $\mathrm{sinc}:t \mapsto \frac{\sin(t)}{t}$ n'est pas intégrable sur $]0, +\infty[$.
\end{prop}

\begin{preuve}
    \marginnote[0cm]{\url{https://www.agreg-maths.fr/uploads/versions/1175/dirichlet.pdf}}
    Soit $N \in \Ne$, alors:
    \begin{align*}
        \int_0^{N \pi} \frac{|\sin x|}{x} \d x &= \sum_{k=0}^{N-1} \int_{k \pi}^{(k+1) \pi} \frac{|\sin x|}{x} \d x \\
        \text{ par changement de variable} &= \sum_{k=0}^{N-1} \int_0^\pi \frac{|\sin x|}{x + k \pi} \d x \\
        &\geqslant \sum_{k=0}^{N-1} \frac{1}{(k+1) \pi} \int_0^\pi \sin x \d x \\
        &\geqslant \frac{2}{\pi} \sum_{k=1}^N \frac{1}{k} \xrightarrow[N \to + \infty]{} + \infty.
    \end{align*}
\end{preuve}

\subsection{Intégrale de \textsc{Dirichlet} via une intégrale à paramètre}

Soit la transformée de \textsc{Laplace} de la fonction sinus cardinal:
$$F:x \to \int_{0}^{+ \infty} \exp(-xt) \frac{\sin (t)}{t} \d t$$
    
\begin{enumerate}
    \item \emph{Montrer que $F$ est définie sur $\Rp$.}
    \begin{itemize}
        \item Si $x > 0$, majorer l'intégrande par $t \mapsto \exp(-xt)$.
        \item Si $x = 0$, montrer le prolongement par continuité de la fonction sinus cardinal en $0$ puis intégrer la fonction sinus cardinal par parties sur $[1, +\infty]$.
    \end{itemize}
    \item \emph{Calculer $F$ sur $\Rpe$, en déduire la valeur de la fonction de \textsc{Dirichlet}}
\end{enumerate}


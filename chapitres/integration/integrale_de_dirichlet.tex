\textcolor{red}{A revoir}
\begin{tcolorbox}
    L'intégrale de \textsc{Dirichlet} (1829) est l'intégrale de la fonction sinus cardinal sur la demi-droite des réels positifs
    $$\int_{0}^{+\infty} \frac{\sin x}{x}\ \d x = \frac{\pi}{2}.$$
\end{tcolorbox}

\begin{enumerate}
    \item Montrer que $\int_{0}^{1} \frac{\sin (t)}{t}\ \d t$ est convergente. 
    \item Deux méthodes.
    \begin{itemize}
        \item Montrer que la série de terme général $\int_{n \pi}^{(n+1) \pi} \frac{\sin(t)}{t}\ \d t$ est convergente. En déduire que $\int_{1}^{+ \infty} \frac{\sin(t)}{t}\ \d t$ converge. 
        \begin{enumerate}
            \item Montrer que $u_n = \int_{n \pi}^{(n+1) \pi} \frac{\sin(t)}{t}\ \d t$ est le terme général d'une série alternée. Donc $\sum u_n$ converge.\\
            Attention: on ne peut pas en déduire directement que $\sum\limits_{n=0}^{+ \infty} \int_{n \pi}^{(n+1) \pi} \frac{\sin(t)}{t}\ \d t = \int_{1}^{+ \infty} \frac{\sin(t)}{t}\ \d t$ car on n'a pas encore démontrer la convergence du deuxième membre (c.f. relation de \textsc{Chasles}).\\
            \item Il faut montrer la convergence de $\int_{\pi}^{x} \frac{\sin (t)}{t}\ \d t$. \textcolor{green}{A compléter.}
        \end{enumerate}
        \item On peut aussi procéder par intégration par parties en posant
        $$
        \begin{drcases}                
            u(t) = \frac{1}{t}\\
            v(t) = - \cos(t)
        \end{drcases}
        \mathscr{C}^1 \text{ sur } [1, +\infty].
        $$
        Bien présicer que $u(t)v(t)=-\frac{\cos(t)}{t}$ admet une limite finie en 1 et en $+ \infty$.\\
        \textcolor{red}{Les IPP préservent la régularité de l'intégrale MAIS ne préservent pas l'intégrabilité.}
    \end{itemize}
    \item On en déduit immédiatement que $\int_{0}^{+ \infty} \frac{\sin (t)}{t}\ \d t$ converge.
    \item De plus, on peut montrer que cette intégrale est semi-convergente (i.e. elle n'est pas intégrable sur $\Rp$). Pour cela, montrer que pour tout entier naturel $n$, $\int_{n \pi}^{(n+1) \pi} \frac{\sin(t)}{t}\ \d t \geqslant \frac{2}{(n+1) \pi}$. 
\end{enumerate}
    
\underline{Remarque:} La fonction sinus cardinal \textbf{n'est pas intégrable} sur $\Rpe$ (cf. le prochain exercice).
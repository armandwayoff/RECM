
\marginnote[-2mm]{Ce théorème a été démontré par le mathématicien italien Guido \textsc{Fubini} en 1907.}
\begin{theo}
    Soit $f: [a,b] \times [c, d] \to \K$ une application continue. Alors,
    $$\int_{a}^{b} \left ( \int_{c}^{d} f(x,y) \d y \right) \d x = \int_{c}^{d} \left ( \int_{a}^{b} f(x,y) \d x \right) \d y.$$
\end{theo}

\textit{Correction du sujet Mines Maths 2 PSI 2021 par Doc Solus.} 
\begin{preuve}
    Pour tout $(x, t) \in [a, b] \times [c, d]$ on pose $\boxed{\varphi(x, t) = \int_{a}^{x} f(u, t) \d u}$. 
    \begin{enumerate}
        \item Montrer que pour tout $x \in [a, b]$, l'application $t \mapsto \varphi(x, t)$ est continue sur $[c, d]$. \\
        Application du \textbf{théorème de continuité des intégrales à paramètre}. \\
        Pour la domination : $f$ est continue sur une partie fermée bornée de $\R^2$, donc d'après le \textbf{théorème des bornes}, $f$ est bornée sur $[a, b] \times [c, d]$ par une constante $M \in \Rp$.
        \item On pose alors, pour tout $x  \in [a, b],\ \boxed{\psi(x) = \int_{c}^{d} \varphi(x, t) \d t}$. Montrer que $\psi$ est de classe $\mathscr{C}^1$ sur $[a, b]$, préciser $\psi'$. \\
        Application du \textbf{théorème de dérivation des intégrales à paramètre} à la fonction $x \mapsto \int_{c}^{d} \varphi(x, t) \d t$:
        \begin{itemize}
            \item $\forall t \in [c, d],\ x \mapsto \varphi(x, t)$ est de classe $\mathscr{C}^1$ sur $[a, b]$ car c'est la \textbf{primitive} s'annulant en $a$ de la fonction continue $x \mapsto f(x, t)$. 
            \item $\frac{\partial \varphi}{\partial x}(x, t) = f(x, t)$
            \item La domination se fait par le même constante $M$ que précédemment. 
        \end{itemize}
        $$\forall x \in [a, b] \quad \psi'(x) = \int_{c}^{d} f(x, t) \d t.$$
        \item En déduire:
        $$\forall x \in [a, b],\ \int_{a}^{x} \left ( \int_{c}^{d} f(u,t) \d t \right) \d u = \int_{c}^{d} \left ( \int_{a}^{x} f(u,t) \d u \right) \d t.$$
        Soit $x \in [a, b]$. D'une part,
        $$\psi(x) = \int_{c}^{d} \left ( \int_{a}^{x} f(u,t) \d u \right) \d t.$$
        D'autre part, d'après la question précédente et le \textbf{théorème fondamental de l'analyse}, 
        \begin{align*}
            \int_{a}^{x} \left ( \int_{c}^{d} f(u,t) \d t \right) \d u &= \int_{a}^{x} \psi'(u) \d u  = \psi(x) - \psi(a) \\
            \text{Or } \psi(a) &= \int_{c}^{d} \varphi(a, t)\ \d t \\
            \text{et } \forall t \in [c, d] \quad \varphi(a, t) &= \int_{a}^{a} f(u, t) \d u = 0
        \end{align*}
        d'où $\psi(a) = 0$ et le résultat. \\
        En particulier, pour $x = b$ on obtient le résultat final.
    \end{enumerate}
\end{preuve}    

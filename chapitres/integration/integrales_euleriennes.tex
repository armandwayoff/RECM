\marginnote[0mm]{Texte de Pierre-Jean \textsc{Hormière}}
De premières tentatives pour définir $n!$ pour des valeurs non entières remontent à \textsc{Stirling} et Daniel \textsc{Bernoulli}. Dans une lettre à Christian \textsc{Goldbach} du 13 octobre 1729, \textsc{Euler} découvre (ou invente ?) une fonction de variable réelle prolongeant de manière naturelle la fonction $n!$. D'abord introduite comme limite de produits, cette fonction fut plus tard présentée sous forme intégrale et reliée à des fonctions voisines. \\
Les fonctions eulériennes sont les plus importantes \say{ fonctions spéciales } de l'analyse classique, réelle et complexe. \textsc{Legendre} les a nommées, classifiées et étudiées. Elles ont aussi été étudiées par \textsc{Gauss}, \textsc{Binet}, \textsc{Plana}, \textsc{Malmsten}, \textsc{Raabe}, \textsc{Weierstrass}, \textsc{Hankel}, H. \textsc{Bohr}, \textsc{Mollerup}, \textsc{Artin} \dots \\
Il y a bien des façons de prolonger la fonction $n!$ au domaine réel, même en se limitant aux fonctions continues. Une idée naturelle est de partir de la formule $\displaystyle n! = \int_{0}^{+ \infty} t^n \me^{-t} \d t$. Cette forme intégrale de la factorielle suggère de considérer la fonction $\displaystyle F(x)=\int_{0}^{+ \infty} t^x \me^{-t} \d t$. Cette fonction, définie sur $]-1, +\infty[$, prolonge intelligemment la factorielle, en ce sens qu'elle possède des propriétés nombreuses et cohérentes. Par commodité, on considère plutôt $\displaystyle \Gamma(x) = \int_{0}^{+ \infty} t^{x-1} \me^{-t} \d t$.

\subsection{Fonction Gamma d'\textsc{Euler}}

%\begin{marginfigure}[3cm]
%    \begin{tikzpicture}[]

\begin{axis}[
xmin = -4.9, xmax = 5.1, 
%ymin = -3.5, ymax = 3.5,  
restrict y to domain=-6:6,
axis lines = middle,
axis line style={-latex},  
xlabel={$x$}, 
ylabel={$\Gamma(x)$},
%enlarge x limits={upper={val=0.2}},
enlarge y limits=0.05,
x label style={at={(ticklabel* cs:1.00)}, inner sep=5pt, anchor=north},
y label style={at={(ticklabel* cs:1.00)}, inner sep=2pt, anchor=south east},
]

\addplot[color=red, samples=222, smooth, 
domain = 0:5] gnuplot{gamma(x)};

\foreach[evaluate={\N=\n-1}] \n in {0,...,-5}{%
\addplot[color=red, samples=555, smooth,  
domain = \n:\N] gnuplot{gamma(x)};
%
\addplot [domain=-6:6, samples=2, densely dashed, thin] (\N, x);
}%
\end{axis}
\end{tikzpicture}
%    \caption*{\centering Graphe de la fonction Gamma}
%\end{marginfigure}

\begin{defi}{Fonction Gamma d'\textsc{Euler}}
    La \emph{fonction Gamma d'\textsc{Euler}} est définie par: 
    $$\Gamma(x) \defeq \int_{0}^{+\infty} t^{x-1} \me^{-t} \d t.$$
\end{defi}

\begin{remarque}
    À un changement de variable près, la fonction $\Gamma$ est la \nameref{transformee_laplace} de la fonction $t \mapsto t^x$. 
\end{remarque} 

\begin{prop}{}
    \begin{itemize}
        \item La fonction $\Gamma$ est définie si et seulement si $x>0$.
        \item Pour tout $x > 0$, $\Gamma(x+1) = x\Gamma(x)$. \\
        En particulier, pour tout $n \in \N$, $\Gamma(n+1) = n!$. 
    \end{itemize}
\end{prop}

\begin{preuve}
    \begin{itemize}
        \item La fonction $f_x:t \mapsto t^{x-1} \me^{-t}$ est continue sur $]0, + \infty[$ comme produit de fonctions qui y sont continues. La fonction $f_x$ est donc intégrable sur tout segment de $]0, +\infty[$. Il reste à étudier son intégrabilité en $0$ et en $+ \infty$:
        \begin{itemize}
            \item[$\triangleright$] En $+\infty:$ par croissances comparées, $f_x(t) = o_{+\infty} \left(\frac{1}{t^2} \right)$. D'après le théorème de comparaison des fonctions à termes positifs, $f_x$ est intégrable au voisinage de $+\infty$.
            \item[$\triangleright$] En $0$: $f_x(t) \sim_0 t^{x-1}$ qui est intégrable d'après \textsc{Riemann} si et seulement si $1-x < 1$ i.e. si et seulement si $x > 0$.
        \end{itemize}

        \item Soit $x > 0$. Calculons $\Gamma(x+1)$ en effectuant une intégration par parties. Posons $u:t \mapsto \me^{-t}$ et $v:t \mapsto t^x$, toutes deux de classe $\mathscr{C}^1$ sur $\Rp$. Vérifions la convergence du \emph{crochet}:
        \begin{align*}
            \text{par croissances comparées } & \lim_{t \to +\infty} u(t) v(t) = 0, \\
            \text{ comme } x > 0 & \lim_{t \to 0} u(t) v(t) = 0.
        \end{align*}
        Ainsi, d'après le théorème d'intégration par parties généralisées, 
        $$\int_{0}^{+\infty} t^{x-1} \me^{-t} \d t = \underbrace{0}_{\text{crochet}} - \int_{0}^{+\infty} xt^{x-1} (-\me^{-t}) \d t.$$
        soit 
        $$\Gamma(x+1) = x \Gamma(x).$$
        En particulier, $\Gamma(1) = 1$ et pour tout $n \in \Ne, \Gamma(n+1) = n \Gamma(n)$. Donc
        $$\forall n \in \Ne, \Gamma(n+1) = n!$$
        Cette fonction, introduite en 1729 par le mathématicien suisse, prolonge la fonction factorielle à l'ensemble des réels strictement positifs.
    \end{itemize}
\end{preuve}

\marginnote[-5cm]{
    \begin{kaobox}[frametitle=Théorème (Intégration par parties généralisées)]
        \cite{acamanes}\\
        Soient $f$ et $g$ deux fonctions de classe $\mathscr{C}^1$ sur $I$. Si la fonction $fg$ a une limite finie en $a$ et en $b$, alors les intégrales
        $$\int_a^b f'(t)g(t) \d t \text{ et } \int_a^b f(t) g'(t) \d t$$
        sont de même nature. Si ces quantités sont convergentes, en notant
        \begin{align*}
            [f(t)g(t)]_a^b = \lim_{x \to b^-} \big(f(x)g(x)\big) - \lim_{x \to a^+} \big(f(x)g(x)\big),
        \end{align*}
        on obtient la relation
        $$\int_a^b f'(t) g(t) \d t = \left[f(t)g(t)\right]_a^b - \int_a^b f(t) g'(t) \d t.$$
    \end{kaobox}
}

\begin{prop}{}
    $$\forall k \in \N,\ \forall x \in \R_+^\star,\ \Gamma^{(k)}(x) = \int_{0}^{+\infty} (\ln t)^k t^{x-1} \me^{-t} \d t$$
\end{prop}

\begin{elem_preuve}
    Utiliser une domination locale sur un segment $[a, A] \subset \R_+^\star$ par la fonction:
    $$\varphi_k:t \mapsto 
    \begin{cases}
        |\ln t |^k \me^{-t} t^{a-1} & \text{si } t \in ]0, 1], \\
        |\ln t |^k \me^{-t} t^{A-1} & \text{si } t > 1.
    \end{cases}
    $$
\end{elem_preuve}

\subsection{Fonction bêta}
\begin{defi}{Fonction bêta}
    Pour tout $(p,q) \in \N^2$, on note
    $$I_{p,q} \defeq \int_{0}^{1} t^p (1-t)^q \d t.$$
\end{defi}

\begin{prop}{}
    Pour tout $(p,q) \in \N^2$,
    $$I_{p,q} = \frac{p! q!}{(p + q + 1)!}.$$
\end{prop}

\begin{preuve}
    Soit $(p,q) \in \N^2$. Nous allons déterminer une relation entre $I_{p,q}$ et $I_{p+1, q-1}$ en faisant une intégration par parties. \\
    On pose $u:t\mapsto \frac{1}{p+1} t^{p+1}$ et $v:t\mapsto (1-t)^q$, toutes deux de classe $\mathscr{C}^1$ sur $[0, 1]$. Alors, 
    \begin{align*}
        I_{p,q} &= \left[ \frac{1}{p+1}t^{p+1} \times (1-t)^q \right]_0^1 + \frac{q}{p+1} \int_0^1 t^{p+1} (1-t)^{q-1} \d t \\
        I_{p,q} &= \frac{q}{p+1} I_{p+1, q-1}.
    \end{align*}
    On en déduit que 
    \begin{align*}
        I_{p,q} &= \frac{q}{p+1} \times \frac{q-1}{p+2} \times \cdots \times \frac{1}{p+q} I_{p+q,0} \\
        I_{p,q} &= \frac{p! q!}{(p + q + 1)!}.
    \end{align*}
\end{preuve}

\url{https://fr.wikipedia.org/wiki/Intégrale_d'Euler} \\
\cite{calcul_infinitesimal} Chapitre IV, 3 Intégrales eulériennes, page 125.

\begin{exercice}
    \marginnote[0cm]{\cite{acamanes}}
    Déterminer la nature de la série $\sum I_{n,n}$ et, le cas échéant, calculer sa somme. 
\end{exercice}

\begin{exercice}
    \marginnote[0cm]{Clémentine Portal feuille d'exo n°7, exo 11}
    Déterminer une expression simplifiée de $\sum\limits_{k=0}^q \binom{q}{k} \frac{(-1)^k}{p+k+1}$, pour tout $(p, q) \in \N^2$.
\end{exercice}

\begin{solution}
\end{solution}

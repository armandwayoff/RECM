\subsection{Fonction Gamma d'\textsc{Euler}}

\begin{marginfigure}[5cm]
    \begin{tikzpicture}[]

\begin{axis}[
xmin = -4.9, xmax = 5.1, 
%ymin = -3.5, ymax = 3.5,  
restrict y to domain=-6:6,
axis lines = middle,
axis line style={-latex},  
xlabel={$x$}, 
ylabel={$\Gamma(x)$},
%enlarge x limits={upper={val=0.2}},
enlarge y limits=0.05,
x label style={at={(ticklabel* cs:1.00)}, inner sep=5pt, anchor=north},
y label style={at={(ticklabel* cs:1.00)}, inner sep=2pt, anchor=south east},
]

\addplot[color=red, samples=222, smooth, 
domain = 0:5] gnuplot{gamma(x)};

\foreach[evaluate={\N=\n-1}] \n in {0,...,-5}{%
\addplot[color=red, samples=555, smooth,  
domain = \n:\N] gnuplot{gamma(x)};
%
\addplot [domain=-6:6, samples=2, densely dashed, thin] (\N, x);
}%
\end{axis}
\end{tikzpicture}
    \caption{Graphe de la fonction Gamma}
\end{marginfigure}

\begin{defi}
    Pour tout $x$ > 0 réel, la \emph{fonction Gamma d'\textsc{Euler}} est définie par: 
    $$\Gamma(x) \defeq \int_{0}^{+\infty} t^{x-1} \me^{-t} \d t.$$
\end{defi}

\marginnote[-2mm]{Cette fonction, introduite en 1729 par le mathématicien suisse, prolonge la fonction factorielle à l'ensemble des nombres complexes (à l'exception des entiers négatifs).}

\begin{remarque}
    À un changement de variable près, la fonction $\Gamma$ est la \nameref{transformee_laplace} de la fonction $t \mapsto t^x$. 
\end{remarque} 
\underline{Propriétés principales:}
\begin{itemize}
    \item $\Gamma$ est définie si et seulement si $x>0$.
    \item Pour tout $x > 0$, $\Gamma(x+1) = x\Gamma(x)$.
    \item En particulier, $\boxed{\forall n \in \N$, $\Gamma(n+1) = n!}$. 
\end{itemize}
\begin{preuve}
    La fonction $f:(x,t) \mapsto t^{x-1} \me^{-t}$ est continue sur $]0, + \infty[$ comme produit de fonctions qui y sont continues. La fonction $f$ est donc intégrable sur tout segment de $]0, +\infty[$. Il reste à étudier son intégrabilité en $0$ et en $+ \infty$:
    \begin{itemize}
        \item En $+\infty:$ par croissances comparées, $t^{x-1} \me^{-t} = o_{+\infty} \left(\frac{1}{t^2} \right)$. D'après le théorème de comparaison des fonctions à termes positifs, $f$ est intégrable au voisinage de $+\infty$.
        \item En $0$: $f(x,t) \sim_0 t^{x-1}$ qui est intégrable d'après \textsc{Riemann} si et seulement si $1-x < 1$ i.e. si et seulement si $\boxed{x > 0}$.
    \end{itemize}
\end{preuve}
\underline{Dérivées successives:} \\
Utiliser une domination locale sur un segment $[a, A] \subset \R_+^\star$ par la fonction:
$$\varphi_k:t \mapsto 
\begin{cases}
    |\ln t |^k \me^{-t} t^{a-1}, & \text{si } t \in ]0, 1] \\
    |\ln t |^k \me^{-t} t^{A-1}, & \text{si } t > 1
\end{cases}
$$
$$\boxed{\forall k \in \N,\ \forall x \in \R_+^\star,\ \Gamma^{(k)}(x) = \int_{0}^{+\infty} (\ln t)^k t^{x-1} \me^{-t} \d t}$$

\underline{Exercice:} \url{https://share.miple.co/content/t8BIcXSjdEslq}

\subsection{Fonction bêta}
\begin{itemize}
    \item Pour tout $(p,q) \in \N^2$, on note 
    $$I_{p,q} \defeq \int_{0}^{1} x^p (1-x)^q \d x$$
    \begin{enumerate}
        \item Déterminer une relation entre $I_{p,q}$ et $I_{p+1, q-1}$ grâce à une IPP.
        \item En déduire l'expression de $I_{p,q}$ à l'aide de factorielles.
        $$\boxed{I_{p,q} = \frac{p! q!}{(p + q + 1)!}}$$
    \end{enumerate}
\end{itemize}
\url{https://fr.wikipedia.org/wiki/Intégrale_d'Euler} \\
\cite{calcul_infinitesimal} Chapitre IV, 3 Intégrales eulériennes, page 125.

\begin{subappendices}
% \section{Théorèmes utilisés}

\printindex[theoremesutilises]

\begin{theo}[Approximation par des fonctions en escalier]
\label{theo:approximationescalier}
\end{theo}

\begin{theo}[Intégration par parties généralisée, Source : \cite{acamanes}]\label{theo:ippgeneralisees}
    Soient $f$ et $g$ deux fonctions de classe $\mathscr{C}^1$ sur $I$. Si la fonction $fg$ a une limite finie en $a$ et en $b$, alors les intégrales
    \[
    \int_a^b f'(t)g(t) \d t \text{ et } \int_a^b f(t) g'(t) \d t
    \]
    sont de même nature. Si ces quantités sont convergentes, en notant
    \[
        [f(t)g(t)]_a^b = \lim_{x \to b^-} \big(f(x)g(x)\big) - \lim_{x \to a^+} \big(f(x)g(x)\big),
    \]
    on obtient la relation
    \[
    \int_a^b f'(t) g(t) \d t = \left[f(t)g(t)\right]_a^b - \int_a^b f(t) g'(t) \d t.
    \]
\end{theo}

% \setchapterstyle{plain} % Output plain chapters from this point onwards

% \begin{subappendices}
\section{Diagrammes de décision}

% Diagrammes de decision
\subsection{Calculs d'intégrales}
%========
\chapter{Calculs d'intégrales}

\bigskip

\begin{center}
\begin{tikzpicture}[scale=0.75, every node/.style={transform shape}]
  % Place nodes
  \node [block] (init) {$\int_a^b f(t) \d t =~?$};
  %% Dérivées usuelles
  \node [decision, below=of init] (usuelle) {$f$ fonction usuelle};
  \node [cloud, right=of usuelle, yshift=1cm] (usuelle_oui) {$f(x) = c \rightsquigarrow \int_a^b f(t) \d t = \left[c t\right]_{a}^{b}$\\
  $f(x) = x^\alpha \rightsquigarrow \int_a^b f(t) \d t = \left[\frac{t^{\alpha+1}}{\alpha+1}\right]_{a}^{b},\, \alpha \neq -1$\\
  $f(x) = \frac{1}{x} \rightsquigarrow \int_a^b f(t) \d t = \left[\ln(t)\right]_{a}^{b}$\\
  $f(x) = \e^{\alpha x} \rightsquigarrow \int_a^b f(t) \d t = \left[\frac{\e^{\alpha x}}{\alpha}\right]_{a}^{b},\, \alpha \neq 0$\\
  };
  %% linearite
  \node [decision, text width=10em, below=of usuelle] (linearite) {$f(x) = a u'(x) + b v'(x)$};
  \node [cloud, right=of linearite] (linearite_oui) {$\int_a^b f(t) \d t = \left[a u(t) + b v(t)\right]_{a}^{b}$};
  %% puissance
  \node [decision, text width=10em, below=of linearite] (puissance) {$f(x) = u'(x) u(x)^\alpha$};
  \node [cloud, right=of puissance] (puissance_oui) {$\int_a^b f(t) \d t = \left[\frac{u(t)^{\alpha+1}}{\alpha+1}\right]_{a}^{b},\, \alpha \neq -1$};
  %% logarithme
  \node [decision, text width=10em, below=of puissance] (logarithme) {$f(x) = \frac{u'(x)}{u(x)}$};
  \node [cloud, right=of logarithme] (logarithme_oui) {$\int_a^b f(t) \d t = \left[\ln\abs{u(t)}\right]_{a}^{b}$};
  %% exponentielle
  \node [decision, text width=10em, below=of logarithme] (exponentielle) {$f(x) = u'(x) \e^{u(x)}$};
  \node [cloud, right=of exponentielle] (exponentielle_oui) {$\int_a^b f(t) \d t = \left[\e^{u(t)}\right]_{a}^{b}$};
  %% produit
  \node [decision, text width=10em, below=of exponentielle] (produit) {$f(x) = u'(x) \times v(x)$};
  \node [cloud, right=of produit] (produit_oui) {Intégration par parties\\
  $u$ et $v$ dérivables et de dérivées continues sur $[a, b]$\\
  $\int_a^b f(t) \d t = \left[u(t) v(t)\right]_{a}^{b} - \int_a^b u(t) v'(t) \d t$\\
  };

  
  % Draw edges
  \path [line] (init) --
      node [color=black, sloped, above] {}
      (usuelle);
  % usuelle
  \path [line] (usuelle) --
      node [color=black, sloped, above] {oui}
      (usuelle_oui);
  \path [line] (usuelle) --
      node [color=black, sloped, above] {non}
      (linearite);
  % linearite
  \path [line] (linearite) --
      node [color=black, sloped, above] {oui}
      (linearite_oui);
  \path [line] (linearite) --
      node [color=black, sloped, above] {non}
      (puissance);
  % \path [draw, very thick, color=black] (linearite_oui.east) --
      % ($(produit_oui.east) + (4,0)$);
  % % produit
  % \path [line] (produit) --
      % node [color=black, sloped, above] {oui}
      % (produit_oui);
  % \path [line] (produit) --
      % node [color=black, sloped, above] {non}
      % (quotient);
  % \path [draw, very thick, color=black] (produit_oui.east) --
      % ($(produit_oui.east) + (4,0)$);
  % % quotient
  % \path [line] (quotient) --
      % node [color=black, sloped, above] {oui}
      % (quotient_oui);
  % \path [line] (quotient) --
      % node [color=black, sloped, above] {non}
      % (puissance);
  % \path [draw, very thick, color=black] (quotient_oui.east) --
      % ($(produit_oui.east) + (4,0)$);
  % puissance
  \path [line] (puissance) --
      node [color=black, sloped, above] {oui}
      (puissance_oui);
  \path [line] (puissance) --
      node [color=black, sloped, above] {non}
      (logarithme);
  % \path [draw, very thick, color=black] (puissance_oui.east) --
      % ($(produit_oui.east) + (4,0)$);
  % logarithme
  \path [line] (logarithme) --
      node [color=black, sloped, above] {oui}
      (logarithme_oui);
  \path [line] (logarithme) --
      node [color=black, sloped, above] {non}
      (exponentielle);
  % \path [draw, very thick, color=black] (logarithme_oui.east) --
      % ($(produit_oui.east) + (4,0)$);
  % exponentielle
  \path [line] (exponentielle) --
      node [color=black, sloped, above] {oui}
      (exponentielle_oui);
  \path [line] (exponentielle) --
      node [color=black, sloped, above] {non}
      (produit);
  % \path [line] (exponentielle_oui.east) --
      % ($(produit_oui.east) + (4,0)$)
      % -- ($(init.east) + (13, 0)$) --
      % node [color=black, sloped, above] {Calcul de $u',\, v'$}
      % (init.east);
  % produit
  \path [line] (produit) --
      node [color=black, sloped, above] {oui}
      (produit_oui);
  \path [line] (produit_oui.east) --
      ($(produit_oui.east) + (3,0)$)
      to[bend right=50]
      ($(init.east) + (12,0)$) --
      node [color=black, sloped, above] {Calcul de la nouvelle intégrale} 
      (init.east);
      \end{tikzpicture}
      \end{center}
\subsection{Intégrales généralisées}
\todoarmand{Reprendre l'espacement entre les blocs}
\begin{center}
\begin{tikzpicture}[scale=1, every node/.style={transform shape}]

  \node [block] (init) {$\displaystyle\int_I f$ ?};
  \node [decision, below =of init] (regularite) {Régularité ?};
  \node [decision, right =of regularite] (decoupage) {$\interoo{a}{b}$\\$=$\\$\interof{a}{c} \cup \interfo{c}{b}$};
  \node [decision, below =of regularite] (reel) {$a \in \R$};
  \node [decision, right =of reel] (plg) {Prolongement \\ continu};
  \node [cloud, right =of plg] (cv) {Converge};
  \node [decision, below =of reel] (primitive) {$\displaystyle\int^x f = F(x)$};
  \node [decision, right =of primitive] (limfct) {$F(a) \to \ell \in \mathbb{R}$};
  \node [decision, below =of primitive] (positive) {$f \geqslant 0$};
  \node [decision, right =of positive] (comparaison) {$f = O_a(g)$};
  \node [cloud, above right =of comparaison, yshift=-1.7cm] (dv) {diverge${}^\ast$};
  \node [decision, left =of positive] (integrable) {$\displaystyle\int_I |f|$};
  \node [decision, below =of positive] (chgtvar) {Changement \\ de variables};
  \node [block, right =of chgtvar] (chgtvarint) {$\displaystyle\int_J |\varphi'|  \cdot f \circ \varphi$\\ converge ?};
  \node [decision, below =of chgtvar] (ipp) {\textsc{i.p.p.}};
  \node [block, right =of ipp] (ippint) {$\displaystyle\int_I u' v$\\ converge ?};
  \node [cloud, left =of ipp, xshift=2cm] (etc) {Autres idées ?};

  % Draw edges
  \path [line] (init) -- (regularite);
  \path [line] ($(regularite)!0.5!(reel)$) coordinate (rrm)
      (regularite) -- node [above] {\contour{white}{$1$ pt}}
                      node [below] {\contour{white}{$\interof{a}{c}$, $\interfo{c}{a}$}}
      (reel);
  \path [line] (regularite) -- node [above] {plusieurs} node [below] {pts} (decoupage);
  \path [draw, very thick] (decoupage) |- (rrm);
  \path [line] (reel) -- node [midway] {\contour{white}{oui}} (plg);
  \path [line] (plg) -- node [midway] {\contour{white}{oui}} (cv);
  \path [line] ($(reel)!0.5!(primitive)$) coordinate (rpm)
    (reel) -- node [near start, sloped, above] {non} (primitive);
  \path [draw, very thick] (plg) |- node [near start, sloped, above] {non} (rpm);
  \path [line] (primitive) -- node [midway] {\contour{white}{oui}} (limfct);
  \path [line] ($(primitive)!0.5!(positive)$) coordinate (ppm)
    (primitive) -- node [near start, sloped, below] {non} (positive);
  \path [line] ([yshift=5pt]limfct.east) -| node [near start, above] {$\ell\in\R$} (cv);
  \path [line] ([yshift=-5pt] limfct.east) -| node [near end, yshift=3pt] {\contour{white}{$\ell\in \{-\infty, +\infty\}$}} (dv);
  \path [draw, very thick] (limfct) |- node [near start, sloped, above] {non} (ppm);
  \path [line] ([yshift=5pt] positive.west) -- node [above] {non (1)} ([yshift=5pt] integrable.east);
  \path [line] ([yshift=-5pt] integrable.east) -- ([yshift=-5pt] positive.west);
  \path [line] (positive) -- (comparaison);
  \path [line] ([yshift=5pt] comparaison.east) -| node [near end, yshift=-5pt] {\contour{white}{$f \sim_a g$ et $\displaystyle\int_I |g|$ diverge}} % node [near start, below] {$\displaystyle\int_I |g|$ diverge} 
  (dv);
  \path [line] ([yshift=-5pt] comparaison.east) -| node [below, sloped, near end] {$\int_I |g|$ converge} (cv);
  \path [line] ($(positive)!0.5!(chgtvar)$) coordinate (pcm)
      (positive) -- node [sloped, below] {non (2)} (chgtvar);


  \path [draw, very thick] (comparaison.south) |- (pcm);
  \path [line] (chgtvar) -- node [midway] {\contour{white}{$\mathscr{C}^1$ bijectif}} (chgtvarint);
  \path [line] (chgtvarint.south west) -- node [midway, sloped] {\contour{white}{non}} (ipp.north east);
  \path [line] (ipp) -- node [above] {crochet} node [below] {converge} (ippint);
  \path [line] (ippint.east) -- ([xshift=5.7cm]ippint.east) |- (init.east);
  \path [line] (ipp.west) -- (etc.east);
  \path [line] (chgtvarint.east) -- ([xshift=5cm]chgtvarint.east) |- (init.east);\node [block] (init) {$\displaystyle\int_I f$ ?};
  \node [decision, below =of init] (regularite) {Régularité ?};

\end{tikzpicture}
\end{center}
%
$\ast$ Si $g = o_a(f)$ sont positives et $\int_a g$ diverge, alors $\int_a f$ diverge.


% \printbibliography[heading=subbibintoc, filter=false
% ,notkeyword={livre},notkeyword={sujetconcours},notkeyword={siteweb}
% ]

\section{Références}
% \addcontentsline{toc}{section}{\protect\numberline{}Références}%
% \addcontentsline{margintoc}{section}{\protect\numberline{}Références}%

\todoarmand{Modifier les formats de citation}

\subsubsection*{Livres}
\printbibliography[keyword={livre},heading=none
% title={Livres}, heading=subbibintoc
]
\subsubsection*{Sujets de concours}
\printbibliography[keyword={sujetconcours},heading=none
% title={Sujets de concours}, heading=subbibintoc
]
\subsubsection*{Sites web}
\printbibliography[keyword={siteweb},heading=none]

\end{subappendices}

% \setchapterstyle{kao} % Choose the default chapter heading style

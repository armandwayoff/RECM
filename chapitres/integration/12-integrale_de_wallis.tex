\section{Intégrale de \nom{Wallis}} \label{integrale_wallis}

\marginnote[0cm]{
Les intégrales de \nom{Wallis} ont été introduites par John \nom{Wallis} (1616--1703), notamment pour développer le nombre $\pi$ en un produit infini de rationnels; le \textsl{produit de \nom{Wallis}}, énoncé en 1656 dans son ouvrage \emph{Arithmetica infinitorum}. \href{https://fr.wikipedia.org/wiki/Intégrale_de_Wallis}{Source}.
}

\begin{defi}[Intégrale de \nom{Wallis}]
Pour tout $n$ entier naturel, on nomme \definir{intégrale de \nom{Wallis}} l'intégrale définie par
\[
\Wallis_n \defeq \int_{0}^{\frac{\pi}{2}} \sin(x)^n \d x.
\]
\end{defi}
\begin{prop}
De façon équivalente, on peut définir l'intégrale de \nom{Wallis} par 
\[
\Wallis_n = \int_{0}^{\frac{\pi}{2}} \cos(x)^n \d x.
\]
\end{prop}
\begin{elemdemo}
Effectuer le changement de variable $t = \frac{\pi}{2} - x$. 
\end{elemdemo}

%-----------
\subsection{Calcul de l'intégrale et démonstration de ses propriétés}

\begin{theo}{} \labprop{prop_wallis} Pour tout $n$ entier naturel, les intégrales de \nom{Wallis} vérifient les propriétés suivantes
\begin{enumerate}[label=(\roman*)]
    \item $(n + 2) \Wallis_{n+2} = (n + 1) \Wallis_n$,  
    \item $\displaystyle \Wallis_{2n} = \frac{\pi}{2^{2n+1}} \binom{2n}{n}$ et $\Wallis_{2n+1} = \frac{2^{2n} (n!)^2}{(2n+1)!}$,
    \item $\displaystyle \Wallis_n \Wallis_{n+1} = \frac{\pi}{2(n+1)}$,
    \item $\Wallis_{n+1} \sim \Wallis_n$ et $\displaystyle \Wallis_n \sim \sqrt{\frac{\pi}{2n}}$.
\end{enumerate}
\end{theo}
Démontrons l'ensemble de ces résultats grâce à l'exercice suivant.
\begin{exercice}\label{exo:propWallis}
\begin{questions}
\item Montrer que la suite $(\Wallis_n)_{n\in\N}$ est décroissante et minorée.

\item Pour tout $n$ entier naturel non nul, montrer que $(n + 2) \Wallis_{n+2} = (n + 1) \Wallis_n$.

\item Pour tout $p \in \N$, en déduire que $\Wallis_{2p} = \frac{1}{2^{2p}} \binom{2p}{p} \frac{\pi}{2}$.

\item Pour tout $p \in \N$, montrer de manière analogue que $\Wallis_{2p+1} = \frac{2^{2p} (p!)^2}{(2p+1)!}$.

\item En déduire que $\Wallis_{n+1} \sim \Wallis_n$.

\item Montrer que $\Wallis_n \Wallis_{n+1} = \frac{\pi}{2 (n + 1)}$.

\item En déduire que $\Wallis_n \sim \sqrt{\frac{\pi}{2n}}$.
\end{questions}
\end{exercice}

\begin{solution}
\begin{reponses}
\item Par linéarité de l'intégrale, pour tout $n \in \N$, $\Wallis_{n+1} - \Wallis_n = \int_0^{\frac{\pi}{2}} \sin(t)^n \big(\sin(t) - 1\big) \d t$. Or, pour tout $t \in \interff{0}{\frac{\pi}{2}}$, $0 \leqslant \sin(t) \leqslant 1$. Ainsi, l'intégrande est à valeurs négatives et en utilisant la croissance de l'intégrale, on obtient bien $0 \leqslant \Wallis_{n+1} \leqslant \Wallis_n$. La suite $(\Wallis_n)_{n\in\N}$ converge donc par le \theoremeutilise{théorème de la limite monotone}{theo:limitemonotone}.
\item Calculons $\Wallis_{n+2}$ en effectuant une intégration par parties. On pose $u(t) = - \cos(t)$ et $v(t) = \sin(t)^{n+1}$, toutes deux de classe $\mathscr{C}^1$ sur $\interff{0}{\frac{\pi}{2}}$ et on calcule
    \begin{align*}
        \Wallis_{n+2} &= \underbrace{\left[ -\cos(t) \sin(t)^{n+1} \right]_0^{\pi/2}}_{=0} + (n+1) \int_0^{\pi/2} \cos(t)^2 \sin(t)^n \d t \\
        &= (n+1) \int_0^{\frac{\pi}{2}} \big(1 - \sin(t)^2 \big) \sin(t)^n \d t \\
        &= (n+1) \Wallis_n - (n+1) \Wallis_{n+2} \\
        \text{soit } (n+2) \Wallis_{n+2} &= (n+1) \Wallis_n.
\end{align*}

\item On remarque que
\[
\Wallis_0 = \int_0^{\frac{\pi}{2}} 1 \d x = \frac{\pi}{2}.
\]
Soit $p \in \N$. Comme la fonction sinus est de signe constant et non identiquement nulle sur $\interff{0}{\frac{\pi}{2}}$, alors $\Wallis_k$ est non nulle pour tout $k$ entier naturel. Alors, d'après la relation de la question précédente,
\[
\frac{\Wallis_{2k}}{\Wallis_{2k-2}} = \frac{2k-1}{2k}
\]
et en utilisant un produit télescopique pour $k \in \interent{1}{p}$,
\[
\prod_{k=1}^p \frac{\Wallis_{2k}}{\Wallis_{2k-2}} = \prod_{k=1}^p \frac{2k-1}{2k} \quad \text{soit} \quad \Wallis_{2p} = \left[ \prod_{k=1}^p \frac{2k-1}{2k} \right] \Wallis_0.
\] 
Nous pouvons récrire le produit de la manière suivante
\begin{align*}
\Wallis_{2p} &= \frac{\prod\limits_{k=1}^p (2k-1)}{\prod\limits_{k=1}^{p} 2 k} \Wallis_0 = \frac{\left[\prod\limits_{k=1}^p (2k-1)\right] \times \left[\prod\limits_{k=1}^{p} (2k)\right]}{\left[\prod\limits_{k=1}^p (2k)\right]^2} \Wallis_0 = \frac{(2p)!}{2^{2p}(p!)^2} \frac{\pi}{2},
\end{align*}
ce qui fournit le résultat demandé. 

\item De manière analogue, on remarque que
\[
\Wallis_1
= \int_0^{\frac{\pi}{2}} \sin(t) \d t
= \left[-\cos(t)\right]_0^{\frac{\pi}{2}}
= 1.
\]

Ainsi, pour $p \in \N$,
\[
\frac{\Wallis_{2p+1}}{\Wallis_{2p-1}} = \frac{2p}{2p+1}
\]
et comme précédemment, en utilisant un produit télescopique, 
\begin{align*}
\prod_{k=1}^p \frac{\Wallis_{2k+1}}{\Wallis_{2k-1}} &= \prod_{k=1}^p \frac{2k}{2k+1}\\
\frac{\Wallis_{2p+1}}{\Wallis_1} &= \frac{\left[\prod\limits_{k=1}^p (2k)\right]^2}{\left[\prod\limits_{k=1}^p (2k+1)\right] \times \left[\prod\limits_{k=1}^p (2k)\right]}\\
\intertext{soit}
\Wallis_{2p+1} &= \frac{2^{2p} (p!)^2}{(2p+1)!}.
\end{align*}

\item D'après les questions précédentes, pour tout $n$ entier naturel non nul,
\begin{align*}
\Wallis_{n-1} &\leqslant \Wallis_n \leqslant \Wallis_{n+1}\\
\frac{\Wallis_{n-1}}{\Wallis_{n+1}} &\leqslant \frac{\Wallis_n}{\Wallis_{n+1}} \leqslant 1.
\end{align*}
Or, d'après la question précédente, $\frac{\Wallis_{n-1}}{\Wallis_{n+1}} = \frac{n+1}{n}$. Ainsi, d'après le \theoremeutilise{théorème d'encadrement}{theo:encadrement}, $\Wallis_n \sim \Wallis_{n+1}$.

\item D'après la question précédente,
\[
\Wallis_{2p} \Wallis_{2p+1} = \frac{\pi}{2 (2 p + 1)}
\quad \text{et} \quad
\Wallis_{2p+1} \Wallis_{2p+2} = \frac{\pi}{2 (2p+2)}.
\]
Ainsi, pour tout $n$ entier naturel,
\[
\Wallis_n \Wallis_{n+1} = \frac{\pi}{2 (n + 1)}.
\]

\item En utilisant les propriétés des équivalents, $\Wallis_n{}^2 \sim \frac{\pi}{2 n}$ soit
\[
\Wallis_n \sim \sqrt{\frac{\pi}{2 n}}.
\]
\end{reponses}
\end{solution}

%-----------
\subsection{Formule de \nom{Stirling}} \label{preuve_stirling}

\begin{theo}[Formule de \nom{Stirling}]
\[
n! \sim \sqrt{2 \pi n} \left(\frac{n}{\e}\right)^n
\]
\end{theo}

\begin{exercice}
Pour tout $n$ entier naturel non nul, on pose $u_n = \frac{n! \e^n}{n^{n + \frac{1}{2}}}$.
\begin{questions}
\item Montrer que $\left(n + \frac{1}{2}\right) \ln\mathopen{}\left(1 + \frac{1}{n}\right) - 1\sim \frac{1}{12 n^2}$.

\item En déduire que la série $\sum \ln \frac{u_{n+1}}{u_n}$ converge.

\item Montrer que la suite $(u_n)_{n\in\Ne}$ converge vers un réel $\ell$ strictement positif.

\item À l'aide des intégrales de \nom{Wallis}, déterminer la valeur~de~$\ell$.

\item En déduire la formule de \nom{Stirling}.
\end{questions}
\end{exercice}

\begin{solution}
\begin{reponses}
\item En utilisant les développements limités classiques,
\begin{align*}
\left(n + \frac{1}{2}\right) \ln\mathopen{}\left(1 + \frac{1}{n}\right) - 1
&= \left(n + \frac{1}{2}\right) \left(\frac{1}{n} - \frac{1}{2 n^2} + \frac{1}{3 n^3} + o\mathopen{}\left(\frac{1}{n^3}\right)\right) - 1\\
&= 1 - \frac{1}{2 n} + \frac{1}{3 n^2} + \frac{1}{2 n} - \frac{1}{4 n^2} + o\mathopen{}\left(\frac{1}{n^2}\right) - 1\\
&= \frac{1}{12 n^2} + o\mathopen{}\left(\frac{1}{n^2}\right).
\end{align*}

On obtient ainsi l'équivalent annoncé.

\item On remarque que
\[
\frac{u_{n+1}}{u_n}
= \frac{(n + 1)!\, \e^{n+1}}{(n+1)^{n+\frac{3}{2}}} \times \frac{n^{n+\frac{1}{2}}}{n!\, \e^n}
= \e \left(1 + \frac{1}{n}\right)^{-\mathopen{}\left(n + \frac{1}{2}\right)}.
\]
Ainsi, d'après la question précédente et le \theoremeutilise{théorème de comparaison aux séries de \nom{Riemann}}{theo:comparaisonseriesriemann}, la série de terme général $\ln\frac{u_{n+1}}{u_n}$ converge.

\item Pour tout $N \geqslant 2$, en reconnaissant une somme télescopique,
\[
\sum_{n=1}^{N-1} \ln\frac{u_{n+1}}{u_n} = \ln(u_N) - \ln(u_1).
\]

Ainsi, d'après la question précédente, la suite $(\ln(u_n))_{n\in\Ne}$ converge vers un réel $\tilde{\ell}$.

D'après la continuité de la fonction exponentielle, la suite $(u_n)_{n\in\Ne}$ converge vers $\ell = \e^{\tilde{\ell}}$ qui est bien un réel strictement positif.

Ainsi, $n! \sim \ell \left(\frac{n}{\e}\right)^n \sqrt{n}$.

\item D'après les résultats sur les intégrales de \nom{Wallis},
\[
\Wallis_{2p}
\sim \sqrt{\frac{\pi}{2\times 2p}}
\sim \sqrt{\frac{\pi}{4 p}}.
\]

Par ailleurs,
\begin{align*}
\Wallis_{2p}
= \frac{(2p)!}{2^{2p} (p!)^2} \frac{\pi}{2}
\sim \frac{\ell\, \e^{2p} (2p)^{2p} \sqrt{2p}}{\ell^2\, \e^{2p} (2p)^{2p} p} \times \frac{\pi}{2}
\sim \frac{\pi}{\ell \sqrt{2 p}}.
\end{align*}

Ainsi,
\begin{align*}
\sqrt{\frac{\pi}{4 p}} \sim \frac{\pi}{\ell \sqrt{2p}}
\quad \text{soit} \quad 
\ell = \sqrt{2 \pi}.
\end{align*}

\item D'après les questions précédentes, on obtient l'équivalent annoncé
\[
n! \sim \sqrt{2 \pi n} \left(\frac{n}{\e}\right)^n.
\]
\end{reponses}
\end{solution}

%-----------
\subsection{Séries génératrices}

% \todoinline{C'est rigolo ! Est-ce qu'on pourrait retrouver l'expression de $\Wallis_n$ à l'aide d'un produit de DSE ? Il y a aussi une application ici : https://math-os.com/coefficient-binomial-central/}

\begin{prop}{}
Pour tout $x \in \interoo{-1}{1}$, la série génératrice des termes impairs est
$$\sum_{p=0}^\infty \Wallis_{2p+1} x^{2p+1} = \frac{\arcsin x}{\sqrt{1-x^2}}.$$

Pour tout $x \in \interoo{-1}{1}$, la série génératrice des termes pairs est 
$$\sum_{p=0}^\infty \Wallis_{2p} x^{2p} = \frac{\pi}{2} \frac{1}{\sqrt{1-x^2}}.$$

Pour tout $x \in \interoo{-1}{1}$,
\[
\sum_{n=0}^{+\infty} \Wallis_n x^n = \frac{2}{\sqrt{1 - x^2}} \arctan\sqrt{\frac{x + 1}{x - 1}}.
\]
\end{prop}

% \todoinline{Citer \url{https://math-os.com/coefficient-binomial-central/}. Je réécris certaines parties pour préciser et détailler la démarche.}

\source{Exercice inspiré de l'article \href{https://math-os.com/coefficient-binomial-central/}{Coefficient Binomial Central : un aperçu} (Annexe 2) du site \href{https://math-os.com/bienvenue/}{math-os.com} de René \textsc{Adad}}
\begin{exercice}
On considère la formule dans le cas des termes impairs. Pour tout $p$ entier naturel, on pose $\fonctionligne[f_p]{t}{\cos(t)^{2p+1} x^{2p+1}}$.
\begin{questions}
\item Montrer que $\sum f_p$ converge normalement sur $\interff{0}{\frac{\pi}{2}}$.

\item En déduire que
\[
\sum_{p=0}^{+\infty} \Wallis_{2p+1} x^{2p+1} = \int_0^{\frac{\pi}{2}} \frac{x \cos(t)}{1 - x^2 \cos(t)^2} \d t.
\]

\item Utiliser le changement de variable $\fonctionligne[\varphi]{u}{\arcsin(u/x)}$ pour conclure.

\item Reprendre le raisonnement précédent avec le changement de variable $\fonctionligne[\varphi]{u}{\tan\frac{u}{2}}$ pour conclure dans le cas pair.

\item Reprendre les raisonnements précédents pour démontrer la dernière formule.
\end{questions}
\end{exercice}

\begin{solution}
\begin{reponses}
\item Comme la fonction cosinus est bornée par $1$,
\[
\norm{f_p}_{\infty} = \module{x}^{2p+1}.
\]

Or, pour tout $x \in \interoo{-1}{1}$, $\sum \module{x}^{2p+1}$ converge. Ainsi, $\sum \norm{f_p}_{\infty}$ converge.

\item Soit $x \in \interoo{-1}{1}$. D'après le \theoremeutilise{théorème d'interversion série / intégrale}{theo:interversionserieintegrale},
\begin{align*}
\sum_{p=0}^\infty \Wallis_{2p+1} x^{2p+1}
&= \sum_{p=0}^\infty x^{2p+1} \int_0^{\pi/2} \cos(t)^{2p+1} \d t\\
&= \int_0^{\pi/2} x\cos(t) \sum_{p=0}^\infty \Big(\big(x\cos(t) \big)^2 \Big)^p \d t \\
&= \int_0^{\pi/2} \frac{x \cos(t)}{1 - x^2 \cos(t)^2} \d t.
\end{align*}

\item Le cas $x = 0$ est trivial. Pour $x \neq 0$, la fonction $\fonctionens[\varphi]{\interff{0}{x}}{\interff{0}{\pi/2}},\, u \mapsto \arcsin\frac{u}{x}$ est de classe $\mathscr{C}^1$. De plus,
\[
\cos(\varphi(u)) = \sqrt{1 - \frac{u^2}{x^2}}
\quad \text{et} \quad 
\varphi'(u) = \frac{1}{x \sqrt{1 - \frac{u^2}{x^2}}}.
\]

Ainsi,
% On voit rapidement que le changement de variable $u = x \cos(t)$ ne permet pas d'aboutir mais ne remplaçant $\cos(t)^2$ par $1 - \sin(t)^2$ au dénominateur puis en posant $u = x \sin(t)$, il vient:
\begin{align*}
\sum_{p=0}^\infty \Wallis_{2p+1} x^{2p+1}
&= \int_0^x \frac{x \sqrt{1 - \frac{u^2}{x^2}}}{1 - x^2 \left(1 - \frac{u^2}{x^2}\right)} \times \frac{1}{x \sqrt{1 - \frac{u^2}{x^2}}}\d u \\
&= \int_0^x \frac{1}{1 - x^2 + u^2} \d u\\
&= \frac{1}{\sqrt{1 - x^2}} \int_0^x \frac{\frac{1}{\sqrt{1 - x^2}}}{1 + \left(\frac{u}{\sqrt{1-x^2}}\right)^2} \d u \\
&= \frac{1}{\sqrt{1 - x^2}} \left[ \arctan \frac{u}{\sqrt{1 - x^2}} \right]^x_0 \\
&= \frac{1}{\sqrt{1 - x^2}} \arctan \frac{x}{\sqrt{1 - x^2}}.
\end{align*}
On obtient le résultat annoncé en utilisant le fait que pour tout $x \in \interoo{-1}{1}$, $\arctan\mathopen{}\Big(\frac{x}{\sqrt{1-x^2}} \Big) = \arcsin(x)$.
\marginnote[-5cm]{
\begin{tikzpicture}[
  scale=1.5,
  my angle/.style={
    every pic quotes/.append style={text=cyan},
    draw=cyan,
    angle radius=1.5cm,
  }]
  \coordinate (C) at (-1.5,-1);
  \coordinate (A) at (1.5,-1);
  \coordinate (B) at (1.5,1);
  \pic [my angle, "$\alpha$"] {angle=A--C--B};
  \draw (C) -- node[above] {$1$} (B) -- node[right] {$x$} (A) -- node[below] {$\sqrt{1-x^2}$} (C);
  \draw (A) +(-.25,0) |- +(0,.25);
  % \pic [my angle, "$\beta$"] {angle=C--B--A};
\end{tikzpicture}
}

\item Pour le cas pair, $\big\Vert t \mapsto \cos(t)^{2p} x^{2p} \big\Vert_{\infty} = \abs{x}^{2p}$ et la série est normalement convergente sur $\interoo{-1}{1}$. Ainsi,
\begin{align*}
\sum_{p=0}^\infty \Wallis_{2p} x^{2p} &= \int_0^{\pi/2} \frac{1}{1 - x^2 \cos(t)^2} \d t \\
&= \int_0^1 \frac{1}{1 - x^2 \left(\frac{1-u^2}{1+u^2}\right)^2} \frac{2}{1 + u^2} \d u \quad \text{en posant } u = \tan \frac{t}{2} \\
&= \int_0^1 \frac{1 + u^2}{(1 + u^2)^2 - x^2(1-u^2)^2} \d u \\
&= \int_0^1 \frac{1 + u^2}{(1 + u^2 - x(1-u^2))(1 + u^2 + x(1-u^2))} \d u \\
&= \frac{1}{2} \int_0^1 \left[ \frac{1}{1 + u^2 - x(1-u^2)} + \frac{1}{1 + u^2 + x(1-u^2)}\right] \d u \\
&= \frac{1}{2} \int_0^1 \left[\frac{1}{1 - x + (1+x)u^2} + \frac{1}{1 + x + (1-x)u^2} \right] \d u \\
&= \frac{1}{\sqrt{1-x^2}} \arctan \sqrt{\frac{x+1}{x-1}} + \frac{1}{\sqrt{1-x^2}} \arctan \sqrt{\frac{x-1}{x+1}}\\
\sum_{p=0}^\infty \Wallis_{2p} x^{2p} &= \frac{\pi}{2} \frac{1}{\sqrt{1-x^2}} \quad \text{car $\arctan u + \arctan \frac{1}{u} = \frac{\pi}{2}$ pour tout $u > 0$}
\end{align*}

\item Pour la dernière formule, $\norm{t \mapsto \cos(t)^p x^p}_{\infty} = \module{x}^p$ et la série converge normalement sur $\interoo{-1}{1}$. Ainsi,
\begin{align*}
\sum_{n=0}^{+ \infty} \Wallis_n x^n &= \sum_{n=0}^{+ \infty} \left[ x^n \int_0^{\pi/2} \cos^n t \d t \right] = \int_0^{\pi/2} \left( \sum_{n=0}^{+\infty} x^n \cos^n t \right) \d t \\
&=\int_0^{\pi/2} \frac{1}{1 - x \cos t} \d t \\
&= \int_0^1 \frac{1}{1 - x \frac{1-u^2}{1+u^2}} \frac{2}{1 + u^2} \d u \quad \text{en posant } u = \tan \frac{t}{2} \\
&= 2 \int_0^1 \frac{1}{(1+x)u^2 + (1-x)} \d u \\
&= 2 \times \frac{1}{1+x} \times \frac{1}{\sqrt{\frac{1-x}{1+x}}} \left[ \arctan\mathopen{}\left( \frac{u}{\sqrt{\frac{1-x}{1+x}}} \right) \right]_0^1 \\
\sum_{n=0}^{+ \infty} \Wallis_n x^n &= \frac{2}{\sqrt{1-x^2}} \arctan \sqrt{\frac{x+1}{x-1}}.
\end{align*}
\end{reponses}
\end{solution}

%---------------

\begin{exercice}
\source{Oral : ENSAM - 2016}
Montrer la convergence et déterminer la somme de la série $\sum (-1)^n \Wallis_n$.
\end{exercice}

\begin{solution}
\begin{enumerate}
\item Posons $u_n = (-1)^n \Wallis_n$. D'après les propriétés des intégrales de \nom{Wallis} démontrées dans l'exercice \ref{exo:propWallis},
\begin{itemize}
\item pour tout $n \in \N$, $u_n u_{n+1} < 0$,
\item la suite de terme général $\abs{u_n} = \Wallis_n$ est décroissante,
\item la suite $(\Wallis_n)_{n \in \N}$ converge vers $0$.
\end{itemize}
Ainsi, d'après le \theoremeutilise{théorème des séries alternées}{theo:seriesalternees}, la série converge.

\item On pose $\fonctionligne[f_n]{t}{\sum\limits_{k=0}^n (-1)^k \cos(t)^k}$. On remarque que
\begin{itemize}
\item la fonction $f_n$ est continue sur~$\interff{0}{\pi/2}$,
\item la suite $(f_n)_{n \in \N}$ converge simplement vers la fonction $t \mapsto \frac{1}{1 + \cos(t)}$ qui est une fonction continue sur~$\interfo{0}{\pi/2}$,
\item d'après le \theoremeutilise{théorème des séries alternées}{theo:seriesalternees}, $\abs{f_n(t)} \leqslant \big|\cos(t)^{n+1}\big| \leqslant 1$, la fonction constante égale à $1$ étant intégrable sur $\interff{0}{\pi/2}$.
\end{itemize}
Ainsi, d'après le \theoremeutilise{théorème de convergence dominée}{theo:convergencedominee},
\[
\sum_{n=0}^{+\infty} (-1)^n \int_0^{\pi/2} \cos(t)^n \d t
= \int_0^{\pi/2} \frac{\d t}{1 + \cos(t)}
= \frac{1}{2} \int_0^{\pi/2} \frac{1}{\cos(t/2)^2} \d t
= 1.
\]
\end{enumerate}
\end{solution}

\begin{remarque}
Pour la majoration de la suite $(f_n)_{n \in \N}$, on peut également utiliser que
\[
\abs{f_n}
= \module{\frac{1 - (-1)^{n+1} \cos^{n+1}(t)}{1 + \cos(t)}}
\leqslant 2.
\]

\end{remarque}

% \begin{exercice}{}
% Calculer $\sum\limits_{n=0}^\infty (-1)^n \Wallis_n$.
% \end{exercice}

% \begin{preuve}
    % \marginnote[0cm]{Source : \href{http://exo7.emath.fr/ficpdf/fic00126.pdf}{Exercices de Jean-Louis \textsc{Rouget} (fic00126) -- \textsf{http://exo7.emath.fr}}}
    % D'après \vrefprop{prop_wallis}, $\Wallis_n \sim \sqrt{\frac{\pi}{2n}}$ et la règle de \textsc{d'Alembert} fournit $R = 1$. Soit $x \in ]-1, 1[$. \\
    % Pour tout $t \in \left[ 0, \frac{\pi}{2} \right]$ et tout entier naturel $n$, $|x^n \cos^n t| \leqslant |x|^n$. Comme la série numérique de terme général $|x|^n$ converge, la série de fonctions de terme général $t \mapsto x^n \cos^n t$ est normalement convergente et donc uniformément convergente sur le segment $\left[ 0, \frac{\pi}{2} \right]$. D'après le théorème d'intégration terme à terme sur un segment, 
    % \begin{align*}
        % \sum_{n=0}^{+ \infty} \Wallis_n x^n &= \sum_{n=0}^{+ \infty} \left[ x^n \int_0^{\pi/2} \cos^n t \d t \right] = \int_0^{\pi/2} \left( \sum_{n=0}^{+\infty} x^n \cos^n t \right) \d t \\
        % &=\int_0^{\pi/2} \frac{1}{1 - x \cos t} \d t \\
        % &= \int_0^1 \frac{1}{1 - x \frac{1-u^2}{1+u^2}} \frac{2}{1 + u^2} \d u \quad \text{en posant } u = \tan \frac{t}{2} \\
        % &= 2 \int_0^1 \frac{1}{(1+x)u^2 + (1-x)} \d u \\
        % &= 2 \times \frac{1}{1+x} \times \frac{1}{\sqrt{\frac{1-x}{1+x}}} \left[ \arctan \left( \frac{u}{\sqrt{\frac{1-x}{1+x}}} \right) \right]_0^1 \\
        % \sum_{n=0}^{+ \infty} \Wallis_n x^n &= \frac{2}{\sqrt{1-x^2}} \arctan \sqrt{\frac{x+1}{x-1}}.
    % \end{align*}
% \end{preuve}

\begin{comment}
\todoinline{Je trouve que ça fait un peu beaucoup ou alors on ne donne pas de preuve pour l'exercice suivant.}

\todoarmand{Nouvel exercice de }

\begin{exercice}
\marginpar[0cm]{Source : \cite{exos_oraux}}
    \textbf{Fonction de \textsc{Bessel} et intégrales de \textsc{Wallis}} \\
    Pour tout $x \in \R$, on note $J(x) = \displaystyle \int_0^{\pi/2} \cos\big(x \sin(t) \big) \d t$.
    \begin{enumerate}
        \item Montrer que $J$ est solution de l'équation différentielles $x y'' + y' + xy = 0$ $(E)$. 
        \item Déterminer les solutions développables en série entière de $(E)$, puis le développement en série entière de $J$. En déduire une expression des intégrales de \textsc{Wallis} d'indice pair. 
    \end{enumerate}
\end{exercice}
\end{comment}

\subsection{Volume d'une boule en dimension \texorpdfstring{$n$}{n}}

La partie suivante constitue une ouverture mais utilise des notions d'intégrales multiples qui dépassent le cadre du programme des classes préparatoires.

\begin{defi}{}
La boule unité de $\R^n$ est définie par
\[
\mathscr{B}_n = \ens[\Big]{(x_1, \dots, x_n) \in \R^n \tq \sum\limits_{i=1}^n x_i{}^2 \leqslant 1}.
\]
Le volume de $\mathscr{B}_n$ est défini par l'intégrale multiple :
\[
\mathscr{V}_n \defeq \idotsint_{x_1{}^2 + \cdots + x_n{}^2 \leqslant 1} \d x_1 \cdots \d x_n.
\]
\end{defi}

\begin{theo}{}
Pour tout $n$ entier naturel non nul,
\[
\mathscr{V}_{2n} = \frac{\pi^n}{n!}
\quad \text{et} \quad
\mathscr{V}_{2n+1} = \frac{2^{2n+1} \pi^n n!}{(2n+1)!}.
\]
\end{theo}

\begin{exercice}
Pour tout $n$ entier naturel non nul, on note
\[
I_n = \int_0^\pi \sin(t)^n \d t.
\]
\begin{questions}
\item Exprimer $I_n$ en fonction de l'intégrale de \nom{Wallis} $\Wallis_n$.

\item En déduire l'expression de $I_n$ en fonction de $n$.

\item Montrer que $\mathscr{V}_n = I_n \times \mathscr{V}_{n-1}$.

\item En déduire l'expression de $\mathscr{V}_n$ en fonction de $n$.
\end{questions}
\end{exercice}

\begin{solution}
\begin{reponses}
\item En utilisant les symétries de la fonction sinus,
\begin{align*}
I_n
% &= \int_0^{\pi} \sin(t)^n \d t\\
&= \int_0^{\frac{\pi}{2}} \sin(t)^n \d t + \int_{\frac{\pi}{2}}^\pi \sin(t)^n \d t
= 2 \int_0^{\frac{\pi}{2}} \sin(t)^n \d t
= 2 \Wallis_n.
\end{align*}

\item D'après la question précédente et les résultats de l'exercice \ref{exo:propWallis}, pour tout $n$ entier naturel,
\begin{align*}
I_{2n} = \frac{(2n)!}{2^{2n} (n!)^2} \frac{\pi}{2}
\quad \text{et} \quad
I_{2n+1} = \frac{2^{2n} (n!)^2}{(2n+1)!}.
\end{align*}

\item D'après la définition,
\begin{align*}
\mathscr{V}_n
&= \int_{-1}^1 \left(\int_{x_2{}^2 + \cdots + x_n{}^2 \leqslant 1 - x_1{}^2} \d x_2 \cdots \d x_n\right) \d x_1.
\end{align*}
Comme $\int_{x_2{}^2 + \cdots + x_n{}^2 \leqslant 1 - x_1{}^2} \d x_2 \cdots \d x_n$ est le volume de la boule de rayon $1 - x_1{}^2$, cette quantité est égale à $\big(1 - x_1{}^2\big)^{\frac{n-1}{2}} \mathscr{V}_{n-1}$. Ainsi,
\begin{align*}
\mathscr{V}_n
&= \mathscr{V}_{n-1} \int_{-1}^1 \big(1 - x_1{}^2\big)^{\frac{n-1}{2}} \d x_1
= \mathscr{V}_{n-1} \times I_n,
\end{align*}
où on a utilisé le changement de variables $x_1 = \cos(\theta)$.

\item D'après la question précédente,
\begin{align*}
\mathscr{V}_n
&= 2 \Wallis_n \mathscr{V}_{n-1}
= 2^{n-2} \prod_{k=3}^n \Wallis_k \times \mathscr{V}_2.
\end{align*}
Or, $\mathscr{V}_2$ est l'aire d'un disque de rayon $1$ qui vaut $\pi$. Alors,
\[
\mathscr{V}_n = \pi\, 2^{n-2} \prod_{k=3}^n \Wallis_k.
\]

Rappelons enfin que $\Wallis_k \Wallis_{k+1} = \frac{\pi}{2(k+1)}$. Ainsi, en distinguant selon la parité de $n$,
\begin{align*}
\mathscr{V}_{2n}
&= \pi\, 2^{2n-2} \prod_{k=3}^{2n} \Wallis_k
= \pi\, 2^{2n-2} \prod_{k=2}^{n} \frac{\pi}{2 \times 2 k}
= \frac{\pi\, 2^{2n-2} \pi^{n-1}}{2^{2(n-1)} \times n!}
= \frac{\pi^n}{n!}
\end{align*}
et
\begin{align*}
\mathscr{V}_{2n+1}
&= 2 \Wallis_{2n+1} \mathscr{V}_{2n}
= 2 \times \frac{2^{2n} (n!)^2}{(2n+1)!} \times \frac{\pi^n}{n!}
= \frac{2^{2n+1} \pi^n n!}{(2n+1)!}.
\end{align*}
\end{reponses}
\end{solution}

\begin{remarque}
Interprétons physiquement ces considérations. En physique statistique, on utilise fréquemment l'espace des phases comprenant comme dimensions les six coordonnées de position et de vitesse de chaque particule. lorsque celles-ci sont en très grand nombre, l'espace des phases a énormément de dimensions. Supposons que nous mesurions la vitesse de chacune des $n$ particules : la moyenne $V$ de ces vitesses donnera une mesure macroscopique comme la témpérature ou la pression (quantités liées à la vitesse). Reportons cette mesure sur chacun des axes d'un espace à $n$ dimensions : les points considérés seront approximativement sur une coquille de rayon $V$ et d'épaisseur $\frac{V}{\sqrt{n}}$ (voir ch. 13). On peut également considérer que les vitesses vont se répartir dans une sphère de rayon inférieur ou égal à la plus grande mesure obtenue $V_\mathrm{max}$, mais comme les points se retrouveront quasiment à la surface, la plupart des vitesses seront environ $V_\mathrm{max}$ qui sera en fait la vitesse moyenne : nous retrouvons là l'hypothèse ergodique de \nompropre{Boltzmann} et \nompropre{Maxwell}. \url{http://promenadesmaths.free.fr/fichiers_pdf/volume%20en%20dim%20n.pdf} \url{https://www.phy.ulaval.ca/fileadmin/phy/documents/Bacc_PHY/Nathalie/Espace-phase-pdf.pdf}
\end{remarque}




\subsection{\textsc{Grain de raisin}: Produit de \nom{Wallis}}

\marginnote[0cm]{
    L'expression \chevron{Grain de raisin} est une référence à l'extrait suivant \url{https://youtube.com/clip/UgkxzacmUJF7Qr3526JmMgMOwLirh_c-gddc?si=AUUGMWZPFH8aNZl7} d'une discussion entre Jean-Pierre \nom{Serre} et Christophe \nom{Ritzenthaler}.
    % \begin{figure}
    % \centering
    \begin{center}
    \includegraphics[width=1cm]{images/qrcode_grainderaisin.png}
    \end{center}
    % \end{figure}
}

\begin{prop}{Produit de \nom{Wallis}}
    Formule énoncée en 1656 par John \nom{Wallis}, dans son ouvrage \emph{Arithmetica infinitorum}
    $$\prod_{n=1}^{\infty} \frac{4n^2}{4n^2-1} = \frac{\pi}{2}.$$
\end{prop}

\begin{demo}
    Puisque $\Wallis_{2n} \sim \Wallis_{2n+1}$, 
    $$\lim_{n \to +\infty} \frac{\Wallis_{2n+1}}{\Wallis_{2n} / \frac{\pi}{2}} = \frac{\pi}{2}.$$
    Or d'après le calcul des intégrales de \nom{Wallis}:
    $$\frac{\Wallis_{2n+1}}{\Wallis_{2n} / \frac{\pi}{2}} = \frac{\frac{2^{2p}(p!)^2}{(2p+1)!}}{\frac{(2p)!}{2^{2p}(p!)^2}} = \prod_{k=1}^n \frac{4k^2}{4k^2-1}.$$
\end{demo}


    %-----------
% \subsection{À trier}

% \todoinline{Cet exercice, je le supprimerais ou alors le mettre avec les intégrales de Gauss ?}

% \begin{exercice}
    % \marginnote[0cm]{Source : \cite{fmaalouf}}
    % Pour $n \in \Ne$ et $R \in \Rpe$ on désigne par $V_n(R)$ le volume de la boule de $\R^n$ de centre $O$ et de rayon $R$, 
    % $$V_n(R) \defeq \idotsint_{x_1^2 + \cdots + x_n^2 \leqslant R^2} \d x_1 \cdots \d x_n.$$
    % Montrer que pour tout $p \in \Ne$, 
    % $$V_{2p}(R) = \frac{\pi^p R^{2p}}{p!}.$$
% \end{exercice}

% \todoinline{Ajouter une preuve.}
% \todoarmand{Je suis entrain de réfléchir à une démo, mais ça me paraît difficile sans la notion de jacobien.}


% \todoarmand{Les exercices 4 et 5 de \url{http://exo7.emath.fr/ficpdf/fic00143.pdf} donnent les liens entre calcul de la surface de la sphère unité de $\R^n$, son volume, la fonction Gamma et les intégrales de Wallis.}

% Exercices recopiés : 

% \begin{exercice}
    % \emph{Cet exercice fournit une autre méthode de calcul du volume de la boule unité $\mathscr{B}_n$ de $\R^n$ et de l'aire de la sphère $\mathscr{S}_{n-1} \subset \R^n$.} On conserve les notations de l'exercice précédent. 
    % \begin{enumerate}
        % \item Montrer que $\mathscr{V}_n = I_n \cdot \mathscr{V}_{n-1}$, où $I_n = \int_0^\pi \sin(t)^n \d t$. 
        % \item Vérifier que $I_n = \frac{n-1}{n}I_{n-2}$.
    % \end{enumerate}
% \end{exercice}

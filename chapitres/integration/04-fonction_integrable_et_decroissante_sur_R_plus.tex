\section{Fonctions décroissantes}

\todoarmand{Pour une raison que je ne comprends pas, je n'arrive plus à compiler deux des figures.}
\todoinline{Pas de problème chez moi}

Lorsque les intégrandes sont décroissantes, la convergence de l'intégrale permet d'obtenir des informations supplémentaires sur la fonction. Nous illustrons ces propriétés à l'aide de deux exemples.

%-----------
\subsection{Décroissante, intégrabilité, rapidité de convergence}

\begin{prop}
\marginnote[0cm]{Source : \cite{exos_oraux} p. 268}
Si $f$ est une fonction continue, décroissante et intégrable sur $\R_+$, alors $\lim\limits_{x\to+\infty} x f(x) = 0$.
\end{prop}

\begin{exercice}
Soit $\fonctionens[f]{\Rp}{\R}$ une fonction continue, décroissante et intégrable.
\begin{questions}
\item Montrer que $f$ admet une limite $\ell$ en $+\infty$.

\item En raisonnant par l'absurde, montrer que $f$ est à valeurs positives.

\item En raisonnant par l'absurde, montrer que $\ell = 0$.

\item En utilisant la monotonie de $f$, montrer que pour tout $x \geqslant 0$,
\[
\int_x^{2 x} f(t) \d t \leqslant x f(x) \leqslant 2 \int_{x/2}^x f(t) \d t.
\]

\item Conclure.
\end{questions}
\end{exercice}

\begin{marginfigure}[-2cm]
    \centering
    % \begin{tikzpicture}[scale=0.5]
  
  \def\xmax{4.5}
  \def\ymax{1.5}
  \def\ymin{-2}
  \def\xzero{2}
  \def\x{5}

    \begin{axis}[
        restrict x to domain=0:8,
        samples=100, % you don't need 1000, it only slows things down
        ticks=none,
        xmin = -1, xmax = \xmax+1,
        ymin = \ymin, ymax = \ymax,
        unbounded coords=jump,
        axis x line=middle,
        axis y line=middle,
        % xlabel={$x$},
        % ylabel={$y$},
        x label style={
          at={(axis cs:5.1,0.2)},
          anchor=west,
        },
        every axis y label/.style={
          at={(axis cs:0,1.5)},
          anchor=south
        },
        legend style={
          at={(axis cs:-5.2,4)},
          anchor=west, font=\scriptsize
        },
        declare function={f(\x)=2*e^(-\x^2/2)-(\x/7)^2-1;},
        ] 
      \addplot[name path=A,very thick,color=blue, mark=none, domain=0:\xmax] {f(x)}
          node [above=8mm,near start] {$f\textcolor{black}{(x)}$};
          
      \addplot[mark=*,fill=white] coordinates {(\xzero,{f(\xzero)})};
      \draw[dashed] (axis cs:0,{f(\xzero)}) node[left=1mm] (l) {$f(x_0)$} -| (axis cs:2,0) node[above] (a) {$x_0$}; 
      \draw (axis cs:\x-0.5,0) node[above] {$x$};

    \path [name path=B] (0,0)--(\xmax, 0);
    \addplot [blue!20!white] fill between [of=A and B, soft clip={domain=\xzero:\x}];

    \fill[red!30!white, pattern=north east lines] (\xzero,0) rectangle (\x-0.5,f(\xzero);
    
    \end{axis}

    \draw[->, black, thick] (5, 4) node[above] {$f(x_0) (x - x_0)$} to [out=-90,in=90] ($(4.5,2.6)$);

    \draw[->, black, thick] (4, 1) node[left] {\color{blue}$\displaystyle \int_{x_0}^x f(t)\, \mathrm{d} t$} to [out=0,in=-90] ($(4.5,1.6)$);
    
  \end{tikzpicture}
    \caption{ébauche}
\end{marginfigure}

\begin{elemsolution}
\begin{reponses}
\item Comme $f$ est décroissante, d'après le théorème de la limite monotone, il existe $\ell \in \R \cup \ens{-\infty}$ tel que $\lim\limits_{x\to+\infty} f(x) = \ell$.

\item Montrons par l'absurde que $f$ est positive. S'il existe $x_0 \geqslant 0$ tel que $f(x_0) < 0$, comme $f$ est décroissante,
\[
\forall x \geqslant x_0,\quad f(x) \leqslant f(x_0).
\]
Ainsi,
\[
\forall x \geqslant x_0,\quad \displaystyle\int_{x_0}^x f(t) \d t \leqslant f(x_0) (x - x_0).
\]
D'après le théorème d'encadrement, $\lim\limits_{x\to+\infty} \displaystyle\int_{x_0}^x f(t) \d t = -\infty$, ce qui est impossible car $f$ est intégrable. Ainsi, $f$ est à valeurs positives et $\ell \geqslant 0$.

\item Supposons par l'absurde que $\ell > 0$. Alors, il existe un réel $A$ tel que
\[
\forall x \geqslant A,\quad f(x) \geqslant \frac{\ell}{2}.
\]
Ainsi,
\[
\forall x \geqslant A,\quad \displaystyle\int_A^x f(t) \d t \geqslant \frac{\ell}{2} (x - A).
\]
D'après le théorème d'encadrement, $\lim\limits_{x\to+\infty} \displaystyle\int_A^x f(t) \mathrm{d}t = +\infty$, ce qui est impossible car $f$ est intégrable.

Finalement, $\lim\limits_{x\to+\infty} f(x) = 0$.

\item Soit $x \geqslant 0$. Comme $f$ est décroissante,
\[
\int_x^{2 x} f(t) \d t \leqslant (2 x - x) f(x).
\]

De même,
\[
\int_{x/2}^x f(t) \d t \geqslant \frac{x}{2} f(x).
\]

\begin{marginfigure}
    \centering
    % \begin{tikzpicture}

  % https://copyprogramming.com/howto/how-do-i-draw-arrows-at-coordinate-on-a-plot
  
  \def\xmax{5.5}
  \def\ymax{1.2}
  \def\ymin{-0.2}
  \def\x{2.5}
  \def\xzero{\x/2}

  \def\colfonc{green!70!black}

    \begin{axis}[
        restrict x to domain=0:10,
        samples=100, % you don't need 1000, it only slows things down
        ticks=none,
        xmin = -1, xmax = \xmax+1,
        ymin = \ymin, ymax = \ymax,
        unbounded coords=jump,
        axis x line=middle,
        axis y line=middle,
        axis line style={-latex},
        % xlabel={$x$},
        % ylabel={$y$},
        x label style={
          at={(axis cs:5.1,0.2)},
          anchor=west,
        },
        every axis y label/.style={
          at={(axis cs:0,1.5)},
          anchor=south
        },
        legend style={
          at={(axis cs:-5.2,4)},
          anchor=west, font=\scriptsize
        },
        % ,declare function={f(\x)=e^(-\x^2/8);},
        ] 

\pgfmathdeclarefunction{f}{1}{%
\pgfmathparse{e^(-#1^2/8)}%
}
      \addplot[name path=A,very thick,color=\colfonc, mark=none, domain=0:\xmax] {f(x)} node [above=2mm,near start] {$f$};
          
      % \addplot[mark=*,fill=white] coordinates {(\xzero,{f(\xzero)})};
      % \draw[dashed] (axis cs:0,{f(\xzero)}) node[left=1mm] (l) {$f(x_0)$} -| (axis cs:2,0) node[above] (a) {$x_0$}; 
      \draw (\xzero,0) node[below] {$x/2$};

      \draw[dashed] (axis cs:0,{f(\x)}) node[left=1mm] (l) {$f(x)$} -| (axis cs:\x,0) node[below] (a) {$\vphantom{x/2}x$};
      
      % \draw (\x,0) node[below] {$x$};
      \draw (2*\x,0) node[below] {$\vphantom{x/2}2x$};

    \path [name path=B] (0,0)--(\xmax, 0);
    \addplot [blue!20!white] fill between [of=A and B, soft clip={domain=\xzero:\x}];

    \addplot [red!20!white] fill between [of=A and B, soft clip={domain=\x:2*\x}];

    \fill[pattern color=blue, pattern=north east lines] (\xzero,0) rectangle (\x,f(\x);
    \fill[pattern color=red, pattern=north east lines] (\x,0) rectangle (2*\x,f(\x);

    \end{axis}

    % \draw[->, black, thick] (5, 4) node[above] {$f(x_0) (x - x_0)$} to [out=-90,in=90] ($(4.5,2.6)$);

    % \draw[->, black, thick] (4, 1) node[left] {\color{blue}$\displaystyle \int_{x_0}^x f(t)\, \mathrm{d} t$} to [out=0,in=-90] ($(4.5,1.6)$);
    
\end{tikzpicture}
    \caption{ébauche}
\end{marginfigure}

\todoinline{Je valide le dessin ici. On peut enlever ce todo à mon avis.}

Ainsi,
\[
\int_x^{2 x} f(t) \d t \leqslant x f(x) \leqslant 2 \int_{x/2}^x f(t) \d t.
\]
En notant $\fonctionligne[F]{x}{\displaystyle\int_0^x f(t) \d t}$, comme $f$ est intégrable, alors $F$ possède une limite finie en $+\infty$.

Ainsi,
\[
F(2 x) - F(x) \leqslant x f(x) \leqslant 2 (F(x) - F(x/2)).
\]

Ainsi, d'après le théorème d'encadrement, $\lim\limits_{x\to 0} x f(x) = 0$.
\end{reponses}
\end{elemsolution}

% \todoarmand{Mettre un lien vers \url{http://ddmaths.free.fr/section115.html} ou pas car c'est très classique}


%-----------
\subsection{Décroissance, intégrabilité, sommes de Riemann}

Le résultat suivant constitue une généralisation du théorème de convergence des sommes de \nom{Riemann} pour des fonctions intégrables sur un intervalle ouvert.

\begin{prop}
    Soit $\fonctionligne[f]{\interof{0}{1}}{\R}$ une fonction continue, décroissante et intégrable. Alors,
    \[
    \lim\limits_{n \to +\infty} \frac{1}{n} \sum\limits_{k=1}^n f\mathopen{}\left(\frac{k}{n}\right) =\int_0^1 f(t) \d t.
    \]
\end{prop}

\begin{marginfigure}
    \centering
    %% Creator: Matplotlib, PGF backend
%%
%% To include the figure in your LaTeX document, write
%%   \input{<filename>.pgf}
%%
%% Make sure the required packages are loaded in your preamble
%%   \usepackage{pgf}
%%
%% Also ensure that all the required font packages are loaded; for instance,
%% the lmodern package is sometimes necessary when using math font.
%%   \usepackage{lmodern}
%%
%% Figures using additional raster images can only be included by \input if
%% they are in the same directory as the main LaTeX file. For loading figures
%% from other directories you can use the `import` package
%%   \usepackage{import}
%%
%% and then include the figures with
%%   \import{<path to file>}{<filename>.pgf}
%%
%% Matplotlib used the following preamble
%%   
%%   \usepackage{fontspec}
%%   \setmainfont{DejaVuSerif.ttf}[Path=\detokenize{/home/wayoff/.pyenv/versions/3.8.10/lib/python3.8/site-packages/matplotlib/mpl-data/fonts/ttf/}]
%%   \setsansfont{DejaVuSans.ttf}[Path=\detokenize{/home/wayoff/.pyenv/versions/3.8.10/lib/python3.8/site-packages/matplotlib/mpl-data/fonts/ttf/}]
%%   \setmonofont{DejaVuSansMono.ttf}[Path=\detokenize{/home/wayoff/.pyenv/versions/3.8.10/lib/python3.8/site-packages/matplotlib/mpl-data/fonts/ttf/}]
%%   \makeatletter\@ifpackageloaded{underscore}{}{\usepackage[strings]{underscore}}\makeatother
%%
\begingroup%
\makeatletter%
\begin{pgfpicture}%
\pgfpathrectangle{\pgfpointorigin}{\pgfqpoint{3.000000in}{3.000000in}}%
\pgfusepath{use as bounding box, clip}%
\begin{pgfscope}%
\pgfsetbuttcap%
\pgfsetmiterjoin%
\definecolor{currentfill}{rgb}{1.000000,1.000000,1.000000}%
\pgfsetfillcolor{currentfill}%
\pgfsetlinewidth{0.000000pt}%
\definecolor{currentstroke}{rgb}{1.000000,1.000000,1.000000}%
\pgfsetstrokecolor{currentstroke}%
\pgfsetdash{}{0pt}%
\pgfpathmoveto{\pgfqpoint{0.000000in}{0.000000in}}%
\pgfpathlineto{\pgfqpoint{3.000000in}{0.000000in}}%
\pgfpathlineto{\pgfqpoint{3.000000in}{3.000000in}}%
\pgfpathlineto{\pgfqpoint{0.000000in}{3.000000in}}%
\pgfpathlineto{\pgfqpoint{0.000000in}{0.000000in}}%
\pgfpathclose%
\pgfusepath{fill}%
\end{pgfscope}%
\begin{pgfscope}%
\pgfsetbuttcap%
\pgfsetmiterjoin%
\definecolor{currentfill}{rgb}{1.000000,1.000000,1.000000}%
\pgfsetfillcolor{currentfill}%
\pgfsetlinewidth{0.000000pt}%
\definecolor{currentstroke}{rgb}{0.000000,0.000000,0.000000}%
\pgfsetstrokecolor{currentstroke}%
\pgfsetstrokeopacity{0.000000}%
\pgfsetdash{}{0pt}%
\pgfpathmoveto{\pgfqpoint{0.580556in}{0.576079in}}%
\pgfpathlineto{\pgfqpoint{2.850000in}{0.576079in}}%
\pgfpathlineto{\pgfqpoint{2.850000in}{2.850000in}}%
\pgfpathlineto{\pgfqpoint{0.580556in}{2.850000in}}%
\pgfpathlineto{\pgfqpoint{0.580556in}{0.576079in}}%
\pgfpathclose%
\pgfusepath{fill}%
\end{pgfscope}%
\begin{pgfscope}%
\pgfpathrectangle{\pgfqpoint{0.580556in}{0.576079in}}{\pgfqpoint{2.269444in}{2.273921in}}%
\pgfusepath{clip}%
\pgfsetbuttcap%
\pgfsetmiterjoin%
\definecolor{currentfill}{rgb}{0.000000,0.501961,0.501961}%
\pgfsetfillcolor{currentfill}%
\pgfsetfillopacity{0.500000}%
\pgfsetlinewidth{1.003750pt}%
\definecolor{currentstroke}{rgb}{0.000000,0.000000,0.000000}%
\pgfsetstrokecolor{currentstroke}%
\pgfsetstrokeopacity{0.500000}%
\pgfsetdash{}{0pt}%
\pgfpathmoveto{\pgfqpoint{0.683713in}{0.576079in}}%
\pgfpathlineto{\pgfqpoint{0.890026in}{0.576079in}}%
\pgfpathlineto{\pgfqpoint{0.890026in}{2.741718in}}%
\pgfpathlineto{\pgfqpoint{0.683713in}{2.741718in}}%
\pgfpathlineto{\pgfqpoint{0.683713in}{0.576079in}}%
\pgfpathclose%
\pgfusepath{stroke,fill}%
\end{pgfscope}%
\begin{pgfscope}%
\pgfpathrectangle{\pgfqpoint{0.580556in}{0.576079in}}{\pgfqpoint{2.269444in}{2.273921in}}%
\pgfusepath{clip}%
\pgfsetbuttcap%
\pgfsetmiterjoin%
\definecolor{currentfill}{rgb}{0.000000,0.501961,0.501961}%
\pgfsetfillcolor{currentfill}%
\pgfsetfillopacity{0.500000}%
\pgfsetlinewidth{1.003750pt}%
\definecolor{currentstroke}{rgb}{0.000000,0.000000,0.000000}%
\pgfsetstrokecolor{currentstroke}%
\pgfsetstrokeopacity{0.500000}%
\pgfsetdash{}{0pt}%
\pgfpathmoveto{\pgfqpoint{0.890026in}{0.576079in}}%
\pgfpathlineto{\pgfqpoint{1.096339in}{0.576079in}}%
\pgfpathlineto{\pgfqpoint{1.096339in}{1.232032in}}%
\pgfpathlineto{\pgfqpoint{0.890026in}{1.232032in}}%
\pgfpathlineto{\pgfqpoint{0.890026in}{0.576079in}}%
\pgfpathclose%
\pgfusepath{stroke,fill}%
\end{pgfscope}%
\begin{pgfscope}%
\pgfpathrectangle{\pgfqpoint{0.580556in}{0.576079in}}{\pgfqpoint{2.269444in}{2.273921in}}%
\pgfusepath{clip}%
\pgfsetbuttcap%
\pgfsetmiterjoin%
\definecolor{currentfill}{rgb}{0.000000,0.501961,0.501961}%
\pgfsetfillcolor{currentfill}%
\pgfsetfillopacity{0.500000}%
\pgfsetlinewidth{1.003750pt}%
\definecolor{currentstroke}{rgb}{0.000000,0.000000,0.000000}%
\pgfsetstrokecolor{currentstroke}%
\pgfsetstrokeopacity{0.500000}%
\pgfsetdash{}{0pt}%
\pgfpathmoveto{\pgfqpoint{1.096339in}{0.576079in}}%
\pgfpathlineto{\pgfqpoint{1.302652in}{0.576079in}}%
\pgfpathlineto{\pgfqpoint{1.302652in}{1.050927in}}%
\pgfpathlineto{\pgfqpoint{1.096339in}{1.050927in}}%
\pgfpathlineto{\pgfqpoint{1.096339in}{0.576079in}}%
\pgfpathclose%
\pgfusepath{stroke,fill}%
\end{pgfscope}%
\begin{pgfscope}%
\pgfpathrectangle{\pgfqpoint{0.580556in}{0.576079in}}{\pgfqpoint{2.269444in}{2.273921in}}%
\pgfusepath{clip}%
\pgfsetbuttcap%
\pgfsetmiterjoin%
\definecolor{currentfill}{rgb}{0.000000,0.501961,0.501961}%
\pgfsetfillcolor{currentfill}%
\pgfsetfillopacity{0.500000}%
\pgfsetlinewidth{1.003750pt}%
\definecolor{currentstroke}{rgb}{0.000000,0.000000,0.000000}%
\pgfsetstrokecolor{currentstroke}%
\pgfsetstrokeopacity{0.500000}%
\pgfsetdash{}{0pt}%
\pgfpathmoveto{\pgfqpoint{1.302652in}{0.576079in}}%
\pgfpathlineto{\pgfqpoint{1.508965in}{0.576079in}}%
\pgfpathlineto{\pgfqpoint{1.508965in}{0.966935in}}%
\pgfpathlineto{\pgfqpoint{1.302652in}{0.966935in}}%
\pgfpathlineto{\pgfqpoint{1.302652in}{0.576079in}}%
\pgfpathclose%
\pgfusepath{stroke,fill}%
\end{pgfscope}%
\begin{pgfscope}%
\pgfpathrectangle{\pgfqpoint{0.580556in}{0.576079in}}{\pgfqpoint{2.269444in}{2.273921in}}%
\pgfusepath{clip}%
\pgfsetbuttcap%
\pgfsetmiterjoin%
\definecolor{currentfill}{rgb}{0.000000,0.501961,0.501961}%
\pgfsetfillcolor{currentfill}%
\pgfsetfillopacity{0.500000}%
\pgfsetlinewidth{1.003750pt}%
\definecolor{currentstroke}{rgb}{0.000000,0.000000,0.000000}%
\pgfsetstrokecolor{currentstroke}%
\pgfsetstrokeopacity{0.500000}%
\pgfsetdash{}{0pt}%
\pgfpathmoveto{\pgfqpoint{1.508965in}{0.576079in}}%
\pgfpathlineto{\pgfqpoint{1.715278in}{0.576079in}}%
\pgfpathlineto{\pgfqpoint{1.715278in}{0.915957in}}%
\pgfpathlineto{\pgfqpoint{1.508965in}{0.915957in}}%
\pgfpathlineto{\pgfqpoint{1.508965in}{0.576079in}}%
\pgfpathclose%
\pgfusepath{stroke,fill}%
\end{pgfscope}%
\begin{pgfscope}%
\pgfpathrectangle{\pgfqpoint{0.580556in}{0.576079in}}{\pgfqpoint{2.269444in}{2.273921in}}%
\pgfusepath{clip}%
\pgfsetbuttcap%
\pgfsetmiterjoin%
\definecolor{currentfill}{rgb}{0.000000,0.501961,0.501961}%
\pgfsetfillcolor{currentfill}%
\pgfsetfillopacity{0.500000}%
\pgfsetlinewidth{1.003750pt}%
\definecolor{currentstroke}{rgb}{0.000000,0.000000,0.000000}%
\pgfsetstrokecolor{currentstroke}%
\pgfsetstrokeopacity{0.500000}%
\pgfsetdash{}{0pt}%
\pgfpathmoveto{\pgfqpoint{1.715278in}{0.576079in}}%
\pgfpathlineto{\pgfqpoint{1.921591in}{0.576079in}}%
\pgfpathlineto{\pgfqpoint{1.921591in}{0.880827in}}%
\pgfpathlineto{\pgfqpoint{1.715278in}{0.880827in}}%
\pgfpathlineto{\pgfqpoint{1.715278in}{0.576079in}}%
\pgfpathclose%
\pgfusepath{stroke,fill}%
\end{pgfscope}%
\begin{pgfscope}%
\pgfpathrectangle{\pgfqpoint{0.580556in}{0.576079in}}{\pgfqpoint{2.269444in}{2.273921in}}%
\pgfusepath{clip}%
\pgfsetbuttcap%
\pgfsetmiterjoin%
\definecolor{currentfill}{rgb}{0.000000,0.501961,0.501961}%
\pgfsetfillcolor{currentfill}%
\pgfsetfillopacity{0.500000}%
\pgfsetlinewidth{1.003750pt}%
\definecolor{currentstroke}{rgb}{0.000000,0.000000,0.000000}%
\pgfsetstrokecolor{currentstroke}%
\pgfsetstrokeopacity{0.500000}%
\pgfsetdash{}{0pt}%
\pgfpathmoveto{\pgfqpoint{1.921591in}{0.576079in}}%
\pgfpathlineto{\pgfqpoint{2.127904in}{0.576079in}}%
\pgfpathlineto{\pgfqpoint{2.127904in}{0.854735in}}%
\pgfpathlineto{\pgfqpoint{1.921591in}{0.854735in}}%
\pgfpathlineto{\pgfqpoint{1.921591in}{0.576079in}}%
\pgfpathclose%
\pgfusepath{stroke,fill}%
\end{pgfscope}%
\begin{pgfscope}%
\pgfpathrectangle{\pgfqpoint{0.580556in}{0.576079in}}{\pgfqpoint{2.269444in}{2.273921in}}%
\pgfusepath{clip}%
\pgfsetbuttcap%
\pgfsetmiterjoin%
\definecolor{currentfill}{rgb}{0.000000,0.501961,0.501961}%
\pgfsetfillcolor{currentfill}%
\pgfsetfillopacity{0.500000}%
\pgfsetlinewidth{1.003750pt}%
\definecolor{currentstroke}{rgb}{0.000000,0.000000,0.000000}%
\pgfsetstrokecolor{currentstroke}%
\pgfsetstrokeopacity{0.500000}%
\pgfsetdash{}{0pt}%
\pgfpathmoveto{\pgfqpoint{2.127904in}{0.576079in}}%
\pgfpathlineto{\pgfqpoint{2.334217in}{0.576079in}}%
\pgfpathlineto{\pgfqpoint{2.334217in}{0.834370in}}%
\pgfpathlineto{\pgfqpoint{2.127904in}{0.834370in}}%
\pgfpathlineto{\pgfqpoint{2.127904in}{0.576079in}}%
\pgfpathclose%
\pgfusepath{stroke,fill}%
\end{pgfscope}%
\begin{pgfscope}%
\pgfpathrectangle{\pgfqpoint{0.580556in}{0.576079in}}{\pgfqpoint{2.269444in}{2.273921in}}%
\pgfusepath{clip}%
\pgfsetbuttcap%
\pgfsetmiterjoin%
\definecolor{currentfill}{rgb}{0.000000,0.501961,0.501961}%
\pgfsetfillcolor{currentfill}%
\pgfsetfillopacity{0.500000}%
\pgfsetlinewidth{1.003750pt}%
\definecolor{currentstroke}{rgb}{0.000000,0.000000,0.000000}%
\pgfsetstrokecolor{currentstroke}%
\pgfsetstrokeopacity{0.500000}%
\pgfsetdash{}{0pt}%
\pgfpathmoveto{\pgfqpoint{2.334217in}{0.576079in}}%
\pgfpathlineto{\pgfqpoint{2.540530in}{0.576079in}}%
\pgfpathlineto{\pgfqpoint{2.540530in}{0.817903in}}%
\pgfpathlineto{\pgfqpoint{2.334217in}{0.817903in}}%
\pgfpathlineto{\pgfqpoint{2.334217in}{0.576079in}}%
\pgfpathclose%
\pgfusepath{stroke,fill}%
\end{pgfscope}%
\begin{pgfscope}%
\pgfpathrectangle{\pgfqpoint{0.580556in}{0.576079in}}{\pgfqpoint{2.269444in}{2.273921in}}%
\pgfusepath{clip}%
\pgfsetbuttcap%
\pgfsetmiterjoin%
\definecolor{currentfill}{rgb}{0.000000,0.501961,0.501961}%
\pgfsetfillcolor{currentfill}%
\pgfsetfillopacity{0.500000}%
\pgfsetlinewidth{1.003750pt}%
\definecolor{currentstroke}{rgb}{0.000000,0.000000,0.000000}%
\pgfsetstrokecolor{currentstroke}%
\pgfsetstrokeopacity{0.500000}%
\pgfsetdash{}{0pt}%
\pgfpathmoveto{\pgfqpoint{2.540530in}{0.576079in}}%
\pgfpathlineto{\pgfqpoint{2.746843in}{0.576079in}}%
\pgfpathlineto{\pgfqpoint{2.746843in}{0.804231in}}%
\pgfpathlineto{\pgfqpoint{2.540530in}{0.804231in}}%
\pgfpathlineto{\pgfqpoint{2.540530in}{0.576079in}}%
\pgfpathclose%
\pgfusepath{stroke,fill}%
\end{pgfscope}%
\begin{pgfscope}%
\pgfpathrectangle{\pgfqpoint{0.580556in}{0.576079in}}{\pgfqpoint{2.269444in}{2.273921in}}%
\pgfusepath{clip}%
\pgfsetrectcap%
\pgfsetroundjoin%
\pgfsetlinewidth{0.803000pt}%
\definecolor{currentstroke}{rgb}{0.690196,0.690196,0.690196}%
\pgfsetstrokecolor{currentstroke}%
\pgfsetdash{}{0pt}%
\pgfpathmoveto{\pgfqpoint{0.662873in}{0.576079in}}%
\pgfpathlineto{\pgfqpoint{0.662873in}{2.850000in}}%
\pgfusepath{stroke}%
\end{pgfscope}%
\begin{pgfscope}%
\pgfsetbuttcap%
\pgfsetroundjoin%
\definecolor{currentfill}{rgb}{0.000000,0.000000,0.000000}%
\pgfsetfillcolor{currentfill}%
\pgfsetlinewidth{0.803000pt}%
\definecolor{currentstroke}{rgb}{0.000000,0.000000,0.000000}%
\pgfsetstrokecolor{currentstroke}%
\pgfsetdash{}{0pt}%
\pgfsys@defobject{currentmarker}{\pgfqpoint{0.000000in}{-0.048611in}}{\pgfqpoint{0.000000in}{0.000000in}}{%
\pgfpathmoveto{\pgfqpoint{0.000000in}{0.000000in}}%
\pgfpathlineto{\pgfqpoint{0.000000in}{-0.048611in}}%
\pgfusepath{stroke,fill}%
}%
\begin{pgfscope}%
\pgfsys@transformshift{0.662873in}{0.576079in}%
\pgfsys@useobject{currentmarker}{}%
\end{pgfscope}%
\end{pgfscope}%
\begin{pgfscope}%
\definecolor{textcolor}{rgb}{0.000000,0.000000,0.000000}%
\pgfsetstrokecolor{textcolor}%
\pgfsetfillcolor{textcolor}%
\pgftext[x=0.662873in,y=0.478857in,,top]{\color{textcolor}\sffamily\fontsize{10.000000}{12.000000}\selectfont \(\displaystyle 0\)}%
\end{pgfscope}%
\begin{pgfscope}%
\pgfpathrectangle{\pgfqpoint{0.580556in}{0.576079in}}{\pgfqpoint{2.269444in}{2.273921in}}%
\pgfusepath{clip}%
\pgfsetrectcap%
\pgfsetroundjoin%
\pgfsetlinewidth{0.803000pt}%
\definecolor{currentstroke}{rgb}{0.690196,0.690196,0.690196}%
\pgfsetstrokecolor{currentstroke}%
\pgfsetdash{}{0pt}%
\pgfpathmoveto{\pgfqpoint{1.183865in}{0.576079in}}%
\pgfpathlineto{\pgfqpoint{1.183865in}{2.850000in}}%
\pgfusepath{stroke}%
\end{pgfscope}%
\begin{pgfscope}%
\pgfsetbuttcap%
\pgfsetroundjoin%
\definecolor{currentfill}{rgb}{0.000000,0.000000,0.000000}%
\pgfsetfillcolor{currentfill}%
\pgfsetlinewidth{0.803000pt}%
\definecolor{currentstroke}{rgb}{0.000000,0.000000,0.000000}%
\pgfsetstrokecolor{currentstroke}%
\pgfsetdash{}{0pt}%
\pgfsys@defobject{currentmarker}{\pgfqpoint{0.000000in}{-0.048611in}}{\pgfqpoint{0.000000in}{0.000000in}}{%
\pgfpathmoveto{\pgfqpoint{0.000000in}{0.000000in}}%
\pgfpathlineto{\pgfqpoint{0.000000in}{-0.048611in}}%
\pgfusepath{stroke,fill}%
}%
\begin{pgfscope}%
\pgfsys@transformshift{1.183865in}{0.576079in}%
\pgfsys@useobject{currentmarker}{}%
\end{pgfscope}%
\end{pgfscope}%
\begin{pgfscope}%
\definecolor{textcolor}{rgb}{0.000000,0.000000,0.000000}%
\pgfsetstrokecolor{textcolor}%
\pgfsetfillcolor{textcolor}%
\pgftext[x=1.183865in,y=0.478857in,,top]{\color{textcolor}\sffamily\fontsize{10.000000}{12.000000}\selectfont \(\displaystyle 1/4\)}%
\end{pgfscope}%
\begin{pgfscope}%
\pgfpathrectangle{\pgfqpoint{0.580556in}{0.576079in}}{\pgfqpoint{2.269444in}{2.273921in}}%
\pgfusepath{clip}%
\pgfsetrectcap%
\pgfsetroundjoin%
\pgfsetlinewidth{0.803000pt}%
\definecolor{currentstroke}{rgb}{0.690196,0.690196,0.690196}%
\pgfsetstrokecolor{currentstroke}%
\pgfsetdash{}{0pt}%
\pgfpathmoveto{\pgfqpoint{1.704858in}{0.576079in}}%
\pgfpathlineto{\pgfqpoint{1.704858in}{2.850000in}}%
\pgfusepath{stroke}%
\end{pgfscope}%
\begin{pgfscope}%
\pgfsetbuttcap%
\pgfsetroundjoin%
\definecolor{currentfill}{rgb}{0.000000,0.000000,0.000000}%
\pgfsetfillcolor{currentfill}%
\pgfsetlinewidth{0.803000pt}%
\definecolor{currentstroke}{rgb}{0.000000,0.000000,0.000000}%
\pgfsetstrokecolor{currentstroke}%
\pgfsetdash{}{0pt}%
\pgfsys@defobject{currentmarker}{\pgfqpoint{0.000000in}{-0.048611in}}{\pgfqpoint{0.000000in}{0.000000in}}{%
\pgfpathmoveto{\pgfqpoint{0.000000in}{0.000000in}}%
\pgfpathlineto{\pgfqpoint{0.000000in}{-0.048611in}}%
\pgfusepath{stroke,fill}%
}%
\begin{pgfscope}%
\pgfsys@transformshift{1.704858in}{0.576079in}%
\pgfsys@useobject{currentmarker}{}%
\end{pgfscope}%
\end{pgfscope}%
\begin{pgfscope}%
\definecolor{textcolor}{rgb}{0.000000,0.000000,0.000000}%
\pgfsetstrokecolor{textcolor}%
\pgfsetfillcolor{textcolor}%
\pgftext[x=1.704858in,y=0.478857in,,top]{\color{textcolor}\sffamily\fontsize{10.000000}{12.000000}\selectfont \(\displaystyle 1/2\)}%
\end{pgfscope}%
\begin{pgfscope}%
\pgfpathrectangle{\pgfqpoint{0.580556in}{0.576079in}}{\pgfqpoint{2.269444in}{2.273921in}}%
\pgfusepath{clip}%
\pgfsetrectcap%
\pgfsetroundjoin%
\pgfsetlinewidth{0.803000pt}%
\definecolor{currentstroke}{rgb}{0.690196,0.690196,0.690196}%
\pgfsetstrokecolor{currentstroke}%
\pgfsetdash{}{0pt}%
\pgfpathmoveto{\pgfqpoint{2.225851in}{0.576079in}}%
\pgfpathlineto{\pgfqpoint{2.225851in}{2.850000in}}%
\pgfusepath{stroke}%
\end{pgfscope}%
\begin{pgfscope}%
\pgfsetbuttcap%
\pgfsetroundjoin%
\definecolor{currentfill}{rgb}{0.000000,0.000000,0.000000}%
\pgfsetfillcolor{currentfill}%
\pgfsetlinewidth{0.803000pt}%
\definecolor{currentstroke}{rgb}{0.000000,0.000000,0.000000}%
\pgfsetstrokecolor{currentstroke}%
\pgfsetdash{}{0pt}%
\pgfsys@defobject{currentmarker}{\pgfqpoint{0.000000in}{-0.048611in}}{\pgfqpoint{0.000000in}{0.000000in}}{%
\pgfpathmoveto{\pgfqpoint{0.000000in}{0.000000in}}%
\pgfpathlineto{\pgfqpoint{0.000000in}{-0.048611in}}%
\pgfusepath{stroke,fill}%
}%
\begin{pgfscope}%
\pgfsys@transformshift{2.225851in}{0.576079in}%
\pgfsys@useobject{currentmarker}{}%
\end{pgfscope}%
\end{pgfscope}%
\begin{pgfscope}%
\definecolor{textcolor}{rgb}{0.000000,0.000000,0.000000}%
\pgfsetstrokecolor{textcolor}%
\pgfsetfillcolor{textcolor}%
\pgftext[x=2.225851in,y=0.478857in,,top]{\color{textcolor}\sffamily\fontsize{10.000000}{12.000000}\selectfont \(\displaystyle 3/4\)}%
\end{pgfscope}%
\begin{pgfscope}%
\pgfpathrectangle{\pgfqpoint{0.580556in}{0.576079in}}{\pgfqpoint{2.269444in}{2.273921in}}%
\pgfusepath{clip}%
\pgfsetrectcap%
\pgfsetroundjoin%
\pgfsetlinewidth{0.803000pt}%
\definecolor{currentstroke}{rgb}{0.690196,0.690196,0.690196}%
\pgfsetstrokecolor{currentstroke}%
\pgfsetdash{}{0pt}%
\pgfpathmoveto{\pgfqpoint{2.746843in}{0.576079in}}%
\pgfpathlineto{\pgfqpoint{2.746843in}{2.850000in}}%
\pgfusepath{stroke}%
\end{pgfscope}%
\begin{pgfscope}%
\pgfsetbuttcap%
\pgfsetroundjoin%
\definecolor{currentfill}{rgb}{0.000000,0.000000,0.000000}%
\pgfsetfillcolor{currentfill}%
\pgfsetlinewidth{0.803000pt}%
\definecolor{currentstroke}{rgb}{0.000000,0.000000,0.000000}%
\pgfsetstrokecolor{currentstroke}%
\pgfsetdash{}{0pt}%
\pgfsys@defobject{currentmarker}{\pgfqpoint{0.000000in}{-0.048611in}}{\pgfqpoint{0.000000in}{0.000000in}}{%
\pgfpathmoveto{\pgfqpoint{0.000000in}{0.000000in}}%
\pgfpathlineto{\pgfqpoint{0.000000in}{-0.048611in}}%
\pgfusepath{stroke,fill}%
}%
\begin{pgfscope}%
\pgfsys@transformshift{2.746843in}{0.576079in}%
\pgfsys@useobject{currentmarker}{}%
\end{pgfscope}%
\end{pgfscope}%
\begin{pgfscope}%
\definecolor{textcolor}{rgb}{0.000000,0.000000,0.000000}%
\pgfsetstrokecolor{textcolor}%
\pgfsetfillcolor{textcolor}%
\pgftext[x=2.746843in,y=0.478857in,,top]{\color{textcolor}\sffamily\fontsize{10.000000}{12.000000}\selectfont \(\displaystyle 1\)}%
\end{pgfscope}%
\begin{pgfscope}%
\definecolor{textcolor}{rgb}{0.000000,0.000000,0.000000}%
\pgfsetstrokecolor{textcolor}%
\pgfsetfillcolor{textcolor}%
\pgftext[x=1.715278in,y=0.284413in,,top]{\color{textcolor}\sffamily\fontsize{10.000000}{12.000000}\selectfont \(\displaystyle x\)}%
\end{pgfscope}%
\begin{pgfscope}%
\pgfpathrectangle{\pgfqpoint{0.580556in}{0.576079in}}{\pgfqpoint{2.269444in}{2.273921in}}%
\pgfusepath{clip}%
\pgfsetrectcap%
\pgfsetroundjoin%
\pgfsetlinewidth{0.803000pt}%
\definecolor{currentstroke}{rgb}{0.690196,0.690196,0.690196}%
\pgfsetstrokecolor{currentstroke}%
\pgfsetdash{}{0pt}%
\pgfpathmoveto{\pgfqpoint{0.580556in}{0.576079in}}%
\pgfpathlineto{\pgfqpoint{2.850000in}{0.576079in}}%
\pgfusepath{stroke}%
\end{pgfscope}%
\begin{pgfscope}%
\pgfsetbuttcap%
\pgfsetroundjoin%
\definecolor{currentfill}{rgb}{0.000000,0.000000,0.000000}%
\pgfsetfillcolor{currentfill}%
\pgfsetlinewidth{0.803000pt}%
\definecolor{currentstroke}{rgb}{0.000000,0.000000,0.000000}%
\pgfsetstrokecolor{currentstroke}%
\pgfsetdash{}{0pt}%
\pgfsys@defobject{currentmarker}{\pgfqpoint{-0.048611in}{0.000000in}}{\pgfqpoint{-0.000000in}{0.000000in}}{%
\pgfpathmoveto{\pgfqpoint{-0.000000in}{0.000000in}}%
\pgfpathlineto{\pgfqpoint{-0.048611in}{0.000000in}}%
\pgfusepath{stroke,fill}%
}%
\begin{pgfscope}%
\pgfsys@transformshift{0.580556in}{0.576079in}%
\pgfsys@useobject{currentmarker}{}%
\end{pgfscope}%
\end{pgfscope}%
\begin{pgfscope}%
\definecolor{textcolor}{rgb}{0.000000,0.000000,0.000000}%
\pgfsetstrokecolor{textcolor}%
\pgfsetfillcolor{textcolor}%
\pgftext[x=0.413889in, y=0.523318in, left, base]{\color{textcolor}\sffamily\fontsize{10.000000}{12.000000}\selectfont \(\displaystyle 0\)}%
\end{pgfscope}%
\begin{pgfscope}%
\pgfpathrectangle{\pgfqpoint{0.580556in}{0.576079in}}{\pgfqpoint{2.269444in}{2.273921in}}%
\pgfusepath{clip}%
\pgfsetrectcap%
\pgfsetroundjoin%
\pgfsetlinewidth{0.803000pt}%
\definecolor{currentstroke}{rgb}{0.690196,0.690196,0.690196}%
\pgfsetstrokecolor{currentstroke}%
\pgfsetdash{}{0pt}%
\pgfpathmoveto{\pgfqpoint{0.580556in}{1.658899in}}%
\pgfpathlineto{\pgfqpoint{2.850000in}{1.658899in}}%
\pgfusepath{stroke}%
\end{pgfscope}%
\begin{pgfscope}%
\pgfsetbuttcap%
\pgfsetroundjoin%
\definecolor{currentfill}{rgb}{0.000000,0.000000,0.000000}%
\pgfsetfillcolor{currentfill}%
\pgfsetlinewidth{0.803000pt}%
\definecolor{currentstroke}{rgb}{0.000000,0.000000,0.000000}%
\pgfsetstrokecolor{currentstroke}%
\pgfsetdash{}{0pt}%
\pgfsys@defobject{currentmarker}{\pgfqpoint{-0.048611in}{0.000000in}}{\pgfqpoint{-0.000000in}{0.000000in}}{%
\pgfpathmoveto{\pgfqpoint{-0.000000in}{0.000000in}}%
\pgfpathlineto{\pgfqpoint{-0.048611in}{0.000000in}}%
\pgfusepath{stroke,fill}%
}%
\begin{pgfscope}%
\pgfsys@transformshift{0.580556in}{1.658899in}%
\pgfsys@useobject{currentmarker}{}%
\end{pgfscope}%
\end{pgfscope}%
\begin{pgfscope}%
\definecolor{textcolor}{rgb}{0.000000,0.000000,0.000000}%
\pgfsetstrokecolor{textcolor}%
\pgfsetfillcolor{textcolor}%
\pgftext[x=0.413889in, y=1.606137in, left, base]{\color{textcolor}\sffamily\fontsize{10.000000}{12.000000}\selectfont \(\displaystyle 5\)}%
\end{pgfscope}%
\begin{pgfscope}%
\pgfpathrectangle{\pgfqpoint{0.580556in}{0.576079in}}{\pgfqpoint{2.269444in}{2.273921in}}%
\pgfusepath{clip}%
\pgfsetrectcap%
\pgfsetroundjoin%
\pgfsetlinewidth{0.803000pt}%
\definecolor{currentstroke}{rgb}{0.690196,0.690196,0.690196}%
\pgfsetstrokecolor{currentstroke}%
\pgfsetdash{}{0pt}%
\pgfpathmoveto{\pgfqpoint{0.580556in}{2.741718in}}%
\pgfpathlineto{\pgfqpoint{2.850000in}{2.741718in}}%
\pgfusepath{stroke}%
\end{pgfscope}%
\begin{pgfscope}%
\pgfsetbuttcap%
\pgfsetroundjoin%
\definecolor{currentfill}{rgb}{0.000000,0.000000,0.000000}%
\pgfsetfillcolor{currentfill}%
\pgfsetlinewidth{0.803000pt}%
\definecolor{currentstroke}{rgb}{0.000000,0.000000,0.000000}%
\pgfsetstrokecolor{currentstroke}%
\pgfsetdash{}{0pt}%
\pgfsys@defobject{currentmarker}{\pgfqpoint{-0.048611in}{0.000000in}}{\pgfqpoint{-0.000000in}{0.000000in}}{%
\pgfpathmoveto{\pgfqpoint{-0.000000in}{0.000000in}}%
\pgfpathlineto{\pgfqpoint{-0.048611in}{0.000000in}}%
\pgfusepath{stroke,fill}%
}%
\begin{pgfscope}%
\pgfsys@transformshift{0.580556in}{2.741718in}%
\pgfsys@useobject{currentmarker}{}%
\end{pgfscope}%
\end{pgfscope}%
\begin{pgfscope}%
\definecolor{textcolor}{rgb}{0.000000,0.000000,0.000000}%
\pgfsetstrokecolor{textcolor}%
\pgfsetfillcolor{textcolor}%
\pgftext[x=0.344444in, y=2.688957in, left, base]{\color{textcolor}\sffamily\fontsize{10.000000}{12.000000}\selectfont \(\displaystyle 10\)}%
\end{pgfscope}%
\begin{pgfscope}%
\definecolor{textcolor}{rgb}{0.000000,0.000000,0.000000}%
\pgfsetstrokecolor{textcolor}%
\pgfsetfillcolor{textcolor}%
\pgftext[x=0.288889in,y=1.713040in,,bottom,rotate=90.000000]{\color{textcolor}\sffamily\fontsize{10.000000}{12.000000}\selectfont \(\displaystyle f(x)\)}%
\end{pgfscope}%
\begin{pgfscope}%
\pgfpathrectangle{\pgfqpoint{0.580556in}{0.576079in}}{\pgfqpoint{2.269444in}{2.273921in}}%
\pgfusepath{clip}%
\pgfsetrectcap%
\pgfsetroundjoin%
\pgfsetlinewidth{1.505625pt}%
\definecolor{currentstroke}{rgb}{0.000000,0.000000,1.000000}%
\pgfsetstrokecolor{currentstroke}%
\pgfsetdash{}{0pt}%
\pgfpathmoveto{\pgfqpoint{0.683713in}{2.741718in}}%
\pgfpathlineto{\pgfqpoint{0.689908in}{2.477448in}}%
\pgfpathlineto{\pgfqpoint{0.696104in}{2.291066in}}%
\pgfpathlineto{\pgfqpoint{0.704364in}{2.110878in}}%
\pgfpathlineto{\pgfqpoint{0.712625in}{1.977681in}}%
\pgfpathlineto{\pgfqpoint{0.722951in}{1.851556in}}%
\pgfpathlineto{\pgfqpoint{0.733277in}{1.754314in}}%
\pgfpathlineto{\pgfqpoint{0.743603in}{1.676385in}}%
\pgfpathlineto{\pgfqpoint{0.755994in}{1.600567in}}%
\pgfpathlineto{\pgfqpoint{0.768386in}{1.538532in}}%
\pgfpathlineto{\pgfqpoint{0.780777in}{1.486554in}}%
\pgfpathlineto{\pgfqpoint{0.795233in}{1.435395in}}%
\pgfpathlineto{\pgfqpoint{0.809689in}{1.391993in}}%
\pgfpathlineto{\pgfqpoint{0.826211in}{1.349628in}}%
\pgfpathlineto{\pgfqpoint{0.842733in}{1.313244in}}%
\pgfpathlineto{\pgfqpoint{0.861319in}{1.277874in}}%
\pgfpathlineto{\pgfqpoint{0.879906in}{1.247151in}}%
\pgfpathlineto{\pgfqpoint{0.900558in}{1.217334in}}%
\pgfpathlineto{\pgfqpoint{0.923275in}{1.188725in}}%
\pgfpathlineto{\pgfqpoint{0.945992in}{1.163632in}}%
\pgfpathlineto{\pgfqpoint{0.970775in}{1.139491in}}%
\pgfpathlineto{\pgfqpoint{0.997622in}{1.116425in}}%
\pgfpathlineto{\pgfqpoint{1.026535in}{1.094501in}}%
\pgfpathlineto{\pgfqpoint{1.057513in}{1.073738in}}%
\pgfpathlineto{\pgfqpoint{1.092621in}{1.052976in}}%
\pgfpathlineto{\pgfqpoint{1.129795in}{1.033599in}}%
\pgfpathlineto{\pgfqpoint{1.171099in}{1.014614in}}%
\pgfpathlineto{\pgfqpoint{1.216533in}{0.996235in}}%
\pgfpathlineto{\pgfqpoint{1.266098in}{0.978604in}}%
\pgfpathlineto{\pgfqpoint{1.321858in}{0.961198in}}%
\pgfpathlineto{\pgfqpoint{1.383814in}{0.944278in}}%
\pgfpathlineto{\pgfqpoint{1.451965in}{0.928019in}}%
\pgfpathlineto{\pgfqpoint{1.528378in}{0.912124in}}%
\pgfpathlineto{\pgfqpoint{1.613051in}{0.896802in}}%
\pgfpathlineto{\pgfqpoint{1.708050in}{0.881879in}}%
\pgfpathlineto{\pgfqpoint{1.815440in}{0.867284in}}%
\pgfpathlineto{\pgfqpoint{1.937287in}{0.853014in}}%
\pgfpathlineto{\pgfqpoint{2.075655in}{0.839103in}}%
\pgfpathlineto{\pgfqpoint{2.232610in}{0.825607in}}%
\pgfpathlineto{\pgfqpoint{2.412282in}{0.812446in}}%
\pgfpathlineto{\pgfqpoint{2.618801in}{0.799619in}}%
\pgfpathlineto{\pgfqpoint{2.746843in}{0.792643in}}%
\pgfpathlineto{\pgfqpoint{2.746843in}{0.792643in}}%
\pgfusepath{stroke}%
\end{pgfscope}%
\begin{pgfscope}%
\pgfsetrectcap%
\pgfsetmiterjoin%
\pgfsetlinewidth{0.803000pt}%
\definecolor{currentstroke}{rgb}{0.000000,0.000000,0.000000}%
\pgfsetstrokecolor{currentstroke}%
\pgfsetdash{}{0pt}%
\pgfpathmoveto{\pgfqpoint{0.580556in}{0.576079in}}%
\pgfpathlineto{\pgfqpoint{0.580556in}{2.850000in}}%
\pgfusepath{stroke}%
\end{pgfscope}%
\begin{pgfscope}%
\pgfsetrectcap%
\pgfsetmiterjoin%
\pgfsetlinewidth{0.803000pt}%
\definecolor{currentstroke}{rgb}{0.000000,0.000000,0.000000}%
\pgfsetstrokecolor{currentstroke}%
\pgfsetdash{}{0pt}%
\pgfpathmoveto{\pgfqpoint{2.850000in}{0.576079in}}%
\pgfpathlineto{\pgfqpoint{2.850000in}{2.850000in}}%
\pgfusepath{stroke}%
\end{pgfscope}%
\begin{pgfscope}%
\pgfsetrectcap%
\pgfsetmiterjoin%
\pgfsetlinewidth{0.803000pt}%
\definecolor{currentstroke}{rgb}{0.000000,0.000000,0.000000}%
\pgfsetstrokecolor{currentstroke}%
\pgfsetdash{}{0pt}%
\pgfpathmoveto{\pgfqpoint{0.580556in}{0.576079in}}%
\pgfpathlineto{\pgfqpoint{2.850000in}{0.576079in}}%
\pgfusepath{stroke}%
\end{pgfscope}%
\begin{pgfscope}%
\pgfsetrectcap%
\pgfsetmiterjoin%
\pgfsetlinewidth{0.803000pt}%
\definecolor{currentstroke}{rgb}{0.000000,0.000000,0.000000}%
\pgfsetstrokecolor{currentstroke}%
\pgfsetdash{}{0pt}%
\pgfpathmoveto{\pgfqpoint{0.580556in}{2.850000in}}%
\pgfpathlineto{\pgfqpoint{2.850000in}{2.850000in}}%
\pgfusepath{stroke}%
\end{pgfscope}%
\begin{pgfscope}%
\pgfsetbuttcap%
\pgfsetmiterjoin%
\definecolor{currentfill}{rgb}{1.000000,1.000000,1.000000}%
\pgfsetfillcolor{currentfill}%
\pgfsetfillopacity{0.800000}%
\pgfsetlinewidth{1.003750pt}%
\definecolor{currentstroke}{rgb}{0.800000,0.800000,0.800000}%
\pgfsetstrokecolor{currentstroke}%
\pgfsetstrokeopacity{0.800000}%
\pgfsetdash{}{0pt}%
\pgfpathmoveto{\pgfqpoint{1.624345in}{2.356814in}}%
\pgfpathlineto{\pgfqpoint{2.752778in}{2.356814in}}%
\pgfpathquadraticcurveto{\pgfqpoint{2.780556in}{2.356814in}}{\pgfqpoint{2.780556in}{2.384591in}}%
\pgfpathlineto{\pgfqpoint{2.780556in}{2.752778in}}%
\pgfpathquadraticcurveto{\pgfqpoint{2.780556in}{2.780556in}}{\pgfqpoint{2.752778in}{2.780556in}}%
\pgfpathlineto{\pgfqpoint{1.624345in}{2.780556in}}%
\pgfpathquadraticcurveto{\pgfqpoint{1.596567in}{2.780556in}}{\pgfqpoint{1.596567in}{2.752778in}}%
\pgfpathlineto{\pgfqpoint{1.596567in}{2.384591in}}%
\pgfpathquadraticcurveto{\pgfqpoint{1.596567in}{2.356814in}}{\pgfqpoint{1.624345in}{2.356814in}}%
\pgfpathlineto{\pgfqpoint{1.624345in}{2.356814in}}%
\pgfpathclose%
\pgfusepath{stroke,fill}%
\end{pgfscope}%
\begin{pgfscope}%
\pgfsetrectcap%
\pgfsetroundjoin%
\pgfsetlinewidth{1.505625pt}%
\definecolor{currentstroke}{rgb}{0.000000,0.000000,1.000000}%
\pgfsetstrokecolor{currentstroke}%
\pgfsetdash{}{0pt}%
\pgfpathmoveto{\pgfqpoint{1.652122in}{2.590146in}}%
\pgfpathlineto{\pgfqpoint{1.791011in}{2.590146in}}%
\pgfpathlineto{\pgfqpoint{1.929900in}{2.590146in}}%
\pgfusepath{stroke}%
\end{pgfscope}%
\begin{pgfscope}%
\definecolor{textcolor}{rgb}{0.000000,0.000000,0.000000}%
\pgfsetstrokecolor{textcolor}%
\pgfsetfillcolor{textcolor}%
\pgftext[x=2.041011in,y=2.541534in,left,base]{\color{textcolor}\sffamily\fontsize{10.000000}{12.000000}\selectfont \(\displaystyle f(x) = \frac{1}{\sqrt{x}}\)}%
\end{pgfscope}%
\end{pgfpicture}%
\makeatother%
\endgroup%

    \caption{Exemple d'une fonction continue, décroissante et intégrable sur $\interof{0}{1}$.}
\end{marginfigure}

\begin{exercice}
Soit $\fonctionligne[f]{\interof{0}{1}}{\R}$ une fonction continue, décroissante et intégrable. On pose \mbox{$R_n(f) = \frac{1}{n} \sum\limits_{k=1}^n f\mathopen{}\left(\frac{k}{n}\right)$}.
\begin{questions}
\item Montrer que
\[
\displaystyle\int_{1/n}^1 f(t) \d t + \frac{f(1)}{n}
\leqslant R_n(f)
\leqslant \displaystyle\int_{1/n}^1 f(t) \d t + \frac{1}{n}  f\mathopen{}\left(\frac{1}{n}\right).
\]

\item Montrer que $\frac{1}{n} f\mathopen{}\left(\frac{1}{n}\right) \leqslant \displaystyle\int_0^{1/n} f(t) \d t$.

\item Conclure.
\end{questions}
\end{exercice}



\begin{elemsolution}
\begin{reponses}
\item Soit $n \geqslant 1$. En utilisant la décroissance de la fonction $f$, pour tout $k \leqslant t \leqslant k + 1 \leqslant n-1$,
\begin{align*}
\frac{1}{n} f\mathopen{}\left(\frac{k+1}{n}\right) &\leqslant \int_{k/n}^{(k+1)/n} f(t) \d t \leqslant \frac{1}{n} f\mathopen{}\left(\frac{k}{n}\right),\\
\intertext{en sommant pour $k \in \interent{1}{n-1}$}
\frac{1}{n} \sum\limits_{k=2}^{n} f\mathopen{}\left(\frac{k}{n}\right) &\leqslant \displaystyle\int_{1/n}^1 f(t) \d t \leqslant \frac{1}{n} \sum\limits_{k=1}^{n-1} f\mathopen{}\left(\frac{k}{n}\right),\\
\intertext{soit}
R_n(f) - \frac{1}{n} f\mathopen{}\left(\frac{1}{n}\right) &\leqslant \displaystyle\int_{1/n}^1 f(t) \d t \leqslant R_n(f) - \frac{f(1)}{n}.
\end{align*}

On obtient ainsi l'encadrement annoncé.

\item Simple conséquence de la décroissance de $f$.

\item Comme $f$ est intégrable sur $\interof{0}{1}$,
\[
\lim\limits_{n\to+\infty} \displaystyle\int_{1/n}^1 f(t) \d t = \displaystyle\int_0^1 f(t) \d t.
\]

\item Ainsi, d'après le théorème d'encadrement,
\[
\lim\limits_{n\to+\infty} R_n(f) = \displaystyle\int_0^1 f(t) \d t.
\]
\end{reponses}
\end{elemsolution}

\begin{remarque}
Une étude plus détaillée de ce résultat et de ses limites est discutée dans~\cite{truc2019} p. 268.
\end{remarque}

%---------------

\begin{exercice}
\marginpar[0cm]{Source : \cite{rms2016-2} Ex.~888 - ENSAM}
Soit $\fonctionligne[f]{x}{\frac{1}{x \sqrt{x^2 - 1}}}$ définie sur $\interoo{1}{+\infty}$.
\begin{questions}
\item Étudier et tracer la fonction $f$.
\end{questions}
Pour tout entier naturel $n$, on pose $S_n = \sum\limits_{k=n+1}^{+\infty} \frac{1}{k \sqrt{k^2 - n^2}}$.
\begin{questions}[resume]
\item Étudier la convergence et la limite de la suite $(S_n)$.

\item Même question avec la suite $(n S_n)$.
\end{questions}
\end{exercice}

\begin{marginfigure}
    \centering
    %% Creator: Matplotlib, PGF backend
%%
%% To include the figure in your LaTeX document, write
%%   \input{<filename>.pgf}
%%
%% Make sure the required packages are loaded in your preamble
%%   \usepackage{pgf}
%%
%% Also ensure that all the required font packages are loaded; for instance,
%% the lmodern package is sometimes necessary when using math font.
%%   \usepackage{lmodern}
%%
%% Figures using additional raster images can only be included by \input if
%% they are in the same directory as the main LaTeX file. For loading figures
%% from other directories you can use the `import` package
%%   \usepackage{import}
%%
%% and then include the figures with
%%   \import{<path to file>}{<filename>.pgf}
%%
%% Matplotlib used the following preamble
%%   
%%   \usepackage{fontspec}
%%   \setmainfont{DejaVuSerif.ttf}[Path=\detokenize{/home/wayoff/.pyenv/versions/3.8.10/lib/python3.8/site-packages/matplotlib/mpl-data/fonts/ttf/}]
%%   \setsansfont{DejaVuSans.ttf}[Path=\detokenize{/home/wayoff/.pyenv/versions/3.8.10/lib/python3.8/site-packages/matplotlib/mpl-data/fonts/ttf/}]
%%   \setmonofont{DejaVuSansMono.ttf}[Path=\detokenize{/home/wayoff/.pyenv/versions/3.8.10/lib/python3.8/site-packages/matplotlib/mpl-data/fonts/ttf/}]
%%   \makeatletter\@ifpackageloaded{underscore}{}{\usepackage[strings]{underscore}}\makeatother
%%
\begingroup%
\makeatletter%
\begin{pgfpicture}%
\pgfpathrectangle{\pgfpointorigin}{\pgfqpoint{3.000000in}{3.000000in}}%
\pgfusepath{use as bounding box, clip}%
\begin{pgfscope}%
\pgfsetbuttcap%
\pgfsetmiterjoin%
\definecolor{currentfill}{rgb}{1.000000,1.000000,1.000000}%
\pgfsetfillcolor{currentfill}%
\pgfsetlinewidth{0.000000pt}%
\definecolor{currentstroke}{rgb}{1.000000,1.000000,1.000000}%
\pgfsetstrokecolor{currentstroke}%
\pgfsetdash{}{0pt}%
\pgfpathmoveto{\pgfqpoint{0.000000in}{0.000000in}}%
\pgfpathlineto{\pgfqpoint{3.000000in}{0.000000in}}%
\pgfpathlineto{\pgfqpoint{3.000000in}{3.000000in}}%
\pgfpathlineto{\pgfqpoint{0.000000in}{3.000000in}}%
\pgfpathlineto{\pgfqpoint{0.000000in}{0.000000in}}%
\pgfpathclose%
\pgfusepath{fill}%
\end{pgfscope}%
\begin{pgfscope}%
\pgfsetbuttcap%
\pgfsetmiterjoin%
\definecolor{currentfill}{rgb}{1.000000,1.000000,1.000000}%
\pgfsetfillcolor{currentfill}%
\pgfsetlinewidth{0.000000pt}%
\definecolor{currentstroke}{rgb}{0.000000,0.000000,0.000000}%
\pgfsetstrokecolor{currentstroke}%
\pgfsetstrokeopacity{0.000000}%
\pgfsetdash{}{0pt}%
\pgfpathmoveto{\pgfqpoint{0.468102in}{0.571603in}}%
\pgfpathlineto{\pgfqpoint{2.850000in}{0.571603in}}%
\pgfpathlineto{\pgfqpoint{2.850000in}{2.832283in}}%
\pgfpathlineto{\pgfqpoint{0.468102in}{2.832283in}}%
\pgfpathlineto{\pgfqpoint{0.468102in}{0.571603in}}%
\pgfpathclose%
\pgfusepath{fill}%
\end{pgfscope}%
\begin{pgfscope}%
\pgfpathrectangle{\pgfqpoint{0.468102in}{0.571603in}}{\pgfqpoint{2.381898in}{2.260680in}}%
\pgfusepath{clip}%
\pgfsetrectcap%
\pgfsetroundjoin%
\pgfsetlinewidth{0.803000pt}%
\definecolor{currentstroke}{rgb}{0.690196,0.690196,0.690196}%
\pgfsetstrokecolor{currentstroke}%
\pgfsetdash{}{0pt}%
\pgfpathmoveto{\pgfqpoint{0.552040in}{0.571603in}}%
\pgfpathlineto{\pgfqpoint{0.552040in}{2.832283in}}%
\pgfusepath{stroke}%
\end{pgfscope}%
\begin{pgfscope}%
\pgfsetbuttcap%
\pgfsetroundjoin%
\definecolor{currentfill}{rgb}{0.000000,0.000000,0.000000}%
\pgfsetfillcolor{currentfill}%
\pgfsetlinewidth{0.803000pt}%
\definecolor{currentstroke}{rgb}{0.000000,0.000000,0.000000}%
\pgfsetstrokecolor{currentstroke}%
\pgfsetdash{}{0pt}%
\pgfsys@defobject{currentmarker}{\pgfqpoint{0.000000in}{-0.048611in}}{\pgfqpoint{0.000000in}{0.000000in}}{%
\pgfpathmoveto{\pgfqpoint{0.000000in}{0.000000in}}%
\pgfpathlineto{\pgfqpoint{0.000000in}{-0.048611in}}%
\pgfusepath{stroke,fill}%
}%
\begin{pgfscope}%
\pgfsys@transformshift{0.552040in}{0.571603in}%
\pgfsys@useobject{currentmarker}{}%
\end{pgfscope}%
\end{pgfscope}%
\begin{pgfscope}%
\definecolor{textcolor}{rgb}{0.000000,0.000000,0.000000}%
\pgfsetstrokecolor{textcolor}%
\pgfsetfillcolor{textcolor}%
\pgftext[x=0.552040in,y=0.474381in,,top]{\color{textcolor}\sffamily\fontsize{10.000000}{12.000000}\selectfont 0}%
\end{pgfscope}%
\begin{pgfscope}%
\pgfpathrectangle{\pgfqpoint{0.468102in}{0.571603in}}{\pgfqpoint{2.381898in}{2.260680in}}%
\pgfusepath{clip}%
\pgfsetrectcap%
\pgfsetroundjoin%
\pgfsetlinewidth{0.803000pt}%
\definecolor{currentstroke}{rgb}{0.690196,0.690196,0.690196}%
\pgfsetstrokecolor{currentstroke}%
\pgfsetdash{}{0pt}%
\pgfpathmoveto{\pgfqpoint{1.038638in}{0.571603in}}%
\pgfpathlineto{\pgfqpoint{1.038638in}{2.832283in}}%
\pgfusepath{stroke}%
\end{pgfscope}%
\begin{pgfscope}%
\pgfsetbuttcap%
\pgfsetroundjoin%
\definecolor{currentfill}{rgb}{0.000000,0.000000,0.000000}%
\pgfsetfillcolor{currentfill}%
\pgfsetlinewidth{0.803000pt}%
\definecolor{currentstroke}{rgb}{0.000000,0.000000,0.000000}%
\pgfsetstrokecolor{currentstroke}%
\pgfsetdash{}{0pt}%
\pgfsys@defobject{currentmarker}{\pgfqpoint{0.000000in}{-0.048611in}}{\pgfqpoint{0.000000in}{0.000000in}}{%
\pgfpathmoveto{\pgfqpoint{0.000000in}{0.000000in}}%
\pgfpathlineto{\pgfqpoint{0.000000in}{-0.048611in}}%
\pgfusepath{stroke,fill}%
}%
\begin{pgfscope}%
\pgfsys@transformshift{1.038638in}{0.571603in}%
\pgfsys@useobject{currentmarker}{}%
\end{pgfscope}%
\end{pgfscope}%
\begin{pgfscope}%
\definecolor{textcolor}{rgb}{0.000000,0.000000,0.000000}%
\pgfsetstrokecolor{textcolor}%
\pgfsetfillcolor{textcolor}%
\pgftext[x=1.038638in,y=0.474381in,,top]{\color{textcolor}\sffamily\fontsize{10.000000}{12.000000}\selectfont 20}%
\end{pgfscope}%
\begin{pgfscope}%
\pgfpathrectangle{\pgfqpoint{0.468102in}{0.571603in}}{\pgfqpoint{2.381898in}{2.260680in}}%
\pgfusepath{clip}%
\pgfsetrectcap%
\pgfsetroundjoin%
\pgfsetlinewidth{0.803000pt}%
\definecolor{currentstroke}{rgb}{0.690196,0.690196,0.690196}%
\pgfsetstrokecolor{currentstroke}%
\pgfsetdash{}{0pt}%
\pgfpathmoveto{\pgfqpoint{1.525236in}{0.571603in}}%
\pgfpathlineto{\pgfqpoint{1.525236in}{2.832283in}}%
\pgfusepath{stroke}%
\end{pgfscope}%
\begin{pgfscope}%
\pgfsetbuttcap%
\pgfsetroundjoin%
\definecolor{currentfill}{rgb}{0.000000,0.000000,0.000000}%
\pgfsetfillcolor{currentfill}%
\pgfsetlinewidth{0.803000pt}%
\definecolor{currentstroke}{rgb}{0.000000,0.000000,0.000000}%
\pgfsetstrokecolor{currentstroke}%
\pgfsetdash{}{0pt}%
\pgfsys@defobject{currentmarker}{\pgfqpoint{0.000000in}{-0.048611in}}{\pgfqpoint{0.000000in}{0.000000in}}{%
\pgfpathmoveto{\pgfqpoint{0.000000in}{0.000000in}}%
\pgfpathlineto{\pgfqpoint{0.000000in}{-0.048611in}}%
\pgfusepath{stroke,fill}%
}%
\begin{pgfscope}%
\pgfsys@transformshift{1.525236in}{0.571603in}%
\pgfsys@useobject{currentmarker}{}%
\end{pgfscope}%
\end{pgfscope}%
\begin{pgfscope}%
\definecolor{textcolor}{rgb}{0.000000,0.000000,0.000000}%
\pgfsetstrokecolor{textcolor}%
\pgfsetfillcolor{textcolor}%
\pgftext[x=1.525236in,y=0.474381in,,top]{\color{textcolor}\sffamily\fontsize{10.000000}{12.000000}\selectfont 40}%
\end{pgfscope}%
\begin{pgfscope}%
\pgfpathrectangle{\pgfqpoint{0.468102in}{0.571603in}}{\pgfqpoint{2.381898in}{2.260680in}}%
\pgfusepath{clip}%
\pgfsetrectcap%
\pgfsetroundjoin%
\pgfsetlinewidth{0.803000pt}%
\definecolor{currentstroke}{rgb}{0.690196,0.690196,0.690196}%
\pgfsetstrokecolor{currentstroke}%
\pgfsetdash{}{0pt}%
\pgfpathmoveto{\pgfqpoint{2.011835in}{0.571603in}}%
\pgfpathlineto{\pgfqpoint{2.011835in}{2.832283in}}%
\pgfusepath{stroke}%
\end{pgfscope}%
\begin{pgfscope}%
\pgfsetbuttcap%
\pgfsetroundjoin%
\definecolor{currentfill}{rgb}{0.000000,0.000000,0.000000}%
\pgfsetfillcolor{currentfill}%
\pgfsetlinewidth{0.803000pt}%
\definecolor{currentstroke}{rgb}{0.000000,0.000000,0.000000}%
\pgfsetstrokecolor{currentstroke}%
\pgfsetdash{}{0pt}%
\pgfsys@defobject{currentmarker}{\pgfqpoint{0.000000in}{-0.048611in}}{\pgfqpoint{0.000000in}{0.000000in}}{%
\pgfpathmoveto{\pgfqpoint{0.000000in}{0.000000in}}%
\pgfpathlineto{\pgfqpoint{0.000000in}{-0.048611in}}%
\pgfusepath{stroke,fill}%
}%
\begin{pgfscope}%
\pgfsys@transformshift{2.011835in}{0.571603in}%
\pgfsys@useobject{currentmarker}{}%
\end{pgfscope}%
\end{pgfscope}%
\begin{pgfscope}%
\definecolor{textcolor}{rgb}{0.000000,0.000000,0.000000}%
\pgfsetstrokecolor{textcolor}%
\pgfsetfillcolor{textcolor}%
\pgftext[x=2.011835in,y=0.474381in,,top]{\color{textcolor}\sffamily\fontsize{10.000000}{12.000000}\selectfont 60}%
\end{pgfscope}%
\begin{pgfscope}%
\pgfpathrectangle{\pgfqpoint{0.468102in}{0.571603in}}{\pgfqpoint{2.381898in}{2.260680in}}%
\pgfusepath{clip}%
\pgfsetrectcap%
\pgfsetroundjoin%
\pgfsetlinewidth{0.803000pt}%
\definecolor{currentstroke}{rgb}{0.690196,0.690196,0.690196}%
\pgfsetstrokecolor{currentstroke}%
\pgfsetdash{}{0pt}%
\pgfpathmoveto{\pgfqpoint{2.498433in}{0.571603in}}%
\pgfpathlineto{\pgfqpoint{2.498433in}{2.832283in}}%
\pgfusepath{stroke}%
\end{pgfscope}%
\begin{pgfscope}%
\pgfsetbuttcap%
\pgfsetroundjoin%
\definecolor{currentfill}{rgb}{0.000000,0.000000,0.000000}%
\pgfsetfillcolor{currentfill}%
\pgfsetlinewidth{0.803000pt}%
\definecolor{currentstroke}{rgb}{0.000000,0.000000,0.000000}%
\pgfsetstrokecolor{currentstroke}%
\pgfsetdash{}{0pt}%
\pgfsys@defobject{currentmarker}{\pgfqpoint{0.000000in}{-0.048611in}}{\pgfqpoint{0.000000in}{0.000000in}}{%
\pgfpathmoveto{\pgfqpoint{0.000000in}{0.000000in}}%
\pgfpathlineto{\pgfqpoint{0.000000in}{-0.048611in}}%
\pgfusepath{stroke,fill}%
}%
\begin{pgfscope}%
\pgfsys@transformshift{2.498433in}{0.571603in}%
\pgfsys@useobject{currentmarker}{}%
\end{pgfscope}%
\end{pgfscope}%
\begin{pgfscope}%
\definecolor{textcolor}{rgb}{0.000000,0.000000,0.000000}%
\pgfsetstrokecolor{textcolor}%
\pgfsetfillcolor{textcolor}%
\pgftext[x=2.498433in,y=0.474381in,,top]{\color{textcolor}\sffamily\fontsize{10.000000}{12.000000}\selectfont 80}%
\end{pgfscope}%
\begin{pgfscope}%
\definecolor{textcolor}{rgb}{0.000000,0.000000,0.000000}%
\pgfsetstrokecolor{textcolor}%
\pgfsetfillcolor{textcolor}%
\pgftext[x=1.659051in,y=0.284413in,,top]{\color{textcolor}\sffamily\fontsize{10.000000}{12.000000}\selectfont \(\displaystyle n\)}%
\end{pgfscope}%
\begin{pgfscope}%
\pgfpathrectangle{\pgfqpoint{0.468102in}{0.571603in}}{\pgfqpoint{2.381898in}{2.260680in}}%
\pgfusepath{clip}%
\pgfsetrectcap%
\pgfsetroundjoin%
\pgfsetlinewidth{0.803000pt}%
\definecolor{currentstroke}{rgb}{0.690196,0.690196,0.690196}%
\pgfsetstrokecolor{currentstroke}%
\pgfsetdash{}{0pt}%
\pgfpathmoveto{\pgfqpoint{0.468102in}{0.922564in}}%
\pgfpathlineto{\pgfqpoint{2.850000in}{0.922564in}}%
\pgfusepath{stroke}%
\end{pgfscope}%
\begin{pgfscope}%
\pgfsetbuttcap%
\pgfsetroundjoin%
\definecolor{currentfill}{rgb}{0.000000,0.000000,0.000000}%
\pgfsetfillcolor{currentfill}%
\pgfsetlinewidth{0.803000pt}%
\definecolor{currentstroke}{rgb}{0.000000,0.000000,0.000000}%
\pgfsetstrokecolor{currentstroke}%
\pgfsetdash{}{0pt}%
\pgfsys@defobject{currentmarker}{\pgfqpoint{-0.048611in}{0.000000in}}{\pgfqpoint{-0.000000in}{0.000000in}}{%
\pgfpathmoveto{\pgfqpoint{-0.000000in}{0.000000in}}%
\pgfpathlineto{\pgfqpoint{-0.048611in}{0.000000in}}%
\pgfusepath{stroke,fill}%
}%
\begin{pgfscope}%
\pgfsys@transformshift{0.468102in}{0.922564in}%
\pgfsys@useobject{currentmarker}{}%
\end{pgfscope}%
\end{pgfscope}%
\begin{pgfscope}%
\definecolor{textcolor}{rgb}{0.000000,0.000000,0.000000}%
\pgfsetstrokecolor{textcolor}%
\pgfsetfillcolor{textcolor}%
\pgftext[x=0.150000in, y=0.869803in, left, base]{\color{textcolor}\sffamily\fontsize{10.000000}{12.000000}\selectfont 0.8}%
\end{pgfscope}%
\begin{pgfscope}%
\pgfpathrectangle{\pgfqpoint{0.468102in}{0.571603in}}{\pgfqpoint{2.381898in}{2.260680in}}%
\pgfusepath{clip}%
\pgfsetrectcap%
\pgfsetroundjoin%
\pgfsetlinewidth{0.803000pt}%
\definecolor{currentstroke}{rgb}{0.690196,0.690196,0.690196}%
\pgfsetstrokecolor{currentstroke}%
\pgfsetdash{}{0pt}%
\pgfpathmoveto{\pgfqpoint{0.468102in}{1.391420in}}%
\pgfpathlineto{\pgfqpoint{2.850000in}{1.391420in}}%
\pgfusepath{stroke}%
\end{pgfscope}%
\begin{pgfscope}%
\pgfsetbuttcap%
\pgfsetroundjoin%
\definecolor{currentfill}{rgb}{0.000000,0.000000,0.000000}%
\pgfsetfillcolor{currentfill}%
\pgfsetlinewidth{0.803000pt}%
\definecolor{currentstroke}{rgb}{0.000000,0.000000,0.000000}%
\pgfsetstrokecolor{currentstroke}%
\pgfsetdash{}{0pt}%
\pgfsys@defobject{currentmarker}{\pgfqpoint{-0.048611in}{0.000000in}}{\pgfqpoint{-0.000000in}{0.000000in}}{%
\pgfpathmoveto{\pgfqpoint{-0.000000in}{0.000000in}}%
\pgfpathlineto{\pgfqpoint{-0.048611in}{0.000000in}}%
\pgfusepath{stroke,fill}%
}%
\begin{pgfscope}%
\pgfsys@transformshift{0.468102in}{1.391420in}%
\pgfsys@useobject{currentmarker}{}%
\end{pgfscope}%
\end{pgfscope}%
\begin{pgfscope}%
\definecolor{textcolor}{rgb}{0.000000,0.000000,0.000000}%
\pgfsetstrokecolor{textcolor}%
\pgfsetfillcolor{textcolor}%
\pgftext[x=0.150000in, y=1.338658in, left, base]{\color{textcolor}\sffamily\fontsize{10.000000}{12.000000}\selectfont 1.0}%
\end{pgfscope}%
\begin{pgfscope}%
\pgfpathrectangle{\pgfqpoint{0.468102in}{0.571603in}}{\pgfqpoint{2.381898in}{2.260680in}}%
\pgfusepath{clip}%
\pgfsetrectcap%
\pgfsetroundjoin%
\pgfsetlinewidth{0.803000pt}%
\definecolor{currentstroke}{rgb}{0.690196,0.690196,0.690196}%
\pgfsetstrokecolor{currentstroke}%
\pgfsetdash{}{0pt}%
\pgfpathmoveto{\pgfqpoint{0.468102in}{1.860275in}}%
\pgfpathlineto{\pgfqpoint{2.850000in}{1.860275in}}%
\pgfusepath{stroke}%
\end{pgfscope}%
\begin{pgfscope}%
\pgfsetbuttcap%
\pgfsetroundjoin%
\definecolor{currentfill}{rgb}{0.000000,0.000000,0.000000}%
\pgfsetfillcolor{currentfill}%
\pgfsetlinewidth{0.803000pt}%
\definecolor{currentstroke}{rgb}{0.000000,0.000000,0.000000}%
\pgfsetstrokecolor{currentstroke}%
\pgfsetdash{}{0pt}%
\pgfsys@defobject{currentmarker}{\pgfqpoint{-0.048611in}{0.000000in}}{\pgfqpoint{-0.000000in}{0.000000in}}{%
\pgfpathmoveto{\pgfqpoint{-0.000000in}{0.000000in}}%
\pgfpathlineto{\pgfqpoint{-0.048611in}{0.000000in}}%
\pgfusepath{stroke,fill}%
}%
\begin{pgfscope}%
\pgfsys@transformshift{0.468102in}{1.860275in}%
\pgfsys@useobject{currentmarker}{}%
\end{pgfscope}%
\end{pgfscope}%
\begin{pgfscope}%
\definecolor{textcolor}{rgb}{0.000000,0.000000,0.000000}%
\pgfsetstrokecolor{textcolor}%
\pgfsetfillcolor{textcolor}%
\pgftext[x=0.150000in, y=1.807514in, left, base]{\color{textcolor}\sffamily\fontsize{10.000000}{12.000000}\selectfont 1.2}%
\end{pgfscope}%
\begin{pgfscope}%
\pgfpathrectangle{\pgfqpoint{0.468102in}{0.571603in}}{\pgfqpoint{2.381898in}{2.260680in}}%
\pgfusepath{clip}%
\pgfsetrectcap%
\pgfsetroundjoin%
\pgfsetlinewidth{0.803000pt}%
\definecolor{currentstroke}{rgb}{0.690196,0.690196,0.690196}%
\pgfsetstrokecolor{currentstroke}%
\pgfsetdash{}{0pt}%
\pgfpathmoveto{\pgfqpoint{0.468102in}{2.329131in}}%
\pgfpathlineto{\pgfqpoint{2.850000in}{2.329131in}}%
\pgfusepath{stroke}%
\end{pgfscope}%
\begin{pgfscope}%
\pgfsetbuttcap%
\pgfsetroundjoin%
\definecolor{currentfill}{rgb}{0.000000,0.000000,0.000000}%
\pgfsetfillcolor{currentfill}%
\pgfsetlinewidth{0.803000pt}%
\definecolor{currentstroke}{rgb}{0.000000,0.000000,0.000000}%
\pgfsetstrokecolor{currentstroke}%
\pgfsetdash{}{0pt}%
\pgfsys@defobject{currentmarker}{\pgfqpoint{-0.048611in}{0.000000in}}{\pgfqpoint{-0.000000in}{0.000000in}}{%
\pgfpathmoveto{\pgfqpoint{-0.000000in}{0.000000in}}%
\pgfpathlineto{\pgfqpoint{-0.048611in}{0.000000in}}%
\pgfusepath{stroke,fill}%
}%
\begin{pgfscope}%
\pgfsys@transformshift{0.468102in}{2.329131in}%
\pgfsys@useobject{currentmarker}{}%
\end{pgfscope}%
\end{pgfscope}%
\begin{pgfscope}%
\definecolor{textcolor}{rgb}{0.000000,0.000000,0.000000}%
\pgfsetstrokecolor{textcolor}%
\pgfsetfillcolor{textcolor}%
\pgftext[x=0.150000in, y=2.276370in, left, base]{\color{textcolor}\sffamily\fontsize{10.000000}{12.000000}\selectfont 1.4}%
\end{pgfscope}%
\begin{pgfscope}%
\pgfpathrectangle{\pgfqpoint{0.468102in}{0.571603in}}{\pgfqpoint{2.381898in}{2.260680in}}%
\pgfusepath{clip}%
\pgfsetrectcap%
\pgfsetroundjoin%
\pgfsetlinewidth{0.803000pt}%
\definecolor{currentstroke}{rgb}{0.690196,0.690196,0.690196}%
\pgfsetstrokecolor{currentstroke}%
\pgfsetdash{}{0pt}%
\pgfpathmoveto{\pgfqpoint{0.468102in}{2.797987in}}%
\pgfpathlineto{\pgfqpoint{2.850000in}{2.797987in}}%
\pgfusepath{stroke}%
\end{pgfscope}%
\begin{pgfscope}%
\pgfsetbuttcap%
\pgfsetroundjoin%
\definecolor{currentfill}{rgb}{0.000000,0.000000,0.000000}%
\pgfsetfillcolor{currentfill}%
\pgfsetlinewidth{0.803000pt}%
\definecolor{currentstroke}{rgb}{0.000000,0.000000,0.000000}%
\pgfsetstrokecolor{currentstroke}%
\pgfsetdash{}{0pt}%
\pgfsys@defobject{currentmarker}{\pgfqpoint{-0.048611in}{0.000000in}}{\pgfqpoint{-0.000000in}{0.000000in}}{%
\pgfpathmoveto{\pgfqpoint{-0.000000in}{0.000000in}}%
\pgfpathlineto{\pgfqpoint{-0.048611in}{0.000000in}}%
\pgfusepath{stroke,fill}%
}%
\begin{pgfscope}%
\pgfsys@transformshift{0.468102in}{2.797987in}%
\pgfsys@useobject{currentmarker}{}%
\end{pgfscope}%
\end{pgfscope}%
\begin{pgfscope}%
\definecolor{textcolor}{rgb}{0.000000,0.000000,0.000000}%
\pgfsetstrokecolor{textcolor}%
\pgfsetfillcolor{textcolor}%
\pgftext[x=0.150000in, y=2.745225in, left, base]{\color{textcolor}\sffamily\fontsize{10.000000}{12.000000}\selectfont 1.6}%
\end{pgfscope}%
\begin{pgfscope}%
\pgfpathrectangle{\pgfqpoint{0.468102in}{0.571603in}}{\pgfqpoint{2.381898in}{2.260680in}}%
\pgfusepath{clip}%
\pgfsetrectcap%
\pgfsetroundjoin%
\pgfsetlinewidth{1.505625pt}%
\definecolor{currentstroke}{rgb}{0.121569,0.466667,0.705882}%
\pgfsetstrokecolor{currentstroke}%
\pgfsetdash{}{0pt}%
\pgfpathmoveto{\pgfqpoint{0.576370in}{0.674362in}}%
\pgfpathlineto{\pgfqpoint{0.625030in}{1.410118in}}%
\pgfpathlineto{\pgfqpoint{0.698019in}{1.768478in}}%
\pgfpathlineto{\pgfqpoint{0.795339in}{1.975105in}}%
\pgfpathlineto{\pgfqpoint{1.038638in}{2.188321in}}%
\pgfpathlineto{\pgfqpoint{1.281937in}{2.283134in}}%
\pgfpathlineto{\pgfqpoint{1.525236in}{2.339089in}}%
\pgfpathlineto{\pgfqpoint{1.768535in}{2.376672in}}%
\pgfpathlineto{\pgfqpoint{2.011835in}{2.403866in}}%
\pgfpathlineto{\pgfqpoint{2.255134in}{2.424513in}}%
\pgfpathlineto{\pgfqpoint{2.498433in}{2.440723in}}%
\pgfpathlineto{\pgfqpoint{2.741732in}{2.453760in}}%
\pgfusepath{stroke}%
\end{pgfscope}%
\begin{pgfscope}%
\pgfpathrectangle{\pgfqpoint{0.468102in}{0.571603in}}{\pgfqpoint{2.381898in}{2.260680in}}%
\pgfusepath{clip}%
\pgfsetbuttcap%
\pgfsetroundjoin%
\definecolor{currentfill}{rgb}{0.121569,0.466667,0.705882}%
\pgfsetfillcolor{currentfill}%
\pgfsetlinewidth{1.003750pt}%
\definecolor{currentstroke}{rgb}{0.121569,0.466667,0.705882}%
\pgfsetstrokecolor{currentstroke}%
\pgfsetdash{}{0pt}%
\pgfsys@defobject{currentmarker}{\pgfqpoint{-0.041667in}{-0.041667in}}{\pgfqpoint{0.041667in}{0.041667in}}{%
\pgfpathmoveto{\pgfqpoint{0.000000in}{-0.041667in}}%
\pgfpathcurveto{\pgfqpoint{0.011050in}{-0.041667in}}{\pgfqpoint{0.021649in}{-0.037276in}}{\pgfqpoint{0.029463in}{-0.029463in}}%
\pgfpathcurveto{\pgfqpoint{0.037276in}{-0.021649in}}{\pgfqpoint{0.041667in}{-0.011050in}}{\pgfqpoint{0.041667in}{0.000000in}}%
\pgfpathcurveto{\pgfqpoint{0.041667in}{0.011050in}}{\pgfqpoint{0.037276in}{0.021649in}}{\pgfqpoint{0.029463in}{0.029463in}}%
\pgfpathcurveto{\pgfqpoint{0.021649in}{0.037276in}}{\pgfqpoint{0.011050in}{0.041667in}}{\pgfqpoint{0.000000in}{0.041667in}}%
\pgfpathcurveto{\pgfqpoint{-0.011050in}{0.041667in}}{\pgfqpoint{-0.021649in}{0.037276in}}{\pgfqpoint{-0.029463in}{0.029463in}}%
\pgfpathcurveto{\pgfqpoint{-0.037276in}{0.021649in}}{\pgfqpoint{-0.041667in}{0.011050in}}{\pgfqpoint{-0.041667in}{0.000000in}}%
\pgfpathcurveto{\pgfqpoint{-0.041667in}{-0.011050in}}{\pgfqpoint{-0.037276in}{-0.021649in}}{\pgfqpoint{-0.029463in}{-0.029463in}}%
\pgfpathcurveto{\pgfqpoint{-0.021649in}{-0.037276in}}{\pgfqpoint{-0.011050in}{-0.041667in}}{\pgfqpoint{0.000000in}{-0.041667in}}%
\pgfpathlineto{\pgfqpoint{0.000000in}{-0.041667in}}%
\pgfpathclose%
\pgfusepath{stroke,fill}%
}%
\begin{pgfscope}%
\pgfsys@transformshift{0.576370in}{0.674362in}%
\pgfsys@useobject{currentmarker}{}%
\end{pgfscope}%
\begin{pgfscope}%
\pgfsys@transformshift{0.625030in}{1.410118in}%
\pgfsys@useobject{currentmarker}{}%
\end{pgfscope}%
\begin{pgfscope}%
\pgfsys@transformshift{0.698019in}{1.768478in}%
\pgfsys@useobject{currentmarker}{}%
\end{pgfscope}%
\begin{pgfscope}%
\pgfsys@transformshift{0.795339in}{1.975105in}%
\pgfsys@useobject{currentmarker}{}%
\end{pgfscope}%
\begin{pgfscope}%
\pgfsys@transformshift{1.038638in}{2.188321in}%
\pgfsys@useobject{currentmarker}{}%
\end{pgfscope}%
\begin{pgfscope}%
\pgfsys@transformshift{1.281937in}{2.283134in}%
\pgfsys@useobject{currentmarker}{}%
\end{pgfscope}%
\begin{pgfscope}%
\pgfsys@transformshift{1.525236in}{2.339089in}%
\pgfsys@useobject{currentmarker}{}%
\end{pgfscope}%
\begin{pgfscope}%
\pgfsys@transformshift{1.768535in}{2.376672in}%
\pgfsys@useobject{currentmarker}{}%
\end{pgfscope}%
\begin{pgfscope}%
\pgfsys@transformshift{2.011835in}{2.403866in}%
\pgfsys@useobject{currentmarker}{}%
\end{pgfscope}%
\begin{pgfscope}%
\pgfsys@transformshift{2.255134in}{2.424513in}%
\pgfsys@useobject{currentmarker}{}%
\end{pgfscope}%
\begin{pgfscope}%
\pgfsys@transformshift{2.498433in}{2.440723in}%
\pgfsys@useobject{currentmarker}{}%
\end{pgfscope}%
\begin{pgfscope}%
\pgfsys@transformshift{2.741732in}{2.453760in}%
\pgfsys@useobject{currentmarker}{}%
\end{pgfscope}%
\end{pgfscope}%
\begin{pgfscope}%
\pgfpathrectangle{\pgfqpoint{0.468102in}{0.571603in}}{\pgfqpoint{2.381898in}{2.260680in}}%
\pgfusepath{clip}%
\pgfsetbuttcap%
\pgfsetroundjoin%
\pgfsetlinewidth{1.505625pt}%
\definecolor{currentstroke}{rgb}{1.000000,0.000000,0.000000}%
\pgfsetstrokecolor{currentstroke}%
\pgfsetdash{{5.550000pt}{2.400000pt}}{0.000000pt}%
\pgfpathmoveto{\pgfqpoint{0.468102in}{2.729525in}}%
\pgfpathlineto{\pgfqpoint{2.850000in}{2.729525in}}%
\pgfusepath{stroke}%
\end{pgfscope}%
\begin{pgfscope}%
\pgfsetrectcap%
\pgfsetmiterjoin%
\pgfsetlinewidth{0.803000pt}%
\definecolor{currentstroke}{rgb}{0.000000,0.000000,0.000000}%
\pgfsetstrokecolor{currentstroke}%
\pgfsetdash{}{0pt}%
\pgfpathmoveto{\pgfqpoint{0.468102in}{0.571603in}}%
\pgfpathlineto{\pgfqpoint{0.468102in}{2.832283in}}%
\pgfusepath{stroke}%
\end{pgfscope}%
\begin{pgfscope}%
\pgfsetrectcap%
\pgfsetmiterjoin%
\pgfsetlinewidth{0.803000pt}%
\definecolor{currentstroke}{rgb}{0.000000,0.000000,0.000000}%
\pgfsetstrokecolor{currentstroke}%
\pgfsetdash{}{0pt}%
\pgfpathmoveto{\pgfqpoint{2.850000in}{0.571603in}}%
\pgfpathlineto{\pgfqpoint{2.850000in}{2.832283in}}%
\pgfusepath{stroke}%
\end{pgfscope}%
\begin{pgfscope}%
\pgfsetrectcap%
\pgfsetmiterjoin%
\pgfsetlinewidth{0.803000pt}%
\definecolor{currentstroke}{rgb}{0.000000,0.000000,0.000000}%
\pgfsetstrokecolor{currentstroke}%
\pgfsetdash{}{0pt}%
\pgfpathmoveto{\pgfqpoint{0.468102in}{0.571603in}}%
\pgfpathlineto{\pgfqpoint{2.850000in}{0.571603in}}%
\pgfusepath{stroke}%
\end{pgfscope}%
\begin{pgfscope}%
\pgfsetrectcap%
\pgfsetmiterjoin%
\pgfsetlinewidth{0.803000pt}%
\definecolor{currentstroke}{rgb}{0.000000,0.000000,0.000000}%
\pgfsetstrokecolor{currentstroke}%
\pgfsetdash{}{0pt}%
\pgfpathmoveto{\pgfqpoint{0.468102in}{2.832283in}}%
\pgfpathlineto{\pgfqpoint{2.850000in}{2.832283in}}%
\pgfusepath{stroke}%
\end{pgfscope}%
\begin{pgfscope}%
\pgfsetbuttcap%
\pgfsetmiterjoin%
\definecolor{currentfill}{rgb}{1.000000,1.000000,1.000000}%
\pgfsetfillcolor{currentfill}%
\pgfsetfillopacity{0.800000}%
\pgfsetlinewidth{1.003750pt}%
\definecolor{currentstroke}{rgb}{0.800000,0.800000,0.800000}%
\pgfsetstrokecolor{currentstroke}%
\pgfsetstrokeopacity{0.800000}%
\pgfsetdash{}{0pt}%
\pgfpathmoveto{\pgfqpoint{1.540027in}{0.641048in}}%
\pgfpathlineto{\pgfqpoint{2.752778in}{0.641048in}}%
\pgfpathquadraticcurveto{\pgfqpoint{2.780556in}{0.641048in}}{\pgfqpoint{2.780556in}{0.668826in}}%
\pgfpathlineto{\pgfqpoint{2.780556in}{1.269677in}}%
\pgfpathquadraticcurveto{\pgfqpoint{2.780556in}{1.297455in}}{\pgfqpoint{2.752778in}{1.297455in}}%
\pgfpathlineto{\pgfqpoint{1.540027in}{1.297455in}}%
\pgfpathquadraticcurveto{\pgfqpoint{1.512249in}{1.297455in}}{\pgfqpoint{1.512249in}{1.269677in}}%
\pgfpathlineto{\pgfqpoint{1.512249in}{0.668826in}}%
\pgfpathquadraticcurveto{\pgfqpoint{1.512249in}{0.641048in}}{\pgfqpoint{1.540027in}{0.641048in}}%
\pgfpathlineto{\pgfqpoint{1.540027in}{0.641048in}}%
\pgfpathclose%
\pgfusepath{stroke,fill}%
\end{pgfscope}%
\begin{pgfscope}%
\pgfsetrectcap%
\pgfsetroundjoin%
\pgfsetlinewidth{1.505625pt}%
\definecolor{currentstroke}{rgb}{0.121569,0.466667,0.705882}%
\pgfsetstrokecolor{currentstroke}%
\pgfsetdash{}{0pt}%
\pgfpathmoveto{\pgfqpoint{1.567804in}{1.184987in}}%
\pgfpathlineto{\pgfqpoint{1.706693in}{1.184987in}}%
\pgfpathlineto{\pgfqpoint{1.845582in}{1.184987in}}%
\pgfusepath{stroke}%
\end{pgfscope}%
\begin{pgfscope}%
\pgfsetbuttcap%
\pgfsetroundjoin%
\definecolor{currentfill}{rgb}{0.121569,0.466667,0.705882}%
\pgfsetfillcolor{currentfill}%
\pgfsetlinewidth{1.003750pt}%
\definecolor{currentstroke}{rgb}{0.121569,0.466667,0.705882}%
\pgfsetstrokecolor{currentstroke}%
\pgfsetdash{}{0pt}%
\pgfsys@defobject{currentmarker}{\pgfqpoint{-0.041667in}{-0.041667in}}{\pgfqpoint{0.041667in}{0.041667in}}{%
\pgfpathmoveto{\pgfqpoint{0.000000in}{-0.041667in}}%
\pgfpathcurveto{\pgfqpoint{0.011050in}{-0.041667in}}{\pgfqpoint{0.021649in}{-0.037276in}}{\pgfqpoint{0.029463in}{-0.029463in}}%
\pgfpathcurveto{\pgfqpoint{0.037276in}{-0.021649in}}{\pgfqpoint{0.041667in}{-0.011050in}}{\pgfqpoint{0.041667in}{0.000000in}}%
\pgfpathcurveto{\pgfqpoint{0.041667in}{0.011050in}}{\pgfqpoint{0.037276in}{0.021649in}}{\pgfqpoint{0.029463in}{0.029463in}}%
\pgfpathcurveto{\pgfqpoint{0.021649in}{0.037276in}}{\pgfqpoint{0.011050in}{0.041667in}}{\pgfqpoint{0.000000in}{0.041667in}}%
\pgfpathcurveto{\pgfqpoint{-0.011050in}{0.041667in}}{\pgfqpoint{-0.021649in}{0.037276in}}{\pgfqpoint{-0.029463in}{0.029463in}}%
\pgfpathcurveto{\pgfqpoint{-0.037276in}{0.021649in}}{\pgfqpoint{-0.041667in}{0.011050in}}{\pgfqpoint{-0.041667in}{0.000000in}}%
\pgfpathcurveto{\pgfqpoint{-0.041667in}{-0.011050in}}{\pgfqpoint{-0.037276in}{-0.021649in}}{\pgfqpoint{-0.029463in}{-0.029463in}}%
\pgfpathcurveto{\pgfqpoint{-0.021649in}{-0.037276in}}{\pgfqpoint{-0.011050in}{-0.041667in}}{\pgfqpoint{0.000000in}{-0.041667in}}%
\pgfpathlineto{\pgfqpoint{0.000000in}{-0.041667in}}%
\pgfpathclose%
\pgfusepath{stroke,fill}%
}%
\begin{pgfscope}%
\pgfsys@transformshift{1.706693in}{1.184987in}%
\pgfsys@useobject{currentmarker}{}%
\end{pgfscope}%
\end{pgfscope}%
\begin{pgfscope}%
\definecolor{textcolor}{rgb}{0.000000,0.000000,0.000000}%
\pgfsetstrokecolor{textcolor}%
\pgfsetfillcolor{textcolor}%
\pgftext[x=1.956693in,y=1.136376in,left,base]{\color{textcolor}\sffamily\fontsize{10.000000}{12.000000}\selectfont \(\displaystyle n S_n\)}%
\end{pgfscope}%
\begin{pgfscope}%
\pgfsetbuttcap%
\pgfsetroundjoin%
\pgfsetlinewidth{1.505625pt}%
\definecolor{currentstroke}{rgb}{1.000000,0.000000,0.000000}%
\pgfsetstrokecolor{currentstroke}%
\pgfsetdash{{5.550000pt}{2.400000pt}}{0.000000pt}%
\pgfpathmoveto{\pgfqpoint{1.567804in}{0.871759in}}%
\pgfpathlineto{\pgfqpoint{1.706693in}{0.871759in}}%
\pgfpathlineto{\pgfqpoint{1.845582in}{0.871759in}}%
\pgfusepath{stroke}%
\end{pgfscope}%
\begin{pgfscope}%
\definecolor{textcolor}{rgb}{0.000000,0.000000,0.000000}%
\pgfsetstrokecolor{textcolor}%
\pgfsetfillcolor{textcolor}%
\pgftext[x=1.956693in,y=0.823148in,left,base]{\color{textcolor}\sffamily\fontsize{10.000000}{12.000000}\selectfont \(\displaystyle \int_{1}^{+\infty} f(t) \, \mathrm{d}t\)}%
\end{pgfscope}%
\end{pgfpicture}%
\makeatother%
\endgroup%

    \caption{La convergence est très lente, donc je ne suis pas sûr que cette illustration soit très convaincante}
\end{marginfigure}

\todoinline{Oui, ça ne converge pas vite. En fait, je m'étais mal exprimé et je pense que $n S_n$ est la somme d'aires de rectangles qu'on peut comparer à l'intégrale sous la courbe de $f$. C'est un peu la même idée que pour la prop précédente mais ici on est sur $[1, +\infty[$ au lieu de $]0, 1]$.}


\begin{elemsolution}
\begin{reponses}
\item La fonction $f$ est décroisante, à valeurs positives, $\lim\limits_{1^+} f = +\infty$ et $\lim\limits_{+\infty} f = 0$. De plus, $f$ est continue et dérivable sur son intervalle de définition.
\item D'après la définition de $f$,
\[
S_n = \frac{1}{n^2} \sum\limits_{k=n+1}^{+\infty} f\mathopen{}\left(\frac{k}{n}\right).
\]
Comme $f$ est décroissante, pour tout $t \in \interff{k/n}{(k+1)/n}$,
\begin{align*}
f\mathopen{}\left(\frac{k+1}{n}\right) &\leqslant f(t) \leqslant f\mathopen{}\left(\frac{k}{n}\right) \\
n^{-1} \sum\limits_{k=n+2}^{N+1} f\mathopen{}\left(\frac{k}{n}\right) &\leqslant \displaystyle\int_{1+1/n}^{N/n} f(t) \d t \leqslant n^{-1} \sum\limits_{k=n+1}^{N} f\mathopen{}\left(\frac{k}{n}\right).
\end{align*}
Comme $f(x) \sim_{+\infty} \frac{1}{x^2}$ l'intégrale $\displaystyle\int_{1+1/n}^{+\infty} f(t) \d t$ converge. Ainsi, la suite $\left(n^{-1}\sum\limits_{k=n+2}^{N+1} f\mathopen{}\left(\frac{k}{n}\right)\right)_N$ est croissante et majorée par $\displaystyle\int_{1+1/n}^{+\infty} f(t) \d t$ donc elle converge. Ainsi, en passant à la limite dans l'inégalité,
\begin{align*}
n S_n - f\mathopen{}\left(\frac{n+1}{n}\right) &\leqslant \displaystyle\int_{1+1/n}^{+\infty} f(t) \d t \leqslant n S_n \\
\frac{1}{n} \displaystyle\int_{1+1/n}^{+\infty} f(t) \d t &\leqslant S_n \leqslant \frac{1}{n} \displaystyle\int_{1+1/n}^{+\infty} f(t) \d t + \frac{1}{n^2} f\mathopen{}\left(\frac{n+1}{n}\right).
\end{align*}
De plus, $f(x) \sim_1 \frac{1}{\sqrt{2 (x - 1)}}$, donc $f$ est intégrable sur $\interff{1}{2}$ et $(S_n)$ converge vers $0$ car $f\mathopen{}\big((n+1)/n\big) \sim n^{1/2}$.

\item En reprenant l'encadrement précédent, la suite $\left(n^{-1} f((n+1)/n)\right)$ converge toujours vers $0$ et $(n S_n)$ converge vers $\displaystyle\int_1^{+\infty} f(t) \d t$.

\textbf{Remarque.} Une primitive de $f$ est donnée par la fonction $x \mapsto - \arctan\frac{1}{\sqrt{x^2 - 1}}$. Alors, $\displaystyle\int_1^{+\infty} f = \frac{\pi}{2}$.
\end{reponses}
\end{elemsolution}
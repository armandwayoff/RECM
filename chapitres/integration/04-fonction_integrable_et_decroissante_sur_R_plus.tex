\section{Fonctions décroissantes}

\todoinline{Ici, on n'a pas de théorème, que des exercices. Ce n'est pas un problème mais chercherait-on une présentation uniforme sur ces premiers thèmes ?}

\begin{exercice}
    \marginnote[0cm]{Source : \cite{exos_oraux} p. 268}
    Soit $f : \Rp \to \R$ une fonction continue, décroissante et intégrable. Montrer que $x f(x) \xrightarrow[x \to +\infty]{} 0$.
\end{exercice}

\begin{marginfigure}[-2cm]
    \centering
    \begin{tikzpicture}[scale=0.5]
  
  \def\xmax{4.5}
  \def\ymax{1.5}
  \def\ymin{-2}
  \def\xzero{2}
  \def\x{5}

    \begin{axis}[
        restrict x to domain=0:8,
        samples=100, % you don't need 1000, it only slows things down
        ticks=none,
        xmin = -1, xmax = \xmax+1,
        ymin = \ymin, ymax = \ymax,
        unbounded coords=jump,
        axis x line=middle,
        axis y line=middle,
        % xlabel={$x$},
        % ylabel={$y$},
        x label style={
          at={(axis cs:5.1,0.2)},
          anchor=west,
        },
        every axis y label/.style={
          at={(axis cs:0,1.5)},
          anchor=south
        },
        legend style={
          at={(axis cs:-5.2,4)},
          anchor=west, font=\scriptsize
        },
        declare function={f(\x)=2*e^(-\x^2/2)-(\x/7)^2-1;},
        ] 
      \addplot[name path=A,very thick,color=blue, mark=none, domain=0:\xmax] {f(x)}
          node [above=8mm,near start] {$f\textcolor{black}{(x)}$};
          
      \addplot[mark=*,fill=white] coordinates {(\xzero,{f(\xzero)})};
      \draw[dashed] (axis cs:0,{f(\xzero)}) node[left=1mm] (l) {$f(x_0)$} -| (axis cs:2,0) node[above] (a) {$x_0$}; 
      \draw (axis cs:\x-0.5,0) node[above] {$x$};

    \path [name path=B] (0,0)--(\xmax, 0);
    \addplot [blue!20!white] fill between [of=A and B, soft clip={domain=\xzero:\x}];

    \fill[red!30!white, pattern=north east lines] (\xzero,0) rectangle (\x-0.5,f(\xzero);
    
    \end{axis}

    \draw[->, black, thick] (5, 4) node[above] {$f(x_0) (x - x_0)$} to [out=-90,in=90] ($(4.5,2.6)$);

    \draw[->, black, thick] (4, 1) node[left] {\color{blue}$\displaystyle \int_{x_0}^x f(t)\, \mathrm{d} t$} to [out=0,in=-90] ($(4.5,1.6)$);
    
  \end{tikzpicture}
    \caption{ébauche}
\end{marginfigure}

\begin{elem_sol}
\begin{itemize}
\item Comme $f$ est décroissante, d'après le théorème de la limite monotone, il existe $\ell \in \R \cup \ens{-\infty}$ tel que $\lim\limits_{x\to+\infty} f(x) = \ell$.

\item Montrons par l'absurde que $f$ est positive. S'il existe $x_0 \geq 0$ tel que $f(x_0) < 0$, comme $f$ est décroissante,
\[
\forall\ x \geq x_0,\, f(x) \leq f(x_0).
\]
Ainsi,
\[
\forall\ x \geq x_0,\, \displaystyle\int_{x_0}^x f(t) \mathrm{d}t \leq f(x_0) (x - x_0).
\]
D'après le théorème d'encadrement, $\lim\limits_{x\to+\infty} \displaystyle\int_{x_0}^x f(t) \mathrm{d}t = -\infty$, ce qui est impossible car $f$ est intégrable. Ainsi, $f$ est à valeurs positives et $\ell \geq 0$.

\item Supposons par l'absurde que $\ell > 0$. Alors, il existe un réel $A$ tel que
\[
\forall\ x \geq A,\, f(x) \geq \frac{\ell}{2}.
\]
Ainsi,
\[
\forall\ x \geq A,\, \displaystyle\int_A^x f(t) \mathrm{d}t \geq \frac{\ell}{2} (x - A).
\]
D'après le théorème d'encadrement, $\lim\limits_{x\to+\infty} \displaystyle\int_A^x f(t) \mathrm{d}t = +\infty$, ce qui est impossible car $f$ est intégrable.

Finalement, $\lim\limits_{x\to+\infty} f(x) = 0$.

\item Soit $x \geq 0$. Comme $f$ est décroissante,
\[
\displaystyle\int_x^{2 x} f(t) \mathrm{d}t \leq (2 x - x) f(x).
\]

De même,
\[
\displaystyle\int_{x/2}^x f(t) \mathrm{d}t \geq \frac{x}{2} f(x).
\]

\begin{marginfigure}
    \centering
    \begin{tikzpicture}

  % https://copyprogramming.com/howto/how-do-i-draw-arrows-at-coordinate-on-a-plot
  
  \def\xmax{5.5}
  \def\ymax{1.2}
  \def\ymin{-0.2}
  \def\x{2.5}
  \def\xzero{\x/2}

  \def\colfonc{green!70!black}

    \begin{axis}[
        restrict x to domain=0:10,
        samples=100, % you don't need 1000, it only slows things down
        ticks=none,
        xmin = -1, xmax = \xmax+1,
        ymin = \ymin, ymax = \ymax,
        unbounded coords=jump,
        axis x line=middle,
        axis y line=middle,
        axis line style={-latex},
        % xlabel={$x$},
        % ylabel={$y$},
        x label style={
          at={(axis cs:5.1,0.2)},
          anchor=west,
        },
        every axis y label/.style={
          at={(axis cs:0,1.5)},
          anchor=south
        },
        legend style={
          at={(axis cs:-5.2,4)},
          anchor=west, font=\scriptsize
        },
        declare function={f(\x)=e^(-\x^2/8);},
        ] 
      \addplot[name path=A,very thick,color=\colfonc, mark=none, domain=0:\xmax] {f(x)} node [above=2mm,near start] {$f$};
          
      % \addplot[mark=*,fill=white] coordinates {(\xzero,{f(\xzero)})};
      % \draw[dashed] (axis cs:0,{f(\xzero)}) node[left=1mm] (l) {$f(x_0)$} -| (axis cs:2,0) node[above] (a) {$x_0$}; 
      \draw (\xzero,0) node[below] {$x/2$};

      \draw[dashed] (axis cs:0,{f(\x)}) node[left=1mm] (l) {$f(x)$} -| (axis cs:\x,0) node[below] (a) {$\vphantom{x/2}x$};
      
      % \draw (\x,0) node[below] {$x$};
      \draw (2*\x,0) node[below] {$\vphantom{x/2}2x$};

    \path [name path=B] (0,0)--(\xmax, 0);
    \addplot [blue!20!white] fill between [of=A and B, soft clip={domain=\xzero:\x}];

    \addplot [red!20!white] fill between [of=A and B, soft clip={domain=\x:2*\x}];

    \fill[pattern color=blue, pattern=north east lines] (\xzero,0) rectangle (\x,f(\x);
    \fill[pattern color=red, pattern=north east lines] (\x,0) rectangle (2*\x,f(\x);

    \end{axis}

    % \draw[->, black, thick] (5, 4) node[above] {$f(x_0) (x - x_0)$} to [out=-90,in=90] ($(4.5,2.6)$);

    % \draw[->, black, thick] (4, 1) node[left] {\color{blue}$\displaystyle \int_{x_0}^x f(t)\, \mathrm{d} t$} to [out=0,in=-90] ($(4.5,1.6)$);
    
\end{tikzpicture}
    \caption{ébauche}
\end{marginfigure}

\todoinline{On pourrait illustrer les deux inégalités précédentes en dessinant les aires sous des rectangles en hachuré et les aires sous les courbes en couleur pastel. On pourrait mettre une couleur pour $[x/2, x]$ et une différente sur $[x, 2x]$}


Ainsi,
\[
\displaystyle\int_x^{2 x} f(t) \mathrm{d}t \leq x f(x) \leq 2 \displaystyle\int_{x/2}^x f(t) \mathrm{d}t.
\]
En notant $F : x \mapsto \displaystyle\int_0^x f(t) \mathrm{d}t$, comme $F$ est intégrable, alors $F$ possède une limite finie en $+\infty$.

Alors,
\[
F(2 x) - F(x) \leq x f(x) \leq 2 (F(x) - F(x/2)).
\]

Ainsi, d'après le théorème d'encadrement, $\lim\limits_{x\to 0} x f(x) = 0$.
\end{itemize}
\end{elem_sol}

\todoarmand{Mettre un lien vers \url{http://ddmaths.free.fr/section115.html} ou pas car c'est très classique}



\begin{exercice}
    \marginnote[0cm]{Source : \cite{truc2019} p. 268}
    Soit $f : \interof{0}{1} \to \R$ une fonction continue, décroissante et intégrable. Alors,
    \[
    \lim\limits_{n \to +\infty} \frac{1}{n} \sum\limits_{k=1}^n f\left(\frac{k}{n}\right) = \displaystyle\int_0^1 f(t) \mathrm{d}t.
    \]
\end{exercice}

\todoinline{Ajouter une illustration avec une fonction type Riemann $x \mapsto \frac{1}{\sqrt{x}}$ sur $]0, 1]$ ?}

\begin{elem_sol}
On note $R_n(f) = \frac{1}{n} \sum\limits_{k=1}^n f\left(\frac{k}{n}\right)$.
\begin{itemize}
\item Soit $n \geq 1$. On utilise la décroissance de la fonction $f$, pour tout $k \leq t \leq k + 1 \leq n-1$,
\begin{align*}
\frac{1}{n} f\left(\frac{k+1}{n}\right) &\leq \displaystyle\int_{k/n}^{(k+1)/n} f(t) \mathrm{d}t \leq \frac{1}{n} f\left(\frac{k}{n}\right)\\
\frac{1}{n} \sum\limits_{k=2}^{n} f\left(\frac{k}{n}\right) &\leq \displaystyle\int_{1/n}^1 f(t) \mathrm{d}t \leq \frac{1}{n} \sum\limits_{k=1}^{n-1} f\left(\frac{k}{n}\right)\\
R_n - \frac{1}{n} f\left(\frac{1}{n}\right) &\leq \displaystyle\int_{1/n}^1 f(t) \mathrm{d}t \leq R_n - \frac{f(1)}{n}.
\end{align*}

Ainsi,
\begin{align*}
\displaystyle\int_{1/n}^1 f(t) \mathrm{d}t + \frac{f(1)}{n} &\leq R_n \leq \displaystyle\int_{1/n} f(t) \mathrm{d}t + \frac{1}{n}  f\left(\frac{1}{n}\right).
\end{align*}

\item Comme $f$ est décroissante et intégrable sur $]0, 1]$,
\[
\frac{1}{n} f\left(\frac{1}{n}\right) \leq \displaystyle\int_0^{1/n} f(t) \mathrm{d}t.
\]

\item Comme $f$ est intégrable sur $]0, 1]$,
\[
\lim\limits_{n\to+\infty} \displaystyle\int_{1/n}^1 f(t) \mathrm{d}t = \displaystyle\int_0^1 f(t) \mathrm{d}t.
\]
\end{itemize}

Ainsi, d'après le théorème d'encadrement,
\[
\lim\limits_{n\to+\infty} R_n(f) = \displaystyle\int_0^1 f(t) \mathrm{d}t.
\]
\end{elem_sol}

\begin{remarque}
Une étude plus détaillée de ce résultats et de ses limites est discutée dans \cite{truc2019} p. 268.
\end{remarque}

%---------------

\begin{exercice}
\cite{RMS 888 2016 - ENSAM}
Soit $f : x \mapsto \frac{1}{x \sqrt{x^2 - 1}}$ définie sur $]1, +\infty[$.
\begin{enumerate}
\item Étudier et tracer la fonction $f$.

\smallskip
Pour tout entier naturel $n$, on pose $S_n = \sum\limits_{k=n+1}^{+\infty} \frac{1}{k \sqrt{k^2 - n^2}}$.
\item Étudier la convergence et la limite de la suite $(S_n)$.

\item Même question avec la suite $(n S_n)$.
\end{enumerate}
\end{exercice}

\todoinline{Illustrer $n S_n$ et $\int_{1}^{+\infty} f(t) \d t$ sur un même graphique ?}

\begin{elem_sol}
\begin{enumerate}
\item $f$ est décroisante, à valeurs positives, $\lim\limits_{1^+} f = +\infty$ et $\lim\limits_{+\infty} f = 0$. De plus, $f$ est continue et dérivable.

\item D'après la définition de $f$,
\[
S_n = \frac{1}{n^2} \sum\limits_{k=n+1}^{+\infty} f(k/n).
\]
Comme $f$ est décroissante, pour tout $t \in [k/n,(k+1)/n]$,
\begin{align*}
f((k+1)/n) &\leq f(t) \leq f(k/n) \\
n^{-1} \sum\limits_{k=n+2}^{N+1} f(k/n) &\leq \displaystyle\int_{1+1/n}^{N/n} f(t) \mathrm{d}t \leq n^{-1} \sum\limits_{k=n+1}^{N} f(k/n).
\end{align*}
Comme $f(x) \sim_{+\infty} \frac{1}{x^2}$, la suite $(S_{n,N})_N$ est croissante et majorée par $\displaystyle\int_{1+1/n}^{+\infty} f(t) \mathrm{d}t$ qui est convergente. Ainsi, en passant à la limite dans l'inégalité,
\begin{align*}
n S_n - f((n+1)/n) &\leq \displaystyle\int_{1+1/n}^{+\infty} f(t) \mathrm{d}t \leq n S_n \\
\frac{1}{n} \displaystyle\int_{1+1/n}^{+\infty} f(t) \mathrm{d}t &\leq S_n \leq \frac{1}{n} \displaystyle\int_{1+1/n}^{+\infty} f(t) \mathrm{d}t + \frac{1}{n^2} f((n+1)/n).
\end{align*}
De plus, $f(x) \sim_1 \frac{1}{\sqrt{2 (x - 1)}}$, donc $f$ est intégrable en $1$ et $(S_n)$ converge vers $0$ car $f((n+1)/n) \sim n^{1/2}$.

\item En reprenant l'encadrement précédent, $\left(n^{-1} f((n+1)/n)\right)$ converge toujours vers $0$ et $(n S_n)$ converge vers $\displaystyle\int_1^{+\infty} f(t) \mathrm{d}t$.

\textbf{Remarque.} $\displaystyle\int^x f(t) \mathrm{d}t = - \arctan\frac{1}{\sqrt{x^2 - 1}}$ et $\displaystyle\int_1^{+\infty} f = \frac{\pi}{2}$.
\end{enumerate}
\end{elem_sol}
\begin{exercice}
    \marginnote[0cm]{Source : \cite{exos_oraux} p. 268}
    Soit $f : \Rp \to \R$ une fonction continue, décroissante et intégrable. Montrer que $x f(x) \xrightarrow[x \to +\infty]{} 0$.
\end{exercice}

\todoinline{On pourrait faire un dessin pour les  points 1 et 2 qui sont classique mais pas très bien compris en minorant/majorant l'aire sous $f$ par l'aire d'un rectangle. On peut aussi illustrer l'encadrement final.}


\begin{elem_sol}
\begin{itemize}
\item Comme $f$ est décroissante, d'après le théorème de la limite monotone, il existe $\ell \in \R \cup \ens{-\infty}$ tel que $\lim\limits_{x\to+\infty} f(x) = \ell$.

\item Montrons par l'absurde que $f$ est positive. S'il existe $x_0 \geq 0$ tel que $f(x_0) < 0$, comme $f$ est décroissante,
\[
\forall\ x \geq x_0,\, f(x) \leq f(x_0).
\]
Ainsi,
\[
\forall\ x \geq x_0,\, \displaystyle\int_{x_0}^x f(t) \mathrm{d}t \leq f(x_0) (x - x_0).
\]
D'après le théorème d'encadrement, $\lim\limits_{x\to+\infty} \displaystyle\int_{x_0}^x f(t) \mathrm{d}t = -\infty$, ce qui est impossible car $f$ est intégrable. Ainsi, $f$ est à valeurs positives et $\ell \geq 0$.

\item Supposons par l'absurde que $\ell > 0$. Alors, il existe un réel $A$ tel que
\[
\forall\ x \geq A,\, f(x) \geq \frac{\ell}{2}.
\]
Ainsi,
\[
\forall\ x \geq A,\, \displaystyle\int_A^x f(t) \mathrm{d}t \geq \frac{\ell}{2} (x - A).
\]
D'après le théorème d'encadrement, $\lim\limits_{x\to+\infty} \displaystyle\int_A^x f(t) \mathrm{d}t = +\infty$, ce qui est impossible car $f$ est intégrable.

Finalement, $\lim\limits_{x\to+\infty} f(x) = 0$.

\item Soit $x \geq 0$. Comme $f$ est décroissante,
\[
\displaystyle\int_x^{2 x} f(t) \mathrm{d}t \leq (2 x - x) f(x).
\]

De même,
\[
\displaystyle\int_{x/2}^x f(t) \mathrm{d}t \geq \frac{x}{2} f(x).
\]

Ainsi,
\[
\displaystyle\int_x^{2 x} f(t) \mathrm{d}t \leq x f(x) \leq 2 \displaystyle\int_{x/2}^x f(t) \mathrm{d}t.
\]
En notant $F : x \mapsto \displaystyle\int_0^x f(t) \mathrm{d}t$, comme $F$ est intégrable, alors $F$ possède une limite finie en $+\infty$.

Alors,
\[
F(2 x) - F(x) \leq x f(x) \leq 2 (F(x) - F(x/2)).
\]

Ainsi, d'après le théorème d'encadrement, $\lim\limits_{x\to 0} x f(x) = 0$.
\end{itemize}

\todoarmand{Correction : exercice 3 de \url{http://ddmaths.free.fr/section115.html}}
\todoinline{J'ai rédigé à ma sauce ci-dessus}


% \begin{itemize}
% \item Montrer que $f$ tend vers 0 en utilisant sa décroissance et son intégrabilité. $f$ est donc à valeurs positives. 
% \item Encadrer $x f(x)$ en écrivant $x=2 \cdot \frac{x}{2}$.
% \end{itemize}
% \end{elem_sol}


\todoinline{Ajouter une illustration avec une fonction type Riemann ?}

\begin{exercice}
    \marginnote[0cm]{Source : \cite{truc2019} p. 268}
    Soit $f : ]0, 1] \to \R$ une fonction continue, décroissante et intégrable. Alors,
    \[
    \lim\limits \frac{1}{n} \sum\limits_{k=1}^n f\left(\frac{k}{n}\right) = \displaystyle\int_0^1 f(t) \mathrm{d}t.
    \]
\end{exercice}

\begin{elem_sol}
On note $R_n(f) = \frac{1}{n} \sum\limits_{k=1}^n f\left(\frac{k}{n}\right)$.
\begin{itemize}
\item Soit $n \geq 1$. On utilise la décroissance de la fonction $f$, pour tout $k \leq t \leq k + 1 \leq n-1$,
\begin{align*}
\frac{1}{n} f\left(\frac{k+1}{n}\right) &\leq \displaystyle\int_{k/n}^{(k+1)/n} f(t) \mathrm{d}t \leq \frac{1}{n} f\left(\frac{k}{n}\right)\\
\frac{1}{n} \sum\limits_{k=2}^{n} f\left(\frac{k}{n}\right) &\leq \displaystyle\int_{1/n}^1 f(t) \mathrm{d}t \leq \frac{1}{n} \sum\limits_{k=1}^{n-1} f\left(\frac{k}{n}\right)\\
R_n - \frac{1}{n} f\left(\frac{1}{n}\right) &\leq \displaystyle\int_{1/n}^1 f(t) \mathrm{d}t \leq R_n - \frac{f(1)}{n}.
\end{align*}

Ainsi,
\begin{align*}
\displaystyle\int_{1/n}^1 f(t) \mathrm{d}t + \frac{f(1)}{n} &\leq R_n \leq \displaystyle\int_{1/n} f(t) \mathrm{d}t + \frac{1}{n}  f\left(\frac{1}{n}\right).
\end{align*}

\item Comme $f$ est décroissante et intégrable sur $]0, 1]$,
\[
\frac{1}{n} f\left(\frac{1}{n}\right) \leq \displaystyle\int_0^{1/n} f(t) \mathrm{d}t.
\]

\item Comme $f$ est intégrable sur $]0, 1]$,
\[
\lim\limits_{n\to+\infty} \displaystyle\int_{1/n}^1 f(t) \mathrm{d}t = \displaystyle\int_0^1 f(t) \mathrm{d}t.
\]
\end{itemize}

Ainsi, d'après le théorème d'encadrement,
\[
\lim\limits_{n\to+\infty} R_n(f) = \displaystyle\int_0^1 f(t) \mathrm{d}t.
\]
\end{elem_sol}


\todoinline{On peut éventuellement ajouter l'exercice suivant. Il s'agit d'une somme de Riemann mais sur $[1, +\infty[$}
%---------------

\begin{exercice}
\cite{RMS 888 2016 - ENSAM}
Soit $f : x \mapsto \frac{1}{x \sqrt{x^2 - 1}}$ définie sur $]1, +\infty[$.
\begin{enumerate}
\item Étudier et tracer la fonction $f$.

\smallskip
Pour tout entier naturel $n$, on pose $S_n = \sum\limits_{k=n+1}^{+\infty} \frac{1}{k \sqrt{k^2 - n^2}}$.
\item Étudier la convergence et la limite de la suite $(S_n)$.

\item Même question avec la suite $(n S_n)$.
\end{enumerate}
\end{exercice}

\begin{elem_sol}
\begin{enumerate}
\item $f$ est décroisante, à valeurs positives, $\lim\limits_{1^+} f = +\infty$ et $\lim\limits_{+\infty} f = 0$. De plus, $f$ est continue et dérivable.

\item D'après la définition de $f$,
\[
S_n = \frac{1}{n^2} \sum\limits_{k=n+1}^{+\infty} f(k/n).
\]
Comme $f$ est décroissante, pour tout $t \in [k/n,(k+1)/n]$,
\begin{align*}
f((k+1)/n) &\leq f(t) \leq f(k/n) \\
n^{-1} \sum\limits_{k=n+2}^{N+1} f(k/n) &\leq \displaystyle\int_{1+1/n}^{N/n} f(t) \mathrm{d}t \leq n^{-1} \sum\limits_{k=n+1}^{N} f(k/n).
\end{align*}
Comme $f(x) \sim_{+\infty} \frac{1}{x^2}$, la suite $(S_{n,N})_N$ est croissante et majorée par $\displaystyle\int_{1+1/n}^{+\infty} f(t) \mathrm{d}t$ qui est convergente. Ainsi, en passant à la limite dans l'inégalité,
\begin{align*}
n S_n - f((n+1)/n) &\leq \displaystyle\int_{1+1/n}^{+\infty} f(t) \mathrm{d}t \leq n S_n \\
\frac{1}{n} \displaystyle\int_{1+1/n}^{+\infty} f(t) \mathrm{d}t &\leq S_n \leq \frac{1}{n} \displaystyle\int_{1+1/n}^{+\infty} f(t) \mathrm{d}t + \frac{1}{n^2} f((n+1)/n).
\end{align*}
De plus, $f(x) \sim_1 \frac{1}{\sqrt{2 (x - 1)}}$, donc $f$ est intégrable en $1$ et $(S_n)$ converge vers $0$ car $f((n+1)/n) \sim n^{1/2}$.

\item En reprenant l'encadrement précédent, $\left(n^{-1} f((n+1)/n)\right)$ converge toujours vers $0$ et $(n S_n)$ converge vers $\displaystyle\int_1^{+\infty} f(t) \mathrm{d}t$.

\textbf{Remarque.} $\displaystyle\int^x f(t) \mathrm{d}t = - \arctan\frac{1}{\sqrt{x^2 - 1}}$ et $\displaystyle\int_1^{+\infty} f = \frac{\pi}{2}$.
\end{enumerate}
\end{elem_sol}

% \todoinline{Concernant les fonctions décroissantes, on pourrait ajouter un exercice qui doit ressembler à (je ne l'ai jamais écrit...) : Soit $f$ décroissante, continue par morceaux et intégrable sur $]0, 1]$. Alors, $\lim\limits_{n\to+\infty} \sum\limits_{k=0}^n f\left(\frac{k}{n}\right) = \int_0^1 f(t) \d t$. Il se prête à une illustration graphique ! Je mets dans un dossier un article qui contient ce résultat. J'ai un exercice d'oral qui utilise ce résultat. À chercher ?}

\section{Intégrale de \nom{Poisson} : une utilisation des sommes de \nom{Riemann}}

\todoarmand{
Un dm \url{http://alain.troesch.free.fr/2023/Fichiers/dm11.pdf} qui montre comment le calcul de cette intégrale peut se ramener à
l’intégrale du noyau de Poisson.
}

\todoinline{Trouver une référence pour le noyau de Poisson.}

%---------------

\begin{prop}[Intégrale de \nom{Poisson}]
Pour tout $x \in \Rp$, on pose $I(x) = \int_0^\pi \ln(1 - 2 x \cos t + x^2) \d t$. Alors, pour tout $x$ réel positif,
\begin{align*}
I(x) &= 2 \pi \ln(x).
\end{align*}
\end{prop}

\begin{exercice}\label{exercice:integralePoisson}
\begin{questions}
\item Montrer que l'intégrale $I$ est bien définie sur $\Rp$.

\item Soit $n \in \Ne$ et $x \neq 1$. Montrer que
\[
\prod_{k=0}^{n-1} \mathopen{}\left(1 - 2 \cos\mathopen{}\left(\frac{k\pi}{n}\right) x + x^2\right)
= \frac{x^{2n} - 1}{x + 1} (x - 1).
\]
% Déterminer une expression simple de $\sum\limits_{k=0}^{n-1} \ln\mathopen{}\big(1 - 2 x \cos\mathopen{}\big(\frac{k\pi}{n}\big) + x^2\big)$.

\item En déduire la valeur de $I(x)$ pour $x \neq 1$.

\item Montrer que $I(1) = 2 \pi\ln(2) + 4 \int_0^{\pi/2} \ln(\sin u) \d u$.

\item En déduire la valeur de $I(1)$.
\end{questions}
\end{exercice}


\begin{solution}
\begin{reponses}
\item En utilisant le discriminant réduit, $\cos(t)^2 - 1 < 0$ sur $\interoo{0}{\pi}$ donc le trinôme n'admet pas de racine et le logarithme est bien défini.

De plus, en $t = 0$, $1 - 2 x + x^2 = (1 - x)^2 > 0$ car $x \neq 1$.
En $t = \pi$, $1 + 2 x + x^2 = (1 + x)^2 > 0$ car $x \neq 1$.

Ainsi, $I(x)$ est bien définie pour $x \neq 1$.

\item En utilisant les factorisations classiques ainsi que les résultats sur les racines $n$-èmes de l'unité,
\begin{align*}
\prod_{k=0}^{n-1} \mathopen{}\left(1 - 2 \cos\mathopen{}\left(\frac{k\pi}{n}\right) x + x^2\right)
&= \prod_{k=0}^{n-1} \mathopen{}\left(x - \e^{\i k\pi/n}\right) \mathopen{}\left(x - \e^{-\i k\pi/n}\right) \\
&= \prod_{k=0}^{n-1} \mathopen{}\left(x - \e^{2\i k\pi/(2n)}\right) \prod_{k=n+1}^{2n} \mathopen{}\left(x - \e^{2\i k\pi/(2n)}\right) \\
&= \frac{x^{2n} - 1}{x + 1} (x - 1)
\end{align*}

\item D'après le calculs précédents,
\begin{align*}
S_n(f)
&\defeq \frac{\pi}{n}\sum\limits_{k=0}^{n-1} \ln\mathopen{}\left(1 - 2 x \cos\mathopen{}\left(\frac{k\pi}{n}\right) + x^2\right)
&= \frac{\pi}{n} \ln\mathopen{}\abs{x^{2n}-1} - \frac{\pi}{n} \ln \mathopen{}\abs{\frac{x+1}{x-1}} \\
&= 2 \pi \ln\mathopen{}\abs{x} + \frac{\pi}{n} \ln \mathopen{}\abs{1 - x^{-2n}} - \frac{\pi}{n} \ln \mathopen{}\abs{\frac{x+1}{x-1}}.
\end{align*}

Ainsi, en utilisant une somme de \nom{Riemann} de pas $\frac{\pi}{n}$, $\lim\limits_{n\to +\infty} S_n(f) = 2 \pi \ln\mathopen{}\abs{x}$.

Finalement, pour tout $x \neq 1$, $I(x) = 2 \pi \ln\mathopen{}\abs{x}$.
\item \marginnote[-7pt]{\hyperref[exercice:integraleEuler]{Intégrale d'\nom{Euler}}}D'après les propriétés des fonctions trigonométriques,
\begin{align*}
\ln(2 - 2 \cos t)
= \ln\mathopen{}\big(2(1 - \cos t)\big)
= \ln\mathopen{}\big(4 \sin(t/2)^2\big)
= \ln(4) + 2 \ln\mathopen{}\big(\sin(t/2)\big)
\end{align*}

\item En effectuant le changement de variable affine $\fonctionligne[\varphi]{u}{2 u}$,
\begin{align*}
\int_0^\pi \ln(2 - 2 \cos t) \d t
= 2 \pi \ln(2) + 2 \int_0^\pi \ln\mathopen{}\big(\sin(t/2)\big) \d t
= 2 \pi \ln(2) + 4 \int_0^{\pi/2} \ln(\sin u) \d u
\end{align*}
Or, $\int_0^{\pi/2} \ln(\sin u) \d u = \int_0^{\pi/2} \ln(\cos u) \d u$. Ainsi,
% \begin{comment}

\begin{align*}
\int_0^{\pi/2} \ln(\sin u) \d u + \int_0^{\pi/2} \ln(\cos u) \d u
&= \int_0^{\pi/2} \ln\mathopen{}\big(\sin(2 t)\big) \d t - \frac{\pi}{2} \ln(2) \\
&= \frac{1}{2} \int_0^\pi \ln(\sin t) \d t - \frac{\pi}{2} \ln(2) \\
&= \begin{multlined}[t]\frac{1}{2} \int_0^{\pi/2} \ln(\sin t) \d t + \frac{1}{2} \int_{\pi/2}^{\pi} \ln(\sin t) \d t \\
- \frac{\pi}{2} \ln(2)
\end{multlined}\\
&= \int_0^{\pi/2} \ln(\sin t) \d t - \frac{\pi}{2} \ln(2)
\end{align*}

Ainsi,
\[
\int_0^{\pi/2} \ln(\sin t) \d t = -\frac{\pi}{2} \ln(2)
\]

D'où $I(1) = 0$.
% \end{comment}
\end{reponses}
\end{solution}

\begin{marginfigure}[-5cm]
    \centering
    \includegraphics[scale=0.08]{illustrations/Journal_de_l_ecole_polytechnique_cahier_17_1815.png}
    % \includepdf[pages={3},scale=.4]{Journal_de_l_ecole_polytechnique_cahier_17_1815.pdf}
    % \caption{\chevron{Suite du mémoire sur les intégrales définies}, Journal de l'École polytechnique, Cahier 17, X (1815), 612-631 (% \url{https://gallica.bnf.fr/ark:/12148/bpt6k433673r/f614.item})}
\end{marginfigure}
\todoarmand{Problème avec la légende de la figure}

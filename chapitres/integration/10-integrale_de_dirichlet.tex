\section{Intégrale de \nom{Dirichlet}}\label{sec:intDirichlet}

\todoarmand{Éléments historiques : \url{https://hsm.stackexchange.com/a/6827}}

\marginnote[0cm]{Dans le cadre de la diffraction de \nom{Fraunhofer}, si l'on appelle $D$ la distance entre l'écran et la fente, alors l'intensité $I$ en un point $x$ de l'écran s'écrit
\[
I(x) = I_0\, \mathrm{sinc}\mathopen{}\left(\frac{\pi \, a}{\lambda \, D}\, x\right)^2
\]
où $\mathrm{sinc}$ est la fonction \textsl{sinus cardinal}, $a$ est la largeur de la fente et $\lambda$  est la longueur d'onde de la radiation lumineuse. \\
\url{https://fr.wikipedia.org/wiki/Diffraction_par_une_fente}, pour des figures: \url{https://physique.ensc-rennes.fr/tp_diffraction.php} et \url{https://tikz.net/optics_diffraction/}
}

\begin{defi}[Sinus cardinal]
    On nomme \definir{sinus cardinal} la fonction définie par
    \[
    \fonction[\mathrm{sinc}]{\Re}{\R}{x}{\frac{\sin(x)}{x}}.
    \]
\end{defi}

\begin{theo}[Intégrale de \nom{Dirichlet} (1829)]
L'intégrale $\displaystyle\int_{0}^{+\infty} \frac{\sin x}{x} \d x$ est convergente et
    \[
    \int_{0}^{+\infty} \frac{\sin x}{x} \d x = \frac{\pi}{2}.
    \]
\end{theo}

\begin{marginfigure}[0cm]
    \begin{tikzpicture}
    
\begin{axis}[
    % width=7.5cm,
    % grid=both,
    xmin=-11,
    xmax=11,
    ymin=-0.25,
    ymax=1.15,
    % ylabel=$\mathrm{sinc}(x)$,
    axis lines=middle,
    axis line style=thick,
    axis line style={-latex},
    xticklabels={},
    xtick={-3*3.141592, -2*3.141592, -3.141592, 3.141592, 2*3.141592, 3*3.141592},
    ytick={0, 1},
    xlabel=$x$,
    every axis x label/.style={at={(current axis.right of origin)},anchor=south},
]
              
  \addplot[domain=-15:15, myblue, samples=200, name path=B] plot[thick] {sin(deg(x))/x};

  \path[name path=xaxis]
      (-15,0) -- (\pgfkeysvalueof{/pgfplots/xmax},0);
    \addplot[mylightblue, opacity=0.9] fill between[of=xaxis and B];
  
  \node[myblue,above left] at (-2,0.5) {$\displaystyle x \mapsto \frac{\sin(x)}{x}$};

  \node[myblue] at ({0},{1/3}) {$\displaystyle \frac{\pi}{2}$};
\end{axis}
\begin{axis}[
    % width=7.5cm,
    % grid=both,
    xmin=-11,
    xmax=11,
    ymin=-0.25,
    ymax=1.15,
    % ylabel=$\mathrm{sinc}(x)$,
    axis lines=middle,
    % axis line style=thick,
    % axis line style={-latex},
    axis line style={draw=none},
    xticklabels={\contour{white}{$-3\pi$}, \contour{white}{$-2\pi$}, \contour{white}{$-\pi$}, \contour{white}{$\pi$}, \contour{white}{$2\pi$}, \contour{white}{$3\pi$}},
    xtick={-3*3.141592, -2*3.141592, -3.141592, 3.141592, 2*3.141592, 3*3.141592},
    ytick={0, 1},
]
\end{axis}
\end{tikzpicture}

\end{marginfigure}

Nous proposons deux démonstrations de l'existence de l'intégrale de \nom{Dirichlet}, une première utilisant le théorème des séries alternées et une deuxième utilisant l'intégration par parties.

\begin{exercice}
Pour tout $n$ entier naturel non nul, on pose $u_n = \displaystyle\int_{n \, \pi}^{(n+1) \, \pi} \mathrm{sinc}(t) \d t$.
\begin{questions}
\item Montrer que la fonction $\mathrm{sinc}$ est intégrable sur $\interof{0}{1}$.

\item Montrer que $\sum u_n$ est une série alternée et en déduire la convergence de l'intégrale de \nom{Dirichlet}.

\item En utilisant une intégration par parties, proposer une autre démonstration de la convergence de l'intégrale de \nom{Dirichlet}.
\end{questions}
\end{exercice}

\begin{solution}
\begin{reponses}
\item La fonction $\mathrm{sinc}$ est continue sur $\interof{0}{1}$. Comme elle est prolongeable par continuité par la valeur $1$ en $0$, elle est intégrable sur $\interof{0}{1}$.

\item Soit $n \geqslant 1$, par un changement de variable affine,
\[
u_n =
\int_{n \pi}^{(n+1) \pi} \frac{\sin t}{t} \d t
= (-1)^n \int_0^\pi \frac{\sin u}{u + n \, \pi} \d u.
\]
En notant $u_n$ cette quantité, comme la fonction sinus est positive sur $\interff{0}{\pi}$,
\begin{comment}
\item $u_n u_{n+1} \leq 0$,
\item $\abs{u_n} \leq \frac{1}{n\pi} \int_0^\pi \sin(u) \d u$ soit $u_n \to 0$,
\item $\abs{u_{n+1}} \leq \abs{u_n}$.
\end{comment}
\[
u_n \, u_{n+1} \leqslant 0, \quad \abs{u_n} \leqslant \frac{1}{n \, \pi} \int_0^\pi \sin(u) \d u \text{ soit } u_n \to 0, \quad \text{et} \quad \abs{u_{n+1}} \leqslant \abs{u_n}.
\]
Ainsi, d'après le \theoremeutilise{théorème des séries alternées}{theo:seriesalternees}, $\sum u_n$ converge.

\medskip

Cependant, la fonction $\displaystyle x \mapsto \int_0^x \frac{\sin t}{t} \d t$ n'est pas monotone et on ne peut donc pas encore conclure directement à la convergence de l'intégrale.

On pose ainsi $\displaystyle n_x = \left\lfloor\frac{x}{\pi}\right\rfloor$. En posant $\displaystyle F(x) = \int_1^{n_x\pi} \frac{\sin t}{t} \d t + \int_{n_x}^x \frac{\sin t}{t} \d t$, le premier terme converge d'après la question précédente. Le second est majoré par une quantité qui tend vers $0$. Ainsi, l'intégrale de \nom{Dirichlet} converge.

\item Les fonctions $\fonctionligne[u]{t}{-\cos(t)}$ et $\fonctionligne[v]{t}{1/t}$ sont de classe $\Cont^1$ sur $\interfo{1}{+\infty}$ et $\lim_{+\infty} u \, v = 0$. Ainsi, par intégration par parties, les intégrales $\displaystyle \int_1^{+\infty} \frac{\sin t}{t} \d t$ et $\displaystyle \int_1^{+\infty} \frac{\cos t}{t^2} \d t$ sont de même nature. \\
D'une part, la fonction $\displaystyle t \mapsto \frac{\cos t}{t^2}$ est continue sur $\interfo{1}{+\infty}$ et d'autre part on a la majoration $\displaystyle \frac{\abs{\cos t}}{t^2} \leqslant \frac{1}{t^2}$ par une fonction intégrable sur $\interfo{1}{+\infty}$. Ainsi, d'après les \theoremeutilise{théorèmes de comparaison}{theo:comparaison}, la fonction $t \mapsto \frac{\abs{\cos t}}{t^2}$ est intégrable et l'intégrale $\displaystyle \int_1^{+\infty} \frac{\cos t}{t^2} \d t$ converge. Finalement, nous avons démontré la convergence de l'intégrale $\displaystyle \int_1^{+\infty} \frac{\sin t}{t} \d t$.
\end{reponses}
\end{solution}

\begin{remarque}
L'intégration par parties préserve la régularité de l'intégrale mais ne préserve pas l'intégrabilité. La section suivante permet d'illustrer cette propriété.
\end{remarque}

%-----------
\subsection{Non intégrabilité}

\begin{prop}
    La fonction sinus cardinal $\fonctionligne[\mathrm{sinc}]{t}{\frac{\sin(t)}{t}}$ n'est pas intégrable sur $\interoo{0}{+\infty}$.
\end{prop}

\begin{exercice}
Soit $N \in \Ne$.
\begin{questions}
\item Montrer que $\displaystyle\int_0^{N\pi} \module{\mathrm{sinc}(x)} \d x \geqslant \frac{2}{\pi} \sum_{k=1}^N \frac{1}{k}$.

\item Conclure.
\end{questions}
\end{exercice}

\begin{demo}
\source{\href{https://www.agreg-maths.fr/uploads/versions/1175/dirichlet.pdf}{Intégrale de \textsc{Dirichlet} -- Florian \textsc{Dussap}}}
\begin{reponses}
\item En utilisant la relation de \nom{Chasles},
    \begin{align*}
        \int_0^{N \pi} \frac{|\sin x|}{x} \d x &= \sum_{k=0}^{N-1} \int_{k \pi}^{(k+1) \pi} \frac{|\sin x|}{x} \d x \\
&= \sum_{k=0}^{N-1} \int_0^\pi \frac{|\sin x|}{x + k \, \pi} \d x        \text{, par un changement de variable }  \\
        &\geqslant \sum_{k=0}^{N-1} \frac{1}{(k+1) \pi} \int_0^\pi \sin x \d x \\
        &\geqslant \frac{2}{\pi} \sum_{k=1}^N \frac{1}{k}.
        \end{align*}

\item Comme $\sum \frac{1}{k}$ diverge, alors $\displaystyle\lim\limits_{N\to+\infty} \int_0^{N\pi} \module{\mathrm{sinc}(x)} \d x = +\infty$.

La fonction $x \mapsto \displaystyle\int_0^x \module{\mathrm{sinc}(t)} \d t$ étant croissante, on obtient même $\displaystyle \lim_{x\to+\infty} \int_0^x \module{\mathrm{sinc}(t)} \d t = +\infty$.
\end{reponses}
\end{demo}

%-----------        
\subsection{Calcul via le lemme de \nom{Lebesgue}}

\marginnote[7pt]{\hyperref[sec:lemmeLebesgue]{\faLink~Lemme de \nom{Lebesgue}}}
\begin{exercice}
Soit $n \in \N$.
\begin{questions}
    \item Calculer $I_n = \displaystyle \int_0^\pi \frac{\sin\mathopen{}\big[(n + 1/2)t\big]}{2 \sin \frac{t}{2}} \d t$. \\
    \emph{Indication :} Calculer $I_{n+1} - I_n$. 

    \item Montrer que la fonction définie sur $\interof{0}{\pi}$ par $f(x) = \frac{1}{x} - \frac{1}{2 \sin \frac{x}{2}}$ peut être prolongée à $\interff{0}{\pi}$ en une fonction de classe $\Cont^1$. 
    \item En déduire la valeur de l'intégrale $\displaystyle I = \int_0^{+\infty} \frac{\sin t}{t} \d t$.\\
    \emph{Indication :} On utilisera le lemme de \nom{Lebesgue}
\end{questions}
\end{exercice}

\begin{solution}
\begin{reponses}
\item En utilisant les formules de trigonométrie,
\begin{align*}
I_{n+1} - I_n
&= \int_0^\pi \frac{\sin\mathopen{}\big[(n + 1 + 1/2)t\big] - \sin\mathopen{}\big[(n + 1/2) t\big]}{2 \sin \frac{t}{2}} \d t\\
&= \int_0^\pi \cos\mathopen{}\big[(n + 1) t\big] \d t
= \frac{1}{n + 1} \Big[\sin\mathopen{}\big[(n + 1) t\big]\Big]_{0}^\pi
= 0.
\end{align*}

Ainsi, la suite $(I_n)$ est constante et $I_0 = \frac{\pi}{2}$. Donc, pour tout $n$ entier naturel, $I_n = \frac{\pi}{2}$.

\item La fonction $f$ est continue et dérivable sur $\interof{0}{\pi}$. De plus, pour tout réel $x \in \interof{0}{\pi}$, $\displaystyle f(x) = \frac{2 \sin(x/2) - x}{2 x \sin(x/2)}$.

En utilisant les développements limités classiques,
\[
f(x)
\sim -\frac{x^3}{24} \times \frac{1}{x^2}
\sim -\frac{x}{24}.
\]
Ainsi, la fonction $f$ est prolongeable par continuité en $0$ par $f(0) = 0$.

\medskip

La fonction $f$ est dérivable sur $\interof{0}{\pi}$ et pour tout $x \in \interof{0}{\pi}$,
\begin{align*}
f'(x) = -\frac{1}{x^2} + \frac{\cos(x/2)}{4 \sin(x/2)^2} = \frac{\cos(x/2) - 4 \sin(x/2)^2}{4 x^2 \sin(x/2)^2} \sim -\frac{1}{24}.
\end{align*}
Ainsi, $\lim_{x\to0} f'(x) = -\frac{1}{24}$. D'après le \theoremeutilise{théorème de prolongement dérivable}{theo:prolongementderivable}, la fonction $f$ est prolongeable en une fonction de classe $\Cont^1$ sur $\interff{0}{\pi}$.

\item En utilisant la fonctin $f$ introduite précédemment,
\begin{align*}
\int_0^{(n + 1/2) \pi} \frac{\sin t}{t} \d t
&= \int_0^\pi \frac{\sin\mathopen{}\big[(n + 1/2) t\big]}{t} \d t\\
&= \int_0^\pi \left[ \frac{\sin\mathopen{}\big[(n + 1/2) t\big]}{t} - \frac{\sin\mathopen{}\big[(n + 1/2) t\big]}{2 \sin (t/2)} + \frac{\sin\mathopen{}\big[(n + 1/2) t\big]}{2 \sin(t/2)} \right] \d t\\
&= \int_0^\pi f(t) \sin\mathopen{}\big[(n + 1/2) t\big] \d t + I_n.
\end{align*}
Comme $f$ est de classe $\Cont^1$, d'après le lemme de \nom{Lebesgue},
\[
\lim_{n\to+\infty} \int_0^\pi f(t) \sin\mathopen{}\big[(n + 1/2) t\big] \d t = 0.
\]

Finalement, $I = \frac{\pi}{2}$.
\end{reponses}
\end{solution}

%-----------
\subsection{Calcul via une intégrale à paramètre}

\begin{comment}
\begin{exercice}
    Exercice 167 p.179 (bestiaire.pdf) \\
    On considère l'application $f(x) = \int_0^\infty \frac{\sin(t)}{t} \e^{-xt} \d t$.
    \begin{enumerate}
        \item Montrer que $f \in \mathscr{C}^1(\Rpe)$.
        \item En déduire une forme explicite de $f$ sur $\Rpe$. 
        \item Montrer que $f$ est continue à l'origine. 
        \item En déduire que $\int_0^\infty \frac{\sin(t)}{t} \d t = \frac{\pi }{2}$.
    \end{enumerate}
\end{exercice}
\end{comment}

\begin{comment}
\begin{exercice}
Soit la transformée de \textsc{Laplace} de la fonction sinus cardinal:
$$F:x \to \int_{0}^{+ \infty} \exp(-xt) \frac{\sin (t)}{t} \d t$$
    
\begin{enumerate}
    \item \emph{Montrer que $F$ est définie sur $\Rp$.}
    \item \emph{Calculer $F$ sur $\Rpe$, en déduire la valeur de la fonction de \textsc{Dirichlet}}
\end{enumerate}
\end{exercice}
\begin{preuve}
\begin{enumerate}
\item Montrons que $F$ est bien définie. D'une part, $\lim_{t \to 0} \exp(-x t) \frac{\sin t}{t} = 1$ donc l'intégrande est prolongeable par continuité en $0$ et est donc intégrable sur $]0, 1]$.
\begin{enumerate}
        \item Si $x > 0$, majorer l'intégrande par $t \mapsto \exp(-xt)$.
        \item Si $x = 0$, montrer le prolongement par continuité de la fonction sinus cardinal en $0$ puis intégrer la fonction sinus cardinal par parties sur $[1, +\infty]$.
    \end{enumerate}
\end{enumerate}
\end{preuve}
\end{comment}

%---------------

% \todoinline{J'ai supprimé la première question de l'exercice qui n'est qu'une redite de la première partie de ce thème. Je détaille aussi les questions pour être plus facile.}

% \todoinline{Mettre au propre la citation ?}

\begin{exercice}
\source{{RMS - Autres Écoles - 178}{IMT}{16}}
On note $I = \int_0^{+\infty} \frac{\sin t}{t} \d t$ et $F(x) = \int_0^{+\infty} \frac{\sin t}{t} (1 - \e^{-x t}) \d t$. On pose $\fonctionligne[u]{t}{\int_0^x \frac{\sin t}{t} \d t}$.
\begin{questions}
% \item Montrer que l'intégrale $I$ est bien définie.
\item Montrer que la fonction $F$ est bien définie sur $\interfo{0}{+\infty}$.

\item Montrer que $F$ est continue sur $\interfo{0}{+\infty}$.\\
\emph{Indication :} Utiliser une intégration par parties avec la fonction $u$.

\item Montrer que $F$ est dérivable sur $\interoo{0}{+\infty}$.

\item En déduire la valeur de $I$.
\end{questions}
\end{exercice}

\begin{solution}
\marginnote[0cm]{Reformulation de la démonstration \ref{demo2existenceIntegraleDirichlet}}
\begin{reponses}
% \item Les fonctions $\fonctionligne[u]{t}{1 - \cos(t)}$ et $\fonctionligne[v]{t}{\frac{1}{t}}$ sont de classe $\mathscr{C}^1$ sur $\Rpe$. De plus, le produit $u v$ a des limites nulles en $0$ et $+\infty$. Ainsi, d'après la formule d'intégration par parties, l'intégrale $I$ a même nature que
% \[
% \int_0^{+\infty} \frac{1 - \cos t}{t^2} \d t,
% \]
% dont l'intégrande est continue sur $\interoo{0}{+\infty}$, prolongeable par continuité par $\frac{1}{2}$ en $0$ et majorée par la fonction $t \mapsto \frac{1}{t^2}$ en $+\infty$, donc est intégrable.

\item 
\begin{enumerate}
\item D'une part, $F(0) = 0$. D'autre part, pour $x > 0$, l'intégrale $F(x)$ est la différence entre $I$ et l'intégrale de $\fonctionligne[f]{(x, t)}{\frac{\sin t}{t} \e^{- x t}}$.
\begin{enumerate}[label=(\roman*)]
\item $f(x, \cdot\,)$ est continue sur $\Rpe$.
\item $f(x, \cdot\,)$ admet pour limite $1$ en $0$.
\item $\abs{f(x, \cdot\,)}$ est un $o(1/t^2)$ en $+\infty$.
\end{enumerate}
Ainsi, l'intégrale $F$ est bien définie sur $\Rp$.

\item Pour la continuité, on ne peut pas espérer appliquer le \theoremeutilise{théorème de convergence dominée}{theo:convergencedominee} directement car le sinus cardinal $t \mapsto \frac{\sin t}{t}$ n'est pas intégrable. On utilise donc une intégration par parties en considérant les fonctions $\fonctionligne[u]{t}{\int_t^{+\infty} \frac{\sin u}{u} \d u}$ et $\fonctionligne[v]{t}{(1 - \e^{-x t})}$. Comme le produit $u\,v$ admet des limites nulles en~$0$ et $+\infty$, 
\begin{align*}
F(x) &= -x \int_{0}^{+\infty} u(t)\, \e^{- x t} \d t \\
&= -\int_0^{+\infty} u\mathopen{}\left(\frac{v}{x}\right) \e^{-v} \d v.
\end{align*}
Comme $u$ possède des limites en $0$ et $+\infty$, elle est bornée sur $\Rp$ et
\[
\abs{u\mathopen{}\left(\frac{v}{x}\right) \e^{-v}} \leqslant M \, \e^{-v},
\]
qui est une fonction intégrable. Ainsi, d'après le \theoremeutilise{théorème de continuité sous le signe intégral}{theo:continuitesoussigneintegral}, la fonction $F$ est continue sur $\Rp$. En particulier,
\[
\lim_{x\to0} F(x) = -\int_0^{+\infty} \lim_{t\to+\infty} u(t) \, \e^{-v} \d v = 0.
\]
Ainsi, $F$ est continue en $0$.

\item Avec les notations de la question précédente,
\[
\forall x \in \interfo{a}{+\infty},\quad \abs{\frac{\partial f}{\partial x}(x, t)} \leqslant \e^{-a t}.
\]
Ainsi, la fonction $F$ est de classe $\Cont^1$ sur $\interfo{a}{+\infty}$, donc sur $\Rpe$.
\end{enumerate}

De plus, pour tout $x > 0$,
\[
F'(x) = \int_0^{+\infty} \sin(t)\, \e^{-x t} \d t = \frac{1}{1 + x^2}.
\]

\item D'après la question précédente, il existe une constante $C \in \R$ telle que
\[
\forall x > 0,\quad F(x) = C + \arctan(x).
\]

Comme $F(0) = 0$, et par continuité de la fonction $F$, alors $C = 0$.

Enfin, $F(x) = I - \int_0^{+\infty} \frac{\sin t}{t}\, \e^{-x t} \d t$ et
\[
\abs{\int_0^{+\infty} \frac{\sin t}{t}\, \e^{-x t} \d t} \leqslant \int_0^{+\infty} \e^{-x t} \d t = \frac{1}{x}.
\]
Ainsi, $\lim\limits_{x\to+\infty} F(x) = I$ et $I = \frac{\pi}{2}$.
\end{reponses}
\end{solution}

%-----------
\subsection{Une preuve par équations différentielles}

\source{Exercice 170 p. 184}
\begin{exercice}
    Soient $\fonctionligne[f]{x}{\displaystyle \int_0^{+\infty} \frac{\sin(t)}{t+x} \d t}$ et $\fonctionligne[g]{x}{\displaystyle \int_0^{+\infty} \frac{\e^{-xt}}{t^2 + 1} \d t}$. 
    \begin{questions}
        \item Montrer que la fonction $g$ est de classe $\Cont^2$ sur $\Re$ et solution de l'équation différentielle $y'' + y = \frac{1}{x}$.
        \item Montrer que la fonction $f$ est aussi de classe $\Cont^2$ sur $\Re$ et solution de l'équation différentielle $y'' + y = \frac{1}{x}$.\\
        \emph{Indication : } On pourra commencer par montrer que $\displaystyle f(x) = \int_0^{+\infty} \frac{1 - \cos(t)}{(t+x)^2} \d t$.
        \item En déduire que la fonction $f-g$ est $2 \pi$-périodique (sur son domaine de définition).
        \item Montrer que les fonctions $f$ et $g$ tendent vers $0$ en $+\infty$, puis que $f = g$.
        \item En déduire la valeur de l'intégrale de \nom{Dirichlet} $\displaystyle \int_0^{+\infty} \frac{\sin(t)}{t} \d t$.
    \end{questions}
\end{exercice}

\begin{solution}
\begin{reponses}
\item On pose $\fonctionligne[G]{(x, t)}{\frac{\e^{-x y}}{t^2 + 1}}$. Alors, pour tout $(x, t) \in \interfo{a}{+\infty} \times \Rp$,
\begin{align*}
\abs{G(x, t)} \leqslant \frac{1}{1 + t^2}, 
\quad \text{puis} \quad
\abs{\frac{\partial G}{\partial x}(x, t)} \leqslant \frac{t \, \e^{-a t}}{1 + t^2},
\quad \text{et enfin} \quad
\abs{\frac{\partial^2 G}{\partial x^2}(x, t)} \leqslant \frac{t^2 \, \e^{-a t}}{1 + t^2}.
\end{align*}
Ainsi, d'après le \theoremeutilise{théorème de dérivation sous le signe intégral}{theo:derivationsoussigneintegrale}, la fonction $g$ est de classe $\Cont^2$ sur $\Rpe$ et
\begin{align*}
g''(x)
&= \int_0^{+\infty} \frac{t^2 \, \e^{-x t}}{1 + t^2} \d t 
= \int_0^{+\infty} \frac{\big(1 + t^2 - 1\big) \e^{-x t}}{1 + t^2} \d t\\
&= \int_0^{+\infty} \e^{-x t} \d t - g(x)\\
&= \frac{1}{x} - g(x).
\end{align*}
\item En utilisant l'intégration par parties vue au début du chapitre,
\begin{align*}
f(x)
&= \left[\frac{1 - \cos(t)}{t + x}\right]_0^{+\infty} + \int_0^{+\infty} \frac{1 - \cos(t)}{(t + x)^2} \d t\\
&= \int_0^{+\infty} \frac{1 - \cos(t)}{(t + x)^2} \d t.
\end{align*}
En posant $\fonctionligne[h]{(x, y)}{\frac{1 - \cos(t)}{(t + x)^2}}$ pour tout $(x, t) \in \interfo{a}{+\infty} \times \Rp$,
\begin{align*}
\abs{h(x, t)} \leqslant \frac{1}{(t + a)^2},
\quad 
\abs{\frac{\partial h}{\partial x}(x, t)} \leqslant \frac{2}{(t + a)^3},
\quad
\abs{\frac{\partial^2 h}{\partial x^2}(x, t)} \leqslant \frac{6}{(t + a)^3}.
\end{align*}
Ainsi, la fonction $f$ est de classe $\Cont^2$ sur $\Rpe$ et, à l'aide d'intégrations par parties,
\begin{align*}
f''(x)
&= \int_0^{+\infty} \frac{6 \, (1 - \cos t)}{(t + x)^4} \d t\\
&= \int_0^{+\infty} \frac{2 \, \sin t}{(t + x)^3} \d t\\
&= \int_0^{+\infty} \frac{\cos t}{(t + x)^2} \d t\\
&= \int_0^{+\infty} \frac{\d t}{(t + x)^2} - \int_0^{+\infty} \frac{1 - \cos t}{(t + x)^2} \d t\\
&= \frac{1}{x} - f(x).
\end{align*}

\item Comme la fonction $f - g$ est solution de l'équation différentielle $y'' - y = 0$, alors $f - g$ est $2\pi$-périodique.

\item En utilisant l'expression précédente,
\begin{align*}
f(x)
&= \frac{1}{x} - \int_0^{+\infty} \frac{2 \sin t}{(t + x)^3} \d t.
\end{align*}
Or, $\displaystyle \abs{\int_0^{+\infty} \frac{2 \sin t}{(t + x)^3} \d t} \leqslant \frac{2}{x^2}$. Ainsi, $f(x) \sim_{+\infty} \frac{1}{x}$.

De même, en $+\infty$,
\begin{align*}
\abs{g(x)}
&\leqslant \int_0^{+\infty} \e^{- x t} \d t
\leqslant \frac{1}{x}.
\end{align*}

D'après les points précédents, $\lim_{x\to+\infty} (f - g) = 0$. Comme $f - g$ est périodique, alors $f - g$ est la fonction nulle.

Ainsi,
\[
\forall\, x > 0,\quad \int_0^{+\infty} \frac{\sin t}{t + x} \d t = \int_0^{+\infty} \frac{\e^{-x t}}{t^2 + 1} \d t.
\]

\item Comme $\abs{\frac{\e^{-x t}}{t^2 + 1}} \leqslant \frac{1}{1 + t^2}$, d'après le \theoremeutilise{théorème de convergence dominée}{theo:convergencedominee},
\[
\lim_{x\to 0} \int_0^{+\infty} \frac{\e^{-x t}}{t^2 + 1} \d t
= \int_0^{+\infty} \frac{\d t}{1 + t^2}
= \frac{\pi}{2}.
\]

De plus,
\begin{align*}
\abs{\int_0^{+\infty} \frac{\sin t}{t + x} \d t - \int_0^{+\infty} \frac{\sin t}{t} \d t}
\leqslant \int_0^1 \frac{x \abs{\sin t}}{t (t + x)} \d t + \int_1^{+\infty} \frac{x \abs{\sin t}}{t (t + x)} \d t.
\end{align*}

Le premier terme tend vers $0$ par convergence dominée, avec la domination $t \mapsto \frac{\abs{\sin t}}{t}$ qui est bien intégrable sur $\interff{0}{1}$.

Le second terme tend vers $0$ par domination par $x \int_1^{+\infty} \frac{\d t}{t^2}$.

Finalement, on obtient
\[
\int_0^{+\infty} \frac{\sin t}{t} \d t
= \int_0^{+\infty} \frac{1}{1 + t^2} \d t
= \frac{\pi}{2}.
\]
\end{reponses}
\end{solution}


%-----------
\subsection{Pour aller plus loin}

À l'aide d'intégrations par parties, on peut montrer que \source{\cite{ensai_1_mp_1996}}
\[
\int_0^{+\infty} \sinc(t) \d t
= \int_0^{+\infty} \sinc^2(t) \d t
= \int_0^{+\infty} \sinc^4(t) \d t
= \frac{\pi}{2}.
\]

Plus généralement, les intégrales de \nom{Borwein} généralisent en un certain sens l'intégrale de \nom{Dirichlet}. En utilisant des calculs utilisant la tranformation de \nom{Fourier}, on peut montrer que (voir~\cite{rioul}) :
\begin{align*}
\int_0^{\pi/2} \sinc(t) \d t &= \pi\\
\int_0^{\pi/2} \sinc(t) \sinc(t/3) \d t &= \pi\\
\int_0^{\pi/2} \sinc(t) \sinc(t/3) \sinc(t/5) \d t &= \pi\\
\int_0^{\pi/2} \sinc(t) \sinc(t/3) \sinc(t/5) \sinc(t/7) \d t &= \pi\\
\int_0^{\pi/2} \sinc(t) \sinc(t/3) \sinc(t/5) \sinc(t/7) \sinc(t/9) \d t &= \pi\\
\int_0^{\pi/2} \sinc(t) \sinc(t/3) \sinc(t/5) \sinc(t/7) \sinc(t/9) \sinc(t/11) \d t &= \pi\\
\int_0^{\pi/2} \sinc(t) \sinc(t/3) \sinc(t/5) \sinc(t/7) \sinc(t/9) \sinc(t/11) \sinc(t/13) \d t &= \pi,
\end{align*}
cette série s'arrêtant ici ! 
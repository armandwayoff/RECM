\section{Permutation somme/intégrale}

\todoinline{Je pense que l'intérêt de l'écriture sur la somme est d'avoir une approximation de l'intégrale. On tente une illustration de cette rapidité de convergence ?}

\begin{prop}{}
\[
\frac{1}{2} \int_{0}^{+ \infty} \frac{\cos (xt)}{\ch t} \d t = \sum_{n=0}^{+ \infty} \frac{(-1)^n (2n+1)}{(2n+1)^2 + x^2}.
\]
\end{prop}

\begin{exercice}
On pose $f : t \mapsto \frac{\cos(x t)}{\e^t + \e^{-t}}$ et $f_n : t \mapsto (-1)^n \cos(x t) \e^{-(n + 1) t}$.
\begin{questions}
\item Justifier la convergence de $\displaystyle\frac{1}{2} \int_{0}^{+ \infty} \frac{\cos (xt)}{\ch t} \d t$.

\item Montrer que $f(x) = \sum\limits_{n=0}^{+\infty} (-1)^n \cos(x t) \e^{-(2 n + 1) t}$.

\item Déterminer $\displaystyle\int_0^{+\infty} f_n(t) \d t$.

\item Conclure en utilisant le théorème des séries alternées.
\end{questions}
\end{exercice}

\begin{elemsolution}
\begin{reponses}
\item La fonction $f \colon t \mapsto \frac{\cos(x t)}{2 \ch(t)}$ est continue sur $[0, +\infty[$. De plus, $|f(t)| \leq \frac{1}{\ch(t)}$ et $\frac{1}{\ch(t)} \sim_{+\infty} 2 \e^{-t}$. Ainsi, la fonction $f$ est intégrable sur $[0, +\infty[$ et l'intégrale converge.

\item En utilisant le développement en série entière de la fonction inverse, comme $0 < \e^{-2t} < 1$ sur $]0, +\infty[$, 
\begin{align*}
f(t)
&= \frac{\cos(x t)}{\e^t + \e^{-t}}
= \e^{-t} \cos(x t) (1 + \e^{-2t})^{-1}\\
&= \e^{-t} \cos(x t) \sum_{n=0}^{+\infty} (-1)^n \e^{-2 n t}\\
&= \sum_{n=0}^{+\infty} (-1)^n \cos(x t) \e^{- (2 n + 1) t}.
\end{align*}

\item En utilisant la formule d'Euler,
\begin{align*}
(-1)^n \int_0^{+\infty} f_n(t) \d t
&= \Reel\left(\int_0^{+\infty} \e^{(\i x - (2 n + 1)) t} \d t\right)\\
&= \Reel\left(-\frac{1}{\i x - (2 n + 1)}\right)\\
&= \frac{2 n + 1}{x^2 + (2 n + 1)^2}.
\end{align*}

\item Remarquons que la série de terme général $(-1)^n \e^{- (2 n + 1) t}$ est une série alternée. Ainsi,
\begin{align*}
\left|\sum_{n=N+1}^{+\infty} (-1)^n \e^{-(2 n + 1) t}\right|
&\leq \e^{-(2 N + 1) t}.
\end{align*}
Alors,
\begin{align*}
\module{\sum_{n=0}^N f_n(t)}
&\leq \module{f(t) - \sum_{n=0}^N f_n(t)} + \module{f(t)}
= \module{\sum_{n=N+1}^{+\infty} f_n(t)} + \module{f(t)}
&\leq |f(t)| + \e^{-(2N+1) t} \leq \module{f(t)} + \e^{-t}.
\end{align*}

Ainsi, $\left(\sum\limits_{n=0}^N f_n\right)_{n\in\N}$ converge simplement vers $f$ et ses sommes partielles sont dominées par une fonction intégrable. Donc, d'après le théorème de convergence dominée,
\begin{align*}
\int_0^{+\infty} f(t) \d t
&= \lim_{N\to+\infty} \sum_{n=0}^N \int_0^{+\infty} f_n(t) \d t\\
\int_0^{+\infty} \frac{\cos(x t)}{2 \ch(t)} \d t
&= \sum_{n=0}^{+\infty} \frac{(-1)^n (2 n + 1)}{x^2 + (2 n + 1)^2}.
\end{align*}
\end{reponses}
\end{elemsolution}
Exercices 51, 52, 53 de \url{http://vonbuhren.free.fr/Prepa/Colles/integration_intervalle_quelconque.pdf}

\begin{exercice}
    Soit $(a, b) \in \R^2$ avec $0<a<b$. Montrer que les intégrales suivantes sont convergentes et calculer leur valeur.
    \[
    I = \int_0^{+\infty} \frac{\e^{-ax} - \e^{-bx}}{x} \d x, \qquad J = \int_0^{+\infty} \frac{\arctan(bx) - \arctan(ax)}{x} \d x.
    \]
\end{exercice}

\begin{exercice}
    Soit $\fonctionens{\interoo{0}{+\infty}}{\R}$ une fonction continue admettant une limite finie $L$ en $+\infty$ et une limite finie $\ell$ en $0$. On considère un couple $(a,b) \in \R^2$ vérifiant $0<a<b$ et les intégrales
    \[
    I = \int_0^{+\infty} \frac{f(at) - f(bt)}{t} \d t \quad \text{et} \quad J = \int_0^1 \frac{t-1}{\ln(t)} \d t.
    \]
    \begin{enumerate}
        \item Montrer que l'intégrale $I$ est convergente et calculer sa valeur. 
        \item En déduire que l'intégrale $J$ converge et calculer sa valeur. 
    \end{enumerate}
\end{exercice}

\begin{exercice}
    Soient $\fonctionens{\interof{0}{+\infty}}{\R}$ une fonction continue, un couple $(a, b) \in \R^2$ avec $0<a<b$ et les intégrales
    \[
    I = \int_1^{+\infty} \frac{f(t)}{t} \d t \quad \text{et} \quad J = \int_0^{+\infty} \frac{f(at)-f(bt)}{t} \d t.
    \]
    On suppose que l'intégrale $I$ est convergente. 
    \begin{enumerate}
        \item Montrer que pour tout $x \in \Rpe$, on a 
        \[
        \int_x^{+\infty} \frac{f(at) - f(bt)}{t} \d t = \int_{ax}^{bx} \frac{f(t)}{t} \d t.
        \]
        \item En déduire que l'intégrale $J$ converge et calculer sa valeur.
    \end{enumerate}
\end{exercice}
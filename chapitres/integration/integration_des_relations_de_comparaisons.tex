\begin{prop}
    Soit $f: \Rp \rightarrow \C$ une fonction continue par morceaux et $g, h:\Rp \rightarrow \Rp$ deux fonctions continues par morceaux, strictement positives. On suppose que $f = o_{+\infty}(g)$ et $f \sim_{+\infty} h$.\\
    Si $g$ et $h$ \textbf{ne sont pas intégrables} sur $\Rp$,
    $$\int_{0}^{x} f = o_{+\infty} \left(\int_{0}^{x} g \right) \text{ et } \int_{0}^{x} f \sim_{+\infty} \int_{0}^{x} h.$$
    Si $g$ et $h$ \textbf{sont intégrables} sur $\Rp$,
    $$\int_{x}^{+\infty} f = o_{+\infty} \left(\int_{x}^{+\infty} g \right) \text{ et } \int_{x}^{+\infty} f \sim_{+\infty} \int_{x}^{+\infty} h.$$
\end{prop} 

La démonstration est analogue à celle de la \nameref{sommation_relations_comparaison}
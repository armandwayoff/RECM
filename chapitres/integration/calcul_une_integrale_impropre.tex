\todoinline{Semble s'appeler l'intégrale d'Euler - \url{https://fr.wikipedia.org/wiki/Table_d'intégrales}}

\begin{exercice}
\cite{Oraux - CCP-PSI-2016}
\begin{enumerate}
\item Montrer que $I$ et $J$ sont convergentes et que $I = J$.
\item Calculer $I + J$ et en déduire $I$ et $J$.
\end{enumerate}
    % Calculer $\displaystyle \int_0^\pi \ln(\sin t) \d t.$
\end{exercice}

\todoinline{Ajouter une représentation graphique de la fonction ?}

\begin{elem_sol}
Soient $I = \int_0^{\pi/2} \ln(\sin(t)) \mathrm{d} t$ et $J = \int_0^{\pi/2} \ln(\cos(t)) \mathrm{d} t$.

\begin{enumerate}
\item La fonction $t \mapsto \ln(\sin(t))$ est continue sur $]0,\pi/2]$. De plus,
\[
\ln(\sin(t)) = \ln(t + o(t)) = \ln(t) + \ln(1 + o(1)) = o(\ln(t)).
\]
Ainsi, $t \mapsto \ln(\sin(t))$ est intégrable en $0$.

La formule de changement de variable, avec $\phi : u \mapsto \pi/2 - u$ assure la convergence de $J$ ainsi que l'égalité $I = J$.

\item Comme ces intégrales sont bien définies, en utilisant la relation de Chasles et la symétrie dans la dernière égalité,
\[
I + J = \int_0^{\pi/2} \ln\left(\frac{\sin(2t)}{2}\right) \mathrm{d} t = \frac{1}{2} \int_0^\pi \ln(\sin(t)) \mathrm{d} t - \frac{\pi}{2} \ln(2) = I - \frac{\pi}{2} \ln(2).
\]
Ainsi, $I = J = -\frac{\pi}{2} \ln(2)$.
\end{enumerate}
\end{elem_sol}

    
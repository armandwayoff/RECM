\documentclass[oneside,11pt]{book}

\ifdefined\correction
  \usepackage[td, nobareme, nosession, noenonce, nosource, correction, exercice]{style}
\else
  \usepackage[td, bareme, enonce, nocorrection, exercice]{style}
\fi
% \usepackage[eleve,poly]{style}

% Raccourcis de langage perso
\usepackage{old_alphabet}

% Permet de ne pas sauter de page entre deux chapitres
\renewcommand{\cleardoublepage}{\newpage}

%%%%%%%%%%%%%%%%%%%%%%%%%%%%%%%%%%%%%%%%%%%%%%%%%%%%%%%%%%%%%%%%%%%
\begin{document}


\begin{exercice}%
\source{RMS 833}%
\session{16}%
\concours{Centrale}%
Pour tout entier naturel $n$, on note $u_n = \dsum_{k=0}^n \frac{1}{\binom{n}{k}}$.
\qu Écrire une fonction \verb?binomial(n, k)? qui renvoie $\binom{n}{k}$. Tracer, pour $n \in \ens{5, 8, 9}$, les points $\left(\binom{n}{k}\right)_{2\leq k \leq n-2}$.

\qu Montrer que, pour tout $2 \leq k \leq n-2$, $\binom{n}{k} \geq \binom{n}{2}$.

\qu Pour tout $n \in \N$, on note $A_n$ le point de coordonnées $(n, u_n)$. Afficher les $31$ premiers termes $A_0,\ldots,A_{30}$. Conjecturer le comportement asymptotique de $(u_n)$.

\qu Démontrer rigoureusement la convergence de $(u_n)$.

Soit $p \geq 2$ et $q \in \N$. On pose $S(p) = \dsum_{n=p}^{+\infty} \frac{1}{\binom{n}{p}}$.

\qu Montrer l'existence de $S(p)$.

\qu On note $S_N = \dsum_{n=p}^N \binom{n}{p}^{-1}$. Tracer $(p-1) S_{200}(p)$ en fonction de $p$ pour $p \in \entiers{2}{50}$.

\qu Exprimer $I(p, q) = \dint_0^1 t^p (1 - t)^q \dt$ en fonction d'un coefficient binomial.

\qu En déduire que $S(p) = \frac{p}{p-1}$.
\end{exercice}

\npagecorr

\begin{solution}
\qu

\qu D'après la définition des coefficients binomiaux,
\[
\frac{\binom{n}{k}}{\binom{n}{2}} = \frac{n (n-1) \cdots (n-k+1) 2}{k (k-1) \cdots 3 \cdot 2 \cdot n (n-1)} \geq 1.
\]

\qu D'après la question précédente,
\[
2 \leq u_{n+1} \leq 2 + \dsum_{k=2}^{n-2} \frac{1}{\binom{n}{2}} \leq 2 + \frac{2(n-3)}{n (n-2)}.
\]
Ainsi, $\dlim_{n\to+\infty} u_n = 2$.

\qu D'après la définition,
\[
\binom{n}{p}^{-1} = \frac{p (p-1) \cdots 2 \cdot 1}{n (n-1) \cdots (n-p+2) (n-p+1)} = o\left(\frac{1}{n^2}\right).
\]
Ainsi, d'après les théorèmes de comparaison des séries à termes positifs, $S(p)$ est bien définie.

\qu À l'aide d'une intégration par parties, $I_{p,q} = \frac{q}{p+1} I_{p+1,q-1}$. De plus, $I_{p,0} = \frac{1}{p+1}$. Ainsi,
\[
I_{p,q} = \frac{q (q-1) \cdots 1}{(p+1) \cdots (p+q)} I_{p+q,0} = \frac{1}{p+q+1} \binom{p+q}{p}^{-1}.
\]

\qu D'après le calcul précédent, si $p \geq 1$,
\begin{align*}
S_N(p) &= \dsum_{n=0}^{N} \binom{n+p}{p}^{-1} = \dsum_{n=0}^{N} (n+p+1) I_{n,p} \\
&= \dint_0^1 \dsum_{n=0}^N (n+1) t^n (1 -t)^p \dt + \dint_0^1 \dsum_{n=0}^N p t^n (1 - t)^p \dt \\
&\to \dint_0^1 (1 -t)^{p-2} \dt + \dint_0^1 p (1 - t)^{p-1} \dt 
= \frac{1}{p-1} + p \frac{1}{p} = \frac{p}{p-1},
\end{align*}
où le théorème de convergence dominée a pu être appliqué car toutes les sommes partielles sont croissantes et leur limite est intégrable car continue.

% \gras{Remarque.} On peut obtenir directement :
% \[
% \binom{n}{p}^{-1} = \binom{n-1}{p-1}^{-1} - \frac{n-p}{n-p+1} \binom{n}{p-1}^{-1}.
% \]
% Ainsi,
% \begin{align*}
% S(p) &= \dsum_{n=0}^{+\infty} \binom{n+p}{n}^{-1} = 1 + \dsum_{n=1}^{+\infty} \binom{n+p-1}{n-1}^{-1} - \frac{p}{p+1} \dsum_{n=1}^{+\infty} \binom{n+p}{p-1}^{-1} \\
% &= 1 + S(p) - \frac{p}{p+1} S(p+1).
% \end{align*}
\end{solution}
\end{document}
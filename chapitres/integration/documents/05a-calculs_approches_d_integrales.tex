\documentclass[oneside,11pt]{book}

\ifdefined\correction
  \usepackage[td, nobareme, nosession, noenonce, nosource, correction, exercice]{style}
\else
  \usepackage[td, bareme, enonce, nocorrection, exercice]{style}. Calculer :
\fi
% \usepackage[eleve,poly]{style}

% Raccourcis de langage perso
\usepackage{old_alphabet}

% Permet de ne pas sauter de page entre deux chapitres
\renewcommand{\cleardoublepage}{\newpage}

%%%%%%%%%%%%%%%%%%%%%%%%%%%%%%%%%%%%%%%%%%%%%%%%%%%%%%%%%%%%%%%%%%%
\begin{document}

\titre{Calculs approchés d'intégrales}
\entete
\begin{refsection}
%%%%%%%

\begin{probleme*}
Dans tout le problème, $a$ et $b$ désignent deux réels tels que $a < b$. Pour tout entier naturel $p$ non nul, on note $(x_i)_{i\in\entiers{0}{p}}$ la subdivision régulière de $[a, b]$ de pas $\frac{b-a}{p}$. Ainsi, pour tout $i \in \entiers{0}{p}$,
\[
x_i = a + i \frac{b-a}{p}.
\]

Une méthode d'intégration est d'\defi{ordre} au moins $n$ si elle est exacte sur les polynômes de degrés inférieurs ou égaux $n$ et non exacte pour au moins un polynôme de degré $n+1$.


%========
\partie{Méthode des rectangles à gauche}


La méthode composée des rectangles à gauche consiste à découper le segment $[a, b]$ en $p$ sous-segments puis à appliquer la méthode simple des rectangles à gauche sur chacune de ces subdivisions :
\begin{colonnes}{2}
\[
I_p^g(f) = \frac{b-a}{p} \sum_{i=0}^{p-1} f(x_i).
\]
%
\columnbreak
%
\begin{center}
  \scalebox{0.3}{\input{figures/approx_rect_g.pdf_t}}
\end{center}
\end{colonnes}

Dans toute cette partie, $f$ désigne une fonction de classe $\CC^1$ sur le segment $[a, b]$. On note $F$ une primitive de $f$ et $M_1 = \dsup_{[a,b]} \abs{f'}$.
\qu Montrer que, pour tout $i \in \entiers{0}{p-1}$,
\[
\abs{F(x_{i+1}) - F(x_i) - (x_{i+1} - x_i) F'(x_i)} \leq \frac{M_1}{2} (x_{i+1}-x_i)^2.
\]


\qu En déduire que
\[
\abs{\int_{[a,b]} f - I_p^g(f)} \leq \frac{M_1 (b-a)^2}{2 p}.
\]

\qu Montrer que cette borne est atteinte pour $f : x \mapsto x - a$.

\qu Montrer que la méthode des rectangles à gauche est d'ordre $0$.


% Par ailleurs, lorsque $g~:~x \mapsto x - a$,
% \begin{align*}
% \int_{[a,b]} g - I_p^g(g) &= \int_a^b (x - a) \dx - \frac{b-a}{p} \sum_{i=0}^{p-1} \left(a + i \frac{b-a}{p} - a\right) \\
    % &= \frac{(b-a)^2}{2} - \frac{(b-a)^2}{p^2} \cdot \frac{(p-1) p}{2} \\
    % &= \frac{(b-a)^2}{2p}.
% \end{align*}
% \end{proof}
  

%=======
\partie{Méthode des rectangles médians}

La méthode composée des rectangles médians consiste à découper le segment $[a, b]$ en $p$ sous-segments puis à appliquer la méthode simple des rectangles médians sur chacune de ces subdivisions :
\begin{colonnes}{2}
\[
I_p^m(f) = \frac{b-a}{p} \sum_{i=0}^{p-1} f\left(\frac{x_i + x_{i+1}}{2} \right).
\]
%
\columnbreak
%
\begin{center}
\scalebox{0.3}{\input{figures/approx_rect_m.pdf_t}}
\end{center}
\end{colonnes}

Dans toute cette partie, $f$ désigne une fonction de classe $\CC^2$ sur le segment $[a, b]$. On note $F$ une primitive de $f$ et $M_2 = \dsup_{[a,b]} \abs{f''}$. Pour tout entier $i \in \entiers{0}{p-1}$, on pose $\gag_i = \frac{x_i + x_{i+1}}{2}$

\qu Soit $i \in \entiers{0}{p-1}$.
\squ Montrer que
\[
(x_{i+1} - x_i) f(\gag_i) = \dint_{x_i}^{x_{i+1}} \left(f(\gag_i) + (t - \gag_i) f'(\gag_i) \right) \dt.
\]

\squ En déduire que
\[
\abs{F(x_{i+1}) - F(x_i) - (x_{i+1} - x_i) F'(\gag_i)}
\leq \frac{M_2}{24} (x_{i+1} - x_i)^3.
\]

\qu Montrer que
\[
\abs{\int_{[a,b]} f - I_p^m(f)} \leq \frac{M_2 (b-a)^3}{24 p^2}
\]

\qu Montrer que cette borne est atteinte pour $f : x \mapsto (x - a)^2$.

\qu Montrer que la méthode des rectangles médians est d'ordre $1$.


%======
\partie{Méthode des trapèzes}

La méthode composée des trapèzes consiste à découper le segment $[a, b]$ en $p$ sous-segments puis à appliquer la méthode simple des trapèzes sur chacune de ces subdivisions :
\begin{colonnes}{2}
\[
I_p^t(f) =  \frac{b-a}{p} \sum_{i=0}^{p-1} \frac{f(x_i) + f(x_{i+1})}{2}.
\]
%
\columnbreak
%
\begin{center}
\scalebox{0.3}{\input{figures/approx_trap.pdf_t}}
\end{center}
\end{colonnes}

On suppose dans cette partie que $f$ est une fonction de classe $\CC^2$ et on note $M_2 = \dsup_{[a,b]} \abs{f''}$. Pour tout $i \in \entiers{0}{p-1}$, on note $\phi_i$ l'approximation affine sur $[x_i, x_{i+1}]$ de $f$ et $g_i = f - \phi_i$.

\qu À l'aide d'intégrations par parties, montrer que, pour tout $i \in \entiers{0}{p-1}$,
\[
\dint_{x_i}^{x_{i+1}} f''(t) (t - x_i) (x_{i+1} - t) \dt = - 2 \dint_{x_i}^{x_{i+1}} g_i(t) \dt.
\]

\qu En déduire que
\[
\abs{\int_{[a,b]} f - I_p^t(f)} \leq \frac{M_2 (b-a)^3}{12 p^2}.
\]

\qu Montrer que cette borne est atteinte pour $f : x \mapsto (x - a)^2$.

\qu Montrer que la méthode des trapèzes est d'ordre $1$.

\qu Montrer que, lorsque $f'' \geq 0$ (en particulier lorsque $f$ est convexe), pour tout entier naturel $p$, $\dint_{[a,b]} f \leq I_p^t(f)$.

\sautdepage

%=======
\partie{Méthode de \cite{Simpson}}

La méthode composée de Simpson consiste à découper le segment $[a, b]$ en $p$ sous-segments puis à approcher, sur chacune de ces subdivisions, la fonction $f$ par un polynôme de degré inférieur ou égal à $2$ :
\[
I_p^s(f) = \frac{b-a}{6 p} \sum_{i=0}^{p-1} \left[f(x_i)+ 4 f\left(\frac{x_i + x_{i+1}}{2}\right) + f(x_{i+1})\right].
\]

\qu Soit $g \in \CC^5([a, b], \R)$ une fonction impaire. On note $K_5 = \dsup_{[a,b]} \abs{g^{(5)}}$. En utilisant la formule de \cite{Taylor} avec reste intégral pour $g$ et $g'$, montrer que
\[
\abs{g(x) - \frac{x}{3} (g'(x) + 2 g'(0))} \leq \frac{K_5}{180} \abs{x}^5.
\]

On suppose dans cette partie que $f$ est une fonction de classe $\CC^4$ sur le segment $[a, b]$. On pose $M_4 = \dsup_{[a,b]} \abs{f^{(4)}}$.

\qu Montrer que, pour tout $i \in \entiers{0}{p-1}$,
\[
\abs{F(x_{i+1}) - F(x_i) - \frac{1}{6p} \left[ f(x_{i+1}) + f(x_i) + 4 f\left(\frac{x_{i+1}+x_i}{2}\right) \right]}
\leq
\frac{M_4 (x_{i+1} - x_i)^5}{2880}.
\]

\indic{Poser $g : t \mapsto F\left(\frac{x_{i+1}+x_i}{2}+t\right) - F\left(\frac{x_{i+1}+x_i}{2}-t\right)$.}

\qu En déduire que
\[
\abs{I_p^s(f) - \dint_a^b f(t) \dt} \leq \frac{M_4 (b-a)^5}{2880 p^4}.
\]

On peut montrer que la méthode de Simpson est d'ordre $3$. On peut augmenter le nombre des n\oe{}uds où est évaluée la fonction à intégrer ($2$ n\oe{}uds pour la méthode des trapèzes, $3$ pour la méthode de Simpson,\ldots). Ces méthodes sont appelées \defi{méthodes de \cite{Newton}-\cite{Cotes}}. Cependant, lorsque le nombre de n\oe{}uds dépasse $8$, des coefficients négatifs apparaissent ce qui engendre des erreurs d'arrondis. 
\end{probleme*}

\printbibliography[title={Mathématiciens}, heading=subbibliography]
\end{refsection}
\end{document}

\documentclass[oneside,11pt]{book}

\ifdefined\correction
  \usepackage[enonce, correction, td, poly]{style}
\else
  \usepackage[enonce, nosession, nocorrection, nosource, noconcours, eleve, poly]{style}
\fi
% \usepackage[eleve,poly]{style}

% Raccourcis de langage perso
\usepackage{old_alphabet}



%%%%%%%%%%%%%%%%%%%%%%%%%%%%%%%%%%%%%%%%%%%%%%%%%%%%%%%%%%%%%%%%%%%
\begin{document}
%==========================================
\title{Cours de PSI}
\niveau{PSI}
\author{A. Camanes}
\date{\anneescolaire}
\etablissement{Stanislas}
\frontmatter
% \maketitle
\renewcommand{\labelitemi}{$\ast$}
\renewcommand{\labelitemii}{$\star$}
\renewcommand{\theenumi}{$(\roman{enumi})$}
%==========================================

\setcounter{exercice}{22}

%============
\supplement{Transformée de \cite{Laplace}}

\begin{exercice}
\source{Dunod - p.734 \& 736}
Pour toute fonction $f \in \CC(\R_+,\R)$, on note, lorsqu'elle converge, $\LL(f)(p) = \dint_0^{+\infty} \e^{-p t} f(t) \dt$. La fonction $\LL(f)$ est la transformée de~\cite{Laplace} de $f$.

\qu Soient $\lg \in \C$ et $n \in \N$. Pour chacune des fonctions suivantes, déterminer leur transformée de~\cite{Laplace} en précisant son domaine de définition :
\begin{colonnes}{3}
\squ $t \mapsto 1$.
\squ $t \mapsto \e^{\lg t}$.
\squ $t \mapsto t^n$.
\end{colonnes}

\qu On suppose que $f$ est bornée. Montrer que $\LL(f)$ est définie et de classe $\CC^\infty$ sur $\R_+^*$.

\qu \gras{Théorème de la valeur finale.} On suppose qu'il existe un réel $\ell$ non nul tel que $\dlim_{+\infty} f(x) = \ell$. Déterminer un équivalent de $\LL(f)$ en $0$.

\medskip
On suppose $f$ continue uniquement sur $\R_+^*$ et qu'il existe $p_0 > 0$ tel que, pour pour tout $p > p_0$, $t \mapsto \e^{-p t} f(t)$ est intégrable sur $\R_+$. 

\qu Montrer que $\LL(f)$ est définie et continue sur $]p_0,+\infty[$.

\qu \gras{Théorème de la valeur initiale.} On note $\ell = \dlim_{t\to0^+} f(t)$. Déterminer la limite de $p \mapsto p \LL(f)(p)$ en $+\infty$.
\end{exercice}

\begin{solution}
\Qu $t \mapsto \e^{-pt}$ est intégrable sur $\R_+$ si et seulement si $p > 0$. Alors, $\LL(f)(p) = \dint_0^{+\infty} \e^{-pt} \dt = \frac{1}{p}$.

\squ $t \mapsto \e^{-(\lg-p)t}$ est intégrable sur $\R_+$ si et seulement si $p > \Re{\lg}$. Alors, $\LL(f)(p) = \dint_0^{+\infty} \e^{-(p-\lg)t} \dt = \frac{1}{p-\lg}$.

\squ Soit $n \in \N^*$ et $f_n : t \mapsto t^n \e^{-pt}$.
\begin{itemize}
\item $f_n$ est continue sur $\R_+^*$.
\item Si $p > 0$, alors $f(t) = o_{+\infty}(1/t^2)$ et $f$ est intégrable sur $\R_+^*$.

Si $p \leq 0$, alors $\dlim_{+\infty} f = +\infty$ et $f$ n'est pas intégrable sur $\R_+^*$.
\end{itemize}
Ainsi, $f_n$ est intégrable si et seulement si $p > 0$. Une récurrence classique avec des intégrations par parties permet de montrer que
\[
\LL(f)(p) = \frac{n!}{p^{n+1}}.
\]

\gras{Remarque.} À un changement de variable près, on retrouve la fonction $\Gg$ d'Euler.

\qu La fonction $f$ est bornée par une constante $M$. On utilise le théorème de dérivation sous le signe intégral. Notons $g : (p, t) \mapsto f(t) \e^{-pt}$.
\begin{itemize}
\item Pour tout $t \in \R_+^*$, $g(\cdot, t) : p \mapsto \e^{-p t} f(t)$ est de classe $\CC^\infty$ sur $\R_+^*$ et
\[
\pourtout j \in \N,\, \frac{\partial^j g}{\partial p^j} g(p, t) = (-1)^j t^j f(t) \e^{-pt}.
\]

\item Pour tout $p > 0$, $g(p, \cdot) : t \mapsto (-1)^j t^j \e^{-p t} f(t)$ est continue sur $[0,+\infty[$.

\item Soit $\tilde{p} > 0$. Alors, pour tout $p \geq \tilde{p}$,
\[
\abs{(-1)^j t^j f(t) \e^{-p t}} \leq M t^j \e^{- \tilde{p} t} M.
\]
De plus, $\phi_j : t \mapsto M t^j \e^{-\tilde{p} t}$ est continue sur $\R_+$, et est négligeable devant $1/t^2$ en $+\infty$. Ainsi, $\phi_j$ est intégrable.
\end{itemize}
Ainsi, $\LL(f)$ est de classe $\CC^\infty$ sur $\R_+^*$.

\qu Comme $f$ est continue sur $\R_+$ et possède une limite en $+\infty$, une généralisation classique du théorème des bornes assure que $f$ est bornée sur $\R_+$. Ainsi, d'après la question précédente, $\LL(f)$ est bien définie sur $\R_+^*$.

\smallskip


On effectue le changement de variable affine $\phi : u \mapsto \frac{u}{p}$. Alors,
\[
p \LL(f)(p) = \dint_0^{+\infty} \e^{-u} f\left(\frac{u}{p}\right) \du.
\]
Soit $(p_n)$ une suite de réels strictement positifs qui tend vers $0$ et $g_n : u \mapsto \e^{-u} f(u/p_n)$.
\begin{itemize}
\item Pour tout $n \in \N$, $g_n$ est continue sur $\R_+$.

\item Soit $u > 0$. D'après les hypothèses sur la fonction $f$, $\dlim_{n\to+\infty} \e^{-u} f(u/p_n) = \e^{-u} \ell$. Ainsi, $(g_n)$ converge simplement sur $\R_+^*$ vers $u \mapsto \e^{-u} \ell$.

\item Comme $f$ est bornée par une constante $M$, $\abs{g_n(u)} \leq M \e^{-u}$ et $u \mapsto \e^{-u}$ est intégrable sur $\R_+^*$.
\end{itemize}
Ainsi, d'après le théorème de convergence dominée,
\[
\dlim_{n\to+\infty} p_n \LL(f)(p_n) = \ell \dint_0^{+\infty} \e^{-u} \du.
\]
En utilisant la caractérisation séquentielle de la limite,
\[
\dlim_{p\to 0} p \LL(f)(p) = \ell \text{ soit } \LL(f)(p) \sim_0 \frac{\ell}{p}.
\]

\qu Soit $g : (p, u) \mapsto f(u) \e^{-p u}$.
\begin{itemize}
\item $\pourtout t \in \R_+^*,\, g(\cdot, u)$ est continue sur $]p_0, +\infty[$.

\item $\pourtout p > p_0,\, g(p, \cdot)$ est continue sur $\R_+^*$.

\item Soit $\tilde{p} > p_0$. Pour tout $p \geq \tilde{p}$ et $t \in \R_+^*$,
\[
\abs{\e^{-p t} f(t)} \leq \e^{- \tilde{p} t} \abs{f(t)}.
\]
De plus, $t \mapsto f(t) \e^{-\tilde{p} t}$ est intégrable sur $\R_+^*$ par hypothèse.
\end{itemize}
D'après le théorème de continuité sous le signe intégral, $\LL(f)$ est continue sur $]p_0, +\infty[$.

\qu Si $f$ est bornée, on peut appliquer la même méthode que pour le théorème de la valeur finale.

\smallskip

Sinon, on raisonne à coup de $\eg$. Soit $\eg > 0$. Comme $\dlim_{0^+} f = \ell$, il existe $h > 0$ tel que
\[
\pourtout t \in ]0, h],\, \abs{f(t) - \ell} \leq \eg.
\]
Alors, pour $p > p_0 + 1$ et $\tilde{p} = p_0 + \frac{1}{2}$,
\begin{align*}
\abs{p \LL(f)(p) - \ell} &\leq \dint_0^{+\infty} p \abs{f(t) - \ell} \e^{-p t} \dt \\
&\leq \dint_0^h p \abs{f(t) - \ell} \e^{-pt} \dt + \dint_h^{+\infty} p \abs{f(t) - \ell} \e^{-pt} \dt\\
&\leq \eg \left(1 - \underbrace{\e^{-p h}}_{\geq 0}\right)
+ p \dint_h^{+\infty} \abs{f(t)} \e^{-p t} \dt
+ p \dint_h^{+\infty} \abs{\ell} \e^{- p t} \dt \\
&\leq \eg
+ \dint_h^{+\infty} \abs{f(t)} \e^{-(p-\tilde{p})t - \tilde{p} t} \dt + \abs{\ell} \e^{-p h}\\
&\leq \eg + p \underbrace{\e^{-(p - \tilde{p}) h}}_{\to 0} \underbrace{\abs{\dint_0^{+\infty} \abs{f(t)} \e^{-\tilde{p} t} \dt}}_{cte} + \abs{f(0)} \e^{-ph} \\
&\leq 3 \eg,
\end{align*}
pour $p$ assez grand. Ainsi, d'après la définition de la limite,
\[
\dlim_{p\to +\infty} p \LL(p) = \ell.
\]
\end{solution}
\end{document}
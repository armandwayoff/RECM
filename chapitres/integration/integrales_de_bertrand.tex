\begin{prop}
    Soient $(\alpha, \beta) \in  \R^2$ et $f:t \to \frac{1}{t^{\alpha} \ln^{\beta} (t)}$. Alors,
    $$\int_{2}^{+ \infty} f \text{ converge si et seulement si }
    \begin{cases}
    \alpha > 1 \\
    \text{ou}\\
    \alpha = 1 \text{ et } \beta > 1
    \end{cases}.
    $$
\end{prop}

\textcolor{red}{à rerédiger}
\begin{preuve}
    Distinguons trois cas selon les valeurs prises par $\alpha$:
    \begin{enumerate}
        \item si $\alpha > 1$, soit $\gamma \in ]1, \alpha[$. On peut montrer que $$\displaystyle \frac{1}{t^{\alpha} \ln^{\beta} (t)} = o_{+ \infty} \left( \frac{1}{t^{\gamma}} \right).$$
        \item si $\alpha < 1$, soit $\gamma \in ]\alpha, 1[$. On peut montrer que 
        $ t^{\gamma} f(t) \xrightarrow[t \to + \infty]{} + \infty $
        donc à partir d'un certain rang, $f(t) \geqslant \frac{1}{t^{\gamma}} > 0$.
        \item si $\alpha = 1$, revenir aux intégrales partielles. Connaître la primitive de $t \mapsto \frac{1}{t \ln^{\beta} (t)}$:
        $$\int_{2}^{X} \frac{1}{t \ln^{\beta} (t)} \d t = 
        \begin{cases}
            \left[ \frac{\ln ^{1-\beta} (t)}{1-\beta} \right]_2 ^X & \text{si } \beta \not = 1, \\
            \left[\ln (\ln(t)) \right]_2 ^X & \text{si } \beta = 1.
        \end{cases}
        $$
        On en déduit que l'intégrale de la fonction $t \mapsto \frac{1}{t \ln^{\beta} (t)}$ converge sur $[2, + \infty[$ si et seulement si $\beta > 1$.
    \end{enumerate}
\end{preuve}

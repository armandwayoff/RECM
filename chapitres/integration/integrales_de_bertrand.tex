\begin{theo}{\textsc{Bertrand}}
    Soient $(\alpha, \beta) \in  \R^2$ et 
    $$f:t \mapsto \frac{1}{t^{\alpha} \ln^{\beta} (t)}.$$
    Alors,
    $$\int_{2}^{+ \infty} f \text{ converge si et seulement si }
    \begin{cases}
    \alpha > 1 \\
    \text{ou}\\
    \alpha = 1 \text{ et } \beta > 1
    \end{cases}.
    $$
\end{theo}

\begin{preuve}
    Distinguons trois cas selon les valeurs prises par $\alpha$:
    \begin{enumerate}
        \item[$\rhd$] \textbf{Cas où $\alpha > 1$.} Soit $\gamma \in ]1, \alpha[$. Par croissances comparées,
        $$\displaystyle \frac{1}{t^{\alpha} \ln^{\beta} (t)} = o_{+ \infty} \left( \frac{1}{t^{\gamma}} \right).$$
        Or, d'après le théorème de \textsc{Riemann}, la fonction $t \mapsto \frac{1}{t^\gamma}$ est intégrable sur $[2, +\infty[$ car $\gamma > 1$. Ainsi, en appliquant les théorèmes de comparaison, $\int_2^{+ \infty} f$ converge.
        \item[$\rhd$] \textbf{Cas où $\alpha < 1$.} Soit $\gamma \in ]\alpha, 1[$.Par croissances comparées,
        $$t^{\gamma} f(t) \xrightarrow[t \to + \infty]{} + \infty$$
        donc à partir d'un certain rang, $f(t) \geqslant \frac{1}{t^{\gamma}} > 0$. Or, d'après le théorème de \textsc{Riemann}, la fonction $t \mapsto \frac{1}{t^\gamma}$ n'est intégrable pas sur $[2, +\infty[$ car $\gamma < 1$. Ainsi, en appliquant les théorèmes de comparaison (les intégrandes sont positives), $\int_2^{+ \infty} f$ diverge.
        \item[$\rhd$] \textbf{Cas où $\alpha = 1$.} Revenons aux intégrales partielles: soit $X > 2$,
        $$\int_{2}^{X} \frac{1}{t \ln^{\beta} (t)} \d t = 
        \begin{cases}
            \left[ \frac{\ln ^{1-\beta} (t)}{1-\beta} \right]_2 ^X & \text{si } \beta \not = 1, \\
            \left[\ln (\ln t) \right]_2 ^X & \text{si } \beta = 1.
        \end{cases}
        $$
        On en déduit que l'intégrale de la fonction $t \mapsto \frac{1}{t \ln^{\beta} (t)}$ converge sur $[2, + \infty[$ si et seulement si $\beta > 1$.
    \end{enumerate}
\end{preuve}

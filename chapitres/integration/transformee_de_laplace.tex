\begin{tcolorbox}
    Pour tout fonction $f \in \mathscr{C}(\Rp, \R)$, on note, lorsqu'elle converge, 
    $$\mathscr{L}(f)(p) = \int_{0}^{+ \infty} \me^{-pt} f(t)\ \d t.$$
    La fonction $\mathscr{L}(f)$ est la \emph{transformée de \textsc{Laplace} de f}.
\end{tcolorbox}

\cite{exos_oraux} + \cite{acamanes} (Exercice cerise Ch. 12) \\
\underline{Démonstration du théorème de la valeur finale:}
\begin{itemize}
    \item Généralisation classique du théorème des bornes $\leadsto$ $f$ est bornée
    \item Changement de variable: $\varphi: u \mapsto \frac{u}{p}$
    \item Caractérisation séquentielle de la limite
    \item Théorème de convergence dominée
\end{itemize}

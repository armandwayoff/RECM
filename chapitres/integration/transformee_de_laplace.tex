\begin{defi}
    Pour tout fonction $f \in \mathscr{C}(\Rp, \R)$, on note, lorsqu'elle converge, 
    $$\mathscr{L}(f)(p) \defeq \int_{0}^{+ \infty} \me^{-pt} f(t) \d t.$$
    La fonction $\mathscr{L}(f)$ est la \emph{transformée de \textsc{Laplace} de f}.
\end{defi}

\section{Exercice oral}
\begin{exercice}
    On pose
    $$F:x \mapsto \int_x^{+\infty} \frac{\me^{-t}}{t} \d t.$$
    \begin{enumerate}
        \item Déterminer l'ensemble de définition de $F$. Étudier brièvement le comportement de la fonction $F$ et tracer sa courbe représentative.
        \item Déterminer un  équivalent de $F$ en $+\infty$.
        \item Montrer que $F(1) - F(x) - \ln x$ converge vers un réel. (pas sûr de cette question).
    \end{enumerate}
\end{exercice}

\begin{elem_solution}
    \begin{enumerate}
        \item $D = ]0, + \infty[$.
        \item Intégration par parties ou comparaison série / intégrale: $F(x) \sim_{+\infty} \frac{\me^{x}}{x}$. \\
        \url{https://www.bibmath.net/ressources/index.php?action=affiche&quoi=bde/analyse/integration/integralesimpropres&type=fexo} exercice 38. (à réécrire)\\
        On remarque d'abord que $\int_{1}^{+\infty} \frac{\me^{-t}}{t} \d t$ converge: en effet, la fonction $t \mapsto \frac{\me^{-t}}{t}$ est continue et positive sur $[1, + \infty[$ et $\lim\limits_{t \to +\infty} t^2 \frac{\me^{-t}}{t} = 0$. On intègre ensuite par parties, en intégrant $t \mapsto \me^{-t}$ et en dérivant $\t \mapsto \frac{1}{t}$. On obtient, pour $x > 1$, 
        \begin{align*}
            \int_x^{+ \infty} \frac{\me^{-t}}{t} \d t &= \left[ -\frac{\me^{-t}}{t} \right]_x^{+\infty} - \int_x^{+\infty} \frac{\me^{-t}}{t^2} \d t \\
            &= \frac{\me^{-x}}{x} - \int_x^{+\infty} \frac{\me^{-t}}{t^2} \d t.
        \end{align*}
        Or, au voisinage de $+ \infty$, 
        $$\frac{\me^{-t}}{t^2} = o\left( \frac{\me^{-t}}{t} \right).$$
        Par intégration des relations de comparaison (les fonctions sont positives et intégrables), on trouve
        $$\int_x^{+\infty} \frac{\me^{-t}}{t^2} \d t = o_{+\infty} \left( \int_x^{+\infty} \frac{\me^{-t}}{t} \d t \right).$$
        On en déduit que
        $$\int_x^{+\infty} \frac{\me^{-t}}{t} \d t \sim_{+\infty} \frac{\me^{-x}}{x}.$$
        \item À faire.
    \end{enumerate}
\end{elem_solution}

\cite{exos_oraux} + \cite{acamanes} (Exercice cerise Ch. 12) \\
\underline{Démonstration du théorème de la valeur finale:}
\begin{itemize}
    \item Généralisation classique du théorème des bornes $\leadsto$ $f$ est bornée
    \item Changement de variable: $\varphi: u \mapsto \frac{u}{p}$
    \item Caractérisation séquentielle de la limite
    \item Théorème de convergence dominée
\end{itemize}

\todoinline{Rechercher l'exercice corrigé sur les théorèmes des valeurs initiales / finales.}

La transformation de \textsc{Laplace} généralise la transformation de \textsc{Fourier} qui est également utilisée pour résoudre les équations différentielles : contrairement à cette dernière, elle tient compte des conditions initiales et peut ainsi être utilisée en théorie des vibrations mécaniques ou en électricité dans l'étude des régimes forcés sans négliger le régime transitoire. De manière générale, ses propriétés vis-à-vis de la dérivation permettent un traitement plus simple de certaines équations différentielles, et elle est de ce fait très utilisée en automatique. \\
Dans ce type d'analyse, la transformation de \textsc{Laplace} est souvent interprétée comme un passage du domaine temps, dans lequel les entrées et sorties sont des fonctions du temps, dans le domaine des fréquences, dans lequel les mêmes entrées et sorties sont des fonctions de la \say{ fréquence } (complexe) $p$. Ainsi; il est possible d'analyser simplement l'effet du système sur l'entrée pour donner la sortie en matière d'opérations algébriques simples (cf. théorie des fonctions de transfert en électronique ou en mécanique). 

\begin{defi}{Transformée de \textsc{Laplace}}
    Pour tout fonction $f \in \mathscr{C}(\Rp, \R)$, on note, lorsqu'elle converge, 
    $$\mathscr{L}(f)(p) \defeq \int_{0}^{+ \infty} \e^{-pt} f(t) \d t.$$
    La fonction $\mathscr{L}(f)$ est la \emph{transformée de \textsc{Laplace} de f}.
\end{defi}

\marginnote[0cm]{Sources : \cite{exos_oraux} + \cite{acamanes} (Exercice cerise Ch. 12)}
\underline{Démonstration du théorème de la valeur finale:}
\begin{itemize}
    \item Généralisation classique du théorème des bornes $\leadsto$ $f$ est bornée
    \item Changement de variable: $\varphi: u \mapsto \frac{u}{p}$
    \item Caractérisation séquentielle de la limite
    \item Théorème de convergence dominée
\end{itemize}

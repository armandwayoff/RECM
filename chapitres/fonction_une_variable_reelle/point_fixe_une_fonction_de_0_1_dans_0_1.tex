\begin{exercice}
    Soit $f$ une fonction dérivable sur $[0, 1]$ telle que:
    $$f(0) = f'(0) = f'(1) = 0 \text{ et } f(1) = 1.$$
    Montrer qu'il existe $c \in ]0, 1[$ tel que $f(c) = c$. 
\end{exercice}

\begin{elem_sol}
    \begin{itemize}
        \item Poser $g(x)= f(x) - x$.
        \item L'objectif est de montrer que $g$ s'annule au moins une fois sur $[0, 1]$ en montrant l'existence de $x_0$ et $x_1$ dans $[0, 1]$ tels que $g(x_0) < 0$ et $g(x_1) > 0$ pour pouvoir appliquer le \textbf{théorème des valeurs intermédiaires}.
        \item Raisonner par l'absurde sur l'existence de $x_0$ et de $x_1$ et écrire la dérivée de $f$ comme la limite de son taux d'accroissement pour aboutir à des contradictions. 
    \end{itemize}
\end{elem_sol}

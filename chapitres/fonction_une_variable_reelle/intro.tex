\marginnote[0mm]{Source : \cite{oraux_x_ens_3}}

\textsl{À partir du \textsc{xvii}$^\e$ siècle, le développment du calcul infinitésimal, motivé par de nombreux problèmes de cinématique, de mécanique, ou de calcul des variations, fait de la \say{ fonction } l'objet central des mathématiques modernes, alors que jusque là, le \say{ nombre } était la base de l'édifice mathématique. Nous devons à \textsc{Bernoulli} et \textsc{Leibniz} le terme même de \say{ fonction }: pour \textsc{Bernoulli} (1698), une fonction de la variable $x$ est \say{ une quantité formée d'une manière quelconque à partir de $x$ et de constantes }. L'écriture $y = f(x)$ est introduite par \textsc{Euler} en 1734. Les fonctions sont représentées par des courbes dans le plan et \textsc{Euler} se demande si une courbe donnée correspond toujours à une fonction. C'est lui qui distingue les courbes continues, des courbes discontinues, qui sont le plus souvent, à cette époque, des graphes de fonctions continues par morceaux. Cette double conception des fonctions, comme expressions analytiques ou comme graphes du plan, ne sera pas vraiment éclaircie avant le \textsc{xix}$^\e$ siècle (c'est \textsc{Dirichlet} qui donnera la définition moderne d'une fonction comme correspondance; il proposera ainsi (en 1837) un exemple de fonction discontinue partout, la fonction $\chi$ définie par $\chi(x) = 1$ pour $x$ rationnel et $\chi(x)=0$ pour $x$ irrationnel). \textsc{Lagrange}, cherchant à établir les fondements de l'Analyse, s'en tient au point de vue formel et refuse de se référer à toute notion de limite. Ces hésitations empêchent les mathématiciens du \textsc{xvii}$^\e$ siècle  de mener jusqu'à leur achèvement certains de leurs travaux, comme l'étude de l'équation des cordes vibrantes. C'est la génération suivante, avec entre autres \textsc{Gauss}, \textsc{Cauchy}, \textsc{Bolzano} et \textsc{Abel}, qui donnera dans la première moitié du \textsc{xix}$^\e$ siècle un statut rigoureux aux notions de convergence, de continuité, \dots Quant au concept de limite d'une fonction numérique, on doit sans doute sa première définition précise à \textsc{Weierstrass}. 
}

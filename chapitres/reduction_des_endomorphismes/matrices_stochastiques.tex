\marginnote[2.5cm]{
    $$
    \begin{pmatrix}
    1/2 & 1/2 & 0 \\
    3/4 & 1/8 & 1/8 \\
    0 & 1/3 & 2/3
    \end{pmatrix}
    $$
}

\begin{marginfigure}[3.5cm]
    \centering
    \resizebox{6.5cm}{6.5cm}{%
    \begin{tikzpicture}
        \node[state] (s1) {1};
        \node[state, below right of=s1] (s2) {2};
        \node[state, below left of=s1] (s3) {3};
    
        \draw (s1) edge[loop above] node {$1/2$} (s1);
        \draw (s1) edge[bend left] node {$1/2$} (s2);
        %\draw (s1) edge[bend right, above left] node {0} (s3);
    
        \draw (s2) edge[bend left, above right] node {$3/4$} (s1);
        \draw (s2) edge[loop right] node {$1/8$} (s2);
        \draw (s2) edge[bend right] node {$1/8$} (s3);
    
        %\draw (s3) edge[bend right] node {0} (s1);
        \draw (s3) edge[bend right] node {$1/3$} (s2);
        \draw (s3) edge[loop left] node {$2/3$} (s3);
    \end{tikzpicture}    
}

  %  \begin{tikzpicture}
  %      \node[state] (s1) {État 1};
  %      \node[state, below right of=s1] (s2) {État 2};
   %     \node[state, below left of=s1] (s3) {État 3};
 %  
 %       \draw (s1) edge[loop above] node {$p_{1,1}$} (s1);
%        \draw (s1) edge[bend left] node {$p_{1,2}$} (s2);
%        \draw (s1) edge[bend right, above left] node {$p_{1,3}$} (s3);
    
 %       \draw (s2) edge[bend left, above right] node {$p_{2,1}$} (s1);
%        \draw (s2) edge[loop right] node {$p_{2,2}$} (s2);
%        \draw (s2) edge[bend right] node {$p_{2,3}$} (s3);
    
%        \draw (s3) edge[bend right] node {$p_{3,1}$} (s1);
%        \draw (s3) edge[bend right] node {$p_{3,2}$} (s2);
%        \draw (s3) edge[loop left] node {$p_{3,3}$} (s3);
%    \end{tikzpicture}    
    \caption*{\centering Une chaîne de \textsc{Markov} et sa matrice de transition.}
\end{marginfigure}

\marginnote[0cm]{Source : Texte de \cite{oraux_x_ens_2} p. 59}
Les matrices stochastiques 
% \marginnote[0cm]{\note Emprunté au grec ancien $\sigma \tau \omicron \chi \alpha \sigma \tau \iota \kappa \omicron \zeta$, stokhastikos (\say{ conjectural }) $\sigma \tau \omicron \chi \omicron \zeta$, stokhos (\say{ but, cible, conjecture }). \\ Un phénomène \emph{stochastique} est un phénomène qui ne se prête qu’à une analyse statistique, par opposition à un phénomène déterministe.}
interviennent en probabilités. Si $X$ et $Y$ sont deux variables aléatoires à valeurs dans $E \defeq \llbracket 1, k \rrbracket$, alors la matrice $A \defeq (a_{i,j})_{1 \leqslant i,j \leqslant n} \in \M_k(\R)$ définie par 
$$a_{i,j} \defeq \P(Y = j \,|\, X = i)$$
est stochastique, ce qui par définition signifie qu'on a $a_{i,j} \geqslant 0$ et $\sum\limits_{j=1}^k a_{i,j} = 1$ pour $1 \leqslant i \leqslant k$. \\
L'évolution d'un système susceptible de prendre un nombre fini d'états notés $1, \dots, k$ est représentée mathématiquement par une suite $(X_n)_{n \geqslant 0}$ de variables aléatoires à valeurs dans $E$. C'est ce qu'on appelle un processus aléatoire (ou stochastique). Si $X_{n+1}$ s'obtient à partir de la valeur de $X_n$ et d'un tirage au sort effectué selon une loi ne dépendant que de cette valeur, on dit que le processus est une chaîne de \textsc{Markov}. Les exemples abondent: marches aléatoires, fortune d'un joueur, modélisation de l'alternance des voyelles et des consonnes dans un poème de \textsc{Pouchkine} (par \textsc{Markov} lui-même), ou prévision (en probabilité) des états  successifs d'un signal pour améliorer la compression en traitement du signal (\textsc{Shannon}) \\
Techniquement, on dit qu'une suite de variables aléatoires $(X_n)$ est une chaîne de \textsc{Markov} si \say{ la loi de l'état $n+1$ conditionnelle au passé ne dépend que de l'état antérieur $n$ }, ce qui se traduit par
$$\P(X_{n+1}=j \,|\, X_0=i_0, \dots, X_n = i_n) = \P(X_{n+1} = j \,|\, X_n = i_n).$$
Si la matrice $A$ est indépendante de $n$, on dit que la chaîne de \textsc{Markov} est stationnaire. Si, dans ce dernier cas, on pose $Y_n \defeq \begin{pmatrix} \P(X_n = 1) \\ \vdots \\ \P(X_n = k) \end{pmatrix}$, pour tout $n \geqslant 0$, on obtient $Y_{n+1} = A Y_n$ et donc $Y_n = A^n Y_0$. \\
Le comportement (probabiliste) d'une chaîne de \textsc{Markov} stationnaire, et notamment son comportement asymptotique, est donc entièrement décrit par la donnée de la loi initiale $Y_0$ et des puissances de la matrice $A$. 

\begin{defi}{Matrice stochastique}
    Une matrice stochastique (matrice de transition d'une \nameref{chaîne_markov}) est une matrice $P \in \M_n([0, 1])$ telle que pour tout $i \in \llbracket 1, n \rrbracket, \sum\limits_{j=1}^{n} p_{i,j} = 1$. \\ Autrement dit, chaque ligne de $P$ est une vecteur de probabilité. \\
    On dit que $P$ est \emph{doublement stochastique} si $P$ et $\Trsp{P}$ sont stochastiques.
\end{defi}

L'exercice suivant étudie le spectre d'une matrice stochastique. 

\begin{exercice}
    \marginnote[0cm]{Source : Exercice 5, TD 11}
    Soit $P$ une matrice stochastique.
    \begin{enumerate}
        \item Montrer que $1$ est valeur propre de $P$.
        \item Soit $v = \Trsp{(v_1 \cdots v_n)}$ un vecteur propre associé à la valeur propre $1$. En considérant $|v_{i_0}| = \max\limits_{1 \leqslant i \leqslant n} |v_i|$, montrer que le sous-espace propre associé à $E_1$ est de dimension $1$.
        \item Montrer que si $\lambda \in \C$ est une valeur propre de $P$, alors $| \lambda | \leqslant 1$.
        \item Soit $\lambda \in \C$ une valeur propre de $P$ telle que $|\lambda| = 1$ et $\title{x}$ un vecteur propre associé.
        \begin{enumerate}
            \item Montrer qu'il existe un vecteur propre associé à $\lambda$ tel que $\Ninf{x} = 1$. 
            \item Montrer qu'il existe $i_0 \in \llbracket 1, n \rrbracket$ tel que $\left| \sum\limits_{j=1}^n p_{i_0,j} x_j \right| = 1$.
            \item Soit $\theta$ l'argument principal de $\sum\limits_{j=1}^n p_{i_0,j} x_j$. Montrer que pour tout $j \in \llbracket 1, n \rrbracket, \Reel \left( \e^{-\i \theta} x_j \right) = 1$.
            \item En déduire que $\lambda = 1$.
        \end{enumerate}
    \end{enumerate}
\end{exercice}

\begin{solution}
    \begin{enumerate}
        \item 1 est valeur propre évidente de $P$ de vecteur propre associé $v = (1, \dots, 1)^\top$.
        \item Montrer que $\dim E_1 = 1$.
        \begin{itemize}
            \item Appliquer la même méthode que la démonstration du lemme d'\textsc{Hadamard}. \\
            Soit $X = \Trsp{(x_1, \dots, x_n)} \in E_1$. Montrons que $X \in \Vect(v)$. \\
            On montre que $|x_{i_0}| = \left| \sum\limits_{j=1}^{n} p_{i_0, j} x_j \right| = \sum\limits_{j=1}^{n} p_{i_0, j} |x_j|$ et on écrit $|x_{i_0}| = |x_{i_0}| \sum\limits_{j=1}^{n} p_{i_0, j}$. D'où, en faisant la différence, pour tout $j \in \llbracket 1, n \rrbracket,\ |x_{i_0}| = |x_j|$. De plus d'après la première relation, il y égalité dans l'inégalité triangulaire et donc les $v_j$ sont \emph{positivement liées}. Finalement, pour tout $j \in \llbracket1, n \rrbracket,\ v_j = v_{i_0}$ soit $\dim E_1 = 1$.
        \end{itemize}
        \item Soit $\lambda \in \C$ une valeurs propre de $P$. \\
        Poser $X = (x_1, \dots, x_n)^\top$ un vecteur propre associé et appliquer encore une fois la même méthode; poser $\displaystyle |x_{i_0}|= \max_{1 \leqslant i \leqslant n} |x_i|$, écrire en module la ligne $i_0$ de l'égalité $\lambda X = P X$, diviser par $|x_{i_0}|$ (qui est non nul d'après la question précédente) puis majorer par $1$. \\
            
        Pour les curieux, lire \cite{matrices} page 59. 
        
        \begin{prop}{}
            Le \nameref{rayon_spectral} stochastique est égal à $1$.
        \end{prop}
    
        \item Soit $\lambda \in \C$ une valeur propre telle que $|\lambda| = 1$ et $\widetilde{x}$ un vecteur propre associé.
        \begin{enumerate}
            \item $\Ninf{\widetilde{x}} \not= 0$ car $\widetilde{x} \not= 0$. 
            $x = \frac{\widetilde{x}}{\Ninf{\widetilde{x}}}$ qui est vecteur propre associé à $\lambda$ et $\Ninf{x} = 1$.
            \item Il existe $i_0 \in \llbracket 1, n \rrbracket$ tel que $|x_{i_0}| = \max\limits_{1 \leqslant i \leqslant n} |x_i| = \Ninf{x} = 1$. \\
            Et $|\lambda x_{i_0}| = |\lambda||x_{i_0}| = 1 = \left| \sum\limits_{j=1}^n p_{i_0, j} x_j \right|$.
            \item On peut écrire
            \begin{align*}
                \sum_{j=1}^n p_{i_0, j} x_j &= \e^{\i \theta} & \text{selon 4)b)} \\
                &= \e^{\i \theta} \sum_{j=1}^n p_{i_0, j} &
            \end{align*}
            et
            $$\sum_{j=1}^n p_{i_0, j} \left( x_j - \e^{\i \theta} \right) = 0,$$
            $$\sum_{j=1}^n p_{i_0, j} \left( x_j \e^{-\i \theta} - 1 \right) = 0$$
            et
            $$\Reel \left( \sum_{j=1}^n p_{i_0, j} \left( x_j \e^{- \i \theta} - 1 \right) \right) = 0$$
            $$\Rightarrow \sum_{j=1}^n p_{i_0, j} \Big( \underbrace{\Reel \big(x_j \e^{- \i \theta} \big)}_{\mathclap{|\cdot| \leqslant |x_j \e^{-\i \theta}| \leqslant |x_j| \leqslant |x_{i_0}| \leqslant 1}} - 1 \Big) = 0$$
            donc $|\Reel(x_j \e^{- \i \theta}) | \leqslant 1$. \\
            Somme de termes négatifs qui est nulle donc
            $$\forall j \in \llbracket 0, n \rrbracket,\ \Reel(x_j \e^{- \i \theta}) = 1.$$
            \item Soit $j \in \llbracket 1, n \rrbracket$.
            $$|x_j \e^{-\i \theta} = |x_j| = \sqrt{ \Reel (x_j \e^{- \i \theta})^2 + \Im(x_j \e^{- \i \theta})^2 } = \sqrt{1 + \Im(x_j \e^{\i \theta})^2} \leqslant 1$$
            donc $\Im(x_j \e^{\i \theta}) = 0$ donc $x_j \e^{\i \theta} = 1$ et $x = \e^{\i \theta} \begin{pmatrix}1 \\ \vdots \\ 1\end{pmatrix}$. $Px = \e^{\i \theta} P \begin{pmatrix}1 \\ \vdots \\ 1\end{pmatrix} = \e^{\i \theta} \begin{pmatrix}1 \\ \vdots \\ 1\end{pmatrix} = x$. 
        \end{enumerate}
    \end{enumerate}
\end{solution}

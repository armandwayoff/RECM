\begin{defi}
    Une matrice stochastique (matrice de transition d'une \nameref{chaîne_markov}) est une matrice $P \in \M_n([0, 1])$ telle que pour tout $i \in \segN{1,n}, \sum\limits_{j=1}^{n} p_{i,j} = 1$. \\ Autrement dit, chaque ligne de $P$ est une vecteur de probabilité. \\
    On dit que $P$ est \emph{doublement stochastique} si $P$ et $\Trsp{P}$ sont stochastiques.
\end{defi}

\begin{exercice}
    \footnote{Exercice 5, TD 11} \\
    Soit $P$ une matrice stochastique.
    \begin{enumerate}
        \item Montrer que $1$ est valeur propre de $P$.
        \item Soit $v = \Trsp{(v_1 \cdots v_n)}$ un vecteur propre associé à la valeur propre $1$. En considérant $|v_{i_0}| = \max\limits_{1 \leqslant i \leqslant n} |v_i|$, montrer que le sous-espace propre associé à $E_1$ est de dimension $1$.
        \item Montrer que si $\lambda \in \C$ est une valeur propre de $P$, alors $| \lambda | \leqslant 1$.
        \item Soit $\lambda \in \C$ une valeur propre de $P$ telle que $|\lambda| = 1$ et $\title{x}$ un vecteur propre associé.
        \begin{enumerate}
            \item Montrer qu'il existe un vecteur propre associé à $\lambda$ tel que $\Ninf{x} = 1$. 
            \item Montrer qu'il existe $i_0 \in \llbracket 1, n \rrbracket$ tel que $\left| \sum\limits_{j=1}^n p_{i_0,j} x_j \right| = 1$.
            \item Soit $\theta$ l'argument principal de $\sum\limits_{j=1}^n p_{i_0,j} x_j$. Montrer que pour tout $j \in \llbracket 1, n \rrbracket, \Reel \left( \me^{-\mi \theta} x_j \right) = 1$.
            \item En déduire que $\lambda = 1$.
        \end{enumerate}
    \end{enumerate}
\end{exercice}

\marginnote[-2cm]{
    $$
    \begin{pmatrix}
    1/2 & 1/2 & 0 \\
    3/4 & 1/8 & 1/8 \\
    0 & 1/3 & 2/3
    \end{pmatrix}
    $$
}

\begin{marginfigure}[-3cm]
    \centering
    \resizebox{5.5cm}{5.5cm}{%
    \begin{tikzpicture}
        \node[state] (s1) {1};
        \node[state, below right of=s1] (s2) {2};
        \node[state, below left of=s1] (s3) {3};
    
        \draw (s1) edge[loop above] node {$1/2$} (s1);
        \draw (s1) edge[bend left] node {$1/2$} (s2);
        %\draw (s1) edge[bend right, above left] node {0} (s3);
    
        \draw (s2) edge[bend left, above right] node {$3/4$} (s1);
        \draw (s2) edge[loop right] node {$1/8$} (s2);
        \draw (s2) edge[bend right] node {$1/8$} (s3);
    
        %\draw (s3) edge[bend right] node {0} (s1);
        \draw (s3) edge[bend right] node {$1/3$} (s2);
        \draw (s3) edge[loop left] node {$2/3$} (s3);
    \end{tikzpicture}    
}

  %  \begin{tikzpicture}
  %      \node[state] (s1) {État 1};
  %      \node[state, below right of=s1] (s2) {État 2};
   %     \node[state, below left of=s1] (s3) {État 3};
 %  
 %       \draw (s1) edge[loop above] node {$p_{1,1}$} (s1);
%        \draw (s1) edge[bend left] node {$p_{1,2}$} (s2);
%        \draw (s1) edge[bend right, above left] node {$p_{1,3}$} (s3);
    
 %       \draw (s2) edge[bend left, above right] node {$p_{2,1}$} (s1);
%        \draw (s2) edge[loop right] node {$p_{2,2}$} (s2);
%        \draw (s2) edge[bend right] node {$p_{2,3}$} (s3);
    
%        \draw (s3) edge[bend right] node {$p_{3,1}$} (s1);
%        \draw (s3) edge[bend right] node {$p_{3,2}$} (s2);
%        \draw (s3) edge[loop left] node {$p_{3,3}$} (s3);
%    \end{tikzpicture}    
    \caption*{\centering Une chaîne de \textsc{Markov} et sa matrice de transition.}
\end{marginfigure}

\begin{solution}
    L'objectif de l'exercice est de montrer que $\boxed{\Sp_{\C}(P) = \{1 \} }$.
    \begin{enumerate}
        \item 1 est valeur propre évidente de $P$ de vecteur propre associé $v = (1, \dots, 1)^\top$.
        \item Montrer que $\dim E_1 = 1$.
        \begin{itemize}
            \item Appliquer la même méthode que la démonstration du lemme d'\textsc{Hadamard}. \\
            Soit $X = \Trsp{(x_1, \dots, x_n)} \in E_1$. Montrons que $X \in \Vect(v)$. \\
            On montre que $|x_{i_0}| = \left| \sum\limits_{j=1}^{n} p_{i_0, j} x_j \right| = \sum\limits_{j=1}^{n} p_{i_0, j} |x_j|$ et on écrit $|x_{i_0}| = |x_{i_0}| \sum\limits_{j=1}^{n} p_{i_0, j}$. D'où, en faisant la différence, pour tout $j \in \llbracket 1, n \rrbracket,\ |x_{i_0}| = |x_j|$. De plus d'après la première relation, il y égalité dans l'inégalité triangulaire et donc les $v_j$ sont \emph{positivement liées}. Finalement, pour tout $j \in \llbracket1, n \rrbracket,\ v_j = v_{i_0}$ soit $\dim E_1 = 1$.
        \end{itemize}
        \item Montrer que si $\lambda \in \C$ est une valeur propre de $P$, alors $|\lambda| \leqslant 1$. \\
        Poser $X = (x_1, \dots, x_n)^\top$ un vecteur propre associé et appliquer encore une fois la même méthode; poser $\displaystyle |x_{i_0}|= \max_{1 \leqslant i \leqslant n} |x_i|$, écrire en module la ligne $i_0$ de l'égalité $\lambda X = P X$, diviser par $|x_{i_0}|$ (qui est non nul d'après la question précédente) puis majorer par $1$. \\
            
        Pour les curieux, lire \cite{matrices} page 59. 
        
        \begin{prop}
            Le \nameref{rayon_spectral} stochastique est égal à $1$.
        \end{prop}
    
        \item Les questions suivantes (à détailler éventuellement) consiste encore au même jeu avec la ligne $i_0$ et les propriétés de matrices stochastiques. 
    \end{enumerate}
\end{solution}

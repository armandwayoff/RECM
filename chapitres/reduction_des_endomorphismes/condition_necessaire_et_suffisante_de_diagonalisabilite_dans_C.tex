\begin{exercice}
    \marginnote[0cm]{Exercice 10 TD 11}
    Soit $f \in \mathscr{L}(\K^n)$.
    \begin{enumerate}
        \item On suppose que $f$ est diagonalisable. Montrer que tout sous-espace de $\K^n$ stable par $f$ admet un supplémentaire stable par $f$.
        \item Que dire de la réciproque dans $\C$.
        \item Décrire un contre-exemple à la réciproque dans $\R$, en dimension 2.
    \end{enumerate}
\end{exercice}  

\begin{enumerate}
    \item Soit $F$ un sev de $E$ stable par $f$.
    \begin{itemize}
        \item Posons $g = f_{\vert F}$ qui est diagonalisable car $f$ l'est par hypothèse. 
        \item Soit $\mathscr{B}_F$ une base de $F$ formée de vep de $g$. 
        \item Soit $\mathscr{B}'$ une base de $E$ formée de vep de $f$ (qui existe car $f$ est diagonalisable).
        \item On \ptnclegras{complète} $\mathscr{B}_F$ est une base $\mathscr{B}$ de E en prenant des vep $(\varepsilon_1, \cdots, \varepsilon_r)$ de $\mathscr{B}'$. On note $(\lambda_1, \dots, \lambda_r)$ les valeurs propres associées. 
        \item On pose G = $\mathrm{Vect}(\varepsilon_1, \dots, \varepsilon_r)$. De cette manière, $F$ et $G$ sont supplémentaires. Montrons que $G$ est stable par $f$. \\
        Soit $x = \sum\limits_{i=1}^{r} \mu_i \varepsilon_i \in G$. Donc $f(x) = \sum\limits_{i=1}^{r} \mu_i f(\varepsilon_i) =  \sum\limits_{i=1}^{r} \mu_i \lambda_i \varepsilon_i \in G$ et $G$ est stable par $f$.
    \end{itemize}
    
    \item Que dire de la réciproque dans $\C$ ? \\
    Montrons que $f$ est diagonalisable. On va montrer que $E = \bigoplus\limits_{\lambda \in \Sp(f)} E_\lambda (f)$.
    
    \begin{enumerate}
        \item \underline{Somme directe:} \\
        On pose $F = \bigoplus\limits_{\lambda \in \Sp(f)} E_\lambda (f)$ et $\Sp(f) = (\lambda_1, \dots, \lambda_r)$. \\
        Soit $x = \sum\limits_{i=1}^{r} x_i \in F$ où $x_i \in E_{\lambda_i}(f)$. Alors $f(x) = \sum\limits_{i=1}^{r} \underbrace{\lambda_i x_i}_{\in E_{\lambda_i}(f)} \in F$. Donc $F$ est stable par $f$.
        \item Montrons que $F = E$. \\
        Par hypothèse, $F$ admet un supplémentaire $G$ dans $\C^n$, stable par $f$. Montrons que $G = \{0\}$ en raisonnant par l'absurde. \\
        On pose $g = f_{\vert F}$. D'après le \ptnclegras{théorème de \textsc{D'Alembert-Gauss}} sur $\C$, $g$ admet au moins une valeur propre $\mu \in \C$ de vep associé $x_\mu$. On montre que $x_\mu \in F \cap G$. Or $F$ et $G$ sont supplémentaires donc $x_\mu = 0_E$: contradiction. D'où le résultat. 
    \end{enumerate}
    \item Décrire un contre-exemple à la réciproque dans $\R$, en dimension 2. \\
    Considérer $R_\theta$. \textcolor{green}{à revoir}
\end{enumerate}
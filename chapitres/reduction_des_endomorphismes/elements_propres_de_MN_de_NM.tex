Soit $M, N \in \M_n(\K)$. 
\begin{enumerate}
    \item ...
    \item \emph{Soit $\lambda \in \Ke$, montrer que $\dim(E_\lambda (MN)) = \dim(E_\lambda (NM))$.} \\
    On remarque que si $X \in E_\lambda (MN)$ alors $NX \in E_\lambda (NM)$. On pose alors:
    \begin{alignat*}{2}
        \varphi\ :\ E_\lambda (MN)\ &\longrightarrow\ E_\lambda (NM)\\
        X\ &\longmapsto\ NX
    \end{alignat*}
    On montre que $\varphi$ est injective et on en déduit que $\dim(E_\lambda (MN)) \leqslant \dim(E_\lambda (NM))$. Par symétrie des rôles de $M$ et de $N$, on montre l'inégalité dans le sens inverse et on en déduit l'égalité.
    \item ...
    \item ...
    \item cf. page 21 de mon cours manuscrit. Un autre démonstration est proposée dans \nameref{spectre_I-uv_et_I-vu}.
    $$\boxed{\forall (M, N) \in \M_n(\K)^2,\ \Sp_\K(MN) = \Sp_\K(NM) \text{ et } \chi_{MN} = \chi_{NM}}$$
    $$\boxed{\overline{\Gl_n(\K)} = \M_n(\K)}$$
\end{enumerate}

\begin{exercice}
    Soit $(A, B) \in \M_n(\K)^2$.
    \begin{enumerate}
        \item Lorsque $A$ est inversible, montrer que $\chi_{AB}=\chi_{BA}$.
        \item En déduire que $\chi_{AB}=\chi_{BA}$ par un argument de continuité.
    \end{enumerate}
\end{exercice}

\begin{solution}
    \begin{enumerate}
        \item \begin{align*}
        \chi_{AB} &= \det(\lambda \I_n - AB) \\
        &= \det(A(\lambda \Inv{A} - B)) \quad \text{car } A \in \Gl_n(\K) \\
        &= \det(A) \det(\lambda \Inv{A} - B) \\
        &= \det(\lambda \Inv{A} - B) \det(A) \\
        &= \det(\lambda \I_n - BA) \\
        \chi_{AB} &= \chi_{BA}
    \end{align*}
    \item Soit $A \in \M_n(\K)$. Son polynôme caractéristique $\chi_A$ admet au plus $n$ racines. \\
    Notons $r \defeq \min \left\{ |\lambda|, \lambda \in \Sp(A) \backslash \{0\} \right\}$. Donc pour tout $t \in ]0,r[, \chi_A(t) \not=0$ soit $t \I_n - A \in \Gl_n(\K)$. \\
    Soit $p_0 \defeq \min \left\{ p, \frac{1}{p} < r \right\}$. Ainsi, en posant $A_p \defeq A - \frac{1}{p} \I_n$, la suite $(A_p)_{p \geqslant p_0}$ est une suite de matrices inversibles qui converge vers $A$. Comme les matrices $A_p$ sont inversibles (pour $p \geqslant p_0$), d'après le premier point, pour tout $p \geqslant p_0$
    $$\chi_{A_p B} = \chi_{B A_p}$$
    soit 
    $$\det(\lambda \I_n - A_p B) = \det(\lambda \I_n - B A_p).$$
    Comme le produit matriciel est une application bilinéaire, $A_p B$ (resp. $B A_p$) tend vers $AB$ (resp. $BA$) quand $p$ tend vers l'infini. Comme le déterminant est une application multilinéaire en dimension finie, elle est continue et $\det(\lambda \I_n - A B) = \det(\lambda \I_n - B A)$ soit $\chi_{A B} = \chi_{B A}$. \\
    On a montré que 
    $$\overline{\Gl_n(\K)} = \M_n(\K).$$
    \end{enumerate}
\end{solution}
{\Large Diagonalisation \sidenote[][-2mm]{Le texte suivant est extrait de \cite{oraux_x_ens_2} p. 169.}} \\

\begin{marginfigure}[1cm]
    \centering
    \includegraphics[width=5cm]{images/camille_jordan.jpg}
    \caption*{\centering Camille \textsc{Jordan} (1838 - 1922)}
\end{marginfigure}

\textsl{On doit à Camille \textsc{Jordan} de nombreux résultats sur la réduction des endomorphismes qu'il découvre notamment à travers l'étude des groupes. Dépassant la notion des groupes de permutations pour en atteindre une plus abstraite, il s'intéresse à la classification des groupes finis à travers leurs représentations linéaires, autrement dit les morphismes entre un groupe fini $G$ et le groupe linéaire $\Gl (E)$ d'un espace vectoriel $E$. Il va même jusqu'à donner le description des classes de similitudes à l'aide des formes dites de \textsc{Jordan}.}\\
\textsl{Le problème fondamental de la réduction est bien celui de caractériser les classes de similitude de l'algèbre $\Endo(E)$ où $E$ est un $\K$-espace vectoriel de dimension finie ou, ce qui revient au même, les classes de similitudes de l'agèbre $\M_n(\K)$. La recherche d'une matrice la plus simple possible pour représenter un endomorphisme donné vise de multiples buts: calculer les puissances successives de cet endomorphisme, son commutant, résoudre des systèmes différentiels linéaires... Une idée naturelle pour essayer de \say{ réduire } l'étude d'un endomorphisme $u$ donné à des choses plus simples consiste à essayer de décomposer l'espace vectoriel $E$ en une somme directe de sous-espaces non triviaux stables par $u$. Cela n'est évidemment pas toujours possible. Les sous-espaces stables les plus simples sont ceux sur lesquels $u$ coïncide avec une homothétie. On est ainsi naturellement amené à la notion de valeur propre. Si $\lambda$ est un scalaire, on s'intéresse donc au sous-espace $E_\lambda = \Ker(u - \lambda \Id_E)$ appelé sous-espace propre pour la valeur propre $\lambda$ lorsque celui-ci n'est pas nul. Le théorème de décomposition des noyaux nous assure que les différents sous-espaces propres d'un endomorphisme sont en somme directe. Le cas où la somme remplit tout l'espace $E$ mène à la notion d'endomorphisme diagonalisable: un tel endomorphisme peut être représenté par une matrice diagonale (il suffit de prendre une base formée de vecteurs propres). Pour les endomorphismes diagonalisables il est alors très facile de répondre à la question initiale de savoir quand ils sont semblables: il faut et suffit qu'ils aient les mêmes valeurs propres et que les espaces propres associés aient la même dimension. Il est aussi facile, en se ramenant à une matrice diagonale, de calculer les puissances d'un tel endomorphisme, son exponentielle (si on travaille sur un sous-corps de $\C$), son commutant...} \\

Soit $E$ un espace vectoriel de dimension finie. Un endomorphisme $u$ de $E$ est trigonalisable si et seulement s'il existe un drapeau total de $E$ stable par $u$. 

\begin{Large}
    Trigonalisation
\end{Large}

\textsl{La diagonalisation ne permet pas de caractériser toutes les classes de similitude de $\M_n(\K)$. Lorsque le corps de base est égal à $\C$, le polynôme caractéristique d'une matrice $A \in \M_n(\C)$ est toujours scindé et $A$ est alors trigonalisable. Le lemme des noyaux permet même dans ce cas de trigonaliser $A$ sous une forme diagonale par blocs avec pour chaque bloc diagonal une unique valeur propre. Cela conduit à la notion de sous-espace caractéristique et à la décomposition de \textsc{Jordan}-\textsc{Dunford}: dans le cas abstrait, tout endomorphisme $u$ d'un $\C$-espace vectoriel de dimension finie s'écrit de manière unique sous la forme $u = d + n$ où $d$ est diagonalisable, $n$ est nilpotent et commute avec $d$. (H.P. CPGE). Cette décomposition amène l'attention sur les classes de similitude des endomorphismes nilpotents. Avec un peu de travail il est alors possible d'obtenir le théorème de \textsc{Jordan} qui règle complètement la question de la détermination des classes de similitude sur un corps algébriquement clos. \\
Dans le cadre général, l'outil fondamental pour élucider les classes de similitude de $\M_n(\K)$ est la notion d'endomorphisme cyclique.
}
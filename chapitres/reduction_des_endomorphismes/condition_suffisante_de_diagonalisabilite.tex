\begin{exercice}
    \begin{enumerate}
        \item Soit $A \in \M_2(\C)$. On suppose que $\Tr{A} = 0$ et $\Tr{A}^2 \not=0$. Montrer que $A$ est diagonalisable. \\
        On cherche à généraliser ce résultat en dimension $n$. Soit $A \in \M_n(\C)$ telle que
        $$
        \begin{cases}
            \Tr{A^k}= 0 \quad \forall k \in \llbracket 1, n-1 \rrbracket, \\
            \Tr{A^n}\not=0.
        \end{cases}
        $$
        \item Montrer que $A$ possède au moins une valeur propre non nulle. 
        \item On note $\alpha_1, \dots, \alpha_k$ les valeurs propres non nulles deux à deux distinctes de $A$, $n_1, \dots, n_k$ leurs multiplicités respectives et 
        $$
        V \defeq \begin{pmatrix}
            \alpha_1 & \cdots & \alpha_k \\
            \vdots & & \vdots \\
            \alpha_1^{n-1} & \cdots & \alpha_k^{n-1}
        \end{pmatrix}
        \in \M_{n-1,k}(\C).
        $$
        Montrer que $\begin{pmatrix} n_1  \\ \vdots \\ n_k \end{pmatrix} \in \Ker{V}$.
        \item En déduire que $A$ est diagonalisable.
        \item Montrer que, si on supprime l'hypothèse $\Tr{A^n} \not=0$, alors soit $A$ est diagonalisable, soit $A^n = 0$.
    \end{enumerate}
\end{exercice}

\newcommand{\vandermondepartiel}{
\left(\begin{gathered}
    \tikzpicture[every node/.style={anchor=south west}]
        \node[minimum width=1.5cm,minimum height=0.5cm] at (0.125,1.25) {\LARGE $V_k$};
        
        \node[minimum width=0.5cm,minimum height=0.5cm] at (0,0) {$\star$};
        \node[minimum width=0.5cm,minimum height=0.5cm] at (0.55,0) {$\cdots$};
        \node[minimum width=0.5cm,minimum height=0.5cm] at (1.25,0) {$\star$};
        
        \node[minimum width=0.5cm,minimum height=0.5cm] at (0,0.375) {$\vdots$};
        \node[minimum width=0.5cm,minimum height=0.5cm] at (1.25,0.375) {$\vdots$};
        
        \node[minimum width=0.5cm,minimum height=0.5cm] at (0,0.75) {$\star$};
        \node[minimum width=0.5cm,minimum height=0.5cm] at (0.55,0.75) {$\cdots$};
        \node[minimum width=0.5cm,minimum height=0.5cm] at (1.25,0.75) {$\star$};

        \draw (0, 1.25) -- (1.75, 1.25);
    \endtikzpicture
    \end{gathered}\right)
}

\begin{solution}
    \begin{enumerate}
        \item Soit $A \in \M_2(\C)$ telle que $\Tr{A} = 0$ et $\Tr{A^2} \not=0$. \\
        La matrice $A$ est trigonalisable dans $\C$ donc semblable à la matrice $\begin{pmatrix} \lambda_1 & \star \\ 0 & \lambda_2 \end{pmatrix}$. \\
        Alors $\Tr{A} = \lambda_1 + \lambda_2 = 0 \Rightarrow \lambda_1 = - \lambda_2$ et $\Tr{A^2} = \lambda_1^2 + \lambda_2^2 = 2 \lambda_1^2 \not=0$ et donc $\lambda_1 \not=0$. \\
        Ainsi, $\lambda_1 \not= \lambda_2$ et la matrice $A$ possède deux valeurs propres distinctes donc est diagonalisable. 
        \item La matrice $A$ est trigonalisable dans $\C$ et on note $\lambda_1, \dots, \lambda_n$ ses valeurs propres. Alors la matrice $A$ est semblable à la matrice 
        $$
        \begin{pmatrix}
            \lambda_1 & \star & \cdots & \star \\
            0 & \lambda_2 & \cdots & \star \\
            \vdots & \ddots &\ddots & \vdots \\
            0 & \cdots & 0 & \lambda_n
        \end{pmatrix}
        $$
        et la matrice $A^n$ est semblable à la matrice
        $$
        \begin{pmatrix}
            \lambda_1^n & \star & \cdots & \star \\
            0 & \lambda_2^n & \cdots & \star \\
            \vdots & \ddots &\ddots & \vdots \\
            0 & \cdots & 0 & \lambda_n^n
        \end{pmatrix}.
        $$
        Or, par hypothèse, $\Tr{A^n} = 0$ donc les $\lambda_i$ ne peuvent pas être tous nuls.
        \item On pose $N \defeq \begin{pmatrix} n_1  \\ \vdots \\ n_k \end{pmatrix}$.
        \begin{align*}
            V N = 
            \begin{pmatrix}
                \sum\limits_{i=1}^n \alpha_i n_i \\ \vdots \\ \sum\limits_{i=1}^n \alpha_i^{n-1} n_i
            \end{pmatrix}
            = 
            \begin{pmatrix}
                \Tr{A} \\ \vdots \\ \Tr{A^{n-1}}
            \end{pmatrix}
            =
            0_n
        \end{align*}
        \item Supposons par l'absurde que $k < n$. \\
        $$V N = \vandermondepartiel N = 0_n$$
        alors 
        $$V_k N = 0_n.$$
        Comme $V_k$ est une matrice de \textsc{Vandermonde} et que les $(\alpha_1, \dots, \alpha_k)$ sont deux à deux distincts alors elle est inversible et donc
        $$N = 0_n.$$
        Ainsi, la matrice $A$ ne possède par de valeurs propres non nulle ce qui est absurde d'après la question 2. \\
        On en déduit que $k \leqslant n$ et comme $k \geqslant n$, $k=n$. La matrice $A$ possède $n$ valeurs propres distinctes donc est diagonalisable.
        \item \textcolor{red}{A compléter}
        \end{enumerate}
\end{solution}

\begin{exercice}
    \cite{reduc_des_endo} p. 114 (je pense qu'il y a un oubli dans l'énoncé)\\
    Soit $A \in \M_n(\C)$ telle que $\Tr{A} = \Tr{A^2} = \cdots = \Tr{A^{n-1}} = 0$. Montrer que $A$ est diagonalisable ou nilpotente. 
\end{exercice}

\begin{elem_sol}
    D'après les formules des sommes de \textsc{Newton} et les relations coefficients-racines (\url{https://fr.wikipedia.org/wiki/Polynôme_caractéristique#Coefficients}), le polynôme caractéristique est 
    $$\chi_A = X^n + (-1)^n \det(A).$$
    Si $\det(A) = 0$, le théorème de \textsc{Cayley}-\textsc{Hamilton} assure que $A$ est nilpotente. Sinon, le polynôme caractéristique est scindé à racines simples donc $A$ est diagonalisable (et d'ailleurs, toutes les valeurs propres ont le même module).
\end{elem_sol}
\chapter{Réduction des endomorphismes}
\labch{reduction_des_endomorphismes}

{\Large Diagonalisation} \\
\marginnote[0cm]{Le texte suivant est extrait de \cite{oraux_x_ens_2} p. 169.}
\begin{marginfigure}[1cm]
    \centering
    \includegraphics[width=5cm]{images/camille_jordan.jpg}
    \caption*{\centering Camille \nom{Jordan} (1838 - 1922)}
\end{marginfigure}
\textsl{On doit à Camille \nom{Jordan} de nombreux résultats sur la réduction des endomorphismes qu'il découvre notamment à travers l'étude des groupes. \\
Le problème fondamental de la réduction est bien celui de caractériser les classes de similitude de l'algèbre $\Endo(E)$ où $E$ est un $\K$-espace vectoriel de dimension finie ou, ce qui revient au même, les classes de similitudes de l'agèbre $\M_n(\K)$. La recherche d'une matrice la plus simple possible pour représenter un endomorphisme donné vise de multiples buts: calculer les puissances successives de cet endomorphisme, son commutant, résoudre des systèmes différentiels linéaires... Une idée naturelle pour essayer de \chevron{réduire} l'étude d'un endomorphisme $u$ donné à des choses plus simples consiste à essayer de décomposer l'espace vectoriel $E$ en une somme directe de sous-espaces non triviaux stables par $u$. Cela n'est évidemment pas toujours possible. Les sous-espaces stables les plus simples sont ceux sur lesquels $u$ coïncide avec une homothétie. On est ainsi naturellement amené à la notion de valeur propre. Si $\lambda$ est un scalaire, on s'intéresse donc au sous-espace $E_\lambda = \Ker(u - \lambda \Id_E)$ appelé sous-espace propre pour la valeur propre $\lambda$ lorsque celui-ci n'est pas nul. Le théorème de décomposition des noyaux nous assure que les différents sous-espaces propres d'un endomorphisme sont en somme directe. Le cas où la somme remplit tout l'espace $E$ mène à la notion d'endomorphisme diagonalisable: un tel endomorphisme peut être représenté par une matrice diagonale (il suffit de prendre une base formée de vecteurs propres). Pour les endomorphismes diagonalisables il est alors très facile de répondre à la question initiale de savoir quand ils sont semblables: il faut et suffit qu'ils aient les mêmes valeurs propres et que les espaces propres associés aient la même dimension. Il est aussi facile, en se ramenant à une matrice diagonale, de calculer les puissances d'un tel endomorphisme, son exponentielle (si on travaille sur un sous-corps de $\C$), son commutant...}

\newpage

\section{Matrice à diagonale dominante}
\input{chapitres/reduction_des_endomorphismes/matrice_a_diagonale_dominante}

\section{Matrices stochastiques}
\marginnote[3cm]{
    $$
    \begin{pmatrix}
    1/2 & 1/2 & 0 \\
    3/4 & 1/8 & 1/8 \\
    0 & 1/3 & 2/3
    \end{pmatrix}
    $$
}

\begin{marginfigure}[5cm]
    \centering
    \resizebox{5.5cm}{5.5cm}{%
    \begin{tikzpicture}
        \node[state] (s1) {1};
        \node[state, below right of=s1] (s2) {2};
        \node[state, below left of=s1] (s3) {3};
    
        \draw (s1) edge[loop above] node {$1/2$} (s1);
        \draw (s1) edge[bend left] node {$1/2$} (s2);
        %\draw (s1) edge[bend right, above left] node {0} (s3);
    
        \draw (s2) edge[bend left, above right] node {$3/4$} (s1);
        \draw (s2) edge[loop right] node {$1/8$} (s2);
        \draw (s2) edge[bend right] node {$1/8$} (s3);
    
        %\draw (s3) edge[bend right] node {0} (s1);
        \draw (s3) edge[bend right] node {$1/3$} (s2);
        \draw (s3) edge[loop left] node {$2/3$} (s3);
    \end{tikzpicture}    
}

  %  \begin{tikzpicture}
  %      \node[state] (s1) {État 1};
  %      \node[state, below right of=s1] (s2) {État 2};
   %     \node[state, below left of=s1] (s3) {État 3};
 %  
 %       \draw (s1) edge[loop above] node {$p_{1,1}$} (s1);
%        \draw (s1) edge[bend left] node {$p_{1,2}$} (s2);
%        \draw (s1) edge[bend right, above left] node {$p_{1,3}$} (s3);
    
 %       \draw (s2) edge[bend left, above right] node {$p_{2,1}$} (s1);
%        \draw (s2) edge[loop right] node {$p_{2,2}$} (s2);
%        \draw (s2) edge[bend right] node {$p_{2,3}$} (s3);
    
%        \draw (s3) edge[bend right] node {$p_{3,1}$} (s1);
%        \draw (s3) edge[bend right] node {$p_{3,2}$} (s2);
%        \draw (s3) edge[loop left] node {$p_{3,3}$} (s3);
%    \end{tikzpicture}    
    \caption*{\centering Une chaîne de \textsc{Markov} et sa matrice de transition.}
\end{marginfigure}

\marginnote[0cm]{Texte de \cite{oraux_x_ens_2} p. 59}
Les matrices stochastiques interviennent en probibilités. Si $X$ et $Y$ sont deux variables aléatoires à valeurs dans $E \defeq \llbracket 1, k \rrbracket$, alors la matrice $A \defeq (a_{i,j}) \in \M_k(\R)$ définie par $a_{i,j} \defeq \P(Y = j | X = i)$ est stochastique, ce qui par définition signifie qu'on a $a_{i,j} \geqslant 0\ (1 \leqslant i, j \leqslant k)$ et $\sum\limits_{j=1}^k a_{i,j} = 1\ (1 \leqslant i \leqslant k)$. \\
L'évolution d'un système susceptible de prendre un nombre fini d'états notés $1, \dots, k$ est représentée mathématiquement par une suite $(X_n)_{n \geqslant 0}$ de variables aléatoires à valeurs dans $E$. C'est ce qu'on appelle un processus aléatoire (ou stochastique). Si $X_{n+1}$ s'obtient à partir de la valeur de $X_n$ et d'un tirage au sort effectué selon une loi ne dépendant que de cette valeur, on dit que le processus est une chaîne de \textsc{Markov}. Les exemples abondent: marches aléatoires, fortune d'un joueur, modélisation de l'alternance des voyelles et des consonnes dans un poème de \textsc{Pouchkine} (par \textsc{Markov} lui-même), ou prévision (en probabilité) des états  successifs d'un signal pour améliorer la compression en traitement du signal (\textsc{Shannon}) \\
Techniquement, on dit qu'une suite de variables aléatoires $(X_n)$ est une chaîne de \textsc{Markov} si \say{ la loi de l'état $n+1$ conditionnelle au passé de dépend que de l'état antérieur $n$ }, ce qui se traduit par
$$\P(X_{n+1}=j | X_0=i_0, \dots, X_n = i_n) = \P(X_{n+1} = j | X_n = i_n).$$
Si la matrice $A \defeq (a_{i,j}) \in \M_k(\R)$ définie par $a_{i,j} \defeq \P(Y = j | X = i)$ est indépendante de $n$, on dit que la chaîne de \textsc{Markov} est stationnaire. Si, dans ce dernier cas, on pose $Y_n \defeq \begin{pmatrix} \P(X_n = 1) \\ \vdots \\ \P(X_n = k) \end{pmatrix}$, pour tout $n \geqslant 0$, on obtient $Y_{n+1} = A Y_n$ et donc $Y_n = A^n Y_0$. \\
Le comportement (probabiliste) d'une chaîne de \textsc{Markov} stationnaire, et notamment son comportement asymptotique, est donc entièrement décrit par la donnée de la loi initiale $Y_0$ et des puissances de la matrice $A$. 

\begin{defi}
    Une matrice stochastique (matrice de transition d'une \nameref{chaîne_markov}) est une matrice $P \in \M_n([0, 1])$ telle que pour tout $i \in \llbracket 1, n \rrbracket, \sum\limits_{j=1}^{n} p_{i,j} = 1$. \\ Autrement dit, chaque ligne de $P$ est une vecteur de probabilité. \\
    On dit que $P$ est \emph{doublement stochastique} si $P$ et $\Trsp{P}$ sont stochastiques.
\end{defi}

\begin{exercice}
    \footnote{Exercice 5, TD 11} \\
    Soit $P$ une matrice stochastique.
    \begin{enumerate}
        \item Montrer que $1$ est valeur propre de $P$.
        \item Soit $v = \Trsp{(v_1 \cdots v_n)}$ un vecteur propre associé à la valeur propre $1$. En considérant $|v_{i_0}| = \max\limits_{1 \leqslant i \leqslant n} |v_i|$, montrer que le sous-espace propre associé à $E_1$ est de dimension $1$.
        \item Montrer que si $\lambda \in \C$ est une valeur propre de $P$, alors $| \lambda | \leqslant 1$.
        \item Soit $\lambda \in \C$ une valeur propre de $P$ telle que $|\lambda| = 1$ et $\title{x}$ un vecteur propre associé.
        \begin{enumerate}
            \item Montrer qu'il existe un vecteur propre associé à $\lambda$ tel que $\Ninf{x} = 1$. 
            \item Montrer qu'il existe $i_0 \in \llbracket 1, n \rrbracket$ tel que $\left| \sum\limits_{j=1}^n p_{i_0,j} x_j \right| = 1$.
            \item Soit $\theta$ l'argument principal de $\sum\limits_{j=1}^n p_{i_0,j} x_j$. Montrer que pour tout $j \in \llbracket 1, n \rrbracket, \Reel \left( \me^{-\mi \theta} x_j \right) = 1$.
            \item En déduire que $\lambda = 1$.
        \end{enumerate}
    \end{enumerate}
\end{exercice}

\begin{solution}
    L'objectif de l'exercice est de montrer que $\boxed{\Sp_{\C}(P) = \{1 \} }$.
    \begin{enumerate}
        \item 1 est valeur propre évidente de $P$ de vecteur propre associé $v = (1, \dots, 1)^\top$.
        \item Montrer que $\dim E_1 = 1$.
        \begin{itemize}
            \item Appliquer la même méthode que la démonstration du lemme d'\textsc{Hadamard}. \\
            Soit $X = \Trsp{(x_1, \dots, x_n)} \in E_1$. Montrons que $X \in \Vect(v)$. \\
            On montre que $|x_{i_0}| = \left| \sum\limits_{j=1}^{n} p_{i_0, j} x_j \right| = \sum\limits_{j=1}^{n} p_{i_0, j} |x_j|$ et on écrit $|x_{i_0}| = |x_{i_0}| \sum\limits_{j=1}^{n} p_{i_0, j}$. D'où, en faisant la différence, pour tout $j \in \llbracket 1, n \rrbracket,\ |x_{i_0}| = |x_j|$. De plus d'après la première relation, il y égalité dans l'inégalité triangulaire et donc les $v_j$ sont \emph{positivement liées}. Finalement, pour tout $j \in \llbracket1, n \rrbracket,\ v_j = v_{i_0}$ soit $\dim E_1 = 1$.
        \end{itemize}
        \item Montrer que si $\lambda \in \C$ est une valeur propre de $P$, alors $|\lambda| \leqslant 1$. \\
        Poser $X = (x_1, \dots, x_n)^\top$ un vecteur propre associé et appliquer encore une fois la même méthode; poser $\displaystyle |x_{i_0}|= \max_{1 \leqslant i \leqslant n} |x_i|$, écrire en module la ligne $i_0$ de l'égalité $\lambda X = P X$, diviser par $|x_{i_0}|$ (qui est non nul d'après la question précédente) puis majorer par $1$. \\
            
        Pour les curieux, lire \cite{matrices} page 59. 
        
        \begin{prop}
            Le \nameref{rayon_spectral} stochastique est égal à $1$.
        \end{prop}
    
        \item Les questions suivantes (à détailler éventuellement) consiste encore au même jeu avec la ligne $i_0$ et les propriétés de matrices stochastiques. 
    \end{enumerate}
\end{solution}


\section{Endomorphismes semi-simples}
\begin{prop}{Critère de diagonalisabilité dans un $\R$-ev}
    Soit $E$ un $\R$-espace vectoriel de dimension finie, soit $f \in \Endo(E)$. Alors $f$ est diagonalisable si et seulement si tout sous-espace vectoriel de $E$ admet un supplémentaire stable par $f$.
\end{prop}

\begin{preuve}
    \marginnote[0cm]{Source : à compléter}
    \begin{itemize}
        \item[$(\Leftarrow)$] Supposons $f$ diagonalisable. Soit $F$ un sous-espace vectoriel quelconque de $E$. Soit $(e_1, \dots, e_m)$ une base de $F$. Puisqu'on a supposé $f$ diagonalisable, il exsite une base $(v_1, \dots, v_n)$ de $E$ formée de vecteurs propres pour $f$. D'après le théorème de la base incomplète, on peut compléter la base $(e_1, \dots, e_m)$ de $F$ en une base $(e_1, \dots, e_m, e_{m+1}, \dots, e_n)$ de $E$, où on rajouté uniquement des vecteurs de notre base de vecteurs propres (c'est-à-dire $e_{m+1}, \dots, e_n$ ont été pris parmi les $v_i$). En effet, la famille $(e_1, \dots, e_m)$ est libre, et elle est contenue dans la famille génératrice $(e_1, \dots, e_m, v_1, \dots, v_n)$, donc il existe une famille $\mathcal{F}$ de vecteurs de $E$, telle que 
        $$(e_1, \dots, e_m) \subseteq \mathcal{F} \subseteq (e_1, \dots, e_m, v_1, \dots, v_n)$$
        et $\mathcal{F}$ à la fois libre et génératrice. \\
        \textcolor{red}{il manque une explication supplémentaire} \\
        On prend alors $G \defeq \Vect(e_{m+1}, \dots, e_n)$. C'est un supplémentaire de $F$ dans $E$, et il est stable par $f$ car $e_{m+1}, \dots, e_n$ sont des vecteurs propres de $f$.
        \item[$(\Rightarrow)$] Réciproquement, supposons que tout sous-espace vectoriel de $E$ admet un supplémentaire stable par $f$. Considérons
        $$F \defeq \bigoplus_{\lambda \in \Sp f} \Ker(f - \lambda \Id)$$
        le sous-espace vectoriel de $E$ formé de la somme directe des sous-espaces propres de $f$. Si $f$ n'était pas diagonalisable, $F$ serait strictement inclus dans $E$. Soit $H$ un hyperplan de $E$ contenant $F$. Alors par hypothèse $H$ admet un supplémentaire stable par $f$. Ce supplémentaire est une droite, engendrée par un vecteur propre de $f$. Mais c'est une contradiction car tous les vecteurs propres de $f$ sont dans $H$. Ainsi, $f$ est diagonalisable. 
    \end{itemize}
\end{preuve}

Nous allons maintenant définir la notion d'endomorphisme semi-simple en relâchant un peu la
condition de l'exercice ci-dessus : nous allons seulement demander aux sous-espaces stables de posséder
un supplémentaire stable.

\begin{defi}{Endomorphisme semi-simple}
    Un endomorphisme $u$ est dit \emph{semi-simple} si tout sous-espace stable par $u$ admet un supplémentaire stable par $u$.
\end{defi}

\begin{prop}{Critère de diagonalisabilité sur un $\C$-ev}
    Soient $E$ un $\C$-espace vectoriel de dimension finie non nulle et $u \in \Endo(E)$. Alors l'endomorphisme $u$ est semi-simple si et seulement s'il est diagonalisable.
\end{prop}

\begin{preuve}
    \begin{itemize}
        \item[$(\Leftarrow)$] On suppose que l'endomorphisme $u$ est diagonalisable. \\
        Son polynôme caractéristique est scindé ce que l'on voit en mettant $u$ sous forme diagonale, et par invariance de $\chi$ par changement de base. \\
        Soit $F$ un sous-espace de $E$. Soit $(e_1, \dots, e_n)$ une base de vecteurs propres de $u$ et $(f_1, \dots, f_p)$ une base de $F$. Par le théorème de la base incomplète, on peut compléter la famille libre $(f_1, \dots, f_p)$ en une base de $E$ en rajoutant $n-p$ vecteurs parmi la base $(e_1, \dots, e_n)$, quitte à réindexer, on peut supposer que c'est $(e_{p+1}, \dots, e_n)$, ces vecteurs engendrent alors un sous-espace stable supplémentaire de $F$.
        \item[$(\Rightarrow)$] On construit une base de vecteurs propres de la manière suivante: prenons un hyperplan $H$ quelconque, il existe une droite stable supplémentaire, donc dirigée par un vecteur propre $e_1$. Si on a construit une famille libre de vecteurs propres $(e_1, \dots, e_k)$, on prend un hyperplan contenant $\Vect(e_1, \dots, e_k)$, et on trouve une droite stable $\Vect(e_{k+1})$ supplémentaire à $H$. On conclut par récurrence.
    \end{itemize}
\end{preuve}

Notes de la correction vue en cours.
\begin{preuve}
    \begin{itemize}
        \item[$(\Leftarrow)$] Soit $F$ un sev de $E$ stable par $f$.
        Posons $g \defeq f_{\vert F}$ qui est diagonalisable car $f$ l'est par hypothèse. \\
        Soit $\mathscr{B}_F$ une base de $F$ formée de vep de $g$. \\
        Soit $\mathscr{B}'$ une base de $E$ formée de vep de $f$ (qui existe car $f$ est diagonalisable). \\
        On complète $\mathscr{B}_F$ est une base $\mathscr{B}$ de E en prenant des vep $(\varepsilon_1, \cdots, \varepsilon_r)$ de $\mathscr{B}'$. On note $(\lambda_1, \dots, \lambda_r)$ les valeurs propres associées. \\
        On pose $G = \mathrm{Vect}(\varepsilon_1, \dots, \varepsilon_r)$. De cette manière, $F$ et $G$ sont supplémentaires. Montrons que $G$ est stable par $f$. \\
        Soit $x = \sum\limits_{i=1}^{r} \mu_i \varepsilon_i \in G$. Donc $f(x) = \sum\limits_{i=1}^{r} \mu_i f(\varepsilon_i) =  \sum\limits_{i=1}^{r} \mu_i \lambda_i \varepsilon_i \in G$ et $G$ est stable par $f$.
    
        \item[$(\Rightarrow)$] Montrons que $f$ est diagonalisable. On va montrer que $E = \bigoplus\limits_{\lambda \in \Sp(f)} E_\lambda (f)$.
        
        \begin{enumerate}
            \item \underline{Somme directe:} \\
            On pose $F = \bigoplus\limits_{\lambda \in \Sp(f)} E_\lambda (f)$ et $\Sp(f) = (\lambda_1, \dots, \lambda_r)$. \\
            Soit $x = \sum\limits_{i=1}^{r} x_i \in F$ où $x_i \in E_{\lambda_i}(f)$. Alors $f(x) = \sum\limits_{i=1}^{r} \underbrace{\lambda_i x_i}_{\in E_{\lambda_i}(f)} \in F$. Donc $F$ est stable par $f$.
            \item Montrons que $F = E$. \\
            Par hypothèse, $F$ admet un supplémentaire $G$ dans $\C^n$, stable par $f$. Montrons que $G = \{0\}$ en raisonnant par l'absurde. \\
            On pose $g = f_{\vert F}$. D'après le théorème de \textsc{D'Alembert-Gauss} sur $\C$, $g$ admet au moins une valeur propre $\mu \in \C$ de vep associé $x_\mu$. On montre que $x_\mu \in F \cap G$. Or $F$ et $G$ sont supplémentaires donc $x_\mu = 0_E$: contradiction. D'où le résultat. 
        \end{enumerate}
        \item Considérer $R_\theta$. \textcolor{green}{à revoir}
    \end{itemize}
\end{preuve}

\begin{exercice}
    Décrire un contre-exemple à la réciproque dans $\R$, en dimension 2.
\end{exercice}  

\section{Autour du commutant}
\begin{defi}[Commutant d'une matrice]
    Soient $A \in \M_n(\R)$ et $C(A) \defeq \ens[\big]{M \in \M_n(\R) \tq MA = AM }$.
\end{defi}

\begin{exercice}
    \source{\cite{exos_oraux} p. 119}
    Soit $E$ un $\K$-espace vectoriel de dimension finie et $u \in \Endo(E)$. Démontrer que $C(u)$ a une structure de $\K$-espace vectoriel puis que, si $u$ est diagonalisable:
    $$\dim C(u) = \smashoperator{\sum_{\lambda \in \Sp(u)}} \dim^2 E_\lambda(u).$$
\end{exercice}

\begin{solution}
    
\end{solution}

\begin{exercice}
    \source{\cite{acamanes} \href{https://acamanes.github.io/psi/psi_doc/exos_e11.pdf}{(Exercice 12 TD 11)}}
    Soit $A \in \M_n(\R)$.
    \begin{enumerate}
        \item 
        \begin{enumerate}
            \item Montrer que $C(A)$ est un sous-espace vectoriel de $\M_n(\R)$ stable par multiplication.
            \item Montrer que si $M \in C(A)$ et $M$ est inversible, alors $\Inv{M} \in C(A)$.
        \end{enumerate}
        \item Soit $D$ une matrice diagonale dont les coefficients diagonaux sont deux à deux distincts.
        \begin{enumerate}
            \item Déterminer $C(D)$.
            \item Montrer que $\big(\I_n, D, \dots, D^{n-1}\big)$ est une base de $C(D)$.
        \end{enumerate}
        \item On se limite au cas $n=2$.
        \begin{enumerate}
            \item Déterminer les matrices $A$ telles que $\dim C(A) = 4$.
            \item Montrer que $\dim C(A) \geqslant 2$. 
            \item On suppose que $\dim C(A) \geqslant 3$. En utilisant $F \defeq \Vect \big\{ \mathrm{E}_{1, 1}, \mathrm{E}_{1, 2} \big\}$ ou $G \defeq \Vect \big\{ \mathrm{E}_{2, 1}, \mathrm{E}_{2, 2} \big\}$, montrer que $A = \lambda \I_2$.
            \item Pour tout $A \in \M_2(\R)$, déterminer une base de $C(A)$.
        \end{enumerate}
    \end{enumerate}
\end{exercice}

\begin{enumerate}
    \item \emph{Montrer que C(A) est un sous-espace vectoriel de $\M_n(\R)$.} \\
    Au lieu de redémontrer les propriétés d'un sev, on peut voir $C(A)$ comme le \textbf{noyau de l'application linéaire} $M \mapsto MA - AM$ ce qui donne directement le résultat. 
    \item On veut montrer que $M^{-1} A = A M^{-1}$ i.e. $A = M A M^{-1}$ ce qui est vrai car $M A = A M$.
    \item
    \begin{itemize}
        \item $C(D) = \mathscr{D}_n$ (l'ensemble des matrices diagonales de taille $n$) \textcolor{red}{(ne pas oublier de montrer la double inclusion)}.
        \item Comme $| \mathscr{B} | = \dim C(D)$, il suffit de montrer la liberté de $\mathscr{B}$. \\
        Soit $(\lambda_0, \dots, \lambda_{n-1}) \in \R^n$ tel que $\sum\limits_{k=0}^{n-1} \lambda_k D^k = 0_n$. \\
        \textcolor{green}{Revoir le caractère générateur avec les polynômes d'interpolation.}
        \begin{itemize}
            \item Pour tout $i \in \llbracket 1, n \rrbracket$, $\sum\limits_{k=0}^{n-1} \lambda_k d_i = 0_n \quad (*)$. Donc le polynôme $P = \sum\limits_{k=0}^{n-1} \lambda_k X^k$ qui est de dégré $n-1$ et prossède $n$ racines distinctes et est donc le polynôme nul. On en déduit que les $\lambda_i$ sont tous nuls. La famille $\mathscr{B}$ est bien libre et forme une base de $C(D)$.
            \item Les relations $(*)$ forment un système de \nom{Vandermonde} de $n$ équations à $n$ inconnues. Comme les coefficients $d_i$ sont deux à deux distincts, le système est inversible et son unique solution est le vecteur colonne nul.
        \end{itemize}
    \end{itemize}
    \item On se limite au cas $n = 2$. 
    \begin{enumerate}
        \item Déterminer les matrices $A$ telles que $\dim C(A) = 4$. \\
        $C(A) = \M_2(\R)$ car $C(A) \subset \M_2(\R)$ et il y égalité des dimensions. \\
        \ptnclegras{Évaluer les commutant en les matrices de la base canoniques de $\M_2(\R)$}: on trouve que A est scalaire. \\
        \textcolor{red}{Ne pas oublier de montrer la réciproque}. 
        \item Montrer que $\dim C(A) \geqslant 2$. \\
        Si $A$ est scalaire, cf. question précédente. \\
        Sinon montrer que la famille $\big\{ \I_2, A \big\} \subset C(A)$ est libre. 
        \item Enoncé... \\
    \end{enumerate}
\end{enumerate}

\section{Que dire si \texorpdfstring{$M^2$}{M^2} est diagonalisable ?}
\begin{prop}
    Soit $n \in \Ne$ et $M \in \M_n(\C)$. On suppose que $M^2$ est diagonalisable. Alors $M$ est diagonalisable si et seulement si $\Ker(M) = \Ker(M^2)$.
\end{prop}

\begin{preuve}
    On note $f$ l'endomorphisme canoniquement associé à $M$. 
    \begin{itemize}
        \item 
        %$(\Rightarrow)$ On suppose que $M$ est diagonalisable (et donc $f$ aussi). Notons $(\lambda_1, \dots, \lambda_n) \in \C^n$ les valeurs propres de $f$. Il existe une base $\mathscr{B}$ telle que $\mathrm{Mat}_{\mathscr{B}}(f) = \mathrm{Diag}(\lambda_1, \dots, \lambda_n)$. Donc $\mathrm{Mat}_{\mathscr{B}}(f^2) = \mathrm{Diag}(\lambda_1^2, \dots, \lambda_n^2)$. \\
        Montrons que $\Ker(f^2) \subset \Ker(f)$ (l'autre inclusion est toujours vraie). \\
        Deux méthodes (je crois que la deuxième fonctionne...)
        \begin{itemize}
            \item On va montrer \textbf{l'égalité des dimensions}. Soit $\mathscr{B}$ une base de diagonalisation de $u$. \textbf{Le rang d'une matrice diagonale étant le nombre de coefficients diagonaux non nuls}, $\Rg \left(\mathrm{Mat}_\mathscr{B}(u^2) \right) = \Rg \left(\mathrm{Mat}_\mathscr{B}(u) \right)$ donc $\Rg(u^2) = \Rg(u)$. Par le \textbf{théorème du rang}, il s'ensuit que $\dim \Ker(u^2) = \dim \Ker(u)$.
            \item Soit $X \in \Ker(M^2)$ i.e. $M^2 X = 0\ (*)$. Montrons que $MX = 0$. L'idée est de \textbf{faire apparaître un produit scalaire} sur l'ensemble des vecteurs colonnes d'une matrice i.e. une produit de la forme $N^\top N$. \\
            Comme $M^2$ est diagonalisable, il existe $P$ inversible et $D$ diagonale telles que $M^2 = PDP^{-1}$. En remplaçant $M^2$ par cette expression dans $(*)$ puis en multipliant à gauche successivement par $P^{-1}$,  $(P^{-1})^\top$ et $X^\top$ on trouve $X^\top (P^{-1})^\top D^2 P^{-1} X = 0$ soit 
            $$(D P^{-1} X)^\top (D P^{-1} X) = 0.$$
            Comme $(C, C') \mapsto C^\top \times C'$ définit un produit scalaire sur l'espace des vecteurs colonnes, on a $D P^{-1} X = 0$ car il est orthogonal à lui-même et donc, en multipliant à gauche par $P$, on obtient bien $MX = 0$. 
        \end{itemize}
        \item $(\Leftarrow)$ On suppose que $\Ker(M) = \Ker(M^2)$. \\
        Raisonner par analyse-synthèse: soit $\lambda$ une valeur propre non nulle de $f^2$. Notons $\mu$ une racine carrée complexe de $\lambda$. Montrons que $E_{\lambda}(f^2) = E_{\mu}(f) \oplus E_{-\mu}(f)$. \\
        On pose $y = \frac{x}{2} + \frac{f(x)}{2 \mu}$ et $z = \frac{x}{2} - \frac{f(x)}{2 \mu}$. \\
        Comme $f^2$ est diagonalisable, $E$ est la somme directe des sous-espaces propres de $f^2$. On décompose chacun de ces sep comme ci-dessus et on en déduit que $E$ est la somme directe des sep de $f$ i.e. $f$ est diagonalisable. 
        \item $(\Leftarrow)$ \cite{reduc_des_endo} p. 100 \\
        Comme $u^2$ est diagonalisable, il existe $Q$ scindé à racines simples vérifiant $Q(0) \not= 0$ tel que $X Q(X)$ annule $u^2$, c'est-à-dire $u^2 \circ Q(u^2) = 0_{\Endo(E)}$. \\
        Alors, pour tout $x \in E$, $Q(u^2)(x) \in \Ker(u^2)$; or $\Ker(u^2) = \Ker(u)$ par hypothèse donc $Q(u^2)(x) \in \Ker(u)$ soit $u \circ Q(u^2) (x) = 0_E$. \\
        Ainsi, $XQ(X^2)$ est un polynôme annulateur de $u$. Il suffit de remarquer que ce polynôme est scindé à racines simples (car les racines complexes de $Q$ sont deux à deux distinctes et non nulles) pour conclure avec le critère algébrique de diagonalisabilité que $u$ est diagonalisable. \\
        \textsl{Une démonstration alternative consiste à montrer que, pour tout $\lambda \in \Ce$ de racines carrées distinctes $\mu$ et $\mu'$, le sous-espace propre $u^2$ associé à $\lambda$ se décompose avec les sous-espaces propres $u$:
        $$\Ker(\lambda \Id_E - u^2) = \Ker(\mu \Id_E - u) \oplus \Ker(\mu' \Id_E - u).$$
        La condition porte alors que le sous-espace propre de $u^2$ associé à $0$, c'est-à-dire $\Ker(u^2)$.
        }
    \end{itemize}
\end{preuve}


\section{Raciné carrée d'une matrice}
\url{https://share.miple.co/content/CtwFAB5leFp4M}

\begin{box_titre}{DS6}
    On note $\mathrm{Rac}(A) = \{ R \in \M_n(\R),\ R^2 = A \}$. \\
    $\blacktriangleright$ Soit $A \in \M_n(\R)$. $\mathrm{Rac}(A)$ est une partie fermée de $\M_n(\R)$. \\
    $\blacktriangleright$ $\mathrm{Rac}(\I_n)$ n'est pas une partie bornée de $\M_n(\R)$ pour $n \geqslant 3$. 
\end{box_titre}

\begin{box_titre}{Racine carrée de matrices symétriques positives}
    Pour tout $A \in \mathscr{S}^+(\R)$, il existe une unique matrice $B \in \mathscr{S}^+(\R)$ telle que $A = B^2$. 
\end{box_titre}

\underline{Exercice 11, TD 11:}\\
Soit $A = 
\begin{pmatrix}
    -1 & 2 & 3 \\
    0 & - 1 & 4 \\
    0 & 0 & 1
\end{pmatrix}. 
$ Montrer que $A$ n'a pas de racine dans $\M_3(\R)$. 

\section{Réduction d'une matrice creuse}
\begin{tcolorbox}
    Soit $n \geqslant 2$. On pose:
    $$
    A = 
        \begin{pmatrix}
              &        &   & c \\
              &  (0)   &   & \vdots \\
              &        &   & c \\
            b & \cdots & b & a
        \end{pmatrix}
        \in \M_n(\R).
    $$
    Étudier la possibilité de diagonaliser $A$ sur $\R$.
\end{tcolorbox}

\underline{Remarques:}\\
$\blacktriangleright$ Si $b = c$ alors $A$ est symétrique réelle donc diagonalisable. \\
$\blacktriangleright$ $A$ est au plus de rang 2. Donc par le \textbf{théorème du rang}, 0 est une valeur propre de $A$ de multiplicité au moins $n-2$.

\section{Vecteurs propres de \texorpdfstring{$\Trsp{\com(A)}$}{la transposée de la comatrice}}
\begin{exercice}
    Soit $A \in \M_n(\K)$. Montrer que 
    $$\Sp(A) \subset \Sp \big( \Trsp{\com(A)} \big).$$
\end{exercice}
\marginnote[0cm]{
    $$A \times\ \Trsp{\com(A)} = \Trsp{\com(A)} \times A = \det(A) \I_n$$
}
\begin{elem_sol}
    Soit $\lambda \in \Sp(A)$ et $X$ un vecteur propre associé. Alors $\lambda (BX) = \det(A) X$. 
        
    \begin{itemize}
        \item $\lambda \not = 0$: $BX = \frac{\det(A)}{\lambda}X$. 
        \item $\lambda = 0$: alors $\det(A) = 0$ et $\Rg(A) \leqslant n-1$. 
        \begin{itemize}
            \item $\Rg(A) \leqslant n-2$: supposer par l'absurde qu'il existe un déterminant mineur de $A$ non nul. 
            \item $\Rg(A) = n-1$: $X \in \Ker(A)$ qui est une droite vectorielle. 
            $$A \text{ et } B \text{ commutent donc } \Ker(A) \text{ est stable par } B$$
        \end{itemize}
    \end{itemize}
\end{elem_sol}


\section{Éléments propres de \texorpdfstring{$MN$}{MN}, de \texorpdfstring{$NM$}{NM}}
\begin{prop}{}
    Soient $M$ et $N$ dans $\M_n(\K)$.
    \begin{itemize}
        \item $$0 \in \Sp(MN) \iff 0 \in \Sp(NM)$$
        \item Soit $\lambda \in \Ke$,
        $$\dim E_\lambda(MN) = \dim E_\lambda(NM)$$
    \end{itemize}
\end{prop}

Soit $M, N \in \M_n(\K)$. 
\begin{enumerate}
    \item ...
    \item \emph{Soit $\lambda \in \Ke$, montrer que $\dim(E_\lambda (MN)) = \dim(E_\lambda (NM))$.} \\
    On remarque que si $X \in E_\lambda (MN)$ alors $NX \in E_\lambda (NM)$. On pose alors:
    $$
        \fonction[\varphi]{E_\lambda (MN)}{E_\lambda (NM)}{X}{NX}.
    $$
    On montre que $\varphi$ est injective et on en déduit que $\dim(E_\lambda (MN)) \leqslant \dim(E_\lambda (NM))$. Par symétrie des rôles de $M$ et de $N$, on montre l'inégalité dans le sens inverse et on en déduit l'égalité.
    \item ...
    \item ...
\end{enumerate}


% \section{Spectre de \texorpdfstring{$\Id_E-uv$ et $\Id_E-vu$}{IdE-uv et IdE - vu}} \label{spectre_I-uv_et_I-vu}
% \begin{exercice}
    \marginnote[0cm]{\cite{exos_oraux}}
    Soient $u$ et $v$ deux endomorphismes d'un espace $E$ de dimension finie $n \in \Ne$. Prouver que $\Id_E - uv$ et $\Id_E - vu$ ont les mêmes valeurs propres. \\
    En déduire que $\Id_E - uv$ est inversible si $\Id_E - vu$ l'est aussi, relier les inverses.
\end{exercice}

Soient $u$ et $v$ deux endomorphismes d'un espace $E$ de dimension $n$.

\begin{enumerate}
    \item Montrer que $\Id_E-uv$ et $\Id_E-vu$ ont les mêmes valeurs propres.
    \begin{itemize}
        \item Montrer que ces deux endomorphismes ont même polynôme caractéristique en posant les deux matrices
        $$
        A = 
        \begin{pmatrix}
            U & \lambda \I_n \\
            \I_n & V
        \end{pmatrix}
        \text{ et }
        B = 
        \begin{pmatrix}
            V & -\lambda \I_n \\
            -\I_n & 0_n
        \end{pmatrix}.
        $$
        \begin{align*}
            \det(AB) &= \det(BA) \\
            (-\lambda)^n \det(AB - \lambda \I_n) &= (-\lambda)^n \det(BA - \lambda \I_n)
        \end{align*}
    \end{itemize}
    \item En déduire que $\Id_E-uv$ est inversible si et seulement si $\Id_E-vu$ l'est, relier les inverses. 
    \begin{itemize}
        \item $f \in \Gl(E) \Longleftrightarrow 0 \not \in \Sp(f)$
        \item Évaluer le résultat précédent en $\lambda = 1$.
        \item Analogie avec l'inversion par sommation géométrique des endomorphismes nilpotents (cf. \nameref{indice_nilpotence})
    \end{itemize}
\end{enumerate}

\section{Réduction simultanée}
\begin{defi}[Endomorphismes codiagonalisables/cotrigonalisables]
    Soit $E$ un espace vectoriel de dimension finie. Soient $u$ et $v$ deux endomorphismes de $E$ diagonalisables (resp. trigonalisables). \\
    On dit que $u$ et $v$ sont \emph{codiagonalisables} (resp. \emph{cotrigonalisables}) s'ils sont diagonalisables (resp. trigonalisables) dans une même base de $E$.
\end{defi}

\begin{prop}[CNS de réduction simultanée]
    Soient $u$ et $v$ deux endomorphimes diagonalisables (resp. trigonalisables). Alors, 
    $$u, v \text{ codiagonalisables (resp. cotrigonalisables)} \iff u, v \text{ commutent}.$$
\end{prop}

\begin{demo} 
    \source{\cite{acamanes} (Thème \emph{Diagonalisation simultanée} Ch. 11)}
    Montrons la résultat pour la codiagonalisation. Raisonnons par double implication.
    \begin{itemize}
        \item[$(\Rightarrow)$] Supponsons que les endomorphismes $u$ et $v$ sont codiagonalisables. On note $D$ (resp. $\widetilde{D}$) la matrice diagonale de $u$ dans une base de $E$ (resp. celle de $v$ dans cette même base). Comme ces matrices sont diagonales, $D \times \widetilde{D} = \widetilde{D} \times D$ \note \marginnote[0cm]{\note Le produit de matrices diagonales commute.} et on en déduit que $u$ et $v$ commutent.
        \item[$(\Leftarrow)$] Supposons que les endomorphismes $u$ et $v$ commutent. \\ 
        Notons $\mathrm{Sp}(u) \defeq \ens[\big]{ \lambda_i \tq i \in \llbracket1, p \rrbracket }$. Comme $u$ est diagonalisable, 
        $$E = \bigoplus\limits_{i = 1}^{p} E_{\lambda_i}(u).$$
        Soit $i \in \llbracket 1, p \rrbracket$. Comme $v$ commute avec $u - \lambda_i \mathrm{Id}_E$ \note 
        \marginnote[0cm]{
            \begin{align*}
                \note v \circ (u - \lambda_i \Id_E) &= v \circ u - \lambda_i v \\
                u \text{ et } v \text{ commutent }&= u \circ v - \lambda_i v \\
                &= (u - \lambda_i \Id_E) \circ v.
            \end{align*}
        }, $E_{\lambda_i}(u) \defeq \Ker(u - \lambda_i \Id_E)$ est stable par $v$ \note. \\
        En notant $v_i$ l'endomorphisme induit par $v$ sur $E_{\lambda_i}(u)$, comme $v$ est diagonalisable, $v_i$ est aussi diagonalisable. Ainsi, il existe $(e_{i, 1}, \dots, e_{i, r_i})$ une base de $E_{\lambda_i}(u)$ formée de vecteurs propres de $v_i$. De plus, $e_{i, j} \in E_{\lambda_i}(u) = \Ker(u - \lambda_i \Id_E)$. Ainsi, $u(e_{i,j}) = \lambda_i e_{i,j}$ et $e_{i,j}$ est un vecteur propre de $u$. \\
        Finalement, $(e_{i, 1}, \dots, e_{i, r_i})_{1 \leqslant i \leqslant p}$ est une base de $E$ constituée de vecteurs propres de $u$ et de $v$. Ainsi, $u$ et $v$ sont diagonalisables dans cette même base. 
    \end{itemize}
\end{demo}

\begin{remarque}
    \source{\cite{objectif_agregation} p. 167}
    \begin{itemize}
        \item La proposition établit que la commutativité est une condition suffisante pour réduire simultanément plusieurs endomorphismes. Remarquez que dans le cas de la diagonalisabilité, elle est aussi nécessaire: si des endomorphismes sont codiagonalisables, alors ils commutent. 
        \item Le seul fait que deux endomorphismes commutent n'implique par leur codiagonalisabilité, ni leur cotrigonalisabilité! Il ne faut pas oublier l'hypothèse faite sur chacun des endomorphismes (diagonalisabilité ou trigonalisabilité).
        \item La preuve de la proposition généralise la démonstration classique de la trigonalisabilité d'un endomorphisme dont le polynôme caractéristique est scindé: on raisonne par récurrence sur la dimension de l'espace. 
        \item Réduire simultanément plusieurs endomorphismes est très utile lorsqu'il s'agit de les ajouter ou de les composer. La proposition permet de démontrer, par exemple, que la somme d'endomorphismes trigonalisables qui commutent est encore trigonalisable, ou que la composée d'endomorphismes diagonalisables qui commutent est encore diagonalisables. 
        \item Supposons que $E$ soit un $\C$-espace vectoriel muni d'un produit scalaire hermitien. Soit $(f_i)_{i \in I}$ une famille d'endomorphismes de $E$ (qui sont donc trigonalisables). Si les $f_i$ commutent deux à deux, alors il existe une base orthonormée de cotrigonalisabilité. 
    \end{itemize}
\end{remarque}

\marginnote[-5cm]{
    \begin{theo}{\note Commutativité \& Stabilité}
        Soient $\varphi$ et $\psi$ deux endomorphismes qui commutent. Alors $\Im \varphi$ et $\Ker{\varphi}$ sont stables par $\psi$.
    \end{theo}
}

\begin{exercice}
    Soient $A$ et $B$ deux matrices de $\M_n(\C)$. On pose $\Phi_{A,B}$ l'endomorphisme de $\M_n(\C)$ défini pour tout $M \in \M_n(\C)$ par
    $$\Phi_{A,B}(M) \defeq AM + MB.$$
    \begin{enumerate}
         \item On suppose que la matrice $A$ est diagonalisable et que $B = 0$. Montrer que $\Phi_{A, B}$ est diagonalisable.
        \item On suppose que les matrices $A$ et $B$ sont diagonalisables. Montrer que $\Phi_{A, B}$ est diagonalisable. 
    \end{enumerate}
\end{exercice}

\begin{solution}
    \begin{enumerate}
        \item \textbf{1$^\text{ère}$ méthode.} Nous allons utiliser l'équivalence
        $$A \text{ diagonalisable} \iff \exists P \in \mathrm{Ann}(A) \text{ \textsc{sars}}.$$
        On remarque que $\Phi_{A, 0}: M \mapsto AM$. On montre par récurrence que 
        $$\Phi_{A, 0}^r(M) = A^r M.$$
        Ainsi, si $P$ est un polynôme, $P(\Phi_{A, 0}) = \Phi_{P(A), 0}$ \note. \\
        \marginnote[-1cm]{
            \note On note $P = \sum\limits_{k=0}^n p_k X^k$. Alors
            \begin{align*}
                P(\Phi_{A, 0}) &= \sum_{k=0}^n p_k \Phi_{A, 0}^k \\
                &= \sum_{k=0}^n p_k A^k M \\
                &= \left( \sum_{k=0}^n p_k A^k \right) M \\
                &= \Phi_{P(A), 0}.
            \end{align*}
        }
        Comme $A$ est diagonalisable, il existe un polynôme $P$ scindé à racines simples qui est un polynôme annulateur de la matrice $A$. Ainsi, $P(\Phi_{A, 0}) = \Phi_{0, 0} = 0_{\Endo \left( \M_n(\C) \right)}$. Comme $P$ est un polynôme scindé à racines simples qui est annulateur de $\Phi_{A, 0}$, alors $\Phi_{A, 0}$ est diagonalisable. \\
        \textbf{2$^\text{e}$ méthode.} Nous allons utiliser l'équivalence 
        $$A \text{  diagonalisable} \iff \exists \text{ une base de $E$ formée de vep de $A$}.$$
        Comme $A$ est diagonalisable, il existe une base $(V_1, \dots, V_n)$ de $\M_{n,1}(\C)$ formée de vecteurs propres de $A$ respectivement associés aux valeurs propres $(\lambda_1, \dots, \lambda_n)$. On définit en colonnes la matrices $M_{k, \ell} \defeq [ 0 \cdots 0\ V_k\ 0 \cdots 0]$ où $V_k$ est la colonne $\ell$ de la matrice. En effectuant une multiplication par blocs, on vérifie que $\Phi_{A, 0}(M_{k, \ell}) = A M_{k, \ell} = \lambda_k M_{k, \ell}$. On vérifie ensuite aisément que $(M_{k, \ell})_{1 \leqslant k, \ell \leqslant n}$ est une base de $\M_n(\C)$. On a ainsi exhibé une base de $\M_n(\C)$ formée de vecteurs propres de $\Phi_{A, 0}$. 
        \item On remarque que $\Phi_{A, 0}$ et $\Phi_{0, B}$ commutent. \\
        D'après la question précédente, $\Phi_{A, 0}$ est diagonalisable et on montrerait de même que $\Phi_{0, B}$ est diagonalisable. Ainsi, d'après la propriété de diagonalisation simultanée, $\Phi_{A, 0}$ et $\Phi_{0, B}$ sont simultanément diagonalisables. \\
        En notant $(M_k)$ une base de vecteurs propres communs, alors
        $$\Phi_{A, B} (M_k) = \Phi_{A, 0} (M_k) + \Phi_{0, B} (M_k) = (\lambda_k + \mu_k) M_k$$
        et $(M_k)$ est une base de vecteurs propres de $\Phi_{A, B}$. Ainsi, $\Phi_{A, B}$ est diagonalisable. 
    \end{enumerate}
\end{solution}

\marginnote[-10cm]{
    \begin{kaobox}[frametitle=Décomposition de $E$ en somme de sous-espaces stables supplémentaire]
        Si $E$ est de dimension finie non nulle et $\chi_f = \prod\limits_{i=1}^{k}(f-\lambda_i)^{\alpha_i}$, alors
        $$E = \bigoplus_{i=1}^{k} \Ker(f-\lambda_i \Id_E)^{\alpha_i}.$$
        \begin{itemize}
            \item Les $\Ker(f-\lambda_i \Id_E)^{\alpha_i}$ sont supplémentaires et stables par $f$. Donc, dans tout base adaptée à cette décomposition, la matrice de $f$ est diagonale par blocs. 
            \item La restriction de $f$ à $\Ker(f-\lambda_i)^{\alpha_i}$ induit un endomorphisme $f_i$ de ce sous-espace. $f_i$ admet une et une seule valeur propre à savoir $\lambda_i$ et $f_i - \lambda_i \Id_{\Ker(f-\lambda_i)^{\alpha_i}}$ est nilpotente d'indice inférieur ou égal à $\alpha_i$. 
        \end{itemize}
    \end{kaobox}
    Source: fiche de \cite{maths-france}.
}

\section{Critère de nilpotence par la trace} \labsec{critere_de_nilpotence_par_la_trace}
Les résultats que nous allons montrer par la suite peuvent être résumés par le diagramme suivant:
\begin{figure*}[h!]
    $$
    \begin{tikzcd}
    	& {\text{si pour tout } k \in \llbracket 1, n-1 \rrbracket, \mathrm{Tr}(A^k)=0 \text{ et si \dots}} & {} \\
    	{A \text{ est nilpotente}} && {A \text{ est diagonalisable}}
    	\arrow["{\dots \mathrm{Tr}(A^n) \not= 0}"{description}, curve={height=6pt}, Rightarrow, from=1-2, to=2-3]
    	\arrow["{\dots \mathrm{Tr}(A^n)=0}"{description}, curve={height=-6pt}, Rightarrow, 2tail reversed, from=1-2, to=2-1]
    \end{tikzcd}
    $$
\end{figure*}

Intéressons-nous d'abord à la branche de gauche.
\begin{prop}{}
    Soit $A \in \M_n(\K)$. La matrice $A$ est nilpotente si et seulement si pour tout $k \in \llbracket 1, n \rrbracket, \Tr(A^k) = 0$.
\end{prop}
\begin{preuve}
    \marginnote[-1cm]{\url{http://vonbuhren.free.fr/Agregation/Developpements/dev_thm_burnside.pdf}}
    Raisonnons par double implication.
    \marginnote[1cm]{
        \note Soit $A$ une matrice semblable à
        $$
        \begin{pmatrix}
            \lambda_1 & \star & \cdots & \star \\
            0 & \lambda_2 & \cdots & \star \\
            \vdots & \ddots &\ddots & \vdots \\
            0 & \cdots & 0 & \lambda_n
        \end{pmatrix}.
        $$
        Alors la matrice $A^k$ est semblable à
        $$
        \begin{pmatrix}
            \lambda_1^k & \star & \cdots & \star \\
            0 & \lambda_2^k & \cdots & \star \\
            \vdots & \ddots &\ddots & \vdots \\
            0 & \cdots & 0 & \lambda_n^k
        \end{pmatrix}.
        $$
    }
    
    %\marginnote[3cm]{
    % https://tex.stackexchange.com/questions/343439/how-to-draw-this-special-matrix-with-two-diagonal-braces
    %}
    \begin{itemize}
        \item[$(\Rightarrow)$] Si la matrice $A$ est nilpotente, son spectre est réduit à $0$ et donc elle est semblable à une matrice strictement triangulaire $T$. Pour tout $k \in \llbracket 1, n \rrbracket$, la matrice $A^k$ est semblable à la matrice $T^k$ dont la diagonale est nulle \note. La trace étant un invariant de similitude, on en déduit que pour tout $k \in \llbracket 1, n \rrbracket$, $\Tr(A^k) = \Tr(T^k) = 0$.
        \item[$(\Leftarrow)$] Réciproquement, supposons que la matrice $A$ n'est pas nilpotente. On désigne par $\lambda_1, \dots, \lambda_r \in \C$ les valeurs propres non nulles deux à deux distinctes de $A$ (qui existent car le polynôme $\chi_A \in \C[X]$ est scindé) et $m_1, \dots, m_r \in \Ne$ leur multiplicité respective. En trigonalisant la matrice $A$, notre hypothèse équivaut à
        $$\forall k \in \llbracket 1, n \rrbracket,\ \sum_{i=1}^r m_i \lambda_i^k = 0.$$
        En effet, la valeur propre $\lambda_i$ est présente $m_i$ fois sur la diagonale. \\
        En particulier, en se limitant à $k \in \llbracket 1, r \rrbracket$, ces relations se traduisent matriciellement par
        $$
        \underbrace{
        \begin{pmatrix}
        \lambda_1 & \cdots & \lambda_r \\
        \vdots & & \vdots \\
        \lambda_1^r & \cdots & \lambda_r^r
        \end{pmatrix}
        }_{\defeq V}
        \underbrace{
        \begin{pmatrix}
            m_1 \\ \vdots \\ m_r
        \end{pmatrix}
        }_{\defeq X}
        = 
        \begin{pmatrix}
        0 \\ \vdots \\ 0
        \end{pmatrix}.
        $$
        
        %$$
        %A^k \sim
        %\begin{tikzpicture}[decoration={brace,amplitude=5pt},baseline=(current bounding box.west)]
        %\matrix (magic) [matrix of math nodes,left delimiter=(,right delimiter=)] {
        %\lambda_1^k \\
        %& \ddots \\
        %& & \lambda_1^k \\
        %& & & \ddots \\
        %& & & & \lambda_r^k \\
        %& & & & & \ddots \\
        %& & & & & & \lambda_r^k \\
        %};
        %\draw[decorate] (magic-1-1.north) -- (magic-3-3.north east) node[above=5pt,midway,sloped] {$m_1$};
        %\draw[decorate] (magic-5-5.north east) -- (magic-7-7.north east) node[above=5pt,midway,sloped] %{$m_r$};
        %\end{tikzpicture}
        %$$

        Ainsi $X \in \Ker(V)$. Or la matrice $V$ est une matrice de \textsc{Vandermonde} inversible car les $\lambda_i$ sont deux à deux distinctes par hypothèse. Nous aboutissons alors à une contradiction car le vecteur $X$ est non nul. On en déduit que la matrice $A$ est nilpotente. \\
        
        Voyons une autre démonstration pour la réciproque. \\ \marginnote[0cm]{\url{https://www.youtube.com/watch?v=d70IfThN_-A}}
        \item[$(\Leftarrow)$] Montrons par récurrence que la matrice $A$ est nilpotente. Plus précisément pour $n \in \N$ on note
        \begin{center}
            $\mathscr{P}_n$: \say{ Soit $A \in \M_n(\K)$. Si pour tout $k \in \llbracket 1, n \rrbracket, \Tr(A^k) = 0$ alors la matrice $A$ est nilpotente }.
        \end{center}
        Montrons d'abord le lemme suivant.
        \begin{lemme}
            Soit $A \in \M_n(\K)$ telle que pour tout $k \in \llbracket 1, n \rrbracket, \Tr(A^k) = 0$. Montrons que $0 \in \Sp(A)$.
        \end{lemme}
        On note $\chi_A(X) \defeq \sum\limits_{k=0}^n a_k X^k$. D'après le théorème de \textsc{Cayley}-\textsc{Hamilton}, $\sum\limits_{k=0}^n a_k A^k = 0$. En composant cette relation par la trace, qui est linéaire, et d'après les hypothèses, 
        $$a_n \times 0 + \cdots + a_1 \times 0 + a_0 \times n = 0.$$
        D'où $a_0 = 0$ et $\chi_A(0) = 0$ donc $0$ est racine du polynôme caractéristique de la matrice $A$ i.e. $0 \in \Sp(A)$. \\
        
        Revenons à la démonstration de la récurrence.
        \begin{itemize}
            \item[$\rhd$] Initialisation pour $n=1$ : $\Tr(A) = 0$ donc $A = 0$ et $A$ est nilpotente. 
            \item[$\rhd$] Hérédité: soit $A \in \M_n(\K)$ telle que pour tout $k \in \llbracket 1, n \rrbracket, \Tr(A^k) = 0$. \\
            D'après le lemme, $0 \in \Sp(A)$. Soit $u$ un vecteur propre de $A$ associé à $0$ et soit $(u, u_2, \dots, u_n)$ une base de $\M_{n,1}(\K)$. Alors en notant $B$ une matrice carrée de taille $n-1$, 
            $$A \sim 
            \begin{pmatrix}
            0 & \star & \cdots & \star \\
            \vdots & & B & \\
            0 & & &
            \end{pmatrix}
            \text{ et }
            A^k \sim 
            \begin{pmatrix}
            0 & \star & \cdots & \star \\
            \vdots & & B^k & \\
            0 & & &
            \end{pmatrix}.
            $$
            D'après les hypothèses sur la trace des $A^k$, pour tout $k \in \llbracket 1, n-1 \rrbracket, \Tr(B^k) = 0$. Ainsi, en appliquant $\mathscr{P}_{n-1}$, la matrice $B$ est nilpotente. On en déduit, d'après les opérations sur les matrices par blocs, que
            $$\chi_A(X) = X \chi_{B}(X) = X \times X^{n-1} = X^n.$$
            Ainsi, d'après le théorème de \textsc{Cayley}-\textsc{Hamilton}, la matrice $A$ est nilpotente. 
        \end{itemize}
    \end{itemize}
\end{preuve}

Intéressons-nous maintenant à la branche de droite du diagramme.
\begin{prop}{}
    Soit $A \in \M_n(\K)$. Si pour tout $k \in \llbracket 1, n-1 \rrbracket, \Tr(A^k) = 0$ et $\Tr(A^n) \not= 0$ alors la matrice $A$ est diagonalisable. 
\end{prop}
Le démonstration de ce résultat reprend la démarche de la première démonstration de la réciproque du résultat précédent.
\newcommand{\vandermondepartiel}{
\left(\begin{gathered}
    \tikzpicture[every node/.style={anchor=south west}]
        \node[minimum width=1.5cm,minimum height=0.5cm] at (0.125,1.25) {\LARGE $V_k$};
        
        \node[minimum width=0.5cm,minimum height=0.5cm] at (0,0) {$\star$};
        \node[minimum width=0.5cm,minimum height=0.5cm] at (0.55,0) {$\cdots$};
        \node[minimum width=0.5cm,minimum height=0.5cm] at (1.25,0) {$\star$};
        
        \node[minimum width=0.5cm,minimum height=0.5cm] at (0,0.375) {$\vdots$};
        \node[minimum width=0.5cm,minimum height=0.5cm] at (1.25,0.375) {$\vdots$};
        
        \node[minimum width=0.5cm,minimum height=0.5cm] at (0,0.75) {$\star$};
        \node[minimum width=0.5cm,minimum height=0.5cm] at (0.55,0.75) {$\cdots$};
        \node[minimum width=0.5cm,minimum height=0.5cm] at (1.25,0.75) {$\star$};

        \draw (0, 1.25) -- (1.75, 1.25);
    \endtikzpicture
    \end{gathered}\right)
}
\begin{preuve}
    \begin{itemize}
        \item Montrons d'abord que la matrice $A$ possède au moins une valeur propre non nulle. \\
        La matrice $A$ est trigonalisable dans $\C$ et on note $\lambda_1, \dots, \lambda_n$ ses valeurs propres. Alors la matrice $A^n$ est semblable à la matrice
        $$
        \begin{pmatrix}
            \lambda_1^n & \star & \cdots & \star \\
            0 & \lambda_2^n & \cdots & \star \\
            \vdots & \ddots &\ddots & \vdots \\
            0 & \cdots & 0 & \lambda_n^n
        \end{pmatrix}.
        $$
        Or, par hypothèse, $\Tr(A^n) \not= 0$ donc les $\lambda_i$ ne peuvent pas être tous nuls et la matrice $A$ possède au moins une valeur propre non nulle.
        \item Montrons maintenant que la matrice $A$ possède $n$ valeurs propres distinctes ce qui assurera sa diagonalisabilité. \\
        On note $\alpha_1, \dots, \alpha_k$ les valeurs propres non nulles deux à deux distinctes de $A$ (qui existent d'après le premier point), $n_1, \dots, n_k$ leurs multiplicités respectives et 
        $$
        V \defeq \begin{pmatrix}
            \alpha_1 & \cdots & \alpha_k \\
            \vdots & & \vdots \\
            \alpha_1^{n-1} & \cdots & \alpha_k^{n-1}
        \end{pmatrix}
        \in \M_{n-1,k}(\C).
        $$
        Montrons que $N \defeq \Trsp{\begin{pmatrix} n_1  \cdots n_k \end{pmatrix}} \in \Ker(V)$. On calcule
        \begin{align*}
            V N = 
            \begin{pmatrix}
                \sum\limits_{i=1}^n \alpha_i n_i \\ 
                \vdots \\ 
                \sum\limits_{i=1}^n \alpha_i^{n-1} n_i
            \end{pmatrix}
            = 
            \begin{pmatrix}
                \Tr{A} \\ \vdots \\ \Tr{A^{n-1}}
            \end{pmatrix}
            =
            0_n
        \end{align*}
        Supposons par l'absurde que $k < n$. \\
        Réécrivons la relation précédente en extrayant de la matrice $V$ une matrice de \textsc{Vandermonde} carrée de taille $k$ notée $V_k$:
        $$V N = \vandermondepartiel N = 0_n$$
        et
        $$V_k N = 0.$$
        Comme $V_k$ est une matrice de \textsc{Vandermonde} et que les $\alpha_1, \dots, \alpha_k$ sont deux à deux distincts alors elle est inversible et donc
        $$N = 0.$$
        Ainsi, la matrice $A$ ne possède pas de valeur propre non nulle ce qui est absurde d'après le premier point. \\
        On en déduit que $k \geqslant n$ et comme $k \leqslant n$, $k=n$. Ainsi, la matrice $A$ possède $n$ valeurs propres distinctes et est donc diagonalisable.
    \end{itemize}
\end{preuve}

On déduit des propositions \textcolor{red}{4.6} et \textcolor{red}{4.7} le résultat suivant.
\begin{prop}{}
    Soit $A \in \M_n(\K)$. Si pour tout $k \in \llbracket 1, n-1 \rrbracket, \Tr(A^k) = 0$ alors la matrice $A$ est nilpotente ou diagonalisable.
\end{prop}

\marginnote[0cm]{(\url{https://fr.wikipedia.org/wiki/Polynôme_caractéristique#Coefficients})
$$\note \det(X \I_n - M) = X^n - f_1(M) X^{n-1} + \cdots + (-1)^n f_n(M)$$
où, en notant $(\lambda_1, \dots, \lambda_n)$ les valeurs propres de $M$ prises avec multiplicité,
$$f_k(M) = s_k(\lambda_1, \dots, \lambda_n)$$
où $s_k$ désigne le $k$-ème polynôme symétrique élémentaire. \\
Grâce aux identités de \textsc{Newton}, les coefficients $f_k(M)$ s'expriment comme des fonctions polynomiales des sommes de \textsc{Newton} des valeurs propres:
$$\sum_{i=1}^n \lambda_i^j = \Tr(M^j).$$
}

\begin{preuve}
    \textcolor{red}{à revoir} \\
    \marginnote[0cm]{\cite{reduc_des_endo} p. 114}
    D'après les formules des sommes de \textsc{Newton} et les relations coefficients-racines \note, le polynôme caractéristique de la matrice $A$ a pour expression
    $$\chi_A = X^n + (-1)^n \det(A).$$
    Si $\det(A) = 0$, le théorème de \textsc{Cayley}-\textsc{Hamilton} assure que $A$ est nilpotente. Sinon, le polynôme caractéristique est scindé à racines simples donc $A$ est diagonalisable (et d'ailleurs, toutes les valeurs propres ont le même module).
\end{preuve}


\section{Supplémentaire stable}

\begin{exercice}
    \marginnote[0cm]{\cite{reduc_des_endo} p. 102}
    Soit $u$ un endomorphisme diagonalisable de $E$ et $F$ un sous-espace de $E$. Montrer que $F$ admet un supplémentaire stable par $u$. 
\end{exercice}

\begin{solution}
    Considérons une base $(e_1, \dots, e_p)$ de $F$ et complétons cette famille libre en une base $(e_1, \dots, e_n)$ de $E$ avec des vecteurs $e_{p+1}, \dots, e_n$ issus d'une base de diagonalisation de $u$. Commes les vecteurs $e_{p+1}, \dots, e_n$ sont des vecteurs propres associés à $u$, le sous-espace $\Vect(e_{p+1}, \dots, e_n)$ est stable par $u$. Comme c'est un supplémentaire de $F$, il répond à la question. 
\end{solution} 

\section{Endomorphisme \texorpdfstring{$\mathrm{ad}_u$}{ad_u}}
\marginnote[0cm]{\cite{reduc_des_endo} p.30}
\begin{defi}{Endomorphisme $\mathrm{ad}_u$}
    Soit $u$ un endomorphisme de $E$. L'endomorphisme $\mathrm{ad}_u$ de $\Endo(E)$ est défini par
    \begin{alignat*}{2}
        \mathrm{ad}_u\ :\ \Endo(E)\ &\longrightarrow\ \Endo(E)\\
        v\ &\longmapsto\ u \circ v - v \circ u
    \end{alignat*}
\end{defi}

Cet endomorphisme nous sert essentiellement à mesurer le défaut de commutativité. Une première remarque dans ce sens est que le commutant de $u$ est $\mathscr{C}(u) = \Ker(\mathrm{ad}_u)$. 

\begin{exercice}
    \marginnote[0cm]{\cite{reduc_des_endo} p. 103}
    Soit $A \in \M_n(\K)$ une matrice diagonalisable. Montrer que $\mathrm{ad}_A$ est diagonalisable.
\end{exercice}

A rajouter:
\begin{itemize}
    \item Matrices dont le spectre est un singleton
    \item Endomorphismes symétriques à valeurs propres positives
    \item Critère de non-diagonalisabilité sur $\C$
    \item Commutant d'un endomorphisme diagonalisable 
\end{itemize}

\newpage

\begin{figure*}
    \begin{Large}
    \begin{align*}
        f \text{ diagonalisable} &\Longleftrightarrow \exists \mathscr{B} \text{ base de } E \text{ telle que } \Mat_\mathscr{B}(f) \text{ diagonale} \\
        &\Longleftrightarrow \exists \mathscr{B} \text{ base de } E \text{ formée de vep de } f\\
        &\Longleftrightarrow E = \bigoplus_{\lambda \in \Sp(f)} E_\lambda(f) \\
        &\Longleftrightarrow \dim E = \sum_{\lambda \in \Sp(f)} \dim E_\lambda(f) \\
        &\Longleftrightarrow 
        \begin{cases}
        \chi_f \text{ est scindé} \\
        \forall \lambda \in \Sp(f), m_{\lambda} = \dim E_\lambda(f)
        \end{cases} \\
        &\Longleftrightarrow \prod_{\lambda \in \Sp(f)}(X-\lambda) \in \mathrm{Ann}(f)\\
        &\Longleftrightarrow \exists P \in \mathrm{Ann}(f) \text{ SARS} \\
        &\Longleftarrow f \text{ possède } n \text{ vap distinctes} \\
        &\Longleftarrow \chi_f \text{ SARS} \\
        &\Longleftarrow \Mat(f) \in \mathscr{S}_n(\R)
    \end{align*}
    \end{Large}
\end{figure*}

\begin{figure*}[h!]
        \begin{tikzcd}[scale cd=0.85, ampersand replacement=\&]
	\& {} \& {} \& {} \\
	\& {\parbox{3.0cm}{\centering $P \in \mathrm{Ann}(f)$ \\ $\mathrm{Sp}(f) \subset \mathrm{Rac}(P)$}} \& {f \in \mathrm{GL} (E) \Leftrightarrow 0 \notin \mathrm{Sp}(f)} \\
	\& {\parbox{5.1cm}{\centering \textcolor{red}{\textsc{Spectre}}\\ $\lambda\in \mathrm{Sp}(f) \Leftrightarrow \exists x \not=0_E, f(x)=\lambda x$}} \& {} \\
	{\parbox{6.0cm}{\centering \textcolor{red}{\textsc{Polynôme caractéristique}} \\ $\chi_f(\lambda) = \det(\lambda\mathrm{Id}_E-f)$ \\ $m_f(\lambda)$: ordre de la racine $\lambda$ dans $\chi_f$}} \&\& {\parbox{4.0cm}{\centering \textcolor{red}{\textsc{Sous-espace propre}} \\ $E_{\lambda}(f)= \mathrm{Ker}(\lambda\mathrm{Id}_E-f)$}} \\
	{\parbox{7.0cm}{\centering $g$ endomorphisme induit par $f$ \\ $\chi_g |\chi _f$ \\ $\chi_f(\lambda) = \lambda^n -\mathrm{Tr}(f) \lambda^{n-1} + \cdots + (-1)^n \det(f)$}} \&\& {\parbox{7.0cm}{\centering Les sep sont des sev stables par $f$ \\ $f \circ g = g \circ f \Rightarrow E_\lambda(f)$ stable par $g$}}
	\arrow["{\dim E_\lambda(f) \leqslant m_f(\lambda)}"{description}, tail reversed, from=4-1, to=4-3]
	\arrow["{\small{\lambda \in \mathrm{Sp}(f) \Leftrightarrow \dim E_\lambda(f) \geqslant 1}}"{description, pos=0.7}, curve={height=-24pt}, tail reversed, from=3-2, to=4-3]
	\arrow[tail reversed, from=4-3, to=5-3]
	\arrow[tail reversed, from=4-1, to=5-1]
	\arrow[curve={height=-12pt}, tail reversed, from=3-2, to=2-3]
	\arrow[from=2-2, to=3-2]
	\arrow["{\parbox{4cm}{\centering \textsc{Cayley}-\textsc{Hamilton} \\ \chi_f (f) = 0}}"{description}, curve={height=30pt}, tail reversed, from=2-2, to=4-1]
	\arrow["{\small{\contour{white}{$\lambda \in \mathrm{Sp}(f) \Leftrightarrow \chi_\lambda(f) = 0$}}}"{description}, shift left=2, curve={height=24pt}, tail reversed, from=3-2, to=4-1]
\end{tikzcd}

\end{figure*}
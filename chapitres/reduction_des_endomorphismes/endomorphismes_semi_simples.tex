\begin{prop}[Critère de diagonalisabilité dans un $\R$-ev]
    Soit $E$ un $\R$-espace vectoriel de dimension finie, soit $f \in \Endo(E)$. Alors $f$ est diagonalisable si et seulement si tout sous-espace vectoriel de $E$ admet un supplémentaire stable par $f$.
\end{prop}

\begin{demo}
    \source{à compléter}
    \begin{itemize}
        \item[$(\Leftarrow)$] Supposons $f$ diagonalisable. Soit $F$ un sous-espace vectoriel quelconque de $E$. Soit $(e_1, \dots, e_m)$ une base de $F$. Puisqu'on a supposé $f$ diagonalisable, il exsite une base $(v_1, \dots, v_n)$ de $E$ formée de vecteurs propres pour $f$. D'après le théorème de la base incomplète, on peut compléter la base $(e_1, \dots, e_m)$ de $F$ en une base $(e_1, \dots, e_m, e_{m+1}, \dots, e_n)$ de $E$, où on rajouté uniquement des vecteurs de notre base de vecteurs propres (c'est-à-dire $e_{m+1}, \dots, e_n$ ont été pris parmi les $v_i$). En effet, la famille $(e_1, \dots, e_m)$ est libre, et elle est contenue dans la famille génératrice $(e_1, \dots, e_m, v_1, \dots, v_n)$, donc il existe une famille $\mathcal{F}$ de vecteurs de $E$, telle que 
        $$(e_1, \dots, e_m) \subseteq \mathcal{F} \subseteq (e_1, \dots, e_m, v_1, \dots, v_n)$$
        et $\mathcal{F}$ à la fois libre et génératrice. \\
        \textcolor{red}{il manque une explication supplémentaire} \\
        On prend alors $G \defeq \Vect(e_{m+1}, \dots, e_n)$. C'est un supplémentaire de $F$ dans $E$, et il est stable par $f$ car $e_{m+1}, \dots, e_n$ sont des vecteurs propres de $f$.
        \item[$(\Rightarrow)$] Réciproquement, supposons que tout sous-espace vectoriel de $E$ admet un supplémentaire stable par $f$. Considérons
        $$F \defeq \smashoperator{\bigoplus_{\lambda \in \Sp f}} \Ker(f - \lambda \Id)$$
        le sous-espace vectoriel de $E$ formé de la somme directe des sous-espaces propres de $f$. Si $f$ n'était pas diagonalisable, $F$ serait strictement inclus dans $E$. Soit $H$ un hyperplan de $E$ contenant $F$. Alors par hypothèse $H$ admet un supplémentaire stable par $f$. Ce supplémentaire est une droite, engendrée par un vecteur propre de $f$. Mais c'est une contradiction car tous les vecteurs propres de $f$ sont dans $H$. Ainsi, $f$ est diagonalisable. 
    \end{itemize}
\end{demo}

Nous allons maintenant définir la notion d'endomorphisme semi-simple en relâchant un peu la
condition de l'exercice ci-dessus : nous allons seulement demander aux sous-espaces stables de posséder
un supplémentaire stable.

\begin{defi}[Endomorphisme semi-simple]
    Un endomorphisme $u$ est dit \emph{semi-simple} si tout sous-espace stable par $u$ admet un supplémentaire stable par $u$.
\end{defi}

\begin{prop}[Critère de diagonalisabilité sur un $\C$-ev]
    Soient $E$ un $\C$-espace vectoriel de dimension finie non nulle et $u \in \Endo(E)$. Alors l'endomorphisme $u$ est semi-simple si et seulement s'il est diagonalisable.
\end{prop}

\begin{demo}
    \begin{itemize}
        \item[$(\Leftarrow)$] On suppose que l'endomorphisme $u$ est diagonalisable. \\
        Son polynôme caractéristique est scindé ce que l'on voit en mettant $u$ sous forme diagonale, et par invariance de $\chi$ par changement de base. \\
        Soit $F$ un sous-espace de $E$. Soit $(e_1, \dots, e_n)$ une base de vecteurs propres de $u$ et $(f_1, \dots, f_p)$ une base de $F$. Par le théorème de la base incomplète, on peut compléter la famille libre $(f_1, \dots, f_p)$ en une base de $E$ en rajoutant $n-p$ vecteurs parmi la base $(e_1, \dots, e_n)$, quitte à réindexer, on peut supposer que c'est $(e_{p+1}, \dots, e_n)$, ces vecteurs engendrent alors un sous-espace stable supplémentaire de $F$.
        \item[$(\Rightarrow)$] On construit une base de vecteurs propres de la manière suivante: prenons un hyperplan $H$ quelconque, il existe une droite stable supplémentaire, donc dirigée par un vecteur propre $e_1$. Si on a construit une famille libre de vecteurs propres $(e_1, \dots, e_k)$, on prend un hyperplan contenant $\Vect(e_1, \dots, e_k)$, et on trouve une droite stable $\Vect(e_{k+1})$ supplémentaire à $H$. On conclut par récurrence.
    \end{itemize}
\end{demo}

Notes de la correction vue en cours.
\begin{demo}
    \begin{itemize}
        \item[$(\Leftarrow)$] Soit $F$ un sev de $E$ stable par $f$.
        Posons $g \defeq f_{\vert F}$ qui est diagonalisable car $f$ l'est par hypothèse. \\
        Soit $\mathscr{B}_F$ une base de $F$ formée de vep de $g$. \\
        Soit $\mathscr{B}'$ une base de $E$ formée de vep de $f$ (qui existe car $f$ est diagonalisable). \\
        On complète $\mathscr{B}_F$ est une base $\mathscr{B}$ de E en prenant des vep $(\varepsilon_1, \cdots, \varepsilon_r)$ de $\mathscr{B}'$. On note $(\lambda_1, \dots, \lambda_r)$ les valeurs propres associées. \\
        On pose $G = \mathrm{Vect}(\varepsilon_1, \dots, \varepsilon_r)$. De cette manière, $F$ et $G$ sont supplémentaires. Montrons que $G$ est stable par $f$. \\
        Soit $x = \sum\limits_{i=1}^{r} \mu_i \varepsilon_i \in G$. Donc $f(x) = \sum\limits_{i=1}^{r} \mu_i f(\varepsilon_i) =  \sum\limits_{i=1}^{r} \mu_i \lambda_i \varepsilon_i \in G$ et $G$ est stable par $f$.
    
        \item[$(\Rightarrow)$] Montrons que $f$ est diagonalisable. On va montrer que $E = \smashoperator{\bigoplus\limits_{\lambda \in \Sp(f)}} E_\lambda (f)$.
        
        \begin{enumerate}
            \item \underline{Somme directe:} \\
            On pose $F = \smashoperator{\bigoplus\limits_{\lambda \in \Sp(f)}} E_\lambda (f)$ et $\Sp(f) = (\lambda_1, \dots, \lambda_r)$. \\
            Soit $x = \sum\limits_{i=1}^{r} x_i \in F$ où $x_i \in E_{\lambda_i}(f)$. Alors $f(x) = \sum\limits_{i=1}^{r} \underbrace{\lambda_i x_i}_{\mathclap{\in E_{\lambda_i}(f)}} \in F$. Donc $F$ est stable par $f$.
            \item Montrons que $F = E$. \\
            Par hypothèse, $F$ admet un supplémentaire $G$ dans $\C^n$, stable par $f$. Montrons que $G = \{0\}$ en raisonnant par l'absurde. \\
            On pose $g = f_{\vert F}$. D'après le théorème de \textsc{D'Alembert-Gauss} sur $\C$, $g$ admet au moins une valeur propre $\mu \in \C$ de vep associé $x_\mu$. On montre que $x_\mu \in F \cap G$. Or $F$ et $G$ sont supplémentaires donc $x_\mu = 0_E$: contradiction. D'où le résultat. 
        \end{enumerate}
        \item Considérer $R_\theta$. \textcolor{green}{à revoir}
    \end{itemize}
\end{demo}

\begin{exercice}
    Décrire un contre-exemple à la réciproque dans $\R$, en dimension 2.
\end{exercice}  
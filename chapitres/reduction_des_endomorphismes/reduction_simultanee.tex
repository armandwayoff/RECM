\begin{defi}{Endomorphismes codiagonalisables/cotrigonalisables}
    Soit $E$ un espace vectoriel de dimension finie. Soient $u$ et $v$ deux endomorphismes de $E$ diagonalisables (resp. trigonalisables). \\
    On dit que $u$ et $v$ sont \emph{codiagonalisables} (resp. \emph{cotrigonalisables}) s'ils sont diagonalisables (resp. trigonalisables) dans une même base de $E$.
\end{defi}

\begin{prop}{CNS de réduction simultanée}
    Soient $u$ et $v$ deux endomorphimes diagonalisables (resp. trigonalisables). Alors, 
    $$u, v \text{ codiagonalisables (resp. cotrigonalisables)} \Longleftrightarrow u, v \text{ commutent}.$$
\end{prop}

\begin{preuve} 
    \marginnote[0cm]{\cite{acamanes} (Thème \emph{Diagonalisation simultanée} Ch. 11)}
    Montrons la résultat pour la codiagonalisation. Raisonnons par double implication.
    \begin{itemize}
        \item[$(\Rightarrow)$] Supponsons que les endomorphismes $u$ et $v$ sont codiagonalisables. On note $D$ (resp. $\widetilde{D}$) la matrice diagonale de $u$ dans une base de $E$ (resp. celle de $v$ dans cette même base). Comme ces matrices sont diagonales, $D \times \widetilde{D} = \widetilde{D} \times D$ \note \marginnote[0cm]{\note Le produit de matrices diagonales commute.} et on en déduit que $u$ et $v$ commutent.
        \item[$(\Leftarrow)$] Supposons que les endomorphismes $u$ et $v$ commutent. \\ 
        Notons $\mathrm{Sp}(u) \defeq \{ \lambda_i,\ i \in \llbracket1, p \rrbracket \}$. Comme $u$ est diagonalisable, 
        $$E = \bigoplus\limits_{i = 1}^{p} E_{\lambda_i}(u).$$
        Soit $i \in \llbracket 1, p \rrbracket$. Comme $v$ commute avec $u - \lambda_i \mathrm{Id}_E$ \note 
        \marginnote[0cm]{
            \begin{align*}
                \note v \circ (u - \lambda_i \Id_E) &= v \circ u - \lambda_i v \\
                u \text{ et } v \text{ commutent }&= u \circ v - \lambda_i v \\
                &= (u - \lambda_i \Id_E) \circ v.
            \end{align*}
        }, $E_{\lambda_i}(u) \defeq \Ker(u - \lambda_i \Id_E)$ est stable par $v$ \note. \\
        En notant $v_i$ l'endomorphisme induit par $v$ sur $E_{\lambda_i}(u)$, comme $v$ est diagonalisable, $v_i$ est aussi diagonalisable. Ainsi, il existe $(e_{i, 1}, \dots, e_{i, r_i})$ une base de $E_{\lambda_i}(u)$ formée de vecteurs propres de $v_i$. De plus, $e_{i, j} \in E_{\lambda_i}(u) = \Ker(u - \lambda_i \Id_E)$. Ainsi, $u(e_{i,j}) = \lambda_i e_{i,j}$ et $e_{i,j}$ est un vecteur propre de $u$. \\
        Finalement, $(e_{i, 1}, \dots, e_{i, r_i})_{1 \leqslant i \leqslant p}$ est une base de $E$ constituée de vecteurs propres de $u$ et de $v$. Ainsi, $u$ et $v$ sont diagonalisables dans cette même base. 
    \end{itemize}
\end{preuve}

\begin{remarque}
    \cite{objectif_agregation} p. 167 
    \begin{itemize}
        \item La proposition établit que la commutativité est une condition suffisante pour réduire simultanément plusieurs endomorphismes. Remarquez que dans le cas de la diagonalisabilité, elle est aussi nécessaire: si des endomorphismes sont codiagonalisables, alors ils commutent. 
        \item Le seul fait que deux endomorphismes commutent n'implique par leur codiagonalisabilité, ni leur cotrigonalisabilité! Il ne faut pas oublier l'hypothèse faite sur chacun des endomorphismes (diagonalisabilité ou trigonalisabilité).
        \item La preuve de la proposition généralise la démonstration classique de la trigonalisabilité d'un endomorphisme dont le polynôme caractéristique est scindé: on raisonne par récurrence sur la dimension de l'espace. 
        \item Réduire simultanément plusieurs endomorphismes est très utile lorsqu'il s'agit de les ajouter ou de les composer. La proposition permet de démontrer, par exemple, que la somme d'endomorphismes trigonalisables qui commutent est encore trigonalisable, ou que la composée d'endomorphismes diagonalisables qui commutent est encore diagonalisables. 
        \item Supposons que $E$ soit un $\C$-espace vectoriel muni d'un produit scalaire hermitien. Soit $(f_i)_{i \in I}$ une famille d'endomorphismes de $E$ (qui sont donc trigonalisables). Si les $f_i$ commutent deux à deux, alors il existe une base orthonormée de cotrigonalisabilité. 
    \end{itemize}
\end{remarque}

\marginnote[-5cm]{
    \begin{theo}{\note Commutativité \& Stabilité}
        Soient $\varphi$ et $\psi$ deux endomorphismes qui commutent. Alors $\Im \varphi$ et $\Ker{\varphi}$ sont stables par $\psi$.
    \end{theo}
}

\begin{exercice}
    Soient $A$ et $B$ deux matrices de $\M_n(\C)$. On pose $\Phi_{A,B}$ l'endomorphisme de $\M_n(\C)$ défini pour tout $M \in \M_n(\C)$ par
    $$\Phi_{A,B}(M) \defeq AM + MB.$$
    \begin{enumerate}
         \item On suppose que la matrice $A$ est diagonalisable et que $B = 0$. Montrer que $\Phi_{A, B}$ est diagonalisable.
        \item On suppose que les matrices $A$ et $B$ sont diagonalisables. Montrer que $\Phi_{A, B}$ est diagonalisable. 
    \end{enumerate}
\end{exercice}

\begin{solution}
    \begin{enumerate}
        \item \textbf{1ère méthode.} Nous allons utiliser l'équivalence
        $$A \text{ diagonalisable} \Longleftrightarrow \exists P \in \mathrm{Ann}(A) \text{ \textsc{sars}}.$$
        On remarque que $\Phi_{A, 0}: M \mapsto AM$. On montre par récurrence que 
        $$\Phi_{A, 0}^r(M) = A^r M.$$
        Ainsi, si $P$ est un polynôme, $P(\Phi_{A, 0}) = \Phi_{P(A), 0}$ \note. \\
        \marginnote[-1cm]{
            \note On note $P = \sum\limits_{k=0}^n p_k X^k$. Alors
            \begin{align*}
                P(\Phi_{A, 0}) &= \sum_{k=0}^n p_k \Phi_{A, 0}^k \\
                &= \sum_{k=0}^n p_k A^k M \\
                &= \left( \sum_{k=0}^n p_k A^k \right) M \\
                &= \Phi_{P(A), 0}.
            \end{align*}
        }
        Comme $A$ est diagonalisable, il existe un polynôme $P$ scindé à racines simples qui est un polynôme annulateur de la matrice $A$. Ainsi, $P(\Phi_{A, 0}) = \Phi_{0, 0} = 0_{\Endo \left( \M_n(\C) \right)}$. Comme $P$ est un polynôme scindé à racines simples qui est annulateur de $\Phi_{A, 0}$, alors $\Phi_{A, 0}$ est diagonalisable. \\
        \textbf{2ème méthode.} Nous allons utiliser l'équivalence 
        $$A \text{  diagonalisable} \Longleftrightarrow \exists \text{ une base de $E$ formée de vep de $A$}.$$
        Comme $A$ est diagonalisable, il existe une base $(V_1, \dots, V_n)$ de $\M_{n,1}(\C)$ formée de vecteurs propres de $A$ respectivement associés aux valeurs propres $(\lambda_1, \dots, \lambda_n)$. On définit en colonnes la matrices $M_{k, \ell} \defeq [ 0 \cdots 0\ V_k\ 0 \cdots 0]$ où $V_k$ est la colonne $\ell$ de la matrice. En effectuant une multiplication par blocs, on vérifie que $\Phi_{A, 0}(M_{k, \ell}) = A M_{k, \ell} = \lambda_k M_{k, \ell}$. On vérifie ensuite aisément que $(M_{k, \ell})_{1 \leqslant k, \ell \leqslant n}$ est une base de $\M_n(\C)$. On a ainsi exhibé une base de $\M_n(\C)$ formée de vecteurs propres de $\Phi_{A, 0}$. 
        \item On remarque que $\Phi_{A, 0}$ et $\Phi_{0, B}$ commutent. \\
        D'après la question précédente, $\Phi_{A, 0}$ est diagonalisable et on montrerait de même que $\Phi_{0, B}$ est diagonalisable. Ainsi, d'après la propriété de diagonalisation simultanée, $\Phi_{A, 0}$ et $\Phi_{0, B}$ sont simultanément diagonalisables. \\
        En notant $(M_k)$ une base de vecteurs propres communs, alors
        $$\Phi_{A, B} (M_k) = \Phi_{A, 0} (M_k) + \Phi_{0, B} (M_k) = (\lambda_k + \mu_k) M_k$$
        et $(M_k)$ est une base de vecteurs propres de $\Phi_{A, B}$. Ainsi, $\Phi_{A, B}$ est diagonalisable. 
    \end{enumerate}
\end{solution}
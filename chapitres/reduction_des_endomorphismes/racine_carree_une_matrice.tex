\url{https://share.miple.co/content/CtwFAB5leFp4M}

\begin{box_titre}{DS6}
    On note $\mathrm{Rac}(A) = \{ R \in \M_n(\R),\ R^2 = A \}$. \\
    $\blacktriangleright$ Soit $A \in \M_n(\R)$. $\mathrm{Rac}(A)$ est une partie fermée de $\M_n(\R)$. \\
    $\blacktriangleright$ $\mathrm{Rac}(\I_n)$ n'est pas une partie bornée de $\M_n(\R)$ pour $n \geqslant 3$. 
\end{box_titre}

\begin{box_titre}{Racine carrée de matrices symétriques positives}
    Pour tout $A \in \mathscr{S}^+(\R)$, il existe une unique matrice $B \in \mathscr{S}^+(\R)$ telle que $A = B^2$. 
\end{box_titre}

\begin{exercice}
    \underline{Exercice 11, TD 11:}\\
    Soit $A = 
    \begin{pmatrix}
        -1 & 2 & 3 \\
        0 & - 1 & 4 \\
        0 & 0 & 1
    \end{pmatrix}. 
    $ Montrer que $A$ n'a pas de racine dans $\M_3(\R)$. 
\end{exercice}

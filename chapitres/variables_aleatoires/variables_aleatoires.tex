\chapter{Variables aléatoires}
\labch{variables_aleatoires}

\section{Inégalités de concentration (et transformées de \textsc{Laplace})}
\begin{itemize}
    \item Inégalité de \textsc{Markov}
    \item Inégalité de \textsc{Bienaymé-Tchebychev}
    \item Loi faible des grands nombres
\end{itemize}

\section{Calcul de \texorpdfstring{$\E \left [ \left (\sum\limits_{i=1}^{n} X_i \right)^4 \right]$}{espérance de la somme puissance 4 de v.a.}}
\begin{exercice}
    On suppose que $(X_i)_{i \in \Ne}$ est une famille de variables aléatoires iid. Calculer $\E \left [ \left (\sum\limits_{i=1}^{n} X_i \right)^4 \right]$.
\end{exercice}

\begin{elem_sol}
    Écrire la somme comme une somme sur quatre indices, utiliser la linéarité de l'espérance i.e. $\smashoperator{\sum\limits_{1 \leqslant i, j, k, l \leqslant n}} \E[X_i X_j X_k X_l]$ et distinguer les cas suivants:
    \begin{itemize}
        \item $|\{i, j, k, l \}| = 4$ (tous les indices sont deux à deux distints). Par l'indépendance des v.a., $\E[X_i X_j X_k X_l] = \E[X_i]^4$.
        \item $|\{i, j, k, l \}| = 3$ (deux des indices sont égaux): $\E[X_i X_j X_k X_l] = \E[X_i^2]\E[X_k]\E[X_l]$.
        \item $|\{i, j, k, l \}| = 2$ (trois des indices sont égaux): $\E[X_i X_j X_k X_l] = \E[X_i^3]\E[X_l]$.
        \item $|\{i, j, k, l \}| = 1$ (tous les indices sont égaux): $\E[X_i X_j X_k X_l] = \E[X_i^4]$.
    \end{itemize}
\end{elem_sol}

\section{Distance en variation totale}
En mathématiques et plus particulièrement en théorie des probabilités et en statistique, la distance en variation totale (ou distance de variation totale ou encore distance de la variation totale) désigne une distance statistique définie sur l'ensemble des mesures de probabilité d'un espace probabilisable. 

\begin{exercice}
\marginnote[0cm]{Source : \cite{acamanes} (\href{https://acamanes.github.io/psi/psi_doc/exos_e07.pdf}{Exercice 8. Chap. VII)}}
Soit $\mathscr{E}$ l'espace des suites réelles $(p_n)_{n \in \N}$ telles que la série $\sum |p_n|$ converge, muni de la norme $\norme{p} = \sum\limits_{n=0}^{+\infty} |p_n|$. Soit $\mathscr{P}$ le sous-ensemble de $\mathscr{E}$ formé des suites réelles positives $(p_n)_{n \in \N}$ telles que $\norme{p}=1$.
\begin{enumerate}
    \item Montrer que $\mathscr{P}$ est borné et convexe. \\
    \item Pour $P, Q \in \mathscr{P}$, on pose $d(P, Q) \defeq \sup\limits_{A \subset \N} \left| \sum\limits_{n \in A} p_n - \sum\limits_{n \in A} q_n \right|$. Montrer que $d(P,Q) \in [0,1]$.
    \item Soit $(p,q) \in [0, 1]^2, P = (1-p, p, 0, \dots)$ et $Q = (1-q, q, 0, \dots)$. Déterminer $d(P, Q)$.
    \item Soient $n \in \N$ et $\lambda \in \Rp$. Montrer l'inégalité $\sum\limits_{k=n+1}^{+\infty} \leqslant \e^{\lambda} \frac{\lambda^{n+1}}{(n+1)!}$.
    \item Soient $X_\lambda$ et $X_\mu$ deux variables aléatoires suivant une loi de \textsc{Poisson} de paramètres respectifs $\lambda$ et $\mu$. Soit $P_\lambda = \big(\P(X_\lambda=n)\big)_{n \in \N}$ et $P_\mu = \big(\P(X_\mu=n)\big)_{n \in \N}$. Soit $n \in \Ne$. Montrer l'inégalité
    $$d(P_\lambda, P_\mu) \leqslant \max_{A \subset \llbracket 0, n \rrbracket} \left| \sum_{k \in A} \P(X_\lambda=k) - \sum_{k \in A} \P(X_\mu=k) \right|$$
    $$+ \frac{\lambda^{n+1}}{(n+1)!} + \frac{\mu^{n+1}}{(n+1)!}.$$
\end{enumerate}
\end{exercice}

\marginnote[-5cm]{
    \begin{defi}{Ensemble convexe}
        Un ensemble $\mathscr{X}$ est dit \emph{convexe} lorsque
        $$\forall (x,y) \in \mathscr{X}^2 \enspace \forall t \in [0,1]$$
        $$tx + (1-t)y \in \mathscr{X}.$$
    \end{defi}
}

\begin{elem_sol}
    \begin{enumerate}
        \item Les éléments de $\mathscr{P}$ sont des suites positives...
        \item Bien justifier l'existence de la borne supérieure.
        \item Considérer une partie $B$ de $\N$ et distinguer quatre cas ($\{0\} \subset B$ et $\{1\} \not\subset B$; $\{0\} \not\subset B$ et $\{1\} \subset B$; $\{0, 1\} \subset B$; $ \{0, 1\} \not\subset B$). Le résultat est $d(P, Q) = |p-q|$.
        \item Passer par la formule de \textsc{Taylor} avec reste intégral.
        \item Soit $B$ une partie de $\N$. Écrire $B = \big(B \cap \llbracket0, n \rrbracket \big) \cup \ens[\big]{k \in B \tq k > n}$. Sommer sur ces deux ensembles, séparer la valeur absolue et majorer à l'aide de la question précédente. (\textcolor{green}{à détailler})
    \end{enumerate}
\end{elem_sol}


\section{Majoration de la variance d'une v.a.d.r.}
\begin{exercice}    
    \emph{Exercice 5. Chap. VII}\\
    Soit $X$ une variable aléatoire discrète réelle à valeurs dans $[a, b]$. Montrer que $\V(X) \leqslant \frac{(b-a)^2}{4}$ et discuter le cas d'égalité.
\end{exercice}

\begin{solution}
    Deux méthodes sont introduites.\\
    \underline{Méthode 1:}
    \begin{itemize}
        \item Poser la fonction $f:t \mapsto \E \left[(X-t)^2 \right]$.
        \item $f$ atteint son minimum en $\E[X]$ et vaut $\V(X)$. 
        \item Comparer $\V(X)$ à $f \left( \frac{a + b}{2} \right)$.
        \item Voir \href{https://stats.stackexchange.com/questions/45588/variance-of-a-bounded-random-variable}{Variance of a bounded random variable -- \textsf{stats.stackexchange.com}}
    \end{itemize}
    
    \underline{Méthode 2:} (vue en cours)
    \begin{itemize}
        \item Justifier l'existence de l'espérance et de la variance\\
        $\sum \gamma^n \P(X=x_k)$ converge, avec $\gamma = \max \{ |a|, |b| \}$ (ne pas oublier les valeurs absolues  car on ne connaît pas les signes de $a$ et de $b$).
        \item On remarque que $\V(X) = \V \left( X - \frac{a+b}{2} \right)$
        \item Or comme $a \leqslant X \leqslant b$, $$\displaystyle \left| X - \frac{a+b}{2}\right| \leqslant \frac{b-a}{2} \text{ et } \displaystyle \left( X - \frac{a+b}{2} \right)^2 \leqslant \left(\frac{b-a}{2}\right)^2$$
        \item Donc par le théorème de \textsc{König}-\textsc{Huygens}, 
        \begin{align*}
            \V \left(X - \frac{a+b}{2} \right) &= \E \Bigg[ \underbrace{\left(X - \frac{a+b}{2} \right)^2}_{\leqslant \left( \frac{b-a}{2} \right)^2} \Bigg] - \underbrace{\E \left[X-\frac{a+b}{2} \right]^2}_{\geqslant 0} \\
            \V(X) &\leqslant \frac{(b-a)^2}{4} \text{ par croissance de l'espérance}
        \end{align*}
    \end{itemize}
    
\marginnote[-1cm]{
    \begin{theo}{Théorème de \textsc{König}-\textsc{Huygens}}
        Soit $X$ une variable aléatoire admettant un moment d'ordre 2. Alors,
        $$\V(X) = \E \left[ X^2 \right] - \E[X]^2.$$
    \end{theo}
}
    
    \textbf{Cas d'égalité}\\
    D'après la majoration du premier terme et la minoration du deuxième terme de la relation ci-dessus, il y a égalité  si et seulement si
    
    \begin{align*}
        \V(X) = \frac{(b-a)^2}{4} &\Longleftrightarrow
        \begin{cases}
            \E \left[ \left(X - \frac{a+b}{2} \right)^2 \right] = \frac{(b-a)^2}{4}\\
            \text{et}\\
            \E \left[X-\frac{a+b}{2} \right]^2 = 0
        \end{cases}\\
        &\Longleftrightarrow
        \begin{cases}
            \P \left( X - \left(\frac{a+b}{2} \right)^2  \left(\frac{b-a}{2} \right)^2 \right)= 1\\
            \text{et}\\
            \E [X] = \frac{a+b}{2}
        \end{cases}\\
        &\Longleftrightarrow
        \begin{cases}
            \P \big( \{ X=b \} \cup \{X=a\} \big)= 1\\
            \text{et}\\
            \E [X] = \frac{a+b}{2}
        \end{cases}\\
        \V(X) = \frac{(b-a)^2}{4} &\Longleftrightarrow \P(X=a)=\P(X=b)=\frac{1}{2}
    \end{align*}
\end{solution}



\section{Identité de \textsc{Wald}}
\begin{prop}{}
    Soient $(X_n)_{n \in \Ne}$ une suite de variables aléatoires, \textbf{mutuellement indépendantes}, de même loi à valeurs dans $\N$, et $T$ une variable aléatoire à valeurs dans $\N$ indépendante des précédentes. La famille $(T, X_n)_{n \in \Ne}$ est une famille de variables aléatoires mutuellement indépendantes.\\
    On note $G_X$ la fonction génératrice commune à toutes les $X_n$.\\
    Pour $n \in \N$ et $\omega \in \Omega$, on pose $S_n(\omega) = \sum\limits_{k=1}^{n} X_k(\omega)$ et $S_0(\omega) = 0$, puis, $S(\omega) = S_{T(\omega)}(\omega)$.\\
    Alors on a $G_S = G_T \circ G_X$ et $\E[S] = \E[T] \E[X]$.
\end{prop}


\section{Allumettes de \textsc{Banach}}
\begin{exercice}
    \marginnote[0cm]{Source : \cite{fmaalouf}}
    On a deux boites d'allumettes $G$ et $D$ chacune contenant $n$ allumettes. On choisit aléatoirement une boite et on en retire une allumette, et on recommence jusqu'à ce que l'une des deux boites soit vide.
    \begin{enumerate}
        \item Déterminer la loi de la variable alétoire $R$ du nombre d'allumettes restant dans l'autre boite.
        \item Dans cette question, on considère qu'on s'arrête lorsque pour la première fois on choisit une boite et on constate qu'elle est vide. Déterminer la loi de la variable aléatoire $R'$ du nombre d'allumettes restant dans l'autre boite.
        \item Calculer l'espérance de $R$, et déduire que $\E(R) \isEquivTo{+ \infty} 2 \sqrt{\frac{n}{\pi}}$.
    \end{enumerate}
\end{exercice}

\section{Urne de \textsc{Pólya}}
\begin{exercice}
    \marginnote[0cm]{Source : \cite{fmaalouf}}
    Soient $a, b, c \in \Ne$. Une urne contient $a$ boules rouge et $b$ boules noires. On tire une boule au hasard, on constate sa couleur, et on la remet dans l'urne avec $c$ nouvelles boules de sa couleur. Soit pour tout $n \in \Ne$, $R_n$ l'événement: \say{ la $n$-ième boule tirée est rouge }.
    \begin{enumerate}
        \item Calculer $\P(R_1)$ et $\P(R_2)$.
        \item Calculer $\P(R_n)$ pour tout $n$.
    \end{enumerate}
\end{exercice}

\newpage

Source : \href{https://acamanes.github.io/psi/psi_doc/chap_e07.pdf}{Variables aléatoires discrètes -- Cours} (\cite{acamanes})
\begin{figure*}
    % \setlength sets the horizontal (column) spacing
    % \arraystretch sets the vertical (row) spacing
    \begingroup
    % \setlength{\tabcolsep}{10pt} % Default value: 6pt
    \renewcommand{\arraystretch}{1.5} % Default value: 1
    \begin{tabular}{|c|c|c|c|c|c|c|c|}
        \hline
        Nom & Paramètres & $X(\Omega)$ & $\P(X=k)$ & $\E[X]$ & $\V(X)$ & $G_X$ & $\rho$\\
        \hline \hline
        Constante & $c$ & $\{c\}$ & $1$ & $c$ & $0$ & $t^c\ (c \in \N)$ & $+\infty$ \\
        \hline
        Uniforme & $a < b \in \N$ & $\llbracket a, b \rrbracket$ & $\frac{1}{b-a+1}$ & $\frac{a+b}{2}$ & $\frac{(b-a+1)^2 - 1}{12}$ & $\frac{t^a - t^{b+1}}{(b-a+1)(1-t)}$ & $+\infty$ \\
        \hline
        \textsc{Bernoulli} & $p \in ]0, 1[$ & $\{0, 1\}$ & $p\ (k=1)$ & $p$ & $pq$ & $q + pt$ & $+\infty$ \\
        \hline
        Binomiale & $(n, p) \in \N \times ]0, 1[$ & $\llbracket 0, n \rrbracket$ & $\binom{n}{k} p^k q^{n-k}$ & $np$ & $npq$ & $(q + pt)^n$ & $+\infty$ \\
        \hline
        Géométrique & $p \in ]0, 1[$ & $\Ne$ & $pq^{k-1}$ & $\frac{1}{p}$ & $\frac{q}{p^2}$ & $\frac{pt}{1-qt}$ & $\frac{1}{q}$ \\
        \hline
        \textsc{Poisson} & $\lambda \in \Rpe$ & $\N$ & $\e^{-\lambda} \frac{\lambda^k}{k!}$ & $\lambda$ & $\lambda$ & $\e^{\lambda(t-1)}$ & $+\infty$ \\
        \hline
    \end{tabular}
    \endgroup
    % The \begingroup ... \endgroup pair ensures the separation
    % parameters only affect this particular table, and not any
    % sebsequent ones in the document.
\end{figure*}



\begin{prop}{}
    Deux matrices réelles semblables dans $\M_n(\C)$ sont semblables dans $\M_n(\R)$.
\end{prop}

\begin{preuve}
    Soient $A$ et $B$ deux matrices réelles semblables dans $\M_n(\C)$. Alors il existe une matrice $P \defeq P_{\mathrm{r}} + \mi P_{\mathrm{i}} \in \Gl_n(\C)$ telle que $AP = PB$ soit $A P_{\mathrm{r}} + \mi A P_{\mathrm{i}} = P_{\mathrm{r}} B + \mi P_{\mathrm{i}} B$ et donc en identifiant parties réelle et imaginaire, $$A P_{\mathrm{r}} = P_{\mathrm{r}} B \text{ et } A P_{\mathrm{i}} = P_{\mathrm{i}} B.$$
    On en déduit que pour tout $x \in \R,\ A(P_{\mathrm{r}} + x P_{\mathrm{i}}) = (P_{\mathrm{r}} + x P_{\mathrm{i}})B$. On pose la fonction 
    $$\delta : z \in \C \mapsto \det(P_{\mathrm{r}} + z P_{\mathrm{i}}).$$ 
    La fonction $\delta$ est polynomiale et est non identiquement nulle car $\delta(\mi) = \det(P) \not=0$. On en déduit qu'il existe un réel $x_0$ tel que $\delta(x_0) = \det(P_{\mathrm{r}} + x_0 P_{\mathrm{i}}) \not=0$. \\
    Ainsi, en posant $\widetilde{P} \defeq P_{\mathrm{r}} + x_0 P_{\mathrm{i}} \in \Gl_n(\R)$ on obtient $A = \widetilde{P}B\Inv{\widetilde{P}}$.
\end{preuve}
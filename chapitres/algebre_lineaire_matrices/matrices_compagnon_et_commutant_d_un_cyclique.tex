\begin{defi}
    Soit $P(X) \defeq X^p + \sum\limits_{k=0}^{p-1} c_kX^k \in \K[X]$. On appelle \emph{matrice compagnon} de $P$ la matrice:
$$ C_P \defeq
\begin{pmatrix}
0 & 0 & \cdots & 0 & -c_0\\
1 & 0 & \cdots & 0 & -c_1\\
0 & 1 & \cdots & 0 & -c_2\\
\vdots & \vdots & \ddots & \vdots & \vdots\\
0 & 0 & \cdots & 1 & -c_{p-1}
\end{pmatrix}.
$$
\end{defi}

\begin{theo}
    Soit $P \in \K[X]$. Le polynôme $P$ est égal au polynôme caractéristique de sa matrice compagnon:
    $$\chi_{C_P}(X) = P(X).$$
\end{theo}    

\begin{preuve}
    On a :
    $$
    \det(X \I_d - C_P) = 
    \begin{vmatrix}
        X & 0 & \cdots & 0 & a_0 \\
        -1 & X & & 0 & a_1 \\
        0 & -1 & \ddots & \vdots & \vdots \\
        \vdots & \ddots & \ddots & X & a_{d-2} \\
        0 & \cdots & 0 & -1 & X + a_{d-1}
    \end{vmatrix}.
    $$
    Notons $D_d(X, a_0, \dots, a_{d-1})$ ce déterminant. \\
    Le $(1,1)$-cofacteur de $(X \I_d - C_P)$ est $D_{d-1}(X, a_1, \dots, a_{d-1})$ et son $(1,d)$-cofacteur est $(-1)^{d+1} \delta$ où $\delta$ est le déterminant d'une matrice triangulaire supérieure de taille $d-1$ et dont tous les éléments valent $-1$; ainsi $\delta = (-1)^{d-1}$ et ce $(1,d)$-cofacteur vaut $1$. Le développement du déterminant $D_d(X, a_0, \dots, a_{d-1})$ par rapport à sa première ligne fournit donc la relation:
    $$D_d(X, a_0, \dots, a_{d-1}) = X D_{d-1}(X, a_1, \dots, a_{d-1}) + a_0.$$
    Comme $D_1(X, a_{d-1}) = X + a_{d-1}$,
    $$\det(X \I_d - C_P) = X \left(X \left(\cdots \left(X(X+a_{d-1}) + a_{d-2} \right) + \right) \cdots \right) + a_0.$$
    On reconnaît la construction de $P$ par le schéma de \textsc{Horner}. Ainsi le polynôme caractéristique de $C_P$ n'est autre que $P$. 
\end{preuve} 

\begin{prop}
    Soit $f$ un endomorphisme cyclique. Tout endomorphisme qui commute avec $f$ est un polynôme en $f$.
\end{prop}

\begin{preuve}
    
\end{preuve}

\begin{theo}
    Théorème de \textsc{Cayley}-\textsc{Hamilton} \\
    Le polynôme caractéristique est un polynôme annulateur.
\end{theo}

L'exercice suivant, issu du premier sujet de l'agrégation interne de 2022, démontre ce résultat. Cette preuve est basée sur le calcul du polynôme caractéristique d'une matrice compagnon et de l'étude du plus petit sous-espace stabilisé par une matrice et contenant un vecteur donné. \\

\marginnote[-3cm]{Note de \cite{contre-exemples} \\
    En recherchant l'inverse d'un quaternion, William \textsc{Hamilton} démontre, en 1853, le résultat pour la dimension 4 sans vraiment l'exprimer. Arthur \textsc{Cayley} énonce le résultat pour de matrices carrées d'ordre $n$, le démontre pour $n=2$, prétend l'avoir fait pour $n=3$ et dit qu'il ne lui semble pas nécessaire de le démontrer dans le cas général \dots Georg \textsc{Frobenius} fournit la première démonstration génrérale en 1878. 
}

\begin{exercice}
    Soit $p$ un entier strictement positif et soit $M$ une matrice de $\M_p(\C)$.
    \begin{enumerate}
        \item Étant donné un élément $x$ quelconque non nul de $\C^p$ on pose
        $$\mu \defeq \min \{ r \geqslant 1\ | (x, Mx, \dots, M^r x) \text{ est liée dans } \C^p\}.$$
        \item Montrer qu'il existe un élément $(\alpha_0, \dots, \alpha_{\mu-1}$ de $\C^{\mu}$ et une matrice $N$ de $\M_{p-\mu}(\C)$ tels que la matrice $M$ soit semblable à une matrice $M'$ de la forme suivante
        $$
        \begin{pmatrix}
        0 & \cdots & \cdots & 0 & -\alpha_0 & \star \\
        1 & 0 & & \vdots & -\alpha_1 & \star \\
        0 & 1 & \ddots & \vdots & \vdots & \vdots \\
        \vdots & \ddots & \ddots & 0 & -\alpha_{\mu-2} & \star \\
        0 & \cdots & 0 & 1 & -\alpha_{\mu-1} & \star \\
        O & \cdots & \cdots & O & O & N
        \end{pmatrix}
        $$
        où les $\star$ représentent des lignes d'éléments de $\C$ et les $O$ représentent des colonnes nulles. 
        \item Montrer que $\chi_M(M)x = 0$.
        \item Montrer que $\chi_M$ est un polynôme annulateur de $M$.
    \end{enumerate}
\end{exercice}

\begin{preuve}
    \marginnote[0cm]{Correction de la RMS 132 3}
\end{preuve}

\begin{box_titre}{Matrice compagnon}
    Soit $P(X) = X^p + \sum\limits_{k=0}^{p-1} c_kX^k \in \K[X]$. On appelle \emph{matrice compagnon} de $P$ la matrice:
    $$
    \begin{pmatrix}
        0 & 0 & \cdots & 0 & -c_0 \\
        1 & 0 & \cdots & 0 & -c_1 \\
        0 & 1 & \cdots & 0 & -c_2 \\
        \vdots & \vdots & \ddots & \vdots & \vdots \\
        0 & 0 & \cdots & 1 & -c_{p-1}
    \end{pmatrix}
    $$
\end{box_titre}

\marginnote[-2mm]{Cette matrice apparaît dans la démonstration du théorème de \textsc{Cayley}-\textsc{Hamilton}.}

\begin{enumerate}
    \item Résultat: soit $f$ un endomorphisme cyclique. Tout endomorphisme qui commute avec $f$ est un polynôme en $f$.
\end{enumerate}
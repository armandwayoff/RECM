\begin{prop}{}
    Soit $E$ un espace vectoriel de dimension finie $n \in \Ne$. On considère $f \in \Endo(E)$.
    \begin{itemize}
        \item La suite $\left( \Ker(f^k) \right)_{k \in \N}$ est une suite croissante pour l'inclusion et stationnaire à partir d'un certain rang $r \in \llbracket 0, n \rrbracket$.
        \item La suite $\left( \Im(f^k) \right)_{k \in \N}$ est une suite décroissante pour l'inclusion et stationnaire à partir du même rang $r$. \\
        (la suite vient de \href{https://bibmath.net/dico/index.php?action=affiche&quoi=./n/noyauxiteres.html}{Noyaux itérés -- \textsf{Bibm@th.net}}) \\
        De plus
        $$\Ker(f^r) \oplus \Im(f^r) = E.$$
        Si on note $d_k \defeq \dim \big(\Ker(f^k) \big)$, alors pour tout $k \in \N$, 
        $$d_{k+1} - d_k \geqslant d_{k+2} - d_{k+1},$$
        autrement dit la suite de la différence des dimensions entre deux noyaux itérés consécutifs est décroissante. 
    \end{itemize}
\end{prop} 

Voir aussi énoncé de \cite{exos_oraux} p. 44.

\begin{demo}
    \begin{itemize}
        \item La monotonie des deux suites est triviale. 
        \item Bien précisier l'existe de $r$ en dimension finie. \\
        Soit $r$ le rang de stationnarité de la suite des noyaux itérés. Montrons que pour tout $k \in \N, \Ker(f^r) = \Ker(f^{r+k})$. \\
        D'après le premier item, l'une des deux inclusions est vérifiée par croissance de la suite des noyaux itérés. Montrons la deuxième. Soient $k \in \N$ et $x \in \Ker(f^{r+k+1})$. Alors $f^{r+k+1}(x) = f^{r+1} \big(f^k(x) \big) = 0$ soit $f^k(x) \in \Ker(f^{r+1}) = \Ker(f^r)$. Ainsi, $f^{r+k}(x) = 0$ et $x \in \Ker(f^{r+k})$.
        \item Montrons que $\Ker(f^r) \oplus \Im(f^r) = E$. \\
        D'après le théorème du rang, $\Rg(f^r) + \dim \Ker(f^r) = n$. Il reste à montrer que $\Ker(f^r) \cap \Im(f^r) = \{ 0 \}$. \\
        Soit $y \in \Ker(f^r) \cap \Im(f^r)$. Alors il existe $x \in E$ tel que $y = f^r(x)$. De plus, $f^r(y) = 0$ donc, en remplaçant $y$ par son expression, $f^{2r}(x) = 0$ i.e. $x \in \Ker(f^{2r})$, qui est égal à $\Ker(f^r)$ par définition de $r$. On en déduit que $y = f^r(x) = 0$. Ainsi $\Ker(f^r) \cap \Im(f^r) = \{ 0 \}$ et on a bien
        $$\Ker(f^r) \oplus \Im(f^r) = E.$$
    \end{itemize}
\end{demo}

\begin{remarque}
    La propriété de somme directe, n'est plus valable dans le cas d'un espace vectoriel de dimension infinie. En effet, dans $\R[X]$, l'application \emph{dérivée} met en défaut cette égalité. 
\end{remarque}

\begin{exercice}
    \marginnote[0cm]{Source : \cite{maths-france} Planche no 2. Révisions algèbre linéaire. Espaces vectoriels}
    Soient $E$ un espace vectoriel et $f$ un endomorphisme de $E$. Pour $k \in \N$, on pose $N_k \defeq \Ker(f^k)$ et $I_k \defeq \Im(f^k)$ puis $N \defeq \bigcup\limits_{k \in \N} N_k$ et $I \defeq \bigcap\limits_{k \in \N} I_k$. ($N$ est le nilespace de $f$ et $I$ le coeur de $f$).
    \begin{enumerate}
        \item 
        \begin{enumerate}
            \item Montrer que les suites $(N_k)_{k \in \N}$ et $(I_k)_{k \in \N}$ sont respectivement croissante et décroissante pour l'inclusion.
            \item Montrer que $N$ et $I$ sont stables par $f$. 
            \item Montrer que pour tout $k \in \N$, 
            $$(N_k = N_{k+1}) \implies (N_{k+1} = N_{k+2}).$$
        \end{enumerate}
        \item On suppose de plus que $\dim E = n$, $n \in \Ne$.
        \begin{enumerate}
            \item Soit 
            \begin{align*}
                A &\defeq \ens[\big]{ k \in \N \tq N_k = N_{k+1} } \\
                \text{et } B &\defeq \ens[\big]{k \in \N \tq I_k = I_{k+1}}.
            \end{align*}
            Montrer qu'il existe un entier $p$ inférieur à $n$ tel que $A = B =  \{ k \in \N \mid k \geqslant p \}$.
            \item Montrer que $E = N_p \oplus I_p$.
            \item Montrer que $f_{\vert N}$ est nilpotent et que $f_{\vert I} \in \Gl(I)$.
        \end{enumerate}
        \item Trouver des exemples où $A$ est vide et $B$ est non vide et où $A$ est non vide et $B$ est vide.
        \item Pour $k \in \N$, on pose $d_k \defeq \dim I_k$. Montrer que la suite $(d_k - d_{k+1})_{k \in \N}$ est décroissante. En déduire le sens de variation de la suite $\big( \dim N_{k+1} - \dim N_k \big)_{k \in \Ne}$.
    \end{enumerate}
\end{exercice}

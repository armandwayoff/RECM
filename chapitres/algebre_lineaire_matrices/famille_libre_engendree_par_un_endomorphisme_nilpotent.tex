\begin{exercice}
    Soit $E$ un $\K$-espace vectoriel de dimension finie $n$ non nulle et $f \in \mathscr{L}(E)$ un endomorphisme nilpotent d'ordre de nilpotence égal à $p > 1$. Montrer qu'il existe un vecteur $x_0 \in E$ tel que la famille $\mathscr{F} \defeq \big(x_0, f(x_0), \dots, f^{p-1}(x_0) \big)$ est une famille libre. En déduire que $p \leqslant n$. 
\end{exercice}    

\begin{solution}
    \begin{itemize}
        \item Justifions tout d'abord l'existence de $x_0 \in E$ tel que $f^{p-1}(x_0) \not= 0$. Par définition, l'ordre de nilpotence $p$ est le plus petit entier tel que $f^p$ est l'endomorphisme nul. Ainsi $f^{p-1}$ n'est pas l'endomorphisme nul et il existe $x_0 \in E$ tel que $f^{p-1}(x_0) \not= 0$.
        \item Soit $(\lambda_0, \dots, \lambda_{p-1}) \in \K^p$ tel que 
        \begin{equation}\tag{\star} \label{relation}
            \lambda_0 x_0 + \lambda_1 f(x_0) + \cdots + \lambda_{p-1} f^{p-1}(x_0) = 0.
        \end{equation}
        L'objectif est de montrer que les $\lambda_i$ sont tous nuls. Pour cela nous allons composer cette relation par une succession d'applications qui permettrons d'isoler chacun des coefficients et de conclure sur leur nullité. \\
        En composant la relation (\ref{relation}) par $f^{p-1}$ on obtient
        $$\lambda_0 f^{p-1}(x_0) + \sum_{k=1}^{p-1} \lambda_k f^{k + p-1}(x_0) = 0.$$
        Or l'endomorphisme $f$ est nilpotent d'ordre $p$ donc le deuxième terme est nul et comme $f^{p-1}(x_0) \not= 0$, nécessairement $\lambda_0 = 0$. \\
        En composant la relation (\ref{relation}) successivement par $f^{p - k}$ pour $k \in \llbracket 2, p-1 \rrbracket$ on obtient $\lambda_1 = 0, \dots, \lambda_{p-1} = 0$. \\
        Finalement, la famille $\mathscr{F}$ est libre.
        \item Nous venons de construire une famille libre de $p$ éléments dans un espace de dimension $n$ donc $p \leqslant n$. 
    \end{itemize}
\end{solution}
On tire de cet exercice un résultat intéressant:
\begin{prop}{Majoration de l'ordre de nilpotence}
    Considérons un endomorphisme nilpotent sur un espace $E$ de dimension finie; son ordre de nilpotence est majoré par la dimension de l'espace $E$.
\end{prop}

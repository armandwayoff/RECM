\begin{theo}
    Soit $n \in \N$. On considère $(n + 1)$ complexes deux à deux distincts, notés $x_0, \dots, x_n$. \\
    Pour tout $i \in \llbracket 0, n \rrbracket$, il existe un unique polynôme $\Lag_i \in \C_n[X]$ tel que 
    $$\forall j \in \llbracket 0, n \rrbracket,\ \Lag_i(x_j) = \delta_{i,j}.$$
    De plus,
        $$\forall (i, j) \in \llbracket 1, n \rrbracket^2,\ \Lag_i = \prod_{j \neq i} \frac{X-x_j}{x_i - x_j}.$$
\end{theo}

\marginnote[-2cm]{
    \begin{kaobox}[frametitle=Symbole de \textsc{Kronecker}]
    $$
    \delta_{i,j} \defeq \begin{cases}
    1 \quad \text{ si } i=j, \\
    0 \quad \text{ sinon}.
    \end{cases}
    $$
    \end{kaobox}
}

\begin{preuve}
\underline{Démonstration de l'existence et de l'unicité par deux méthodes:} 
\begin{itemize}
    \item Par construction...\cite{maths-france} \\
    Soit $i \in \llbracket 1, n \rrbracket$. Le polynôme $\Lag_i$ est de degré au plus $n$ et admet les $n$ complexes deux à deux distincts $x_j$ pour racines. Par suite, nécessairement, il existe une constante $C$ telle que 
    $$\Lag_i = C \prod_{i \not= j} (X-x_j).$$
    L'égalité $\Lag_i(x_i) = 1$ fournit alors $C = \left[ \prod\limits_{j \not=i}(x_i - x_j) \right]^{-1}$ et donc nécessairement
    $$\Lag_i = \prod_{j \neq i} \frac{X-x_j}{x_i - x_j}.$$
    Réciproquement, si pour tout $i \in \llbracket 0, n\rrbracket,\ \Lag_i = \prod\limits_{j \neq i} \frac{X-x_j}{x_i - x_j}$, alors $\Lag_i$ est bien défini car les $x_j$ sont deux à deux distincts, de degré $n$ exactement et enfin les polynômes $\Lag_i$ vérifient clairement les égalités de dualité.
    \item En passant par un peu d'algèbre linéaire... \footnote{\url{https://www.youtube.com/watch?v=blB2SAYpobA}}
    \begin{alignat*}{2}
        \text{On considère l'application } \varphi\ :\ \R_n[X]\ &\longrightarrow\ \R^{n+1}\\
        P\ &\longmapsto\ (P(x_0), \dots, P(x_n)).
    \end{alignat*}
    \textcolor{green}{La suite n'est pas très bien rédigée...} \\
    L'application $\varphi$ est linéaire et injective (raisonner sur le nombre de racines distinctes d'un polynôme) ce qui assure \textbf{l'unicité} des polynômes interpolateurs de \textsc{Lagrange}. De plus, $\varphi$ est un isomorphisme (égalité des dimensions des espaces de départ et d'arrivée) donc $\varphi$ est sujective ce qui assure \textbf{existence} de ces polynômes. 
\end{itemize}
\end{preuve}

\begin{prop}
    La famille $(\Lag_0, \dots, \Lag_n)$ des $(n+1)$ premiers polynômes de \textsc{Lagrange} forme une base de $\C_n[X]$.
\end{prop}

\begin{preuve} 
    Par construction, la famille $\mathscr{L} \defeq (\Lag_0, \dots, \Lag_n)$ est échelonnée en dégré ce qui assure sa liberté d'après le \textcolor{red}{lemme}. De plus, $|\mathscr{L}| = \dim \C_n[X]$ donc la famille $\mathscr{L}$ forme bien une base de $\C_n[X]$.
\end{preuve}

Coordonées d'un polynôme dans la base de \textsc{Lagrange}

\begin{prop}

Soit $(x_0, \dots, x_n)$, $(n+1)$ complexes deux à deux distincts et $(y_0, \dots, y_n)$, $(n+1)$ complexes. Il existe un et un seul polynôme $P \in \R_n[X]$ tel que pour tout $i \in \llbracket 0, n \rrbracket, P(x_i) = y_i$. Ce polynôme à pour expression dans la base de \textsc{Lagrange}
$$P = \sum_{i=0}^n y_i \Lag_i.$$
\end{prop}

\begin{marginfigure}[-5cm]
    \begin{tikzpicture}
    \begin{axis}[width=6.5cm,
        axis lines=middle,
        grid=major,
        xmin=-1.2, xmax=1.2,
        ymin=-1.1, ymax=1.1,
        % xlabel=$x$, xlabel style={right},
        % ylabel=$y$, ylabel style={above},
        %tick style={thick},
        %ticklabel style={font=\normalsize},
        xtick=\empty, 
        ytick=\empty,
        axis line style={-latex}
    ]
    
    \def\a{-1.1}
    \def\b{1.1}
    \def\colour{BrickRed}
    
    \addplot[red,thick,samples=100,domain=\a:\b] {
    (x+5/8)*(x-1/8)*(x-1/2)*(x-4/5)/((-7/8+5/8)*(-7/8-1/8)*(-7/8-1/2)*(-7/8-4/5))*(-1/2)
    + (x+7/8)*(x-1/8)*(x-1/2)*(x-4/5)/((-5/8+7/8)*(-5/8-1/8)*(-5/8-1/2)*(-5/8-4/5))*(-1/9)
    + (x+7/8)*(x+5/8)*(x-1/2)*(x-4/5)/((1/8+7/8)*(1/8+5/8)*(1/8-1/2)*(1/8-4/5))*(-1/8)
    + (x+7/8)*(x+5/8)*(x-1/8)*(x-4/5)/((1/2+7/8)*(1/2+5/8)*(1/2-1/8)*(1/2-4/5))*(1/2)
    + (x+7/8)*(x+5/8)*(x-1/8)*(x-1/2)/((4/5+7/8)*(4/5+5/8)*(4/5-1/8)*(4/5-1/2))*(2/9)
    };
    
    \addplot[\colour,mark=*] coordinates {(-7/8,-1/2)} node[left] {$M_0$};
    \addplot[\colour,mark=*] coordinates {(-5/8,-1/9)} node[below] {$M_1$};
    \addplot[\colour,mark=*] coordinates {(1/8,-1/8)} node[below] {\contour{white}{$M_2$}};
    \addplot[\colour,mark=*] coordinates {(1/2,1/2)} node[above] {\contour{white}{$M_3$}};
    \addplot[\colour,mark=*] coordinates {(4/5,2/9)} node[right] {$M_4$};
    
    \draw[blue, thick, dotted] (-7/8,-1/2) -- (-7/8, 0);
    \draw[blue, thick, dotted] (-7/8,-1/2) -- (0, -1/2);

    \draw[blue, thick, dotted] (-5/8,-1/9) -- (-5/8, 0);
    \draw[blue, thick, dotted] (-5/8,-1/9) -- (0, -1/9);

    \draw[blue, thick, dotted] (1/8,-1/8) -- (1/8, 0);
    \draw[blue, thick, dotted] (1/8,-1/8) -- (0,-1/8);

    \draw[blue, thick, dotted] (1/2,1/2) -- (1/2, 0) node[below] {$a_3$};
    \draw[blue, thick, dotted] (1/2,1/2) -- (0, 1/2) node[left] {$f_3$};
    
    \draw[blue, thick, dotted] (4/5,2/9) -- (4/5, 0);
    \draw[blue, thick, dotted] (4/5,2/9) -- (0, 2/9);
    
    \draw[black, thick] (-0.8,0.5) node[above] 
    {\footnotesize \contour{white}{{\parbox{2cm}{\centering Polynôme \\ interpolateur}}}} to [out=640,in=800] ($(-0.3,-1/4)$);
    \end{axis}
    
\end{tikzpicture}
\end{marginfigure}

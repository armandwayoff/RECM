%========
\todoinline{Intégré au chapitre polynôme - Supprimer ce thème ?}

\source{\href{https://perso.math.univ-toulouse.fr/fdelebec/files/2018/03/chap01-L2.pdf}{extrait d'un cours de F. \textsc{Delebecque}}}
\subsection{Motivations de l'interpolation polynomiale}
En analyse numérique, une fonction $f$ inconnue explicitement est souvent connue seulement en certains points $x_0, \dots, x_d$, ou évaluable uniquement au moyen de l'appel à un code coûteux. \\
Mais dans de nombreux cas, on a besoin d'effectuer des opérations (dérivation, intégration, \dots) sur la fonction $f$. \\
On cherche donc à reconstruire cette fonction $f$ par une autre fonction $f_r$ simple et facile à évaluer à partir des données discrètes de $f$. On espère que le modèle $f_r$ ne sera pas trop éloigné de la fonction $f$ aux autres points.
\begin{center}
    Pourquoi utiliser des polynômes pour reconstruire la fonction $f$ ?    
\end{center}
\marginnote[0cm]{\note cf. p. [...]}
\begin{itemize}
    \item \textbf{Le théorème d'approximation de \nom{Weierstrass}:} \note \\
    pour toute fonction $f$ définie et continue sur un intervalle $[a, b]$ et pour tout $\varepsilon > 0$, il existe un polynôme $P$ tel que 
    $$\forall x \in [a, b],\ |f(x) - P(x)| < \varepsilon.$$
    Plus $\varepsilon$ est petit, plus le degré du polynôme est grand.
    \item \textbf{La simplicité de l'évaluation d'un polynôme par le schéma de \nom{Hörner}:}
    $$\sum_{j=0}^n c_j x^j = \Big( \cdots \big( (c_n x + c_{n-1})x + c_{n-2} \big)x + \cdots c_1 \Big)x + c_0.$$
\end{itemize}

\begin{marginfigure}[-1cm]
    \centering
    % \begin{tikzpicture}
    \begin{axis}[width=6.5cm,
        axis lines=middle,
        grid=major,
        xmin=-1.2, xmax=1.2,
        ymin=-1.1, ymax=1.1,
        % xlabel=$x$, xlabel style={right},
        % ylabel=$y$, ylabel style={above},
        %tick style={thick},
        %ticklabel style={font=\normalsize},
        xtick=\empty, 
        ytick=\empty,
        axis line style={-latex}
    ]
    
    \def\a{-1.1}
    \def\b{1.1}
    \def\colour{BrickRed}
    
    \addplot[red,thick,samples=100,domain=\a:\b] {
    (x+5/8)*(x-1/8)*(x-1/2)*(x-4/5)/((-7/8+5/8)*(-7/8-1/8)*(-7/8-1/2)*(-7/8-4/5))*(-1/2)
    + (x+7/8)*(x-1/8)*(x-1/2)*(x-4/5)/((-5/8+7/8)*(-5/8-1/8)*(-5/8-1/2)*(-5/8-4/5))*(-1/9)
    + (x+7/8)*(x+5/8)*(x-1/2)*(x-4/5)/((1/8+7/8)*(1/8+5/8)*(1/8-1/2)*(1/8-4/5))*(-1/8)
    + (x+7/8)*(x+5/8)*(x-1/8)*(x-4/5)/((1/2+7/8)*(1/2+5/8)*(1/2-1/8)*(1/2-4/5))*(1/2)
    + (x+7/8)*(x+5/8)*(x-1/8)*(x-1/2)/((4/5+7/8)*(4/5+5/8)*(4/5-1/8)*(4/5-1/2))*(2/9)
    };
    
    \addplot[\colour,mark=*] coordinates {(-7/8,-1/2)} node[left] {$M_0$};
    \addplot[\colour,mark=*] coordinates {(-5/8,-1/9)} node[below] {$M_1$};
    \addplot[\colour,mark=*] coordinates {(1/8,-1/8)} node[below] {\contour{white}{$M_2$}};
    \addplot[\colour,mark=*] coordinates {(1/2,1/2)} node[above] {\contour{white}{$M_3$}};
    \addplot[\colour,mark=*] coordinates {(4/5,2/9)} node[right] {$M_4$};
    
    \draw[blue, thick, dotted] (-7/8,-1/2) -- (-7/8, 0);
    \draw[blue, thick, dotted] (-7/8,-1/2) -- (0, -1/2);

    \draw[blue, thick, dotted] (-5/8,-1/9) -- (-5/8, 0);
    \draw[blue, thick, dotted] (-5/8,-1/9) -- (0, -1/9);

    \draw[blue, thick, dotted] (1/8,-1/8) -- (1/8, 0);
    \draw[blue, thick, dotted] (1/8,-1/8) -- (0,-1/8);

    \draw[blue, thick, dotted] (1/2,1/2) -- (1/2, 0) node[below] {$a_3$};
    \draw[blue, thick, dotted] (1/2,1/2) -- (0, 1/2) node[left] {$f_3$};
    
    \draw[blue, thick, dotted] (4/5,2/9) -- (4/5, 0);
    \draw[blue, thick, dotted] (4/5,2/9) -- (0, 2/9);
    
    \draw[black, thick] (-0.8,0.5) node[above] 
    {\footnotesize \contour{white}{{\parbox{2cm}{\centering Polynôme \\ interpolateur}}}} to [out=640,in=800] ($(-0.3,-1/4)$);
    \end{axis}
    
\end{tikzpicture}
    \caption{Polynômes de \nom{Lagrange}}
\end{marginfigure}

Plus précisément, étant donnés $d+1$ points d'abscisses distinctes $M_i \defeq (a_i, f_i)$ pour $i \in \llbracket 0, d \rrbracket$ dans le plan, le problème de l'interpolation polynomiale consiste à trouver un polynôme de degré inférieur ou égal à $m$ dont le graphe passe par les $d+1$ points $M_i$.\\

\subsection{Interpolation lagrangienne}

Les polynômes de \nom{Lagrange} permettent d'interpoler une série de $n+1$ points par un polynôme de degré $n$ qui passe exactement par ces points.

\begin{theo}[Propriété fondamentale des polynômes de \nom{Lagrange}]
    Soient $n \in \N$ et $(x_0, \dots, x_n)$ des complexes deux à deux distincts. \\
    Pour tout $i \in \llbracket 0, n \rrbracket$, il existe un unique polynôme $\Lag_i \in \C_n[X]$ tel que 
    \begin{equation} \label{prop_fondamentale}
        \forall j \in \llbracket 0, n \rrbracket,\ \Lag_i(x_j) = \delta_{i,j}.\ \note
    \end{equation}
\end{theo}

\marginnote[-4cm]{
    \begin{defi}[\note Symbole de \textsc{Kronecker}]
    $$
    \delta_{i,j} \defeq \begin{cases}
    1 \quad \text{ si } i=j, \\
    0 \quad \text{ si } i \not= j.
    \end{cases}
    $$
    \end{defi}
}

\begin{demo}
    \source{\href{https://www.youtube.com/watch?v=blB2SAYpobA}{\textsl{[UT\#29] Unicité des polynômes de \textsc{Lagrange}} -- Øljen - Les maths en finesse}}
    On considère $n + 1$ complexes deux à deux distincts, notés $x_0, \dots, x_n$. Soit l'application
    \[
    \fonction[\varphi]{\R_n[X]}{\R^{n+1}}{P}{\big(P(x_0), \dots, P(x_n) \big)}.
    \]
    \begin{itemize}
        \item[$\rhd$] Soit $P \in \Ker \varphi$. Alors le polynôme $P$ a $n+1$ racines discintes. Or il est de degré inférieur à $n$ donc est le polynôme nul. On en déduit que $\varphi$ est injective ce qui assure l'\ptnclegras{unicité} des polynômes interpolateurs de \nom{Lagrange}.
        \item[$\rhd$] Comme $\dim \R_n[X] = \dim \R^{n+1}$ et que l'application $\varphi$ est injective, c'est un isomorphisme \note.
        \item[$\rhd$] En particulier, l'application $\varphi$ est surjective ce qui assure l'\ptnclegras{existence} de ces polynômes. 
    \end{itemize}
    \marginnote[-4cm]{
        \begin{prop}
            \note Si $f$ est une application linéaire d'un espace de dimension finie $E$ dans un espace de dimension finie $F$ avec $\dim E = \dim F$, il suffit que $f$ soit injective ou surjective pour que $f$ soit un isomorphisme.
        \end{prop}
    }
\end{demo}

\begin{defi}[Polynômes de \nom{Lagrange}]
    Soient $n \in \N$ et $(x_0, \dots, x_n)$ des complexes deux à deux distincts. On appelle famille des \emph{polynômes de \nom{Lagrange}} la suite de polynômes $(\Lag_n)_{n \in \N}$ vérifiant la relation (\ref{prop_fondamentale}).
\end{defi}

\begin{remarque}
    Une famille de polynômes de \nom{Lagrange} est définie à partir d'une famille de complexes deux à deux distincts. 
\end{remarque}

\begin{prop}[Expression des polynômes de \nom{Lagrange}]
    Soient $n \in \N$ et $(x_0, \dots, x_n)$ des complexes deux à deux distincts. Les polynômes de \nom{Lagrange} associés à cette famille ont pour expression
    \[
    forall i \in \llbracket 0, n \rrbracket,\ \Lag_i(X) = \prod_{j \neq i} \frac{X-x_j}{x_i - x_j}.
    \]
\end{prop}

\begin{demo}
    \source{\cite{maths-france}}
    \begin{itemize}
        \item[$\rhd$] On considère $n + 1$ complexes deux à deux distincts, notés $x_0, \dots, x_n$. \\
        Soit $i \in \llbracket 1, n \rrbracket$. Le polynôme $\Lag_i$ est de degré au plus $n$ et admet $n$ complexes deux à deux distincts $x_j$ pour racines, avec $j \not= i$. Alors, nécessairement, il existe une constante $C$ telle que 
        $$\Lag_i = C \prod_{i \not= j} (X-x_j).$$
        L'égalité $\Lag_i(x_i) = 1$ fournit $C = \left[ \prod\limits_{j \not=i}(x_i - x_j) \right]^{-1}$ et donc 
        $$\Lag_i = \prod_{j \neq i} \frac{X-x_j}{x_i - x_j}.$$
        \item[$\rhd$] Réciproquement, si pour tout $i \in \llbracket 0, n\rrbracket$ on pose $\Lag_i \defeq \prod\limits_{j \neq i} \frac{X-x_j}{x_i - x_j}$, alors le polynôme $\Lag_i$ est bien défini car les $x_j$ sont deux à deux distincts, est bien de degré $n$ et enfin les polynômes $\Lag_i$ vérifient clairement les égalités de dualité.
    \end{itemize}
\end{demo}

\subsection{Coordonées d'un polynôme dans la base de \nom{Lagrange}}

\begin{prop}[$(\Lag_i)_{0 \leqslant i \leqslant n}$ forme une base de $\C_n{[X]}$]
    La famille $(\Lag_0, \dots, \Lag_n)$ des $n+1$ premiers polynômes de \textsc{Lagrange} forme une base de $\C_n[X]$.
\end{prop}

\begin{demo} 
    Par construction, la famille $\mathscr{L} \defeq (\Lag_0, \dots, \Lag_n)$ est échelonnée en dégré ce qui assure sa liberté d'après le \reflemme{famille_deg_echelonnes_est_libre}. De plus, $|\mathscr{L}| = \dim \C_n[X]$ donc la famille $\mathscr{L}$ forme bien une base de $\C_n[X]$.
\end{demo}

\begin{theo}[Interpolation lagrangienne]
Soient $n \in \N$ et $(x_0, \dots, x_n)$ des complexes deux à deux distincts et $(y_0, \dots, y_n)$ des complexes deux à deux distincts. \\
Il existe un et un seul polynôme $P \in \R_n[X]$ tel que 
\[
\forall i \in \llbracket 0, n \rrbracket, P(x_i) = y_i.
\]
Ce polynôme à pour expression dans la base des polynômes de \nom{Lagrange} associée
\[
P = \sum_{i=0}^n y_i \Lag_i = \sum_{i=0}^n P(x_i) \Lag_i.
\]
\end{theo}

\subsection{Lien avec les déterminants de \nom{Vandermonde}}

\source{\cite{maths-france}}
En appliquant la formule des coordonnées d'un polynôme de degré au plus $n$ dans la base $(\Lag_i)_{i \in \llbracket 0, n \rrbracket}$ au cas particulier où le polynôme $P$ est l'un des éléments de la base canonique $(X^j)_{j \in \llbracket 0, n \rrbracket}$ de $\C_n[X]$, on obtient $\sum\limits_{i=0}^{n} \Lag_i = 1$ et plus généralement, 
$$\forall j \in \llbracket 0, n \rrbracket,\ X^j = \sum_{i=0}^{n} x_i ^j \Lag_i.$$
Ainsi, 
\marginnote[0cm]{
    $$\Mat_{(\Lag_i)_0^n, (X^j)_0^n} \begin{pmatrix}
    1 & x_1 & x_1^2 & \cdots & x_1^{n-1} \\
    \vdots & \vdots & \vdots & \ddots & \vdots \\
    1 & x_n & x_n^2 & \cdots & x_n^{n-1}
    \end{pmatrix}.$$
}
\begin{prop}
    La matrice de passage de la base  $(\Lag_i)_{i \in \llbracket 0, n \rrbracket}$ à la base  $(X^j)_{j \in \llbracket 0, n \rrbracket}$ est la matrice de \nom{Vandermonde} associée à la famille $(x_i)_{i \in \llbracket 0, n \rrbracket}$.
\end{prop}

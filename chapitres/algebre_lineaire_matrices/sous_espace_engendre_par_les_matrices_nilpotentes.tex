\begin{exercice}
    Déterminer le sous-espace vectoriel de $\M_n(\K)$ engendré par les matrices nilpotentes.
\end{exercice}

\begin{solution}
    On note 
    \begin{align*}
        \mathcal{N} & \text{ l'ensemble des matrices nilpotentes}, \\
        \mathcal{T} & \text{ l'ensemble des matrices de trace nulle.}
    \end{align*}
    \marginnote[0cm]{
        \begin{defi}{Hyperplan}
            Soit $E$ un $\K$-espace vectoriel de dimension $n \in \Ne$.
            Un sous-espace vectoriel $H$ de $E$ est un \emph{hyperplan} de $E$ si $\dim H = n-1$.
        \end{defi}
        \begin{theo}{Hyperplan \& Formes linéaires}
            Si $\varphi$ est une forme linéaire sur $E$ non nulle, alors $\Ker \varphi$ est un hyperplan de $E$.
        \end{theo}
    }
    Nous allons montrer que $\Vect(\mathcal{N}) = \mathcal{T}$. \\
    L'ensemble $\mathcal{T}$ est le noyau de la forme linéaire non nulle qu'est la \emph{trace}, c'est donc un hyperplan de $\M_n(\K)$ et est de dimension $n^2-1$. \\
    Raisonnons par double inclusion.
    \begin{itemize}
        \item[$(\subset)$] Comme le spectre d'une matrice nilpotente est réduit à $0$, toute matrice nilpotente est semblable à une matrice triangulaire à éléments diagonaux nuls. La \emph{trace} étant un invariant de similitude, toute matrice nilpotente est de trace nulle. \\
        On en déduit le même résultat sur la trace de toute combinaison linéaire de matrices nilpotentes.
        \marginnote[0cm]{\cite{oraux_x_ens_2} p. 12.}
        \item[$(\supset)$] Les matrices $\mathrm{E}_{ij}$ de la base canonique avec $i \not= j$ sont nilpotentes d'ordre $2$. Ces matrices engendrent l'espace des matrices de diagonale nulle. Il nous reste donc à obtenir l'espace des matrices diagonales de trace nulle. L'ensemble $\mathcal{T}$ est engendré par
        $$\big\{ \mathrm{E}_{k \ell} + \mathrm{E}_{ii} - \mathrm{E}_{jj}\ |\ k \not= \ell, i \not= j \big\}.$$
        Les matrices $\mathrm{E}_{ii} - \mathrm{E}_{jj}$ avec $i \not= j$ engendrent l'espace des matrices diagonales de trace nulle. Ces matrices ne sont pas nilpotentes mais on va pouvoir les obtenir comme combinaisons linéaires de matrices nilpotentes. \\
        Regardons déjà le cas $n = 2$. Une matrice nilpotente de trace nulle avec $1$ et $-1$ sur la diagonale, doit être de rang $1$. On constate effectivement que $\begin{pmatrix} 1 & -1 \\ 1 & -1 \end{pmatrix}$ est bien nilpotente. Construisons une matrice analogue de taille $n$. \\
        Pour $2 \leqslant i \leqslant n$, considérons alors la matrice
        $$F_i \defeq \mathrm{E}_{11} - \mathrm{E}_{1i} + \mathrm{E}_{i1} - \mathrm{E}_{ii}.$$
        Cette matrice étant de rang égal à $1$, d'après \vrefexercice{carre_matrice_de_rang_1} $F_i^2 = \Tr (F_i) F_i = 0$ puisque $\Tr(F_i) = 0$. Les matrices $F_i$ sont donc nilpotentes. \\
        Ainsi l'espace $\Vect(\mathcal{N})$ contient $F_i + \mathrm{E}_{1i} - \mathrm{E}_{i1} = \mathrm{E}_{11}-\mathrm{E}_{ii}$ pour $2 \leqslant i \leqslant n$ et donc l'espace des matrices diagonales de trace nulle. 
    \end{itemize}
    On a ainsi montré que $\Vect(\mathcal{N}) = \mathcal{T}$.
\end{solution}

\marginnote[-7cm]{
    $$
    F_i=
    \begin{pmatrix}
        1 & 0 & \cdots & 0 & -1 & 0 & \cdots & 0 \\
        0 & & & & 0 & & & 0 \\
        \vdots & & & & \vdots & & & \vdots \\
        0 & & & & 0 & & & 0 \\
        1 & 0 & \cdots & 0 & -1 & 0 & \cdots & 0 \\
        0 & & & & 0 & & & 0 \\
        \vdots & & & & \vdots & & & \vdots \\
        0 & & \cdots & & 0 & & \cdots & 0 \\
    \end{pmatrix}
    $$
}
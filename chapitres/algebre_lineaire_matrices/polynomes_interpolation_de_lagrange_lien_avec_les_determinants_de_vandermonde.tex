\marginnote[0cm]{\cite{maths-france}}
En appliquant la formule des coordonnées d'un polynôme de degré au plus $n$ dans la base $(\Lag_i)_{i \in \llbracket 0, n \rrbracket}$ au cas particulier où le polynôme $P$ est l'un des éléments de la base canonique $(X^j)_{j \in \llbracket 0, n \rrbracket}$ de $\C_n[X]$, on obtient $\sum\limits_{i=0}^{n} \Lag_i = 1$ et plus généralement, 
$$\forall j \in \llbracket 0, n \rrbracket,\ X^j = \sum_{i=0}^{n} x_i ^j \Lag_i.$$
Ainsi, 
\begin{prop}
    La matrice de passage de la base  $(\Lag_i)_{i \in \llbracket 0, n \rrbracket}$ à la base  $(X^j)_{j \in \llbracket 0, n \rrbracket}$ est la matrice de \textsc{Vandermonde} associée à la famille $(x_i)_{i \in \llbracket 0, n \rrbracket}$.
\end{prop}

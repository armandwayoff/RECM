\begin{prop}[Caractérisation des homothéties]
    Soit $E$ un espace vectoriel et $f \in \Endo(E)$. L'application $f$ est une homothétie si et seulement si la famille $\big(x, f(x) \big)$ est liée pour tout $x \in E$.
\end{prop}

\begin{marginfigure}
    \centering
    % \pgfmathsetmacro{\SCALE}{1}
\resizebox{6.5cm}{6.5cm}{%
\begin{tikzpicture}
\tkzDefPoints{0/0/O, 1/2/A, 4/3/B, 3/1/C}
\tkzDefPointBy[homothety=center O ratio \SCALE](A)   \tkzGetPoint{A'}
\tkzDefPointBy[homothety=center O ratio \SCALE](B)   \tkzGetPoint{B'}
\tkzDefPointBy[homothety=center O ratio \SCALE](C)   \tkzGetPoint{C'}

\tkzDrawPolygon[fill=red!30](A,B,C)
\tkzDrawPolygon(A',B',C')
\tkzDrawPoints(O)
\tkzDrawLines[dashed,blue,add= 0 and .3](O,A' O,B' O,C')
\tkzDrawPoints(A,B,C,A',B',C')
\tkzLabelPoints(A,B,C,A',B',C')
\end{tikzpicture}
}
    \caption{Figure à compiler}
\end{marginfigure}

\begin{demo}
    Raisonnons par double implication. 
    \begin{itemize}
        \item[$(\Rightarrow)$] On suppose que l'application $f$ est une homothétie. Alors il existe un scalaire non nul $\lambda$ tel que pour tout $x \in E, f(x) = \lambda x$. Ainsi, la famille $\big(x, f(x) \big)$ est liée pour tout $x \in E$. 
        \item[$(\Leftarrow)$] \source{\href{http://marocprepa.com/site/homothetiecarac}{Une caractérisation des homothéties -- \textsf{marocprepa.com}}} On suppose que la famille $\big(x, f(x) \big)$ est liée pour tout $x \in E$. On pose 
        $$\fonctionligne{x}{\lambda_x x}.$$
        Soient $y \in E$ non nul et $x \in E$. Montrons que $\lambda_x = \lambda_{y}$ ce qui assurera que $f$ est une homothétie. \\
        Distinguons deux cas:
        \begin{itemize}
            \item \textbf{Cas où $x \in \Vect(y)$.} Alors il existe un scalaire non nul $\lambda$ tel que $x = \lambda y$. En composant cette relation par $f$ on obtient
            $$f(x) = \lambda f(y)$$
            \begin{align*}
                &\text{soit : } \lambda_x x = \lambda \lambda_{y} y, \\
                &\text{en remplaçant $x$ : } \lambda_x \lambda y = \lambda \lambda_{y} y, \\
                &\text{en factorisant : } \lambda (\lambda_x - \lambda_{y}) y = 0.
            \end{align*}
            Or $\lambda \not= 0$ et $y \not= 0_E$ donc nécessairement $\lambda_x = \lambda_{y}$.
            \item \textbf{Cas où $x \not \in \Vect(y)$.} On pose $z \defeq x + y$. Par définition de l'application $f$, $f(z) = \lambda_z z$ soit $f(z) = \lambda_z(x + y)$. \\
            On peut aussi écrire $f(z) = f(x) + f(y) = \lambda_x x + \lambda_y y$. \\
            En égalant ces deux expressions, on obtient 
            \[
            (\lambda_x - \lambda_z) x + (\lambda_{y} - \lambda_z) y = 0
            \]
            et par liberté de la famille $(x, y)$, on aboutit à $\lambda_x = \lambda_{y}$.
        \end{itemize}
    \end{itemize}
\end{demo}

\begin{exercice}
    Déterminer toutes les matrices $M \in \M_3(\K)$ telles que $M^2=0$.
\end{exercice}

\begin{solution}
    La correction qui suit est issue de \cite{ellipses}. \\
    Si $M$ est la matrice nulle, $M$ est solution de l'équation. \\
    Supposons que $M \not= 0$. Soit $\varphi$ l'endomorphisme canoniquement associé à $M$. \\
    Comme $M$ est non nulle, existe donc un vecteur $x$ tel que $\varphi(x) \not= 0$. \\
    Comme $M^2 = 0$, $\Im \varphi \subset \Ker \varphi$. Par le théorème du rang, $\Rg \varphi + \dim \Ker \varphi = 3$ et comme $\Rg \varphi \geqslant 1$ (car $M \not=0$), nécessairement, $\Rg \varphi = 1$ et $\dim \Ker \varphi = 2$. \\
    Comme $\varphi(x) \in \Ker \varphi$, il existe $z \in \Ker \varphi$ tel que la famille $\{ \varphi(x), z \}$ soit une base de $\Ker \varphi$. La famille $\mathscr{F} = \{x, \varphi(x), z \}$ est libre (facile à montrer) et la matrice de $\varphi$ par rapport à cette base est 
    $A = 
    \begin{pmatrix}
    0 & 0 & 0 \\
    1 & 0 & 0 \\ 
    0 & 0 & 0
    \end{pmatrix}
    $. 
    Finalement, les matrices solutions sont
    $$\boxed{\mathscr{S} = \left \{ \{0\} \cup \{\Inv{P} A P, P \in \Gl_3(\K) \} \right \}.}$$
\end{solution}

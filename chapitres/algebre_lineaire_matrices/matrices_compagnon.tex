\subsection{Motivation}

On considère une équation différentielle linéaire d'ordre $n$ à coefficients constants
\[
x^{(n)}(t) + a_1 x^{(n-1)}(t) + \cdots + a_{n-1} x'(t) + a_n x(t) = 0
\]
où la fonction inconnue est une fonction $t \mapsto x(t)$ de $\R$ dans $\R$ et $n$ fois dérivable. \\
On introduit les fonctions auxiliaires
\[
\begin{cases}
    x_1 &= x \\
    x_2 &= x_1' = x' \\
    \vdots \\
    x_{n-1} &= x'_{n-2} = x^{(n-2)} \\
    x_n &= x'_{n-1} = x^{(n-1)}
\end{cases}
\]

\begin{defi}{Matrice compagnon}
    Soit $P$ le polynôme définit par 
    $$P(X) \defeq X^p + \sum\limits_{k=0}^{p-1} c_k X^k \in \K[X].$$
    On appelle \emph{matrice compagnon} de $P$ la matrice
$$ C_P \defeq
\begin{pmatrix}
0 & 0 & \cdots & 0 & -c_0\\
1 & 0 & \cdots & 0 & -c_1\\
0 & 1 & \cdots & 0 & -c_2\\
\vdots & \vdots & \ddots & \vdots & \vdots\\
0 & 0 & \cdots & 1 & -c_{p-1}
\end{pmatrix}.
$$
\end{defi}

\begin{theo}{} \labthm{poly_carac_mat_comp}
    Soit $P \in \K[X]$. Le polynôme $P$ est égal au polynôme caractéristique de sa matrice compagnon:
    $$\chi_{C_P}(X) = P(X).$$
\end{theo}   

\marginnote[2cm]{
    \begin{defi}{Cofacteurs}
        \cite{acamanes} ch4 \\
        Soient $A \in \M_n(\K)$ et $i, j \in \llbracket 1, n \rrbracket$. On note $\Delta_{i,j}$ le déterminant de la matrice de $\M_{n-1}(\K)$ obtenue à partir de la matrice $A$ en supprimant la ligne $i$ et la colonne $j$.
        \begin{itemize}
            \item Le \emph{mineur} d'indice $i,j$ de la matrice $A$ est $\Delta_{i,j}$.
            \item Le \emph{cofacteur} d'indice $i,j$ de la matrice $A$ est $(-1)^{i+j}\Delta_{i,j}$.
        \end{itemize}
    \end{defi}
    \begin{prop}{Développement selon une ligne / colonne}
        Soit $A \defeq (a_{i,j})_{1 \leqslant i, j \leqslant n} \in \M_n(\K)$.
        $$\forall j \in \llbracket 1, n \rrbracket, \det(A) = \sum\limits_{i=1}^n a_{i,j}(-1)^{i+j} \Delta_{i,j}.$$
        $$\forall i \in \llbracket 1, n \rrbracket, \det(A) = \sum\limits_{j=1}^n a_{i,j}(-1)^{i+j} \Delta_{i,j}.$$
    \end{prop}
}

\begin{preuve}
    Par définition,
    $$
    \chi_{C_P}(X) = \det(X \I_p - C_P) = 
    \begin{vmatrix}
        X & 0 & \cdots & 0 & c_0 \\
        -1 & X & & 0 & c_1 \\
        0 & -1 & \ddots & \vdots & \vdots \\
        \vdots & \ddots & \ddots & X & c_{d-2} \\
        0 & \cdots & 0 & -1 & X + c_{d-1}
    \end{vmatrix}.
    $$
    Notons $D_p(X, c_0, \dots, c_{p-1})$ ce déterminant. \\
    Le $(1,1)$-cofacteur de $(X \I_p - C_P)$ est $D_{p-1}(X, c_1, \dots, c_{p-1})$ et son $(1,p)$-cofacteur est $(-1)^{p+1} \delta$ où $\delta$ est le déterminant d'une matrice triangulaire supérieure de taille $d-1$ et dont tous les éléments valent $-1$; ainsi $\delta = (-1)^{p-1}$ et ce $(1,p)$-cofacteur vaut $1$. \\
    Le développement du déterminant $D_p(X, c_0, \dots, c_{p-1})$ par rapport à sa première ligne fournit donc la relation:
    $$D_p(X, c_0, \dots, c_{p-1}) = X D_{p-1}(X, c_1, \dots, c_{p-1}) + c_0.$$
    Comme $D_1(X, c_{p-1}) = X + c_{p-1}$,
    $$\det(X \I_p - C_P) = X \bigg(X \Big(\cdots \big(X(X+c_{p-1}) + c_{p-2} \big) \cdots \Big) + c_1 \bigg) + c_0.$$
    On reconnaît la construction de $P$ par le schéma de \textsc{Hörner}. Ainsi le polynôme caractéristique de $C_P$ n'est autre que $P$. 
\end{preuve} 

\subsection{Endomorphisme cyclique}

\begin{defi}{Endomorphisme cyclique}
    \marginnote[0cm]{Source : \href{https://bibmath.net/dico/index.php?action=affiche&quoi=./c/cyclique.html}{Endomorphisme cyclique -- \textsf{Bibm@th.net}}}
    Soit $E$ un $\K$-espace vectoriel de dimension finie $n$ et soit $u$ un endomorphisme de $E$. On dit que $u$ est \emph{cyclique} s'il existe $x \in E$ tel que $\big(x, u(x), \dots, u^{n-1}(x) \big)$ est une base de $E$. 
\end{defi}

Les endomorphismes cycliques admettent des matrices particulières dans la base précédente :

\begin{prop}{} 
    Soit $u$ un endomorphisme de $E$. Alors $u$ est cyclique si et seulement s'il existe une base de $E$ dans laquelle la matrice de $u$ est la matrice compagnon de son polynôme caractéristique.
\end{prop}

\begin{prop}{}
    Soit $f$ un endomorphisme cyclique. Tout endomorphisme qui commute avec $f$ est un polynôme en $f$.
\end{prop}

\begin{preuve}
    \marginnote[0cm]{Source : \cite{exos_oraux} p. 58}
    Tous les polynômes en $f$ commutent avec $f$. \\
    Réciproquement, on va montrer que seuls les polynômes en $f$ commutent avec $f$. On note $\mathscr{B} \defeq (e_0, \dots, e_{n-1})$ la base dans laquelle $f$ admet pour matrice la matrice compagnon de son polynôme caractéristique (qui existe d'après \textcolor{red}{lien}). Par définition, on a donc:
    $$\forall p \in \llbracket 0, n-2 \rrbracket, f(e_p) = e_{p+1}. \quad (\star)$$
    Soit $g \in \Endo(E)$ tel que $f \circ g = g \circ f$. Le vecteur $g(e_0)$ se décompose sur la base $\mathscr{B}$:
    $$\exists (\lambda_0, \dots, \lambda_{n-1}) \in \K^n, g(e_0) = \sum_{k=0}^{n-1} \lambda_k e_k.$$
    Nous allons montrer que $g = \sum\limits_{k=0}^{n-1} \lambda_k f^k$. Pour cela, nous allons prouver que:
    $$\forall p \in \llbracket 0, n-1 \rrbracket, g(e_p) = \left( \sum_{k=0}^{n-1} \lambda_k f^k \right)(e_p)$$
    par récurrence sur $p$: l'égalité proviendra du fait que les deux applications linéaires $\sum\limits_{k=0}^{n-1} \lambda_k f^k$ et $g$ coïncident sur une base. 
    \begin{itemize}
        \item[$\rhd$] On a $f^k (e_0) = e_k$ par récurrence immédiate sur $k \in \llbracket 0, n-1 \rrbracket$. On a donc
        $$g(e_0) = \sum_{k=0}^{n-1} \lambda_k f^k(e_0).$$
        \item[$\rhd$] Supposons que $0 \leqslant p \leqslant n-2$ et que $g(e_p) = \sum\limits_{k=0}^{n-1} \lambda_k f^k (e_p)$. On compose par $f$, qui est linéaire: $(f \circ g)(e_p) = \sum\limits_{k=0}^{n-1} \lambda_k f \circ f^k (e_p)$. On utilise l'hypothèse de commutativité:
        $$(g \circ f)(e_p) = \sum_{k=0}^{n-1} \lambda_k f^k \big( f(e_p) \big).$$
        Cela s'écrit, grâce à $(\star)$: $g(e_{p+1}) = \sum\limits_{k=0}^{n-1} \lambda_k f^k(e_{p+1})$. La récurrence est établie et $g = \sum\limits_{k=0}^{n-1} \lambda_k f^k$. \\
        Tout endomorphisme qui commute avec $f$ est donc un polynôme en $f$. Ainsi l'ensemble des endomorphismes commutant avec $f$ est:
        $$\Vect \big( \Id_E, f, f^2, \dots, f^{n-1}\big),$$
        on retrouve ainsi que c'est un sous-espace vectoriel de $\Endo(E)$.
    \end{itemize}
\end{preuve}

\subsection{Théorème de \textsc{Cayley}-\textsc{Hamilton}}

\begin{theo}{\textsc{Cayley}-\textsc{Hamilton}}
    Le polynôme caractéristique est un polynôme annulateur.
\end{theo}

\marginnote[0cm]{Source : note de \cite{contre-exemples}}
En recherchant l'inverse d'un quaternion, William \textsc{Hamilton} démontre, en 1853, le résultat pour la dimension 4 sans vraiment l'exprimer. Arthur \textsc{Cayley} énonce le résultat pour de matrices carrées d'ordre $n$, le démontre pour $n=2$, prétend l'avoir fait pour $n=3$ et dit qu'il ne lui semble pas nécessaire de le démontrer dans le cas général \dots Georg \textsc{Frobenius} fournit la première démonstration génrérale en 1878. \\

L'exercice suivant, issu du premier sujet de l'agrégation interne de 2022, démontre ce résultat. Cette preuve est basée sur le calcul du polynôme caractéristique d'une matrice compagnon et de l'étude du plus petit sous-espace stabilisé par une matrice et contenant un vecteur donné. \\

\begin{exercice}
    Soit $p$ un entier strictement positif et soit $M$ une matrice de $\M_p(\C)$. \\
    Étant donné un élément $x$ quelconque non nul de $\C^p$ on pose
    $$\mu \defeq \min \ens[\Big]{ r \geqslant 1 \tq \big(x, Mx, \dots, M^r x \big) \text{ est liée dans } \C^p}.$$
    \begin{enumerate}
        \item Montrer qu'il existe un élément $(\alpha_0, \dots, \alpha_{\mu-1})$ de $\C^{\mu}$ et une matrice $N$ de $\M_{p-\mu}(\C)$ tels que la matrice $M$ soit semblable à une matrice $M'$ de la forme suivante
        $$
        \begin{pmatrix}
        0 & \cdots & \cdots & 0 & -\alpha_0 & \star \\
        1 & 0 & & \vdots & -\alpha_1 & \star \\
        0 & 1 & \ddots & \vdots & \vdots & \vdots \\
        \vdots & \ddots & \ddots & 0 & -\alpha_{\mu-2} & \star \\
        0 & \cdots & 0 & 1 & -\alpha_{\mu-1} & \star \\
        O & \cdots & \cdots & O & O & N
        \end{pmatrix}
        $$
        où les $\star$ représentent des lignes d'éléments de $\C$ et les $O$ représentent des colonnes nulles. 
        \item Montrer que $\chi_M(M)x = 0$.
        \item Montrer que $\chi_M$ est un polynôme annulateur de $M$.
    \end{enumerate}
\end{exercice}

\begin{solution}
    \marginnote[0cm]{Source : Correction de la RMS 132 3}
    \begin{enumerate}
        \item 
        \begin{itemize}
            \item La famille $\big(x, Mx, \dots, M^p x\big)$ est un système de $p+1$ vecteurs de $\C^p$ d'un espace vectoriel de dimension $p$. Il est donc lié. Il existe donc $\mu$, minimum de $r$ tel que $(x, Mx, \dots, M^r x)$ est lié et $\mu \leqslant p$. 
            \item Par construction, la famille  $\big(x, Mx, \dots, M^{\mu-1} x\big)$ est libre et $\big(x, Mx, \dots, M^{\mu-1} x, M^\mu x\big)$ est liée. Il en résulte une égalité: 
            $$M^\mu(x) = \alpha_0 x + \alpha_1 M x + \cdots + \alpha_{\mu-1} M^{\mu-1} x$$
            pour un unique $\mu$-uplet $(\alpha_0, \dots, \alpha_{\mu-1})$ de $\C^\mu$. \\
            \item Notons $\fonctionligne{x}{Mx}$ l'endomorphisme de $\C^p$ et 
            $$V \defeq \Vect\big(x, Mx, \dots, M^{\mu-1} x\big);$$
            l'application $f$ envoie chaque $M^i x$, pour $0 \leqslant i \leqslant \mu - 1$, sur un élément de $V$. Ainsi $V$ est stabilisé par $f$.
            Notons $e_1 \defeq x, e_2 \defeq M x, \dots, e_\mu \defeq M^{\mu-1} x$. Le système $(e_1, \dots, e_\mu)$ est une base de $V$. Complétons ce système par $(e_{\mu+1}, \dots, e_p)$ tels que $(e_1, \dots, e_p)$ est une base de $\C^p$. 
            $$f(e_1) = e_2, \dots, f(e_{\mu-1}) = e_\mu, f(e_\mu) = -\alpha_{\mu-1} e_{\mu-1} - \cdots - \alpha_0 e_1.$$
            \item Notons $\Pi$ le polynôme $X^\mu + \alpha_{\mu-1} X^{\mu-1} + \cdots + \alpha_1 X + \alpha_0$. La matrice de $f$ dans la base $(e_1, \dots, e_p)$ est une matrice 
            $M' \defeq
            \begin{pmatrix}
                C_\Pi & L \\
                O & N
            \end{pmatrix}
            $. Elle est bien du type voulu. \\
            La matrice $M'$ est semblable à $M$ car $M$ et $M'$ représentent toutes deux $f$, l'une dans la base canonique et l'autre dans la base des $e_i$. 
        \end{itemize}
        \item On a $\Pi(M)x = M^\mu x - \alpha_{\mu-1}M^{\mu-1}x - \cdots - \alpha_0 x$. Or d'après \vrefthm{poly_carac_mat_comp}, on d'une part $\Pi = \chi_{C_\Pi}$ et d'autre part 
        $$X \I_p - M' = \begin{pmatrix}
            X \I_\mu - C_\Pi & -L \\
            O & X \I_{p-\mu} - N
        \end{pmatrix}.$$
        Donc $\det(X \I_p - M') = \det(X \I_\mu - C_\Pi) \det(X I_{p-\mu} - N)$. Ainsi $\Pi$ divise $\chi_{M'}$ et comme $M'$ est semblable à $M$, $\chi_M = \chi_{M'}$. Il en résulte: $\chi_M(M)x = 0$.
        \item Pour tout $x \in \C^p$, on a $\chi_M(M)x = 0$ c'est-à-dire que $\chi_M(M)$ est la matrice nulle.
    \end{enumerate}
\end{solution}

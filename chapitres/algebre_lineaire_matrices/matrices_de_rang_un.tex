\begin{prop}
    Une matrice $A$ de $\M_{n,p}(\R)$ est de rang $1$ si et seulement s'il existe deux matrices colonnes non nulles (pas nécessairement uniques) $X \in \M_{n,1}(\R)$ et $Y \in \M_{p,1}(\R)$ telles que $A = X \Trsp{Y}$. 
\end{prop}

\begin{preuve}
    à revoir
    \begin{itemize}
        \item[$(\Rightarrow)$] Soit $A \in \M_{n,p}(\R)$, une matrice de rang $1$. En notant $C_1, \dots, C_p$ ses colonnes, comme $\Rg A = 1$, il existe un vecteur $X \defeq \Trsp{(x_1 \cdots x_n)}\in \M_{n,1}(\R)$ et $(\lambda_1, \dots, \lambda_p) \in (\Re)^p$ tels que pour tout $k \in \llbracket 1, p \rrbracket, C_k = \lambda_k X$. \\
        En posant $Y \defeq \Trsp{(\lambda_1 \cdots \lambda_p)}$, on obtient $A = X \Trsp{Y}$.
        \item[$(\Leftarrow)$] Soient $X \in \M_{n,1}(\R)$ et $Y \in \M_{p,1}(\R)$, deux matrices colonnes non nulles...
    \end{itemize}
    Par contre, il n'y a pas unicité du couple $X, Y$...
\end{preuve}

\begin{exercice}
    Soit $A$ une matrice carrée de rang $1$. Montrer qu'il existe $\lambda \in \K$ tel que $A^2 = \lambda A$.
\end{exercice}

\begin{solution}
    Il existe une colonne $X$ telle que $AX \not= 0$ et alors $\Im(A) = \Vect(AX)$. \\
    Ainsi, la matrice $A^2X \in \Im(A)$ et donc il existe $\lambda \in \K$ tel que $A^2X = \lambda AX$. De plus, pour $Y \in \Ker(A)$, $A^2Y = 0 = \lambda AY$. \\
    Enfin $\Ker(A)$ et $\Vect(X)$ sont supplémentaires dans $\M_{n,1}(\K)$ donc $A^2 = \lambda A$.
\end{solution}

\url{http://ddmaths.free.fr/section373.html} \\
Compléter avec le problème 2 Partie I de CCINP PSI 2022.

\begin{exercice}
    \marginnote[0cm]{\url{https://www.elearning-cpge.com/classique/classiques-reduction/}} \\
    \begin{enumerate}
        \item Montrer que $A^2 = \Tr (A) A$.
        \item En déduire que la matrice $A$ est diagonalisable si et seulement si $\Tr A \not= 0$.
        \item Montrer que si $\Tr A \not= 0$, alors la matrice $A$ est semblable dans $\M_n(\R)$ à la matrice $\Diag(0, \dots, 0, \Tr A)$.
        \item On suppose que $\Tr A = 0$ et on désigne par $f$ l'endomorphisme de $\M_{n,1}(\K)$ canoniquement associé à $A$. 
        \begin{enumerate}
            \item Montrer que $U \in \Ker f$ et justifier l'existence d'une base de $\Ker f$ de la forme $(E_1, \dots, E_{n-2}, U)$. 
            \item Soit $W \defeq \frac{1}{\Trsp{V}V}V$. Montrer que $(E_1, \dots, E_{n-2}, U, W)$ est une base de $\M_{n,1}(\K)$ et écrire la matrice de $f$ dans cette base. 
            \item En déduire que deux matrices de rang $1$ et de trace nulle sont semblables dans $\M_n(\K)$.
        \end{enumerate}
    \end{enumerate}
\end{exercice}
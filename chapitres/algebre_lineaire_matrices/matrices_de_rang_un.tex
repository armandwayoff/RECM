\begin{prop}{Caractérisation des matrices de rang $1$} \labprop{caracterisation_matrice_de_rang_1}
    Une matrice $A \in \M_{n,p}(\R)$ est de rang $1$ si et seulement s'il existe deux matrices colonnes non nulles (pas nécessairement uniques) $X \in \M_{n,1}(\R)$ et $Y \in \M_{p,1}(\R)$ telles que $A = X \Trsp{Y}$. 
\end{prop}

\begin{preuve}
    \begin{itemize}
        \item[$(\Rightarrow)$] Soit $A \in \M_{n,p}(\R)$, une matrice de rang $1$. On note $C_1, \dots, C_p$ ses colonnes. L'hypothèse $\Rg A = 1$ se traduit par le fait que les colonnes de la matrice $A$ sont proportionnelles. Autrement dit, il existe un vecteur $X \defeq \Trsp{(x_1 \cdots x_n)}$ et $(\lambda_1, \dots, \lambda_p)$ tous non nuls tels que 
        $$\forall k \in \llbracket 1, p \rrbracket,\ C_k = \lambda_k X.$$
        En posant $Y \defeq \Trsp{(\lambda_1 \cdots \lambda_p)}$, on peut écrire $A = X \Trsp{Y}$.
        \item[$(\Leftarrow)$] Soient $X \in \M_{n,1}(\R)$ et $Y \in \M_{p,1}(\R)$, deux matrices colonnes non nulles. \\
        On note $Y \defeq \Trsp{(y_1 \cdots y_p)}$ où aucun des $y_i$ n'est nul. Alors,
        $$
        X \Trsp{Y} = X
        \times
        (y_1 \cdots y_p)
         = \big[y_1 X & \cdots & y_p X \big]
        $$
        Les colonnes de la matrice résultante sont proportionnelles donc cette matrice est de rang égal à $1$. 
    \end{itemize}
\end{preuve}

\begin{remarque}
    Il n'y a pas unicité du couple $(X, Y)$. En effet si ce couple convient, le couple $\big(2X, \frac{1}{2}Y \big)$ convient aussi.
\end{remarque}

\begin{exercice} \labexercice{carre_matrice_de_rang_1}
    Soit $A$ une matrice carrée de rang $1$. Montrer qu'il existe $\lambda \in \K$ tel que $A^2 = \lambda A$.
\end{exercice}

\begin{solution}
    D'après \vrefprop{caracterisation_matrice_de_rang_1}, il existe $X \in \M_{n,1}(\R)$ et $Y \in \M_{p,1}(\R)$ telles que $A = X \Trsp{Y}$. Ainsi, 
    \begin{align*}
        A^2 &= \big(X \Trsp{Y}\big) \big(X \Trsp{Y}\big) \\
        &= X \underbrace{\big( \Trsp{Y} X \big)}_{ \defeq \lambda \in \K} \Trsp{Y} \\
        A^2 &= \lambda A
    \end{align*}
    De plus $\lambda = \Tr A$.
\end{solution}

Compléter avec le problème 2 Partie I de \textsc{ccinp} \textsc{psi} 2022.

\begin{exercice}
    \marginnote[0cm]{Source : \href{https://www.elearning-cpge.com/classique/classiques-reduction/}{Classiques réduction -- \textsf{elearning-cpge.com}}} \\
    \begin{enumerate}
        \item Montrer que $A^2 = \Tr (A) A$.
        \item En déduire que la matrice $A$ est diagonalisable si et seulement si $\Tr A \not= 0$.
        \item Montrer que si $\Tr A \not= 0$, alors la matrice $A$ est semblable dans $\M_n(\R)$ à la matrice $\Diag(0, \dots, 0, \Tr A)$.
        \item On suppose que $\Tr A = 0$ et on désigne par $f$ l'endomorphisme de $\M_{n,1}(\K)$ canoniquement associé à $A$. 
        \begin{enumerate}
            \item Montrer que $U \in \Ker f$ et justifier l'existence d'une base de $\Ker f$ de la forme $(E_1, \dots, E_{n-2}, U)$. 
            \item Soit $W \defeq \frac{1}{\Trsp{V}V}V$. Montrer que $(E_1, \dots, E_{n-2}, U, W)$ est une base de $\M_{n,1}(\K)$ et écrire la matrice de $f$ dans cette base. 
            \item En déduire que deux matrices de rang $1$ et de trace nulle sont semblables dans $\M_n(\K)$.
        \end{enumerate}
    \end{enumerate}
\end{exercice}

\begin{solution}
    
\end{solution}
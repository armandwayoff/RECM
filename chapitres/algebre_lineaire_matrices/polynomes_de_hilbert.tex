\begin{prop} \labprop{hilbert_base}
    La famille $(\Hilb_0, \dots, \Hilb_n)$ des $n+1$ premiers polynômes de \textsc{Hilbert} forme une base de $\C_n[X]$.
\end{prop}

\begin{lemme} \lablemme{famille_deg_echelonnes_est_libre}
    Toute famille de polynômes non nuls à degrés échelonnés est libre.
\end{lemme}

\begin{preuve}
    \marginnote[0cm]{\url{https://www.bibmath.net/ressources/justeunexo.php?id=815}}
    Soit $(P_1, \dots, P_n)$ une famille de polynômes de $\C_n[X]$ non nuls, à degrés échelonnés, i.e. $0 \leqslant \deg P_1 < \cdots < \deg P_n$. \\
    Soit $(\lambda_1, \dots, \lambda_n)$ des scalaires tels que
    $$\lambda_1 P_1 + \cdots + \lambda_n P_n = 0. \quad (\star)$$
    Supposons par l'absurde que $\lambda_n \not= 0$. Alors le membre de gauche de l'égalité $(\star)$ est un polynôme de degré $\deg P_n \not= - \infty$ puisque tous les polynômes sont supposés non nuls. Ce membre ne peut donc pas être le polynôme nul. On aboutit à une contradiction et $\lambda_n = 0$. \\ 
    En itérant le raisonnement, on trouve successivement $\lambda_{n-1} = 0, \dots, \lambda_1 = 0$ ce qui assure la liberté de la famille $(P_1, \dots, P_n)$.
\end{preuve}

Revenons à la démonstration de la \refprop{hilbert_base}.

\begin{preuve}
    \marginnote[0cm]{
        \note Par définition, 
        $$\Hilb_0 \defeq 1$$
        $$\forall n \in \Ne,\ \Hilb_n \defeq \frac{1}{n!}X(X-1)\cdots(X-n+1).$$
        Donc pour tout $n \in \N$, $\deg \Hilb_n = n$.
    }
    Par construction, la famille $\mathscr{H} \defeq (\Hilb_0, \dots, \Hilb_n)$ est échelonnée en dégré \note ce qui assure sa liberté d'après le lemme. De plus, $|\mathscr{H}| = \dim \C_n[X]$ donc la famille $\mathscr{H}$ forme bien une base de $\C_n[X]$.
\end{preuve}
\begin{lemme}
    Si la somme des probabilités d'une suite $(A_n)_{n \in \N}$ d'événements d'un espace probabilisé $(\Omega, \mathscr{F}, \mathbb{P})$ est finie, alors la probabilité qu'une infinité d'entre eux se réalisent simultanément est nulle.
\end{lemme}

\begin{itemize}
    \item \href{https://www.youtube.com/watch?v=Yw2qk42EZcM}{\textsl{APPLICATION: Lemme de Borel-Cantelli - Partie 1} -- Hattab \textsc{Mouajria}}
    \item \href{https://www.youtube.com/watch?v=2GqPQY-mBpk}{\textsl{APPLICATION: Lemme de Borel-Cantelli - Partie 2} -- Hattab \textsc{Mouajria}}
    \item \cite{intro_graph_alea} II) §5 p. 34.
\end{itemize}
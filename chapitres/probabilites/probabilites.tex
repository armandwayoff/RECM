\chapter{Probabilités}
\labch{probabilites}

\section{Loi d'un maximum/minimum}
\input{chapitres/probabilites/loi_un_maximum_minimum}

\section{Lemmes de \textsc{Borel-Cantelli}}
\begin{lemme}
    Si la somme des probabilités d'une suite $(A_n)_{n \in \N}$ d'événements d'un espace probabilisé $(\Omega, \mathscr{F}, \mathbb{P})$ est finie, alors la probabilité qu'une infinité d'entre eux se réalisent simultanément est nulle.
\end{lemme}

\begin{itemize}
    \item \url{https://www.youtube.com/watch?v=Yw2qk42EZcM}
    \item \url{https://www.youtube.com/watch?v=2GqPQY-mBpk}
    \item \cite{intro_graph_alea} II) §5 page 34.
\end{itemize}

\section{Chaîne de \textsc{Markov}} \label{chaîne_markov}
% \begin{tikzpicture}
    \node[state] (s1) {État 1};
    \node[state, below right of=s1] (s2) {État 2};
    \node[state, below left of=s1] (s3) {État 3};
    
    \draw (s1) edge[loop above] node {$p_{1,1}$} (s1);
    \draw (s1) edge[bend left] node {$p_{1,2}$} (s2);
    \draw (s1) edge[bend right, above left] node {$p_{1,3}$} (s3);
    
    \draw (s2) edge[bend left, above right] node {$p_{2,1}$} (s1);
    \draw (s2) edge[loop right] node {$p_{2,2}$} (s2);
    \draw (s2) edge[bend right] node {$p_{2,3}$} (s3);
    
    \draw (s3) edge[bend right] node {$p_{3,1}$} (s1);
    \draw (s3) edge[bend right] node {$p_{3,2}$} (s2);
    \draw (s3) edge[loop left] node {$p_{3,3}$} (s3);
\end{tikzpicture}


\section{Exercice d'oral}
\marginnote[0cm]{\cite{acamanes}}
\begin{exercice}
    Soit $(\Omega, \mathscr{A}, \P)$ un espace probabilisé. 
    \begin{enumerate}
        \item Soit $B$ un ensemble non vide et $(A_{\beta})_{\beta \in B}$ une famille d'éléments deux à deux disjoints de $\mathscr{A}$ telle que pour tout $\beta \in B, \P(A_\beta) > 0$. Montrer que $B$ est au plus dénombrable. 
        \item Soit $X$ une variable aléatoire indépendante d'elle même. Montrer que $X$ est constante. 
    \end{enumerate}
\end{exercice} 

\marginnote{
\begin{methode}
    Écrire l'ensemble sous la forme d'une union finie ou dénombrable d'ensembles dénombrables ou finis.
\end{methode}
}

\begin{solution}
\begin{enumerate}
    \item Soit $I$ un ensemble non vide. 
    $$I = \bigcup_{n \in \Ne} \underbrace{\left \{ \beta \in B, \P(A_\beta) \geqslant \frac{1}{n} \right \}}_{\defeq I_n}.$$
    Soient $n, p \in \Ne$ et $(\beta_1, \dots, \beta_p) \in (I_n)^p$ deux à deux distincts. Alors
    \begin{align*}
        1 \geqslant \P \left( \bigsqcup_{i=1}^p A_{\beta_i} \right) &= \sum_{i=1}^p \P(A_{\beta_i}) \geqslant \sum_{i=1}^p \frac{1}{n}
    \end{align*}
    donc $p \leqslant n$. \\
    Ainsi, $I_n$ est fini et $|I_n| \leqslant n$ i.e. $I$ est au plus dénombrable. 
    \item \textcolor{red}{À vérifier} Soit $\omega \in X(\Omega)$. Par indépendance de $X$ avec elle-même,
    $$\P(X = \omega) = \P \big( \{X=\omega\} \cap \{X=\omega\} \big) = \P(X=\omega)^2.$$
    On en déduit que $\P(X=\omega) \in \{ 0, 1 \}$. \\
    De plus, $\sum\limits_{\omega \in X(\Omega)} \P(X=\omega) = 1$ et donc il existe un unique $\omega_0 \in X(\Omega)$ tel que $X=\omega_0$ presque sûrement i.e. $X$ est presque sûrement constante. 
\end{enumerate}
\end{solution}

\section{Fonction indicatrice d'\textsc{Euler}}
\begin{defi}{Fonction indicatrice d'\textsc{Euler}}
    La \emph{fonction indicatrice d'\textsc{Euler}} est une fonction arithmétique de la théorie des nombres, qui à tout entier naturel $n$ non nul associe le nombre d'entiers compris entre 1 et $n$ et premiers avec $n$. \\
    Autrement dit, 
    \begin{alignat*}{2}
        \varphi\ :\ \Ne\ &\longrightarrow\ \Ne\\
        n\ &\longmapsto\ \mathrm{Card} \big( \{m \in \Ne\ |\ m \leqslant n \text{ et } m \text{ premier avec } n \} \big).
    \end{alignat*}
\end{defi}

\marginnote{faire un graphe de $\varphi$}

\begin{prop}{}
    La fonction indicatrice d'\textsc{Euler} peut s'écrire
    $$\varphi\ :\ n \longmapsto n \cdot \prod_{p \in \mathscr{P}_n} \left(1 - \frac{1}{p} \right)$$
    en notant $\mathscr{P}_n$ l'ensemble des nombres premiers divisant $n$.
\end{prop}

\begin{preuve}
    (Wikipedia) \\
    La valeur de l'indicatrice d'\textsc{Euler} s'obtient à partir de la décomposition en facteurs premiers de $n$. On note $n = \prod\limits_{p \in \mathscr{P}_n} p^{k_i}$. Alors, $\varphi(n) = $
\end{preuve}

\begin{exercice}
    \marginnote[0cm]{Issu de la RMS 132 3 p.32.}
    Montrer que pour tout $n \in \Ne, \varphi(n) \geqslant \frac{\sqrt{n}}{2}$.
\end{exercice}

\begin{exercice}
    \marginnote[0cm]{\emph{Exercice 17. Chap. VI}}
    On note $\varphi$ la fonction indicatrice d'\textsc{Euler}. Montrer que 
    $$\forall n \in \Ne \quad n = \sum_{d|n} \varphi(d).$$
\end{exercice}

Les points suivants ne peuvent être compris qu'avec la correction.
\begin{itemize}
    \item Q2): Résultat à retenir (bien que rappeler dans le DS5):\\
    $$\forall p \in J \subset \P,\ p \text{ divise } a \Longleftrightarrow \prod_{p \in J} p \text{ divise } a$$
    \item Q3): un élément de $\Omega$ est premier avec $n$ si et seulement si il n'est divisible par aucun des diviseurs premiers de $n$. D'où
    $$\mathbb{P} \left(\{ m \in \Omega\ ;\ m \wedge n = 1 \} \right) = \prod_{p \in \mathscr{P}_n} \left(1 - \frac{1}{p} \right).$$
    \item Q4): Calculer le cardinal de $B_j = \{ j\cdot d,\ j \in \llbracket 1, k \rrbracket\ ;\ j \wedge k = 1 \}$.
    \item Q5): Montrer que $(B_d)_{d|n}$ forme un SCE de $\Omega$ et en déduire la formule de l'énoncé. $\displaystyle (\Omega = \bigcup_{d|n} B_d)$ \textcolor{green}{à compléter}
\end{itemize}

\section{\emph{Exercice 4. Chap. VII:}}
\begin{exercice}
    Lors d'une élection, 700 électeurs votent pour $A$ et 300 pour $B$. Quelle est la probabilité que, pendant le dépouillement, $A$ soit toujours strictement en tête?
\end{exercice}

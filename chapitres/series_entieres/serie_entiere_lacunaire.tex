\begin{exercice}
Calculer $\sum\limits_{n=0}^{+ \infty} \frac{x^{3n}}{(3n)!}$ pour tout $x \in \R$.
\end{exercice}

\begin{solution}
Soit $x \in \R$. 
\begin{align} \label{decomposition_somme_expo}
    \e^x = \sum_{n=0}^{+ \infty} \frac{x^n}{n!} = \underbrace{\sum_{n=0}^{+ \infty} \frac{x^{3n}}{(3n)!}}_{ \defeq A} + \underbrace{\sum_{n=0}^{+ \infty} \frac{x^{3n+1}}{(3n+1)!}}_{\defeq B} + \underbrace{\sum_{n=0}^{+ \infty} \frac{x^{3n+2}}{(3n+2)!}}_{\defeq C}.
\end{align}

%%% Commente par Alain car pb de compilation %%%
% \begin{marginfigure}
    % \begin{tikzpicture}[scale=0.8,declare function={angle=120;},bullet/.style={inner
    sep=1pt,fill,draw,circle,solid}]
    %--------------------------------------------------------------------------------------------
    %Axis
    \draw[thick,-latex,black] (-3.7,0)--(3.7,0); % x axis
    \draw[thick,-latex,black] (0,-3.7)--(0,3.7); % y axis
    %--------------------------------------------------------------------------------------------
    % Rest
    \path (0,0) coordinate (O);
    \draw (O) circle [radius=3cm]  (1,0);
    \draw[-latex] (O) -- (angle:3) node[above] {$\mj = \mj^{3n+1}$};
    \draw[-latex] (O) -- (2*angle:3) node[below] {\contour{white}{$\mj^2 = \overline{\mj} = \mj^{3n+2}$}};
    \draw[-{Latex[bend]}] (angle/2:0.6) node{$\frac{2\pi}{3}$} (0:1) arc(0:angle:1);
    \draw[thick,red] (angle:3) -- (3,0) node[above] {\contour{white}{\textcolor{black}{$1 = \mj^{3n}$}}};
    \draw[thick,red] (2*angle:3) -- (3,0);
    \draw[thick,red] (angle:3) -- (2*angle:3);
\end{tikzpicture}


% \end{marginfigure}

On note $\j \defeq \e^{\i \frac{2\pi}{3}}$. Soit $n \in \N$. On remarque que $(\j x)^{3n} = x^{3n}$, $(\j x)^{3n+1} = \j x^{3n+1}$ et $(\j x)^{3n+2} = \j^2 x^{3n+2}$ (la figure ci-contre permet de se rendre compte géométriquement de cette cyclicité). Donc en évaluant (\ref{decomposition_somme_expo}) respectivement en $\j x$ et en $\j^2 x$ on obtient deux nouvelles équations
$$\begin{cases}
    \e^x &= A + B + C \\
    \e^{\j x} &= A + \j B + \j^2 C \\
    \e^{\j^2 x} &= A + \j^2 B + \j C
\end{cases}.$$
Comme $1 + \j + \j^2 = 0$, en sommant ces trois relations on obtient
$$A = \frac{1}{3} \left(\e^x + \e^{\j x}+ \e^{\j^2 x} \right).$$
Cette quantité est bien réelle car $\j^2 = \overline{\j}$.
\end{solution}

\begin{remarque}
    On étend facilement le résultat aux séries entières $p$-lacunaires $\sum\limits_{n=0}^{+ \infty} \frac{x^{pn}}{(pn)!}$.
    \end{remarque}


    % -----------------

    %---------------

\begin{exercice}
{X-ENS}
{16}%
Soient $f$ et $g$ les fonctions définies pour tout $x \in \R_+$ par $f(x) = \int_0^1 \frac{e^{-(t^2+1) x^2}}{1 + t^2} \d t$ et $g(x) = \int_0^x e^{-t^2} \d t$.
\begin{questions}
\item Calculer $f(0)$ puis $\lim_{x\to+\infty} f(x)$.

\item Montrer que $f$ est de classe $\mathscr{C}^1$ sur $\R_+$ et que, pour tout $x \in \R_+$, $-2 g'(x) g(x) = f'(x)$.

\item En déduire $I = \int_0^{+\infty} e^{-t^2} \d t$.

Soit $h$ une fonction continue par morceaux, décroissante sur $\R_+$ telle que $\int_0^{+\infty} h(t) \d t$ soit convergente et non nulle.

\item Montrer que $h$ est à valeurs positives.

Pour tout réel positif $t$ non nul, on pose $S(t) = \sum_{n=1}^{+\infty} h(n t)$.
\item Montrer que $S$ existe.

\item Déterminer un équivalent de $S(t)$ lorsque $t$ tend vers $0^+$.

\item Déterminer un équivalent de $\sum_{n=1}^{+\infty} x^{n^2}$ lors que $x$ tend vers $1^-$.
\end{questions}
\end{exercice}

\begin{solution}
\begin{reponses}
\item D'après la définition, $f(0) = \int_0^1 \frac{1}{1 + t^2} \d t = \arctan(1) = \frac{\pi}{4}$.

On remarque que l'intégrande est majorée par $t \mapsto \frac{1}{1+ t^2}$ qui est intégrable donc en appliquant le théorème de convergence dominée,
\[
\lim_{x\to+\infty} f(x) = 0.
\]

\item Sur $[a, b]$, on majore la dérivée par $b^2$ qui est intégrable sur $[0, 1]$. Ainsi, d'après le théorème de dérivation sous le signe intégral,
\begin{align*}
f'(x) &= - \int_0^1 2 x e^{-(t^2+1) x^2} \d t \\
&= - 2 x e^{-x^2} \int_0^1 e^{-(t x)^2} \d t \\
&= - 2 x e^{-x^2} \int_0^x e^{-t^2} \frac{\d t}{x} \\
&= - 2 g'(x) g(x).
\end{align*}

\item D'après la question précédente,
\[
f(x) - f(0) = - (g(x)^2 - g(0)^2).
\]
Ainsi,
\[
\frac{\pi}{4} - f(x) = g(x)^2.
\]
La fonction $g$ étant à valeurs positives, $I = \frac{\sqrt{\pi}}{2}$.

\item $h$ est décroissante sur $\R_+$. Elle admet donc une limite en $+\infty$. Si cette limite (dans $\bar{\R}$) est égale à $\ell < 0$, alors $h(t) \leq \frac{\ell}{2}$ pour $t$ assez grand et $\int_0^{+\infty} h(t) \d t$ diverge. On raisonne de même pour $\ell > 0$. Ainsi, $h$ tend vers $0$ en $+\infty$ et $h$ est à valeurs positives.

\medskip

{2ème méthode (si $h$ est continue).} En utilisant le théorème des accroissements finis, $H(n+1) - H(n) = h(c_n)$ et le membre de gauche tend vers $0$ donc $\ell = 0$.

\item D'après la décroissance de $h$,
\begin{align*}
\sum_{n=1}^N h(n t) &= \sum_{n=1}^N (n t - (n-1)t) \frac{1}{t} h(n t) \\
&\leq \frac{1}{t} \sum_{n=1}^N \int_{(n-1)t}^{nt} h(u) \d u \\
&\leq \frac{1}{t} \int_0^{Nt} h(u) \d u.
\end{align*}
Ainsi, comme $\int_0^{+\infty} h(t) \d t$ converge, alors d'après les théorèmes sur les séries à termes positifs, $S$ converge.

\item D'après la question précédente, en utilisant une minoration,
\[
J - c t \leq t S(t) \leq J.
\]
Ainsi, comme $J \neq 0$, alors
\[
S(t) \sim_{t\to0} \frac{J}{t}.
\]

\item En posant $h(t) = e^{-t^2}$, la fonction $h$ est bien continue, décroissante et d'intégrale sur $\R_+$ convergente. Ainsi, d'après la question précédente, pour $x \in ]0, 1[$,
\[
S(\sqrt{-\ln(x)}) = \sum_{n=1}^{+\infty} x^{n^2}.
\]
Ainsi, d'après la question précédente,
\[
\sum_{n=1}^{+\infty} x^{n^2} \sim_1 \frac{\sqrt{\pi}}{2 \sqrt{\abs{\ln(x)}}}.
\]
\end{reponses}
\end{solution}
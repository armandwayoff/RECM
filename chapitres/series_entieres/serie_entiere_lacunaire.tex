\begin{exercice}
Calculer $\sum\limits_{n=0}^{+ \infty} \frac{x^{3n}}{(3n)!}$ pour tout $x \in \R$.
\end{exercice}

\begin{solution}
Soit $x \in \R$. 
\begin{align} \label{decomposition_somme_expo}
    \e^x = \sum_{n=0}^{+ \infty} \frac{x^n}{n!} = \underbrace{\sum_{n=0}^{+ \infty} \frac{x^{3n}}{(3n)!}}_{ \defeq A} + \underbrace{\sum_{n=0}^{+ \infty} \frac{x^{3n+1}}{(3n+1)!}}_{\defeq B} + \underbrace{\sum_{n=0}^{+ \infty} \frac{x^{3n+2}}{(3n+2)!}}_{\defeq C}.
\end{align}

\begin{marginfigure}
    \begin{tikzpicture}[scale=0.65,declare function={angle=120;},bullet/.style={inner
    sep=1pt,fill,draw,circle,solid}]
    %--------------------------------------------------------------------------------------------
    %Axis
    \draw[thick,-latex,black] (-3.7,0)--(3.7,0); % x axis
    \draw[thick,-latex,black] (0,-3.7)--(0,3.7); % y axis
    %--------------------------------------------------------------------------------------------
    % Rest
    \path (0,0) coordinate (O);
    \draw (O) circle [radius=3cm]  (1,0);
    \draw[-latex] (O) -- (angle:3) node[above] {$\mj = \mj^{3n+1}$};
    \draw[-latex] (O) -- (2*angle:3) node[below] {\contour{white}{$\mj^2 = \overline{\mj} = \mj^{3n+2}$}};
    \draw[-{Latex[bend]}] (angle/2:0.6) node{$\frac{2\pi}{3}$} (0:1) arc(0:angle:1);
    \draw[thick,red] (angle:3) -- (3,0) node[above] {\contour{white}{\textcolor{black}{$1 = \mj^{3n}$}}};
    \draw[thick,red] (2*angle:3) -- (3,0);
    \draw[thick,red] (angle:3) -- (2*angle:3);
\end{tikzpicture}


\end{marginfigure}

On note $\j \defeq \e^{\i \frac{2\pi}{3}}$. Soit $n \in \N$. On remarque que $(\j x)^{3n} = x^{3n}$, $(\j x)^{3n+1} = \j x^{3n+1}$ et $(\j x)^{3n+2} = \j^2 x^{3n+2}$ (la figure ci-contre permet de se rendre compte géométriquement de cette cyclicité). Donc en évaluant (\ref{decomposition_somme_expo}) respectivement en $\j x$ et en $\j^2 x$ on obtient deux nouvelles équations
$$\begin{cases}
    \e^x &= A + B + C \\
    \e^{\j x} &= A + \j B + \j^2 C \\
    \e^{\j^2 x} &= A + \j^2 B + \j C
\end{cases}.$$
Comme $1 + \j + \j^2 = 0$, en sommant ces trois relations on obtient
$$A = \frac{1}{3} \left(\e^x + \e^{\j x}+ \e^{\j^2 x} \right).$$
Cette quantité est bien réelle car $\j^2 = \overline{\j}$.
\end{solution}

\begin{remarque}
    On étend facilement le résultat aux séries entières $p$-lacunaires $\sum\limits_{n=0}^{+ \infty} \frac{x^{pn}}{(pn)!}$.
\end{remarque}
\begin{exercice}    
    Soient $\sum a_n$ et $\sum b_n$ des séries à termes positifs, on pose:
    $$f:x \mapsto \sum_{n=0}^{+\infty} a_n x^n \text{ et } g:x \mapsto \sum_{n=0}^{+\infty} b_n x^n.$$
    On suppose que les rayons de convergence valent $1$, que $\sum b_n$ diverge et que $a_n = o(b_n)$.\\
    Montrer que $g(x) \xrightarrow[x \to 1^-]{} + \infty$ et que $f = o_{1^-}(g)$.
\end{exercice}

\begin{elem_sol}
    \begin{itemize}
        \item Même méthode que pour la fonction zêta alternée.
        \item Soit $\varepsilon > 0$. Alors il existe $n_0 \in \N$ tel que pour tout $n \geqslant n_0$, $0 \leqslant a_n \leqslant \frac{\varepsilon}{2} b_n$. (détailler le dernier argument qui permet de conclure rigoureusement).
    \end{itemize}
\end{elem_sol}
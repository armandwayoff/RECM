\chapter{Séries entières}
\labch{series_entieres}


\subsection{Prolongement par continuité d'une fonction à variable réelle / Régularité du sinus cardinal sur $\R$}

La fonction $\fonction[f]{\Re}{\R}{x}{\frac{\sin(x)}{x}}$ admet un prolongement par continuité en $0$, appelé \emph{sinus cardinal}. En effet, considérons la fonction
\[
    g(x) \defeq
    \begin{cases} 
        \frac{\sin x}{x} &\text{si } x \not= 0 \\ 
        1 &\text{sinon} 
    \end{cases}.
\]
Cette fonction est continue en $0$ puisque, pour tout $x \in \Re$,
\[
\module{g(x) - g(0)} = \module{\frac{\sin(x) - 0}{x - 0} - 1} \xrightarrow[x \to 0]{} \module{\sin'(0) - 1} = \module{\cos(0) - 1} = 0. 
\]
La fonction $g$ constitue donc un prolongement de la fonction $f$ en $0$. \\
De plus, en utilisant le développement en série entière de la fonction sinus, on peut écrire pour $x$ réel non nul, 
\[
g(x) = \sum\limits_{n=0}^{+ \infty} (-1)^n \frac{x^{2n}}{(2n+1)!},
\]
ce qui reste vrai pour $x = 0$. La fonction $g$ est donc développable en série entière sur $\R$ et en particulier, la fonction $g$ est de classe $\mathscr{C}^\infty$ sur $\R$.
\marginnote[0cm]{fic00126}


\section{Matrice et série entière}
\begin{exercice}
    Soit $m \in \Ne$ et $A \in \M_m(\R)$ vérifiant $A^3 + A = 0$.\\
    Montrer que son rang est pair. Étudier la convergence et la somme de $\sum\limits_{n=0}^{+\infty} x^n \Tr (A^n)$.
\end{exercice}

\begin{solution}
    D'après l'énoncé, le polynôme $P(X) \defeq X^3 + X$ est annulateur de la matrice $A$. Le polynôme $P = X(X-\mi)(X+\mi)$ est scindé à racines simples dans $\C$ donc la matrice $A$ est diagonalisable dans $\C$. \\
    La diagonalisabilité de la matrice $A$ équivaut à $\sum\limits_{\lambda \in \Sp A} \dim E_\lambda(A) = m$ soit $\dim E_0(A) + \dim_{-\mi}(A) + \dim_{\mi}(A) = 0$. Comme $E_0(A) = \Ker A$, d'après le théorème du rang, $\dim E_0(A) = m - \Rg A$. On sait aussi (\textcolor{red}{à détailler peut être}) que $p \defeq \dim E_{-\mi}(A) = \dim E_{\mi}(A)$. On obtient alors
    \begin{align*}
        & m - \Rg A + 2p = m \\
        \text{soit } & \Rg A = 2p.
    \end{align*}
\end{solution}

\section{Série génératrice des polynômes d'\textsc{Hermite}}
\begin{defi}{Polynômes d'\textsc{Hermite}}
    La suite des polynômes d'\textsc{Hermite}, notée $(\Hermite_n)_{n \in \N}$, est définie comme l'unique suite de polynômes réels tels que:
    $$\forall(x, t) \in \R^2,\ \exp(tx-t^2/2) = \sum_{n=0}^{+\infty} t^n \Hermite_n(x) \qquad (*)$$
\end{defi}

\begin{preuve}
    \begin{itemize}
    \item $\blacktriangleright$ L'existence se prouve en faisant le produit de \textsc{Cauchy} des développements en série entière de $\exp(tx)$ et de $\exp(-t^2/2)$.\\
    $\blacktriangleright$ L'unicité se justifie par l'unicité des coefficients d'un développement en série entière. 
    \item Pour montrer que pour tout $n \in \Ne,\ (n+1)\Hermite_{n+1}=X\Hermite_n-\Hermite_{n-1}$, il faut dériver $(*)$ par rapport à $t$, effectuer des changements d'indices et utiliser l'unicité des coefficients du DES. 
    \item On peut montrer en dérivant  la série de fonctions $\sum \big(x \mapsto t^n \Hermite_n(x) \big)$ terme à terme sur $\R$ que pour tout $n \in \Ne,\ \Hermite'_n=\Hermite_{n-1}$.\\
    La dérivation est justifiée par l'application du \ptnclegras{théorème de dérivation terme à terme}, en particulier montrer soigneusement la CN de $\sum f'_n$ sur tout segment $[-a, a] \subset \R$ (qui entraîne la CU sur $\R$) $\left(|\Hermite'_n(x)| \leqslant \e^{|x|} \right)$.\\
    On en déduit que $(\Hermite_n)_n$ forme une base de $\R[X]$.\\
    \end{itemize}
\end{preuve}


\section{Théorème abélien ou taubérien sur les séries numériques}
Soit $f(x) = \sum\limits_{n=0}^{+\infty} a_n x^n$ une série entière de rayon de convergence égal à $1$.
\begin{itemize}
    \item \underline{Un théorème abélien:} si $\sum a_n$ est convergente, alors $f$ est définie et continue en $1$.
    \item \underline{Un théorème taubérien:} si $(a_n)_n$ est à termes positifs et $f$ a une limite à gauche en $1$, alors $\sum a_n$ est convergente, de somme $\displaystyle \lim_{1^-}f$.
\end{itemize}

\section{Fonction non développable en série entière}
\cite{contre-exemples} p. 263
\begin{alignat*}{2}
    \text{Soit } f\ :\ \R\ &\longrightarrow\ \R\\
    x\ &\longmapsto\ 
    \begin{cases}
        0 &\text{ si } x \leqslant 0,\\
        \exp \left(-\frac{1}{x^2}\right) &\text{ sinon}.
    \end{cases}
\end{alignat*}
        

\section{Comparaison de séries entières au bord}
Soient $\sum a_n$ et $\sum b_n$ des séries à termes positifs, on pose:
$$f:x \mapsto \sum_{n=0}^{+\infty} a_n x^n \text{ et } g:x \mapsto \sum_{n=0}^{+\infty} b_n x^n.$$
On suppose que les rayons de convergence valent $1$, que $\sum b_n$ diverge et que $a_n = o(b_n)$.\\
Montrer que $g(x) \xrightarrow[x \to 1^-]{} + \infty$ et que $f = o_{1^-}(g)$.

\begin{itemize}
    \item Même méthode que pour la fonction zêta alternée.
    \item Soit $\varepsilon > 0$. Alors il existe $n_0 \in \N$ tel que pour tout $n \geqslant n_0$, $0 \leqslant a_n \leqslant \frac{\varepsilon}{2} b_n$. (détailler le dernier argument qui permet de conclure rigoureusement).
\end{itemize}

\section{Développement en série entière}
\begin{exercice}  
    La fonction $f$ définie par:
    $$f(x) = \exp(x^2) \int_{x}^{+ \infty} \exp(-t^2)\ \d t$$
    est-elle développable en série entière ? Calculer les coefficients du développement en série entière à l'aide de factorielles, on utilisera que $\int_{0}^{+\infty} \exp(-t^2)\ \d t = \frac{\sqrt{\pi}}{2}$. 
\end{exercice}

\begin{itemize}
    \item Bien justifier l'existence de $f$ sur $\R$.
    \item L'écriture de $f$ sous la forme:
    $$\forall x \in \R,\ f(x) = \exp(x^2) \times \left(\int_{0}^{+ \infty} \exp(-t^2) \d t + \int_{0}^{x} \exp(-t^2) \d t\right)$$
    permet de justifier que $f$ est DSE (comme primitive d'une fonction DSE et d'un produit de fonctions DSE).
    \item Remarquer que $f$ vérifie
    $$f' -2xf + 1 = 0$$
    \item On en déduit que 
    $$\forall p \in \N,\ a_{2p} = \frac{\sqrt{\pi}}{2p!} \text{ et } a_{2p+1} = -\frac{2^{2p} p!}{(2p+1)!}.$$
\end{itemize}


\section{Série entière lacunaire}
\begin{exercice}
Calculer $\sum\limits_{n=0}^{+ \infty} \frac{x^{3n}}{(3n)!}$ pour tout $x \in \R$.
\end{exercice}

\begin{solution}
Soit $x \in \R$. 
\begin{align} \label{decomposition_somme_expo}
    \me^x = \sum_{n=0}^{+ \infty} \frac{x^n}{n!} = \underbrace{\sum_{n=0}^{+ \infty} \frac{x^{3n}}{(3n)!}}_{ \defeq A} + \underbrace{\sum_{n=0}^{+ \infty} \frac{x^{3n+1}}{(3n+1)!}}_{\defeq B} + \underbrace{\sum_{n=0}^{+ \infty} \frac{x^{3n+2}}{(3n+2)!}}_{\defeq C}.
\end{align}

\begin{marginfigure}
    \begin{tikzpicture}[scale=0.8,declare function={angle=120;},bullet/.style={inner
    sep=1pt,fill,draw,circle,solid}]
    %--------------------------------------------------------------------------------------------
    %Axis
    \draw[thick,-latex,black] (-3.7,0)--(3.7,0); % x axis
    \draw[thick,-latex,black] (0,-3.7)--(0,3.7); % y axis
    %--------------------------------------------------------------------------------------------
    % Rest
    \path (0,0) coordinate (O);
    \draw (O) circle [radius=3cm]  (1,0);
    \draw[-latex] (O) -- (angle:3) node[above] {$\mj = \mj^{3n+1}$};
    \draw[-latex] (O) -- (2*angle:3) node[below] {\contour{white}{$\mj^2 = \overline{\mj} = \mj^{3n+2}$}};
    \draw[-{Latex[bend]}] (angle/2:0.6) node{$\frac{2\pi}{3}$} (0:1) arc(0:angle:1);
    \draw[thick,red] (angle:3) -- (3,0) node[above] {\contour{white}{\textcolor{black}{$1 = \mj^{3n}$}}};
    \draw[thick,red] (2*angle:3) -- (3,0);
    \draw[thick,red] (angle:3) -- (2*angle:3);
\end{tikzpicture}


\end{marginfigure}

On note $\mj \defeq \me^{\mi \frac{2\pi}{3}}$. Soit $n \in \N$. On remarque que $(\mj x)^{3n} = x^{3n}$, $(\mj x)^{3n+1} = \mj x^{3n+1}$ et $(\mj x)^{3n+2} = \mj^2 x^{3n+2}$ (la figure ci-contre permet de se rendre compte géométriquement de cette cyclicité). Donc en évaluant (\ref{decomposition_somme_expo}) respectivement en $\mj x$ et en $\mj^2 x$ on obtient deux nouvelles équations
$$\begin{cases}
    \me^x &= A + B + C \\
    \me^{\mj x} &= A + \mj B + \mj^2 C \\
    \me^{\mj^2 x} &= A + \mj^2 B + \mj C
\end{cases}.$$
Comme $1 + \mj + \mj^2 = 0$, en sommant ces trois relations on obtient
$$A = \frac{1}{3} \left(\me^x + \me^{\mj x}+ \me^{\mj^2 x} \right).$$
Cette quantité est bien réelle car $\mj^2 = \overline{\mj}$.
\end{solution}

\begin{remarque}
    On étend facilement le résultat aux séries entières $p$-lacunaires $\sum\limits_{n=0}^{+ \infty} \frac{x^{pn}}{(pn)!}$.
\end{remarque}


\section{A rajouter:}
\begin{itemize}
    \item Transformée de \textsc{Laplace} d'une série entière
    \item Formule de \textsc{Cauchy} et applications
    \item Fonction de \textsc{Bessel} et intégrale de \textsc{Wallis}
\end{itemize}

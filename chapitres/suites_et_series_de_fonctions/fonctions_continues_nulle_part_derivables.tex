\epigraph{\emph{``Je me détourne avec effroi et horreur de cette plaie lamentable des fonctions continues qui n'ont point de dérivées.''}}{--- Charles \textsc{Hermite} (1893)}
    
\epigraph{\emph{``C’est un cas où il est vraiment naturel de penser à ces fonctions continues sans dérivées que les mathématiciens ont imaginées, et que l’on regardait à tort comme de simples curiosités mathématiques, puisque l’expérience peut les suggérer.''}}{--- Jean \textsc{Perrin} \footnote{Physicien, chimiste et homme politique français (1870 - 1942), prix Nobel de physique 1926.}, au sujet du mouvement brownien}
    
- thème Ch. 8 \\
- DS MPSI 2015 \\
- \url{https://share.miple.co/content/XEZ7y9BayeSN1} \\
- \url{http://christophebertault.fr/documents/articles/Article - Une famille nombreuse de fonctions continues partout derivables nulle part.pdf}

\begin{itemize}
    \item Intégrale à paramètre vs. série de fonctions
    \item Développements asymptotiques de sommes de séries de fonctions
\end{itemize}

%\begin{figure}[!h]
%   \begin{floatrow}
%\ffigbox{\input{bolzano-lebesgue}}%
%        {\caption{first figure}\label{fig:example-1}}
%\hfill
%\ffigbox{\input{blanc-manger}}%
%        {\caption{second figure}\label{fig:example-2}}
%   \end{floatrow}
%\end{figure}

Pour découvrir d'autres \emph{courbes remarquables}, lire le paragraphe éponyme dans \cite{contre-exemples} page 350. 
\marginnote[0cm]{
    \emph{Je me détourne avec effroi et horreur de cette plaie lamentable des fonctions continues qui n'ont point de dérivées.} \\
    Charles \textsc{Hermite} (1893) \\
    \emph{C’est un cas où il est vraiment naturel de penser à ces fonctions continues sans dérivées que les mathématiciens ont imaginées, et que l’on regardait à tort comme de 
    simples curiosités mathématiques, puisque l’expérience peut les suggérer.} \\
    Jean \textsc{Perrin} \footnote{Physicien, chimiste et homme politique français (1870 - 1942), prix Nobel de physique 1926.}, au sujet du mouvement brownien
}

- thème Ch. 8 \\
- DS MPSI 2015 \\
- \url{https://share.miple.co/content/XEZ7y9BayeSN1} \\
- \url{http://christophebertault.fr/documents/articles/Article - Une famille nombreuse de fonctions continues partout derivables nulle part.pdf}

\begin{itemize}
    \item Intégrale à paramètre vs. série de fonctions
    \item Développements asymptotiques de sommes de séries de fonctions
\end{itemize}

\begin{marginfigure}[-5cm]
	\input{illustrations/i_blanc_manger}
\end{marginfigure}

\begin{marginfigure}[1cm]
	\begin{tikzpicture}

\definecolor{gris}{RGB}{245,245,245}

\begin{axis}[width=6.5cm 
title={Courbe de \textsc{Bolzano-Lebesgue} d'ordre 4},
xtick={0, 1/3, 2/3, 1}, 
ytick={0, 1/3, 2/3, 1}
]
\addplot [line width=0.0032pt, cyan]
table {%
0 0
0.0123456790123457 0.197530864197531
0.0246913580246914 0.0987654320987654
0.037037037037037 0.296296296296296
0.0493827160493827 0.197530864197531
0.0617283950617284 0.246913580246914
0.0740740740740741 0.148148148148148
0.0864197530864197 0.345679012345679
0.0987654320987654 0.246913580246914
0.111111111111111 0.444444444444444
0.123456790123457 0.345679012345679
0.135802469135802 0.395061728395062
0.148148148148148 0.296296296296296
0.160493827160494 0.345679012345679
0.172839506172839 0.320987654320988
0.185185185185185 0.37037037037037
0.197530864197531 0.271604938271605
0.209876543209877 0.320987654320988
0.222222222222222 0.222222222222222
0.234567901234568 0.419753086419753
0.246913580246914 0.320987654320988
0.259259259259259 0.518518518518518
0.271604938271605 0.419753086419753
0.283950617283951 0.469135802469136
0.296296296296296 0.37037037037037
0.308641975308642 0.567901234567901
0.320987654320988 0.469135802469136
0.333333333333333 0.666666666666667
0.345679012345679 0.567901234567901
0.358024691358025 0.617283950617284
0.37037037037037 0.518518518518518
0.382716049382716 0.567901234567901
0.395061728395062 0.54320987654321
0.407407407407407 0.592592592592593
0.419753086419753 0.493827160493827
0.432098765432099 0.54320987654321
0.444444444444444 0.444444444444444
0.45679012345679 0.493827160493827
0.469135802469136 0.469135802469136
0.481481481481481 0.518518518518518
0.493827160493827 0.493827160493827
0.506172839506173 0.506172839506173
0.518518518518518 0.481481481481481
0.530864197530864 0.530864197530864
0.54320987654321 0.506172839506173
0.555555555555556 0.555555555555556
0.567901234567901 0.45679012345679
0.580246913580247 0.506172839506173
0.592592592592593 0.407407407407407
0.604938271604938 0.45679012345679
0.617283950617284 0.432098765432099
0.62962962962963 0.481481481481481
0.641975308641975 0.382716049382716
0.654320987654321 0.432098765432099
0.666666666666667 0.333333333333333
0.679012345679012 0.530864197530864
0.691358024691358 0.432098765432099
0.703703703703704 0.62962962962963
0.716049382716049 0.530864197530864
0.728395061728395 0.580246913580247
0.740740740740741 0.481481481481481
0.753086419753086 0.679012345679012
0.765432098765432 0.580246913580247
0.777777777777778 0.777777777777778
0.790123456790123 0.679012345679012
0.802469135802469 0.728395061728395
0.814814814814815 0.62962962962963
0.827160493827161 0.679012345679012
0.839506172839506 0.654320987654321
0.851851851851852 0.703703703703704
0.864197530864197 0.604938271604938
0.876543209876543 0.654320987654321
0.888888888888889 0.555555555555556
0.901234567901235 0.753086419753086
0.91358024691358 0.654320987654321
0.925925925925926 0.851851851851852
0.938271604938272 0.753086419753086
0.950617283950617 0.802469135802469
0.962962962962963 0.703703703703704
0.975308641975309 0.901234567901235
0.987654320987654 0.802469135802469
1 1
};
\end{axis}

\end{tikzpicture}


\end{marginfigure}

Pour découvrir d'autres \emph{courbes remarquables}, lire le paragraphe éponyme dans \cite{contre-exemples} page 350. 
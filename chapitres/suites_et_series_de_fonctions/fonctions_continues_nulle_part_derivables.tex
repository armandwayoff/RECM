\marginnote[0cm]{
\emph{\say{ Je me détourne avec effroi et horreur de cette plaie lamentable des fonctions continues qui n'ont point de dérivées. }} \\
\hspace*{\fill} Charles \textsc{Hermite} (1893) \\
\emph{\say{ C’est un cas où il est vraiment naturel de penser à ces fonctions continues sans dérivées que les mathématiciens ont imaginées, et que l’on regardait à tort comme de
simples curiosités mathématiques, puisque l’expérience peut les suggérer. }} \\
\hspace*{\fill} Jean \textsc{Perrin} \sidenote{Physicien, chimiste et homme politique français (1870 - 1942), prix Nobel de physique 1926.}, au sujet du mouvement brownien.
}

\cite{contre-exemples} p. 160 \\
Jusqu'à la moitié du \textsc{xix}$^\me$ siècle, on pensait généralement qu'une fonction continue était dérivable sauf peut-être en quelques points. \textsc{Ampère} prétendit même l'avoir démontré en 1806. \textsc{Bolzano} donne vers 1830 un exemple de fonction continue, mais dérivable nulle part; cependant ses écrits restent méconnus. En 1854, Bernhard \textsc{Riemann}, propose sans preuve, la fonction:
$$R:x \mapsto R(x) \defeq \sum_{n=1}^{+\infty} \frac{\sin(n^2 x)}{n^2}.$$
Karl \textsc{Weierstrass} se déclare incapable de la démontrer. Il faut attendre 1971 pour savoir que $R$ n'est pas dérivable sauf en certains points. En 1872, Karl \textsc{Weierstrass} démontre que si $a$ et $b$ sont des réels tels que $a > 0$, $b > 0$ et $ab > 1 + (3 \pi) / 2$, la fonction:
$$f : x \mapsto f(x) \defeq \sum_{n=1}^{+\infty} b^n \cos(a^n x)$$
est continue sur $\R$ et n'est dérivable en aucun point de $\R$.

\textbf{Notations.}\\
On note $E$ l'ensemble des fonctions continues sur $[0, 1]$ et $F$ l'ensemble des fonctions continues sur $[0, 1]$, nulle part dérivables sur $[0, 1]$. 

\begin{theo}{}
    L'ensemble $F$ est non vide.
\end{theo}

\begin{preuve}
    \marginnote[0cm]{\url{https://share.miple.co/content/XEZ7y9BayeSN1}}
    Notons:
    \begin{itemize}
        \item $\forall x \in \Rp, \Delta(x) \defeq \min\limits_{n \in \N} |x - n|$,
        \item $\forall x \in \Rp, \Delta_n(x) \defeq \frac{\Delta(2^n x)}{2^n x}$,
        \item $\forall x \in [0, 1], W(x) \defeq \sum\limits_{n=0}^\infty \Delta_n(x)$.
    \end{itemize}
    Nous allons montrer que la fonction $W$ est un élément de $F$.
    \begin{itemize}
    \item[$\rhd$] \textbf{Continuité.} La fonction $\Delta$ est minimale sur $\N$ où elle vaut $0$, et est maximale sur $\N + \frac{1}{2}$ où elle vaut $\frac{1}{2}$, d'où $\Ninf{\Delta} \leqslant \frac{1}{2}$ et donc $\Ninf{\Delta_n} \leqslant \frac{1}{2^{n+1}}$. Les fonction $\Delta_n$ étant continues et $\sum \Delta_n$ convergeant normalement, la fonction $W$ est continue. 
    \item[$\rhd$] \textbf{Non-dérivabilité.} Soit $x \in [0, 1]$. On note, pour tout $p \in \N$, 
    $$x_p \defeq \frac{\lfloor 2^p x \rfloor}{2^p} \text{ et } y_p \defeq x_p + \frac{1}{2^p}.$$ 
    Étudions la limite de $\limits \frac{W(x_p) - W(y_p)}{x_p - y_p}$ quand $p$ tend vers $+ \infty$:
    \begin{itemize}
        \item Si $n \geqslant p$, alors $2^n x_p$ et $2^n y_p$ sont des entiers. Donc $\Delta_n(x_p) = \Delta_n(y_p) = 0$ et $\frac{\Delta_n(x_p) - \Delta_n(y_p)}{x_p - y_p} = 0$.
        \item Si $n < p$, alors $x_p, y_p \in I_1 \defeq \left[ x_n, \frac{x_n + y_n}{2} \right]$ ou $x_p, y_p \in I_2 \defeq \left[ \frac{x_n + y_n}{2}, y_n \right]$. $\Delta_n$ a une pente $1$ sur $I_1$ et une pense $-1$ sur $I_2$. Donc $\frac{\Delta_n(x_p) - \Delta_n(y_p)}{x_p - y_p} = (-1)^{\lfloor 2^{n+1} x \rfloor}$. 
    \end{itemize}
    Donc:
    $$\frac{W(x_p) - W(y_p)}{x_p - y_p} = \sum_{n=0}^{p-1} (-1)^{\lfloor 2^{n+1} x \rfloor}.$$
    Notons $r_p \defeq \frac{W(y_p) - W(x_p)}{y_p - x_p}$. La suite $(r_p)_{p \in \N}$ ne converge par puisque $r_{p+1} - r_p = (-1)^{\lfloor 2^{n+1} x \rfloor}$ ne tend pas vers $0$. \\
    Comme $x_p \xrightarrow[p \to \infty]{} x$ et $y_p \xrightarrow[p \to \infty]{} x$, on en conclut que $W$ n'est pas dérivable en $x$ et donc est nulle part dérivable sur $[0, 1]$.
    \end{itemize}
\end{preuve}

\subsection{Courbe du blancmanger ou de \textsc{Takagi}}

\marginnote[0cm]{\cite{contre-exemples} p. 160}
En 1903, le mathématicien japonais Teiji \textsc{Takagi} (1875-1960) propose les fonctions:
$$f : x \mapsto f(x) \defeq \sum_{n=0}^{+ \infty} b^n g(a^n x)$$
où $g$ est la fonction de $g : x \mapsto d(x, \Z)$ (distance de $x$ à $\Z$) de $\R$ dans $\R$ et $a$ et $b$ des réels tels que $0 < b < 1$ et $a \leqslant 4$. Lorsque de plus $ab > 2$, la fonction $f$ a des dérivées supérieures égales à $+\infty$ et inférieures égales à $- \infty$ en tout point de $\R$. \\

\marginnote[0cm]{Alain \textsc{Camanes} DS MPSI1 10/01/2015}
La fonction de \textsc{Takagi} a été introduite par Teiji \textsc{Takagi} en 1903 motivé par les fonctions nulle part dérivables de \textsc{Weierstrass} après une visite en Allemagne de 1897 à 1901. \\
Une variante de la fonction de \textsc{Takagi}, utilisant la base $10$ et non la base $2$, a été introduite par \textsc{Van der Waerden} en  1930. Bien que non différentiables, ces fonctions possèdent des dérivées infinies en de nombreux points (tout comme la fonction de \textsc{Weierstrass}). C'est pourquoi \textsc{Knopp} a introduit en 1918 une variante de la fonction de \textsc{Takagi} qui, en tout point, n'admet ni de dérivée finie ni de dérivée infinie.

\begin{figure*}[h!]
    \centering
    \begin{tikzpicture}[scale=1]

    \begin{scope}[local bounding box=struct, scale=3]
        \draw[red] (0, 0) -- (1/2, 1/2) -- (1, 0);
    \end{scope}
    
    \begin{scope}[shift={($(struct.east)+(1,-3/4)$)}, scale=3]
        \draw[dashed] (1/4, 1/4) -- (1/2, 1/2) -- (3/4, 1/4);
        \draw[red] (0, 0) -- (1/4, 1/4) -- (1/2, 0) -- (3/4, 1/4) -- (1, 0);
        \draw (0, 0) -- (1/4, 1/2) -- (3/4, 1/2) -- (1, 0);
    \end{scope}
    
    \begin{scope}[shift={($(struct.east)+(5,-3/4)$)}, scale=3]
        \draw[dashed] (0, 0) -- (1/4, 1/2) -- (3/4, 1/2) -- (1, 0);
        \draw[red] (0, 0) -- (1/8, 1/8) -- (1/4, 0) -- (3/8, 1/8) -- (1/2, 0) -- (5/8, 1/8) -- (6/8, 0) -- (7/8, 1/8) -- (1, 0);
        \draw (0, 0) -- (1/8, 3/8) -- (3/8, 5/8) -- (1/2, 1/2) -- (5/8, 5/8) -- (7/8, 3/8) -- (1, 0);
    \end{scope}
    
    \begin{scope}[shift={($(struct.east)+(9,-3/4)$)}, scale=3]
        \draw[dashed] (0, 0) -- (1/8, 3/8) -- (3/8, 5/8) -- (1/2, 1/2) -- (5/8, 5/8) -- (7/8, 3/8) -- (1, 0);
        \draw[red] (0, 0) -- (1/16, 1/16) -- (2/16, 0) -- (3/16, 1/16) -- (4/16, 0) -- (5/16, 1/16) -- (6/16, 0) -- (7/16, 1/16) -- (8/16, 0) -- (9/16, 1/16) -- (10/16, 0) -- (11/16, 1/16) -- (12/16, 0) -- (13/16, 1/16) -- (14/16, 0) -- (15/16, 1/16) -- (1, 0);
        \draw (0, 0) -- (1/16, 4/16) -- (3/16, 1/2) -- (4/16, 1/2) -- (5/16, 5/8) -- (7/16, 5/8) -- (1/2, 1/2) -- (9/16, 5/8) -- (11/16, 5/8) -- (12/16, 1/2) -- (13/16, 1/2) -- (15/16, 4/16) -- (1, 0);
    \end{scope}
\end{tikzpicture}
    \caption*{\centering Construction graphique}
\end{figure*}

\begin{figure}[h!]
    \centering
	\input{illustrations/i_blanc_manger}
\end{figure}

\begin{defi}{Distance à $\Z$}
    Pour tout $x \in \R$, on définit la fonction $\ll x \gg$ distance de $x$ à $\Z$. 
\end{defi}

\begin{prop}{Expression de $\ll \cdot \gg$}
$$\ll \cdot \gg : x \mapsto \left| x - \left\lfloor x + \frac{1}{2} \right\rfloor \right|$$
\end{prop}

\begin{preuve}
    
\end{preuve}

\begin{defi}{Courbe du blancmanger ou de \textsc{Takagi}}
On définit cette courbe par la fonction
    \begin{alignat*}{2}
        \tau\ :\ [0,1]\ &\longrightarrow\ [0,1]\\
        x\ &\longmapsto\ \sum\limits_{k=0}^\infty \frac{1}{2^k} \ll 2^k x \gg.
    \end{alignat*}
\end{defi}

\subsection{Courbe de \textsc{Bolzano}-\textsc{Lebesgue}}    

\begin{defi}{Courbe de \textsc{Bolzano}-\textsc{Lebesgue}}
    On pose $I \defeq [0, 1]$ et $(f_n)$ la suite de fonctions définie par
    \begin{itemize}
        \item $f_0(x) = x.$
        \item $f_n$ est affine sur $\left[ \frac{k}{3^n}, \frac{k+1}{3^n} \right]$ pour tout $k \in \llbracket 0, 3^n - 1 \rrbracket$.
        \item $f_n$ et $f_{n-1}$ sont égales en $\frac{3k}{3^n}$, $\frac{3k+1}{3^n}$ et $\frac{3k+2}{3^n}$ pour tout $k \in \llbracket 0, 3^n-1 \rrbracket$.
    \end{itemize}
\end{defi}

\subsection{Densité de $F$ dans $E$}

\begin{theo}{}
    $F$ est dense dans $E$ pour la topologie uniforme.
\end{theo}

\begin{preuve}
    \marginnote[0cm]{\url{https://share.miple.co/content/XEZ7y9BayeSN1}}
    Soient $f \in E$ et $W \in F$. La fonction $f - W$ est continue sur $[0, 1]$ donc peut être approchée uniformément par une suite $(A_n)_{n \in \N}$ de fonctions polynomiales définies sur $[0, 1]$ d'après le théorème de \textsc{Weierstrass}. \\
    Pour tout $n \in \N$, notons $B_n \defeq A_n + W$. La fonction $B_n$ est continue et nulle part dérivable puisque si $(B_n)_{n \in \N}$ était dérivable en $x \in [0, 1]$, la fonction $W$ le serait aussi. La suite $(B_n)_{n \in \N}$ est donc une suite de fonctions continues sur $[0,1]$ et nulle part dérivable qui converge uniformément vers la fonction $f$.
\end{preuve}

- \url{http://christophebertault.fr/documents/articles/Article - Une famille nombreuse de fonctions continues partout derivables nulle part.pdf} \\
- lire le paragraphe éponyme dans \cite{contre-exemples} page 350. \\

\section{A rajouter}

\begin{itemize}
    \item Intégrale à paramètre vs. série de fonctions
    \item Développements asymptotiques de sommes de séries de fonctions
\end{itemize}
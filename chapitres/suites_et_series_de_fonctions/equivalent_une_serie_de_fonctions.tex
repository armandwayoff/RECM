\begin{exercice}
    On pose $f(x) = \sum\limits_{n=1}^{+\infty}\frac{x}{n(1+nx^2)}$. Donner un équivalent de $f$ et $0^+$.
\end{exercice}

\begin{solution}
    Éffectuer une comparaison série/intégrale aux termes de la somme pour encadrer $f$ (ne pas oublier de justifier l'intégrabilité des fonctions sur $[1, +\infty[$):
    $$\int_{2}^{+\infty} \frac{x}{t(1+tx^2)} \d t + \frac{x}{1+x^2} \leqslant f(x) \leqslant \int_{1}^{+\infty} \frac{x}{t(1+tx^2)} \d t + \frac{x}{1+x^2}.$$ 
    Décomposer les intégrandes en éléments simples et calculer les intégrales:
    $$-2x \ln(x) + o_0(x\ln(x)) \leqslant f(x) \leqslant -2x \ln(x) + o_0(x\ln(x))$$
    Finalement, 
    $$f(x) \sim_{0^+} -2x\ln(x)$$
\end{solution}
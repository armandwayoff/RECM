\begin{prop}{}
    \marginnote[0cm]{Source : \cite{contre-exemples} p.257}
    Soit la suite $(f_n)_{n \geqslant 0}$ d'applications continues de $[0,1]$ dans $\R$, de terme général
    \begin{alignat*}{2}
        f_n\ :\ [0,1]\ &\longrightarrow\ \R\\
        x\ &\longmapsto\ f_n(x) = (1-x)x^n.
    \end{alignat*}
    La somme des $f_n$ est discontinue.
\end{prop}

\begin{preuve}
    Pour tout point $x$ de $[0,1[$, la série $\sum f_n(x)$ est le produit par $(1-x)$ d'une série géométrique de raison $x$ où $0 \leqslant x < 1$, donc elle converge et sa somme est égale à $(1-x)/(1-x) = 1$. De plus $f_n(1) = 0$ pour tout entier naturel $n$. La série de fonctions $\sum f_n$ converge donc simplement sur le segment $[0,1]$ et sa somme est l'application 
    \begin{alignat*}{2}
        S\ :\ [0,1]\ &\longrightarrow\ \R\\
        x\ &\longmapsto\ S(x) =
        \begin{cases}
            1 &\text{ si } x \in [0,1[,\\
            0 &\text{ si } x = 1,
        \end{cases}
    \end{alignat*}
    qui est discontinue en $1$.
\end{preuve}


\textsl{Le Calcul infinitésimal est l'apprentissage du maniement des \emph{inégalités} bien plus que des égalités, et on pourrait se résumer en trois mots:}
\begin{center}
        \textsc{majorer, minorer, approcher.}
\end{center}
\marginnote[0cm]{V.1 Écart de deux fonctions de  \cite{calcul_infinitesimal}}
\textsl{De même qu'on cherche à \emph{approcher} un \emph{nombre} inconnu (défini par un procédé quelconque) à l'aide de nombres décimaux (ou rationnels), de même il est naturel en Analyse de chercher à \say{ approcher } une \emph{fonction} complexe inconnue (qui peut être définie par des procédés variés, somme de série, intégrale dépendant d'un paramètre, solution d'équation différentielle, etc.) à l'aide de fonctions que l'on considère comme \emph{connues} (polynômes, fonctions exponentielles, fonctions trigonométriques, etc.). Mais il faut préciser ce qu'on entend par \say{ approcher }, c'est-à-dire \say{ mesurer } en quelque sorte l'\say{ écart } de deux fonctions, de même que la valeur absolue $|x-y|$ mesure l'écart de deux nombres réels ou complexes. \\
L'idée la plus naturelle est que si une fonction $g$ \say{ approche } une fonction $f$ dans un ensemble $E$ où elles sont toutes deux définies, alors, pour chaque $x_0 \in E$, la \emph{valeur} $g(x_0)$ de $g$ doit \emph{approcher} la \emph{valeur} $f(x_0)$ de $f$ au sens usuel, c'est-à-dire que $|f(x_0)-g(x_0)|$ doit être \say{ petit }. Comme ceci doit avoir lieu en \emph{chaque} poit $x_0$ de $E$, on est conduit à prendre pour \say{ écart } de deux fonctions complexes $f$, $g$ définies dans $E$ le nombre
$$d(f, g) \defeq \sup_{x \in E} |f(x)-g(x)|.$$
Lorsqu'il s'agit de fonctions \emph{réelles} $f, g$ définies dans un intervalle $E = [a, b]$ de $\R$, l'idée d'\say{ écart } que nous venons de définir peut se concrétiser graphiquement de la façon suivante: dire que $d(f,g) \leqslant \varepsilon$ signifie que pour tout $x \in E$ on a $g(x)-\varepsilon \leqslant f(x) \leqslant g(x) + \varepsilon$, c'est-à-dire que le graphe de $f$ est \emph{tout entier} contenu dans la \say{ bande } de demi-largeur $\varepsilon$ autour du graphe de $g$. \\
Pour distinguer cette idée d'\say{ approximations } d'autres notions, nous dirons qu'il s'agit d'\emph{approximation uniforme} d'une fonction par une autre dans un ensemble $E$ où elles sont toues deux définies; il est important de remarquer que cette notion \emph{dépend essentiellement} de l'ensemble $E$ que l'on considère: si $f$ et $g$ sont toutes deux définies dans un ensemble plus grand $E'$, la relation $|f(x) - g(x)| \leqslant \varepsilon$ pour $x \in E$ n'entraîne nullement $|f(x)-g(x)| \leqslant \varepsilon$ pour $x \in E'$. \\
Étant donnés deux ensembles de fonctions $\mathscr{F}$ (les fonctions \say{ inconnues }) et $\mathscr{G}$ (les fonctions \say{ connues }) toutes définies dans un même ensemble $E$, nous dirons pour abréger qu'\emph{on peut approcher uniformément dans} $E$ les fonctions de $\mathscr{F}$ par les fonctions de $\mathscr{G}$ si, pour \emph{toute fonction} $f \in \mathscr{F}$ et \emph{tout nombre} $\varepsilon > 0$, il existe une fonction $g \in \mathscr{G}$ (dépendant de $f$ et de $\varepsilon$) telle que l'écart $d(f,g) \leqslant \varepsilon$, c'est-à-dire que 
$$|f(x) - g(x)| \leqslant \varepsilon \quad \text{pour tout } x \in E.$$
}
% https://tex.stackexchange.com/questions/383609/draw-illustration-uniform-convergence

\begin{marginfigure}[-10cm]
    % https://tex.stackexchange.com/questions/383609/draw-illustration-uniform-convergence

\begin{tikzpicture}[auto,
B/.style = {decorate,
            decoration={calligraphic brace, amplitude=3pt,
            pre=moveto,pre length=1pt,post=moveto,post length=1pt,
            raise=1mm}},
   domain = -30:30, samples=10, smooth,
     font = \footnotesize
                        ]
% coordinates
\draw[-latex] (-0.2,0) -- (6,0) node[right]{$x$};
\draw[-latex] (0,-0.2) -- (0,4) node[above]{$\textcolor{red}{f(x)}\ \textcolor{blue}{g(x)}$};
% curve
\draw[thick, red] 
    plot ({(45+\x)/15},{1+rand/5+2*cos(\x)});
    % node[coordinate,pin=0:$f$] {};
% convergence borders
\draw[dashed]  plot ({(45+\x)/15},{1.5+2*cos(\x)}) coordinate (e1);
\draw[thick, blue] plot ({(45+\x)/15},{1.0+2*cos(\x)}) coordinate (e2);
\draw[dashed]  plot ({(45+\x)/15},{0.5+2*cos(\x)}) coordinate (e3);
% labels on the left side
% \draw[B] (1,{1.25+cos(-10)}) -- node[left=1.5mm]   {$g$} ++ (0,1);
% labels on the right
\draw[B] (e1) -- node[right=1.5mm]   {$\varepsilon$} (e2);
\draw[B] (e2) -- node[right=1.5mm]   {$\varepsilon$} (e3);
% domain
\draw[dashed]   (1,{0.5+2*cos(30)}) -- (1,0)
                (5,{0.5+2*cos(30)}) -- (5,0);
\draw[very thick]    (1,0) to ["$E$" '] (5,0);
\end{tikzpicture}
\end{marginfigure}

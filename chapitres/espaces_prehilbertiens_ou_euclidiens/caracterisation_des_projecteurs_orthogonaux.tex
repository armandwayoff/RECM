\begin{prop}
    Soient $E$ un espace euclidien et $p$ un projecteur de $E$. Alors $p$ est un projecteur orthogonal si et seulement si, pour tout $x \in E$,
    $$\norme{p(x)} \leqslant \norme{x}.$$
\end{prop}

\begin{preuve}
    \begin{itemize}
        \item $(\Rightarrow)$ Il existe $F$ un sev de $E$ tel que $p$ soit la projection sur $F$ parallèlement à $F^\perp$. On décompose tout vecteur de $E$ comme la somme unique d'un élément de $F$ et de $F^\perp$ puis on applique le théorème de \textsc{Pythagore}. 
        \item $(\Leftarrow)$ 
        \begin{itemize}
            \item (Exos incontournables SUP) Raisonner par l'absurde. Soit $F$ et $G$ tels que $p$ soit la projection sur $F$ parallèlement à $G$. Considérer un vecteur de $G^\perp\ \backslash\ F$ et aboutir à une contradiction.
            \item (Ellipses p.176) Soit $p$ une projection telle que pour tout $x \in E, \norme{p(x)} \leqslant \norme{x}$. \\
            Nous allons poser un vecteur \say{ variable }.
            \marginnote{$p(x+ty) = ty$}
            Soit $x \in \Ker p$ et $y \in \Im p$, pour tout $t \in \R$, 
            \begin{align*}
                \norme{ty} \leqslant \norme{x + ty} &\Leftrightarrow t^2 \norme{y}^2 \leqslant \norme{x+ty}^2 \\
                &\Leftrightarrow t^2 \norme{y}^2 \leqslant \norme{x}^2 + t^2 \norme{y}^2 + 2t \langle x, y \rangle \\
                &\Leftrightarrow \norme{x}^2 + 2t \langle x, y \rangle \geqslant 0 \\
                &\Rightarrow \langle x, y \rangle = 0 \text{ car cette inégalité est vraie pour tout } t \in \R
            \end{align*}
            Donc $\Ker p$ et $\Im p$ sont orthogonaux et $p$ est un projecteur orthogonal.
        \end{itemize}
    \end{itemize}
\end{preuve}

% \begin{marginfigure}[-4cm]
    % % \tdplotsetmaincoords{70}{200}

% \end{marginfigure}

%\begin{marginfigure}[-1cm]
%    \includegraphics[width=5cm]{images/jacques_hadamard.jpg}
%    \caption{Jacques \textsc{Hadamard}}
%\end{marginfigure}

\begin{prop}
    Soit $M \in \M_n(\C)$ et soient $X_1, \dots, X_n$ les vecteurs colonnes de $M$. Alors,
    $$|\mathrm{det}(M)| \leqslant \prod_{i=1}^{n} \Vert X_i \Vert$$
    avec égalité si et seulement si la famille $(X_i)$ est orthogonale.
\end{prop}

%\begin{marginfigure}[-2cm]
%    \begin{tikzpicture}[scale=0.7]
    \node[block] (gram) {Déterminant \\ de \textsc{Gram}};
    \node[block, right of=s1] (iwasama) {Décomposition \\ d'\textsc{Iwasama}};
    \node[block, below right of=gram] (hadamard) {Inégalité \\ d'\textsc{Hadamard}};
    \draw (gram) edge[bend right, above left] node {} (hadamard);
    \draw (iwasama) edge[bend left, above right] node {} (hadamard);
\end{tikzpicture}    
%\end{marginfigure}

\begin{preuve}

\end{preuve}

L'inégalité d'\textsc{Hadamard} nous apprend en fait que le volume du parallélotope défini par les vecteurs colonnes est inférieur ou égal au produit des normes de ses vecteurs et il y a égalité si et seulement si la matrice est diagonale, ou encore que le parallélotope est rectangle. 

\marginnote[-2cm]{
    \begin{kaobox}[frametitle=Parallélotope]
        Soit $(x_1, \dots, x_n)$ une famille libre. Le parallélotope engendré par cette famille est défini par
        $$P \defeq \left\{ x = \sum_{i=1}^n t_i x_i,\ \forall i\ t_i \in [0,1]\right\}.$$
    \end{kaobox}
}

\begin{prop}
    Soient $\mathscr{S}_n ^{++} (\R)$ l'ensemble des matrices réelles d'ordre $n$ symétriques à valeurs propres strictement positives et $A = (a_{i,j}) \in \mathscr{S}_n ^{++} (\R)$. Alors,
    $$\det(A) \leqslant \prod_{i=1}^{n} a_{i,i}.$$
\end{prop}

\underline{Démonstration:} (exercice 4, TD 14 \cite{acamanes})

\begin{enumerate}
    \item Soit $(\gamma_1, \dots, \gamma_n) \in (\Re)^n$. Montrer que $B = (\gamma_i \gamma_j a_{i,j}) \in \mathscr{S}_n^{++}(\R)$. 
    \item Montrer que $\det(A)^{1/n} \leqslant \frac{\Tr(A)}{n}$. \\
    \emph{On pourra utiliser l'inégalité arithmético-géométrique}.
    
    \marginnote[-2cm]{
    	\begin{kaobox}[frametitle=Inégalité arithmético-géométrique]
            Soient $n \in \Ne$ et $x_1, \dots, x_n$ des réels positifs. Alors, 
            $$\frac{x_1 + \cdots + x_n}{n} \geqslant \sqrt[n]{x_1 \cdots x_n}.$$
            Il y a égalité si et seulement si tous les $x_i$ sont égaux.
        \end{kaobox}
        Pour d'autres inégalités, lire le chapitre 16, p.117 de la deuxième édition de Raisonnements divins (en fr)
    }
    
    \item Montrer que pour tout $i \in \llbracket 1, n \rrbracket, a_{i,i} > 0$. On pose $\gamma_i = \frac{1}{\sqrt{a_{i,i}}}$. En déduire l'inégalité d'\textsc{Hadamard}.
\end{enumerate}
\begin{tcolorbox}
    Soient $\mathscr{S}_n ^{++} (\R)$ l'ensemble des matrices réelles d'ordre $n$ symétriques à valeurs propres strictement positives et $A = (a_{i,j}) \in \mathscr{S}_n ^{++} (\R)$. Alors,
    $$\det(A) \leqslant \prod_{i=1}^{n} a_{i,i}.$$
\end{tcolorbox}

\underline{Démonstration:} (exercice 4, TD 14 \cite{acamanes})

\begin{enumerate}
    \item Soit $(\gamma_1, \dots, \gamma_n) \in (\Re)^n$. Montrer que $B = (\gamma_i \gamma_j a_{i,j}) \in \mathscr{S}_n^{++}(\R)$. 
    \item Montrer que $\det(A)^{1/n} \leqslant \frac{\Tr(A)}{n}$. \\
    \emph{On pourra utiliser l'inégalité arithmético-géométrique}.
    
    \marginnote[-2cm]{
    	\begin{kaobox}[frametitle=Inégalité arithmético-géométrique]
            Soient $n \in \Ne$ et $x_1, \dots, x_n$ des réels positifs. Alors, 
            $$\frac{x_1 + \cdots + x_n}{n} \geqslant \sqrt[n]{x_1 \cdots x_n}.$$
            Il y a égalité si et seulement si tous les $x_i$ sont égaux.
        \end{kaobox}
        Pour d'autres inégalités, lire le chapitre 16, p.117 de la deuxième édition de Raisonnements divins (en fr)
    }

    \item Montrer que pour tout $i \in \llbracket 1, n \rrbracket, a_{i,i} > 0$. On pose $\gamma_i = \frac{1}{\sqrt{a_{i,i}}}$. En déduire l'inégalité d'\textsc{Hadamard}.
\end{enumerate}
\begin{tcolorbox}
    Soit $E$ un espace euclidien de dimension $n \geqslant 1$ et $\mathscr{B}$ une base orthonormée de $E$. Pour tout $n$-uplet $(x_1, \dots, x_n)$, on a:
    $$\left| \mathrm{det}_{\mathscr{B}}(x_1, \dots, x_n) \right| \leqslant \prod_{i=1}^{n} \Vert x_i \Vert.$$
\end{tcolorbox}

Détailler le lien avec la \nameref{matrice_gram}. \\

\underline{Démonstration:} (\url{https://bibmath.net/dico/index.php?action=affiche&quoi=./i/ineghadamard.html}) \\

\textcolor{green}{Notations et rédaction à revoir}

Dans le cas où aucun des $x_i$ n'est nul, on a égalité si, et seulement si, les vecteurs $(x_1, \dots  ,x_n)$ sont orthogonaux deux à deux. \\
La démonstration de cette inégalité n'est pas très difficile : si les vecteurs colonnes $(x_1, \dots  ,x_n)$ sont liés, le déterminant est nul et l'inégalité est triviale. Sinon, $(x_1, \dots  ,x_n)$ forme une base de $\R^n$ à laquelle on peut appliquer le procédé de \textsc{Gram}-\textsc{Schmidt} pour obtenir une base orthonormale $(\widetilde{x}_1, \dots, \widetilde{x}_n)$ telle que pour tout $i \in \llbracket 1, n \rrbracket$, 
$$\Vect(x_1, \dots  ,x_i) = \Vect(\widetilde{x}_1, \dots, \widetilde{x}_i)$$.

Si $P$ est la matrice de passage de $(\widetilde{x}_1, \dots, \widetilde{x}_n)$ à $(x_1, \dots  ,x_i)$, $P$ est une matrice triangulaire supérieure. Remarquons que $M$ est la matrice de passage de la base canonique $(e_1, \dots, e_n)$ à $(x_1, \dots  ,x_i)$, et notons $B$ la matrice des vecteurs colonnes $(\widetilde{x}_1, \dots, \widetilde{x}_i)$. Les formules de changement de base donnent $M=BP$. D'autre part, $B$ est une matrice orthogonale (c'est une matrice de passage d'une base orthonormale à une autre), et en particulier$|\det(B)|=1$. On a donc : 
$$| \det(M) | = | \det(P) | = |p_{1,1} | \cdots |p_{n,n}|$$
puisque $P$ est triangulaire supérieure. Mais, 
$$|p_{i,i}| = |\langle x_i, \widetilde{x}_i \rangle \leqslant \Vert x_i \Vert$$
et ceci achève la démonstration. \\
Géométriquement, l'inégalité exprime que, pour des côtés de longueur donnée, un parallélépipède est de volume maximal s'il est rectangle. \textcolor{green}{en faire une représentation}
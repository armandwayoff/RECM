\marginnote[0cm]{(Planche n°14. Espaces euclidiens de \cite{maths-france})}
Si on interprète le terme général de la matrice de \textsc{Hilbert} (cf. \nameref{matrice_hilbert}) comme 
$$\Hilb_{i,j} = \int_{0}^{1} x^{i+j-2} \d x$$
on peut y reconnaître une \nameref{matrice_gram} pour les fonctions puissances et le produit scalaire usuel sur l'espace des fonctions de $[0, 1]$ dans $\R$ de carré intégrable. Puisque les fonctions puissances sont linéairement indépendantes, les matrices de \textsc{Hilbert} sont donc définies positives.

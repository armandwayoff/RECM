\frefsec{polynomes_de_legendre}

$$\forall n \in \N, \Leg_n(X) = \frac{1}{2^n n!} U_n^{(n)}(X)$$
où $U_n(X) = (X^2-1)^n$.

Les polynômes de \textsc{Legendre} constituent l'exemple le plus simple d'une suite de polynômes orthogonaux. \\
La suite vient de \url{https://bibmath.net/dico/index.php?action=affiche&quoi=./l/legendrepoly.html}.
Les polynômes de \textsc{Legendre} sont orthogonaux pour le produit scalaire
$$\langle P, Q \rangle = \int_{-1}^1 P(t) Q(t) \d t$$
mais ne sont toutefois par orthonormaux car
$$\langle \Leg_n, \Leg_n \rangle = \frac{2}{2n+1}.$$

\begin{enumerate}
    \item Montrer que $(\Leg_n)_{n \in \N)}$ est une famille orthogonale. \\
    $-1$ et $+1$ sont des racines d'ordre $n$ de $U_n$ donc:
    $$\forall i \in \llbracket 1, n-1 \rrbracket,\ U_n^{(n)}(-1) = U_n^{(n)}(1) = 0 \quad (*)$$
    On calcule $\int_{-1}^{1} U_n^{(n)}(t) \times U_m^{(m)}(t)\d t$ en faisant une \textbf{intégration par parties} un intégrant $U_m^{(m)}$. D'après $(*)$, le crochet est nul. On répète l'opération $n+1$ fois. On obtient alors en facteur dans l'intégrande $U_n^{(2n+1)} = 0$ car $\mathrm{deg}(U_n) = 2n$.
\end{enumerate}
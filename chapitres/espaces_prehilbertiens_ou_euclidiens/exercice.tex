\begin{exercice}
    Issu de la RMS 132 3. \\
    Agrégation Interne de Mathématiques (première épreuve 2022) \\
    Vrai ou faux: \say{ les matrices carrées et symétriques à coefficients dans $\C$ sont diagonalisables. }
\end{exercice}

\begin{solution}
    \newline
    \begin{lemme}
        Une matrice nilpotente non nulle n'est pas diagonalisable.
    \end{lemme}
    
    \begin{preuve}
        Soit $A \in \M_n(\R)$ une matrice nilpotente diagonalisable. \\
        Alors il existe $P \in \Gl_n(\R)$ et $D$ une matrice diagonale telles que $A = PD\Inv{P}$. Or $A$ est nilpotente donc il existe $p \in \N$ tel que $A^p = P D^p \Inv{P} = 0$. Donc $D^p = 0$ soit $D = 0$ et $A = 0$. \\
        On peut aussi dire qu'une matrice ayant une unique valeur propre (comme c'est la cas des matrices nilpotentes) est diagonalisable si et seulement si elle est diagonale.
    \end{preuve}
    Cette affirmation est fausse. En effet en taille $2$, la matrice $A = \begin{pmatrix}
        1 & \mi \\
        \mi & -1
    \end{pmatrix}$ est symétrique et non nulle; elle vérifie $A^2 = 0$. Or d'après le lemme, une matrice nilpotente non nulle n'est pas diagonalisable. En taille $n > 2$ la matrice $B$ telle que $[B]_{i,j} = [A]_{i,j}$ si $1 \leqslant i, j \leqslant 2$ et $[B]_{i,j} = 0$ sinon est elle aussi symétrique, nilpotente et non nulle et n'est donc n'est pas diagonalisable. 
\end{solution}   

\marginnote[-2cm]{
    $$
    B=
    \begin{pmatrix}
    1 & \mi & 0 & \cdots & 0 \\
    \mi & -1 & 0 & \cdots & 0 \\
    0 & 0 & 0 & \cdots & 0 \\
    \vdots & \vdots & \vdots & \ddots & \vdots \\
    0 & 0 & 0 & \cdots & 0
    \end{pmatrix}
    $$
}
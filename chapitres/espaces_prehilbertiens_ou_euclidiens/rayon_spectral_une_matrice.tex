\begin{defi}{Rayon spectral}
    Soient $n \geqslant 2$ et $M \in \M_n(\C)$. On définit le \emph{rayon spectral} de la matrice $M$ par
    $$\rho(M) \defeq \max \ens[\big]{ |\lambda | \tq \lambda \in \Sp_{\C}(M) }.$$
\end{defi}

\begin{exercice}
    \marginnote[0cm]{Source : \cite{exos_oraux} p. 182}
    Montrer que $M^p \xrightarrow[p \to +\infty]{} 0$ implique $\rho(M) < 1$. Montrer la réciproque dans les situations successives:
    \begin{itemize}
        \item $M$ est diagonalisable;
        \item $M$ ne possède qu'une unique valeur propre complexe;
        \item $M \in \M_n(\C)$ vérifie $\rho(M) < 1$.
    \end{itemize}
\end{exercice}                                    

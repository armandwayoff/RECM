\begin{exercice}
    \marginnote[0cm]{Exercice 17 Ch 13 \cite{acamanes}}
    Soient $I$ un intervalle non vide de $\R$ et $w \in \mathscr{C}(I, \Rpe)$. On suppose que, pour tout entier naturel $n$, $\int_I |x|^n w(x) \d x$ converge. On note $\mathscr{H} = \left\{ f \in \mathscr{C}(I, \R),\ \int_I f^2 w \text{ converge} \right\}$. Pour tout $(P, Q) \in \R[X]^2$, on pose $\langle P, Q \rangle = \int_I P(t) Q(t) w(t) \d t$.
    \begin{enumerate}
        \item Montrer que $\langle \cdot, \cdot \rangle$ est un produit scalaire sur $\R[X]$.
        \item Montrer qu'il existe une suite $(P_n)_{n \in \N}$ de polynômes tels que 
        \begin{itemize}
            \item pour tout $n \in \N, \deg P_n = n$,
            \item pour tout $(n, m) \in \N^2, n \not= m, \langle P_n, P_m \rangle = 0$,
            \item pour tout $n \in \N$, $P_n$ est unitaire.
        \end{itemize}
        Soit $n$ un entier naturel.
        \item Montrer que $\Vect(P_0, \dots, P_n) = \R_n[X]$.
        \item Montrer que $P_{n+1} \in \R_n[X]^\perp$.
        \item \textbf{Racines.} On note $(\alpha_i)_{i \leqslant i \leqslant k}$ les racines de $P_n$ qui appartiennent à $\mathring{I}$ et qui sont de multiplicité impaire. On pose $Q \defeq \prod\limits_{i=1}^k (X - \alpha_i)$.
        \begin{enumerate}
            \item Déterminer le degré de $Q$.
            \item Déterminer le signe de $P_n Q$ sur $I$.
            \item En déduire que $k = n$ et que $P_n$ a toutes ses racines réelles et simples dans $\mathring{I}$.
        \end{enumerate}
        \item \textbf{Relation de récurrence.}
        \begin{enumerate}
            \item Montrer que $(P_0, \dots, P_{n-1}, X P_{n-1})$ forme une base de $\R_n[X]$. \\
        On note $P_n = \sum\limits_{k=0}^{n-1} \alpha_k P_k + \alpha_n X P_{n-1}$.
            \item Montrer que, pour tout $j \in \llbracket 0, n - 3 \rrbracket, \alpha_j = 0$.
            \item En déduire qu'il existe trois suites réelles $(a_n), (b_n)$ et $(c_n)$ telles que 
            $$\forall n \in \N,\ P_{n+2} = (a_n X + b_n) P_{n+1} + c_n P_n.$$
        \end{enumerate}
    \end{enumerate}
\end{exercice}

\newpage

\begin{figure*}
    % \setlength sets the horizontal (column) spacing
    % \arraystretch sets the vertical (row) spacing
    \begingroup
    % \setlength{\tabcolsep}{10pt} % Default value: 6pt
    \renewcommand{\arraystretch}{1.5} % Default value: 1
    \begin{tabular}{|c|c|c|c|}
        \hline
        Nom & $I$ & $w(x)$ & Relation de récurrence\\
        \hline \hline
        \textsc{Legendre} & $[-1, 1]$ & $1$ & $(n+2) \Leg_{n+2} = (2n+3) \Leg_{n+1} - (n+1)\Leg_n$\\
        \hline
        \textsc{Tchebychev} & $]-1, 1[$ & $\frac{1}{\sqrt{1-x^2}}$ & $\Tcheby_{n+2} = 2X \Tcheby_{n+1} - \Tcheby_n$ \\
        \hline
        \textsc{Laguerre} & $\Rp$ & $\me^{-x}$ & $(n+2) \Lag_{n+2} = (-X+2n+3) \Lag_{n+1} - (n+1) \Lag_n$ \\
        \hline
        \textsc{Hermite} & $\R$ & $\me^{-x^2}$ & $\Hermite_{n+2} = 2X \Hermite_{n+1} - 2(n+1) \Hermite_n$\\
        \hline
    \end{tabular}
    \endgroup
    % The \begingroup ... \endgroup pair ensures the separation
    % parameters only affect this particular table, and not any
    % sebsequent ones in the document.
\end{figure*}

Ces familles de polynômes sont utilisées, via les formules de quadrature, pour calculer des valeurs approchées d'intégrales.


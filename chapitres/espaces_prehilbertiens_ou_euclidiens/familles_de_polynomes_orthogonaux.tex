Soient $I$ un intervalle non vide de $\R$ et $w \in \mathscr{C}(I, \Rpe)$. On suppose que, pour tout entier naturel $n$, $\int_I |x|^n w(x) \d x$ converge. On note
$$\mathscr{H} \defeq \ens[\Bigg]{ f \in \mathscr{C}(I, \R) \tq \int_I f^2 w \text{ converge}}.$$
Pour tout $(P, Q) \in \R[X]^2$, on pose 
$$\langle P, Q \rangle \defeq \int_I P(t) Q(t) w(t) \d t.$$

\subsection{Construction}

\begin{exercice}
    \marginnote[0cm]{Source : \cite{acamanes} \href{https://acamanes.github.io/psi/psi_doc/chap_e13.pdf}{(Exercice 17 Ch 13)}}
    \begin{enumerate}
        \item Montrer que $\langle \cdot, \cdot \rangle$ est un produit scalaire sur $\R[X]$.
        \item Montrer qu'il existe une suite $(P_n)_{n \in \N}$ de polynômes tels que 
        \begin{itemize}
            \item pour tout $n \in \N, \deg P_n = n$,
            \item pour tout $(n, m) \in \N^2, n \not= m, \langle P_n, P_m \rangle = 0$,
            \item pour tout $n \in \N$, $P_n$ est unitaire.
        \end{itemize}
        Soit $n$ un entier naturel.
        \item Montrer que $\Vect(P_0, \dots, P_n) = \R_n[X]$.
        \item Montrer que $P_{n+1} \in \R_n[X]^\perp$.
    \end{enumerate}
\end{exercice}

\begin{solution}
    \marginnote[0cm]{Source : Solution de \cite{acamanes}}
    \begin{enumerate}
        \item 
        \begin{itemize}
            \item[$\rhd$] Montrons que $\langle \cdot, \cdot \rangle$ est bien définie. D'une part, $PQw$ est une fonction continue sur $I$. \\
            De plus, $|PQ| \leqslant \frac{P^2 + Q^2}{2}$. Comme $w$ est à valeurs positives, alors
            $$|PQw| \leqslant \frac{1}{2} \Big[ P^2 w + Q^2 w\Big].$$
            Comme $P^2 w$ et $Q^2 w$ sont intégrables sur $I$, d'après les théorèmes de comparaison, $PQw$ est intégrable sur $I$.
            \item[$\rhd$] $\langle \cdot, \cdot \rangle$ est bien symétrique par commutativité du produit de deux polynômes. 
            \item[$\rhd$] est bilinéaire par linéarité des intégrales convergentes.
            \item[$\rhd$] Soit $P \in \R_n[X]$. Comme $w \geqslant 0$, par croissance de l'intégrale, 
            $$\int_I P(x)^2 w(x) \d x \geqslant 0.$$
            De plus, si $\displaystyle \int_I P^2 w = 0$, comme $P$ est une fonction polynomiale donc continue et $w$ est continue, d'après la positivité de l'intégrale,
            $$\forall t \in I, P(t)^2 w(t) = 0.$$
            De plus, $w$ est à valeurs strictement positives, donc
            $$\forall t \in I, P(t) = 0.$$
            Ainsi, $P$ possède une infinité de racines distinctes et $P$ est le polynôme nul.
        \end{itemize}
        Finalement, $\langle \cdot, \cdot \rangle$ est une forme bilinéaire définie positive, donc elle définit un produit scalaire. 
        \item La famille $(X^n)_{n \in \N}$ est la base canonique de $\R[X]$. En appliquant le procédé d'orthogonalisation de \textsc{Gram}-\textsc{Schmidt} à cette famille, on construit une famille de polynôme $(P_n)_{n \in \N}$ telle que, pour tout $n \in \N$, il existe $(\lambda_0, \dots, \lambda_{n-1}) \in \R^n$ tel que
        $$P_n \defeq X^n + \sum_{j=0}^{n-1} \lambda_j P_j.$$
        Ainsi, pour tout $n \in \N$, $\deg P_n = n$ et $P_n$ est unitaire. \\
        De plus, $(P_n)_{n \in \N}$ est une famille orthogonale et 
        $$\forall (m, n) \in \N^2, m \not= n, \langle P_n, P_m \rangle = 0.$$
        \item D'après le procédé de \textsc{Gram}-\textsc{Schmidt}, pour tout $n \in \N$,
        \begin{align*}
            \Vect(P_0, \dots, P_n) &= \Vect(1, X, \dots, X^n)\\
            &= \R_n[X].
        \end{align*}
        \item Soit $P \in \R_n[X]$. D'après la question précédente, il existe $(\mu_0, \dots, \mu_n) \in \R^{n+1}$ tel que
        $$P = \sum_{k=0}^n \mu_k P_k.$$
        Alors, par linéarité du produit scalaire,
        \begin{align*}
            \langle P_{n+1}, P \rangle &= \sum_{k=0}^n \mu_k \langle P_k, P_{n+1} \rangle \\
            &= 0.
        \end{align*}
        Ainsi, $P_{n+1} \in \R_n[X]^\perp$.
    \end{enumerate}
\end{solution}

\subsection{Racines}

\begin{exercice}
    On note $(\alpha_i)_{i \leqslant i \leqslant k}$ les racines de $P_n$ qui appartiennent à $\mathring{I}$ et qui sont de multiplicité impaire. On pose $Q \defeq \prod\limits_{i=1}^k (X - \alpha_i)$.
    \begin{enumerate}
        \item Majorer le degré de $Q$.
        \item Déterminer le signe de $P_n Q$ sur $I$.
        \item En déduire que $k = n$ et que $P_n$ a toutes ses racines réelles et simples dans $\mathring{I}$.
    \end{enumerate}
\end{exercice}

\begin{solution}
    Nous allons montrer que $P_n$ admet au moins $n$ changements de signe dans $I$. C'est pour cela que nous considérons les racines de multiplicité impaire, ce sont elles qui correspondent aux changements de signe. 
    \begin{enumerate}
        \item Comme $\deg P_n = n$, le polynôme $P_n$ possède au plus $n$ racines réelles distinctes. Ainsi, $\deg Q \leqslant n$.
        \item En notant $a_1, \dots, a_p$ les racines réelles distinctes de $P$, on écrit sous forme irréductible:
        $$P_n = \prod_{i=1}^p (X - a_i)^{r_i} \prod_{i=1}^s (X^2 + b_i X + c_i)^{\ell_i}.$$
        Ainsi, pour tout $i \in \llbracket 1,p \rrbracket$, il existe $d_i \in \R$ non nul tel que
        $$P_n(x) \isEquivTo{a_i} d_i (x - a_i)^{r_i}.$$
        Alors, pour tout $i \in \llbracket 1, k$, il existe $\tilde{d_i} \in \R$ non nul tel que
        $$P_n(x) Q(x) \isEquivTo{\alpha_i} \tilde{d_i} (x - \alpha_i)^{r_i+1}.$$
        Comme $r_i$ est impair, le polynôme $P_nQ$ ne change pas de signe au voisinage de $\alpha_i$. \\
        De plus, si $a_i$ est une racine de $P_n$ de multiplicité paire, alors $P_n$ ne change pas de signe au voisinage de $a_i$, d'après la \textcolor{red}{première} question. \\
        Finalement, $P_nQ$ garde un signe constant sur $I$.
        \item Supposons par l'absurde que $k < n$.  Alors, $Q \in \R_k[X] \subset \R_{n-1}[X]$ et, d'après la question $4.$, $P_n \in \R_{n-1}[X]^\perp$. Alors,
        \begin{align*}
            \langle P_n, Q \rangle &= 0 \\
            \int_I P_n(t) Q(t) w(t) \d t &= 0.
        \end{align*}
        D'après la question précédente, $P_n Q w$ est une fonction continue et de signe constant. Ainsi, d'après la positivité de l'intégrale, $P_n Q w = 0$ sur $I$. Comme $w$ est à valeurs strictement positives,
        $$\forall x \in I, P_n(x) Q(x) = 0.$$
        Ainsi, $P_n Q$ possède une infinité de racines soit $P_n Q = 0$. Or $Q \not= 0$, soit $P_n = 0$, ce qui est absurde. Finalement, $k=n$ \note \marginnote[0cm]{\note $P_n$ a au moins $n$ changements de signe sur $I$; par le théorème des valeurs intermédiaires, $P_n$ a au moins $n$ racines dans $I$ et comme $P_n$ est de degré $n$, il y a exactement $n$ racines simples.} et $\deg Q = n$, donc toutes les racines de $P_n$ sont simples et appartiennent à $\mathring{I}$. 
    \end{enumerate}
\end{solution}

\subsection{Relation de récurrence}

\begin{exercice}
    \begin{enumerate}
        \item Montrer que $(P_0, \dots, P_{n-1}, X P_{n-1})$ forme une base de $\R_n[X]$. \\
        On note $P_n = \sum\limits_{k=0}^{n-1} \alpha_k P_k + \alpha_n X P_{n-1}$.
        \item Montrer que, pour tout $j \in \llbracket 0, n - 3 \rrbracket, \alpha_j = 0$.
        \item En déduire qu'il existe trois suites réelles $(a_n), (b_n)$ et $(c_n)$ telles que 
        $$\forall n \in \N,\ P_{n+2} = (a_n X + b_n) P_{n+1} + c_n P_n.$$
    \end{enumerate}
\end{exercice}

\begin{solution}
    \begin{enumerate}
        \item $(P_0, \dots, P_{n-1}, X P_{n-1})$ est une famille de $n+1$ polynômes appartenant à $\R_n[X]$ et de degrés échelonnés. Ainsi, $(P_0, \dots, P_{n-1}, X P_{n-1})$ est une base de $\R_n[X]$.
        \item Soit $j \leqslant n-3$. D'après les définitions,
        \begin{align*}
            \langle X P_{n-1}, P_j \rangle &= \int_I t P_{n-1}(t) P_j(t) \d t \\
            &= \langle \underbrace{X P_j}_{\mathclap{\in \R_{n-2}[X]}}, P_{n-1} \rangle.
        \end{align*}
        Or, d'après la question $4.$, $P_{n-1} \in \R_{n-2}[X]^\perp$. \\
        Alors, comme $(P_0, \dots, P_n)$ est orthogonale,
        \begin{align*}
            0 &= \langle P_n, P_j \rangle \\
            &= \sum_{k=0}^{n-1} \alpha_k \langle P_k, P_j \rangle + \langle X P_{n-1}, P_j \rangle \\
            &= \alpha_j \norme{P_j}^2.
        \end{align*}
        Comme $P_j \not= 0$, alors $\alpha_j = 0$.
        \item D'après la question précédente, il existe
        $(\alpha_{n-2}, \alpha_{n-1}, \alpha_n) \in \R^3$ tel que 
        \begin{align*}
            P_n &= \alpha_{n-2} P_{n-2} + \alpha_{n-1} P_{n-1} + \alpha_n X P_{n-1} \\
            &= (\alpha_n X + \alpha_{n-1}) P_{n-1} + \alpha_{n-2} P_{n-2}.
        \end{align*}
        Ainsi, la suite $(P_n)_{n \in \N}$ satisfait une relation de récurrence d'ordre $2$. 
    \end{enumerate}
\end{solution}

\begin{exercice}    
    \marginnote[0cm]{Source : \cite{exos_oraux} p. 155}
    Soient $f \in \mathscr{C}^0(I, \R)$ et $(P_n)_{n \in \N}$ une famille de polynômes orthogonaux. Montrer que $\sum \langle f, P_n \rangle^2$ converge et prouver l'égalité de \textsc{Parseval}:
    $$\forall f \in \mathscr{C}^0(I, \R), \norme{f}^2 = \sum_{n=0}^{+ \infty} \langle f, P_n \rangle^2.$$
\end{exercice}

\begin{solution}
    
\end{solution}

\subsection{Équation différentielle}
\marginnote[0cm]{Source : Polynômes orthogonaux -- Fabien \textsc{Pucci}}

Soient $a$ et $b$ deux fonctions définies sur un intervalle $I \defeq ] \alpha, \beta [$ de $\R$ borné ou non, avec $\alpha > 0$ sur $I$. On se propose d'étudier les valeurs propres de l'opérateur différentiel:
\begin{align} \label{def_T}
    T(y) \defeq ay'' + by'.
\end{align}
On introduit pour cela une fonction résolvante $w$ à valeurs strictement positives sur $I$ qui permet d'écrire l'opérateur $T$ sous une forme dont nous verrons bientôt l'utilité
\begin{align*}
    T(y) &= \frac{1}{w} \big( a w y' \big)' \\
    &= \frac{1}{w} \big( a'w y' + a w' y' + a w y'' \big) \\
    T(y) &= ay'' + a'y' + \frac{aw'}{w}y'.
\end{align*}
En égalisant avec \ref{def_T}, la fonction $w$ doit être solution de l'équation différentielle linéaire du premier ordre:
$$a w' + (a' - b) w = 0,$$
donc de la forme $w = \e^A$, où $A$ est une primitive de $\frac{b-a'}{a}$. \\
On voit alors que pour le produit scalaire $\langle f, g \rangle \defeq \int_I f(x) g(x) w(x) \d x$, on a:
$$\langle T(f), g \rangle = \int_I (a w f')'(x) g(x) \d x = \big[ a w f' g \big]_I - \int_I a(x) w(x) f'(x) g'(x) \d x.$$
Si de plus la fonction $a w$ s'annule aux bornes de $I$ (ou tend vers $0$ en ses bornes si $I$ est infini), on a par intégration par parties
$$\langle T(f), g \rangle = - \int_I a(x) w(x) f'(x) g'(x) \d x = \langle f, T(f) \rangle,$$
autrement dit, l'opérateur $T$ est symétrique. \\
Bien sûr, il faudrait préciser un peu les hypothèses sur les fonctions $a$ et $b$ pour que tout cela ait un sens, et en particulier préciser sur quel domaine est défini le produit scalaire précédent. \\
Nous nous limiterons ici au cas où $a$ et $b$ sont des fonctions polynomiales de la forme
$$a(x) \defeq a_2 x^2 + a_1 x + a_0 \quad \text{et} \quad b(x) \defeq b_1 x + b_0.$$
Dans ce cas, l'opérateur $T$ est, pour tout entier $n \in \N$, une application linéaire de l'espace $\mathscr{P}_n$ des fonctions polynomiales de degré inférieur à $n$ dans lui-même et le produit scalaire $\langle \cdot, \cdot \rangle$ est défini sur cet espace si $\int_I |x|^k w(x) \d x$ converge pour tout $k \leqslant n$. \\
Sous cette hypothèse, $\mathscr{P}_n$ muni de ce produit scalaire est un espace vectoriel euclidien de dimension finie $n + 1$, et l'opérateur $T$ est un endomorphisme symétrique de cet espace. Il existe alors une base orthonormée de $\mathscr{P}_n$ constituée de vecteurs propres de $T$. \\
En particulier, il existe au moins un vecteur propre $P_n$ de degré $n$, qu'on peut choisir unitaire, et qui vérifie donc:
$$T(P_n) = \lambda_n P_n \Longleftrightarrow a P_n'' + b P_n' = \lambda_n P_n.$$
En considérant le terme de degré $n$ dans cette égalité, on obtient:
$$\lambda_n = n \big( a_2(n-1) + b_1 \big).$$
On sait que les sous-espaces propres d'un endomorphisme symétrique associés à des valeurs propres distinctes sont orthogonaux. Il en résulte qui si les valeurs propres $\lambda_n$ sont toutes distinctes, les polynômes $P_n$ seront deux à deux orthogonaux i.e.
$$\int_I P_n(x) P_n(x) w(x) \d x = 0 \text{ pour } m \not= n.$$
Ce sera le cas si l'équation:
$$a_2 n (n-1) + b_1 n = a_2 m (m-1) + b_1 m \iff (n-m) \big( a_2 (n+m-1) + b_1 \big) = 0$$
n'admet pas de solutions entières positives $(n, m)$ avec $n \not= m$.

\begin{remarque}
    On peut bien sûr ajouter à l'opérateur $T$ un terme $cy$, avec $c$ constant, sans changer la symétrie de $T$, ni les vecteurs propres. On en fait alors que translater les valeurs propres.
\end{remarque}

\subsection{Conclusion}

Pour chaque choix de $w$, on construit ainsi un produit scalaire appelé \emph{produit scalaire usuel avec poids $w$} sur $\mathscr{C}^0(I, \R)$. Pour chacun de ces choix, l'orthogonalisation de \textsc{Gram}-\textsc{Schmidt} appliqué à la base canonique $(1, X, X^2, \dots)$ fait apparaître des familles de polynômes orthogonaux. \\
C'est ainsi qu'il existe beaucoup de familles connues de polynômes orthogonaux dont l'introduction a été motivée par la résolution d'équations différentielles issues de la physique. Ces familles de polynômes sont aussi utilisées, via les formules de quadrature, pour calculer des valeurs approchées d'intégrales. \\
Le tableau ci-dessous rassemble quelques exemples de ces familles. 

\begin{figure*}[h!]
    % \setlength sets the horizontal (column) spacing
    % \arraystretch sets the vertical (row) spacing
    \begingroup
    % \setlength{\tabcolsep}{10pt} % Default value: 6pt
    \renewcommand{\arraystretch}{1.2} % Default value: 1
    \begin{tabular}{|c|c|c|c|c|}
        \hline
        Nom & $I$ & $w(x)$ & Relation de récurrence & Équation différentielle\\
        \hline \hline
        \textsc{Legendre} & $[-1, 1]$ & $1$ & $(n+2) \Leg_{n+2} = (2n+3) \Leg_{n+1} - (n+1)\Leg_n$ & $(1-x^2) y'' - 2xy' + n(n+1) y = 0$\\
        \hline
        \textsc{Tchebychev} & $]-1, 1[$ & $\frac{1}{\sqrt{1-x^2}}$ & $\Tcheby_{n+2} = 2X \Tcheby_{n+1} - \Tcheby_n$ & $(1-x^2)y'' - xy' + n^2y = 0$ \\
        \hline
        \textsc{Laguerre} & $\Rp$ & $\e^{-x}$ & $(n+2) \Lag_{n+2} = (-X+2n+3) \Lag_{n+1} - (n+1) \Lag_n$ & $xy'' + (1-x)y' + ny = 0$\\
        \hline
        \textsc{Hermite} & $\R$ & $\e^{-x^2}$ & $\Hermite_{n+2} = 2X \Hermite_{n+1} - 2(n+1) \Hermite_n$ & $y'' - 2xy' + 2ny = 0$\\
        \hline
    \end{tabular}
    \endgroup
    % The \begingroup ... \endgroup pair ensures the separation
    % parameters only affect this particular table, and not any
    % sebsequent ones in the document.
\end{figure*}

Voir le thème 23 de \cite{acamanes}.
\subsection{Décomposition d'\textsc{Iwasama}}
\begin{prop}
    Soient $n \in \Ne$ et $M \in \Gl_n(\R)$. Il existe un \textbf{unique} couple $(T, O)$ tel que:
    $$M = OT,$$
    avec $T$ triangulaire supérieure à coefficients diagonaux strictement positifs et $O$ matrice orthogonale. 
\end{prop}

Comme $M$ est inversible, c'est une matrice de changement de base. \\
Le produit et l'inversibilité sont stables dans $\mathscr{T}_n^+$.

\marginnote[-2cm]{
    \begin{kaobox}[frametitle=Le procédé de \textsc{Gram}-\textsc{Schmidt}]
        Soit $E$ un espace vectoriel préhilbertien et soit $\mathscr{F} = (e_i)_{i \in I}$ une famille libre dans $E$; il existe une unique famille orthonormale $\mathscr{G} = (\varepsilon_i)_{i \in I}$ telle que pour tout $k \in \llbracket 1, n \rrbracket$,
        \begin{itemize}
            \item $\Vect(e_1, \dots, e_k) = \Vect(\varepsilon_1, \dots, \varepsilon_k)$,
            \item $\langle e_k, \varepsilon_k \rangle > 0$.
        \end{itemize}
        La famille $\mathscr{G}$ est appelée l'\emph{orthonormalisée (de \textsc{Gram}-\textsc{Schmidt})} de $\mathscr{F}$.
    \end{kaobox}
}

\begin{preuve}
    \begin{itemize}
        \item \underline{Existence:} \\
        On note $\mathscr{B}$ la base canonique de $\R^n$. Soit $\mathscr{C} \defeq (C_1, \dots, C_n)$ la famille des vecteurs colonnes de $M$ exprimés dans $\mathscr{B}$. Comme $M$ est inversible, $\mathscr{C}$ forme une \textbf{base} de $\R^n$. Appliquons-lui le \textbf{procédé d'orthonormalisation de \textsc{Gram}-\textsc{Schmidt}}. \\
        Il existe une base orthonormée $\mathscr{B}_O = (O_1, \dots, O_n)$ telle que pour tout $i \in \llbracket 1, n \rrbracket$
        $$\mathrm{Vect}(C_1, \dots, C_i) = \mathrm{Vect}(O_1,\dots, O_i) \quad (1) \quad \text{et} \quad \langle C_i, O_i \rangle > 0 \quad (2).$$
        On écrit $M = P_{\mathscr{B} \to \mathscr{C}} = P_{\mathscr{B} \to \mathscr{B}_O} \times P_{\mathscr{B}_O \to \mathscr{C}} = OT$. \\
        Le caractère triangulaire de $T = P_{\mathscr{B}_O \to \mathscr{C}}$ vient de $(1)$ et la stricte positivité de sa diagonale de $(2)$.
        \item \underline{Unicité:} \textcolor{green}{à compléter} \\
        Soit $M = OT = O'T'$. $T$ est inversible. $(O')^{-1}O =T'\Inv{T}$. Le premier terme est une matrice orthogonale et le second triangulaire supérieure car ces deux ensembles sont des groupes multiplicatifs. $B = T'\Inv{T}$ est diagonale (schéma) de coeff...
    \end{itemize} 
\end{preuve}

\subsection{Décomposition polaire d'une matrice}
Lire chapitre 7 de \cite{matrices} page 77.
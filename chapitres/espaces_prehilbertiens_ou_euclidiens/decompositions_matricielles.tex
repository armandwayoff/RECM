Voir le thème 23 de \cite{acamanes}.
\subsection{Décomposition d'\textsc{Iwasama}}
\begin{tcolorbox}
    Soient $n \in \Ne$ et $M \in \Gl_n(\R)$. Il existe un \textbf{unique} couple $(T, O)$ tel que:
    $$M = OT,$$
    avec $T$ triangulaire supérieure à coefficients diagonaux strictement positifs et $O$ matrice orthogonale. 
\end{tcolorbox}

Comme $M$ est inversible, c'est une matrice de changement de base. \\
Faire un encadré sur le procédé de G.S. \\
Le produit et l'inversibilité sont stables dans $\mathscr{T}_n^+$.

\begin{itemize}
    \item \underline{Existence:} \\
    On note $\mathscr{B}$ la base canonique de $\R^n$. Soit $\mathscr{C} = (C_1, \dots, C_n)$ la famille des vecteurs colonnes de $M$ exprimés dans $\mathscr{B}$. Comme $M$ est inversible, $\mathscr{C}$ forme un \textbf{base} de $\R^n$. Appliquons-lui le \textbf{procédé d'orthonormalisation de \textsc{Gram}-\textsc{Schmidt}}. \\
    Il existe une base orthonormée $\mathscr{B}_o = (o_1, \dots, o_n)$ telle que pour tout $i \in \llbracket 1, n \rrbracket$
    $$\mathrm{Vect}(C_1, \dots, C_i) = \mathrm{Vect}(o_1,\dots, o_i) \quad (1) \quad \text{et} \quad \langle C_i, o_i \rangle > 0 \quad (2).$$
    On écrit $M = P_{\mathscr{B} \to \mathscr{C}} = P_{\mathscr{B} \to \mathscr{B}_o} \times P_{\mathscr{B}_o \to \mathscr{C}} = OT$. \\
    Le caractère triangulaire de $T = P_{\mathscr{B}_o \to \mathscr{C}}$ vient de $(1)$ et la stricte positivité de sa diagonale de $(2)$.
    \item \underline{Unicité:} \textcolor{green}{à compléter} \\
    Soit $M = OT = O'T'$. $T$ est inversible. $(O')^{-1}O =T'\Inv{T}$. Le premier terme est une matrice orthogonale et le second triangulaire supérieure car ces deux ensembles sont des groupes multiplicatifs. $B = T'\Inv{T}$ est diagonale (schéma) de coeff...
\end{itemize} 

\subsection{Décomposition polaire d'une matrice}
Lire chapitre 7 de \cite{matrices} page 77.
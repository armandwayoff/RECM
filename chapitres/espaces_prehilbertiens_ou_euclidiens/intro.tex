Lorsque $E$ est un espace euclidien, le procédé de \textsc{Gram}-\textsc{Schmidt} permet, à partir d'une base adaptée à un drapeau total de $E$, d'obtenir une base orthonormale adaptée à ce même drapeau. \\
Si l'on combine avec le théorème de trigonalisation utilisant les drapeaux, on constate que tout endomorphisme trigonalisable peut être trigonalisé dans une base orthonormale. 

\marginnote[-3cm]{
    \begin{kaobox}[frametitle=Drapeau]
        (Wiki) \\
        Un \emph{drapeau} d'un espace vectoriel $E$ de dimension finie est une suite finie strictement croissante de sous-espaces vectoriels de $E$, commençant par l'espace nul $\{0_E\}$ et se terminant par l'espace total $E$:
        $$\{0_E\} = E_0 \subsetneq E_1 \subsetneq \cdots \subsetneq E_k = E.$$
        Si $\dim(E)=n$ et si pour tout $i \in \llbracket 1, k \rrbracket$, $\dim(E_i)=i$, alors le drapeau est dit \emph{total}.
    \end{kaobox}
}

\chapter{Espaces préhilbertiens ou euclidiens}
\labch{espaces_prehilbertiens_ou_euclidiens}

Lorsque $E$ est un espace euclidien, le procédé de \textsc{Gram}-\textsc{Schmidt} permet, à partir d'une base adaptée à un drapeau total de $E$, d'obtenir une base orthonormale adaptée à ce même drapeau. \\
Si l'on combine avec le théorème de trigonalisation utilisant les drapeaux, on constate que tout endomorphisme trigonalisable peut être trigonalisé dans une base orthonormale. 

\marginnote[-3cm]{
    \begin{kaobox}[frametitle=Drapeau]
        (Wiki) \\
        Un \emph{drapeau} d'un espace vectoriel $E$ de dimension finie est une suite finie strictement croissante de sous-espaces vectoriels de $E$, commençant par l'espace nul $\{0_E\}$ et se terminant par l'espace total $E$:
        $$\{0_E\} = E_0 \subsetneq E_1 \subsetneq \cdots \subsetneq E_k = E.$$
        Si $\dim(E)=n$ et si pour tout $i \in \llbracket 1, k \rrbracket$, $\dim(E_i)=i$, alors le drapeau est dit \emph{total}.
    \end{kaobox}
}

\newpage

\section{Déterminant de \textsc{Gram}} \label{matrice_gram}
\begin{defi}{Matrice de \textsc{Gram}}
    Soient $E$ un espace euclidien et $(x_1, \dots, x_p) \in E^n$. On définit la matrice de \textsc{Gram} par
    $$\Gram(x_1, \dots, x_p) \defeq \big( \langle x_i, x_j \rangle \big)_{i,j \in \llbracket 1, p \rrbracket}.$$
\end{defi}
Le déterminant de \textsc{Gram} permet de calculer des volumes et de tester l'indépendance linéaire d'une famille de vecteurs.
\begin{prop}{}
    $$\Rg \big( \Gram(x_1, \dots, x_n) \big) = \Rg(x_1, \dots, x_n)$$
\end{prop}

\begin{preuve}
    \marginnote[0cm]{Source : \cite{maths-france} Planche 6 Maths SPE}
    Soient $F \defeq \Vect(x_1, \dots, x_n)$ et $m \defeq \dim F$. Soient $\mathscr{B} \defeq (e_i)_{1 \leqslant i \leqslant m}$ une base orthonormée de $F$ puis $M$ la matrice de la famille $(x_j)_{1 \leqslant j \leqslant n}$ dans la base $\mathscr{B}$. La matrice $M$ est une matrice rectangulaire de format $(m, n)$. \\
    Soit $(i, j) \in \llbracket 1, m \rrbracket \times \llbracket 1, n \rrbracket$. Puisque la base $\mathscr{B}$ est orthonormée, le coefficient $[ \Trsp{M} M ]_{i,j}$ est 
    $$\sum_{k=1}^m m_{k,i} m_{k,j} = \langle x_i, x_j \rangle,$$
    et on a donc
    $$\Gram(x_1, \dots, x_n) = \Trsp{M} M.$$
    Puisque $\Rg(x_1, \dots, x_n) = \Rg M$, il s'agit de vérifier que $\Rg(\Trsp{M} M) = \Rg M$. Pour cela, montrons que les matrices $M$ et $\Trsp{M} M$ ont le même noyau. \\
    Soit $X \in \M_{n,1}(\R)$.
    \begin{align*}
        X \in \Ker M &\implies MX = 0 \\
        &\implies \Trsp{M}MX = 0 \\
        &\implies X \in \Ker(\Trsp{M}M)
    \end{align*}
    et aussi
    \begin{align*}
        X \in \Ker \big( \Trsp{M}M \big) &\implies \Trsp{M}MX = 0 \\
        &\implies \Trsp{X}\Trsp{M}MX = 0 \\
        &\implies \Trsp{(MX)}MX = 0 \\
        &\implies \norme{MX}^2 = 0 \\
        &\implies MX = 0 \\
        &\implies X \in \Ker M.
    \end{align*}
    Finalement, $\Ker(\Trsp{M}M) = \Ker M$ et donc, d'après le théorème du rang,
    $$\Rg(x_1, \dots, x_n) = \Rg M = \Rg(\Trsp{M}M) = \Rg \big( \Gram(x_1, \dots, x_n) \big).$$
\end{preuve}

\begin{prop}{}
     La famille $(x_1, \dots, x_p)$ est liée si et seulement si $\det \Gram(x_1, \dots, x_p) = 0$ et est libre si et seulement si $\det \Gram(x_1, \dots, x_p) > 0$.
\end{prop}

\begin{preuve}
    \marginnote[0cm]{Source : \cite{maths-france} Planche 6 Maths SPE}
    D'après (??) et puisque $\Trsp{M}M \in \M_n(\K)$,
    \begin{align*}
        (x_1, \dots, x_n) \text{ liée } &\iff \Rg (x_1, \dots, x_n) < n \\
        &\iff \Rg \Gram(x_1, \dots, x_n) < n \\
        &\iff \Gram(x_1, \dots, x_n) \not \in \Gl_n(\R) \\
        &\iff \det \Gram(x_1, \dots, x_n) = 0.
    \end{align*}
    De plus, quand la famille $(x_1, \dots, x_n)$ libre, avec les notations de (??), on a $m=n$ et la matrice $M$ est une matrice carrée, inversible. On peut donc écrire
    $$\det \Gram(x_1, \dots, x_n) = \det \big( \Trsp{M} M \big) = \det(M)^2 > 0.$$
\end{preuve}

\begin{theo}{Distance à un sous-espace vectoriel} \labthm{distance_a_un_sous_espace_vectoriel}
    Soit $E$ un espace préhilbertien. Soit $F$ un sous-espace vectoriel de $E$ de dimension finie $p \in \Ne$ et soit $(e_1, \dots, e_p)$ une base de $F$. Alors pour tout $x \in E$,
    $$d(x, F)^2 = \frac{\Gram(e_1, \dots, e_p, x)}{\Gram(e_1, \dots, e_p)}.$$
\end{theo}

\begin{preuve}
    Soit $\pi_F(x)$ le projeté orthogonal de $x$ sur $F$. \\
    Alors $d(x, F)^2 = \norme{x - \pi_F(x)}^2$ et par \textsc{Pythagore},
    $$\norme{x}^2 = \norme{\pi_F(x)}^2 + \norme{x - \pi_F(x)}^2.$$
    De plus, 
    $$\forall k \in \llbracket 1, p \rrbracket,\ \langle x , e_k \rangle = \langle \pi_F(x) , e_k \rangle.$$
    On obtient alors
    \begin{align*}
        \Gram(e_1, \dots, e_p, x) &= 
        \begin{vmatrix}
          \begin{matrix}
            & & \\
            & \langle e_i, e_j \rangle & \\
            & &
          \end{matrix}
          & \rvline & \langle e_i, x \rangle \\
        \hline
          \langle x, e_j \rangle & \rvline &
          \begin{matrix}
          \norme{x}^2
          \end{matrix}
        \end{vmatrix} \\
        &=
        \begin{vmatrix}
          \begin{matrix}
            & & \\
            & \langle e_i, e_j \rangle & \\
            & &
          \end{matrix}
          & \rvline & \langle e_i, \pi_F(x) \rangle + 0 \\
        \hline
          \langle x, e_j \rangle & \rvline &
          \begin{matrix}
          \norme{\pi_F(x)}^2 + \norme{x - \pi_F(x)}^2
          \end{matrix}
        \end{vmatrix} 
    \end{align*}
    \marginnote[0cm]{
        \note On écrit la dernière colonne sous la forme
        $$
        \begin{pmatrix}
            \langle e_1, \pi_F(x) \rangle \\
            \vdots \\
            \langle e_p, \pi_F(x) \rangle \\
            \norme{\pi_F(x)}^2
        \end{pmatrix}
        + 
        \begin{pmatrix}
            0 \\
            \vdots \\
            0 \\
            \norme{x - \pi_F(x)}^2
        \end{pmatrix}.
        $$
    }
    Par linéarité du déterminant par rapport à la dernière colonne \note on obtient
    $$\Gram(e_1, \dots, e_p, x) = \Gram \big(e_1, \dots, e_p, \pi_F(x) \big) + \norme{x - \pi_F(x)}^2 \Gram(e_1, \dots, e_p).$$
    Comme $\pi_F(x) \in \Vect(e_1, \dots, e_p)$, le premier terme est nul et donc 
    $$d(x, F)^2 = \frac{\Gram(e_1, \dots, e_p, x)}{\Gram(e_1, \dots, e_p)}.$$
\end{preuve}

\begin{corol} \labthm{inegalite_gram}
    Soit $(x_1, \dots, x_n) \in E^n$. Alors,
    $$\Gram(x_1, \dots, x_n) \leqslant \prod_{i=1}^n \norme{x_i}^2$$
    avec égalité si et seulement si la famille $(x_1, \dots, x_n)$ est orthogonale. 
\end{corol}

Compléter avec \cite{objectif_agregation} p. 185.

\begin{preuve}
    \marginnote[0cm]{Source : \href{http://vonbuhren.free.fr/Agregation/Developpements/dev_determinant_gram.pdf}{Développement : Déterminant de \textsc{Gram} -- Jérôme \textsc{Von Buhren}, \textsf{vonbuhren.free.fr}}}
    \begin{itemize}
        \item Si la famille $(x_1, \dots, x_n)$ est liée, le résultat est immédiat.
        \item Raisonnons par récurrence sur $n \in \Ne$ sur la propriété
        \begin{center}
            $\mathscr{P}_n$: \say{ pour toute famille libre $(x_1, \dots, x_n)$ de $E$, on a $\Gram(x_1, \dots, x_n) \leqslant \prod\limits_{i=1}^n \norme{x_i}^2$ }.
        \end{center}
        \begin{itemize}
            \item[$\rhd$] Initialisation pour $n = 1$: soit $x_1 \in E$. Par définition, $\Gram(x_1) = \langle x_1, x_1 \rangle = \norme{x_1}^2$ donc $\mathscr{P}_1$ est vérifiée.
            \item[$\rhd$] Hérédité: supponsons $\mathscr{P}_n$ vraie. Soit $(x_1, \dots, x_n, x_{n+1})$ une famille libre de $E$. En notant $F \defeq \Vect(x_1, \dots, x_n)$, il existe $(f, \pi_F) \in F \times F^\perp$ tel que $x_{n+1} = f + \pi_F$. Par le théorème \vrefthm{distance_a_un_sous_espace_vectoriel} et par $\mathscr{P}_n$, 
            $$\Gram(x_1, \dots, x_{n+1}) = \Gram(x_1, \dots, x_n) \norme{\pi_F} \leqslant \norme{x_1}^2 \cdots \norme{x_n}^2 \norme{x_{n+1}}^2$$
            car par le théorème de \textsc{Pythagore}, $\norme{\pi_F} \leqslant \norme{x_{n+1}}$. On conclut que $\mathscr{P}_{n+1}$ est vraie, d'où le résultat. 
        \end{itemize}
    \end{itemize}
    \textcolor{red}{cas d'égalité}
\end{preuve}

\begin{prop}{}
    La matrice de \textsc{Gram} est symétrique positive.
\end{prop}

\marginnote[0cm]{
    \begin{defi}{Matrices symétriques positives}
        L'ensemble des \emph{matrices symétriques positives} est noté $\mathscr{S}_n^+(\R)$. Une matrice $M \in \mathscr{S}_n^+(\R)$ équivaut à chacune des propriétés suivantes:
        \begin{itemize}
            \item pour tout $X \in \M_n(\R), \Trsp{X} M X \geqslant 0$,
            \item $\Sp(M) \subset \Rp.$
        \end{itemize}
    \end{defi}
}

\begin{preuve}
    \begin{itemize}
        \item La matrice de \textsc{Gram} est symétrique par symétrie du produit scalaire.
        \item Montrons la positivité de $\Gram$. Soit $X = \Trsp{(\alpha_1 \cdots \alpha_n)} \in \M_{n,1}(\R)$. Montrons que $\Trsp{X} \Gram X \geqslant 0$. 
        \begin{align*}
            \Trsp{X} \Gram X &= \sum_{i=1}^{n} \sum_{j=1}^{n} \langle x_i, x_j \rangle \alpha_i \alpha_j \\ 
            &= \sum_{i=1}^{n} \sum_{j=1}^{n} \langle \alpha_i x_i, \alpha_j x_j \rangle \\
            &= \left\Vert \sum_{i=1}^{n}x_i \alpha_i \right\Vert^2 \geqslant 0.
        \end{align*}
    \end{itemize}
   
    Ce qui montre bien que $\Gram$ est symétrique positive.
\end{preuve}


\section{Positivité de la matrice de \textsc{Hilbert}}
\marginnote[0cm]{(Planche n°14. Espaces euclidiens de \cite{maths-france})}
Si on interprète le terme général de la matrice de \textsc{Hilbert} (cf. \nameref{matrice_hilbert}) comme 
$$\Hilb_{i,j} = \int_{0}^{1} x^{i+j-2} \d x$$
on peut y reconnaître une \nameref{matrice_gram} pour les fonctions puissances et le produit scalaire usuel sur l'espace des fonctions de $[0, 1]$ dans $\R$ de carré intégrable. Puisque les fonctions puissances sont linéairement indépendantes, les matrices de \textsc{Hilbert} sont donc définies positives.


\section{Décompositions matricielles}
Voir le thème 23 de \cite{acamanes}.
\subsection{Décomposition d'\textsc{Iwasama}}
\begin{tcolorbox}
    Soient $n \in \Ne$ et $M \in \Gl_n(\R)$. Il existe un \textbf{unique} couple $(T, O)$ tel que:
    $$M = OT,$$
    avec $T$ triangulaire supérieure à coefficients diagonaux strictement positifs et $O$ matrice orthogonale. 
\end{tcolorbox}

Comme $M$ est inversible, c'est une matrice de changement de base. \\
Faire un encadré sur le procédé de G.S. \\
Le produit et l'inversibilité sont stables dans $\mathscr{T}_n^+$.

\begin{itemize}
    \item \underline{Existence:} \\
    On note $\mathscr{B}$ la base canonique de $\R^n$. Soit $\mathscr{C} = (C_1, \dots, C_n)$ la famille des vecteurs colonnes de $M$ exprimés dans $\mathscr{B}$. Comme $M$ est inversible, $\mathscr{C}$ forme un \textbf{base} de $\R^n$. Appliquons-lui le \textbf{procédé d'orthonormalisation de \textsc{Gram}-\textsc{Schmidt}}. \\
    Il existe une base orthonormée $\mathscr{B}_o = (o_1, \dots, o_n)$ telle que pour tout $i \in \llbracket 1, n \rrbracket$
    $$\mathrm{Vect}(C_1, \dots, C_i) = \mathrm{Vect}(o_1,\dots, o_i) \quad (1) \quad \text{et} \quad \langle C_i, o_i \rangle > 0 \quad (2).$$
    On écrit $M = P_{\mathscr{B} \to \mathscr{C}} = P_{\mathscr{B} \to \mathscr{B}_o} \times P_{\mathscr{B}_o \to \mathscr{C}} = OT$. \\
    Le caractère triangulaire de $T = P_{\mathscr{B}_o \to \mathscr{C}}$ vient de $(1)$ et la stricte positivité de sa diagonale de $(2)$.
    \item \underline{Unicité:} \textcolor{green}{à compléter} \\
    Soit $M = OT = O'T'$. $T$ est inversible. $(O')^{-1}O =T'\Inv{T}$. Le premier terme est une matrice orthogonale et le second triangulaire supérieure car ces deux ensembles sont des groupes multiplicatifs. $B = T'\Inv{T}$ est diagonale (schéma) de coeff...
\end{itemize} 

\subsection{Décomposition polaire d'une matrice}
Lire chapitre 7 de \cite{matrices} page 77.

\section{Inégalité d'\textsc{Hadamard}}
%\begin{marginfigure}[-1cm]
%    \includegraphics[width=5cm]{images/jacques_hadamard.jpg}
%    \caption{Jacques \textsc{Hadamard}}
%\end{marginfigure}

\begin{theo}{Inégalité d'\textsc{Hadamard}}
    Soit $M \in \M_n(\C)$ et soient $X_1, \dots, X_n$ les vecteurs colonnes de $M$. Alors,
    $$|\mathrm{det}(M)| \leqslant \prod_{i=1}^{n} \Vert X_i \Vert$$
    avec égalité si et seulement si la famille $(X_i)$ est orthogonale.
\end{theo}

%\begin{marginfigure}[-2cm]
%    \begin{tikzpicture}[scale=0.7]
    \node[block] (gram) {Déterminant \\ de \textsc{Gram}};
    \node[block, right of=s1] (iwasama) {Décomposition \\ d'\textsc{Iwasama}};
    \node[block, below right of=gram] (hadamard) {Inégalité \\ d'\textsc{Hadamard}};
    \draw (gram) edge[bend right, above left] node {} (hadamard);
    \draw (iwasama) edge[bend left, above right] node {} (hadamard);
\end{tikzpicture}    
%\end{marginfigure}

Voyons deux démonstrations de ce résultat; une première en utilisant la décomposition d'\textsc{Iwasama} et une deuxième le \emph{corollaire 5.1}.

\begin{preuve}
    Si la matrice $M$ n'est pas inversible, le résultat est immédiat. Supposons que $M$ est inversible. D'après la décomposition d'\textsc{Iwasama}, il existe une matrice $O \in \mathscr{O}_n(\C)$ et $T \defeq (t_{i,j})$, triangulaire supérieure dont les coefficients diagonaux sont strictement positifs telles que $M = OT$. D'après la multiplicité du déterminant, 
    $$\det(M) = \underbrace{\det(O)}_{= \pm 1} \det(T)$$
    donc
    \begin{equation} \label{det}
        |\det(M)| = |\det(T)| = \prod_{i=1}^{n} |t_{i,i}|.
    \end{equation}
    Par construction, pour tout $i \in \llbracket 1, n \rrbracket, t_{i,i} = \langle X_i, O_i \rangle$ où $O_i$ est un vecteur unitaire. D'après l'inégalité de \textsc{Cauchy}-\textsc{Schwarz}, pour tout $i \in \llbracket 1, n \rrbracket$, 
    $$|t_{i,i}| = |\langle X_i, O_i \rangle | \leqslant \norme{X_i} \underbrace{\norme{O_i}}_{=1}.$$
    Ainsi d'après (\ref{det}), 
    $$|\det(M)| \leqslant \prod_{i=1}^n \norme{X_i}.$$
\end{preuve}

\begin{preuve}
    Si la matrice $M$ n'est pas inversible, le résultat est immédiat. Supposons que $M$ est inversible. On a $\Trsp{M} M = \Gram(X_1, \dots, X_n)\ (\star)$, la matrice de \textsc{Gram} de la famille des colonnes de la matrice $M$. D'après le (\ref{inegalite_gram}), 
    $$\Gram(X_1, \dots, X_n) \leqslant \prod_{i=1}^n \norme{X_i}^2.$$
    Ainsi, en composant $(\star)$ par le déterminant, 
    $$\det(\Trsp{M}M) = \det(M)^2 = \det \Gram(X_1, \dots, X_n) \leqslant \prod_{i=1}^n \norme{X_i}^2.$$
    En passant à la racine on obtient l'inégalité d'\textsc{Hadamard}.
\end{preuve}

L'inégalité d'\textsc{Hadamard} nous apprend en fait que le volume du parallélotope défini par les vecteurs colonnes est inférieur ou égal au produit des normes de ses vecteurs et il y a égalité si et seulement si la matrice est diagonale, ou encore que le parallélotope est rectangle. 

\marginnote[-2cm]{
    \begin{kaobox}[frametitle=Parallélotope]
        Soit $(x_1, \dots, x_n)$ une famille libre. Le parallélotope engendré par cette famille est défini par
        $$P \defeq \left\{ x = \sum_{i=1}^n t_i x_i,\ \forall i\ t_i \in [0,1]\right\}.$$
    \end{kaobox}
}

\begin{prop}
    Soient $\mathscr{S}_n ^{++} (\R)$ l'ensemble des matrices réelles d'ordre $n$ symétriques à valeurs propres strictement positives et $A = (a_{i,j}) \in \mathscr{S}_n ^{++} (\R)$. Alors,
    $$\det(A) \leqslant \prod_{i=1}^{n} a_{i,i}.$$
\end{prop}

\begin{exercice}    
exercice 4, TD 14 \cite{acamanes}
\begin{enumerate}
    \item Soit $(\gamma_1, \dots, \gamma_n) \in (\Re)^n$. Montrer que $B = (\gamma_i \gamma_j a_{i,j}) \in \mathscr{S}_n^{++}(\R)$. 
    \item Montrer que $\det(A)^{1/n} \leqslant \frac{\Tr(A)}{n}$. \\
    \emph{On pourra utiliser l'inégalité arithmético-géométrique}.
    
    \marginnote[-2cm]{
    	\begin{kaobox}[frametitle=Inégalité arithmético-géométrique]
            Soient $n \in \Ne$ et $x_1, \dots, x_n$ des réels positifs. Alors, 
            $$\frac{x_1 + \cdots + x_n}{n} \geqslant \sqrt[n]{x_1 \cdots x_n}.$$
            Il y a égalité si et seulement si tous les $x_i$ sont égaux.
        \end{kaobox}
        Pour d'autres inégalités, lire le chapitre 16, p.117 de la deuxième édition de Raisonnements divins (en fr)
    }
    \item Montrer que pour tout $i \in \llbracket 1, n \rrbracket, a_{i,i} > 0$. On pose $\gamma_i = \frac{1}{\sqrt{a_{i,i}}}$. En déduire l'inégalité d'\textsc{Hadamard}.
\end{enumerate}
\end{exercice}


\section{Familles de polynômes orthogonaux}
\begin{exercice}
    Exercice 17 Ch 13 \cite{acamanes}. \\
    Soient $I$ un intervalle non vide de $\R$ et $w \in \mathscr{C}(I, \Rpe)$. On suppose que, pour tout entier naturel $n$, $\int_I |x|^n w(x) \d x$ converge. On note $\mathscr{H} = \left\{ f \in \mathscr{C}(I, \R),\ \int_I f^2 w \text{ converge} \right\}$. Pour tout $(P, Q) \in \R[X]^2$, on pose $\langle P, Q \rangle = \int_I P(t) Q(t) w(t) \d t$.
    \begin{enumerate}
        \item Montrer que $\langle \cdot, \cdot \rangle$ est un produit scalaire sur $\R[X]$.
        \item Montrer qu'il existe une suite $(P_n)_{n \in \N}$ de polynômes tels que 
        \begin{itemize}
            \item pour tout $n \in \N, \deg P_n = n$,
            \item pour tout $(n, m) \in \N^2, n \not= m, \langle P_n, P_m \rangle = 0$,
            \item pour tout $n \in \N$, $P_n$ est unitaire.
        \end{itemize}
        Soit $n$ un entier naturel.
        \item Montrer que $\Vect(P_0, \dots, P_n) = \R_n[X]$.
        \item Montrer que $P_{n+1} \in \R_n[X]^\perp$.
        \item \textbf{Racines.} On note $(\alpha_i)_{i \leqslant i \leqslant k}$ les racines de $P_n$ qui appartiennent à $\mathring{I}$ et qui sont de multiplicité impaire. On pose $\Q = \prod\limits_{i=1}^k (X - \alpha_i)$.
        \begin{enumerate}
            \item Déterminer le degré de $Q$.
            \item Déterminer le signe de $P_n Q$ sur $I$.
            \item En déduire que $k = n$ et que $P_n$ a toutes ses racines réelles et simples dans $\mathring{I}$.
        \end{enumerate}
        \item \textbf{Relation de récurrence.}
        \begin{enumerate}
            \item Montrer que $(P_0, \dots, P_{n-1}, X P_{n-1})$ forme une base de $\R_n[X]$. \\
        On note $P_n = \sum\limits_{k=0}^{n-1} \alpha_k P_k + \alpha_n X P_{n-1}$.
            \item Montrer que, pour tout $j \in \llbracket 0, n - 3 \rrbracket, \alpha_j = 0$.
            \item En déduire qu'il existe trois suites réelles $(a_n), (b_n)$ et $(c_n)$ telles que 
            $$\forall n \in \N,\ P_{n+2} = (a_n X + b_n) P_{n+1} + c_n P_n.$$
        \end{enumerate}
    \end{enumerate}
\end{exercice}


\section{Polynômes orthogonaux associés à un poids}
\begin{defi}{}
    Soit $E = \mathscr{C}^0 \big( [-1, 1], \R \big)$ et $w$ continue et intégrable sur $]-1, 1[$, vérifiant pour tout $ t \in ]-1, 1[,\ w(t) > 0$. On définit: 
    $$\forall (f,g) \in E^2,\ \langle f, g \rangle = \int_{-1}^{1} f(t)g(t)w(t) \d t.$$
\end{defi}

\section{Polynômes de \textsc{Legendre} (bis)}
$$\forall n \in \N, \Leg_n(X) = \frac{1}{2^n n!} U_n^{(n)}(X)$$
où $U_n(X) = (X^2-1)^n$.

\begin{enumerate}
    \item Montrer que $(\Leg_n)_{n \in \N)}$ est une famille orthogonale. \\
    $-1$ et $+1$ sont des racines d'ordre $n$ de $U_n$ donc:
    $$\forall i \in \llbracket 1, n-1 \rrbracket,\ U_n^{(n)}(-1) = U_n^{(n)}(1) = 0 \quad (*)$$
    On calcule $\int_{-1}^{1} U_n^{(n)}(t) \times U_m^{(m)}(t)\ \mathrm{d}t$ en faisant une \textbf{intégration par parties} un intégrant $U_m^{(m)}$. D'après $(*)$, le crochet est nul. On répète l'opération $n+1$ fois. On obtient alors en facteur dans l'intégrande $U_n^{(2n+1)} = 0$ car $\mathrm{deg}(U_n) = 2n$.
\end{enumerate}

\section{Rayon spectral d'une matrice} \label{rayon_spectral}
\begin{tcolorbox}
    Soient $n \geqslant 2, M \in \M_n(\C)$. On définit son \emph{rayon spectral}:
    $$\rho(M) = \max \{ |\lambda |,\ \lambda \in \Sp_{\C}(M) \}.$$
\end{tcolorbox}

\section{Caractétisation des projecteurs orthogonaux}
\begin{prop}{}
    Soient $E$ un espace euclidien et $p$ un projecteur de $E$. Alors $p$ est un projecteur orthogonal si et seulement si, pour tout $x \in E$,
    $$\norme{p(x)} \leqslant \norme{x}.$$
\end{prop}

\begin{preuve}
    \begin{itemize}
        \item[$(\Rightarrow)$] Il existe $F$ un sev de $E$ tel que $p$ soit la projection sur $F$ parallèlement à $F^\perp$. On décompose tout vecteur de $E$ comme la somme unique d'un élément de $F$ et de $F^\perp$ puis on applique le théorème de \textsc{Pythagore}. 
        \item[$(\Leftarrow)$]
        \begin{itemize}
            \item \marginnote[0cm]{(Exos incontournables SUP)} Raisonner par l'absurde. Soit $F$ et $G$ tels que $p$ soit la projection sur $F$ parallèlement à $G$. Considérer un vecteur de $G^\perp \setminus F$ et aboutir à une contradiction.
            \item \marginnote[0cm]{(Ellipses p.176)} Soit $p$ une projection telle que pour tout $x \in E, \norme{p(x)} \leqslant \norme{x}$. \\
            Nous allons poser un vecteur dont la composante selon $\Im p$ sera variable.
            Soit $x \in \Ker p$ et $y \in \Im p$, pour tout $t \in \R$, 
            \begin{align*}
                \norme{ty} \leqslant \norme{x + ty} &\Leftrightarrow t^2 \norme{y}^2 \leqslant \norme{x+ty}^2 \\
                &\Leftrightarrow t^2 \norme{y}^2 \leqslant \norme{x}^2 + t^2 \norme{y}^2 + 2t \langle x, y \rangle \\
                &\Leftrightarrow \norme{x}^2 + 2t \langle x, y \rangle \geqslant 0 \\
                &\Rightarrow \langle x, y \rangle = 0 \text{ car cette inégalité est vraie pour tout } t \in \R
            \end{align*}
            Donc $\Ker p$ et $\Im p$ sont orthogonaux et $p$ est un projecteur orthogonal.
        \end{itemize}
    \end{itemize}
\end{preuve}

% \begin{marginfigure}[-4cm]
    % % \tdplotsetmaincoords{70}{200}

% \end{marginfigure}


\section{Famille obtusangle}
\begin{defi}
    Soit $E$ un espace euclidien de dimension $n \geqslant 2$. Soit $(x_1, \dots, x_p)$ une famille de vecteurs de $E$. On dit que cette famille est \emph{obtusangle} si et seulement si pour tout $i \not= j, \langle x_i, x_j \rangle < 0$. 
\end{defi}

\begin{exercice0}
    Soit $E$ un espace vectoriel de dimension $n$ et soit $(x_1, \dots, x_p)$ une famille obtusangle de $E$. Montrer que $p \leqslant n + 1$. 
\end{exercice0}


\section{Exercice 6.28 du ELLIPSES}
\begin{exercice}
    Soit $A \in \M_{1,n} (\R)$. Montrer que $B = A^\top A$ est diagonalisable et déterminer une matrice diagonale semblable à $B$.
\end{exercice}

\begin{solution}
    \begin{itemize}
        \item On montre facilement que $B$ est symétrique et comme cette matrice est réelle, elle est diagonalisable.
        \item Toutes les lignes de $B$ sont proportionnelles, et colinéaires à $A$; donc $B$ est de rang $1$ et $\dim E_0(B) = n-1$; la deuxième valeur propre de $B$ est: $\mathrm{Tr}(B) = \sum\limits_{k = 1}^{n} a_k^2$ en posant $A = (a_1, \dots, a_n)$. Enfin, $\Diag \left(\sum\limits_{k = 1}^{n} a_k^2, 0, \dots, 0 \right)$ est semblable à $B$.
    \end{itemize}
\end{solution}


\section{Exercice}
\begin{exercice}
    Issu de la RMS 132 3. \\
    Agrégation Interne de Mathématiques (première épreuve 2022) \\
    Vrai ou faux: \say{ les matrices carrées et symétriques à coefficients dans $\C$ sont diagonalisables. }
\end{exercice}

\begin{solution}
    \newline
    \begin{lemme}
        Une matrice nilpotente non nulle n'est pas diagonalisable.
    \end{lemme}
    
    \begin{preuve}
        Soit $A \in \M_n(\R)$ une matrice nilpotente diagonalisable. \\
        Alors il existe $P \in \Gl_n(\R)$ et $D$ une matrice diagonale telles que $A = PD\Inv{P}$. Or $A$ est nilpotente donc il existe $p \in \N$ tel que $A^p = P D^p \Inv{P} = 0$. Donc $D^p = 0$ soit $D = 0$ et $A = 0$. \\
        On peut aussi dire qu'une matrice ayant une unique valeur propre (comme c'est la cas des matrices nilpotentes) est diagonalisable si et seulement si elle est diagonale.
    \end{preuve}
    Cette affirmation est fausse. \\
    En effet en taille $2$, la matrice $A \defeq \begin{pmatrix}
        1 & \mi \\
        \mi & -1
    \end{pmatrix}$ est symétrique et non nulle; elle vérifie $A^2 = 0$. Or d'après le lemme, une matrice nilpotente non nulle n'est pas diagonalisable. En taille $n > 2$ la matrice $B$ telle que $[B]_{i,j} = [A]_{i,j}$ si $1 \leqslant i, j \leqslant 2$ et $[B]_{i,j} = 0$ sinon est elle aussi symétrique, nilpotente et non nulle et n'est donc n'est pas diagonalisable. 
\end{solution}   

\marginnote[-2cm]{
    $$
    B \defeq
    \begin{pmatrix}
    1 & \mi & 0 & \cdots & 0 \\
    \mi & -1 & 0 & \cdots & 0 \\
    0 & 0 & 0 & \cdots & 0 \\
    \vdots & \vdots & \vdots & \ddots & \vdots \\
    0 & 0 & 0 & \cdots & 0
    \end{pmatrix}
    $$
}

A rajouter:
\begin{itemize}
    \item Extremums d'une fonction sur les fonctions continues
    \item Racine carrée d'un endomorphisme autoadjoint positif
    \item Endomorphismes et matrices antisymétriques
\end{itemize}
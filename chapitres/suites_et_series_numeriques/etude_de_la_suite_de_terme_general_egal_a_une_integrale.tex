 \begin{exercice}
    \emph{Exercice 9. TD I}\\
    Soit $f$ une supposée continue et positive sur $[a, b]$. Étudier la suite de terme général $u_n = \left( \frac{1}{b-a} \int_{a}^{b} f(x)^n \d x \right)^{1/n}$.
 \end{exercice}

\begin{elem_sol}
    \begin{itemize}
        \item La démarche générale consiste à encadrer $u_n$. 
        \item \underline{Majoration:} $f$ est continue sur un segment donc est en particuler bornée par un réel positif $M$. On peut montrer que $u_n \leqslant M$ \emph{(ne pas oublier l'argument de la continuité lors du passage à l'intégrale)}.
        \item \underline{Minoration:} soit $\varepsilon > 0$, soit $x_0$ tel que $f(x_0) = M$. Comme $f$ est continue en $x_0$, il existe $[c, d] \subset [a, b]$ tel que $x_0 \in [c, d]$ et pour tout $x \in [c, d]$, $f(x) \geqslant M - \varepsilon$ \emph{(un dessin permet de bien comprendre la stratégie)}.\\
        On peut ensuite montrer que $u_n \geqslant \left(\frac{d-c}{b-a} \right)^{1/n}(M-\varepsilon) \xrightarrow[n \to + \infty]{} M-\varepsilon$.
        \item Finalement, $u_n \displaystyle \longrightarrow M = \max_{[ a, b ]} f = \Ninf{f}$.
    \end{itemize}
\end{elem_sol}

\begin{prop}
    Soient $(a_n)_{n \in \Ne}$ et $(b_n)_{n \in \Ne}$ deux suites à valeurs positives telles que $a_n \sim b_n$.\\
    Si $ \sum a_n$ diverge, alors $\sum\limits_{k=1}^{n} a_k \sim \sum\limits_{k=1}^{n} b_k$. \\
    Si $ \sum a_n$ converge, alors $\sum\limits_{k=n+1}^{+ \infty} a_k \sim \sum\limits_{k=n+1}^{+ \infty} b_k$. \\
    \emph{Il y a des résultats analogues si $a_n = o(b_n)$ ou si $a_n = \mathcal{O}(b_n)$.}
\end{prop}

\begin{marginfigure}[3cm]
    \centering
    \caption*{\centering Diagramme de la démonstration}
    \begin{tikzcd}
    a_n \sim b_n \arrow[r, Rightarrow, "1"] & a_n - b_n = o(a_n) \arrow[d, Rightarrow, "2"]\\
    A_n \sim B_n \arrow[r, Leftarrow, "3"] & A_n - B_n = o (A_n)
    \end{tikzcd}
\end{marginfigure}

\begin{preuve}
    On suppose que $\sum a_n$ diverge. On sait que $a_n \sim b_n$ est équivalent à $a_n -b_n = o(b_n)$, autrement dit, pour $\varepsilon > 0$, il existe un rang $n_0$ à partir duquel $|a_n -b_n| \leqslant \varepsilon a_n$. \\
    Pour tout $n \in \Ne$, on note $A_n \defeq \sum\limits_{k=1}^n a_k$ et $B_n \defeq \sum\limits_{k=1}^n b_k$. \\
    Soit $n > n_0$,
    \begin{align*}
        |A_n - B_n| &= \left|\sum_{k=1}^n (a_k - b_k) \right| \\
        \text{par l'inégalité triangulaire} &\leqslant \sum_{k=1}^n |a_n-b_n| \\
        &\leqslant \underbrace{\sum_{k=1}^{n_0-1} |a_k - b_k|}_{\defeq C} + \sum_{k=n_0}^n \underbrace{|a_k - b_k|}_{\leqslant \varepsilon a_k} \\
        &\leqslant C + \varepsilon \sum_{k=n_0}^n a_k \\
        \text{comme $(a_n)$ est à valeurs positives} &\leqslant C + \varepsilon A_n
    \end{align*}
    Comme $\sum a_n$ est divergente et à valeurs positives, $A_n \longrightarrow +\infty$ et donc à partir d'un certain rang $n_1$, $A_n \varepsilon \geqslant C$. Ainsi pour $n \geqslant \max \{ n_0, n_1 \}$, $|A_n - B_n| \leqslant 2 \varepsilon A_n$. \\ 
    Donc $A_n - B_n = o(A_n)$ ce qui équivaut à $A_n \sim B_n$.
\end{preuve}

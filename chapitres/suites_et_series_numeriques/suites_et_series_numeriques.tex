\chapter{Suites \& Séries numériques}
\labch{suites_et_series_numeriques}

\emph{\say{ Les séries divergentes sont des inventions du diable, et c'est une honte que l'on ose fonder sur elles la moindre démonstration. On peut en tirer  tout ce qu'on veut quand on les emploie et ce sont elles qui ont produit tant d'échecs et tant de paradoxes. }}
\begin{flushright}
\textsc{--- Niels Abel}, \emph{Œuvres, 1881}
\end{flushright}

\newpage

\section{Lemme de \textsc{Cesàro}} \label{lemme_cesaro}
\begin{lemme}
    Soit $(u_n)_{n \in \Ne}$ une suite réelle ou complexe convergeant vers $\ell$.
    Alors la suite de terme général $\frac{1}{n} \sum\limits_{k=1}^{n} u_k$ converge aussi vers $\ell$.
\end{lemme}

\begin{preuve}
    Soit $\varepsilon > 0$. Comme la suite $(u_n)$ converge vers $\ell$, il existe un rang $n_0 \in \Ne$ tel que pour tout $n \geqslant n_0,\ |u_n - \ell| \leqslant \varepsilon$. \\
    Soit $n \geqslant n_0$,
    \begin{align*}
        \left| \frac{1}{n} \sum_{k=1}^n u_k - \ell \right| &= \left| \frac{1}{n} \sum_{k=1}^n (u_k - \ell) \right| \\
        \text{par l'inégalité triangulaire} &\leqslant \frac{1}{n} \sum_{k=1}^n |u_k - \ell| \\
        &\leqslant \frac{1}{n} \Bigg( \underbrace{\sum_{k=1}^{n_0-1} |u_k - \ell|}_{\defeq K} + \sum_{k=n_0}^n \underbrace{|u_k - \ell|}_{\leqslant \varepsilon} \Bigg) \\
        &\leqslant \frac{K}{n} + \varepsilon
    \end{align*}
    Or $\lim\limits_{n \to \infty} \frac{K}{n} = 0$ donc il existe un rang $n_1 \in \Ne$ tel que pour tout $n \geqslant n_1, \left| \frac{K}{n} \right| \leqslant \varepsilon$. \\
    Ainsi pour tout $n \geqslant \max \{ n_0, n_1 \}$, 
    $$\left| \frac{1}{n} \sum_{k=1}^n u_k - \ell \right| \leqslant 2 \varepsilon.$$
    On en déduit que la suite $\Bigg( \frac{1}{n} \sum\limits_{k=1}^{n} u_k \Bigg)_{n \in \Ne}$ converge vers $\ell$.
\end{preuve}

\begin{remarque}
    \textcolor{red}{à réécrire}
    Attention, la réciproque du lemme de \textsc{Cesàro} est fausse. Une suite $(u_n)$ peut converger au sens de \textsc{Cesàro} i.e. $\Bigg( \frac{1}{n} \sum\limits_{k=1}^{n} u_k \Bigg)_{n \in \Ne}$ converge sans pour autant que la suite $(u_n)$ converge. Par exemple, $(u_n) \defeq \left((-1)^n\right)_n$.
\end{remarque}

\section{Constante d'\textsc{Euler}}
\begin{tcolorbox}
    La constante d'\textsc{Euler} $\gamma$ est définie par:
    $$\gamma = \lim_{n \to \infty} \left(\sum_{k=1}^{n} \frac{1}{k} - \ln(n) \right) \approx 0,577 215 664 \dots$$
\end{tcolorbox}

\begin{enumerate}
    \item Poser $v_n = H_n - \ln(n)$
    \item Montrer que $v_{n+1}-v_n = \mathcal{O}\left(\frac{1}{n^2}\right)$\\
    On peut aussi montrer...
    \item ... la décroissance de la suite $(v_n)$. \\
    Soit $n \in \N$. 
    $$v_n - v_{n+1} = \ln(n+1) - \ln(n) - \frac{1}{n+1}$$
    Deux méthodes:
    \begin{itemize}
        \item On transforme $\ln(n+1) - \ln(n)$ en intégrale:
        $$v_n - v_{n+1} = \int_{n}^{n+1} \underbrace{\left( \frac{1}{t} - \frac{1}{n+1} \right)}_{\geqslant 0}\ \d t > 0.$$
        \item D'après le \textbf{théorème des accroissements finis}, il existe $c \in ]n, n+1[$ tel que 
        $$\ln(n+1) - \ln(n) = \ln'(c)((n+1) - n) = \frac{1}{c}$$
        d'où l'on tire que 
        $$v_n - v_{n+1} = \frac{1}{c} - \frac{1}{n+1} > 0.$$
    \end{itemize}
    \item ... que $\boxed{\forall n \in \Ne,\ \ln(n+1) \leqslant H_n \leqslant \ln(n) + 1}$ grâce à l'\textbf{encadrement de l'intégrale} sur $[k, k+1]$ de la fonction $t \mapsto \frac{1}{t}$.
\end{enumerate}

\section{Séries de \textsc{Bertrand}}
\begin{defi}{Séries de \textsc{Bertrand}}
    Soient $\alpha$ et $\beta$ deux réels. On nomme \emph{série de \textsc{Bertrand}} la série de terme général $\displaystyle \frac{1}{n^\alpha \ln^\beta (n)}$ pour $n \geqslant 2$. 
\end{defi}

\begin{theo}{}
    La série de \textsc{Bertrand} converge si et seulement si \begin{cases} \alpha > 1 \\
    \text{ou} \\ \alpha = 1 \text{ et } \beta > 1 \end{cases}.
\end{theo}

\begin{preuve}
    Distinguons trois cas selon les valeurs prises par $\alpha$:
    \begin{enumerate}
        \item[$\rhd$] si $\alpha > 1$, soit $\gamma \in ]1, \alpha[$. Par croissances comparées,
        $$\displaystyle \frac{1}{t^{\alpha} \ln^{\beta} (t)} = o_{+ \infty} \left( \frac{1}{t^{\gamma}} \right).$$
        Or, d'après le théorème de \textsc{Riemann}, la fonction $t \mapsto \frac{1}{t^\gamma}$ est intégrable sur $[2, +\infty[$ car $\gamma > 1$. Ainsi, en appliquant les théorèmes de comparaison, $\int_2^{+ \infty} f$ converge.
        \item[$\rhd$] si $\alpha < 1$, soit $\gamma \in ]\alpha, 1[$.Par croissances comparées,
        $t^{\gamma} f(t) \xrightarrow[t \to + \infty]{} + \infty$
        donc à partir d'un certain rang, $f(t) \geqslant \frac{1}{t^{\gamma}} > 0$. Or, d'après le théorème de \textsc{Riemann}, la fonction $t \mapsto \frac{1}{t^\gamma}$ n'est intégrable pas sur $[2, +\infty[$ car $\gamma < 1$. Ainsi, en appliquant les théorèmes de comparaison (les intégrandes sont positives), $\int_2^{+ \infty} f$ diverge.
        \item[$\rhd$] si $\alpha = 1$, revenons aux intégrales partielles:
        $$\int_{2}^{X} \frac{1}{t \ln^{\beta} (t)} \d t = 
        \begin{cases}
            \left[ \frac{\ln ^{1-\beta} (t)}{1-\beta} \right]_2 ^X & \text{si } \beta \not = 1, \\
            \left[\ln (\ln(t)) \right]_2 ^X & \text{si } \beta = 1.
        \end{cases}
        $$
        On en déduit que l'intégrale de la fonction $t \mapsto \frac{1}{t \ln^{\beta} (t)}$ converge sur $[2, + \infty[$ si et seulement si $\beta > 1$.
    \end{enumerate}
\end{preuve}


\begin{exercice}
    \marginnote[0cm]{\cite{acamanes}}
    On note $h : x \mapsto \sum\limits_{n=2}^{+ \infty} \frac{1}{n^x \ln n}$.
    \begin{enumerate}
        \item Étudier la continuité de $h$ sur son domaine de définition.
        \item Étudier les limites de $h$ aux bornes de son intervalle de définition.
        \item Déterminer des équivalents de $h$ aux bornes de son intervalle de définition.
    \end{enumerate}
\end{exercice}

\begin{solution}
\begin{enumerate}
    On note $f_x : t \mapsto \frac{1}{t^x \ln t}$.
    \item D'après le théorème de \textsc{Bertrand} sur les séries numériques, l'ensemble de définition de $h$ est $\mathcal{D}_h \defeq ]1, + \infty[$. \\
    Soit $a > 1$. On se place sur $I \defeq [a, +\infty[$. Pour tout $x \in I$,
    $$\left| \frac{1}{n^x \ln n} \right| \leqslant \frac{1}{n^\alpha \ln n}$$
    comme $a > 1$, d'après le théorème de \textsc{Bertrand} sur les séries numériques, la série du terme majorant converge et donc par théorème de comparaison, la suite $(f_x)$ converge normalement sur tout segment de la forme de $I$. On en déduit que $h$ est continue sur $\mathcal{D}_h$. 
    \begin{itemize}
        \item En $+ \infty$: comme la série des $f_x$ converge uniformément sur $[2, + \infty[$ (on aurait pu choisir une valeur que $2$), d'après le théorème de la double limite
        $$\lim_{x \to + \infty} h(x) = \sum_{n=2}^{+ \infty} \left[\lim_{x \to +\infty} f_x(n) \right] = 0.$$
        \item En $1^+$: On montre que la fonction $f_x$ est décroissante et donc $h$ aussi. On note $\ell \defeq \lim\limits_{1^+} h$. D'après le théorème de la limite monotone, $\ell \in \R \cup \{ + \infty \}$. \\
        Supponson que $\ell \in \R$, alors
        $$h(x) = \sum_{n=2}^{+\infty} \frac{1}{n^x \ln n} \geqslant \sum_{n=2}^N \frac{1}{n^x \ln n}$$
        et en passant à la limite quand $x$ tend vers $1^+$ dans l'inégalité (ce qui est licite) on obtient
        $$\ell \geqslant \sum_{n=2}^N \frac{1}{n \ln n}.$$
        Nous aboutissons donc à une contradiction car la somme minorante diverge quand $N$ tend vers $+ \infty$. Finalement,
        $$\lim_{1^+} h = + \infty.$$
    \end{itemize}
    \item 
    \begin{itemize}
        \item En $+ \infty$: on intuite que la premier terme de la somme domine les autres. On a
        $$2^x \ln(2) h(x) = 1 + \sum_{n=3}^{+\infty} \left(\frac{2}{n}\right)^x \frac{\ln 2}{\ln n}.$$
        On peut montrer(...) que la somme converge normalement sur $[2, +\infty[$. On en déduit que 
        $$h(x) \isEquivTo{+\infty} \frac{1}{2^x \ln 2}.$$
        \item En $1^+$: un encadrement par la méthode des rectangles permet de trouver
        $$\int_{3}^{+\infty} f_x(t) \d t + \frac{1}{2^x \ln 2} \leqslant h(x) \leqslant \int_{2}^{+\infty} f_x(t) \d t + \frac{1}{2^x \ln 2}.$$
        On en déduit que 
        $$h(x) \isEquivTo{1^+} \int_2^{+\infty} f_x(t) \d t$$
        soit après calculs (...)
        $$h(x) \isEquivTo{1^+} - \ln(x-1).$$
    \end{itemize}
\end{enumerate}
\end{solution}

\section{Deux sommes} \label{deux_sommes}
$\displaystyle \sum_{n=1}^{+\infty} \frac{(-1)^n}{n}$ et $\displaystyle \sum_{n=0}^{+\infty} \frac{(-1)^n}{2n+1}$

\begin{itemize}
    \item Exprimer les termes généraux avec une intégrale. 
    \item (1)$= -\ln(2)$, (2)$= \frac{\pi}{4}$.
\end{itemize}

\section{Sommation des relations de comparaison} \label{sommation_relations_comparaison}
\begin{prop}{}
    Soient $(a_n)_{n \in \Ne}$ et $(b_n)_{n \in \Ne}$ deux suites à valeurs positives telles que $a_n \sim b_n$.
    \begin{itemize}
        \item Si $ \sum a_n$ diverge, alors $\sum\limits_{k=1}^{n} a_k \sim \sum\limits_{k=1}^{n} b_k$. 
        \item Si $ \sum a_n$ converge, alors $\sum\limits_{k=n+1}^{+ \infty} a_k \sim \sum\limits_{k=n+1}^{+ \infty} b_k$. 
    \end{itemize}
\end{prop}

\emph{Il y a des résultats analogues si $a_n = o(b_n)$ ou si $a_n = \mathcal{O}(b_n)$.}

\begin{marginfigure}[3cm]
    \centering
    \caption*{\centering Diagramme de la démonstration}
    \begin{tikzcd}
    a_n \sim b_n \arrow[r, Rightarrow, "1"] & a_n - b_n = o(a_n) \arrow[d, Rightarrow, "2"]\\
    A_n \sim B_n \arrow[r, Leftarrow, "3"] & A_n - B_n = o (A_n)
    \end{tikzcd}
\end{marginfigure}

\begin{preuve}
    On suppose que $\sum a_n$ diverge. On sait que $a_n \sim b_n$ est équivalent à $a_n -b_n = o(b_n)$, autrement dit, pour $\varepsilon > 0$, il existe un rang $n_0$ à partir duquel $|a_n -b_n| \leqslant \varepsilon a_n$. \\
    Pour tout $n \in \Ne$, on note $A_n \defeq \sum\limits_{k=1}^n a_k$ et $B_n \defeq \sum\limits_{k=1}^n b_k$. \\
    Soit $n > n_0$,
    \begin{align*}
        |A_n - B_n| &= \left|\sum_{k=1}^n (a_k - b_k) \right| \\
        \text{par l'inégalité triangulaire} &\leqslant \sum_{k=1}^n |a_n-b_n| \\
        &\leqslant \underbrace{\sum_{k=1}^{n_0-1} |a_k - b_k|}_{\defeq C} + \sum_{k=n_0}^n \underbrace{|a_k - b_k|}_{\leqslant \varepsilon a_k} \\
        &\leqslant C + \varepsilon \sum_{k=n_0}^n a_k \\
        \text{comme $(a_n)$ est à valeurs positives} &\leqslant C + \varepsilon A_n
    \end{align*}
    Comme $\sum a_n$ est divergente et à valeurs positives, $A_n \longrightarrow +\infty$ et donc à partir d'un certain rang $n_1$, $A_n \varepsilon \geqslant C$. Ainsi pour $n \geqslant \max \{ n_0, n_1 \}$, $|A_n - B_n| \leqslant 2 \varepsilon A_n$. \\ 
    Donc $A_n - B_n = o(A_n)$ ce qui équivaut à $A_n \sim B_n$.
\end{preuve}


\section{Formule de \textsc{Stirling}}
\label{preuve_stirling}
\begin{theo}{}
    $$n! \sim \left(\frac{n}{\me}\right)^n \sqrt{2 \pi n}$$
\end{theo}

\begin{elem_preuve}
    \begin{enumerate}
        \item Montrer que $\left( \frac{n!\me^n}{\sqrt{n}n^n} \right)_{n \in \Ne}$ converge vers $\ell \in \R$
        \item Montrer que $\ell = \sqrt{2 \pi}$ en utilisant l'\nameref{integrale_wallis}:
        $$\Wallis_n \defeq \int_{0}^{\pi/2}\sin^n(t) \d t$$
        \begin{enumerate}
            \item Montrer que $(\Wallis_n)_{n \in \N}$ est décroissante
            \item Exprimer $\Wallis_{n+2}$ en fonction de $\Wallis_n$ grâce à une IPP: $$(n+2)\Wallis_{n+2} = (n+1)\Wallis_n$$
            \item Exprimer $\Wallis_{2p}$ et $\Wallis_{2p+1}$ en fonction de $p$:\\
            $$\Wallis_{2p} = \frac{\binom{2p}{p}}{2^{2p}}\frac{\pi}{2} \text{ et } \Wallis_{2p+1} = \frac{2^{2p} (p!)^2}{(2p+1)!}$$
            \item Utiliser les points (a) et (b) pour montrer que $\frac{\Wallis_n}{\Wallis_{n+1}} \longrightarrow 1$
            \item Utiliser les points (c) et (d) pour montrer que $\left ( \frac{2^n n!}{n ((2n)!)^2} \right)^4 \longrightarrow \pi$
            \item Utiliser le point 1. pour déterminer $\ell$
        \end{enumerate}
    \end{enumerate}
\end{elem_preuve}

\begin{preuve}
    Calculons $\Wallis_{n+2}$ en effectuant une intégration par parties. On pose $u(t) \defeq - \cos(t)$ et $v(t) \defeq \sin^{n+1}(t)$, toutes deux de classe $\mathscr{C}^1$ sur $\left[0, \frac{\pi}{2} \right]$. 
    \begin{align*}
        \Wallis_{n+2} &= \underbrace{\left[ -\cos(t) \sin^{n+1}(t) \right]_0^{\pi/2}}_{=0} + (n+1) \int_0^{\pi/2} \cos^2 (t) \sin^n(t) \d t \\
        &= (n+1) \int_0^{\pi/2} (1 - \sin^2(t)) \sin^n(t) \d t \\
        &= (n+1) \Wallis_n - (n+1) \Wallis_{n+2} \\
        \text{soit } (n+2) \Wallis_{n+2} &= (n+1) \Wallis_n.
\end{align*}
Soit $p \in \N$. D'après la relation précédente, 
\begin{figure*}[h!]
\begin{multicols}{2}
\begin{align*}
    \Wallis_{2p} &= \frac{2p-1}{2p} \Wallis_{2p-2} \\
    &= \frac{2p-1}{2p} \times \frac{2p-3}{2p-2} \times \cdots \times \frac{1}{2} \times \underbrace{\Wallis_0}_{=\pi/2} \\
    &= \frac{\prod\limits_{k=1}^p (2k+1)}{\prod\limits_{k=1}^{p+1} (2k)} \frac{\pi}{2} \\
    &= \frac{\left[\prod\limits_{k=1}^p (2k+1) \right] \times \left[ \prod\limits_{k=1}^{p+1} (2k) \right]}{\left[\prod\limits_{k=1}^{p+1} (2k) \right]^2} \frac{\pi}{2} \\
    \Wallis_{2p} &= \frac{(2p)!}{2^{2p}(p!)^2} \frac{\pi}{2}.
\end{align*}
\begin{align*}
    \Wallis_{2p+1} &= \frac{2p}{2p+1} \Wallis_{2p-1} \\
    &= \frac{2p}{2p+1} \times \frac{2p-2}{2p-1} \times \cdots \times \frac{2}{3} \times \underbrace{\Wallis_1}_{=1} \\
    &= \frac{\prod\limits_{k=1}^p (2k)}{\prod\limits_{k=0}^p (2k+1)} \\
    &= \frac{\left[ \prod\limits_{k=1}^p (2k) \right]^2}{\left[ \prod\limits_{k=0}^p (2k+1) \right] \left[ \prod\limits_{k=1}^p (2k) \right]} \\
    \Wallis_{2p+1} &= \frac{2^{2p}(p!)^2}{(2p+1)!}.
\end{align*}
\end{multicols}
\end{figure*}
\end{preuve}


\section{Règle de \textsc{Raabe-Duhamel}}
\begin{theo}{Règle de \textsc{d'Alembert}}
    On suppose que $\lim\limits_{n \to + \infty} \frac{u_{n+1}}{u_n} = \ell$.
    \begin{itemize}
        \item Si $\ell < 1$, alors $\sum u_n$ converge.
        \item Si $\ell > 1$, alors $\sum u_n$ diverge.
    \end{itemize}
\end{theo}

Lorsque $\ell = 1$, on ne peut pas conclure. En effet, 
$$\sum \frac{1}{n} \text{ diverge et } \sum \frac{1}{n^2} \text{ converge}.$$

\begin{theo}{}
    Soit $\alpha$ un réel et $(u_n)_{n \in \N}$ une suite de réels strictement positifs. On suppose que
    $$\displaystyle \frac{u_{n+1}}{u_n} = 1 - \frac{\alpha}{n} + \mathcal{O} \left( \frac{1}{n^2} \right).$$ Alors $\sum u_n$ converge si et seulement si $\alpha > 1$. 
\end{theo}
\marginnote[-1cm]{Voir exercice 3.43. \cite{oraux_x_ens_3}}
\begin{preuve}
    \begin{enumerate}
        \item[($\Rightarrow$)] Montrer que si $u_n=\frac{K}{n^{\alpha}}$ avec $K>0$ et $\alpha > 1$ alors $(u_n)$ vérifie la relation.
        \item[($\Leftarrow$)] Soit $(v_n)$ une suite vérifiant les hypothèses. Montrer qu'il existe $K>0$ tel que $v_n \sim \frac{K}{n^{\alpha}}$ avec $\alpha > 0$. Pour cela, étudier la série de terme général $\ln (v_n)$.
    \end{enumerate}
\end{preuve}

On ne peut pas conclure si $\alpha=1$.

\section{Suites sous-additive}
\emph{Exercice 2. TD I \cite{acamanes}}\\

Une suite $(u_n)_{n\geqslant1}$ est dite sous-additive si pour tout couple d'entiers non nuls $(n, m)$, $u_{n+m} \leqslant u_n + u_m$.\\
1.a)\\ 
\indent $(\Leftarrow)$ Étudier les cas où $n=m$.\\
\indent $(\Rightarrow)$ Étudier la fonction $f:x \mapsto 1+x^{\alpha} - (1+x)^{\alpha}$.\\
1.b)\\
\indent Étudier le cas où $m=1$ et montrer que $w_n = n w_1$.\\
2)\\
\indent Montrer que la suite $(v_n)$ est décroissante.\\ 
4.b)\\
\indent Soit $(n, m) \in (\Ne)^2$. D'après le théorème de la division euclidienne, il existe un unique couple $(k, r) \in \Ne \times \llbracket0, m-1 \rrbracket$ tel que $n=km+r$.\\
\indent Utiliser successivment la définition d'une suite sous-additive et les résultats des questions 3) et 4.a).

\section{Étude de la suite de terme général \texorpdfstring{$u_n = \left( \frac{1}{b-a} \int_{a}^{b} f(x)^n \d x \right)^{1/n}$}{égal à une intégrale}}
 \begin{exercice}
    \marginnote[0cm]{\emph{Exercice 9. TD I}}
    Soit $f$ une supposée continue et positive sur $[a, b]$. Étudier la suite de terme général $u_n = \left( \frac{1}{b-a} \int_{a}^{b} f(x)^n \d x \right)^{1/n}$.
 \end{exercice}

\begin{elem_sol}
    \begin{itemize}
        \item La démarche générale consiste à encadrer $u_n$. 
        \item \underline{Majoration:} $f$ est continue sur un segment donc est en particuler bornée par un réel positif $M$. On peut montrer que $u_n \leqslant M$ \emph{(ne pas oublier l'argument de la continuité lors du passage à l'intégrale)}.
        \item \underline{Minoration:} soit $\varepsilon > 0$, soit $x_0$ tel que $f(x_0) = M$. Comme $f$ est continue en $x_0$, il existe $[c, d] \subset [a, b]$ tel que $x_0 \in [c, d]$ et pour tout $x \in [c, d]$, $f(x) \geqslant M - \varepsilon$ \emph{(un dessin permet de bien comprendre la stratégie)}.\\
        On peut ensuite montrer que $u_n \geqslant \left(\frac{d-c}{b-a} \right)^{1/n}(M-\varepsilon) \xrightarrow[n \to + \infty]{} M-\varepsilon$.
        \item Finalement, $u_n \displaystyle \longrightarrow M = \max_{[ a, b ]} f = \Ninf{f}$.
    \end{itemize}
\end{elem_sol}


\section{Transformation d'\textsc{Abel}} \label{transformation_abel}
\begin{tcolorbox}
    Si $(\varepsilon_n)$ et $(v_n)$ sont des suites à valeurs réelles, si $(u_n)$ et $(S_n)$ sont les suites de termes généraux:
    $$u_n = \varepsilon_n v_n \text{ et } S_n = \sum_{k=0}^{n} \varepsilon_k,$$
    si $(S_n)$ est bornée dans $\R$, si $(v_n)$ converge vers $0$ et si la série $\sum |v_{n+1} - v_n|$ converge, alors la série $\sum u_n$ converge.
\end{tcolorbox}

\begin{itemize}
    \item Savoir énoncer et démontrer la \textbf{formule de sommation par parties} (analogie avec l'intégration par parties) (sans raisonner par récurrence).
    $$\boxed{\forall n \in \N,\ \sum_{k=0}^{n} v_k \varepsilon_k = \sum_{k=0}^{n-1} (v_k -v_{k+1})S_k + v_nS_n}$$
    \item Critère d'\textsc{Abel} et test de \textsc{Dirichlet}.
\end{itemize}

\section{Convergence et calcul de  \texorpdfstring{$\sum \frac{r}{2^r}$}{de la série de terme général r/2^r}}
Montrer la convergence de la série de terme général $\frac{r}{2^r}$ et prouver que $\sum\limits_{r=1}^{+ \infty} \frac{r}{2^r} = 2$. Deux méthodes de résolution sont possibles bien que la première soit plus élégante. 

\begin{itemize}
    \item La série $\sum \frac{1}{2^r}$ est une série géométrique absolument convergente. Ainsi, d'après le résultat sur les produits de \textsc{Cauchy}, 
    $$4 = \left( \sum_{r=0}^{+ \infty} \frac{1}{2^r} \right)^2 = \sum_{r=0}^{+ \infty} \sum_{k=0}^{r} \frac{1}{2^k \cdot 2^{r-k}} = \sum_{r=0}^{+ \infty} \frac{r}{2^{r-1}}.$$
    \item On peut également étudier la fonction $g : x \to \sum\limits_{r=0}^{n} \frac{x^r}{2^r}$.
\end{itemize}

\section{Suites du type \texorpdfstring{$f(x_n) = n$}{f(x_n) = n}}
\begin{exercice}    
    \marginnote[0cm]{\emph{Exercice X 1 p.226 de \cite{exos_oraux}}}
    Soit $n \in \N$, montrer que l'équation $x \me^x = n$ admet une unique solution $x_n \in \R$, en donner un équivalent puis un équivalent de $y_n = x_n - \ln(n)$.  
\end{exercice}

\begin{exercice}
    Soit $n \in \Ne$, montrer que l'équation $x + \ln(x) = n$ admet une unique solution $x_n \in \R$, en donner un équivalent.
\end{exercice}
    


\section{Suites définies implicitement}

\begin{methode}
    \marginnote[0cm]{Texte de \cite{oraux_x_ens_3} p. 181.}
    \begin{enumerate}
        \item Montrer l'existence de $x_n$
        \item Démontrer la convergence de la suite $x_n$
        \item Déterminer un équivalent 
        \item Déterminer un développement asymptotique de $x_n$ \\
        Pour cela il faut commencer par déterminer dans la relation qui définit $x_n$ quels sont les termes prépondérants. 
    \end{enumerate}
\end{methode}

\section{Sommation par paquets}
\begin{exercice}
    \marginnote[0cm]{Source : \cite{exos_oraux} p. 340}
    Soit $\sum\limits_{n \geqslant 1} u_n$ une série à termes réels positifs, telle que $(u_n)_{n \geqslant 1}$ est décroissante. Montrer que les séries $\sum\limits_{n \geqslant 1} u_n$ et $\sum\limits_{n \geqslant 2} 2^n u_{2^n}$ sont de même nature. 
\end{exercice}

\section{Plan d'étude des suites \texorpdfstring{$u_{n+1} = f(u_n)$}{u_(n+1) = f(u_n)}}

La fonction $f$ doit être continue. \\

\begin{enumerate}
    \item Trouver un invervalle $I$ stable par $f$ ($f(I) \subset I$). Le segment $I$ doit contenir, au moins à partir d'un certain rang, tous les termes de la suite.
    \item Recherche des points fixes de la fonction $f$ dans $I$.
    \item ...
\end{enumerate}

\section{Techniques classiques}
\marginnote[0cm]{Arnaud Guyader}
\subsection{Développements asymptotiques}

Dans de nombreuses situations, on conclut sur la nature d'une série en se ramenant à une série plus simple. On a vu que pour les séries à termers positifs, il suffit de se ramener à un équivalent. Ceci n'est plus le cas avec des séries à termes de signe quelconque. Par ailleurs, un équivalent correspond à une approximation au premier ordre, laquelle ne permet pas forcément de conclure. \\
Dans ces deux situations, il suffit souvent d'écrire un développement asymptotique du terme général, c'est-à-dire d'être plus précis dans l'approximation. Celui-ci est généralement en $\frac{1}{n}$ ou en $\frac{1}{\sqrt{n}}$ et s'arrête au premier terme absolument convergent, en $\frac{1}{n^2}$ ou $\frac{1}{n^{3/2}}$.

Exemples: \\
$$\sum_{n \geqslant 2} \ln \left(1 + \frac{(-1)^n}{\sqrt{n}}\right) \text{ est divergente}.$$
$$\sum_{n \geqslant 1} \left(\frac{1}{\sqrt{n}} - \sqrt{n} \sin \frac{1}{n} \right) \text{ est absolument convergente}.$$

\subsection{Groupements de termes}

Considérons une série numérique $\sum u_n$ dont on veut déterminer la nature. On commence par s'assurer que le terme général $(u_n)$ tend vers zéro, sinon la série est trivialement divergente. Ceci fait, il faudrait montrer que la suite $(s_N)$ des sommes partielles est convergente, ce qui n'est pas toujours facile. En particulier, il est parfois plus simple de montrer qu'une sous-suite de $(s_N)$ converge, par exemple en effectuant des regroupements de termes, et de conclure ensuite. \\
Cadre typique d'application: on réussit à montrer que $(s_{2N})$ converge, disons vers $s$. Alors pour la sous-suite $(s_{2N+1})$, il suffit d'écrire:
$$s_{2N+1} = s_{2N} + u_{2N+1},$$
et si la série ne diverge par trivialement, on a
$$\lim_{N \to + \infty} s_{2N+1} = \lim_{N \to + \infty} s_{2N} = s,$$
c'est-à-dire que 
$$\lim_{N \to \infty} s_N = s,$$
la série converge. 


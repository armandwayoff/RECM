\chapter{Suites \& Séries numériques}
\labch{suites_et_series_numeriques}

\emph{\say{ Les séries divergentes sont des inventions du diable, et c'est une honte que l'on ose fonder sur elles la moindre démonstration. On peut en tirer  tout ce qu'on veut quand on les emploie et ce sont elles qui ont produit tant d'échecs et tant de paradoxes. }}
\begin{flushright}
\textsc{--- Niels Abel}, \emph{Œuvres, 1881}
\end{flushright}

\newpage

\section{Lemme de \textsc{Cesàro}} \label{lemme_cesaro}
\begin{lemme}
    Soit $(u_n)_{n \in \Ne}$ une suite réelle ou complexe convergeant vers $\ell$.
    Alors la suite de terme général $\frac{1}{n} \sum\limits_{k=1}^{n} u_k$ converge aussi vers $\ell$.
\end{lemme}

\begin{preuve}
    Soit $\varepsilon > 0$. Comme la suite $(u_n)$ converge vers $\ell$, il existe un rang $n_0 \in \Ne$ tel que pour tout $n \geqslant n_0,\ |u_n - \ell| \leqslant \varepsilon$. \\
    Soit $n \geqslant n_0$,
    \begin{align*}
        \left| \frac{1}{n} \sum_{k=1}^n u_k - \ell \right| &= \left| \frac{1}{n} \sum_{k=1}^n (u_k - \ell) \right| \\
        \text{par l'inégalité triangulaire} &\leqslant \frac{1}{n} \sum_{k=1}^n |u_k - \ell| \\
        &\leqslant \frac{1}{n} \Bigg( \underbrace{\sum_{k=1}^{n_0-1} |u_k - \ell|}_{\defeq K} + \sum_{k=n_0}^n \underbrace{|u_k - \ell|}_{\leqslant \varepsilon} \Bigg) \\
        &\leqslant \frac{K}{n} + \varepsilon
    \end{align*}
    Or $\lim\limits_{n \to \infty} \frac{K}{n} = 0$ donc il existe un rang $n_1 \in \Ne$ tel que pour tout $n \geqslant n_1, \left| \frac{K}{n} \right| \leqslant \varepsilon$. \\
    Ainsi pour tout $n \geqslant \max \{ n_0, n_1 \}$, 
    $$\left| \frac{1}{n} \sum_{k=1}^n u_k - \ell \right| \leqslant 2 \varepsilon.$$
    On en déduit que la suite $\Bigg( \frac{1}{n} \sum\limits_{k=1}^{n} u_k \Bigg)_{n \in \Ne}$ converge vers $\ell$.
\end{preuve}

\begin{remarque}
    \textcolor{red}{à réécrire}
    Attention, la réciproque du lemme de \textsc{Cesàro} est fausse. Une suite $(u_n)$ peut converger au sens de \textsc{Cesàro} i.e. $\Bigg( \frac{1}{n} \sum\limits_{k=1}^{n} u_k \Bigg)_{n \in \Ne}$ converge sans pour autant que la suite $(u_n)$ converge. Par exemple, $(u_n) \defeq \left((-1)^n\right)_n$.
\end{remarque}

\section{Constante d'\textsc{Euler}}
\begin{tcolorbox}
    La constante d'\textsc{Euler} $\gamma$ est définie par:
    $$\gamma = \lim_{n \to \infty} \left(\sum_{k=1}^{n} \frac{1}{k} - \ln(n) \right) \approx 0,577 215 664 \dots$$
\end{tcolorbox}

\begin{enumerate}
    \item Poser $v_n = H_n - \ln(n)$
    \item Montrer que $v_{n+1}-v_n = \mathcal{O}\left(\frac{1}{n^2}\right)$\\
    On peut aussi montrer...
    \item ... la décroissance de la suite $(v_n)$. \\
    Soit $n \in \N$. 
    $$v_n - v_{n+1} = \ln(n+1) - \ln(n) - \frac{1}{n+1}$$
    Deux méthodes:
    \begin{itemize}
        \item On transforme $\ln(n+1) - \ln(n)$ en intégrale:
        $$v_n - v_{n+1} = \int_{n}^{n+1} \underbrace{\left( \frac{1}{t} - \frac{1}{n+1} \right)}_{\geqslant 0}\ \d t > 0.$$
        \item D'après le \textbf{théorème des accroissements finis}, il existe $c \in ]n, n+1[$ tel que 
        $$\ln(n+1) - \ln(n) = \ln'(c)((n+1) - n) = \frac{1}{c}$$
        d'où l'on tire que 
        $$v_n - v_{n+1} = \frac{1}{c} - \frac{1}{n+1} > 0.$$
    \end{itemize}
    \item ... que $\boxed{\forall n \in \Ne,\ \ln(n+1) \leqslant H_n \leqslant \ln(n) + 1}$ grâce à l'\textbf{encadrement de l'intégrale} sur $[k, k+1]$ de la fonction $t \mapsto \frac{1}{t}$.
\end{enumerate}

\section{Séries de \textsc{Bertrand}}
\begin{defi}{Séries de \textsc{Bertrand}}
    Soient $\alpha$ et $\beta$ deux réels. On nomme \emph{série de \textsc{Bertrand}} la série de terme général $\displaystyle \frac{1}{n^\alpha \ln^\beta (n)}$ pour $n \geqslant 2$. 
\end{defi}

\begin{theo}{}
    La série de \textsc{Bertrand} converge si et seulement si \begin{cases} \alpha > 1 \\
    \text{ou} \\ \alpha = 1 \text{ et } \beta > 1 \end{cases}.
\end{theo}

\begin{preuve}
    Distinguons trois cas selon les valeurs prises par $\alpha$:
    \begin{enumerate}
        \item[$\rhd$] si $\alpha > 1$, soit $\gamma \in ]1, \alpha[$. Par croissances comparées,
        $$\displaystyle \frac{1}{t^{\alpha} \ln^{\beta} (t)} = o_{+ \infty} \left( \frac{1}{t^{\gamma}} \right).$$
        Or, d'après le théorème de \textsc{Riemann}, la fonction $t \mapsto \frac{1}{t^\gamma}$ est intégrable sur $[2, +\infty[$ car $\gamma > 1$. Ainsi, en appliquant les théorèmes de comparaison, $\int_2^{+ \infty} f$ converge.
        \item[$\rhd$] si $\alpha < 1$, soit $\gamma \in ]\alpha, 1[$.Par croissances comparées,
        $t^{\gamma} f(t) \xrightarrow[t \to + \infty]{} + \infty$
        donc à partir d'un certain rang, $f(t) \geqslant \frac{1}{t^{\gamma}} > 0$. Or, d'après le théorème de \textsc{Riemann}, la fonction $t \mapsto \frac{1}{t^\gamma}$ n'est intégrable pas sur $[2, +\infty[$ car $\gamma < 1$. Ainsi, en appliquant les théorèmes de comparaison (les intégrandes sont positives), $\int_2^{+ \infty} f$ diverge.
        \item[$\rhd$] si $\alpha = 1$, revenons aux intégrales partielles:
        $$\int_{2}^{X} \frac{1}{t \ln^{\beta} (t)} \d t = 
        \begin{cases}
            \left[ \frac{\ln ^{1-\beta} (t)}{1-\beta} \right]_2 ^X & \text{si } \beta \not = 1, \\
            \left[\ln (\ln(t)) \right]_2 ^X & \text{si } \beta = 1.
        \end{cases}
        $$
        On en déduit que l'intégrale de la fonction $t \mapsto \frac{1}{t \ln^{\beta} (t)}$ converge sur $[2, + \infty[$ si et seulement si $\beta > 1$.
    \end{enumerate}
\end{preuve}


\begin{exercice}
    \marginnote[0cm]{\cite{acamanes}}
    On note $h : x \mapsto \sum\limits_{n=2}^{+ \infty} \frac{1}{n^x \ln n}$.
    \begin{enumerate}
        \item Étudier la continuité de $h$ sur son domaine de définition.
        \item Étudier les limites de $h$ aux bornes de son intervalle de définition.
        \item Déterminer des équivalents de $h$ aux bornes de son intervalle de définition.
    \end{enumerate}
\end{exercice}

\begin{solution}
\begin{enumerate}
    On note $f_x : t \mapsto \frac{1}{t^x \ln t}$.
    \item D'après le théorème de \textsc{Bertrand} sur les séries numériques, l'ensemble de définition de $h$ est $\mathcal{D}_h \defeq ]1, + \infty[$. \\
    Soit $a > 1$. On se place sur $I \defeq [a, +\infty[$. Pour tout $x \in I$,
    $$\left| \frac{1}{n^x \ln n} \right| \leqslant \frac{1}{n^\alpha \ln n}$$
    comme $a > 1$, d'après le théorème de \textsc{Bertrand} sur les séries numériques, la série du terme majorant converge et donc par théorème de comparaison, la suite $(f_x)$ converge normalement sur tout segment de la forme de $I$. On en déduit que $h$ est continue sur $\mathcal{D}_h$. 
    \begin{itemize}
        \item En $+ \infty$: comme la série des $f_x$ converge uniformément sur $[2, + \infty[$ (on aurait pu choisir une valeur que $2$), d'après le théorème de la double limite
        $$\lim_{x \to + \infty} h(x) = \sum_{n=2}^{+ \infty} \left[\lim_{x \to +\infty} f_x(n) \right] = 0.$$
        \item En $1^+$: On montre que la fonction $f_x$ est décroissante et donc $h$ aussi. On note $\ell \defeq \lim\limits_{1^+} h$. D'après le théorème de la limite monotone, $\ell \in \R \cup \{ + \infty \}$. \\
        Supponson que $\ell \in \R$, alors
        $$h(x) = \sum_{n=2}^{+\infty} \frac{1}{n^x \ln n} \geqslant \sum_{n=2}^N \frac{1}{n^x \ln n}$$
        et en passant à la limite quand $x$ tend vers $1^+$ dans l'inégalité (ce qui est licite) on obtient
        $$\ell \geqslant \sum_{n=2}^N \frac{1}{n \ln n}.$$
        Nous aboutissons donc à une contradiction car la somme minorante diverge quand $N$ tend vers $+ \infty$. Finalement,
        $$\lim_{1^+} h = + \infty.$$
    \end{itemize}
    \item 
    \begin{itemize}
        \item En $+ \infty$: on intuite que la premier terme de la somme domine les autres. On a
        $$2^x \ln(2) h(x) = 1 + \sum_{n=3}^{+\infty} \left(\frac{2}{n}\right)^x \frac{\ln 2}{\ln n}.$$
        On peut montrer(...) que la somme converge normalement sur $[2, +\infty[$. On en déduit que 
        $$h(x) \isEquivTo{+\infty} \frac{1}{2^x \ln 2}.$$
        \item En $1^+$: un encadrement par la méthode des rectangles permet de trouver
        $$\int_{3}^{+\infty} f_x(t) \d t + \frac{1}{2^x \ln 2} \leqslant h(x) \leqslant \int_{2}^{+\infty} f_x(t) \d t + \frac{1}{2^x \ln 2}.$$
        On en déduit que 
        $$h(x) \isEquivTo{1^+} \int_2^{+\infty} f_x(t) \d t$$
        soit après calculs (...)
        $$h(x) \isEquivTo{1^+} - \ln(x-1).$$
    \end{itemize}
\end{enumerate}
\end{solution}

\section{Deux sommes} \label{deux_sommes}
\begin{exercice}
    Calculer $\displaystyle \sum_{n=1}^{+\infty} \frac{(-1)^n}{n}$ et $\displaystyle \sum_{n=0}^{+\infty} \frac{(-1)^n}{2n+1}$.
\end{exercice}

\begin{elem_sol}
    \begin{itemize}
        \item Exprimer les termes généraux avec une intégrale. 
        \item (1)$= -\ln(2)$, (2)$= \frac{\pi}{4}$.
    \end{itemize}
\end{elem_sol}


\section{Sommation des relations de comparaison} \label{sommation_relations_comparaison}
\begin{tcolorbox}
    Soient $(a_n)_{n \in \Ne}$ et $(b_n)_{n \in \Ne}$ deux suites à valeurs positives telles que $a_n \sim b_n$.\\
    Si $ \sum a_k$ diverge, alors $\sum\limits_{k=1}^{n} a_k \sim \sum\limits_{k=1}^{n} b_k$. \\
    Si $ \sum a_k$ converge, alors $\sum\limits_{k=n+1}^{+ \infty} a_k \sim \sum\limits_{k=n+1}^{+ \infty} b_k$. \\
    \emph{Il y a des résultats analogues si $u_n = o(v_n)$ ou si $u_n = \mathcal{O}(v_n)$.}
\end{tcolorbox}

\begin{itemize}
    \item Revenir à la définition de l'équivalence avec le $o$: $a_n \sim b_n \Longleftrightarrow a_n -b_n = o(b_n)$.
\end{itemize}

\section{Preuve de la formule de \textsc{Stirling} par \textsc{Wallis}}
\label{preuve_stirling}
\begin{box_titre}{Formule de \textsc{Stirling}}
    $$n! \sim \left(\frac{n}{\me}\right)^n \sqrt{2 \pi n}$$
\end{box_titre}

\begin{preuve}
    \begin{enumerate}
        \item Montrer que $\left( \frac{n!\me^n}{\sqrt{n}n^n} \right)_{n \in \Ne}$ converge vers $\ell \in \R$
        \item Montrer que $\ell = \sqrt{2 \pi}$ en utilisant l'\nameref{integrale_wallis}:
        $$\boxed{\Wallis_n = \int_{0}^{\pi/2}\sin^n(t) \, \d t}$$
        \begin{enumerate}
            \item Montrer que $(\Wallis_n)_{n \in \N}$ est décroissante
            \item Exprimer $\Wallis_{n+2}$ en fonction de $\Wallis_n$ grâce à une IPP: $$(n+2)\Wallis_{n+2} = (n+1)\Wallis_n$$
            \item Exprimer $\Wallis_{2p}$ et $\Wallis_{2p+1}$ en fonction de $p$:\\
            $$\Wallis_{2p} = \frac{\binom{2p}{p}}{2^{2p}}\frac{\pi}{2} \text{ et } \Wallis_{2p+1} = \frac{2^{2p} (p!)^2}{(2p+1)!}$$
            \item Utiliser les points (a) et (b) pour montrer que $\frac{\Wallis_n}{\Wallis_{n+1}} \longrightarrow 1$
            \item Utiliser les points (c) et (d) pour montrer que $\left ( \frac{2^n n!}{n ((2n)!)^2} \right)^4 \longrightarrow \pi$
            \item Utiliser le point 1. pour déterminer $\ell$
        \end{enumerate}
    \end{enumerate}
\end{preuve}


\section{Règle de \textsc{Raabe-Duhamel}}
\begin{theo}
    Soit $\alpha$ un réel et $(u_n)_{n \in \N}$ une suite de réels strictement positifs. On suppose que
    $$\displaystyle \frac{u_{n+1}}{u_n} = 1 - \frac{\alpha}{n} + \mathcal{O} \left( \frac{1}{n^2} \right).$$ Alors $\sum u_n$ converge si et seulement si $\alpha > 1$. 
\end{theo}
\marginnote[-1cm]{Voir exercice 3.43. \cite{oraux_x_ens_3}}
\begin{enumerate}
    \item[($\Rightarrow$)] Montrer que si $u_n=\frac{K}{n^{\alpha}}$ avec $K>0$ et $\alpha > 1$ alors $(u_n)$ vérifie la relation.
    \item[($\Leftarrow$)] Soit $(v_n)$ une suite vérifiant les hypothèses. Montrer qu'il existe $K>0$ tel que $v_n \sim \frac{K}{n^{\alpha}}$ avec $\alpha > 0$. Pour cela, étudier la série de terme général $\ln (v_n)$.
\end{enumerate}

Être capable de donner deux séries montrant qu'on ne peut pas conclure si $\alpha=1$.

\section{Suites sous-additive}
\emph{Exercice 2. TD I \cite{acamanes}}\\

\begin{defi}
    Une suite $(u_n)_{n\geqslant1}$ est dite \emph{sous-additive} si pour tout couple d'entiers non nuls $(n, m)$, $u_{n+m} \leqslant u_n + u_m$.
\end{defi}

\begin{exercice}
    Soit $(u_n)_{n \geqslant 1}$ une suite sous-additive. On pose $b_n = \min\limits_{k \in \llbracket 1, n \rrbracket} \frac{u_k}{k}$.
    \begin{enumerate}
        \item \begin{enumerate}
            \item Soit $\alpha \in \Rp$ et, pour $n \in \Ne, t_n = n^{\alpha}$. Montrer que $(t_n)$ est sous-additive si et seulement si $\alpha \leqslant 1$. Déterminer alors la limite de la suite $(t_n/n)$.
            \item Soit $(w_n)$ une suite réelle telle que pour tout $(n,m) \in (\Ne)^2, w_{n+m} = w_n + w_m$. Montrer que $(w_n)$ est sous-additive et calculer la limite de la suite $(w_n / n)$.
        \end{enumerate}
        \item Montrer qu'il existe $\ell \in \R \cup \{ - \infty \}$ telle que $\lim\limits_{n \to +\infty} v_n = \ell$.
        \item Montrer que pour tout $(m,n) \in (\Ne)^2, u_{nm} \leqslant m u_n$.
        \item On suppose que $\ell \not= - \infty$. Soit $\varepsilon > 0$.
        \begin{enumerate}
            \item Montrer qu'il existe $m \in \N$ tel que $\frac{u_m}{m} \leqslant \ell + \varepsilon$. 
            \item En utilisant le théorème de la division euclidienne, montrer que $\left( \frac{u_n}{n} \right)_{n \geqslant 1}$ converge vers $\ell$.
        \end{enumerate}
    \end{enumerate}
\end{exercice}

\begin{solution}
    1.a)\\ 
    \indent $(\Leftarrow)$ Étudier les cas où $n=m$.\\
    \indent $(\Rightarrow)$ Étudier la fonction $f:x \mapsto 1+x^{\alpha} - (1+x)^{\alpha}$.\\
    1.b)\\
    \indent Étudier le cas où $m=1$ et montrer que $w_n = n w_1$.\\
    2)\\
    \indent Montrer que la suite $(v_n)$ est décroissante.\\ 
    4.b)\\
    \indent Soit $(n, m) \in (\Ne)^2$. D'après le théorème de la division euclidienne, il existe un unique couple $(k, r) \in \Ne \times \llbracket0, m-1 \rrbracket$ tel que $n=km+r$.\\
    \indent Utiliser successivment la définition d'une suite sous-additive et les résultats des questions 3) et 4.a).
\end{solution}


\section{Étude de la suite de terme général \texorpdfstring{$u_n = \left( \frac{1}{b-a} \int_{a}^{b} f(x)^n \d x \right)^{1/n}$}{égal à une intégrale}}
\emph{Exercice 9. TD I}\\
La fonction $f$ est supposée continue et positive sur $[a, b]$.

\begin{itemize}
    \item La démarche générale consiste à encadrer $u_n$. 
    \item \underline{Majoration:} $f$ est continue sur un segment donc est en particuler bornée par un réel positif $M$. On peut montrer que $u_n \leqslant M$ \emph{(ne pas oublier l'argument de la continuité lors du passage à l'intégrale)}.
    \item \underline{Minoration:} soit $\varepsilon > 0$, soit $x_0$ tel que $f(x_0) = M$. Comme $f$ est continue en $x_0$, il existe $[c, d] \subset [a, b]$ tel que $x_0 \in [c, d]$ et pour tout $x \in [c, d]$, $f(x) \geqslant M - \varepsilon$ \emph{(un dessin permet de bien comprendre la stratégie)}.\\
    On peut ensuite montrer que $u_n \geqslant \left(\frac{d-c}{b-a} \right)^{1/n}(M-\varepsilon) \xrightarrow[n \to + \infty]{} M-\varepsilon$.
    \item Finalement, $\boxed{u_n \displaystyle \longrightarrow M = \max_{[ a, b ]} f = \Ninf{f}}$.
\end{itemize}

\section{Transformation d'\textsc{Abel}} \label{transformation_abel}
\marginnote[0cm]{Texte de \cite{oraux_x_ens_3} p. 262.}
La technique des transformations d'\textsc{Abel} peut être vue comme des intégrations par parties discrètes. On s'intéresse à la nature de la série $\sum a_n b_n$ où $(a_n)_{n \in \N}$ et $(b_n)_{n \in \N}$ sont deux suites réelles ou complexes. Pour $n \geqslant 0$ on note $A_n \defeq \sum\limits_{k=0}^n a_k$ la $n$-ième somme partielle de la série $\sum a_n$. On peut alors écrire $a_n = A_n - A_{n-1}$ (avec la convention $A_{-1} = 0$) et ainsi, pour tout entier $N$ on a
\begin{align*}
    \sum_{n=0}^N a_n b_n &= \sum_{n=0}^N (A_n - A_{n-1})b_n =  \sum_{n=0}^N A_n b_n -  \sum_{n=0}^N A_{n-1}b_n \\
    &= \sum_{n=0}^N A_n b_n - \sum_{n=0}^{N-1} A_n b_{n+1} \\
    \sum_{n=0}^N a_n b_n &= A_N b_{N+1} - \sum_{n=0}^N A_n(b_{n+1}-b_n). \quad (\star)
\end{align*}
La suite $(A_n)$ joue le rôle de la \say{ primitive } de $a_n$ et $b_{n+1} - b_n$ celui de la \say{ dérivée } de $b_n$. \\
Une application classique correspond à ce qu'on appelle parfois théorème d'\textsc{Abel} ou test de \textsc{Dirichlet}: 
\begin{theo}
    Lorsque la suite $(b_n)_{n \in \N}$ est décroissante de limite nulle et la suite des sommes partielles $(A_n)$ est bornée, alors la série $\sum a_n b_n$ converge. 
\end{theo}

En effet, dans $(\star)$ le terme $A_N b_{N+1}$ converge vers $0$ quand $N$ tend vers l'infini et la série $\sum A_n(b_n - b_{n+1})$ est absolument convergente car le terme général est un $\mathcal{O}(b_n - b_{n+1})$ avec la série à termes positifs $\sum(b_n - b_{n+1})$ qui est convergente. \\
Appliquons ce théorème aux séries trigonométriques de la forme $\sum \frac{\me^{\mi n x}}{n^\alpha}$ avec $\alpha > 0$ et $x \not \equiv 0 [2\pi]$ en prenant $a_n \defeq \me^{\mi n x}$ et $b_n \defeq \frac{1}{n^\alpha}$. Les sommes partielles $(A_n)$ sont effectivement bornées puisque
$$|A_n| = \left| \sum_{k=0}^n \me^{\mi k x}\right| = \left| \frac{1 - \me^{\mi (n+1)x}}{1-\me^{\mi x}} \right| = \left| \frac{\sin \left( \frac{n+1}{2} x\right)}{\sin \left( \frac{x}{2} \right)} \right| \leqslant \frac{1}{\left| \sin \left( \frac{x}{2} \right) \right|}.$$
Historiquement cette transformation fut utilisée par \textsc{Abel} en 1826 pour donner un exemple de série de fonctions continues dont la somme n'est pas continue \footnote{\textsc{Cauchy} affirme, en 1821, que la somme d'une série de fonctions continue est toujours continue (\textcolor{green}{rajouter la référence})}, à savoir $\sum \frac{\sin nx}{n}$.


\section{Mise en application d'une transformation d'\textsc{Abel} \cite{rms}}
Soient $(\varepsilon_n)_{n \in \N}$ une suite à termes dans $\{-1, 1\}$, et $(a_n)_{n \in \N}$ une suite décroissante de réels positifs telle que $\sum \varepsilon_n a_n$ converge. Montrer que $a_n \sum\limits_{k=0}^{n} \varepsilon_n \xrightarrow[n \to + \infty]{} 0$.


\section{Convergence et calcul de  \texorpdfstring{$\sum \frac{r}{2^r}$}{de la série de terme général r/2^r}}
\begin{exercice}
    Montrer la convergence de la série de terme général $\frac{r}{2^r}$ et prouver que $\sum\limits_{r=1}^{+ \infty} \frac{r}{2^r} = 2$. 
\end{exercice}

\marginnote[2cm]{
    \begin{theo}{}
        Soient $\sum u_n$ et $\sum v_n$ deux séries telles que $\sum u_n$ et $\sum v_n$ convergent absolument. Alors, en posant $w_n \defeq \sum\limits_{k=0}^n u_k v_{n-k}$, la série $\sum w_n$ converge absolument et
        $$\sum_{n=0}^{+\infty} u_n \cdot \sum_{n=0}^{+\infty} v_n = \sum_{n=0}^{+\infty} w_n.$$
    \end{theo}
}

\begin{elem_sol}
    Deux méthodes de résolution sont possibles bien que la première soit plus élégante. 
    \begin{itemize}
        \item La série $\sum \frac{1}{2^r}$ est une série géométrique absolument convergente. Ainsi, d'après le résultat sur les produits de \textsc{Cauchy}, 
        $$\sum_{r=0}^{+ \infty} \frac{r}{2^{r-1}} = \sum_{r=0}^{+ \infty} \sum_{k=0}^{r} \frac{1}{2^k \cdot 2^{r-k}} = \left( \sum_{r=0}^{+ \infty} \frac{1}{2^r} \right)^2 = 4.$$
        \item On peut également étudier la fonction $g : x \to \sum\limits_{r=0}^{n} \frac{x^r}{2^r}$.
    \end{itemize}
\end{elem_sol}


\section{Suites du type \texorpdfstring{$f(x_n) = n$}{f(x_n) = n}}
\begin{itemize}
    \item \emph{Exercice X 1 p.226 de \cite{exos_oraux}}\\
    Soit $n \in \N$, montrer que l'équation $x \me^x = n$ admet une unique solution $x_n \in \R$, en donner un équivalent puis un équivalent de $y_n = x_n - \ln(n)$. 
    \item Soit $n \in \Ne$, montrer que l'équation $x + \ln(x) = n$ admet une unique solution $x_n \in \R$, en donner un équivalent.
\end{itemize}

\section{Suites définies implicitement}

\begin{methode}
    Texte de \cite{oraux_x_ens_3} p. 181.
    \begin{enumerate}
        \item Montrer l'existence de $x_n$
        \item Démontrer la convergence de la suite $x_n$
        \item Déterminer un équivalent 
        \item Déterminer un développement asymptotique de $x_n$ \\
        Pour cela il faut commencer par déterminer dans la relation qui définit $x_n$ quels sont les termes prépondérants. 
    \end{enumerate}
\end{methode}

\section{Sommation par paquets}
\begin{exercice}
    \marginnote[0cm]{\cite{exos_oraux} p. 340}
    Soit $\sum\limits_{n \geqslant 1} u_n$ une série à termes réels positifs, telle que $(u_n)_{n \geqslant 1}$ est décroissante. Montrer que les séries $\sum\limits_{n \geqslant 1} u_n$ et $\sum\limits_{n \geqslant 2} 2^n u_{2^n}$ sont de même nature. 
\end{exercice}
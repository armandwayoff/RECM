Texte de \cite{oraux_x_ens_3} p. 262. \\

La technique des transformations d'\textsc{Abel} peut être vue comme des intégrations par parties discrètes. On s'intéresse à la nature de la série $\sum a_n b_n$ où $(a_n)_{n \in \N}$ et $(b_n)_{n \in \N}$ sont deux suites réelles ou complexes. Pour $n \geqslant 0$ on note $A_n \defeq \sum\limits_{k=0}^n a_k$ la $n$-ième somme partielle de la série $\sum a_n$. On peut alors écrire $a_n = A_n - A_{n-1}$ (avec la convention $A_{-1} = 0$) et ainsi, pour tout entier $N$ on a
\begin{align*}
    \sum_{n=0}^N a_n b_n &= \sum_{n=0}^N (A_n - A_{n-1})b_n =  \sum_{n=0}^N A_n b_n -  \sum_{n=0}^N A_{n-1}b_n \\
    &= \sum_{n=0}^N A_n b_n - \sum_{n=0}^{N-1} A_n b_{n+1} \\
    \sum_{n=0}^N a_n b_n &= A_N b_{N+1} - \sum_{n=0}^N A_n(b_{n+1}-b_n). \quad (\star)
\end{align*}
La suite $(A_n)$ joue le rôle de la \say{ primitive } de $a_n$ et $b_{n+1} - b_n$ celui de la \say{ dérivée } de $b_n$. \\
Une application classique correspond à ce qu'on appelle parfois théorème d'\textsc{Abel} ou test de \textsc{Dirichlet}: 
\begin{theo}
    Lorsque la suite $(b_n)_{n \in \N}$ est décroissante de limite nulle et la suite des sommes partielles $(A_n)$ est bornée, alors la série $\sum a_n b_n$ converge. 
\end{theo}

En effet, dans $(\star)$ le terme $A_N b_{N+1}$ converge vers $0$ quand $N$ tend vers l'infini et la série $\sum A_n(b_n - b_{n+1})$ est absolument convergente car le terme général est un $\mathcal{O}(b_n - b_{n+1})$ avec la série à termes positifs $\sum(b_n - b_{n+1})$ qui est convergente. \\
Appliquons ce théorème aux séries trigonométriques de la forme $\sum \frac{\me^{\mi n x}}{n^\alpha}$ avec $\alpha > 0$ et $x \not \equiv 0 [2\pi]$ en prenant $a_n \defeq \me^{\mi n x}$ et $b_n \defeq \frac{1}{n^\alpha}$. Les sommes partielles $(A_n)$ sont effectivement bornées puisque
$$|A_n| = \left| \sum_{k=0}^n \me^{\mi k x}\right| = \left| \frac{1 - \me^{\mi (n+1)x}}{1-\me^{\mi x}} \right| = \left| \frac{\sin \left( \frac{n+1}{2} x\right)}{\sin \left( \frac{x}{2} \right)} \right| \leqslant \frac{1}{\left| \sin \left( \frac{x}{2} \right) \right|}.$$
Historiquement cette transformation fut utilisée par \textsc{Abel} en 1826 pour donner un exemple de série de fonctions continues dont la somme n'est pas continue \footnote{\textsc{Cauchy} affirme, en 1821, que la somme d'une série de fonctions continue est toujours continue (\textcolor{green}{rajouter la référence})}, à savoir $\sum \frac{\sin nx}{n}$.

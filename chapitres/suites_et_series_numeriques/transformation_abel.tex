\begin{tcolorbox}
    Si $(\varepsilon_n)$ et $(v_n)$ sont des suites à valeurs réelles, si $(u_n)$ et $(S_n)$ sont les suites de termes généraux:
    $$u_n = \varepsilon_n v_n \text{ et } S_n = \sum_{k=0}^{n} \varepsilon_k,$$
    si $(S_n)$ est bornée dans $\R$, si $(v_n)$ converge vers $0$ et si la série $\sum |v_{n+1} - v_n|$ converge, alors la série $\sum u_n$ converge.
\end{tcolorbox}

\begin{itemize}
    \item Savoir énoncer et démontrer la \textbf{formule de sommation par parties} (analogie avec l'intégration par parties) (sans raisonner par récurrence).
    $$\boxed{\forall n \in \N,\ \sum_{k=0}^{n} v_k \varepsilon_k = \sum_{k=0}^{n-1} (v_k -v_{k+1})S_k + v_nS_n}$$
    \item Critère d'\textsc{Abel} et test de \textsc{Dirichlet}.
\end{itemize}
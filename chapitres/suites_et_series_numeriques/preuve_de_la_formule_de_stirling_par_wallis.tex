\begin{theo}
    $$n! \sim \left(\frac{n}{\me}\right)^n \sqrt{2 \pi n}$$
\end{theo}

\begin{preuve}
    \begin{enumerate}
        \item Montrer que $\left( \frac{n!\me^n}{\sqrt{n}n^n} \right)_{n \in \Ne}$ converge vers $\ell \in \R$
        \item Montrer que $\ell = \sqrt{2 \pi}$ en utilisant l'\nameref{integrale_wallis}:
        $$\Wallis_n \defeq \int_{0}^{\pi/2}\sin^n(t) \d t$$
        \begin{enumerate}
            \item Montrer que $(\Wallis_n)_{n \in \N}$ est décroissante
            \item Exprimer $\Wallis_{n+2}$ en fonction de $\Wallis_n$ grâce à une IPP: $$(n+2)\Wallis_{n+2} = (n+1)\Wallis_n$$
            \item Exprimer $\Wallis_{2p}$ et $\Wallis_{2p+1}$ en fonction de $p$:\\
            $$\Wallis_{2p} = \frac{\binom{2p}{p}}{2^{2p}}\frac{\pi}{2} \text{ et } \Wallis_{2p+1} = \frac{2^{2p} (p!)^2}{(2p+1)!}$$
            \item Utiliser les points (a) et (b) pour montrer que $\frac{\Wallis_n}{\Wallis_{n+1}} \longrightarrow 1$
            \item Utiliser les points (c) et (d) pour montrer que $\left ( \frac{2^n n!}{n ((2n)!)^2} \right)^4 \longrightarrow \pi$
            \item Utiliser le point 1. pour déterminer $\ell$
        \end{enumerate}
    \end{enumerate}
\end{preuve}

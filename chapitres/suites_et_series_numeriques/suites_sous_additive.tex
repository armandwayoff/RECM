\emph{Exercice 2. TD I \cite{acamanes}}\\

Une suite $(u_n)_{n\geqslant1}$ est dite sous-additive si pour tout couple d'entiers non nuls $(n, m)$, $u_{n+m} \leqslant u_n + u_m$.\\
1.a)\\ 
\indent $(\Leftarrow)$ Étudier les cas où $n=m$.\\
\indent $(\Rightarrow)$ Étudier la fonction $f:x \mapsto 1+x^{\alpha} - (1+x)^{\alpha}$.\\
1.b)\\
\indent Étudier le cas où $m=1$ et montrer que $w_n = n w_1$.\\
2)\\
\indent Montrer que la suite $(v_n)$ est décroissante.\\ 
4.b)\\
\indent Soit $(n, m) \in (\Ne)^2$. D'après le théorème de la division euclidienne, il existe un unique couple $(k, r) \in \Ne \times \llbracket0, m-1 \rrbracket$ tel que $n=km+r$.\\
\indent Utiliser successivment la définition d'une suite sous-additive et les résultats des questions 3) et 4.a).
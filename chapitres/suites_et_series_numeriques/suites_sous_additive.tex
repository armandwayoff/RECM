\marginnote[0cm]{\emph{Exercice 2. TD I \cite{acamanes}}}
\begin{defi}
    Une suite $(u_n)_{n\geqslant1}$ est dite \emph{sous-additive} si pour tout couple d'entiers non nuls $(n, m)$, $u_{n+m} \leqslant u_n + u_m$.
\end{defi}

\begin{exercice}
    Soit $(u_n)_{n \geqslant 1}$ une suite sous-additive. On pose $b_n \defeq \min\limits_{k \in \llbracket 1, n \rrbracket} \frac{u_k}{k}$.
    \begin{enumerate}
        \item \begin{enumerate}
            \item Soit $\alpha \in \Rp$ et, pour $n \in \Ne, t_n \defeq n^{\alpha}$. Montrer que $(t_n)$ est sous-additive si et seulement si $\alpha \leqslant 1$. Déterminer alors la limite de la suite $(t_n/n)$.
            \item Soit $(w_n)$ une suite réelle telle que pour tout $(n,m) \in (\Ne)^2, w_{n+m} = w_n + w_m$. Montrer que $(w_n)$ est sous-additive et calculer la limite de la suite $(w_n / n)$.
        \end{enumerate}
        \item Montrer qu'il existe $\ell \in \R \cup \{ - \infty \}$ telle que $\lim\limits_{n \to +\infty} v_n = \ell$.
        \item Montrer que pour tout $(m,n) \in (\Ne)^2, u_{nm} \leqslant m u_n$.
        \item On suppose que $\ell \not= - \infty$. Soit $\varepsilon > 0$.
        \begin{enumerate}
            \item Montrer qu'il existe $m \in \N$ tel que $\frac{u_m}{m} \leqslant \ell + \varepsilon$. 
            \item En utilisant le théorème de la division euclidienne, montrer que $\left( \frac{u_n}{n} \right)_{n \geqslant 1}$ converge vers $\ell$.
        \end{enumerate}
    \end{enumerate}
\end{exercice}

\begin{solution}
    1.a)\\ 
    \indent $(\Leftarrow)$ Étudier les cas où $n=m$.\\
    \indent $(\Rightarrow)$ Étudier la fonction $f:x \mapsto 1+x^{\alpha} - (1+x)^{\alpha}$.\\
    1.b)\\
    \indent Étudier le cas où $m=1$ et montrer que $w_n = n w_1$.\\
    2)\\
    \indent Montrer que la suite $(v_n)$ est décroissante.\\ 
    4.b)\\
    \indent Soit $(n, m) \in (\Ne)^2$. D'après le théorème de la division euclidienne, il existe un unique couple $(k, r) \in \Ne \times \llbracket0, m-1 \rrbracket$ tel que $n=km+r$.\\
    \indent Utiliser successivment la définition d'une suite sous-additive et les résultats des questions 3) et 4.a).
\end{solution}

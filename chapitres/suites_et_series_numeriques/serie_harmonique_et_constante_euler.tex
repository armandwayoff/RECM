\begin{defi}{Série harmonique}
    $$\Harmonique_n \defeq \sum_{k=1}^n \frac{1}{k}.$$
\end{defi}

\begin{marginfigure}[6cm]
	\def\a{0.1}
\def\b{11}

\begin{tikzpicture}
    \begin{axis}[width=6.5cm,
        axis lines=middle,
        grid=major,
        xmin=\a, xmax=\b+1,
        ymin=0, ymax=3.5,
        xlabel=$n$, xlabel style={right},
        % ylabel=$y$, ylabel style={above},
        xtick={1,...,11},
        yticklabels={$H_1$, $H_4$, $H_{11}$},
        ytick={1, 1+1/2+1/3+1/4, 1+1/2+1/3+1/4+1/5+1/6+1/7+1/8+1/9+1/10+1/11},
        tick style={thick},
        ticklabel style={font=\normalsize},
    ]
    \addplot[blue,thick,samples=100,domain=\a:\b] {ln(x)} node (m);
    
    \addplot[color=red,mark=*, mark size=1.5pt] coordinates {
    	(1,1)
    	(2,1+1/2)
    	(3,1+1/2+1/3)
    	(4,1+1/2+1/3+1/4)
    	(5,1+1/2+1/3+1/4+1/5)
    	(6,1+1/2+1/3+1/4+1/5+1/6)
    	(7,1+1/2+1/3+1/4+1/5+1/6+1/7)
    	(8,1+1/2+1/3+1/4+1/5+1/6+1/7+1/8)
    	(9,1+1/2+1/3+1/4+1/5+1/6+1/7+1/8+1/9)
    	(10,1+1/2+1/3+1/4+1/5+1/6+1/7+1/8+1/9+1/10)
    	(11,1+1/2+1/3+1/4+1/5+1/6+1/7+1/8+1/9+1/10+1/11)
    } node (l) {};
    \draw [latex-latex] (l) -- (m) node [left] {$\gamma$};
    % \node [left] at (l) {$\ln x$};
    \end{axis}
\end{tikzpicture}
\end{marginfigure}

\begin{prop}{}
    $\lim\limits_{n \to + \infty} \Harmonique_n = + \infty$.
\end{prop}

\begin{prop}{Équivalent de la série harmonique}
    $\forall n \geqslant 1, \ln(n+1) \leqslant \Harmonique_n \leqslant \ln n + 1$ et 
    $$\Harmonique_n \isEquivTo{n \to + \infty} \ln n.$$
\end{prop}

\begin{defi}{Constante d'\textsc{Euler}}
    La constante d'\textsc{Euler} $\gamma$ est définie par:
    $$\gamma \defeq \lim_{n \to \infty} \big(\Harmonique_n - \ln(n) \big) \approx 0,577\ 215\ 664 \dots$$
\end{defi}

\begin{prop}{Développement asymptotique de la série harmonique}
    $$\Harmonique_n =_{n \to + \infty} \ln n + \gamma + \frac{1}{2n} + o\left(\frac{1}{n}\right).$$
\end{prop}

\marginnote[2cm]{
    \begin{theo}{Comparaison série / intégrale}
        \cite{acamanes} ch 2. \\
        Soit $f$ une fonction continue par morceaux, décroissante de $\Rp$ à valeurs dans $\Rp$. Alors, la série de terme général 
        $$w_n \defeq \int_{n-1}^n f(t) \d t - f(n)$$
        est convergente. En particulier, $\sum f(n)$ converge si et seulement si $\displaystyle x \mapsto \int_0^x f(t) \d t$ admet une limite finie en $+ \infty$.
    \end{theo}
    \begin{methode}
        Le théorème de comparaison série / intégrale permet
        \begin{itemize}
            \item de montrer qu'une série converge ou diverge,
            \item d'obtenir un équivalent d'une somme partielle de série divergente,
            \item d'obtenir un équivalent du reste d'une série convergente.
        \end{itemize}
    \end{methode}
}

\begin{preuve}
    \item Poser $v_n = H_n - \ln(n)$
    \item Montrer que $v_{n+1}-v_n = \mathcal{O}\left(\frac{1}{n^2}\right)$\\
    \begin{align*}
        v_n &= \sum_{k=1}^n\frac{1}{k} -\ln(n) \\
        &= \sum_{k=1}^n \frac{1}{k} -\sum_{k=1}^n  \ln\left( \frac{k+1}{k} \right) \\
        &= \frac{1}{n} + \sum_{k=1}^n \underbrace{\left( \frac{1}{k} - \ln\left(1-\frac{1}{k}\right)\right)}_{\mathcal{O}\left(\frac{1}{k^2}\right)}
    \end{align*}
    On peut aussi montrer...
    \item ... la décroissance de la suite $(v_n)$. \\
    Soit $n \in \N$. 
    $$v_n - v_{n+1} = \ln(n+1) - \ln(n) - \frac{1}{n+1}$$
    Deux méthodes:
    \begin{itemize}
        \item On transforme $\ln(n+1) - \ln(n)$ en intégrale:
        $$v_n - v_{n+1} = \int_{n}^{n+1} \underbrace{\left( \frac{1}{t} - \frac{1}{n+1} \right)}_{\geqslant 0} \d t > 0.$$
        \item D'après le \textbf{théorème des accroissements finis}, il existe $c \in ]n, n+1[$ tel que 
        $$\ln(n+1) - \ln(n) = \ln'(c)((n+1) - n) = \frac{1}{c}$$
        d'où l'on tire que 
        $$v_n - v_{n+1} = \frac{1}{c} - \frac{1}{n+1} > 0.$$
    \end{itemize}
    \item ... que $\forall n \in \Ne,\ \ln(n+1) \leqslant H_n \leqslant \ln(n) + 1$ grâce à l'\textbf{encadrement de l'intégrale} sur $[k, k+1]$ de la fonction $t \mapsto \frac{1}{t}$.
\end{preuve}

\marginnote[-1cm]{\url{https://www.maths-france.fr/MathSpe/GrandsClassiquesDeConcours/SeriesNumeriques/SerieHarmonique.pdf}}

\begin{exercice}
    \marginnote[0cm]{\cite{exos_oraux} p. 334}
    Montrer que l'intégrale $\displaystyle \int_1^{+\infty} \bigg( \frac{1}{\lfloor x \rfloor} - \frac{1}{x}\bigg) \d x$ est égale à $\gamma$.
\end{exercice}

\begin{methode}
    Quand un exercice amène à manipuler des parties entières, il est très souvent judicieux d'utiliser les encadrements suivants
    $$x-1 < \lfloor x \rfloor \leqslant x,$$
    $$\lfloor x \rfloor \leqslant x < \lfloor x \rfloor + 1.$$
\end{methode}

\begin{solution}
\end{solution}
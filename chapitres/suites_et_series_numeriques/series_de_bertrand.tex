\begin{defi}{Séries de \textsc{Bertrand}}
    Soient $\alpha$ et $\beta$ deux réels. On nomme \emph{série de \textsc{Bertrand}} la série de terme général $\displaystyle \frac{1}{n^\alpha \ln^\beta (n)}$ pour $n \geqslant 2$. 
\end{defi}

\begin{theo}
    La série de \textsc{Bertrand} converge si et seulement si \begin{cases} \alpha > 1 \\
    \text{ou} \\ \alpha = 1 \text{ et } \beta > 1 \end{cases}.
\end{theo}

\begin{preuve}
\end{preuve}

\begin{exercice}
    \marginnote[0cm]{\cite{acamanes}}
    On note $h : x \mapsto \sum\limits_{n=2}^{+ \infty} \frac{1}{n^x \ln n}$.
    \begin{enumerate}
        \item Étudier la continuité de $h$ sur son domaine de définition.
        \item Étudier les limites de $h$ aux bornes de son intervalle de définition.
        \item Déterminer des équivalents de $h$ aux bornes de son intervalle de définition.
    \end{enumerate}
\end{exercice}

\begin{solution}
\begin{enumerate}
    On note $f_x : t \mapsto \frac{1}{t^x \ln t}$.
    \item D'après le théorème de \textsc{Bertrand} sur les séries numériques, l'ensemble de définition de $h$ est $\mathcal{D}_h \defeq ]1, + \infty[$. \\
    Soit $a > 1$. On se place sur $I \defeq [a, +\infty[$. Pour tout $x \in I$,
    $$\left| \frac{1}{n^x \ln n} \right| \leqslant \frac{1}{n^\alpha \ln n}$$
    comme $a > 1$, d'après le théorème de \textsc{Bertrand} sur les séries numériques, la série du terme majorant converge et donc par théorème de comparaison, la suite $(f_x)$ converge normalement sur tout segment de la forme de $I$. On en déduit que $h$ est continue sur $\mathcal{D}_h$. 
    \begin{itemize}
        \item En $+ \infty$: comme la série des $f_x$ converge uniformément sur $[2, + \infty[$ (on aurait pu choisir une valeur que $2$), d'après le théorème de la double limite
        $$\lim_{x \to + \infty} h(x) = \sum_{n=2}^{+ \infty} \left[\lim_{x \to +\infty} f_x(n) \right] = 0.$$
        \item En $1^+$: On montre que la fonction $f_x$ est décroissante et donc $h$ aussi. On note $\ell \defeq \lim\limits_{1^+} h$. D'après le théorème de la limite monotone, $\ell \in \R \cup \{ + \infty \}$. \\
        Supponson que $\ell \in \R$, alors
        $$h(x) = \sum_{n=2}^{+\infty} \frac{1}{n^x \ln n} \geqslant \sum_{n=2}^N \frac{1}{n^x \ln n}$$
        et en passant à la limite quand $x$ tend vers $1^+$ dans l'inégalité (ce qui est licite) on obtient
        $$\ell \geqslant \sum_{n=2}^N \frac{1}{n \ln n}.$$
        Nous aboutissons donc à une contradiction car la somme minorante diverge quand $N$ tend vers $+ \infty$. Finalement,
        $$\lim_{1^+} h = + \infty.$$
    \end{itemize}
    \item 
    \begin{itemize}
        \item En $+ \infty$: on intuite que la premier terme de la somme domine les autres. On a
        $$2^x \ln(2) h(x) = 1 + \sum_{n=3}^{+\infty} \left(\frac{2}{n}\right)^x \frac{\ln 2}{\ln n}.$$
        On peut montrer(...) que la somme converge normalement sur $[2, +\infty[$. On en déduit que 
        $$h(x) \isEquivTo{+\infty} \frac{1}{2^x \ln 2}.$$
        \item En $1^+$: un encadrement par la méthode des rectangles permet de trouver
        $$\int_{3}^{+\infty} f_x(t) \d t + \frac{1}{2^x \ln 2} \leqslant h(x) \leqslant \int_{2}^{+\infty} f_x(t) \d t + \frac{1}{2^x \ln 2}.$$
        On en déduit que 
        $$h(x) \isEquivTo{1^+} \int_2^{+\infty} f_x(t) \d t$$
        soit après calculs (...)
        $$h(x) \isEquivTo{1^+} - \ln(x-1).$$
    \end{itemize}
\end{enumerate}
\end{solution}
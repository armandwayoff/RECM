\begin{theo}
    Soit $\alpha$ un réel et $(u_n)_{n \in \N}$ une suite de réels strictement positifs. On suppose que
    $$\displaystyle \frac{u_{n+1}}{u_n} = 1 - \frac{\alpha}{n} + \mathcal{O} \left( \frac{1}{n^2} \right).$$ Alors $\sum u_n$ converge si et seulement si $\alpha > 1$. 
\end{theo}

Voir exercice 3.43. \cite{oraux_x_ens_3}.
\begin{enumerate}
    \item ($\Rightarrow$) Montrer que si $u_n=\frac{K}{n^{\alpha}}$ avec $K>0$ et $\alpha > 1$ alors $(u_n)$ vérifie la relation.
    \item ($\Leftarrow$) Soit $(v_n)$ une suite vérifiant les hypothèses. Montrer qu'il existe $K>0$ tel que $v_n \sim \frac{K}{n^{\alpha}}$ avec $\alpha > 0$. Pour cela, étudier la série de terme général $\ln (v_n)$.
\end{enumerate}

Être capable de donner deux séries montrant qu'on ne peut pas conclure si $\alpha=1$.
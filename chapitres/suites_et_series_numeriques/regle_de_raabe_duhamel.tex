\begin{theo}{Règle de \textsc{d'Alembert}}
    On suppose que $\lim\limits_{n \to + \infty} \frac{u_{n+1}}{u_n} = \ell$.
    \begin{itemize}
        \item Si $\ell < 1$, alors $\sum u_n$ converge.
        \item Si $\ell > 1$, alors $\sum u_n$ diverge.
    \end{itemize}
\end{theo}

Lorsque $\ell = 1$, on ne peut pas conclure. En effet, 
$$\sum \frac{1}{n} \text{ diverge et } \sum \frac{1}{n^2} \text{ converge}.$$

\begin{theo}{}
    Soit $\alpha$ un réel et $(u_n)_{n \in \N}$ une suite de réels strictement positifs. On suppose que
    $$\displaystyle \frac{u_{n+1}}{u_n} = 1 - \frac{\alpha}{n} + \mathcal{O} \left( \frac{1}{n^2} \right).$$ Alors $\sum u_n$ converge si et seulement si $\alpha > 1$. 
\end{theo}
\marginnote[-1cm]{Voir exercice 3.43. \cite{oraux_x_ens_3}}
\begin{preuve}
    \begin{enumerate}
        \item[($\Rightarrow$)] Montrer que si $u_n=\frac{K}{n^{\alpha}}$ avec $K>0$ et $\alpha > 1$ alors $(u_n)$ vérifie la relation.
        \item[($\Leftarrow$)] Soit $(v_n)$ une suite vérifiant les hypothèses. Montrer qu'il existe $K>0$ tel que $v_n \sim \frac{K}{n^{\alpha}}$ avec $\alpha > 0$. Pour cela, étudier la série de terme général $\ln (v_n)$.
    \end{enumerate}
\end{preuve}

On ne peut pas conclure si $\alpha=1$.
\begin{prop}{}
    Le nombre $\me \defeq \exp(1)$ est irrationnel.
\end{prop}
\marginnote[0cm]{Une version de la preuve est dans Proofs from the BOOK (p.47)}
La démonstration suivante est due à Joseph \textsc{Fourier} (1815).
\begin{preuve}
    Supposons par l'absurde qu'il existe deux entiers $a$ et $b$ non nuls tels que $\me = \frac{a}{b}$. Alors, pour tout $n \geqslant 0$,
    $$n! b \me = n! a.$$
    Le terme de droite est un entier et le terme de gauche s'écrit \note \marginnote[0cm]{$\displaystyle \note\ \me = \sum_{n=0}^{+ \infty} \frac{1}{n!}$}
    $$n! b \left(1 + \frac{1}{1!} + \frac{1}{2!} + \cdots + \frac{1}{n!} + \frac{1}{(n+1)!} + \cdots \right)$$
    qui se décompose en la somme d'un entier
    $$b n! \left(1 + \frac{1}{1!} + \frac{1}{2!} + \cdots + \frac{1}{n!} \right)$$
    et d'un second membre
    $$b \left( \frac{1}{n+1} + \frac{1}{(n+1)(n+2)} + \frac{1}{(n+1)(n+2)(n+3)}+ \cdots \right).$$
    Or ce second membre n'est pas entier car pour $n > 1$,
    \begin{align*}
        0 &< \frac{1}{n+1} + \frac{1}{(n+1)(n+2)} + \frac{1}{(n+1)(n+2)(n+3)} + \cdots \\
        & < \frac{1}{n+1} + \frac{1}{(n+1)^2} + \frac{1}{(n+1)^3} + \cdots = \frac{1}{n+1} \cdot \frac{1}{1-\frac{1}{n+1}} = \frac{1}{n}.
    \end{align*}
    Ainsi le membre de gauche n'est pas entier et on aboutit à une contradiction. On en déduit que le nombre $\me$ est irrationnel.
\end{preuve}

\marginnote{
J. \textsc{Liouville} montre en 1840 que $\me^2$ est également irrationnel. (à compléter)
Charles \textsc{Hermite} montre en 1873 que $\me$ est \emph{transcendant}
\begin{defi}{Nombre transcendant}
\end{defi}
}
La suite définie une norme sur $\K^n$ et établit une généralisation de l'inégalité de \textsc{Cauchy}-\textsc{Schwarz}. \\
\textbf{Notations.} Soit $x \in \K^n$. Dans tout la suite on note
$$
x = 
\begin{pmatrix}
    x_1 \\
    \vdots \\
    x_n
\end{pmatrix}.
$$
\begin{defi}{}
    Soit $u \in \K^n$. Pour tout réel $p \geqslant 1$, on définit l'application $\Vert \bm{\cdot} \Vert_p$ par
    $$\Vert u \Vert_p \defeq \left (\sum_{k=1}^{n} |u_i|^p \right)^{1/p}.$$
\end{defi}

\marginnote[0cm]{
    \begin{defi}{Norme}
        L'application $N : E \to \Rp$ est une \emph{norme} sur $E$ si
        \begin{itemize}
            \item[(\textsc{i})] \textbf{Séparabilité :} $\forall x \in E, N(x) = 0 \Leftrightarrow x = 0_E$.
            \item[(\textsc{ii})] \textbf{Homogénéité :} $\forall x \in E, \forall \lambda \in \K, N(\lambda x) = |\lambda| N(x).$
            \item[(\textsc{iii})] \textbf{Inégalité triangulaire :} $\forall (x, y) \in E^2, N(x+y) \leqslant N(x) + N(y)$.
        \end{itemize}
    \end{defi}
}

\begin{prop}{Normes de \textsc{Hölder}}
    L'application $\Vert \bm{\cdot} \Vert_p$ définie une norme sur $\K^n$.
\end{prop}

Pour montrer que l'application $\Vert \bm{\cdot} \Vert_p$ définie bien une norme sur $\K^n$, il faut entre autre montrer qu'elle vérifie l'inégalité triangulaire et pour montrer cela nous allons d'abord démontrer l'inégalité de \textsc{Hölder}:

\begin{prop}{Inégalité de \textsc{Hölder}}
    Soient deux réels $p > 1$ et $q > 1$ tels que $\frac{1}{p} + \frac{1}{q} = 1$ ($p$ et $q$ sont \emph{conjugués}). Pour tout $u, v \in \K^n$
    $$\sum_{k=1}^{n} |u_k v_k| \leqslant \Vert u \Vert_p \Vert v \Vert_q.$$
\end{prop}

%\begin{marginfigure}[0cm]
%    \centering
%    % https://tex.stackexchange.com/questions/394923/how-one-can-draw-a-convex-function

\begin{tikzpicture}
\begin{axis}[width=5in,axis equal image,
    axis lines=middle,
    xmin=0,xmax=8,
    xlabel=$x$,ylabel=$y$,
    ymin=-0.25,ymax=4,
    xtick={\empty},ytick={\empty}, axis on top
]

% 
\addplot[thick,domain=0.25:7,blue,name path = A]  {-x/3 + 2.75} coordinate[pos=0.4] (m) ;
\draw[thick,blue, name path =B] (0.15,4) .. controls (1,1) and (4,0) .. (6,2) node[pos=0.95, color=black, right]  {$f(x)$} coordinate[pos=0.075] (a1)  coordinate[pos=0.95] (a2);
\path [name intersections={of=A and B, by={a,b}}];

% 
\draw[densely dashed] (0,0) -| node[pos=0.5, color=black, label=below:$a$] {}(a1);
\draw[densely dashed] (0,0) -| node[pos=0.5, color=black, label=below:$x_{1}$] {}(a);
\draw[densely dashed, name path=D] (3,0) -|node[pos=0.5, color=black, label=below:$\lambda x_{1}+ (1-\lambda)x_{2}$] {} node[pos=1, fill,circle,inner sep=1pt] {}(m);
\draw[densely dashed] (0,0) -|node[pos=0.5, color=black, label=below:$x_{2}$] {}(b);
\draw[densely dashed] (0,0) -|node[pos=0.5, color=black, label=below:$b$] {}(a2);

% 
\path [name intersections={of=B and D, by={c}}] node[fill,circle,inner sep=1pt] at (c) {}; 

% 
\node[anchor=south west, text=black] (d) at (0.75,3) {$f[\lambda x_{1}+(1-\lambda)x_{2}]$};
\node[anchor=south west, text=black] (e) at (5,2.5) {$\lambda f(x_{1})+(1-\lambda)f(x_{2})$};
\draw[-{Latex[width=4pt,length=6pt]}, densely dashed] (d) -- (c);
\draw[-{Latex[width=4pt,length=6pt]}, densely dashed] (e) -- (m);
\end{axis}
\end{tikzpicture}
%\end{marginfigure}

\begin{preuve}
    \marginnote[0cm]{\url{https://bibmath.net/dico/index.php?action=affiche&quoi=./h/holder.html}}
    \begin{itemize}
        \item Un lemme fondamental: \\
        D'après la concavité de la fonction logarithme, 
        $$\forall (x, y) \in \Rpe^2,\ \ln \left( \frac{x^p}{p} + \frac{y^q}{q} \right) \geqslant \frac{\ln(x^p)}{p} + \frac{\ln(y^q)}{q} = \ln(xy).$$
        En passant à l'exponentielle, on obtient
        $$xy \leqslant \frac{x^p}{p} + \frac{y^q}{q}.$$
        \item Un cas particulier crutial: \\
        Nous démontrons pour le moment l'inégalité dans le cas où $\sum\limits_{k=1}^n |u_k|^p = 1$ et $\sum\limits_{k=1}^n |v_k|^q = 1$. \\
        D'après le lemme, pour tout $k \in \llbracket 1, n \rrbracket$,
        $$|u_k| |v_k| \leqslant \frac{|u_k|^p}{p} + \frac{|v_k|^q}{q}.$$
        En sommant cette inégalité pour $k$ allant de $1$ à $n$, on obtient le résultat.
        \item Raisonnement par homogénéité: \\
        Pour obtenir le cas général, il suffit d'appliquer la cas particulier avec 
        $$|u'_k| = \frac{u_k}{\Vert u \Vert_p} \text{ et } |v'_k| = \frac{v_k}{\Vert v \Vert_p}.$$
    \end{itemize}
\end{preuve}

\begin{remarque}
    Pour $p = q = 2$, on retrouve l'inégalité de \textsc{Cauchy}-\textsc{Schwarz}.
\end{remarque}

Montrons maintenant que $\Vert \bm{\cdot} \Vert_p$ définie bien une norme sur $\K^n$.

\begin{preuve}
    \marginnote[-1cm]{(lire aussi \emph{Chapitre 4 - Normes} page 39 \cite{matrices}) Exercice 4.81 page 377 \cite{oraux_x_ens_3}}
    Tout d'abord, on s'assure que $\Vert \bm{\cdot} \Vert_p$ est bien à valeurs positives.
    \begin{itemize}
        \item[(\textsc{i})] \textbf{Séparabilité :}
        \begin{itemize}
            \item[$(\Leftarrow)$] Immédiat.
            \item[$(\Rightarrow)$] Soit $u \in \K^n$ tel que $\Vert \lambda u \Vert_p = 0$. Alors
            $$\sum_{k=1}^{n} |\lambda u_i|^p = 0$$
            qui est une somme de termes positifs donc chacun des termes est nul et $$\forall k \in \llbracket 1, n \rrbracket,\ u_k = 0$$
            ce qui assure que $u = 0$.
        \end{itemize}
        \item[(\textsc{ii})] \textbf{Homogénéité :} soient $u \in \K^n$ et $\lambda \in \K$,
        $$\Vert \lambda u \Vert_p = \left (\sum_{k=1}^{n} |\lambda u_i|^p \right)^{1/p} ...$$
        \item[(\textsc{iii})] \textbf{Inégalité triangulaire :} 
        Soient $u, v \in \K^n$, montrons que 
        $$\Vert u + v \Vert_p \leqslant \Vert u \Vert_p \Vert v \Vert_p.$$
        Soit $k \in \llbracket 1, n \rrbracket$,
        $$|u_k + v_k|^p = |u_k| \times |u_k + v_k|^{p-1} + |v_k| \times |u_k + v_k|^{p-1}$$ 
        et sommer pour $k$ allant de $1$ à $n$. Appliquer le résultat précédent à chaque somme, factoriser et multiplier l'inégalité par une somme judicieuse. 
    \end{itemize}
\end{preuve}    

\begin{prop}{}
    \marginnote[0cm]{Proposition 4.1.3. \cite{matrices}}
    Toutes les normes de $E = \K^n$ sont équivalentes. Par exemple:
    $$\Vert x \Vert_\infty \leqslant \Vert x \Vert_p \leqslant p^{1/p} \Vert x \Vert_\infty$$
\end{prop}

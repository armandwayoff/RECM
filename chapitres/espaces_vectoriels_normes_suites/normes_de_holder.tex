Deux résultats: une norme sur $\K^n$ et une généralisation de l'inégalité de \textsc{Cauchy}-\textsc{Schwarz}. 
\begin{defi}
    Soit $\bm{v} \in \K^n$. Pour tout réel $p \geqslant 1$, l'application $\Vert \bm{\cdot} \Vert_p$ définie par
    $$\Vert \bm{v} \Vert_p = \left (\sum_{k=1}^{n} |v_i|^p \right)^{1/p}$$
    est une norme sur $\K^n$.
\end{defi}

\begin{prop}
    Soient deux réels $p > 1$ et $q > 1$ tels que $\frac{1}{p} + \frac{1}{q} = 1$ ($p$ et $q$ sont \emph{conjugués}). Pour tout $\bm{u}, \bm{v} \in \K^n$
    $$\sum_{k=1}^{n} |u_k v_k| \leqslant \Vert \bm{u} \Vert_p \Vert \bm{v} \Vert_q.$$
\end{prop}

\begin{preuve}(lire aussi \emph{Chapitre 4 - Normes} page 39 \cite{matrices}) \\
    Exercice 4.81 page 377 \cite{oraux_x_ens_3}.
    \begin{enumerate}
        \item Prouver que pour tous $a \geqslant 0, b \geqslant 0: ab \leqslant \frac{a^p}{p} + \frac{b^q}{q}$.
        \begin{itemize}
            \item Utiliser l'\textbf{inégalité de convexité de l'exponentielle}.
        \end{itemize}
        \item Démontrer l'inégalité de \textsc{Hölder}.
        \begin{itemize}
            \item Poser $u'_k = \frac{u_k}{\Vert \bm{u} \Vert_p}$ et $v'_k = \frac{v_k}{\Vert \bm{v} \Vert_p}$ et appliquer le résultat précédent. 
        \end{itemize}
        \item Montrer que $\Vert \bm{\cdot} \Vert$ définit une norme sur $\K^n$.
        \begin{itemize}
            \item Inégalité triangulaire: écrire 
            $$|u_k + v_k|^p = |u_k| \times |u_k + v_k|^{p-1} + |v_k| \times |u_k + v_k|^{p-1}$$ 
            et sommer pour $k$ allant de $1$ à $n$. Appliquer le résultat précédent à chaque somme, factoriser et multiplier l'inégalité par une somme judicieuse. 
        \end{itemize}
    \end{enumerate}
\end{preuve}     

\begin{box_titre}{Proposition 4.1.3. \cite{matrices}}
    Toutes les normes de $E = \K^n$ sont équivalentes. Par exemple:
    $$\Vert x \Vert_\infty \leqslant \Vert x \Vert_p \leqslant p^{1/p} \Vert x \Vert_\infty$$
\end{box_titre}

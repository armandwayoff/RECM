\marginnote[1cm]{Grossièrement on voit que \say{ ça marche } car 
$$v_n \longrightarrow \frac{1}{2^n} \underbrace{\sum_{k=0}^n \binom{n}{k}}_{=2^n} \times \ell = \ell.$$}

\begin{exercice}
    Soit $(u_n)_{n \in \N}$ une suite réelle convergeant vers $\ell$. On définit une suite $(v_n)_{n \in \N}$ par 
    $$\forall n \in \N, v_n \defeq \frac{1}{2^n} \sum_{k=0}^{n} \binom{n}{k} u_k.$$
    Montrer que $\displaystyle \lim_{n \rightarrow + \infty} v_n = \ell$.
\end{exercice}

\begin{elem_sol}
    Le méthode consiste à se ramener au cas où $\ell = 0$ en posant deux suites auxiliares $u_n'=u_n - \ell$ et $v_n' = v_n - \ell$. La démarche est en suite analogue à la démonstration du \nameref{lemme_cesaro}.
\end{elem_sol}

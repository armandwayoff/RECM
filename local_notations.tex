% Pour les matrices par blocs
\newcommand{\rvline}{\hspace*{-\arraycolsep}\vline\hspace*{-\arraycolsep}}

\DeclareMathOperator{\sinc}{\mathrm{sinc}}

\newcommand{\todoinline}[1]{\todo[color=green!40, bordercolor=red, inline]{#1}}
\newcommand{\entiers}[2]{\llbracket#1,#2\rrbracket}

\newcommand{\todoarmand}[1]{\todo[color=blue!20, bordercolor=red, inline]{#1}}

\newcommand\coauteur{%
    \raisebox{-0.25ex}{$\quad \scalebox{1.2}{$\cdot$} \quad$~}%
}

% Pour automatiser l'espace après les quantificateurs
\let\oldforall\forall
\renewcommand{\forall}{\oldforall \, }

\let\oldexist\exists
\renewcommand{\exists}{\oldexist \: }

\newcommand\pageblanche{
    \null
    \thispagestyle{empty}
    \addtocounter{page}{-1}
    \newpage
}

% Pour pouvoir enlever l'indentation dans un itemize
% https://tex.stackexchange.com/questions/131637/no-indentation-for-non-item-within-itemize
\newcommand\NoIndent[1]{%
  \par\vbox{\parbox[t]{\linewidth}{#1}}%
}


\newcommand{\nompropre}[1]{\textsc{#1}}

% \newcommand{\nom}[1]{\textsc{#1}\index[mathematiciens]{\textsc{#1}}}
\newcommand{\nom}[1]{\gls{#1}}

%%%%%%%%%%%%%%% Théorèmes utilisés %%%%%%%%%%%%%%%%%
\newcommand{\etoile}[1]{
  \ifthenelse{\isodd{\thepage}}
    {\begin{flushright}#1 $\bigstar$\end{flushright}}
    {$\bigstar$ #1}
}

\DeclareRobustCommand{\indice}[1]{%
    % Liste des préfixes à couper
    \StrSubstitute{#1}{théorème d'}{}[\temp]%
    \StrSubstitute{\temp}{théorème de la }{}[\temp]%
    \StrSubstitute{\temp}{théorème de }{}[\temp]%
    \StrSubstitute{\temp}{théorème des }{}[\temp]%
    \StrSubstitute{\temp}{théorèmes d'}{}[\temp]%
    \StrSubstitute{\temp}{théorèmes de }{}[\temp]%
    \StrSubstitute{\temp}{théorème }{}[\temp]%
    \index[theoremesutilises]{\temp}%
}

\DeclareRobustCommand{\theoremeutilise}[3][-7pt]{%
    #2\marginnote[#1]{
        \etoile{
            % \hyperref[#3]{ % pour renvoyer vers l'énoncé du théorème
                \expandafter\MakeUppercase#2
            % }
            \indice{#2}
        }
    }%
}
% \theoremeutilise[arg. optionel : décalage vertical de la marginnote]{nom du théorème}{label}
%%%%%%%%%%%%%%%%%%%%%%%%%%%%%%%%%%%%%%%%%%%%%%%%%%%%


% Commande pour indiquer une source
\newcommand{\source}[1]{
\marginnote[0cm]{\textsc{\textbf{Source --}} #1}
}


% Groupes pour l'index des notations
% https://cs.overleaf.com/learn/latex/Nomenclatures
% -----------------------------------------
\renewcommand\nomgroup[1]{%
  \item[\bfseries
  \ifstrequal{#1}{A}{Analyse}{%
  \ifstrequal{#1}{B}{Algèbre}{%
  \ifstrequal{#1}{C}{Probabilités}{}}}%
]}
% -----------------------------------------

%%%%%% Flowcharts %%%%%%
\newcounter{boxlblcounter}  
\newcommand{\makeboxlabel}[1]{\fbox{#1}\hfill}% \hfill fills the label box
\newenvironment{boxlabel}
  {\begin{list}
    {\arabic{boxlblcounter}}
    {\usecounter{boxlblcounter}
     \setlength{\labelwidth}{3em}
     \setlength{\labelsep}{0em}
     \setlength{\itemsep}{2pt}
     \setlength{\leftmargin}{1.5cm}
     \setlength{\rightmargin}{2cm}
     \setlength{\itemindent}{0em} 
     \let\makelabel=\makeboxlabel
    }
  }
{\end{list}}

%== Blocs
\tikzstyle{decision} = [
  rectangle, rounded corners, dashed,
  inner sep = 10pt,
  draw=colordef, very thick,
  minimum height=5ex,
  align=center,
  node distance=1.5cm and 2cm,
  ]
  \tikzstyle{block} =
  [rectangle,
  draw=colorexample, very thick,
  fill=colorexample!5!white,
  align=center,
  inner sep = 10pt,
  rounded corners, 
  minimum height=4ex]
  
\tikzstyle{line} = [draw, very thick, color=black, -latex']

\tikzstyle{cloud} =
  [ellipse, draw=colortheorem,
  very thick,
  align=center,
  minimum height=2em, inner sep=1pt]

\definecolor{colordef}{RGB}{0,153,153}
% DarkSlateGray
\definecolor{colortitle}{RGB}{51,102,102}
% 
\definecolor{colorproposition}{RGB}{255,51,0}
\definecolor{colortheorem}{RGB}{255,0,0}
\definecolor{colorexample}{RGB}{51,153,102}
\definecolor{colorinterpretation}{RGB}{204,255,51}
\definecolor{colorstrategie}{RGB}{204,153,0}

% \usepgfplotslibrary{external} 
% \tikzexternalize

  
%%%%%%%%%%%%%%%%%%%%%%%%%%%%%%%%%%%%%%%


\def\tick#1#2{\draw (#1) ++ (#2:0.125) --++ (#2-180:0.25)} % pour les "ticks" dans les figures type graphe de fonction

\section{Remarques générales}

% \todoinline{
% * Prévoir un truc (un lien hypertexte pour le pdf, une façon de numéroter qui relie les exercices) qui permet de relier des exercices entre eux, par ex. Hölder dans un chapitre convexité et les $L^p$ ?\\

% * Prévoir de mettre des références (livres, articles ?) par chapitre ? Par section ?\\

% * Prévoir une annexe avec les théorèmes utilisés ?\\

% * Niveau présentation, je pense qu'il faut choisir entre des propriétés énoncées puis démontrées ou des exercices. Je pencherais pour des petites propriétés démontrées pas à pas, comme le serait un problème et, éventuellement, à la fin, un exercice d'application (non corrigé ?)\\
% }

\todoinline{Pour mémoire, les environnements utilisés :
"demo", "elemdemo" et "elemsolution"\\
Titre optionnel entre crochets à tous les environnements: théorème, proposition, ..., exercice, remarque, démonstration, ...\\
environnements "questions" et "reponses" à la place des simples "enumerate".}

\todoarmand{J'ai trouvé ce livre de ECS \url{https://excerpts.numilog.com/books/9782340005853.pdf} qui propose 17 thèmes classiques qui recouvrent l'ensemble de leur programme}

\todoinline{
* Reprendre les $\ell^p$ pour éviter les redites avec les $L^p$.\\

* Pour les $\ell^p(\K^n)$, lorsque $n = 2$, on peut illustrer la forme des différentes boules. J'ai une animation sur ce sujet qui traîne. \\
}

\todoarmand{
Ajouter les numéros de page aux liens hypertextes
}
% https://tex.stackexchange.com/questions/88281/how-to-change-font-for-the-integral-symbol
\makeatletter
\def\upintkern@{\mkern-7mu\mathchoice{\mkern-3.5mu}{}{}{}}
\def\upintdots@{\mathchoice{\mkern-4mu\@cdots\mkern-4mu}%
 {{\cdotp}\mkern1.5mu{\cdotp}\mkern1.5mu{\cdotp}}%
 {{\cdotp}\mkern1mu{\cdotp}\mkern1mu{\cdotp}}%
 {{\cdotp}\mkern1mu{\cdotp}\mkern1mu{\cdotp}}}
\newcommand{\upiint}{\DOTSI\protect\UpMultiIntegral{2}}
\newcommand{\upiiint}{\DOTSI\protect\UpMultiIntegral{3}}
\newcommand{\upiiiint}{\DOTSI\protect\UpMultiIntegral{4}}
\newcommand{\upidotsint}{\DOTSI\protect\UpMultiIntegral{0}}
\newcommand{\UpMultiIntegral}[1]{%
  \edef\ints@c{\noexpand\upintop
    \ifnum#1=\z@\noexpand\upintdots@\else\noexpand\upintkern@\fi
    \ifnum#1>\tw@\noexpand\upintop\noexpand\upintkern@\fi
    \ifnum#1>\thr@@\noexpand\upintop\noexpand\upintkern@\fi
    \noexpand\upintop
    \noexpand\ilimits@
  }%
  \futurelet\@let@token\ints@a
}
\makeatother

\DeclareFontFamily{OMX}{mdbch}{}
\DeclareFontShape{OMX}{mdbch}{m}{n}{ <->s * [0.8]  mdbchr7v }{}
\DeclareFontShape{OMX}{mdbch}{b}{n}{ <->s * [0.8]  mdbchb7v }{}
\DeclareFontShape{OMX}{mdbch}{bx}{n}{<->ssub * mdbch/b/n}{}

\DeclareSymbolFont{uplargesymbols}{OMX}{mdbch}{m}{n}
\SetSymbolFont{uplargesymbols}{bold}{OMX}{mdbch}{b}{n}
\DeclareMathSymbol{\upintop}{\mathop}{uplargesymbols}{82}

\makeatletter
\newcommand{\upint}{\DOTSI\upintop\ilimits@}
\makeatother

\let\int\upint
\let\idotsint\upidotsint
%%%%%%%%%%%%%%%%%%%%%%%%%%%%%%%%%%%%%%%%%%%%%%%%%%%%%%%%%%%%%

% https://tex.stackexchange.com/questions/253077/how-do-you-create-a-set-in-latex
\DeclarePairedDelimiterX\ens[1]\lbrace\rbrace{\def\tq{\;\delimsize\vert\;}#1}

\DeclareMathOperator{\ch}{ch}

\newcommand{\me}{\mathrm{e}}
\newcommand{\mi}{\mathrm{i}\,}
\newcommand{\mj}{\mathrm{j}} % racine troisième de l'unité

\newcommand{\R}{\mathbf{R}}
\newcommand{\Rp}{\R_+}
\renewcommand{\Re}{\R^\star}
\newcommand{\Rpe}{\Re_+}
\newcommand{\C}{\mathbf{C}}
\newcommand{\Ce}{\C^\star}
\newcommand{\K}{\mathbf{K}}
\newcommand{\Ke}{\K^\star}
\newcommand{\N}{\mathbf{N}}
\newcommand{\Ne}{\N^\star}
\newcommand{\Z}{\mathbf{Z}}
\newcommand{\Q}{\mathbf{Q}}
\newcommand{\A}{\mathbb{A}} % Nombres algébriques
\newcommand{\Premier}{\mathbb{P}} % Nombres premiers

\renewcommand{\d}{\, \mathrm{d}}

\newcommand{\Ninf}[1]{\Vert #1 \Vert_\infty}
\newcommand{\norme}[1]{\Vert #1 \Vert}
\newcommand{\segN}[2]{\llbracket #1 , #2 \rrbracket}

% Complexes
\newcommand{\Reel}{\mathrm{Re}}

\newcommand{\ptnclegras}[1]{\textbf{#1}}
\newcommand{\ptnclecadre}[1]{\boxed{#1}}

% Algèbre
% \newcommand{\Gl}{\mathscr{G}\kern-0.16em\ell}
\newcommand{\Gl}{\mathrm{GL}}
\newcommand{\I}{\mathrm{I}}
\newcommand{\Id}{\mathrm{Id}}
\newcommand{\Rg}{\mathrm{Rg}\,}
\newcommand{\Ker}{\mathrm{Ker}\,}
\newcommand{\Vect}{\mathrm{Vect}}
\newcommand{\Sp}{\mathrm{Sp}}
\newcommand{\M}{\mathscr{M}}
\newcommand{\Endo}{\mathscr{L}}
\newcommand{\Trsp}[1]{#1^\top}
\newcommand{\Inv}[1]{#1^{-1}}
\renewcommand{\Tr}{\mathrm{Tr}}
\newcommand{\com}{\mathrm{com}}
\newcommand{\Mat}{\mathrm{Mat}}
\newcommand{\Diag}{\mathrm{Diag}}
\newcommand{\Ortho}{\mathrm{O}}
\newcommand{\Sym}{\mathscr{S}}

% Probas
\newcommand{\E}{\mathbf{E}}
\newcommand{\V}{\mathbf{V}}
\renewcommand{\P}{\mathbf{P}}

% Suites remarquables
\newcommand{\Leg}{\mathrm{L}}
\newcommand{\Lag}{\mathrm{L}}
\newcommand{\Hilb}{\mathrm{H}}
\newcommand{\Hermite}{\mathrm{H}}
\newcommand{\Bern}{\mathrm{B}}
\newcommand{\Tcheby}{\mathrm{T}}
\newcommand{\Wallis}{\mathrm{W}}
\newcommand{\Cauchy}{\mathrm{C}}
\newcommand{\Gram}{\mathrm{G}}
\newcommand{\Bernstein}{\mathrm{B}}
\newcommand{\Vandermonde}{\mathrm{V}}
\newcommand{\Harmonique}{\mathrm{H}}

% Symboles
%https://tex.stackexchange.com/questions/4216/how-to-typeset-correctly
% :=
\newcommand*{\defeq}{\mathrel{\vcenter{\baselineskip0.5ex \lineskiplimit0pt
                     \hbox{\scriptsize.}\hbox{\scriptsize.}}}%
                     =}
% =:
\newcommand*{\defeqright}{=\mathrel{\vcenter{\baselineskip0.5ex \lineskiplimit0pt
                     \hbox{\scriptsize.}\hbox{\scriptsize.}}}%
                      }

% https://tex.stackexchange.com/questions/349747/equivalence-of-sequence-sim-with-some-text-under      
\newcommand{\isEquivTo}[1]{%
  \mathpalette\isEquivToInner{#1}%
}
\newcommand{\isEquivToInner}[2]{%
  \ifx#1\displaystyle
    \underset{#2}{\sim}
  \else
    \sim_{#2}
  \fi
}

\newcommand{\note}{\hbox{\scriptsize $\blacklozenge$\ }}
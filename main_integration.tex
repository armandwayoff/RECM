\documentclass[
	a4paper, % Format du livre
	fontsize=8pt, % Taille de la police (texte de base)
	twoside=true,
	chapterentrydots=true,
	numbers=noenddot,
]{kaobook}

\usepackage[T1]{fontenc} % Encodage de la police pour les caractères internationaux

\usepackage{anyfontsize} % Pour le titre de la page de couverture

% Pour faire des pages blanches
\usepackage{afterpage}
\newcommand\pageblanche{
    \null
    \thispagestyle{empty}
    \addtocounter{page}{-1}
    \newpage
}

% Load the bibliography package
\usepackage{kaobiblio}
\addbibresource{main.bib} % Bibliography file

% Load mathematical packages for theorems and related environments
\usepackage[framed=true]{kaotheorems}

%%%% Ajoute par Alain pour pouvoir compiler le kaorefs %%%%
\usepackage[french]{babel}

\usepackage[dvipsnames]{xcolor}
\usepackage{ragged2e}

% Pour la fonction Gamma
\usepackage{pgfplots}
\pgfplotsset{compat=newest}

% Pour la figure homothétie
\usepackage{tkz-euclide}

\usepackage{hyperref}
\usepackage{multicol}
\usepgfplotslibrary{fillbetween}

\usepackage{tikz}
\usetikzlibrary{automata, 
                decorations.pathreplacing, % pour les curly braces de la fig du déterminant
                calligraphy, % pour les curly braces de la fig du déterminant
                arrows, % customizing arrows
                positioning, % positioning nodes
                calc,
                bending, % figure racines troisième de l'unité
                matrix % critère de nilpotence par la trace
                }
% Pour le schéma de la preuve de la densité des matrices diagonalisables
\usetikzlibrary{decorations.markings,arrows}
\tikzset{
    arrowMe/.style={
        postaction=decorate,
        decoration={
            markings,
            mark=at position .45 with {\arrow[thick]{#1}}
        }
    }
}
                
% Pour la figure de projection orthogonale
%\usepackage[active,tightpage]{preview}

% Pour pouvoir enlever l'indentation dans un itemize
% https://tex.stackexchange.com/questions/131637/no-indentation-for-non-item-within-itemize
\newcommand\NoIndent[1]{%
  \par\vbox{\parbox[t]{\linewidth}{#1}}%
}

\usepackage[outline]{contour} % glow around text
\contourlength{1.0pt}

% Pour les longues équations (théorème d'intégration par parties généralisées) https://tex.stackexchange.com/questions/3782/how-can-i-split-an-equation-over-two-or-more-lines
%\usepackage{breqn}

% Pour l'arbre de Calkin-Wilf
\usepackage{forest}

\tikzset{node distance=3.5cm, % Minimum distance between two nodes. Change if necessary.
    every state/.style={ % Sets the properties for each state
    semithick,
    fill=red!20},
    initial text={},     % No label on start arrow
    double distance=1pt, % Adjust appearance of accept states
    every edge/.style={  % Sets the properties for each transition
    draw,
    ->,>=stealth',     % Makes edges directed with bold arrowheads
    auto,
    semithick}}
          
\tikzset{
    block/.style={
    draw, 
    rectangle, 
    minimum height=1.5cm, 
    minimum width=3cm, align=center
    }, 
    line/.style={->,>=latex'}
}
           
\usepackage{tikz-3dplot}

% Pour le diagramme quiver
\usepackage{tikz-cd}
\usepackage{quiver}

% Scale pour le diagramme sur le red des endos
% https://tex.stackexchange.com/questions/325297/how-to-scale-a-tikzcd-diagram
\tikzcdset{scale cd/.style={every label/.append style={scale=#1},
    cells={nodes={scale=#1}}}}

%%%% Commente par Alain car conflit, deja charge par koa ? %%%%
% \usepackage{fancyhdr}
\usepackage{amsfonts} 
\usepackage{amsmath}
\usepackage{amsthm}
\usepackage{amssymb}

% \usepackage{mathtools}
\usepackage{bm}
\usepackage{stmaryrd}
%\usepackage{mathrsfs}
\usepackage{euscript}
\usepackage{enumitem}
\usepackage{xurl}
\usepackage{dsfont}  % Pour la fonction indicatrice \mathds{1}


\usepackage{cfr-lm} % https://tex.stackexchange.com/questions/301699/oldstyle-numbers-in-body-text-computer-modern

% Pour les matrices par blocs
\newcommand{\rvline}{\hspace*{-\arraycolsep}\vline\hspace*{-\arraycolsep}}

% Figures
\usepackage{physics}
\usepackage{mathdots}
\usepackage{cancel}
\usepackage{color}
\usepackage{siunitx}
\usepackage{array}
\usepackage{multirow}
\usepackage{gensymb}
\usepackage{tabularx}
\usepackage{extarrows}

%\newenvironment{preuve}[1][\proofname]{%
%  \begin{proof}[#1]$ $\par\nobreak\ignorespaces
%}{%
%  \end{proof}
%}

\newenvironment{preuve}
  {\begin{proof}[$\blacksquare$\ \textbf{\textsc{\emph{Démonstration}}}]}
  {\end{proof}}
  
\newenvironment{elem_preuve}
  {\begin{proof}[$\blacksquare$\ \textbf{\textsc{\emph{Éléments de démonstration}}}]}
  {\end{proof}}

\newenvironment{solution}
  {\renewcommand\qedsymbol{$\lhd$}\begin{proof}[$\blacktriangleright$ \textbf{\textsc{\emph{Solution}}}]}
  {\end{proof}}

\newenvironment{elem_sol}
  {\renewcommand\qedsymbol{$\lhd$}\begin{proof}[$\blacktriangleright$ \textbf{\textsc{\emph{Éléments de solution}}}]}
  {\end{proof}}

%%% Commentes car deja charges precedemment %%%
% \usepackage[top=15mm, bottom=20mm, left=15mm, right=25mm]{geometry}
% \usepackage[french]{babel}
% \usepackage[T1]{fontenc} % Encodage de la police pour les caractères internationaux
% \usepackage{amsmath} % pour \overset notamment
% \usepackage{amsthm} % pour les environnements démonstration, solution, ...
% \usepackage{amsfonts} % mathbb
% \usepackage{amssymb} % exemple : \blacksquare
% \usepackage{mathrsfs} % mathscr
% \usepackage{stmaryrd} % ll/rrbracket
% \usepackage{hyperref} % pour des liens cliquables

\usepackage{mathtools} % pour \ens notamment

% \usepackage{cfr-lm} % https://tex.stackexchange.com/questions/301699/oldstyle-numbers-in-body-text-computer-modern
\usepackage{hfoldsty}

\usepackage{etoolbox} % pour créer des listes de macros (https://tex.stackexchange.com/questions/48/how-can-i-specify-a-long-list-of-math-operators)

\usepackage{array} % pour la commande fonction[f]

\usepackage{xfrac} % pour \sfrac{}{}

\usepackage{physics} % pour les \dv{}{} ...

% \usepackage[scr=rsfs,cal=cm,frak=euler,bb=ams]{mathalfa} % https://tex.stackexchange.com/questions/418352/my-pdf-reader-does-not-show-calligraphic-letter-produced-by-latex-correctly
% https://mirror.ibcp.fr/pub/CTAN/macros/latex/contrib/mathalpha/doc/mathalpha-doc.pdf

\usepackage[dvipsnames]{xcolor}

% J'ai commenté car empechait d'afficher la toc d'un chapitre dans la marge
% \usepackage{sectsty} % style sections, sous-sections, ...

\usepackage{enumitem} % pour les labels des listes

\usepackage{cancel} % pour les expressions barrées

\usepackage{relsize} % pour les chiffres capitaux

\usepackage{xurl} % voir conflit avec hyperref

\usepackage{csquotes} % Package biblatex Warning: 'babel/polyglossia' detected but 'csquotes' missing. Loading 'csquotes' recommended.

%%%% Commente par Alain car deja charge par math_packages %%%%
% \usepackage{fancyhdr} % pour le formatage des entêtes et bas de pages


% guillements
\usepackage[
    left = \flqq{},% 
    right = \frqq{},% 
    leftsub = \flq{},% 
    rightsub = \frq{} %
]{dirtytalk}
\DeclareMathOperator{\ch}{ch}

\newcommand{\e}{\mathrm{e}}
\renewcommand{\i}{\mathrm{i}}
\renewcommand{\j}{\mathrm{j}} % racine troisième de l'unité

\newcommand{\R}{\mathbb{R}}
\newcommand{\Rp}{\R_+}
\newcommand{\Rm}{\R_-}
\renewcommand{\Re}{\R^\star}
\newcommand{\Rpe}{\Re_+}
%%% Ajout du re par Alain %%%
\renewcommand{\C}{\mathbb{C}}
\newcommand{\Ce}{\C^\star}
\newcommand{\K}{\mathbb{K}}
\newcommand{\Ke}{\K^\star}
\newcommand{\N}{\mathbb{N}}
\newcommand{\Ne}{\N^\star}
\newcommand{\Z}{\mathbb{Z}}
\newcommand{\Q}{\mathbb{Q}}
\newcommand{\A}{\mathbb{A}} % Nombres algébriques
\newcommand{\Premier}{\mathbb{P}} % Nombres premiers

\renewcommand{\d}{\, \mathrm{d}}

\newcommand{\norme}[1]{\Vert #1 \Vert}
%%% Ajout par Alain car definition manquante %%%
\newcommand{\Ninf}[1]{\Vert #1 \Vert_\infty}

% Complexes
% \DeclareMathOperator{\Reel}{\mathfrak{Re}}
% \DeclareMathOperator{\Imaginaire}{\mathfrak{Im}}
%%% Modifie par Alain car pb de compilation %%%
\DeclareMathOperator{\Reel}{\mathscr{Re}}
\DeclareMathOperator{\Imaginaire}{\mathscr{Im}}


\newcommand{\ptnclegras}[1]{\textbf{#1}}
\newcommand{\ptnclecadre}[1]{\boxed{#1}}

% Algèbre
% \newcommand{\Gl}{\mathscr{G}\kern-0.16em\ell}
\newcommand{\Gl}{\mathrm{GL}}
\newcommand{\I}{\mathrm{I}}
\newcommand{\Id}{\mathrm{Id}}
\newcommand{\M}{\mathscr{M}}
\newcommand{\Endo}{\mathscr{L}}
\newcommand{\Ortho}{\mathrm{O}}
\newcommand{\Sym}{\mathscr{S}}

% https://tex.stackexchange.com/questions/48/how-can-i-specify-a-long-list-of-math-operators
\newcommand{\DeclareMyOperator}[1]{%
  \expandafter\DeclareMathOperator\csname #1\endcsname{#1}
}
\newcommand{\DeclareMathOperators}{\forcsvlist{\DeclareMyOperator}}

%%% Modifie par Alain : suppression de Tr, det dans la liste car conflit %%%
\DeclareMathOperators{Rg,Ker,Vect,Sp,com,Diag,Mat} % conflit entre physics et Tr

% Probas
\newcommand{\E}{\mathbf{E}}
\newcommand{\V}{\mathbf{V}}
\renewcommand{\P}{\mathbf{P}}
\newcommand{\indicatrice}{\mathbf{1}}


%%%%%%%%%%%%%%%%%%%%%%%%%%%%%%%%%%%%%%%%%%%%%%%%%%%%%%%%%%%%%%%%%%%%%%%%%%%%%%%%%%%%%%%%%%%%%%%%%%%
%https://tex.stackexchange.com/questions/4216/how-to-typeset-correctly
% :=
%\newcommand*{\defeq}{\mathrel{\vcenter{\baselineskip0.5ex \lineskiplimit0pt
%                     \hbox{\scriptsize.}\hbox{\scriptsize.}}}                 =}
% =:
%\newcommand*{\defeqright}{=\mathrel{\vcenter{\baselineskip0.5ex \lineskiplimit0pt
%                     \hbox{\scriptsize.}\hbox{\scriptsize.}}}%
%                      }
\newcommand{\defeq}{\overset{\mathrm{\tiny def}}{=}}
\newcommand{\defeqright}{\defeq}
%%%%%%%%%%%%%%%%%%%%%%%%%%%%%%%%%%%%%%%%%%%%%%%%%%%%%%%%%%%%%%%%%%%%%%%%%%%%%%%%%%%%%%%%%%%%%%%%%%%

% https://tex.stackexchange.com/questions/349747/equivalence-of-sequence-sim-with-some-text-under      
\newcommand{\isEquivTo}[1]{%
  \mathpalette\isEquivToInner{#1}%
}
\newcommand{\isEquivToInner}[2]{%
  \ifx#1\displaystyle
    \underset{#2}{\sim}
  \else
    \sim_{#2}
  \fi
}


%%%%%%%%%%%%%%%%%%%%%%%%%%%%%%%%%%%%%%%%%%%%%%%%%%%%%%%%%%%%%%%%%%%%%%%%
\newcommand{\interoo}[2]{\left]#1\,;#2\right[}
\newcommand{\interff}[2]{\left[#1\,;#2\right]}
\newcommand{\interof}[2]{\left]#1\,;#2\right]}
\newcommand{\interfo}[2]{\left[#1\,;#2\right[}
\newcommand{\interent}[2]{\llbracket #1\,; #2 \rrbracket}

%\newcommand{\intervalle}[4]{%
%\mathopen{#1}#2\,;#3\mathclose{#4}}
%\newcommand{\intervalleff}[2]{%
%\intervalle{[}{#1}{#2}{]}}
%\newcommand{\intervallefo}[2]{%
%\intervalle{[}{#1}{#2}{[}}
%\newcommand{\intervalleof}[2]{%
%\intervalle{]}{#1}{#2}{]}}
%\newcommand{\intervalleoo}[2]{%
%\intervalle{]}{#1}{#2}{[}}
%%%%%%%%%%%%%%%%%%%%%%%%%%%%%%%%%%%%%%%%%%%%%%%%%%%%%%%%%%%%%%%%%%%%%%%%


\newcommand{\Trsp}[1]{#1^\top}

\newcommand{\Inv}[1]{#1^{-1}}

\newcommand{\fact}[1]{#1\,!}

\newcommand{\chevron}[1]{\say{\,#1\,}}

\newcommand{\note}{\hbox{\scriptsize $\blacklozenge$\ }}

% A REVOIR %%%%%%%%%%%%%%%%%%%%%%%%%%%%%%%%%%%%%%%%%
\newcommand{\suite}[3]{(#1_#2)_{#3}}

% \newcommand{\somme}[2][x]{%#1_1+\cdots+#1_#2}

\newcommand{\convolution}[2]{#1 \ast #2}

\newcommand{\conjugue}[1]{\overline{#1}}

\newcommand{\module}[1]{\left| #1 \right|}

\DeclarePairedDelimiterX\ens[1]\lbrace\rbrace{\def\tq{\;\delimsize\vert\;}#1} % https://tex.stackexchange.com/questions/253077/how-do-you-create-a-set-in-latex

\newcommand{\fonctionligne}[3][f]{#1 \colon #2 \mapsto #3}
\newcommand{\fonctionens}[3][f]{#1 \colon #2 \rightarrow #3}

\newcommand{\fonction}[5][f]{
    #1 \colon \begin{array}{>{\displaystyle}r @{} >{{}}c<{{}} @{} >{\displaystyle}l} 
          #2 &\longrightarrow& #3 \\[\medskipamount]
          #4 &\longmapsto& #5 
         \end{array}
} 

%%%%%%%%%%%%%%%%%%%%%%%%%%%%%%%
%%%%%%%%%%%%%%%%%%%%%%%%%%%
%%%%%%%%%%%%%%%%%%%%%%%%
% \langle \rangle

\newcommand*{\limite}[4][]{\displaystyle\lim_{\substack{#2\to#3\\#1}}#4}

\newcommand{\faibletoile}[1]{
    \xrightharpoonup[#1]{}^\star
}

% Suites remarquables
\newcommand{\Leg}{\mathrm{L}}
\newcommand{\Lag}{\mathrm{L}}
\newcommand{\Hilb}{\mathrm{H}}
\newcommand{\Hermite}{\mathrm{H}}
\newcommand{\Bern}{\mathrm{B}}
\newcommand{\Tcheby}{\mathrm{T}}
\newcommand{\Wallis}{\mathrm{W}}
\newcommand{\Cauchy}{\mathrm{C}}
\newcommand{\Gram}{\mathrm{G}}
\newcommand{\Bernstein}{\mathrm{B}}
\newcommand{\Vandermonde}{\mathrm{V}}
\newcommand{\Harmonique}{\mathrm{H}}


\newcommand{\scnums}[1]{\ifmmode\mathsmaller{\newstylenums{#1}}\else\textsmaller{\newstylenums{#1}}\fi} % https://comp.text.tex.narkive.com/idxlcygk/numerals-in-small-caps-font

\newcommand{\definir}[1]{\emph{#1}}


% \sectionfont{\color{BlueViolet}\sffamily}
% \subsectionfont{\color{BrickRed}\sffamily}
% \subsubsectionfont{\sffamily}

\renewcommand{\bfdefault}{b} % https://tex.stackexchange.com/questions/27843/level-of-boldness-changeable

\renewcommand{\sfdefault}{xcmss} % https://ctan.tetaneutral.net/fonts/sansmathfonts/sansmathfonts.pdf

%%%%%%%%%%%%%%%%%%%%%%%%%%%%%%%%%%%%%%%%%%%%%%%%%%%%%%%%%%%%%%
% https://tex.stackexchange.com/questions/88281/how-to-change-font-for-the-integral-symbol
% \makeatletter
% \def\upintkern@{\mkern-7mu\mathchoice{\mkern-3.5mu}{}{}{}}
% \def\upintdots@{\mathchoice{\mkern-4mu\@cdots\mkern-4mu}%
%  {{\cdotp}\mkern1.5mu{\cdotp}\mkern1.5mu{\cdotp}}%
%  {{\cdotp}\mkern1mu{\cdotp}\mkern1mu{\cdotp}}%
%  {{\cdotp}\mkern1mu{\cdotp}\mkern1mu{\cdotp}}}
% \newcommand{\upiint}{\DOTSI\protect\UpMultiIntegral{2}}
% \newcommand{\upiiint}{\DOTSI\protect\UpMultiIntegral{3}}
% \newcommand{\upiiiint}{\DOTSI\protect\UpMultiIntegral{4}}
% \newcommand{\upidotsint}{\DOTSI\protect\UpMultiIntegral{0}}
% \newcommand{\UpMultiIntegral}[1]{%
%   \edef\ints@c{\noexpand\upintop
%     \ifnum#1=\z@\noexpand\upintdots@\else\noexpand\upintkern@\fi
%     \ifnum#1>\tw@\noexpand\upintop\noexpand\upintkern@\fi
%     \ifnum#1>\thr@@\noexpand\upintop\noexpand\upintkern@\fi
%     \noexpand\upintop
%     \noexpand\ilimits@
%   }%
%   \futurelet\@let@token\ints@a
% }
% \makeatother

% \DeclareFontFamily{OMX}{mdbch}{}
% \DeclareFontShape{OMX}{mdbch}{m}{n}{ <->s * [0.8]  mdbchr7v }{}
% \DeclareFontShape{OMX}{mdbch}{b}{n}{ <->s * [0.8]  mdbchb7v }{}
% \DeclareFontShape{OMX}{mdbch}{bx}{n}{<->ssub * mdbch/b/n}{}

% \DeclareSymbolFont{uplargesymbols}{OMX}{mdbch}{m}{n}
% \SetSymbolFont{uplargesymbols}{bold}{OMX}{mdbch}{b}{n}
% \DeclareMathSymbol{\upintop}{\mathop}{uplargesymbols}{82}

% \makeatletter
% \newcommand{\upint}{\DOTSI\upintop\ilimits@}
% \makeatother

% \let\int\upint
% \let\idotsint\upidotsint
%%%%%%%%%%%%%%%%%%%%%%%%%%%%%%%%%%%%%%%%%%%%%%%%%%%%%%%%%%%%%

% https://tex.stackexchange.com/questions/124049/applying-options-to-already-loaded-package
\usepackage[framemethod=TikZ]{mdframed}
\usetikzlibrary{shadows}
\usepackage{amsthm}
\usepackage{thmtools}
\usepackage{xcolor}

\newcommand{\declaretheoremstylewithcolor}[3]{
  \declaretheoremstyle[
      numberwithin=section,
      headfont=\bfseries\color{#2},
      postheadhook=\leavevmode,
      notefont=\normalfont\bfseries,
      headformat=\NAME~\newstylenums{\NUMBER}\NOTE,
      bodyfont=#3,
      mdframed={%
          backgroundcolor=#2!5!white,
          linecolor=#2,
          linewidth=1pt,
          topline=false,
          bottomline=false,
          roundcorner=3pt,
          skipabove=5pt,
          innertopmargin=3pt,
          innerbottommargin=5pt
      }
  ]{#1} 
}

\declaretheoremstylewithcolor{styleMahogany}{Mahogany}{\normalfont}
\declaretheoremstylewithcolor{stylePeriwinkle}{Periwinkle}{\itshape}
\declaretheoremstylewithcolor{styleMulberry}{Mulberry}{\normalfont}
\declaretheoremstylewithcolor{styleJungleGreen}{JungleGreen}{\normalfont}
\declaretheoremstylewithcolor{styledarkgray}{darkgray}{\normalfont}

\declaretheorem[style=styleMahogany,    name=Définition]{defi}
\declaretheorem[style=styleMahogany,    name=Définitions]{defns}
\declaretheorem[style=stylePeriwinkle,  name=Théorème]{theo}
\declaretheorem[style=stylePeriwinkle,  name=Lemme]{lemme}
\declaretheorem[style=stylePeriwinkle,  name=Corollaire]{corol}
\declaretheorem[style=stylePeriwinkle,  name=Proposition]{prop}
\declaretheorem[style=styleMulberry,    name=Exercice]{exercice}
\declaretheorem[style=styleJungleGreen, name=Remarque]{remarque}
\declaretheorem[style=styleJungleGreen, name=Remarques]{remarques}
\declaretheorem[style=styledarkgray,    name=Méthode]{methode}

\declaretheoremstyle[
    numberwithin=section,
    headfont=\bfseries,%\scshape,
    notefont=\normalfont\bfseries,%\itshape, 
    headformat=\NAME~\newstylenums{\NUMBER}\NOTE,
    postheadhook=\leavevmode,
    mdframed={%
        topline=false,
        linewidth=1pt,
        %leftline=false,
        rightline=false,
        bottomline=false,
        % roundcorner=2pt,
        % splittopskip=20pt, 
        skipabove = 5pt, % to adjust the above skip
        innertopmargin=0pt,
        innerbottommargin=0pt
        }
]{styledefault}

\declaretheoremstyle[
    % spaceabove=6pt, spacebelow=6pt,
    headfont=\normalfont\bfseries,%\scshape,
    notefont=\normalfont\bfseries,%\itshape, 
    notebraces={(}{)},
    bodyfont=\normalfont,
    postheadhook=\leavevmode
    % postheadspace=1em
]{styledemosoluex}

\declaretheoremstyle[
    headformat=\!\!\NOTE,
    notefont=\normalfont\bfseries,%\scshape, 
    notebraces={}{},
    bodyfont=\normalfont,
    postheadhook=\leavevmode
    % postheadspace=1em
]{styleenvide}

\declaretheorem[
    style=styledemosoluex,
    qed=\qedsymbol,
    name=$\blacksquare$\ Démonstration,
    numbered=no
]{demo}

\declaretheorem[
    style=styledemosoluex,
    qed=\qedsymbol,
    name=$\blacksquare$\ Éléments de démonstration,
    numbered=no
]{elemdemo}

\declaretheorem[
    style=styledemosoluex,
    qed=$\lhd$,
    name=$\blacktriangleright$\ Solution,
    numbered=no
]{solution}

\declaretheorem[
    style=styledemosoluex,
    qed=$\lhd$,
    name=$\blacktriangleright$\ Éléments de solution,
    numbered=no
]{elemsolution}
 
\declaretheorem[
    style=styledemosoluex,
    qed=$\Diamond$,
    name=$\blacklozenge$\ Exemple,
    numbered=no
]{exemple}

\declaretheorem[
    style=styledemosoluex,
    qed=$\Diamond$,
    name=$\blacklozenge$\ Exemples,
    numbered=no
]{exemples}

\declaretheorem[
    style=styledemosoluex,
    qed=$\Diamond$,
    name=$\blacklozenge$\ Contre-exemple,
    numbered=no
]{contreexemple}

\declaretheorem[
    style=styledemosoluex,
    qed=$\Diamond$,
    name=$\blacklozenge$\ Contre-exemples,
    numbered=no
]{contreexemples}

\declaretheorem[
    style=styleenvide,
    numbered=no,
]{envide}


% Load the package for hyperreferences
\usepackage{kaorefs}

% \makeindex[columns=3, title=Alphabetical Index, intoc] % Make LaTeX produce the files required to compile the index

% \makeglossaries % Make LaTeX produce the files required to compile the glossary
% \input{glossary.tex} % Include the glossary definitions

% \makenomenclature % Make LaTeX produce the files required to compile the nomenclature

% Reset sidenote counter at chapters
%\counterwithin*{sidenote}{chapter}

%%% Ajoute par Alain
\usepackage{color}
\usepackage{todonotes}
\newcommand{\todoinline}[1]{\todo[color=green!40, bordercolor=red, inline]{#1}}

\newcommand{\entiers}[2]{\llbracket#1,#2\rrbracket}
\newcommand{\todoarmand}[1]{\todo[color=blue!20, bordercolor=red, inline]{#1}}

\usepackage{mathrsfs}


% \usepgfplotslibrary{external} 
% \tikzexternalize



%----------------------------------------------------------------------------------------
\begin{document}

%----------------------------------------------------------------------------------------
%	BOOK INFORMATION
%----------------------------------------------------------------------------------------

\title[Recueil des Exercices et Résultats Classiques de Mathématiques]{
Recueil d'Exercices et Résultats Classiques de Mathématiques
}

\author[Armand Wayoff]{Armand \textsc{Wayoff} \\ \url{https://armandwayoff.github.io/}}

\date{}

% \publishers{}

%----------------------------------------------------------------------------------------

\frontmatter % Denotes the start of the pre-document content, uses roman numerals

%----------------------------------------------------------------------------------------
%	OPENING PAGE
%----------------------------------------------------------------------------------------

\maketitle


\mainmatter % Indique le début du contenu principal du document, réinitialise la numérotation des pages et utilise les chiffres arabes.
\setchapterstyle{kao} % Choose the default chapter heading style


\section{Remarques générales}

% \todoinline{
% * Prévoir un truc (un lien hypertexte pour le pdf, une façon de numéroter qui relie les exercices) qui permet de relier des exercices entre eux, par ex. Hölder dans un chapitre convexité et les $L^p$ ?\\

% * Prévoir de mettre des références (livres, articles ?) par chapitre ? Par section ?\\

% * Prévoir une annexe avec les théorèmes utilisés ?\\

% * Niveau présentation, je pense qu'il faut choisir entre des propriétés énoncées puis démontrées ou des exercices. Je pencherais pour des petites propriétés démontrées pas à pas, comme le serait un problème et, éventuellement, à la fin, un exercice d'application (non corrigé ?)\\
% }

\todoinline{Pour mémoire, les environnements utilisés :
"demo", "elemdemo" et "elemsolution"\\
Titre optionnel entre crochets à tous les environnements: théorème, proposition, ..., exercice, remarque, démonstration, ...\\
environnements "questions" et "reponses" à la place des simples "enumerate".}

\todoarmand{J'ai trouvé ce livre de ECS \url{https://excerpts.numilog.com/books/9782340005853.pdf} qui propose 17 thèmes classiques qui recouvrent l'ensemble de leur programme}

\todoinline{
* Reprendre les $\ell^p$ pour éviter les redites avec les $L^p$.\\

* Pour les $\ell^p(\K^n)$, lorsque $n = 2$, on peut illustrer la forme des différentes boules. J'ai une animation sur ce sujet qui traîne. \\
}

\todoarmand{
Ajouter les numéros de page aux liens hypertextes
}

\pagelayout{wide}
\addpart{Analyse}
\pagelayout{margin}

\setchapterpreamble[u]{\margintoc}
\chapter{Intégration}
\labch{integration}

\todoinline{
Remarques générales :
}
\todoarmand{Les trois dernières animations du chapitre "Intégration sur un intervalle quelconque" sur votre site ne sont pas accessibles, est-ce normal ?}

\todoarmand{
Inclure le flow chart \url{https://acamanes.github.io/psi/psi_doc/fc03.pdf} ? \\
J'ai vu sur votre site que vous aviez fait plein de diagrammes pour les ECT. On pourrait inclure ceux sur l'intégration ?
}

\section{Une propriété géométique de l'intégrale}

\begin{exercice}
    Soit $f$ de classe $\mathscr{C}^1$ sur $[a, b]$ telle que $f'$ soit strictement positive sur $[a, b]$. Calculer:
    $$\int_{a}^{b} f(t) \d t + \int_{f(a)}^{f(b)} f^{-1}(t) \d t.$$
\end{exercice}

\begin{elem_sol}
    \begin{enumerate}
    \item Comme $f' > 0$, alors $f$ est strictement croissante. Comme $f$ est continue, d'après le théorème de la bijection monotone, $f$ réalise une bijection de $[a, b]$ sur $[f(a), f(b)]$.
    
    \item On utilise le changement de variable $\phi : [a, b] \to [f(a), f(b)],\, u \mapsto f(u)$. Alors, $\phi$ est de classe $\mathscr{C}^1$ et
    \begin{align*}
    \int_{f(a)}^{f(b)} f^{-1}(t) \d t
    &= \int_a^b f^{-1}(f(u)) f'(u) \d u\\
    &= \int_a^b u f'(u) \d u\\
    &= \left[u f(u)\right]_a^b - \int_a^b f(u) \d u,
    \end{align*}
    où on a réalisé une intégration par parties.
    \end{enumerate}

    Finalement,
    \[
    \int_a^b f(t) \d t + \int_{f(a)}^{f(b)} f^{-1}(t) \d t = b f(b) - a f(a).
    \]
\end{elem_sol}



% \todoinline{Une version plus claire de mon dessin ?}

% \includegraphics{./chapitres/integration/documents/propriete_geometrique.jpg}

\todoinline{J'ai modifié la position (et l'orientation!) de certaines écritures et j'ai centré la figure. Si ça te va, supprime cette remarque !}

\begin{center}
\begin{tikzpicture}[line cap=round, >=latex]

\begin{scope}[local bounding box=struct, scale=1]

    \def\u{0.5}
    \def\v{0.2}

    \def\xa{5}
    \def\ya{0.5}
    \def\xb{6.5}
    \def\yb{1.5}
    \def\xc{8}
    \def\yc{4}

    \path (\xa, \ya)    coordinate (A)
        (\xb, \yb)    coordinate (B)
        (\xc, \yc)  coordinate (C)
        (\ya, \xa)    coordinate (A')
        (\yb, \xb)    coordinate (B')
        (\yc, \xc)  coordinate (C');

    \begin{scope}
        \clip (A') ..controls +(0.05*\u, 0.05*\u) and ( $(B') + (-5*\v, -0.5*\u)$ )..
        (B') ..controls +(5*\v, 0.5*\u) and ( $(C') + (-2*\v, -2*\v)$ )..
        (C') |- cycle;
        %\clip (A') ..controls +(0.05*\u, 0.05*\u) and ( $(B') + (-0.5*\v, -0.5*\u)$ )..
        %(B') ..controls +(5*\v, 0.5*\u) and ( $(C') + (-2*\v, -2*\v)$ )..
        %(C') |- cycle;
        \foreach \x in {\ya00, \ya01,...,\yc.000}   
            \draw[blue!15!white, opacity=0.5] (\x,0) -- ++(0,8);
    \end{scope}

    \fill[blue!15!white] (\ya,0) rectangle ++(\yc-\ya,\xa);

    \begin{scope}
        \clip (A) ..controls +(0.05*\u, 0.05*\u) and ( $(B) + (-0.5*\u, -5*\v)$ )..
        (B) ..controls +(0.5*\u, 5*\v) and ( $(C) + (-2*\v, -2*\v)$ )..
            (C) |- cycle;
        \foreach \x in {\xa.000, \xa.001,...,\xc.000}   
            \draw[red!15!white, opacity=0.5] (\x,-\ya) -- ++(0,10);
    \end{scope}

    \draw[thick, blue] 
        (A') ..controls +(0.05*\u, 0.05*\u) and ( $(B') + (-5*\v, -0.5*\u)$ )..
        (B') ..controls +(5*\v, 0.5*\u) and ( $(C') + (-2*\v, -2*\v)$ )..
        (C') node[right] {$\mathcal{C}_f^{-1}$};
        
    \draw[thick, red] 
        (A) ..controls +(0.05*\u, 0.05*\u) and ( $(B) + (-0.5*\u, -5*\v)$ )..
        (B) ..controls +(0.5*\u, 5*\v) and ( $(C) + (-2*\v, -2*\v)$ )..
            (C) node[above] {$\mathcal{C}_f$};


    \fill[red!15!white] (\xa,0) rectangle ++(\xc-\xa,\ya);

    \draw[gray] (-.5, -.5) -- (\xc + 0.5, \xc.5) node[below left] {\contour{white}{\footnotesize$y = x$}};

    \draw[dashed] (\xa, 0) node[black,below]{\footnotesize$a$} -- (A);
    \draw[dashed] (0, \ya) node[black,left]{\footnotesize$f(a)$} -- (A);
    
    \draw[dashed] (\xc, 0) node[black,below]{\footnotesize$b$} -- (C);
    \draw[dashed] (0, \yc) node[black,left]{\footnotesize$f(b)$} -- (C);

    \draw[dashed] (0, \xa) node[black,left]{\footnotesize$a$} -- (A');
    \draw[dashed] (\ya, 0) node[black,below]{\footnotesize$f(a)$} -- (A');
    
    \draw[dashed] (0, \xc) node[black,left]{\footnotesize$b$} -- (C');
    \draw[dashed] (\yc, 0) node[black,below]{\footnotesize$f(b)$} -- (C');

    \draw[->, black] (5.5,2) node[above] 
    {\footnotesize \contour{white}{$\displaystyle \int_a^b f(t)\, \mathrm{d} t$}} to [out=-90,in=180] ($(7,1.5)$);

    % \draw[->, black] (3,-0.5) node[below] 
    % {\footnotesize \contour{white}{$\displaystyle \int_{f(a)}^{f(b)} f^{-1}(t)\, \mathrm{d} t$}} to [out=70,in=-70] ($(3,1.5)$);
    \node at ((2.5, 5) {\footnotesize \contour{blue!15!white}{$\displaystyle \int_{f(a)}^{f(b)} f^{-1}(t)\, \mathrm{d} t$}};

    \draw[thick, ->] (-.5, 0) -- (\xc + 0.5, 0) node[above] {$x$};
    \draw[thick, ->] (0, -.5) -- (0, \xc + 0.5) node[left] {$y$};

\end{scope}

\end{tikzpicture}
\end{center}


\section{Lemme de \textsc{Lebesgue}}

\todoinline{Parvenir à décommenter le marginnote suivant. D'après mes recherches sur internet, il faudrait "externaliser" la compilation des figures en pdflatex qui demandent trop de ressources.}

\begin{lemme}
% \marginnote[0cm]{Source : \cite{maths-france} Planche no 37. Intégration sur un segment}
Soit $a < b$.
\begin{enumerate}
\item On suppose que $f$ est une fonction de classe $\mathscr{C}^1$ sur $[a, b]$. Alors,
\[
\lim_{ \lambda \to +\infty} \int_a^b \sin(\lambda t) f(t) \d t = 0.
\]

\item Redémontrer le même résultat en supposant simplement que la fonction $f$ est continue par morceaux sur~$[a, b]$.
\end{enumerate}
\end{lemme}

\todoinline{
Ajout du graphique suivant, à supprimer ou améliorer en coloriant les aires positives et négatives ?
}

On constate sur la figure suivante que plus $\lambda$ est grand, plus les oscillations sont élevées et plus les aires comptées positivement et négativement se compensent.
\begin{center}
\includegraphics[width=0.75\textwidth]{illustrations/integration-02_lebesgue.png}
\end{center}

\begin{solution}
    \begin{enumerate}
        \item Puisque la fonction $f$ est de classe $\mathscr{C}^1$ sur $[a, b]$, on peut effectuer une intégration par parties qui fournit pour $\lambda > 0$:
        $$\left| \int_a^b f(t) \sin(\lambda t) \d t \right| = \left| \frac{1}{\lambda} \left( -\big[ \cos(\lambda t) f(t) \big]_a^b + \int_a^b f'(t) \cos(\lambda t) \d t  \right) \right| \leqslant \frac{1}{\lambda} \left( |f(a)| + |f(b)| + \int_a^b |f'(t)| \d t \right).$$
        Cette dernière expression tend vers $0$ quand $\lambda$ tend vers $+ \infty$, et donc $\int_a^b f(t) \sin(\lambda t) \d t$ tend vers $0$ quand $\lambda$ tend vers $+\infty$.
        \item Si la fonction $f$ est simplement supposée continue par morceaux, on ne peut donc plus effectuer une intégration par parties. \\
        Le résultat est clair si $f = 1$, car pour $\lambda > 0$, $\left| \int_a^b \sin(\lambda t) \d t \right| = \left| \frac{\cos(\lambda a) - \cos(\lambda b)}{\lambda} \right| \leqslant \frac{2}{\lambda}$. \\
        Le résultat s'étend aux fonctions constantes par linéarité de l'intégrale puis aux fonctions constantes par morceaux par additivité par rapport à l'intervalle d'intégration, c'est-à-dire aux fonctions en escaliers. Pour toute fonction $g$ continue par morceaux sur $[a, b]$, on note $\|g\|_{\infty} = \sup_{[a, b]} |g|$.\\
        Soit alors $f$ une fonction continue par morceaux sur $[a, b]$. \\
        \todoinline{Là on admet un théorème d'approximation non trivial et hors programme en PCSI. Il faut voir ce qu'on indique en introduction du chapitre ?}
        Soit $\varepsilon > 0$. Il existe une fonction en escalier $\varphi$ telle que $\|f - \varphi\|_\infty \leq \varepsilon$. De plus, d'après le point précédent, il existe un réel $\lambda_0$ strictement positif tel que pour tout $\lambda > \lambda_0$,
        \[
        \left|\int_a^b \sin(\lambda t) \varphi(t) \d t\right| \leq \varepsilon.
        \]
        Finalement, d'après l'inégalité triangulaire, pour tout $\lambda > \lambda_0$,
        \begin{align*}
        \left|\int_a^b f(t) \sin(\lambda t) \d t\right|
        &\leq         \left|\int_a^b (f(t) - \varphi(t)) \sin(\lambda t) \d t\right| + \left|\int_a^b \varphi(t) \sin(\lambda t) \d t\right|\\
        &\leq \norme{f - \varphi}_\infty (b - a) + \varepsilon\\
        &\leq \varepsilon (1 + b - a).
        \end{align*}
Finalement, $\lim\limits_{\lambda \to +\infty} \int_a^b f(t) \sin(\lambda t) \d t = 0$.
        \end{enumerate}
\end{solution}


% \todoinline{
% La variante proposée ci-dessous me semble difficile.
%
% Le calcul de $\sum \frac{1}{n^2}$ est classique et pourrait être directement généralisé avec St Cyr 1995 - Je mets une version dans le dossier /chapitres/integration/documents. En plus on ferait un peu de polynômes de Bernoulli.
%
% Le calcul de l'intégrale de Dirichlet est top.
% }

% \todoarmand{Cela me convient, nous pouvons supprimer la variante. On pourrait la remplacer par le  mais en renvoyant vers un exercice du chapitre polynômes.}

\bigskip


\todoinline{Ajouter des liens dans le texte ci-dessous.}

Nous allons utiliser le lemme de Lebesgue pour calculer certaines valeurs de la fonction $\zeta$ de Riemann. Nous verrons ultérieurement une autre utilisation au calcul de la valeur de l'intégrale de Dirichlet.

\todoinline{Partie re-rédigée, à relire !}

\todoinline{Citer sujet St Cyr 1995}

\todoinline{Faire référence à une section sur $\zeta$ et une section sur Bernoulli}

\todoinline{Dans la partie sur Bernoulli, il faudra :
la symétrie, la dérivée et les valeurs de $B_2(0)$ et $B_4(0)$}

On note $(B_n)$ la suite des polynômes de Bernoulli et, pour tout $x > 1$, $\zeta(x) = \sum\limits_{k=1}^{+\infty} \frac{1}{k^x}$.

\begin{theo}
Pour tout $m \geq 1$, $\zeta(2m) = (-1)^{m-1} (2 \pi)^{2m} \frac{B_{2m}(0)}{2}$.
\end{theo}

On rappelle que 
\begin{align*}
B_n(1 - x) &= (-1)^n B_n(x),\\
B_n'(x) &= n B_{n-1}(x).
\end{align*}

\begin{elem_sol}
Pour $k$ et $n$ entiers strictement positifs, on défnit:
\[
I_{n, k} = \int_0^1 B_n(t) \cos(2 k \pi t) \d t.
\]

\begin{enumerate}
\item Pour tout entier $p > 0$,
\[
I_{2p, k} = \frac{(-1)^{p-1}}{(2 k \pi)^{2p}} \quad \text{et} \quad I_{2p+1,k} = 0.
\]

En effet, en utilisant deux intégrations par parties successives,
\[
I_{n,k} = \frac{1}{4k^2 \pi^2} \big(B_{n-1}(1) - B_{n-1}(0) - I_{n-2, k} \big).
\]
De plus, $I_{0,k} = 0$, $I_{1,k} = 0$, $I_{2,k} = \frac{1}{4 \pi^2}$? Donc,
\[
\forall n \geqslant 3,\, I_{n,k} = - \frac{1}{4 k^2 \pi^2}I_{n-2, k}.
\]
On obtient ainsi le résultat annoncé
\end{enumerate}

Pour tout entier naturel $n$ strictement positif, on pose:
\[
\forall t \in \interoo{0}{1}, \quad \varphi_n(t) = \frac{B_n(t) - B_n(0)}{\sin(\pi t)}.
\]

\begin{enumerate}[resume]
\item Pour tout $n \geq 2$, la fonction $\varphi_n$ est prolongeable par continuité à $\interff{0}{1}$ et que le prolongement est de classe $\mathscr{C}^1$. En effet,

\begin{itemize}
\item D'après les quotients de fonctions de classe $\mathscr{C}^1$ dont le dénominateur ne s'annule pas, la fonction $\varphi_n$ est de classe $\mathscr{C}^1$ sur $\interoo{0}{1}$.

\item La fonction $B_n$ étant polynomiale, elle est de classe $\mathscr{C}^1$ en $0$ et, en utilisant la formule de \textsc{Taylor}--\textsc{Young},
\begin{align*}
\varphi_n(t) &= \frac{B'_n(0)t + \frac{B''_n(0)}{2}t^2 + o(t^2)}{\pi t \big(1 + o(t) \big)} \\
&= \frac{B'_n(0)}{\pi} + \frac{B''_n(0)}{2 \pi}t + o(t).
\end{align*}

Ainsi, $\lim\limits_0 \varphi_n = \frac{B'_n(0)}{\pi}$ et $\varphi_n$ est une fonction prolongeable par continuité en $0$.

\item
% De plus, $\lim\limits_{t \to 0} \frac{\varphi_n(t) - \frac{B'_n(0)}{\pi}}{t} = \frac{B''_n(0)}{2 \pi}$.
%
De plus, pour tout $t$ non nul, $\varphi'_n(t) = \frac{B'_n(t) \sin(\pi t) - \big(B_n(t) - B_n(0) \big) \pi \cos(\pi t)}{\sin(\pi t)^2}$. Ainsi, en effectuant un développement limité à l'ordre $2$ du numérateur, alors $\lim\limits_{t \to 0} \varphi'_n(t) = \frac{1}{2 \pi} B''_n(0)$.

D'après le théorème de prolongement des dérivées, $\varphi_n$ est prolongeable en une fonction de classe $\mathscr{C}^1$ sur $\interfo{0}{1}$.

Enfin, $\varphi_n(1-t) = (-1)^n \frac{B_n(t) - B_n(1)}{\sin(\pi t)}$. Comme, pour tout $n \geqslant 2$, $B_n(0) = B_n(1)$, alors la fonction $\varphi_n$ est bien prolongeable en une fonction de classe $\mathscr{C}^1$ sur $\interff{0}{1}$.
\end{itemize}
\end{enumerate}

Pour tout $N$ entier naturel non nul et $t \in \interoo{0}{1}$, on pose :
\[
D_n(t) = 1 + 2 \sum_{k=1}^N \cos(2k \pi t) = \frac{\sin\big((2N+1) \pi t \big)}{\sin(\pi t)}.
\]

\begin{enumerate}[resume]
\item Cette quantité, appelée noyau de Dirichlet, s'exprime simplement à l'aide de la fonction sinus :
\[
D_n(t) = \frac{\sin\big((2N+1) \pi t \big)}{\sin(\pi t)}.
\]
En effet, pour tout $t \in \interoo{0}{1}$, $\e^{2 \i k \pi t} \not= 1$. Ainsi, d'après la somme des termes d'une suite géométrique, 
    \begin{align*}
        \sum_{k=0}^N \e^{2 \i k \pi t} &= \frac{\e^{2 \i (N+1) \pi} - 1}{\e^{2 \i \pi} - 1} \\
        &= \e^{\i N \pi} \frac{\sin(N+1) \pi t}{\sin(\pi t)}. \\
        \sum_{k=0}^N \cos(2 k \pi t) &= \cos(N \pi t) \frac{\sin \big((N+1) \pi t \big)}{\sin(\pi t)}\text{, en prenant les parties réelles,} \\
        1 + 2 \sum_{k=1}^N \cos(2 k \pi t) &= 2 \frac{\cos(N \pi t) \sin \big((N+1) \pi t\big)}{\sin(\pi t)} - 1 \\
        &= \frac{\sin\big((2N+1) \pi t \big) + \sin( \pi t)}{\sin(\pi t)} - 1 \\
        &= \frac{\sin\big((2N+1) \pi t \big)}{\sin(\pi t)}.
    \end{align*}

\item Ainsi,
\[
\int_0^1 \varphi_{2m}(t) \sin \big((2N+1) \pi t \big) \d t
= - B_{2m}(0) + 2 \sum_{k=1}^N \frac{(-1)^{m-1}}{(2 k \pi)^{2m}}.
\]

En effet, d'après la définition de $\varphi_{2m}$,
\begin{align*}
\int_0^1 \varphi_{2m}(t) \sin \big((2N+1) \pi t \big) \d t &= \int_0^1 \big(B_{2m}(t) - B_{2m}(0) \big) \frac{\sin\big((2N+1) \pi t \big)}{\sin(\pi t)} \d t \\
&= \int_0^1 \big( B_{2m}(t) - B_{2m}(0) \big) \d t + \cdots \\
&\cdots + 2 \sum_{k=1}^N \int_0^1 \big(B_{2m}(t) - B_{2m}(0) \big) \cos(2 k \pi t) \d t \\
&= - B_{2m}(0) + 2 \sum_{k=1}^N \frac{(-1)^{m-1}}{(2 k \pi)^{2m}}.
\end{align*}

\item D'après le lemme de Lebesgue,
\[
\lim_{N\to+\infty} \int_0^1 \varphi_{2m}(t) \sin \big((2N+1) \pi t \big) \d t = 0
\]
et on obtient bien le résultat annoncé.
\end{enumerate}
\end{elem_sol}


\begin{remarque}
En utilisant les valeurs remarquables des premiers polynômes de Bernoulli, on obtient
\begin{align*}
\sum_{k=1}^{+\infty} \frac{1}{k^2} &= 2 \pi^2 B_2(0) = \frac{\pi^2}{6} \\
\sum_{k=1}^{+\infty} \frac{1}{k^4} &= -2^3 \pi^4 B_4(0) = \frac{\pi^4}{90}.
\end{align*}
\end{remarque}

\section{Calculs approchés d'intégrales}


\todoinline{Texte réécrit, à relire avec attention en raison des copier-coller...}

\todoinline{OK, je relirai également. Je viens de penser aussi à  Centrale - PC - 2021 qui fait les méthodes de quadrature.}

\todoarmand{Le II.C fait un lien avec la section 5.5 sur les familles de polynômes orthogonaux. \\
Le III permet d'avoir une application des polynômes / nombres de Bernoulli}

Nous abordons dans cette section quelques méthodes dont le but est d’estimer la valeur numérique de l’intégrale d'une fonction donnée $f$ définie sur un domaine borné $\interff{a}{b}$:
\[
I = \int_a^b f(t) \d t.
\]
Ces méthodes nous fourniront une valeur approchée $\widetilde{I}$ de l'intégrale $I$ de sorte que 
\[
\widetilde{I} \approx I.
\]

Soient $a$ et $b$ désignent deux réels tels que $a < b$. Pour tout entier naturel $p$ non nul, on note $(x_i)_{i\in\llbracket 0, p \rrbracket}$ la subdivision régulière de $[a, b]$ de pas $\frac{b-a}{p}$. Ainsi, pour tout $i \in \llbracket 0, p \rrbracket$,
\[
x_i = a + i \frac{b-a}{p}.
\]

\todoinline{Doit-on définir la notion d'"exacte" dans la définition suivante ?}

\todoarmand{C'est peut être mieux, j'ai fait une proposition qui permet de ne pas trop rallonger la définition mais qui demande d'avoir défini $I$ et $\widetilde{I}$ avant.}

\begin{defi}{}
Une méthode d'intégration est d'\emph{ordre} au moins $n$ si elle est exacte (\emph{i.e.} $\widetilde{I} = I$) pour les polynômes de degrés inférieurs ou égaux $n$ et non exacte pour au moins un polynôme de degré $n+1$.
\end{defi}

%-----------
\subsection{Méthode des rectangles à gauche}

La méthode des rectangles à gauche consiste, pour chacun des intervalles de la subdivision, à approcher l'aire sous la courbe représentative de $f$ par celle d'un rectangle dont la hauteur correspond à la valeur de $f$ sur la borne de gauche. Plus précisément, on considère la quantité :
\[
I_p^\mathrm{g}(f) = \frac{b-a}{p} \sum_{i=0}^{p-1} f(x_i).
\]

\begin{marginfigure}[0cm]
    \centering
    %https://tex.stackexchange.com/questions/476702/riemann-sum-approaches-area-under-curve

\begin{tikzpicture}[scale=0.8,declare function={f(\x)=((1/3)*(\x)^(3)-3*(\x)^(2)+8*\x-3;}]
\coordinate (start) at (.8,{f(.8)});
\coordinate (x0) at (1,{f(1)});
\coordinate (x1) at (2,{f(2)});
\coordinate (x2) at (3,{f(3)});
\coordinate (x3) at (4,{f(4)});
\coordinate (x4) at (5,{f(5)});
\coordinate (end) at (5.05,{f(5.05)});
\draw[fill=teal!20!white] (1,0) rectangle (2,{f(1)});
\draw[fill=teal!20!white] (2,0) rectangle (3,{f(2)});
\draw[fill=teal!20!white] (3,0) rectangle (4,{f(3)});
\draw[fill=teal!20!white] (4,0) rectangle (5,{f(4)});
%\draw (5,0)--(5,{f(5)});
\draw [-latex] (-0.5,0) -- (6,0) node (xaxis) [below] {$x$};
\draw [-latex] (0,-0.5) -- (0,5) node [left] {$f(x)$};
\foreach \x/\xtext in {1/a=x_0 ,2/x_1, 3/x_2 , 4/x_3 , 5/b=x_4}
 \draw[xshift=\x cm] node[below=2pt,fill=white,font=\small, anchor=south, yshift=-5mm] {$\xtext$};
\draw[domain=.5:5.3,samples=200,variable=\x,blue,thick] plot ({\x},{f(\x)});                 
\foreach \n in {0,1,2,3}
\draw[blue,fill=blue] (x\n) circle (2pt) node[font=\normalsize] {$ $};    
\draw[latex-latex] (2,1)--(3,1) node[midway, anchor=south] {$\frac{b-a}{p}$};      
\end{tikzpicture}
    \caption{Illustration de la méthode des rectangles à gauche}
\end{marginfigure}
\marginnote[6cm]{Animation de la méthode des rectangles à gauche : \url{https://acamanes.github.io/psi/psi_doc/animations/integration_segment/01-methode_des_rectangles_a_gauche.mp4}}

\begin{prop}{}{}
La méthode des rectangles à gauche est d'ordre $0$. De plus, si $f$ est de classe $\mathscr{C}^1$, l'erreur commise est en $O(1/p)$.
\end{prop}

\begin{elem_sol}
On suppose que $f$ désigne une fonction de classe $\mathscr{C}^1$ sur le segment $\interff{a}{b}$. On note $F$ une primitive de $f$ et $M_1 = \sup_{\interff{a}{b}} \module{f'}$.

\begin{enumerate}
\item D'après la formule de \textsc{Taylor} avec reste intégal, pour tout $i \in \interent{0}{p-1}$,
\[
\module{F(x_{i+1}) - F(x_i) - (x_{i+1} - x_i) F'(x_i)} \leq \frac{M_1}{2} (x_{i+1}-x_i)^2.
\]

\item En utilisant la relation de \textsc{Chasles},
\begin{align*}
\module{\int_{\interff{a}{b}} f - I_p^\mathrm{g}(f)}
&\leq \sum_{i=1}^{p-1} \module{\int_{\interff{x_i}{x_{i+1}}} f - (x_{i+1} - x_i) f(x_i)}\\
&\leq \sum_{i=1}^{p-1} \frac{M_1}{2} (x_{i+1} - x_i)^2\\
% &\leq \frac{M_1}{2} (b - a) \sum_{i=1}^{p-1} (x_{i+1} - x_i)\\
&\leq \frac{M_1 (b-a)^2}{2 p}.
\end{align*}

\item De plus, on montre que cette borne est atteinte pour $f : x \mapsto x - a$.

\item La méthode des rectangles à gauche est exacte si $f$ est constante. Cependant, le calcul précédent montre que si $f : x \mapsto x - a$, alors la méthode ne donne pas la valeur exacte de l'intégrale. La méthode est donc d'ordre $0$.
\end{enumerate}
\end{elem_sol}

%-----------
\subsection{Méthode des rectangles médians}

La méthode des rectangles médians consiste, pour chacun des intervalles de la subdivision, à approcher l'aire sous la courbe représentative de $f$ par celle d'un rectangle dont la hauteur correspond à la valeur de $f$ au milieu de la subdivision. Plus précisément, on considère la quantité :
\[
I_p^\mathrm{m}(f) = \frac{b-a}{p} \sum_{i=0}^{p-1} f\left(\frac{x_i + x_{i+1}}{2} \right).
\]

\begin{prop}{}{}
La méthode des rectangles médians est d'ordre $1$. De plus, si $f$ est de classe $\mathscr{C}^2$, l'erreur commise est en $O(1/p^2)$.
\end{prop}

\begin{marginfigure}[0cm]
    \centering
    \input{illustrations/i_rectangles_medians}
    \caption{Illustration de la méthode des rectangles médians}
\end{marginfigure}
\marginnote[6cm]{Animation de la méthode des rectangles médians : \url{https://acamanes.github.io/psi/psi_doc/animations/integration_segment/02-methode_des_rectangles_medians.mp4}}


\begin{elem_sol}
On suppose que $f$ désigne une fonction de classe $\mathscr{C}^2$ sur le segment $\interff{a}{b}$. On note $F$ une primitive de $f$ et $M_2 = \sup_{\interff{a}{b}} \module{f''}$. Pour tout entier $i \in \interent{0}{p-1}$, on pose $\gamma_i = \frac{x_i + x_{i+1}}{2}$ le milieu du segment $\interff{x_i}{x_{i+1}}$.

\begin{enumerate}
\item Soit $i \in \interent{0}{p-1}$. Un simple calcul ou le graphique suivant permet de montrer que :
\[
 (x_{i+1} - x_i) f(\gamma_i) = \int_{x_i}^{x_{i+1}} \left(f(\gamma_i) + (t - \gamma_i) f'(\gamma_i) \right) \d t.
    \]

\begin{figure}
    \centering
    \begin{tikzpicture}[scale=0.8,declare function={f(\x)=((1/3)*(\x)^(3)-3*(\x)^(2)+8*\x-3;}, declare function={g(\x)=-\x+6;},]
\coordinate (xm) at (3,{f(3)});

\draw[fill=teal!20!white] (2,0) rectangle (4,{f(3)});

\draw[dashed] (3, 0) -- (3,{f(3)});
\draw[dashed] (2, {f(3)}) -- (2,{g(2)});

\draw [-latex] (-0.5,0) -- (6,0) node (xaxis) [below] {$t$};
\draw [-latex] (0,-0.5) -- (0,5);
% \foreach \x/\xtext in {1/a=x_0 ,2/x_1, 3/x_2 , 4/x_3 , 5/b=x_4}
\draw[xshift=2 cm] node[below=2pt,fill=white,font=\small, anchor=south, yshift=-5mm] {$x_i$};
\draw[xshift=3 cm] node[below=2pt,fill=white,font=\small, anchor=south, yshift=-5mm] {$\gamma_i$};
\draw[xshift=4 cm] node[below=2pt,fill=white,font=\small, anchor=south, yshift=-5mm] {$x_{i+1}$};
\draw[domain=.5:5.3,samples=200,variable=\x,blue,thick] plot ({\x},{f(\x)}) node[right] {$f(t)$};  
\draw[domain=1.5:4.5,samples=200,variable=\x,red,thick] plot ({\x},{g(\x)}) node[right] {$f(\gamma_i) + (t - \gamma_i) f'(\gamma_i)$}; 
% \foreach \n in {0,1,2,3}
\draw[blue,fill=blue] (xm) circle (2pt) node[font=\normalsize] {$ $};    
% \draw[latex-latex] (2,1)--(3,1) node[midway, anchor=south] {$\frac{b-a}{p}$};      
\end{tikzpicture}
\end{figure}
    

\item Ainsi, d'après la formule de \textsc{Taylor} avec reste intégral,
\begin{align*}
\module{F(x_{i+1}) - F(x_i) - (x_{i+1} - x_i) F'(\gamma_i)}
&= \module{\int_{x_i}^{x_{i+1}} \left(f(t) - f(\gamma_i) - (t - \gamma_i) f'(\gamma_i)\right) \d t}\\
&\leq \frac{M_2}{24} (x_{i+1} - x_i)^3.
\end{align*}

\item Comme pour la méthode des rectangles à gauche, la formule de \textsc{Chasles} permet de montrer que
\[
\module{\int_{[a,b]} f - I_p^\mathrm{m}(f)} \leq \frac{M_2 (b-a)^3}{24 p^2}.
\]

\item De plus, on montre que cette borne est atteinte pour $f : x \mapsto (x - a)^2$.

\item La méthode des rectangles médians est exacte si $f$ est un polynôme de degré $1$. Cependant, si $f$ est la fonction $x \mapsto (x - a)^2$, le calcul précédent montre que la méthode des rectangles médians ne donne pas la valeur exacte de l'intégrale. La méthode est donc d'ordre $1$.
\end{enumerate}
\end{elem_sol}

%-----------
\subsection{Méthode des trapèzes}

La méthode des trapèzes consiste, pour chacun des intervalles de la subdivision, à approcher l'aire sous la courbe représentative de $f$ par celle d'un trapèze. Plus précisément, on considère la quantité :
\[
I_p^\mathrm{t}(f) =  \frac{b-a}{p} \sum_{i=0}^{p-1} \frac{f(x_i) + f(x_{i+1})}{2}.
\]

\begin{marginfigure}[0cm]
    \centering
    \input{illustrations/i_trapezes}
    \caption{Illustration de la méthode des trapèzes}
\end{marginfigure}
\marginnote[6cm]{Animation de la méthode des trapèzes : \url{https://acamanes.github.io/psi/psi_doc/animations/integration_segment/03-methode_des_trapezes.mp4}}

\begin{prop}{}{}
La méthode des trapèzes est d'ordre $2$. De plus, si $f$ est de classe $\mathscr{C}^2$, l'erreur commise est en $O(1/p^2)$.
\end{prop}

\begin{elem_sol}
On suppose que $f$ est une fonction de classe $\mathscr{C}^2$ et on note $M_2 = \sup_{[a,b]} \module{f''}$. Pour tout $i \in \interent{0}{p-1}$, on note $\phi_i$ l'approximation affine sur $\interff{x_i}{x_{i+1}}$ de $f$ et $g_i = f - \phi_i$.

\begin{enumerate}
\item À l'aide de deux intégrations par parties successives, on montre que, pour tout $i \in \interent{0}{p-1}$,
\[
\int_{x_i}^{x_{i+1}} f''(t) (t - x_i) (x_{i+1} - t) \d t = - 2 \int_{x_i}^{x_{i+1}} g_i(t) \d t.
\]

\item D'après la relation précédente, on établit que
\begin{align*}
\module{\int_{x_i}^{x_{i+1}} f(t) \d t - I_p^\mathrm{t}(f)}
&\leq \int_{x_i}^{x_{i+1}} \module{f(t) - \phi_i(t)} \d t\\
&\leq \frac{M_2}{2} \int_{x_i}^{x_{i+1}} (t - x_i) (x_{i+1} - t) \d t\\
&\leq \frac{M_2}{2} \cdot \frac{(b - a)^3}{6}.
\end{align*}

\item Comme pour les méthodes précédentes, on utilise ensuite la relation de \textsc{Chasles} pour obtenir
\[
\module{\int_{[a,b]} f - I_p^\mathrm{t}(f)} \leq \frac{M_2 (b-a)^3}{12 p^2}.
\]

\item De plus, on montre que cette borne est atteinte pour $f : x \mapsto (x - a)^2$.

\item La méthode des trapèzes est exacte si $f$ est un polynôme de degré $1$. Cependant, si $f$ est la fonction $x \mapsto (x - a)^2$, le calcul précédent montre qur la méthode des trapèzes ne donne pas la valeur exacte de l'intégrale. La méthode est donc d'ordre $2$.
\end{enumerate}
\end{elem_sol}

\begin{remarque}
Lorsque $f$ est de classe $\mathscr{C}^2$ et convexe, alors $f'' \geq 0$ et, pour tout $p$ entier naturel, \mbox{$\int_{[a,b]} f \leq I_p^\mathrm{t}(f)$}. On obtient ainsi une valeur approchée par excès de l'intégrale.
\end{remarque}

\subsection{Méthode de \textsc{Simpson}}

La méthode de \textsc{Simpson} consiste, pour chacun des intervalles de la subdivision, à approcher la fonction $f$ par un polynôme de degré inférieur ou égal à $2$. Plus précisément, on considère la quantité :
\[
I_p^\mathrm{s}(f) = \frac{b-a}{6 p} \sum_{i=0}^{p-1} \left[f(x_i)+ 4 f\left(\frac{x_i + x_{i+1}}{2}\right) + f(x_{i+1})\right].
\]

\begin{prop}{}{}
Dans la méthode de Simpson, si $f$ est de classe $\mathscr{C}^4$, l'erreur commise est en $O(1/p^4)$.
\end{prop}

\begin{elem_sol}
On suppose que $f$ est une fonction de classe $\mathscr{C}^4$ sur le segment $[a, b]$. On pose $M_4 = \sup_{[a,b]} \module{f^{(4)}}$.

Pour tout $i \in \interent{0}{p-1}$, notons $\gamma_i = \frac{x_i + x_{i+1}}{2}$ le milieu de la subdivision et $h_i = \frac{x_{i+1} - x_i}{2}$ la moitié de sa longueur.
\begin{enumerate}
\item D'après la formule de Taylor avec reste intégral appliquée à une primitive $F$ de $f$,
\begin{align*}
F(\gamma_i + h_i)
&= \begin{aligned}[t]F(\gamma_i) + h_i f(\gamma_i) + \frac{h_i{}^2}{2} f'(\gamma_i) + \frac{h_i{}^3}{6} f''(\gamma_i) + \frac{h_i{}^4}{24} f'''(\gamma_i) \\ + \frac{h_i{}^5}{24} \int_0^1 (1 - t)^4 f^{(4)}(\gamma_i + t h_i) \d t,\\
\end{aligned} \\
f(\gamma_i - h_i)
&= \begin{aligned}[t]F(\gamma_i) - h_i f(\gamma_i) + \frac{h_i{}^2}{2} f'(\gamma_i) - \frac{h_i{}^3}{6} f''(\gamma_i) + \frac{h_i{}^4}{24} f'''(\gamma_i) \\ - \frac{h_i{}^5}{24} \int_0^1 (1 - t)^4 f^{(4)}(\gamma_i - t h_i) \d t.
\end{aligned}
\end{align*}

Comme $\gamma_i - h_i = x_i$ et $\gamma_i + h_i = x_{i+1}$, en soustrayant les deux relations précédentes, on obtient
\begin{align*}
F(x_{i+1}) - F(x_i)
&= 2 h_i f(\gamma_i) + \frac{h_i^3}{3} f''(\gamma_i) + \frac{h_i^5}{24} \int_0^1 (1 - t)^4 \left[f^{(4)}(\gamma_i + t h_i) + f^{(4)}(\gamma_i - t h_i)\right] \d t.
% \\
% \int_{x_i}^{x_{i+1}} f(t) \d t
% &= \frac{b - a}{p} f(\gamma_i) + \frac{(b - a)^3}{24 p^3} f''(\gamma_i) + \frac{(b - a)^5}{32 \times 24 p^5} \int_0^1 (1 - t)^4 \left[f^{(4)}(\gamma_i + t h_i) + f^{(4)}(\gamma_i - t h_i)\right] \d t.
\end{align*}

Ainsi,
\begin{align*}
&\module{\int_{x_i}^{x_{i+1}} f(t) \d t - \frac{h_i}{3} \left[f(x_i) + 4 f(\gamma_i) + f(x_{i+1})\right]}\\
&\leq \module{\frac{h_i}{3} \left[f(x_i) - 2 f(\gamma_i) + f(x_{i+1}) - h_i^2 f''(\gamma_i)\right]} + \frac{h_i^5 2 M_4}{5! p^5}.
\end{align*}

\item En utilisant la formule de Taylor avec reste intégral pour la fonction $f$ sur $[\gamma_i - h_i, \gamma_i]$ et $[\gamma_i, \gamma_i + h_i]$, on obtient comme précédemment
\[
\module{f(x_{i+1}) + f(x_i) - 2 f(\gamma_i) - h_i^2 f(\gamma_i)} \leq \frac{h_i^4 \times 2 M_4}{4!}.
\]

\item Finalement,
\begin{align*}
\module{\int_{x_i}^{x_{i+1}} f(t) \d t - \frac{h_i}{3} \left[f(x_i) + 4 f(\gamma_i) + f(x_{i+1})\right]}
&\leq h_i^5 \left[\frac{2}{3 \times 4!} + \frac{2}{5!}\right]\\
&\leq \frac{2 h_i^5}{45}
\leq \frac{(x_{i+1} - x_i)^5}{720}.
\end{align*}

\todoinline{Je crois qu'on peut atteindre un facteur $\frac{1}{2880}$ mais je ne parviens pas à faire mieux sans regarder en détail les restes intégraux !}

\item On conclut à l'aide de la relation de Chasles :
\[
\module{I_p^s(f) - \int_a^b f(t) \d t} \leq \frac{M_4 (b-a)^5}{720 p^4}.
\]
% \[
% \module{I_p^s(f) - \int_a^b f(t) \d t} \leq \frac{M_4 (b-a)^5}{2880 p^4}.
% \]
\end{enumerate}
\end{elem_sol}

%-----------
\subsection{Et ensuite ?}

Nous constatons que, pour chacune des méthodes précédentes, la stratégie est identique :
\begin{itemize}
\item découper le segment en une subdivision régulière $a = x_0 \leq \cdots \leq x_n = b$,

\item sur chacun des intervalles de cette subdivision, approcher la fonction par une fonction dont l'intégrale est plus simple.

Sur l'intervalle $[x_i, x_{i+1}]$ : pour la méthode des rectangles, on approche la fonction par une droite horizontale, pour celle des trapèzes, par une droite affine passant par les points $(x_i, f(x_i))$ et $(x_{i+1}, f(x_{i+1}))$.
\end{itemize}

Plus généralement, on peut découper le segment $[x_i, x_{i+1}]$ en une subdivision régulière $x_i = y_{i,0} \leq \ldots \leq y_{i,p} \leq x_{i+1}$. On peut ensuite approcher la fonction par le polynôme d'interpolation de Lagrange qui passe par les points de coordonnées $(y_{i,0}, f(y_{i,0}), \ldots, (y_{i,p}, f(y_{i,p}))$.

Cette méthode est appelée \emph{méthode de \textsc{Newton}--\textsc{Cotes}}.

Plus précisément, on considère une subdivision $0 = y_0 \leq \ldots \leq y_p = 1$ de l'intervalle $\interff{0}{1}$ et on note $(L_0,\ldots,L_p)$ la famille des \hyperref[sec:polynomes_de_lagrange]{polynômes d'interpolation de \textsc{Lagrange}} associée à cette subdivision, i.e.
\[
\forall\ i \in \interent{0}{p},\, L_i(X) = \prod_{j \neq i} \frac{X - y_j}{y_i - y_j}.
\]

On pose alors $\lg_j = \int_0^1 L_j(t) \d t$.

On approche alors l'intégrale de $f$ sur $[x_i, x_{i+1}]$ par la somme
\[
I_p^i(f) = (x_{i+1} - x_i) \sum_{j=0}^p \lg_j g(x_i + (x_{i+1} - x_i) y_j),
\]
puis l'intégrale sur le segment $[a, b]$ par
\[
I_p(f) = \sum_{i=0}^{n-1} (x_{i+1} - x_i) \sum_{j=0}^p \lg_j f(x_i + (x_{i+1} - x_i) y_j).
\] 

On peut montrer que :
\begin{itemize}
\item lorsque $n = 1$, on retrouve la formule des trapèzes.

\item lorque $n = 2$, on retrouve la méthode de Simpson.
\end{itemize}

On peut montrer que la méthode de Simpson est d'ordre $3$. On peut augmenter le nombre des n\oe{}uds où est évaluée la fonction à intégrer ($2$ n\oe{}uds pour la méthode des trapèzes, $3$ pour la méthode de Simpson,\ldots). Cependant, lorsque le nombre de n\oe{}uds dépasse $8$, des coefficients négatifs apparaissent ce qui engendre des erreurs d'arrondis. \\

\todoinline{On pourrait renvoyer vers le livre Demailly - Analyse numérique et équations différentielles}


\todoinline{Je pense qu'il faudrait mettre les quadratures ailleurs pour éviter un thème trop long ? Un thème à part ? Une application dans la partie polynômes orthogonaux ?}


\section{Fonctions décroissantes}

\todoinline{Ici, on n'a pas de théorème, que des exercices. Ce n'est pas un problème mais chercherait-on une présentation uniforme sur ces premiers thèmes ?}

\begin{exercice}
    \marginnote[0cm]{Source : \cite{exos_oraux} p. 268}
    Soit $f : \Rp \to \R$ une fonction continue, décroissante et intégrable. Montrer que $x f(x) \xrightarrow[x \to +\infty]{} 0$.
\end{exercice}

\begin{marginfigure}[-2cm]
    \centering
    \begin{tikzpicture}[scale=0.5]
  
  \def\xmax{4.5}
  \def\ymax{1.5}
  \def\ymin{-2}
  \def\xzero{2}
  \def\x{5}

    \begin{axis}[
        restrict x to domain=0:8,
        samples=100, % you don't need 1000, it only slows things down
        ticks=none,
        xmin = -1, xmax = \xmax+1,
        ymin = \ymin, ymax = \ymax,
        unbounded coords=jump,
        axis x line=middle,
        axis y line=middle,
        % xlabel={$x$},
        % ylabel={$y$},
        x label style={
          at={(axis cs:5.1,0.2)},
          anchor=west,
        },
        every axis y label/.style={
          at={(axis cs:0,1.5)},
          anchor=south
        },
        legend style={
          at={(axis cs:-5.2,4)},
          anchor=west, font=\scriptsize
        },
        declare function={f(\x)=2*e^(-\x^2/2)-(\x/7)^2-1;},
        ] 
      \addplot[name path=A,very thick,color=blue, mark=none, domain=0:\xmax] {f(x)}
          node [above=8mm,near start] {$f\textcolor{black}{(x)}$};
          
      \addplot[mark=*,fill=white] coordinates {(\xzero,{f(\xzero)})};
      \draw[dashed] (axis cs:0,{f(\xzero)}) node[left=1mm] (l) {$f(x_0)$} -| (axis cs:2,0) node[above] (a) {$x_0$}; 
      \draw (axis cs:\x-0.5,0) node[above] {$x$};

    \path [name path=B] (0,0)--(\xmax, 0);
    \addplot [blue!20!white] fill between [of=A and B, soft clip={domain=\xzero:\x}];

    \fill[red!30!white, pattern=north east lines] (\xzero,0) rectangle (\x-0.5,f(\xzero);
    
    \end{axis}

    \draw[->, black, thick] (5, 4) node[above] {$f(x_0) (x - x_0)$} to [out=-90,in=90] ($(4.5,2.6)$);

    \draw[->, black, thick] (4, 1) node[left] {\color{blue}$\displaystyle \int_{x_0}^x f(t)\, \mathrm{d} t$} to [out=0,in=-90] ($(4.5,1.6)$);
    
  \end{tikzpicture}
    \caption{ébauche}
\end{marginfigure}

\begin{elem_sol}
\begin{itemize}
\item Comme $f$ est décroissante, d'après le théorème de la limite monotone, il existe $\ell \in \R \cup \ens{-\infty}$ tel que $\lim\limits_{x\to+\infty} f(x) = \ell$.

\item Montrons par l'absurde que $f$ est positive. S'il existe $x_0 \geq 0$ tel que $f(x_0) < 0$, comme $f$ est décroissante,
\[
\forall\ x \geq x_0,\, f(x) \leq f(x_0).
\]
Ainsi,
\[
\forall\ x \geq x_0,\, \displaystyle\int_{x_0}^x f(t) \mathrm{d}t \leq f(x_0) (x - x_0).
\]
D'après le théorème d'encadrement, $\lim\limits_{x\to+\infty} \displaystyle\int_{x_0}^x f(t) \mathrm{d}t = -\infty$, ce qui est impossible car $f$ est intégrable. Ainsi, $f$ est à valeurs positives et $\ell \geq 0$.

\item Supposons par l'absurde que $\ell > 0$. Alors, il existe un réel $A$ tel que
\[
\forall\ x \geq A,\, f(x) \geq \frac{\ell}{2}.
\]
Ainsi,
\[
\forall\ x \geq A,\, \displaystyle\int_A^x f(t) \mathrm{d}t \geq \frac{\ell}{2} (x - A).
\]
D'après le théorème d'encadrement, $\lim\limits_{x\to+\infty} \displaystyle\int_A^x f(t) \mathrm{d}t = +\infty$, ce qui est impossible car $f$ est intégrable.

Finalement, $\lim\limits_{x\to+\infty} f(x) = 0$.

\item Soit $x \geq 0$. Comme $f$ est décroissante,
\[
\displaystyle\int_x^{2 x} f(t) \mathrm{d}t \leq (2 x - x) f(x).
\]

De même,
\[
\displaystyle\int_{x/2}^x f(t) \mathrm{d}t \geq \frac{x}{2} f(x).
\]

\begin{marginfigure}
    \centering
    \begin{tikzpicture}

  % https://copyprogramming.com/howto/how-do-i-draw-arrows-at-coordinate-on-a-plot
  
  \def\xmax{5.5}
  \def\ymax{1.2}
  \def\ymin{-0.2}
  \def\x{2.5}
  \def\xzero{\x/2}

  \def\colfonc{green!70!black}

    \begin{axis}[
        restrict x to domain=0:10,
        samples=100, % you don't need 1000, it only slows things down
        ticks=none,
        xmin = -1, xmax = \xmax+1,
        ymin = \ymin, ymax = \ymax,
        unbounded coords=jump,
        axis x line=middle,
        axis y line=middle,
        axis line style={-latex},
        % xlabel={$x$},
        % ylabel={$y$},
        x label style={
          at={(axis cs:5.1,0.2)},
          anchor=west,
        },
        every axis y label/.style={
          at={(axis cs:0,1.5)},
          anchor=south
        },
        legend style={
          at={(axis cs:-5.2,4)},
          anchor=west, font=\scriptsize
        },
        declare function={f(\x)=e^(-\x^2/8);},
        ] 
      \addplot[name path=A,very thick,color=\colfonc, mark=none, domain=0:\xmax] {f(x)} node [above=2mm,near start] {$f$};
          
      % \addplot[mark=*,fill=white] coordinates {(\xzero,{f(\xzero)})};
      % \draw[dashed] (axis cs:0,{f(\xzero)}) node[left=1mm] (l) {$f(x_0)$} -| (axis cs:2,0) node[above] (a) {$x_0$}; 
      \draw (\xzero,0) node[below] {$x/2$};

      \draw[dashed] (axis cs:0,{f(\x)}) node[left=1mm] (l) {$f(x)$} -| (axis cs:\x,0) node[below] (a) {$\vphantom{x/2}x$};
      
      % \draw (\x,0) node[below] {$x$};
      \draw (2*\x,0) node[below] {$\vphantom{x/2}2x$};

    \path [name path=B] (0,0)--(\xmax, 0);
    \addplot [blue!20!white] fill between [of=A and B, soft clip={domain=\xzero:\x}];

    \addplot [red!20!white] fill between [of=A and B, soft clip={domain=\x:2*\x}];

    \fill[pattern color=blue, pattern=north east lines] (\xzero,0) rectangle (\x,f(\x);
    \fill[pattern color=red, pattern=north east lines] (\x,0) rectangle (2*\x,f(\x);

    \end{axis}

    % \draw[->, black, thick] (5, 4) node[above] {$f(x_0) (x - x_0)$} to [out=-90,in=90] ($(4.5,2.6)$);

    % \draw[->, black, thick] (4, 1) node[left] {\color{blue}$\displaystyle \int_{x_0}^x f(t)\, \mathrm{d} t$} to [out=0,in=-90] ($(4.5,1.6)$);
    
\end{tikzpicture}
    \caption{ébauche}
\end{marginfigure}

\todoinline{On pourrait illustrer les deux inégalités précédentes en dessinant les aires sous des rectangles en hachuré et les aires sous les courbes en couleur pastel. On pourrait mettre une couleur pour $[x/2, x]$ et une différente sur $[x, 2x]$}


Ainsi,
\[
\displaystyle\int_x^{2 x} f(t) \mathrm{d}t \leq x f(x) \leq 2 \displaystyle\int_{x/2}^x f(t) \mathrm{d}t.
\]
En notant $F : x \mapsto \displaystyle\int_0^x f(t) \mathrm{d}t$, comme $F$ est intégrable, alors $F$ possède une limite finie en $+\infty$.

Alors,
\[
F(2 x) - F(x) \leq x f(x) \leq 2 (F(x) - F(x/2)).
\]

Ainsi, d'après le théorème d'encadrement, $\lim\limits_{x\to 0} x f(x) = 0$.
\end{itemize}
\end{elem_sol}

\todoarmand{Mettre un lien vers \url{http://ddmaths.free.fr/section115.html} ou pas car c'est très classique}



\begin{exercice}
    \marginnote[0cm]{Source : \cite{truc2019} p. 268}
    Soit $f : \interof{0}{1} \to \R$ une fonction continue, décroissante et intégrable. Alors,
    \[
    \lim\limits_{n \to +\infty} \frac{1}{n} \sum\limits_{k=1}^n f\left(\frac{k}{n}\right) = \displaystyle\int_0^1 f(t) \mathrm{d}t.
    \]
\end{exercice}

\todoinline{Ajouter une illustration avec une fonction type Riemann $x \mapsto \frac{1}{\sqrt{x}}$ sur $]0, 1]$ ?}

\begin{elem_sol}
On note $R_n(f) = \frac{1}{n} \sum\limits_{k=1}^n f\left(\frac{k}{n}\right)$.
\begin{itemize}
\item Soit $n \geq 1$. On utilise la décroissance de la fonction $f$, pour tout $k \leq t \leq k + 1 \leq n-1$,
\begin{align*}
\frac{1}{n} f\left(\frac{k+1}{n}\right) &\leq \displaystyle\int_{k/n}^{(k+1)/n} f(t) \mathrm{d}t \leq \frac{1}{n} f\left(\frac{k}{n}\right)\\
\frac{1}{n} \sum\limits_{k=2}^{n} f\left(\frac{k}{n}\right) &\leq \displaystyle\int_{1/n}^1 f(t) \mathrm{d}t \leq \frac{1}{n} \sum\limits_{k=1}^{n-1} f\left(\frac{k}{n}\right)\\
R_n - \frac{1}{n} f\left(\frac{1}{n}\right) &\leq \displaystyle\int_{1/n}^1 f(t) \mathrm{d}t \leq R_n - \frac{f(1)}{n}.
\end{align*}

Ainsi,
\begin{align*}
\displaystyle\int_{1/n}^1 f(t) \mathrm{d}t + \frac{f(1)}{n} &\leq R_n \leq \displaystyle\int_{1/n} f(t) \mathrm{d}t + \frac{1}{n}  f\left(\frac{1}{n}\right).
\end{align*}

\item Comme $f$ est décroissante et intégrable sur $]0, 1]$,
\[
\frac{1}{n} f\left(\frac{1}{n}\right) \leq \displaystyle\int_0^{1/n} f(t) \mathrm{d}t.
\]

\item Comme $f$ est intégrable sur $]0, 1]$,
\[
\lim\limits_{n\to+\infty} \displaystyle\int_{1/n}^1 f(t) \mathrm{d}t = \displaystyle\int_0^1 f(t) \mathrm{d}t.
\]
\end{itemize}

Ainsi, d'après le théorème d'encadrement,
\[
\lim\limits_{n\to+\infty} R_n(f) = \displaystyle\int_0^1 f(t) \mathrm{d}t.
\]
\end{elem_sol}

\begin{remarque}
Une étude plus détaillée de ce résultats et de ses limites est discutée dans \cite{truc2019} p. 268.
\end{remarque}

%---------------

\begin{exercice}
\cite{RMS 888 2016 - ENSAM}
Soit $f : x \mapsto \frac{1}{x \sqrt{x^2 - 1}}$ définie sur $]1, +\infty[$.
\begin{enumerate}
\item Étudier et tracer la fonction $f$.

\smallskip
Pour tout entier naturel $n$, on pose $S_n = \sum\limits_{k=n+1}^{+\infty} \frac{1}{k \sqrt{k^2 - n^2}}$.
\item Étudier la convergence et la limite de la suite $(S_n)$.

\item Même question avec la suite $(n S_n)$.
\end{enumerate}
\end{exercice}

\todoinline{Illustrer $n S_n$ et $\int_{1}^{+\infty} f(t) \d t$ sur un même graphique ?}

\begin{elem_sol}
\begin{enumerate}
\item $f$ est décroisante, à valeurs positives, $\lim\limits_{1^+} f = +\infty$ et $\lim\limits_{+\infty} f = 0$. De plus, $f$ est continue et dérivable.

\item D'après la définition de $f$,
\[
S_n = \frac{1}{n^2} \sum\limits_{k=n+1}^{+\infty} f(k/n).
\]
Comme $f$ est décroissante, pour tout $t \in [k/n,(k+1)/n]$,
\begin{align*}
f((k+1)/n) &\leq f(t) \leq f(k/n) \\
n^{-1} \sum\limits_{k=n+2}^{N+1} f(k/n) &\leq \displaystyle\int_{1+1/n}^{N/n} f(t) \mathrm{d}t \leq n^{-1} \sum\limits_{k=n+1}^{N} f(k/n).
\end{align*}
Comme $f(x) \sim_{+\infty} \frac{1}{x^2}$, la suite $(S_{n,N})_N$ est croissante et majorée par $\displaystyle\int_{1+1/n}^{+\infty} f(t) \mathrm{d}t$ qui est convergente. Ainsi, en passant à la limite dans l'inégalité,
\begin{align*}
n S_n - f((n+1)/n) &\leq \displaystyle\int_{1+1/n}^{+\infty} f(t) \mathrm{d}t \leq n S_n \\
\frac{1}{n} \displaystyle\int_{1+1/n}^{+\infty} f(t) \mathrm{d}t &\leq S_n \leq \frac{1}{n} \displaystyle\int_{1+1/n}^{+\infty} f(t) \mathrm{d}t + \frac{1}{n^2} f((n+1)/n).
\end{align*}
De plus, $f(x) \sim_1 \frac{1}{\sqrt{2 (x - 1)}}$, donc $f$ est intégrable en $1$ et $(S_n)$ converge vers $0$ car $f((n+1)/n) \sim n^{1/2}$.

\item En reprenant l'encadrement précédent, $\left(n^{-1} f((n+1)/n)\right)$ converge toujours vers $0$ et $(n S_n)$ converge vers $\displaystyle\int_1^{+\infty} f(t) \mathrm{d}t$.

\textbf{Remarque.} $\displaystyle\int^x f(t) \mathrm{d}t = - \arctan\frac{1}{\sqrt{x^2 - 1}}$ et $\displaystyle\int_1^{+\infty} f = \frac{\pi}{2}$.
\end{enumerate}
\end{elem_sol}

\section{Calcul d'une intégrale impropre}

\todoinline{Semble s'appeler l'intégrale d'Euler - \url{https://fr.wikipedia.org/wiki/Table_d'intégrales}}

\todoinline{Illustrer par un graphique ces intégrales ?}

\todoinline{À mon avis, l'intérêt de ce calcul est d'utiliser les symétries associées aux fonctions sinus / cosinus pour pouvoir calculer. Dans le même genre il y a une intégrale sur un segment que je mets ci-dessous.

Je me demande donc s'il ne vaudrait pas mieux renommer cette section en intégrales de fonctions trigonométriques sans primitiver. C'est pas très sexy comme titre, on doit pouvoir l'améliorer ;-)}

\begin{exercice}
\cite{Oraux - CCP-PSI-2016}
    Soient $I = \int_0^{\pi/2} \ln\sin(t)) \d t$ et $J = \int_0^{\pi/2} \ln(\cos(t)) \d t$.
    \begin{enumerate}
        \item Montrer que $I$ et $J$ sont convergentes et que $I = J$.
        \item Calculer $I + J$ et en déduire $I$ et $J$.
    \end{enumerate}
\end{exercice}

\begin{marginfigure}[0cm]
    \begin{tikzpicture}
\begin{axis}[
    unit vector ratio*=1 1 1,
    xlabel={},
    ylabel={},
    xmin=-0.5, xmax=2,
    ymin=-5, ymax=1,
    xtick={0,0.7854,1.5708}, 
    xticklabel style={above},
    xticklabels={$0$,$\displaystyle \frac{\pi}{4}$,$\displaystyle \frac{\pi}{2}$},
    ytick={-1,-4},
    yticklabels={$-1$, $-4$},
    xmajorgrids,
    axis lines=middle,
    samples=100,
    legend style={
            at={(1.5708,0.5)},
            anchor=north,
            legend cell align=left,
            draw=none % Unterdrücke Box
        },
]

\addplot[blue,thick, domain=0.01:1.5708] {ln(sin(deg(x)))};
\addplot[red,thick, domain=0.01:1.5708] {ln(cos(deg(x)))};
\legend{$\ln\!\big(\!\sin(x)\!\big)$, $\ln\!\big(\!\cos(x)\!\big)$}

\end{axis}
\end{tikzpicture}

\end{marginfigure}

% \todoinline{En mettre un peu plus sur la démo ? J'ai la version suivante à relire et changer les dt (CCP-PSI-2016) :}

\begin{elem_sol}
\begin{enumerate}
\item La fonction $t \mapsto \ln(\sin(t))$ est continue sur $]0,\pi/2]$. De plus,
\[
\ln(\sin(t)) = \ln(t + o(t)) = \ln(t) + \ln(1 + o(1)) = o(\ln(t)).
\]
Ainsi, $t \mapsto \ln(\sin(t))$ est intégrable en $0$.

La formule de changement de variable, avec $\phi : u \mapsto \pi/2 - u$ assure la convergence de $J$ ainsi que l'égalité $I = J$.

\item Comme ces intégrales sont bien définies, en utilisant la relation de Chasles et la symétrie dans la dernière égalité,
\[
I + J = \int_0^{\pi/2} \ln\left(\frac{\sin(2t)}{2}\right) \mathrm{d} t = \frac{1}{2} \int_0^\pi \ln(\sin(t)) \mathrm{d} t - \frac{\pi}{2} \ln(2) = I - \frac{\pi}{2} \ln(2).
\]
Ainsi, $I = J = -\frac{\pi}{2} \ln(2)$.
\end{enumerate}
\end{elem_sol}


\todoinline{L'exercice dont je parlais plus haut}

\begin{exercice}
Calculer $\int\limits_0^{\pi/4} \ln(1+\tan x) \d x$.
\end{exercice}

\begin{elem_sol}
Ici, les règles de Bioche ne marchent pas. Il va falloir ruser. On commence par supprimer la première fonction qui nous dérange~:~la tangente.
\begin{align*}
\int\limits_0^{\pi/4} \ln(1+\tan x) \d x &= \int\limits_0^{\pi/4} \ln(1+\frac{\sin x}{\cos x}) \d x\\
 &= \int\limits_0^{\pi/4} \ln(\sin x+\cos x) \d x - \int\limits_0^{\pi/4} \ln(\cos x)\d x.
\end{align*}

Maintenant, il faut essayer d'éliminer l'intégrale du logarithme (toute tentative de Bioche échoue à nouveau). On remarque alors que
\[
\cos x + \sin x = \sqrt{2} \cos\left(x-\frac{\pi}{4}\right).
\]
On obtient ainsi
\begin{align*}
\int\limits_0^{\pi/4} \ln(1+\tan x) \d x &= \int\limits_0^{\pi/4} \ln(\sqrt{2}) \d x + \int\limits_0^{\pi/4} \ln(\cos x) \d x + \cdots\\
& \cdots - \int\limits_0^{\pi/4} \ln(\cos x) \d x\\
&= \frac{\pi}{8} \ln 2.
\end{align*}
\end{elem_sol}

\todoinline{Si on décide d'élargir à l'utilisation des symétries, il y a l'exercice suivant que j'aime bien.}

\begin{exercice}
Justifier l'existence puis calculer l'intégale $\int\limits_0^{+\infty} \frac{x \ln(x)}{(1 + x^2)^2} \d x$.
\end{exercice}

\todoinline{Ajouter l'existence dans preuve et relire le calcul.}

\begin{elem_sol}
D'après la relation de Chasles,
\[
\int\limits_0^{+\infty} \frac{x \ln(x)}{(1 + x^2)^2} \d x
= \int\limits_0^1 \frac{x \ln(x)}{(1 + x^2)^2} \d x + \int\limits_1^{+\infty} \frac{x \ln(x)}{(1 + x^2)^2} \d x.
\]

Dans la seconde intégrale, on effectue le changement de variable $\phi : [1, +\infty[ \to ]0, 1],\, u \mapsto \frac{1}{u}$ qui est bien $\mathscr{C}^1$ et bijectif :
\begin{align*}
\int\limits_0^{+\infty} \frac{x \ln(x)}{(1 + x^2)^2} \d x
&= \int\limits_0^1 \frac{x \ln(x)}{(1 + x^2)^2} \d x + \int\limits_0^1 \frac{\frac{1}{u} \ln \frac{1}{u}}{(1 + \frac{1}{u^2})^2} \times \frac{1}{u^2} \d u\\
&= \int\limits_0^1 \frac{x \ln(x)}{(1 + x^2)^2} \d x - \int\limits_0^1 \frac{u \ln(u)}{(u^2 + 1)^2} \d u\\
&= 0.
\end{align*}
\end{elem_sol}

\section{Permutation somme/intégrale}

\todoinline{Je pense que l'intérêt de l'écriture sur la somme est d'avoir une approximation de l'intégrale. On tente une illustration de cette rapidité de convergence ?}

\begin{prop}{}
\[
\frac{1}{2} \int_{0}^{+ \infty} \frac{\cos (xt)}{\ch t} \d t = \sum_{n=0}^{+ \infty} \frac{(-1)^n (2n+1)}{(2n+1)^2 + x^2}.
\]
\end{prop}

\begin{exercice}
On pose $f : t \mapsto \frac{\cos(x t)}{\e^t + \e^{-t}}$ et $f_n : t \mapsto (-1)^n \cos(x t) \e^{-(n + 1) t}$.
\begin{questions}
\item Justifier la convergence de $\displaystyle\frac{1}{2} \int_{0}^{+ \infty} \frac{\cos (xt)}{\ch t} \d t$.

\item Montrer que $f(x) = \sum\limits_{n=0}^{+\infty} (-1)^n \cos(x t) \e^{-(2 n + 1) t}$.

\item Déterminer $\displaystyle\int_0^{+\infty} f_n(t) \d t$.

\item Conclure en utilisant le théorème des séries alternées.
\end{questions}
\end{exercice}

\begin{elemsolution}
\begin{reponses}
\item La fonction $f \colon t \mapsto \frac{\cos(x t)}{2 \ch(t)}$ est continue sur $[0, +\infty[$. De plus, $|f(t)| \leq \frac{1}{\ch(t)}$ et $\frac{1}{\ch(t)} \sim_{+\infty} 2 \e^{-t}$. Ainsi, la fonction $f$ est intégrable sur $[0, +\infty[$ et l'intégrale converge.

\item En utilisant le développement en série entière de la fonction inverse, comme $0 < \e^{-2t} < 1$ sur $]0, +\infty[$, 
\begin{align*}
f(t)
&= \frac{\cos(x t)}{\e^t + \e^{-t}}
= \e^{-t} \cos(x t) (1 + \e^{-2t})^{-1}\\
&= \e^{-t} \cos(x t) \sum_{n=0}^{+\infty} (-1)^n \e^{-2 n t}\\
&= \sum_{n=0}^{+\infty} (-1)^n \cos(x t) \e^{- (2 n + 1) t}.
\end{align*}

\item En utilisant la formule d'Euler,
\begin{align*}
(-1)^n \int_0^{+\infty} f_n(t) \d t
&= \Reel\left(\int_0^{+\infty} \e^{(\i x - (2 n + 1)) t} \d t\right)\\
&= \Reel\left(-\frac{1}{\i x - (2 n + 1)}\right)\\
&= \frac{2 n + 1}{x^2 + (2 n + 1)^2}.
\end{align*}

\item Remarquons que la série de terme général $(-1)^n \e^{- (2 n + 1) t}$ est une série alternée. Ainsi,
\begin{align*}
\left|\sum_{n=N+1}^{+\infty} (-1)^n \e^{-(2 n + 1) t}\right|
&\leq \e^{-(2 N + 1) t}.
\end{align*}
Alors,
\begin{align*}
\module{\sum_{n=0}^N f_n(t)}
&\leq \module{f(t) - \sum_{n=0}^N f_n(t)} + \module{f(t)}
= \module{\sum_{n=N+1}^{+\infty} f_n(t)} + \module{f(t)}
&\leq |f(t)| + \e^{-(2N+1) t} \leq \module{f(t)} + \e^{-t}.
\end{align*}

Ainsi, $\left(\sum\limits_{n=0}^N f_n\right)_{n\in\N}$ converge simplement vers $f$ et ses sommes partielles sont dominées par une fonction intégrable. Donc, d'après le théorème de convergence dominée,
\begin{align*}
\int_0^{+\infty} f(t) \d t
&= \lim_{N\to+\infty} \sum_{n=0}^N \int_0^{+\infty} f_n(t) \d t\\
\int_0^{+\infty} \frac{\cos(x t)}{2 \ch(t)} \d t
&= \sum_{n=0}^{+\infty} \frac{(-1)^n (2 n + 1)}{x^2 + (2 n + 1)^2}.
\end{align*}
\end{reponses}
\end{elemsolution}

%--------
\section{Comparaisons Séries / Intégrales}
% \section{Intégration des relations de comparaisons}

\todoinline{Le sujet de centrale est très prometteur au début puis utilise des résultats beaucoup plus faibles que ceux démontrés en préliminaires. Je propose de déplacer ce résultat dans la partie Séries numériques}

% Source : Centrale - PC - Maths 1 - 2003
\begin{prop}
Il existe deux constantes $\gamma$ et $\delta$ telles que
\[
n! = \delta n^{n + \frac{1}{2}} \e^{-n} \left(1 + \frac{1}{12 n} + o\left(\frac{1}{n}\right)\right).
\]
\end{prop}

\begin{itemize}
\item On suppose que $f$ et $g$ sont positives et vérifient $f(x) \sim_b g(x)$. Si $\int_a^b f(t) \d t$ converge, alors $\int_x^b f(t) \d t \sim_b \int_x^b g(t) \d t$.

Soit $\epsilon > 0$. La fonction $g$ étant strictement positive et comme $f \sim g$, il existe $\beta \in [a, b[$ tel que
\[
\forall\, t \in [\beta, b[,\, 0 < \abs{f(t) - g(t)} \leq \epsilon g(t).
\]
Ainsi, comme $g$ est intégrable sur $[\beta, b[$, alors $f - g$ l'est également. D'après la croissance de l'intégrale et l'inégalité triangulaire,
\[
\forall\, x \in [\beta, b[,\, 0 \leq \abs{\int_x^b f(t) \d t - \int_x^b g(t) \d t} \leq \epsilon \int_x^b g(t) \d t.
\]
Ainsi, $\int_x^b f(t) \d t \sim \int_x^b g(t) \d t$.

\item Si $a_n \sim b_n$ sont positifs et les termes généraux de séries convergentes, alors $\sum_{k=n+1}^{+\infty} a_k \sim \sum_{k=n+1}^{+\infty} b_k$.

Il suffit d'appliquer le résultat précédent aux fonctions en escalier
\[
f(t) = \sum_{k=0}^{+\infty} a_k \indicatrice{[k,k+1[}
\text{ et }
g(t) = \sum_{k=0}^{+\infty} b_k \indicatrice{[k, k+1[}.
\]

\item On pose $b_n = 1 + \left(n - \frac{1}{2}\right) \ln\left(1 - \frac{1}{n}\right)$ pour $n \geq 2$ et $b_0 = 0$, $b_1 = 1$.

On remarque que $b_n \sim -\frac{1}{12 n^2}$. Comme $\sum a_n$ et $\sum b_n$ convergent, alors
\[
\sum_{k=n+1}^{+\infty} a_n \sim \sum_{k=n+1}^{+\infty} b_n.
\]

\item En utilisant les comparaisons séries / intégrales,
\[
\sum_{k=n+1}^{+\infty} \frac{1}{k^2} \sim \int_n^{+\infty} \frac{\d t}{t^2} = \frac{1}{n}.
\]
\todoinline{À justifier}


Ainsi, il existe un réel $\ell$ tel que
\[
\sum_{k=0}^n b_k = \ln\left(\frac{n! \e^n}{n^{n+\frac{1}{2}}}\right) = \ell + \frac{1}{12 n} + o\left(\frac{1}{n}\right).
\]

En utilisant la fonction expontentielle,
\[
\frac{n! \e^n}{n^{n+\frac{1}{2}}} = \e^{\ell} \times \e^{\frac{1}{12 n} + o\left(\frac{1}{n}\right)}.
\]

On obtient le résultat attendu en utilisant un développement limité de la fonction exponentielle.
\end{itemize}



\begin{prop}
    Soit $f: \Rp \rightarrow \C$ une fonction continue par morceaux et $g, h:\Rp \rightarrow \Rp$ deux fonctions continues par morceaux, strictement positives. On suppose que $f = o_{+\infty}(g)$ et $f \sim_{+\infty} h$.\\
    \begin{itemize}
        \item Si $g$ et $h$ ne sont pas intégrables sur $\Rp$,
        $$\int_{0}^{x} f = o_{+\infty} \left(\int_{0}^{x} g \right) \text{ et } \int_{0}^{x} f \sim_{+\infty} \int_{0}^{x} h.$$
        \item Si $g$ et $h$ sont intégrables sur $\Rp$,
        $$\int_{x}^{+\infty} f = o_{+\infty} \left(\int_{x}^{+\infty} g \right) \text{ et } \int_{x}^{+\infty} f \sim_{+\infty} \int_{x}^{+\infty} h.$$
    \end{itemize}
\end{prop} 

La démonstration est analogue à celle de la \nameref{sommation_relations_comparaison}


\todoinline{J'ai relu. Sans illustrations et sans application, est-ce qu'on le laisse ? Ou alors on trouve une application, mais je n'en ai pas sous le coude à l'instant !}

\begin{theo}[Intégrales de \textsc{Bertrand}]
    Soient $(\alpha, \beta) \in  \R^2$ et 
    $$f:t \mapsto \frac{1}{t^{\alpha} \ln^{\beta} (t)}.$$
    Alors,
    $$\int_{2}^{+ \infty} f(t) \d t \text{ converge si et seulement si }
    \begin{cases}
    \alpha > 1 \\
    \text{ou}\\
    \alpha = 1 \text{ et } \beta > 1
    \end{cases}.
    $$
\end{theo}

% \todoinline{On écrit plutôt $\int_2^{+\infty} f(t) dt$ et $\int_{[2,+\infty[} f$.}

\begin{demo}
    Distinguons trois cas selon les valeurs prises par $\alpha$:
    \begin{enumerate}
        \item[$\rhd$] \textbf{Cas où $\alpha > 1$.} Soit $\gamma \in \interoo{1}{\alpha}$. Par croissances comparées,
        $$\displaystyle \frac{1}{t^{\alpha} \ln^{\beta} (t)} = o_{+ \infty} \left( \frac{1}{t^{\gamma}} \right).$$
        Or, d'après la convergence des intégrales de \textsc{Riemann}, la fonction $t \mapsto \frac{1}{t^\gamma}$ est intégrable sur $[2, +\infty[$ car $\gamma > 1$. Ainsi, en appliquant les théorèmes de comparaison, $\int_2^{+ \infty} f$ converge.

        \item[$\rhd$] \textbf{Cas où $\alpha < 1$.} Soit $\gamma \in \interoo{\alpha}{1}$. Par croissances comparées,
        $$t^{\gamma} f(t) \xrightarrow[t \to + \infty]{} + \infty$$
        donc à partir d'un certain rang, $f(t) \geqslant \frac{1}{t^{\gamma}} > 0$. Or, d'après la convergence des intégrales de \textsc{Riemann}, la fonction $t \mapsto \frac{1}{t^\gamma}$ n'est intégrable pas sur $[2, +\infty[$ car $\gamma < 1$. Ainsi, en appliquant les théorèmes de comparaison (les intégrandes sont positives), $\int_2^{+ \infty} f$ diverge.
        
        \item[$\rhd$] \textbf{Cas où $\alpha = 1$.} Revenons aux intégrales partielles: soit $X > 2$,
        $$\int_{2}^{X} \frac{1}{t \ln^{\beta} (t)} \d t = 
        \begin{cases}
            \left[ \frac{\ln ^{1-\beta} (t)}{1-\beta} \right]_2 ^X & \text{si } \beta \not = 1, \\
            \left[\ln (\ln t) \right]_2 ^X & \text{si } \beta = 1.
        \end{cases}
        $$
        On en déduit que l'intégrale de la fonction $t \mapsto \frac{1}{t \ln^{\beta} (t)}$ converge sur $[2, + \infty[$ si et seulement si $\beta > 1$.
    \end{enumerate}
\end{demo}

\todoinline{J'ai mis dans "documents" le sujet Centrale PC 2003 - Il fait à la fois des relations de comparaisons, de l'intégrale de Bertrand à la fin et une intégrale fonction des bornes. C'est peut être une bonne idée !}

\todoarmand{Effectivement, c'est une bonne idée. Ça permettrait de supprimer l'exercice sur l'intégrale de Bertrand et de l'intégrer avec celui-ci}

\begin{exercice}
Soient $a\in \R,\, b \in \interoo{a}{+\infty} \cup \ens{+\infty}$ et $f, g$ deux applications continues par morceaux sur $\interfo{a}{b}$ à valeurs strictement positives.
\begin{enumerate}
    \item On suppose que $g$ est intégrable sur $\interfo{a}{b}$.
    \begin{enumerate}
        \item Montrer que, en $b$, la relation $f = o(g)$ entraîne $\int_x^{b} f = o\left(\int_x^b g\right)$. \\
        \emph{
        On n'hésitera pas à raisonner en utilisant des $\eg$.
        }
        \item Montrer que, en $b$, la relation $f \sim g$ entraîne $\int_x^{b}f \sim \left(\int_x^b g\right)$. \\
        \emph{
        On justifiera l'intégrabilité de $f$ sur les intervalles $\interfo{x}{b}$ considérés.
        }
    \end{enumerate}
    \item On suppose que $g$ n'est pas intégrable sur $\interfo{a}{b}$
    \begin{enumerate}
        \item Montrer que, en $b$, la relation $f = o(g)$ entraîne $\int_a^{x} f(t) \d t = o\left(\int_a^x g(t) \d t\right)$.
        Montrer à l'aide d'exemples que l'on ne peut rien dire de l'intégrabilité de $f$ sur $\interfo{a}{b}$.
        \item Montrer que, en $b$, la relation $f \sim g$ entraîne $\int_a^{x} f(t) \d t \sim \int_a^x g(t) \d t$.
        Que peut-on dire de l'intégrabilité de $f$ sur $\interfo{a}{b}$ ?
    \end{enumerate}
\end{enumerate}
\end{exercice}

\begin{exercice}
Soit $a$ un nombre réel et $f$ une application de classe $\mathscr{C}^1$ sur $\interfo{a}{+\infty}$ à valeurs strictement positives. On suppose que le quotient $\frac{x f'(x)}{f(x)}$ tend vers une limite finie $\alpha$ en $+\infty.$
\begin{enumerate}
    \item Montrer, à l'aide des préliminaires que, en $+\infty$, $\frac{\ln(f(x))}{\ln(x)}$ tend vers $\alpha$. \\
    \emph{
    On peut distinguer le cas $\alpha = 0$.
    }
    \item On suppose dans cette question $\alpha < -1.$
    \begin{enumerate}
        \item Montrer que $f$ est intégrable sur $\interfo{a}{+\infty}$.
        \item Montrer que, en $+\infty$, on a $\int_x^{+\infty} f(t) \d t \sim -\frac{x f(x)}{\alpha + 1}$. \\
        \emph{
        On pourrra considérer $\frac{x f(x)}{\alpha+1}$ et utiliser les préliminaires.
        } 
    \end{enumerate}
    \item On suppose dans cette question $\alpha > -1.$
    \begin{enumerate}
        \item Étudier l'intégrabilité de $f$ sur $\interfo{a}{+\infty}.$
        \item Montrer que, en $+\infty$, on a $\int_a^{x} f(t) \d t \sim \frac{x f(x)}{\alpha + 1}$.
        \item Donner un exemple d'application $f$ de classe $\mathscr{C}^1$ sur $\interfo{a}{+\infty}$ à valeurs positives telle qu'en $+\infty$ le quotient $\frac{\ln(f(x))}{\ln(x)}$ tend vers $\alpha > -1,$ mais telle que l'on n'ait pas $\int_a^{x} f(t) \d t \sim \frac{x f(x)}{\alpha+1}$.
    \end{enumerate}
    \item Étudier l'intégrabilité sur $\interfo{2}{+\infty}$ des applications $ x \mapsto \frac{1}{x (\ln x)^{\beta}}$ selon les valeurs du réel $\beta$.
    \begin{enumerate}
        \item Étudier, à l'aide des questions précédentes, l'intégrabilité sur $\interfo{2}{+\infty}$ des applications $x \mapsto \frac{1}{x^{\gamma}(\ln x)^{\beta}}$, selon les valeurs des réels $\beta$ et $\gamma$.
        \item Que conclure quant à l'intégrabilité de $f$ sur $\interfo{a}{+\infty}$ dans le cas $\alpha =-1$ ?
    \end{enumerate}
\end{enumerate}
\end{exercice}


% Le fichier suivant est vide
% \section{Transformée de \textsc{Fourier} de la loi normale}


\section{Intégrale de \textsc{Dirichlet}}\label{sec:intDirichlet}

\todoarmand{Éléments historiques : \url{https://hsm.stackexchange.com/a/6827}}

\marginnote[0cm]{Dans le cadre de la diffraction de \textsc{Fraunhofer}, si l'on appelle $D$ la distance entre l'écran et la fente, alors l'intensité $I$ en un point $x$ de l'écran s'écrit
\[
I(x) = I_0\, \mathrm{sinc}\mathopen{}\left(\frac{\pi a}{\lambda D}\, x\right)^2
\]
où $\mathrm{sinc}$ est la fonction \textsl{sinus cardinal}, $a$ est la largeur de la fente et $\lambda$  est la longueur d'onde de la radiation lumineuse. \\
\url{https://fr.wikipedia.org/wiki/Diffraction_par_une_fente}, pour des figures: \url{https://physique.ensc-rennes.fr/tp_diffraction.php} et \url{https://tikz.net/optics_diffraction/}
}

\begin{defi}[Sinus cardinal]
    On nomme \definir{sinus cardinal} la fonction définie par
    \[
    \fonction[\mathrm{sinc}]{\Re}{\R}{x}{\frac{\sin(x)}{x}}.
    \]
\end{defi}

\begin{theo}[Intégrale de \textsc{Dirichlet} (1829)]
L'intégrale $\displaystyle\int_{0}^{+\infty} \frac{\sin x}{x} \d x$ est convergente et
    \[
    \int_{0}^{+\infty} \frac{\sin x}{x} \d x = \frac{\pi}{2}.
    \]
\end{theo}

\begin{marginfigure}[0cm]
    \begin{tikzpicture}
    
\begin{axis}[
    % width=7.5cm,
    % grid=both,
    xmin=-11,
    xmax=11,
    ymin=-0.25,
    ymax=1.15,
    % ylabel=$\mathrm{sinc}(x)$,
    axis lines=middle,
    axis line style=thick,
    axis line style={-latex},
    xticklabels={},
    xtick={-3*3.141592, -2*3.141592, -3.141592, 3.141592, 2*3.141592, 3*3.141592},
    ytick={0, 1},
    xlabel=$x$,
    every axis x label/.style={at={(current axis.right of origin)},anchor=south},
]
              
  \addplot[domain=-15:15, blue, samples=200, name path=B] plot[thick] {sin(deg(x))/x};

  \path[name path=xaxis]
      (-15,0) -- (\pgfkeysvalueof{/pgfplots/xmax},0);
    \addplot[blue!25, opacity=0.9] fill between[of=xaxis and B];
  
  \node[blue,above left] at (-2,0.5) {$\displaystyle x \mapsto \frac{\sin(x)}{x}$};
\end{axis}
\begin{axis}[
    % width=7.5cm,
    % grid=both,
    xmin=-11,
    xmax=11,
    ymin=-0.25,
    ymax=1.15,
    % ylabel=$\mathrm{sinc}(x)$,
    axis lines=middle,
    % axis line style=thick,
    % axis line style={-latex},
    axis line style={draw=none},
    xticklabels={\contour{white}{$-3\pi$}, \contour{white}{$-2\pi$}, \contour{white}{$-\pi$}, \contour{white}{$\pi$}, \contour{white}{$2\pi$}, \contour{white}{$3\pi$}},
    xtick={-3*3.141592, -2*3.141592, -3.141592, 3.141592, 2*3.141592, 3*3.141592},
    ytick={0, 1},
]
\end{axis}
\end{tikzpicture}
\end{marginfigure}

Nous proposons deux démonstrations de l'existence de l'intégrale de \textsc{Dirichlet}, une première utilisant le théorème des séries alternées et une deuxième utilisant l'intégration par parties.

\todoinline{Ajout ici d'un format exercice / solution}

\begin{exercice}
Pour tout $n$ entier naturel non nul, on pose $u_n = \displaystyle\int_{n\pi}^{(n+1)\pi} \mathrm{sinc}(t) \d t$.
\begin{questions}
\item Montrer que $\mathrm{sinc}$ est intégrable sur $\interof{0}{1}$.

\item Montrer que $\sum u_n$ est une série alternée et en déduire la convergence de l'intégrale de Dirichlet.

\item En utilisant une intégration par parties, proposer une autre démonstration de la convergence de l'intégrale de Dirichlet.
\end{questions}
\end{exercice}

\begin{solution}
\begin{reponses}
\item La fonction $\mathrm{sinc}$ est continue sur $\interof{0}{1}$. Comme elle est prolongeable par continuité par la valeur $1$ en $0$, elle est intégrable sur $\interof{0}{1}$.

\item Soit $n \geqslant 1$, par un changement de variable affine,
\[
u_n =
\int_{n\pi}^{(n+1)\pi} \frac{\sin t}{t} \d t
= (-1)^n \int_0^\pi \frac{\sin u}{u + n \pi} \d u.
\]
En notant $u_n$ cette quantité, comme la fonction sinus est positive sur $\interff{0}{\pi}$,
\begin{comment}
\item $u_n u_{n+1} \leq 0$,
\item $\abs{u_n} \leq \frac{1}{n\pi} \int_0^\pi \sin(u) \d u$ soit $u_n \to 0$,
\item $\abs{u_{n+1}} \leq \abs{u_n}$.
\end{comment}
\[
u_n u_{n+1} \leqslant 0, \quad \abs{u_n} \leqslant \frac{1}{n\pi} \int_0^\pi \sin(u) \d u \text{ soit } u_n \to 0, \quad \text{et} \quad \abs{u_{n+1}} \leqslant \abs{u_n}.
\]
\marginnote[-7pt]{\theoremeutilise{\hyperref[thm:seriesAlternees]{Théorème des séries alternées}}}Ainsi, d'après le théorème des séries alternées, $\sum u_n$ converge.

\medskip

Cependant, la fonction $x \mapsto \int_0^x \frac{\sin t}{t} \d t$ n'est pas monotone et on ne peut donc pas encore conclure directement à la convergence de l'intégrale.

On pose ainsi $n_x = \left\lfloor\frac{x}{\pi}\right\rfloor$. En posant $F(x) = \int_1^{n_x\pi} \frac{\sin t}{t} \d t + \int_{n_x}^x \frac{\sin t}{t} \d t$, le premier terme converge d'après la question précédente. Le second est majoré par une quantité qui tend vers $0$. Ainsi, l'intégrale de Dirichlet converge.

\item Les fonctions $\fonctionligne[u]{t}{-\cos(t)}$ et $\fonctionligne[v]{t}{1/t}$ sont de classe $\mathscr{C}^1$ sur $\interfo{1}{+\infty}$ et $\lim_{+\infty} u v = 0$. Ainsi, par intégration par parties, les intégrales $\int_1^{+\infty} \frac{\sin t}{t} \d t$ et $\int_1^{+\infty} \frac{\cos t}{t^2} \d t$ sont de même nature. \\
D'une part, la fonction $t \mapsto \frac{\cos t}{t^2}$ est continue sur $\interfo{1}{+\infty}$ et d'autre part on a la majoration $\frac{\abs{\cos t}}{t^2} \leqslant \frac{1}{t^2}$ par une fonction intégrable sur $\interfo{1}{+\infty}$. Ainsi, d'après les théorèmes de comparaison, la fonction $t \mapsto \frac{\abs{\cos t}}{t^2}$ est intégrable et l'intégrale $\int_1^{+\infty} \frac{\cos t}{t^2} \d t$ converge. Finalement, nous avons démontré la convergence de l'intégrale $\int_1^{+\infty} \frac{\sin t}{t} \d t$.
\end{reponses}
\end{solution}

\begin{remarque}
L'intégration par parties préserve la régularité de l'intégrale mais ne préserve pas l'intégrabilité. La section suivante permet d'illustrer cette propriété.
\end{remarque}

%-----------
\subsection{Non intégrabilité}

\begin{prop}
    La fonction sinus cardinal $\fonctionligne[\mathrm{sinc}]{t}{\frac{\sin(t)}{t}}$ n'est pas intégrable sur $\interoo{0}{+\infty}$.
\end{prop}

\todoinline{Ajout de l'exercice, je trouvais difficile de résoudre le problème seul.}

\begin{exercice}
Soit $N \in \Ne$.
\begin{questions}
\item Montrer que $\displaystyle\int_0^{N\pi} \module{\mathrm{sinc}(x)} \d x \geq \frac{2}{\pi} \sum_{k=1}^N \frac{1}{k}$.

\item Conclure.
\end{questions}
\end{exercice}

\begin{demo}
\marginnote[0cm]{Source : \href{https://www.agreg-maths.fr/uploads/versions/1175/dirichlet.pdf}{Intégrale de \textsc{Dirichlet} -- Florian \textsc{Dussap}}}
\begin{reponses}
\item En utilisant la relation de Chasles,
    \begin{align*}
        \int_0^{N \pi} \frac{|\sin x|}{x} \d x &= \sum_{k=0}^{N-1} \int_{k \pi}^{(k+1) \pi} \frac{|\sin x|}{x} \d x \\
&= \sum_{k=0}^{N-1} \int_0^\pi \frac{|\sin x|}{x + k \pi} \d x        \text{, par un changement de variable }  \\
        &\geqslant \sum_{k=0}^{N-1} \frac{1}{(k+1) \pi} \int_0^\pi \sin x \d x \\
        &\geqslant \frac{2}{\pi} \sum_{k=1}^N \frac{1}{k}.
        \end{align*}

\item Comme $\sum \frac{1}{k}$ diverge, alors $\displaystyle\lim\limits_{N\to+\infty} \int_0^{N\pi} \module{\mathrm{sinc}(x)} \d x = +\infty$.

La fonction $x \mapsto \displaystyle\int_0^x \module{\mathrm{sinc}(t)} \d t$ étant croissante, on obtient même $\displaystyle \lim_{x\to+\infty} \int_0^x \module{\mathrm{sinc}(t)} \d t = +\infty$.
\end{reponses}
\end{demo}

%-----------        
\subsection{Calcul via le lemme de \textsc{Lebesgue}}

\marginnote[7pt]{\hyperref[sec:lemmeLebesgue]{\faLink~Lemme de \textsc{Lebesgue}}}
\begin{exercice}
Soit $n \in \N$.
\begin{questions}
    \item Calculer $I_n = \displaystyle \int_0^\pi \frac{\sin\mathopen{}\big[(n + 1/2)t\big]}{2 \sin \frac{t}{2}} \d t$. \\
    \emph{Indication :} Calculer $I_{n+1} - I_n$. 

    \item Montrer que la fonction définie sur $\interof{0}{\pi}$ par $f(x) = \frac{1}{x} - \frac{1}{2 \sin \frac{x}{2}}$ peut être prolongée à $\interff{0}{\pi}$ en une fonction de classe $\mathscr{C}^1$. 
    \item En déduire la valeur de l'intégrale $I = \int_0^{+\infty} \frac{\sin t}{t} \d t$.\\
    \emph{Indication :} On utilisera le lemme de \textsc{Lebesgue}
\end{questions}
\end{exercice}

\begin{solution}
\begin{reponses}
\item En utilisant les formules de trigonométrie,
\begin{align*}
I_{n+1} - I_n
&= \int_0^\pi \frac{\sin\mathopen{}\big[(n + 1 + 1/2)t\big] - \sin\mathopen{}\big[(n + 1/2) t\big]}{2 \sin \frac{t}{2}} \d t\\
&= \int_0^\pi \cos\mathopen{}\big[(n + 1) t\big] \d t
= \frac{1}{n + 1} \Big[\sin\mathopen{}\big[(n + 1) t\big]\Big]_{0}^\pi
= 0.
\end{align*}

Ainsi, la suite $(I_n)$ est constante et $I_0 = \frac{\pi}{2}$. Donc, pour tout $n$ entier naturel, $I_n = \frac{\pi}{2}$.

\item La fonction $f$ est continue et dérivable sur $\interof{0}{\pi}$. De plus, pour tout réel $x \in \interof{0}{\pi}$, $f(x) = \frac{2 \sin(x/2) - x}{2 x \sin(x/2)}$.

En utilisant les développements limités classiques,
\[
f(x)
\sim -\frac{x^3}{24} \times \frac{1}{x^2}
\sim -\frac{x}{24}.
\]
Ainsi, la fonction $f$ est prolongeable par continuité en $0$ par $f(0) = 0$.

\medskip

La fonction $f$ est dérivable sur $\interof{0}{\pi}$ et pour tout $x \in \interof{0}{\pi}$,
\begin{align*}
f'(x) = -\frac{1}{x^2} + \frac{\cos(x/2)}{4 \sin(x/2)^2} = \frac{\cos(x/2) - 4 \sin(x/2)^2}{4 x^2 \sin(x/2)^2} \sim -\frac{1}{24}.
\end{align*}

\marginnote[-7pt]{\theoremeutilise{Théorème de prolongement dérivable}}Ainsi, $\lim_{x\to0} f'(x) = -\frac{1}{24}$. D'après le théorème de prolongement dérivable, la fonction $f$ est prolongeable en une fonction de classe $\mathscr{C}^1$ sur $\interff{0}{\pi}$.

\item En utilisant la fonctin $f$ introduite précédemment,
\begin{align*}
\int_0^{(n + 1/2) \pi} \frac{\sin t}{t} \d t
&= \int_0^\pi \frac{\sin\mathopen{}\big[(n + 1/2) t\big]}{t} \d t\\
&= \int_0^\pi \left[ \frac{\sin\mathopen{}\big[(n + 1/2) t\big]}{t} - \frac{\sin\mathopen{}\big[(n + 1/2) t\big]}{2 \sin (t/2)} + \frac{\sin\mathopen{}\big[(n + 1/2) t\big]}{2 \sin(t/2)} \right] \d t\\
&= \int_0^\pi f(t) \sin\mathopen{}\big[(n + 1/2) t\big] \d t + I_n.
\end{align*}
Comme $f$ est de classe $\mathscr{C}^1$, d'après le lemme de \textsc{Lebesgue},
\[
\lim_{n\to+\infty} \int_0^\pi f(t) \sin\mathopen{}\big[(n + 1/2) t\big] \d t = 0.
\]

Finalement, $I = \frac{\pi}{2}$.
\end{reponses}
\end{solution}

%-----------
\subsection{Calcul via une intégrale à paramètre}

\begin{comment}
\begin{exercice}
    Exercice 167 p.179 (bestiaire.pdf) \\
    On considère l'application $f(x) = \int_0^\infty \frac{\sin(t)}{t} \e^{-xt} \d t$.
    \begin{enumerate}
        \item Montrer que $f \in \mathscr{C}^1(\Rpe)$.
        \item En déduire une forme explicite de $f$ sur $\Rpe$. 
        \item Montrer que $f$ est continue à l'origine. 
        \item En déduire que $\int_0^\infty \frac{\sin(t)}{t} \d t = \frac{\pi }{2}$.
    \end{enumerate}
\end{exercice}
\end{comment}

\begin{comment}
\begin{exercice}
Soit la transformée de \textsc{Laplace} de la fonction sinus cardinal:
$$F:x \to \int_{0}^{+ \infty} \exp(-xt) \frac{\sin (t)}{t} \d t$$
    
\begin{enumerate}
    \item \emph{Montrer que $F$ est définie sur $\Rp$.}
    \item \emph{Calculer $F$ sur $\Rpe$, en déduire la valeur de la fonction de \textsc{Dirichlet}}
\end{enumerate}
\end{exercice}
\begin{preuve}
\begin{enumerate}
\item Montrons que $F$ est bien définie. D'une part, $\lim_{t \to 0} \exp(-x t) \frac{\sin t}{t} = 1$ donc l'intégrande est prolongeable par continuité en $0$ et est donc intégrable sur $]0, 1]$.
\begin{enumerate}
        \item Si $x > 0$, majorer l'intégrande par $t \mapsto \exp(-xt)$.
        \item Si $x = 0$, montrer le prolongement par continuité de la fonction sinus cardinal en $0$ puis intégrer la fonction sinus cardinal par parties sur $[1, +\infty]$.
    \end{enumerate}
\end{enumerate}
\end{preuve}
\end{comment}

%---------------

\todoinline{J'ai supprimé la première question de l'exercice qui n'est qu'une redite de la première partie de ce thème. Je détaille aussi les questions pour être plus facile.}

\todoinline{Mettre au propre la citation ?}

\begin{exercice}[{RMS - Autres Écoles - 178}{IMT}{16}]
On note $I = \int_0^{+\infty} \frac{\sin t}{t} \d t$ et $F(x) = \int_0^{+\infty} \frac{\sin t}{t} (1 - \e^{-x t}) \d t$. On pose $u : t \mapsto \int_0^x \frac{\sin t}{t} \d t$.
\begin{questions}
% \item Montrer que l'intégrale $I$ est bien définie.
\item Montrer que la fonction $F$ est bien définie sur $\interfo{0}{+\infty}$.

\item Montrer que $F$ est continue sur $\interfo{0}{+\infty}$.\\
\emph{Indication :} Utiliser une intégration par parties avec la fonction $u$.

\item Montrer que $F$ est dérivable sur $\interoo{0}{+\infty}$.

\item En déduire la valeur de $I$.
\end{questions}
\end{exercice}

\begin{solution}
\marginnote[0cm]{Reformulation de la démonstration \ref{demo2existenceIntegraleDirichlet}}
\begin{reponses}
% \item Les fonctions $\fonctionligne[u]{t}{1 - \cos(t)}$ et $\fonctionligne[v]{t}{\frac{1}{t}}$ sont de classe $\mathscr{C}^1$ sur $\Rpe$. De plus, le produit $u v$ a des limites nulles en $0$ et $+\infty$. Ainsi, d'après la formule d'intégration par parties, l'intégrale $I$ a même nature que
% \[
% \int_0^{+\infty} \frac{1 - \cos t}{t^2} \d t,
% \]
% dont l'intégrande est continue sur $\interoo{0}{+\infty}$, prolongeable par continuité par $\frac{1}{2}$ en $0$ et majorée par la fonction $t \mapsto \frac{1}{t^2}$ en $+\infty$, donc est intégrable.

\item 
\begin{enumerate}
\item D'une part, $F(0) = 0$. D'autre part, pour $x > 0$, l'intégrale $F(x)$ est la différence entre $I$ et l'intégrale de $\fonctionligne[f]{(x, t)}{\frac{\sin t}{t} \e^{- x t}}$.
\begin{enumerate}[label=(\roman*)]
\item $f(x, \cdot\,)$ est continue sur $\Rpe$.
\item $f(x, \cdot\,)$ admet pour limite $1$ en $0$.
\item $\abs{f(x, \cdot\,)}$ est un $o(1/t^2)$ en $+\infty$.
\end{enumerate}
Ainsi, l'intégrale $F$ est bien définie sur $\Rp$.

\item Pour la continuité, on ne peut pas espérer appliquer le théorème de convergence dominée directement car le sinus cardinal $t \mapsto \frac{\sin t}{t}$ n'est pas intégrable. On utilise donc une intégration par parties en considérant les fonctions $\fonctionligne[u]{t}{\int_t^{+\infty} \frac{\sin u}{u} \d u}$ et $\fonctionligne[v]{t}{(1 - \e^{-x t})}$. Comme le produit $u\,v$ admet des limites nulles en~$0$ et $+\infty$, 
\begin{align*}
F(x) &= -x \int_{0}^{+\infty} u(t)\, \e^{- x t} \d t \\
&= -\int_0^{+\infty} u\mathopen{}\left(\frac{v}{x}\right) \e^{-v} \d v.
\end{align*}
Comme $u$ possède des limites en $0$ et $+\infty$, elle est bornée sur $\R_+$ et
\[
\abs{u\mathopen{}\left(\frac{v}{x}\right) \e^{-v}} \leqslant M \e^{-v},
\]
qui est une fonction intégrable. Ainsi, d'après le théorème de continuité sous le signe intégral, la fonction $F$ est continue sur $\R+$. En particulier,
\[
\lim_{x\to0} F(x) = -\int_0^{+\infty} \lim_{t\to+\infty} u(t)\, \e^{-v} \d v = 0.
\]
Ainsi, $F$ est continue en $0$.

\item Avec les notations de la question précédente,
\[
\forall x \in \interfo{a}{+\infty},\quad \abs{\frac{\partial f}{\partial x}(x, t)} \leqslant \e^{-a t}.
\]
Ainsi, la fonction $F$ est de classe $\mathscr{C}^1$ sur $\interfo{a}{+\infty}$, donc sur $\Rpe$.
\end{enumerate}

De plus, pour tout $x > 0$,
\[
F'(x) = \int_0^{+\infty} \sin(t)\, \e^{-x t} \d t = \frac{1}{1 + x^2}.
\]

\item D'après la question précédente, il existe une constante $C \in \R$ telle que
\[
\forall x > 0,\quad F(x) = C + \arctan(x).
\]

Comme $F(0) = 0$, et par continuité de la fonction $F$, alors $C = 0$.

Enfin, $F(x) = I - \int_0^{+\infty} \frac{\sin t}{t}\, \e^{-x t} \d t$ et
\[
\abs{\int_0^{+\infty} \frac{\sin t}{t}\, \e^{-x t} \d t} \leqslant \int_0^{+\infty} \e^{-x t} \d t = \frac{1}{x}.
\]
Ainsi, $\lim\limits_{x\to+\infty} F(x) = I$ et $I = \frac{\pi}{2}$.
\end{reponses}
\end{solution}

%-----------
\subsection{Une preuve par équations différentielles}

\begin{exercice}
    Exercice 170 p. 184 \\
    Soient $f : x \mapsto \int_0^\infty \frac{\sin(t)}{t+x} \d t$ et $g : x \mapsto \int_0^\infty \frac{\e^{-xt}}{t^2 + 1} \d t$. 
    \begin{questions}
        \item Montrer que $g$ est de classe $\mathscr{C}^2$ sur $\Re$ et solution de l'équation différentielle $y'' + y = \frac{1}{x}$.
        
        \item Montrer que $f$ est de classe $\mathscr{C}^2$ sur $\Re$ et solution de l'équation différentielle $y'' + y = \frac{1}{x}$.\\
        \emph{Indication : } On pourra commencer par montrer que $f(x) = \int_0^\infty \frac{1 - \cos(t)}{(t+x)^2} \d t$.
        \item En déduire que $f-g$ est $2 \pi$-périodique (sur son domaine de définition).

        \item Montrer que $f$ et $g$ tendent vers $0$ en $+\infty$, puis que $f = g$.
        \item En déduire la valeur de $\int_0^\infty \frac{\sin(t)}{t} \d t$.
    \end{questions}
\end{exercice}

\begin{solution}
\begin{reponses}
\item On pose $\fonctionligne[g]{(x, t)}{\frac{\e^{-x y}}{t^2 + 1}}$. Alors, pour tout $(x, t) \in \interfo{a}{+\infty} \times \Rp$,
\begin{align*}
\abs{g(x, t)} &\leqslant \frac{1}{1 + t^2},\\
\abs{\frac{\partial g}{\partial x}(x, t)} &\leqslant \frac{t \e^{-a t}}{1 + t^2},\\
\abs{\frac{\partial^2 g}{\partial x^2}(x, t)} &\leqslant \frac{t^2 \e^{-a t}}{1 + t^2}.
\end{align*}
\marginnote[-7pt]{\theoremeutilise{Théorème de dérivation sous le signe intégral}}Ainsi, d'après le théorème de dérivation sous le signe intégral, la fonction $g$ est de classe $\mathscr{C}^2$ sur $\Rpe$ et
\begin{align*}
g''(x)
&= \int_0^{+\infty} \frac{t^2 \e^{-x t}}{1 + t^2} \d t\\
&= \int_0^{+\infty} \frac{(1 + t^2 - 1) \e^{-x t}}{1 + t^2} \d t\\
&= \int_0^{+\infty} \e^{-x t} \d t - g(x)\\
&= \frac{1}{x} - g(x).
\end{align*}

\item En utilisant l'intégration par parties vue au début du chapitre,
\begin{align*}
f(x)
&= \left[\frac{1 - \cos(t)}{t + x}\right]_0^{+\infty} + \int_0^{+\infty} \frac{1 - \cos(t)}{(t + x)^2} \d t\\
&= \int_0^{+\infty} \frac{1 - \cos(t)}{(t + x)^2} \d t.
\end{align*}
En posant $h(x, y) = \frac{1 - \cos(t)}{(t + x)^2}$ pour tout $(x, t) \in \interfo{a}{+\infty} \times \Rp$,
\begin{align*}
\abs{h(x, t)} &\leqslant \frac{1}{(t + a)^2}\\
\abs{\frac{\partial h}{\partial x}(x, t)} &\leqslant \frac{2}{(t + a)^3}\\
\abs{\frac{\partial^2 h}{\partial x^2}(x, t)} &\leqslant \frac{6}{(t + a)^3}.
\end{align*}
Ainsi, la fonction $f$ est de classe $\mathscr{C}^2$ sur $\R_+^*$ et, à l'aide d'intégrations par parties,
\begin{align*}
f''(x)
&= \int_0^{+\infty} \frac{6 (1 - \cos t)}{(t + x)^4} \d t\\
&= \int_0^{+\infty} \frac{2 \sin t}{(t + x)^3} \d t\\
&= \int_0^{+\infty} \frac{\cos t}{(t + x)^2} \d t\\
&= \int_0^{+\infty} \frac{\d t}{(t + x)^2} - \int_0^{+\infty} \frac{1 - \cos t}{(t + x)^2} \d t\\
&= \frac{1}{x} - f(x).
\end{align*}

\item Comme $f - g$ est solution de l'équation différentielle $y'' - y = 0$, alors $f - g$ est $2\pi$-périodique.

\item En utilisant l'expression précédente,
\begin{align*}
f(x)
&= \frac{1}{x} - \int_0^{+\infty} \frac{2 \sin t}{(t + x)^3} \d t.
\end{align*}
Or, $\abs{\int_0^{+\infty} \frac{2 \sin t}{(t + x)^3} \d t} \leqslant \frac{2}{x^2}$. Ainsi, $f(x) \sim_{+\infty} \frac{1}{x}$.

De même, en $+\infty$,
\begin{align*}
\abs{g(x)}
&\leqslant \int_0^{+\infty} \e^{- x t} \d t
\leqslant \frac{1}{x}.
\end{align*}

D'après les points précédents, $\lim_{x\to+\infty} (f - g) = 0$. Comme $f - g$ est périodique, alors $f - g$ est la fonction nulle.

Ainsi,
\[
\forall\, x > 0,\quad \int_0^{+\infty} \frac{\sin t}{t + x} \d t = \int_0^{+\infty} \frac{\e^{-x t}}{t^2 + 1} \d t.
\]

\item \marginnote[-7pt]{\theoremeutilise{Théorème de convergence dominée}}Comme $\abs{\frac{\e^{-x t}}{t^2 + 1}} \leqslant \frac{1}{1 + t^2}$, d'après le théorème de convergence dominée,
\[
\lim_{x\to 0} \int_0^{+\infty} \frac{\e^{-x t}}{t^2 + 1} \d t
= \int_0^{+\infty} \frac{\d t}{1 + t^2}
= \frac{\pi}{2}.
\]

De plus,
\begin{align*}
\abs{\int_0^{+\infty} \frac{\sin t}{t + x} \d t - \int_0^{+\infty} \frac{\sin t}{t} \d t}
\leqslant \int_0^1 \frac{x \abs{\sin t}}{t (t + x)} \d t + \int_1^{+\infty} \frac{x \abs{\sin t}}{t (t + x)} \d t.
\end{align*}

Le premier terme tend vers $0$ par convergence dominée, avec la domination $t \mapsto \frac{\abs{\sin t}}{t}$ qui est bien intégrable sur $\interff{0}{1}$.

Le second terme tend vers $0$ par domination par $x \int_1^{+\infty} \frac{\d t}{t^2}$.

Finalement, on obtient
\[
\int_0^{+\infty} \frac{\sin t}{t} \d t
= \int_0^{+\infty} \frac{1}{1 + t^2} \d t
= \frac{\pi}{2}.
\]
\end{reponses}
\end{solution}


%-----------
\subsection{Pour aller plus loin}

\todoinline{Mettre une citation pour ENSAIT-MP-1996}
À l'aide d'intégrations par parties, on peut montrer que
\[
\int_0^{+\infty} \sinc(t) \d t
= \int_0^{+\infty} \sinc^2(t) \d t
= \int_0^{+\infty} \sinc^4(t) \d t
= \frac{\pi}{2}.
\]

Plus généralement, les intégrales de \cite{Browein} généralisent en un certain sens l'intégrale de Dirichlet. En utilisant des calculs utilisant la tranformation de Fourier, on peut montrer que
\todoinline{Mettre une citation pour \url{https://perso.telecom-paristech.fr/rioul/publis/202301rioul.pdf}}
\begin{align*}
\int_0^{\pi/2} \sinc(t) \d t &= \pi\\
\int_0^{\pi/2} \sinc(t) \sinc(t/3) \d t &= \pi\\
\int_0^{\pi/2} \sinc(t) \sinc(t/3) \sinc(t/5) \d t &= \pi\\
\int_0^{\pi/2} \sinc(t) \sinc(t/3) \sinc(t/5) \sinc(t/7) \d t &= \pi\\
\int_0^{\pi/2} \sinc(t) \sinc(t/3) \sinc(t/5) \sinc(t/7) \sinc(t/9) \d t &= \pi\\
\int_0^{\pi/2} \sinc(t) \sinc(t/3) \sinc(t/5) \sinc(t/7) \sinc(t/9) \sinc(t/11) \d t &= \pi\\
\int_0^{\pi/2} \sinc(t) \sinc(t/3) \sinc(t/5) \sinc(t/7) \sinc(t/9) \sinc(t/11) \sinc(t/13) \d t &= \pi,
\end{align*}
cette série s'arrêtant ici ! 

\section{Intégrale de \textsc{Gauss}}

\begin{marginfigure}[0cm]
    \centering
    % Author: Izaak Neutelings (August, 2017)


\tikzset{>=latex} % for LaTeX arrow head
\contourlength{1.2pt}
\usetikzlibrary{positioning,calc}
\usetikzlibrary{backgrounds}% required for 'inner frame sep'
%\usepackage{adjustbox} % add whitespace (trim)

% define gaussian pdf and cdf
\pgfmathdeclarefunction{gauss}{3}{%
  \pgfmathparse{1/(#3*sqrt(2*pi))*exp(-((#1-#2)^2)/(2*#3^2))}%
}
\colorlet{mydarkblue}{blue!30!black}

% to fill an area under function
\usepgfplotslibrary{fillbetween}
\usetikzlibrary{patterns}
\pgfplotsset{compat=1.12} % TikZ coordinates <-> axes coordinates
% https://tex.stackexchange.com/questions/240642/add-vertical-line-of-equation-x-2-and-shade-a-region-in-graph-by-pgfplots

% plot aspect ratio
%\def\axisdefaultwidth{8cm}
%\def\axisdefaultheight{6cm}

% number of sample points
\def\N{50}

\begin{tikzpicture}[scale=1]
  \message{Cumulative probability^^J}
  
  \def\B{0};
  \def\Bs{6.0};
  \def\xmax{\B+3.5*\Bs};
  \def\ymin{{-0.1*gauss(\B,\B,\Bs)}};
  \def\h{0.07*gauss(\B,\B,\Bs)};
  \def\a{\B-0.8*\Bs};
  
  \begin{axis}[every axis plot post/.append style={
               mark=none,
               domain={-1*(\xmax)}:{1*(\xmax)},samples=\N,smooth},
               % xmin={-1*(\xmax)}, 
               xmax={1.06*(\xmax)},
               ymin=\ymin, ymax={1.3*gauss(\B,\B,\Bs)},
               axis lines=middle,
               axis line style=thick,
               axis line style={-latex},
               ticks=none,
               xlabel=$x$,
               every axis x label/.style={at={(current axis.right of origin)},anchor=north},
               y=700pt,
               clip=false
              ]
    
    % PLOTS
    \addplot[blue,thick,name path=B] {gauss(x,\B,\Bs)};
    % FILL
    \path[name path=xaxis]
      (0,0) -- (\pgfkeysvalueof{/pgfplots/xmax},0);
    \addplot[blue!25, opacity=0.9] fill between[of=xaxis and B];
    % LINES
    \node[blue,above left] at ({-0.7*(\B+\Bs)},{1.2*gauss(\B+\Bs,\B,\Bs)}) {$x \mapsto \mathrm{e}^{-x^2}$};
    \node[blue!60!black] at ({0},{0.6*gauss(0.85*(\a),\B,\Bs)}) {$\sqrt{\pi}$};
    
  \end{axis}
\end{tikzpicture}
    \caption{Intégrale de \textsc{Gauss}}
\end{marginfigure}

\begin{theo}{}
On montre que
\[
\int_0^{+\infty} \e^{-x^2} \d x = \lim_{n\to+\infty} \int_0^{\sqrt{n}} \left(1 - \frac{x^2}{n}\right)^n \d x.
\]

Ainsi,
\begin{equation}\label{eqIntGauss}
        \int_{0}^{+\infty} \e^{-x^2} \d x = \frac{\sqrt{\pi}}{2}        
    \end{equation}
\end{theo}

%-----------
\subsection{Calculs de l'intégrale}

\begin{exercice}
\marginnote[0cm]{Source : \cite{maths-france} Planche no 13. Suites et séries d’intégrales}
On note $I = \int_0^{+\infty} \e^{-x^2} \d x$ et on pose $f : x \mapsto \e^{-x^2}$.

\begin{enumerate}
\item Montrer que $I$ est une intégrale convergente.

On propose ensuite de déterminer la valeur de $I$.

\item \textbf{Première méthode:} \say{ à la main }. \\ 
Pour $n \in \Ne$, on pose
$$
g_n(x) \defeq
\begin{cases}
\left(1 - \frac{x}{n} \right)^n &\text{si } x \in [0, n] \\
0 &\text{si } x \geqslant n
\end{cases}.
$$

On pose $g : x \mapsto \e^{-x}$ et $h_n = g - g_n$·
\begin{enumerate}
\item Montrer que $h_n$ atteint son maximum sur $[0, n]$. On notera $x_n$ l'abscisse d'un point en laquelle $h_n$ atteint ce maximum.

\item Montrer que $h_n$ est à valeurs positives.

\item Montrer que $h_n(x_n) = \frac{x_n \e^{-x_n}}{n}$.

\item En étudiant la fonction $u \mapsto u \e^{-u}$, en déduire que
\[
\module{h_n} \leq \frac{1}{n \e}.
\]

\item On pose $I_n= \int_0^{+\infty} g_n(x^2) \d x$. Montrer que $\module{I_n - I} \leq \frac{1}{\e \sqrt{n}} + \int_{\sqrt{n}}^{+\infty} \e^{-x^2} \d x$.

\item En déduire le résultat attendu.
\end{enumerate}

\item \textbf{Deuxième méthode:} \say{ avec le théorème de convergence dominée et les intégrales de Wallis }. \\
On pose $f_n(x) = \left(1 - \frac{x^2}{n}\right)^n \indicatrice{[0,\sqrt{n}[}(x)$.
\begin{enumerate}
\item Montrer que $(f_n)_{n\in\N}$ converge simplement vers $f$.

\item À l'aide du théorème de convergence dominée, en déduire que 
\[
\lim\limits_{n\to+\infty} \int_{\R_+} f_n = \int_0^{+\infty} \e^{-x^2} \d x.
\]
\end{enumerate}

\item On rappelle que, d'après les résultats sur les intégrales de Wallis,
\[
\lim_{n\to+\infty} \sqrt{n} \int_0^{\frac{\pi}{2}} \cos^{2n+1}(t) \d t = \frac{\sqrt{\pi}}{2}.
\]

En déduire la valeur de $I$.
\end{enumerate}
% Pour $n \in \Ne$, on pose
        % $$
        % f_n(x) \defeq
        % \begin{cases}
            % \left(1 - \frac{x^2}{n} \right)^n &\text{si } x \in [0, \sqrt{n}] \\
            % 0 &\text{si } x > \sqrt{n}
        % \end{cases}.
        % $$
        % \begin{enumerate}
            % \item Montrer que la suite $(f_n)_{n \in \Ne}$ converge simplement sur $\Rp$ vers la fonction $f:x \mapsto \e^{-x^2}$.
            % \item À l'aide de la convergence dominée, calculer l'intégrale de \textsc{Gauss}.
        % \end{enumerate}
\end{exercice}

\begin{marginfigure}[2cm]
    \centering
    %% Creator: Matplotlib, PGF backend
%%
%% To include the figure in your LaTeX document, write
%%   \input{<filename>.pgf}
%%
%% Make sure the required packages are loaded in your preamble
%%   \usepackage{pgf}
%%
%% Also ensure that all the required font packages are loaded; for instance,
%% the lmodern package is sometimes necessary when using math font.
%%   \usepackage{lmodern}
%%
%% Figures using additional raster images can only be included by \input if
%% they are in the same directory as the main LaTeX file. For loading figures
%% from other directories you can use the `import` package
%%   \usepackage{import}
%%
%% and then include the figures with
%%   \import{<path to file>}{<filename>.pgf}
%%
%% Matplotlib used the following preamble
%%   
%%   \usepackage{fontspec}
%%   \setmainfont{DejaVuSerif.ttf}[Path=\detokenize{/home/wayoff/.pyenv/versions/3.8.10/lib/python3.8/site-packages/matplotlib/mpl-data/fonts/ttf/}]
%%   \setsansfont{DejaVuSans.ttf}[Path=\detokenize{/home/wayoff/.pyenv/versions/3.8.10/lib/python3.8/site-packages/matplotlib/mpl-data/fonts/ttf/}]
%%   \setmonofont{DejaVuSansMono.ttf}[Path=\detokenize{/home/wayoff/.pyenv/versions/3.8.10/lib/python3.8/site-packages/matplotlib/mpl-data/fonts/ttf/}]
%%   \makeatletter\@ifpackageloaded{underscore}{}{\usepackage[strings]{underscore}}\makeatother
%%
\begingroup%
\makeatletter%
\begin{pgfpicture}%
\pgfpathrectangle{\pgfpointorigin}{\pgfqpoint{3.000000in}{2.500000in}}%
\pgfusepath{use as bounding box, clip}%
\begin{pgfscope}%
\pgfsetbuttcap%
\pgfsetmiterjoin%
\definecolor{currentfill}{rgb}{1.000000,1.000000,1.000000}%
\pgfsetfillcolor{currentfill}%
\pgfsetlinewidth{0.000000pt}%
\definecolor{currentstroke}{rgb}{1.000000,1.000000,1.000000}%
\pgfsetstrokecolor{currentstroke}%
\pgfsetdash{}{0pt}%
\pgfpathmoveto{\pgfqpoint{0.000000in}{0.000000in}}%
\pgfpathlineto{\pgfqpoint{3.000000in}{0.000000in}}%
\pgfpathlineto{\pgfqpoint{3.000000in}{2.500000in}}%
\pgfpathlineto{\pgfqpoint{0.000000in}{2.500000in}}%
\pgfpathlineto{\pgfqpoint{0.000000in}{0.000000in}}%
\pgfpathclose%
\pgfusepath{fill}%
\end{pgfscope}%
\begin{pgfscope}%
\pgfsetbuttcap%
\pgfsetmiterjoin%
\definecolor{currentfill}{rgb}{1.000000,1.000000,1.000000}%
\pgfsetfillcolor{currentfill}%
\pgfsetlinewidth{0.000000pt}%
\definecolor{currentstroke}{rgb}{0.000000,0.000000,0.000000}%
\pgfsetstrokecolor{currentstroke}%
\pgfsetstrokeopacity{0.000000}%
\pgfsetdash{}{0pt}%
\pgfpathmoveto{\pgfqpoint{0.316667in}{0.571603in}}%
\pgfpathlineto{\pgfqpoint{2.850000in}{0.571603in}}%
\pgfpathlineto{\pgfqpoint{2.850000in}{2.350000in}}%
\pgfpathlineto{\pgfqpoint{0.316667in}{2.350000in}}%
\pgfpathlineto{\pgfqpoint{0.316667in}{0.571603in}}%
\pgfpathclose%
\pgfusepath{fill}%
\end{pgfscope}%
\begin{pgfscope}%
\pgfpathrectangle{\pgfqpoint{0.316667in}{0.571603in}}{\pgfqpoint{2.533333in}{1.778397in}}%
\pgfusepath{clip}%
\pgfsetrectcap%
\pgfsetroundjoin%
\pgfsetlinewidth{0.803000pt}%
\definecolor{currentstroke}{rgb}{0.690196,0.690196,0.690196}%
\pgfsetstrokecolor{currentstroke}%
\pgfsetdash{}{0pt}%
\pgfpathmoveto{\pgfqpoint{0.431818in}{0.571603in}}%
\pgfpathlineto{\pgfqpoint{0.431818in}{2.350000in}}%
\pgfusepath{stroke}%
\end{pgfscope}%
\begin{pgfscope}%
\pgfsetbuttcap%
\pgfsetroundjoin%
\definecolor{currentfill}{rgb}{0.000000,0.000000,0.000000}%
\pgfsetfillcolor{currentfill}%
\pgfsetlinewidth{0.803000pt}%
\definecolor{currentstroke}{rgb}{0.000000,0.000000,0.000000}%
\pgfsetstrokecolor{currentstroke}%
\pgfsetdash{}{0pt}%
\pgfsys@defobject{currentmarker}{\pgfqpoint{0.000000in}{-0.048611in}}{\pgfqpoint{0.000000in}{0.000000in}}{%
\pgfpathmoveto{\pgfqpoint{0.000000in}{0.000000in}}%
\pgfpathlineto{\pgfqpoint{0.000000in}{-0.048611in}}%
\pgfusepath{stroke,fill}%
}%
\begin{pgfscope}%
\pgfsys@transformshift{0.431818in}{0.571603in}%
\pgfsys@useobject{currentmarker}{}%
\end{pgfscope}%
\end{pgfscope}%
\begin{pgfscope}%
\definecolor{textcolor}{rgb}{0.000000,0.000000,0.000000}%
\pgfsetstrokecolor{textcolor}%
\pgfsetfillcolor{textcolor}%
\pgftext[x=0.431818in,y=0.474381in,,top]{\color{textcolor}\sffamily\fontsize{10.000000}{12.000000}\selectfont \(\displaystyle 0\)}%
\end{pgfscope}%
\begin{pgfscope}%
\definecolor{textcolor}{rgb}{0.000000,0.000000,0.000000}%
\pgfsetstrokecolor{textcolor}%
\pgfsetfillcolor{textcolor}%
\pgftext[x=1.583333in,y=0.284413in,,top]{\color{textcolor}\sffamily\fontsize{10.000000}{12.000000}\selectfont \(\displaystyle t\)}%
\end{pgfscope}%
\begin{pgfscope}%
\pgfpathrectangle{\pgfqpoint{0.316667in}{0.571603in}}{\pgfqpoint{2.533333in}{1.778397in}}%
\pgfusepath{clip}%
\pgfsetrectcap%
\pgfsetroundjoin%
\pgfsetlinewidth{0.803000pt}%
\definecolor{currentstroke}{rgb}{0.690196,0.690196,0.690196}%
\pgfsetstrokecolor{currentstroke}%
\pgfsetdash{}{0pt}%
\pgfpathmoveto{\pgfqpoint{0.316667in}{0.571603in}}%
\pgfpathlineto{\pgfqpoint{2.850000in}{0.571603in}}%
\pgfusepath{stroke}%
\end{pgfscope}%
\begin{pgfscope}%
\pgfsetbuttcap%
\pgfsetroundjoin%
\definecolor{currentfill}{rgb}{0.000000,0.000000,0.000000}%
\pgfsetfillcolor{currentfill}%
\pgfsetlinewidth{0.803000pt}%
\definecolor{currentstroke}{rgb}{0.000000,0.000000,0.000000}%
\pgfsetstrokecolor{currentstroke}%
\pgfsetdash{}{0pt}%
\pgfsys@defobject{currentmarker}{\pgfqpoint{-0.048611in}{0.000000in}}{\pgfqpoint{-0.000000in}{0.000000in}}{%
\pgfpathmoveto{\pgfqpoint{-0.000000in}{0.000000in}}%
\pgfpathlineto{\pgfqpoint{-0.048611in}{0.000000in}}%
\pgfusepath{stroke,fill}%
}%
\begin{pgfscope}%
\pgfsys@transformshift{0.316667in}{0.571603in}%
\pgfsys@useobject{currentmarker}{}%
\end{pgfscope}%
\end{pgfscope}%
\begin{pgfscope}%
\definecolor{textcolor}{rgb}{0.000000,0.000000,0.000000}%
\pgfsetstrokecolor{textcolor}%
\pgfsetfillcolor{textcolor}%
\pgftext[x=0.150000in, y=0.518842in, left, base]{\color{textcolor}\sffamily\fontsize{10.000000}{12.000000}\selectfont \(\displaystyle 0\)}%
\end{pgfscope}%
\begin{pgfscope}%
\pgfpathrectangle{\pgfqpoint{0.316667in}{0.571603in}}{\pgfqpoint{2.533333in}{1.778397in}}%
\pgfusepath{clip}%
\pgfsetrectcap%
\pgfsetroundjoin%
\pgfsetlinewidth{0.803000pt}%
\definecolor{currentstroke}{rgb}{0.690196,0.690196,0.690196}%
\pgfsetstrokecolor{currentstroke}%
\pgfsetdash{}{0pt}%
\pgfpathmoveto{\pgfqpoint{0.316667in}{2.265314in}}%
\pgfpathlineto{\pgfqpoint{2.850000in}{2.265314in}}%
\pgfusepath{stroke}%
\end{pgfscope}%
\begin{pgfscope}%
\pgfsetbuttcap%
\pgfsetroundjoin%
\definecolor{currentfill}{rgb}{0.000000,0.000000,0.000000}%
\pgfsetfillcolor{currentfill}%
\pgfsetlinewidth{0.803000pt}%
\definecolor{currentstroke}{rgb}{0.000000,0.000000,0.000000}%
\pgfsetstrokecolor{currentstroke}%
\pgfsetdash{}{0pt}%
\pgfsys@defobject{currentmarker}{\pgfqpoint{-0.048611in}{0.000000in}}{\pgfqpoint{-0.000000in}{0.000000in}}{%
\pgfpathmoveto{\pgfqpoint{-0.000000in}{0.000000in}}%
\pgfpathlineto{\pgfqpoint{-0.048611in}{0.000000in}}%
\pgfusepath{stroke,fill}%
}%
\begin{pgfscope}%
\pgfsys@transformshift{0.316667in}{2.265314in}%
\pgfsys@useobject{currentmarker}{}%
\end{pgfscope}%
\end{pgfscope}%
\begin{pgfscope}%
\definecolor{textcolor}{rgb}{0.000000,0.000000,0.000000}%
\pgfsetstrokecolor{textcolor}%
\pgfsetfillcolor{textcolor}%
\pgftext[x=0.150000in, y=2.212553in, left, base]{\color{textcolor}\sffamily\fontsize{10.000000}{12.000000}\selectfont \(\displaystyle 1\)}%
\end{pgfscope}%
\begin{pgfscope}%
\pgfpathrectangle{\pgfqpoint{0.316667in}{0.571603in}}{\pgfqpoint{2.533333in}{1.778397in}}%
\pgfusepath{clip}%
\pgfsetrectcap%
\pgfsetroundjoin%
\pgfsetlinewidth{1.505625pt}%
\definecolor{currentstroke}{rgb}{0.579608,0.770196,0.873725}%
\pgfsetstrokecolor{currentstroke}%
\pgfsetdash{}{0pt}%
\pgfpathmoveto{\pgfqpoint{0.431818in}{2.265314in}}%
\pgfpathlineto{\pgfqpoint{0.454872in}{2.264254in}}%
\pgfpathlineto{\pgfqpoint{0.477925in}{2.261072in}}%
\pgfpathlineto{\pgfqpoint{0.500979in}{2.255768in}}%
\pgfpathlineto{\pgfqpoint{0.524032in}{2.248343in}}%
\pgfpathlineto{\pgfqpoint{0.549391in}{2.237726in}}%
\pgfpathlineto{\pgfqpoint{0.574749in}{2.224542in}}%
\pgfpathlineto{\pgfqpoint{0.600108in}{2.208790in}}%
\pgfpathlineto{\pgfqpoint{0.625467in}{2.190472in}}%
\pgfpathlineto{\pgfqpoint{0.653131in}{2.167561in}}%
\pgfpathlineto{\pgfqpoint{0.680795in}{2.141596in}}%
\pgfpathlineto{\pgfqpoint{0.708459in}{2.112575in}}%
\pgfpathlineto{\pgfqpoint{0.738428in}{2.077689in}}%
\pgfpathlineto{\pgfqpoint{0.768397in}{2.039218in}}%
\pgfpathlineto{\pgfqpoint{0.798367in}{1.997161in}}%
\pgfpathlineto{\pgfqpoint{0.830641in}{1.947861in}}%
\pgfpathlineto{\pgfqpoint{0.862916in}{1.894402in}}%
\pgfpathlineto{\pgfqpoint{0.895191in}{1.836785in}}%
\pgfpathlineto{\pgfqpoint{0.929771in}{1.770439in}}%
\pgfpathlineto{\pgfqpoint{0.964351in}{1.699320in}}%
\pgfpathlineto{\pgfqpoint{1.001236in}{1.618198in}}%
\pgfpathlineto{\pgfqpoint{1.038122in}{1.531646in}}%
\pgfpathlineto{\pgfqpoint{1.077312in}{1.433734in}}%
\pgfpathlineto{\pgfqpoint{1.116503in}{1.329691in}}%
\pgfpathlineto{\pgfqpoint{1.155694in}{1.219517in}}%
\pgfpathlineto{\pgfqpoint{1.197190in}{1.096180in}}%
\pgfpathlineto{\pgfqpoint{1.238686in}{0.965969in}}%
\pgfpathlineto{\pgfqpoint{1.282487in}{0.821069in}}%
\pgfpathlineto{\pgfqpoint{1.326289in}{0.668510in}}%
\pgfpathlineto{\pgfqpoint{1.353953in}{0.571603in}}%
\pgfpathlineto{\pgfqpoint{2.734848in}{0.571603in}}%
\pgfpathlineto{\pgfqpoint{2.734848in}{0.571603in}}%
\pgfusepath{stroke}%
\end{pgfscope}%
\begin{pgfscope}%
\pgfpathrectangle{\pgfqpoint{0.316667in}{0.571603in}}{\pgfqpoint{2.533333in}{1.778397in}}%
\pgfusepath{clip}%
\pgfsetrectcap%
\pgfsetroundjoin%
\pgfsetlinewidth{1.505625pt}%
\definecolor{currentstroke}{rgb}{0.290980,0.594510,0.789020}%
\pgfsetstrokecolor{currentstroke}%
\pgfsetdash{}{0pt}%
\pgfpathmoveto{\pgfqpoint{0.431818in}{2.265314in}}%
\pgfpathlineto{\pgfqpoint{0.454872in}{2.264254in}}%
\pgfpathlineto{\pgfqpoint{0.477925in}{2.261074in}}%
\pgfpathlineto{\pgfqpoint{0.500979in}{2.255782in}}%
\pgfpathlineto{\pgfqpoint{0.524032in}{2.248386in}}%
\pgfpathlineto{\pgfqpoint{0.549391in}{2.237838in}}%
\pgfpathlineto{\pgfqpoint{0.574749in}{2.224787in}}%
\pgfpathlineto{\pgfqpoint{0.600108in}{2.209262in}}%
\pgfpathlineto{\pgfqpoint{0.627772in}{2.189546in}}%
\pgfpathlineto{\pgfqpoint{0.655436in}{2.166984in}}%
\pgfpathlineto{\pgfqpoint{0.683100in}{2.141638in}}%
\pgfpathlineto{\pgfqpoint{0.713069in}{2.111120in}}%
\pgfpathlineto{\pgfqpoint{0.743039in}{2.077520in}}%
\pgfpathlineto{\pgfqpoint{0.775313in}{2.038016in}}%
\pgfpathlineto{\pgfqpoint{0.809893in}{1.992044in}}%
\pgfpathlineto{\pgfqpoint{0.846779in}{1.939084in}}%
\pgfpathlineto{\pgfqpoint{0.885970in}{1.878683in}}%
\pgfpathlineto{\pgfqpoint{0.927466in}{1.810494in}}%
\pgfpathlineto{\pgfqpoint{0.973572in}{1.730195in}}%
\pgfpathlineto{\pgfqpoint{1.024290in}{1.637185in}}%
\pgfpathlineto{\pgfqpoint{1.084228in}{1.522337in}}%
\pgfpathlineto{\pgfqpoint{1.167220in}{1.357911in}}%
\pgfpathlineto{\pgfqpoint{1.296319in}{1.102119in}}%
\pgfpathlineto{\pgfqpoint{1.351647in}{0.997576in}}%
\pgfpathlineto{\pgfqpoint{1.397754in}{0.914996in}}%
\pgfpathlineto{\pgfqpoint{1.436945in}{0.849085in}}%
\pgfpathlineto{\pgfqpoint{1.471525in}{0.794903in}}%
\pgfpathlineto{\pgfqpoint{1.503799in}{0.748245in}}%
\pgfpathlineto{\pgfqpoint{1.533769in}{0.708748in}}%
\pgfpathlineto{\pgfqpoint{1.561433in}{0.675928in}}%
\pgfpathlineto{\pgfqpoint{1.586791in}{0.649205in}}%
\pgfpathlineto{\pgfqpoint{1.609845in}{0.627929in}}%
\pgfpathlineto{\pgfqpoint{1.632898in}{0.609736in}}%
\pgfpathlineto{\pgfqpoint{1.653646in}{0.596168in}}%
\pgfpathlineto{\pgfqpoint{1.674394in}{0.585415in}}%
\pgfpathlineto{\pgfqpoint{1.692837in}{0.578347in}}%
\pgfpathlineto{\pgfqpoint{1.711280in}{0.573737in}}%
\pgfpathlineto{\pgfqpoint{1.727417in}{0.571809in}}%
\pgfpathlineto{\pgfqpoint{1.752776in}{0.571603in}}%
\pgfpathlineto{\pgfqpoint{2.734848in}{0.571603in}}%
\pgfpathlineto{\pgfqpoint{2.734848in}{0.571603in}}%
\pgfusepath{stroke}%
\end{pgfscope}%
\begin{pgfscope}%
\pgfpathrectangle{\pgfqpoint{0.316667in}{0.571603in}}{\pgfqpoint{2.533333in}{1.778397in}}%
\pgfusepath{clip}%
\pgfsetrectcap%
\pgfsetroundjoin%
\pgfsetlinewidth{1.505625pt}%
\definecolor{currentstroke}{rgb}{0.090196,0.392941,0.670588}%
\pgfsetstrokecolor{currentstroke}%
\pgfsetdash{}{0pt}%
\pgfpathmoveto{\pgfqpoint{0.431818in}{2.265314in}}%
\pgfpathlineto{\pgfqpoint{0.454872in}{2.264254in}}%
\pgfpathlineto{\pgfqpoint{0.477925in}{2.261075in}}%
\pgfpathlineto{\pgfqpoint{0.500979in}{2.255786in}}%
\pgfpathlineto{\pgfqpoint{0.526337in}{2.247547in}}%
\pgfpathlineto{\pgfqpoint{0.551696in}{2.236795in}}%
\pgfpathlineto{\pgfqpoint{0.577055in}{2.223564in}}%
\pgfpathlineto{\pgfqpoint{0.602413in}{2.207893in}}%
\pgfpathlineto{\pgfqpoint{0.630077in}{2.188071in}}%
\pgfpathlineto{\pgfqpoint{0.657741in}{2.165474in}}%
\pgfpathlineto{\pgfqpoint{0.687711in}{2.137959in}}%
\pgfpathlineto{\pgfqpoint{0.717680in}{2.107402in}}%
\pgfpathlineto{\pgfqpoint{0.749955in}{2.071241in}}%
\pgfpathlineto{\pgfqpoint{0.784535in}{2.028953in}}%
\pgfpathlineto{\pgfqpoint{0.821420in}{1.980074in}}%
\pgfpathlineto{\pgfqpoint{0.860611in}{1.924222in}}%
\pgfpathlineto{\pgfqpoint{0.902107in}{1.861135in}}%
\pgfpathlineto{\pgfqpoint{0.948214in}{1.786902in}}%
\pgfpathlineto{\pgfqpoint{1.001236in}{1.697114in}}%
\pgfpathlineto{\pgfqpoint{1.065786in}{1.583137in}}%
\pgfpathlineto{\pgfqpoint{1.176442in}{1.382217in}}%
\pgfpathlineto{\pgfqpoint{1.266350in}{1.220906in}}%
\pgfpathlineto{\pgfqpoint{1.326289in}{1.117757in}}%
\pgfpathlineto{\pgfqpoint{1.377006in}{1.034788in}}%
\pgfpathlineto{\pgfqpoint{1.420807in}{0.967140in}}%
\pgfpathlineto{\pgfqpoint{1.462303in}{0.907044in}}%
\pgfpathlineto{\pgfqpoint{1.501494in}{0.854269in}}%
\pgfpathlineto{\pgfqpoint{1.538379in}{0.808434in}}%
\pgfpathlineto{\pgfqpoint{1.572959in}{0.769053in}}%
\pgfpathlineto{\pgfqpoint{1.605234in}{0.735568in}}%
\pgfpathlineto{\pgfqpoint{1.637509in}{0.705334in}}%
\pgfpathlineto{\pgfqpoint{1.667478in}{0.680219in}}%
\pgfpathlineto{\pgfqpoint{1.697448in}{0.657969in}}%
\pgfpathlineto{\pgfqpoint{1.727417in}{0.638569in}}%
\pgfpathlineto{\pgfqpoint{1.757386in}{0.621972in}}%
\pgfpathlineto{\pgfqpoint{1.787356in}{0.608094in}}%
\pgfpathlineto{\pgfqpoint{1.817325in}{0.596815in}}%
\pgfpathlineto{\pgfqpoint{1.847294in}{0.587975in}}%
\pgfpathlineto{\pgfqpoint{1.877264in}{0.581373in}}%
\pgfpathlineto{\pgfqpoint{1.909538in}{0.576482in}}%
\pgfpathlineto{\pgfqpoint{1.946424in}{0.573243in}}%
\pgfpathlineto{\pgfqpoint{1.990225in}{0.571769in}}%
\pgfpathlineto{\pgfqpoint{2.091660in}{0.571603in}}%
\pgfpathlineto{\pgfqpoint{2.734848in}{0.571603in}}%
\pgfpathlineto{\pgfqpoint{2.734848in}{0.571603in}}%
\pgfusepath{stroke}%
\end{pgfscope}%
\begin{pgfscope}%
\pgfpathrectangle{\pgfqpoint{0.316667in}{0.571603in}}{\pgfqpoint{2.533333in}{1.778397in}}%
\pgfusepath{clip}%
\pgfsetrectcap%
\pgfsetroundjoin%
\pgfsetlinewidth{1.505625pt}%
\definecolor{currentstroke}{rgb}{0.031373,0.188235,0.419608}%
\pgfsetstrokecolor{currentstroke}%
\pgfsetdash{}{0pt}%
\pgfpathmoveto{\pgfqpoint{0.431818in}{2.265314in}}%
\pgfpathlineto{\pgfqpoint{0.454872in}{2.264254in}}%
\pgfpathlineto{\pgfqpoint{0.477925in}{2.261076in}}%
\pgfpathlineto{\pgfqpoint{0.500979in}{2.255790in}}%
\pgfpathlineto{\pgfqpoint{0.526337in}{2.247559in}}%
\pgfpathlineto{\pgfqpoint{0.551696in}{2.236827in}}%
\pgfpathlineto{\pgfqpoint{0.577055in}{2.223632in}}%
\pgfpathlineto{\pgfqpoint{0.602413in}{2.208022in}}%
\pgfpathlineto{\pgfqpoint{0.630077in}{2.188306in}}%
\pgfpathlineto{\pgfqpoint{0.657741in}{2.165867in}}%
\pgfpathlineto{\pgfqpoint{0.687711in}{2.138599in}}%
\pgfpathlineto{\pgfqpoint{0.717680in}{2.108385in}}%
\pgfpathlineto{\pgfqpoint{0.749955in}{2.072726in}}%
\pgfpathlineto{\pgfqpoint{0.784535in}{2.031157in}}%
\pgfpathlineto{\pgfqpoint{0.821420in}{1.983284in}}%
\pgfpathlineto{\pgfqpoint{0.860611in}{1.928811in}}%
\pgfpathlineto{\pgfqpoint{0.904412in}{1.864080in}}%
\pgfpathlineto{\pgfqpoint{0.955130in}{1.784894in}}%
\pgfpathlineto{\pgfqpoint{1.015068in}{1.686857in}}%
\pgfpathlineto{\pgfqpoint{1.098060in}{1.546368in}}%
\pgfpathlineto{\pgfqpoint{1.238686in}{1.308064in}}%
\pgfpathlineto{\pgfqpoint{1.300930in}{1.207158in}}%
\pgfpathlineto{\pgfqpoint{1.353953in}{1.125210in}}%
\pgfpathlineto{\pgfqpoint{1.402365in}{1.054386in}}%
\pgfpathlineto{\pgfqpoint{1.446166in}{0.994074in}}%
\pgfpathlineto{\pgfqpoint{1.487662in}{0.940557in}}%
\pgfpathlineto{\pgfqpoint{1.526853in}{0.893457in}}%
\pgfpathlineto{\pgfqpoint{1.566043in}{0.849846in}}%
\pgfpathlineto{\pgfqpoint{1.602929in}{0.812071in}}%
\pgfpathlineto{\pgfqpoint{1.639814in}{0.777507in}}%
\pgfpathlineto{\pgfqpoint{1.674394in}{0.748020in}}%
\pgfpathlineto{\pgfqpoint{1.708974in}{0.721325in}}%
\pgfpathlineto{\pgfqpoint{1.743554in}{0.697367in}}%
\pgfpathlineto{\pgfqpoint{1.778134in}{0.676064in}}%
\pgfpathlineto{\pgfqpoint{1.812714in}{0.657311in}}%
\pgfpathlineto{\pgfqpoint{1.847294in}{0.640983in}}%
\pgfpathlineto{\pgfqpoint{1.884180in}{0.626074in}}%
\pgfpathlineto{\pgfqpoint{1.921065in}{0.613564in}}%
\pgfpathlineto{\pgfqpoint{1.960256in}{0.602662in}}%
\pgfpathlineto{\pgfqpoint{2.001752in}{0.593512in}}%
\pgfpathlineto{\pgfqpoint{2.045553in}{0.586168in}}%
\pgfpathlineto{\pgfqpoint{2.093965in}{0.580360in}}%
\pgfpathlineto{\pgfqpoint{2.149293in}{0.576061in}}%
\pgfpathlineto{\pgfqpoint{2.216148in}{0.573246in}}%
\pgfpathlineto{\pgfqpoint{2.306056in}{0.571859in}}%
\pgfpathlineto{\pgfqpoint{2.518147in}{0.571603in}}%
\pgfpathlineto{\pgfqpoint{2.734848in}{0.571603in}}%
\pgfpathlineto{\pgfqpoint{2.734848in}{0.571603in}}%
\pgfusepath{stroke}%
\end{pgfscope}%
\begin{pgfscope}%
\pgfpathrectangle{\pgfqpoint{0.316667in}{0.571603in}}{\pgfqpoint{2.533333in}{1.778397in}}%
\pgfusepath{clip}%
\pgfsetrectcap%
\pgfsetroundjoin%
\pgfsetlinewidth{1.505625pt}%
\definecolor{currentstroke}{rgb}{0.031373,0.188235,0.419608}%
\pgfsetstrokecolor{currentstroke}%
\pgfsetdash{}{0pt}%
\pgfpathmoveto{\pgfqpoint{0.431818in}{2.265314in}}%
\pgfpathlineto{\pgfqpoint{0.454872in}{2.264254in}}%
\pgfpathlineto{\pgfqpoint{0.477925in}{2.261076in}}%
\pgfpathlineto{\pgfqpoint{0.500979in}{2.255792in}}%
\pgfpathlineto{\pgfqpoint{0.526337in}{2.247568in}}%
\pgfpathlineto{\pgfqpoint{0.551696in}{2.236851in}}%
\pgfpathlineto{\pgfqpoint{0.577055in}{2.223683in}}%
\pgfpathlineto{\pgfqpoint{0.602413in}{2.208119in}}%
\pgfpathlineto{\pgfqpoint{0.630077in}{2.188481in}}%
\pgfpathlineto{\pgfqpoint{0.657741in}{2.166159in}}%
\pgfpathlineto{\pgfqpoint{0.687711in}{2.139073in}}%
\pgfpathlineto{\pgfqpoint{0.719985in}{2.106692in}}%
\pgfpathlineto{\pgfqpoint{0.752260in}{2.071185in}}%
\pgfpathlineto{\pgfqpoint{0.786840in}{2.029925in}}%
\pgfpathlineto{\pgfqpoint{0.826031in}{1.979508in}}%
\pgfpathlineto{\pgfqpoint{0.867527in}{1.922378in}}%
\pgfpathlineto{\pgfqpoint{0.913634in}{1.855061in}}%
\pgfpathlineto{\pgfqpoint{0.966656in}{1.773663in}}%
\pgfpathlineto{\pgfqpoint{1.035816in}{1.663072in}}%
\pgfpathlineto{\pgfqpoint{1.264045in}{1.294528in}}%
\pgfpathlineto{\pgfqpoint{1.321678in}{1.207567in}}%
\pgfpathlineto{\pgfqpoint{1.372395in}{1.134872in}}%
\pgfpathlineto{\pgfqpoint{1.420807in}{1.069358in}}%
\pgfpathlineto{\pgfqpoint{1.464609in}{1.013670in}}%
\pgfpathlineto{\pgfqpoint{1.508410in}{0.961606in}}%
\pgfpathlineto{\pgfqpoint{1.549906in}{0.915756in}}%
\pgfpathlineto{\pgfqpoint{1.589097in}{0.875621in}}%
\pgfpathlineto{\pgfqpoint{1.628288in}{0.838585in}}%
\pgfpathlineto{\pgfqpoint{1.667478in}{0.804629in}}%
\pgfpathlineto{\pgfqpoint{1.706669in}{0.773704in}}%
\pgfpathlineto{\pgfqpoint{1.745860in}{0.745730in}}%
\pgfpathlineto{\pgfqpoint{1.785050in}{0.720601in}}%
\pgfpathlineto{\pgfqpoint{1.824241in}{0.698189in}}%
\pgfpathlineto{\pgfqpoint{1.863432in}{0.678348in}}%
\pgfpathlineto{\pgfqpoint{1.902622in}{0.660916in}}%
\pgfpathlineto{\pgfqpoint{1.944118in}{0.644894in}}%
\pgfpathlineto{\pgfqpoint{1.987920in}{0.630468in}}%
\pgfpathlineto{\pgfqpoint{2.031721in}{0.618347in}}%
\pgfpathlineto{\pgfqpoint{2.080133in}{0.607325in}}%
\pgfpathlineto{\pgfqpoint{2.130851in}{0.598102in}}%
\pgfpathlineto{\pgfqpoint{2.186179in}{0.590329in}}%
\pgfpathlineto{\pgfqpoint{2.248423in}{0.583912in}}%
\pgfpathlineto{\pgfqpoint{2.319888in}{0.578892in}}%
\pgfpathlineto{\pgfqpoint{2.405186in}{0.575249in}}%
\pgfpathlineto{\pgfqpoint{2.515842in}{0.572906in}}%
\pgfpathlineto{\pgfqpoint{2.686436in}{0.571785in}}%
\pgfpathlineto{\pgfqpoint{2.734848in}{0.571696in}}%
\pgfpathlineto{\pgfqpoint{2.734848in}{0.571696in}}%
\pgfusepath{stroke}%
\end{pgfscope}%
\begin{pgfscope}%
\pgfpathrectangle{\pgfqpoint{0.316667in}{0.571603in}}{\pgfqpoint{2.533333in}{1.778397in}}%
\pgfusepath{clip}%
\pgfsetrectcap%
\pgfsetroundjoin%
\pgfsetlinewidth{1.505625pt}%
\definecolor{currentstroke}{rgb}{1.000000,0.000000,0.000000}%
\pgfsetstrokecolor{currentstroke}%
\pgfsetdash{}{0pt}%
\pgfpathmoveto{\pgfqpoint{0.431818in}{2.265314in}}%
\pgfpathlineto{\pgfqpoint{0.454872in}{2.264254in}}%
\pgfpathlineto{\pgfqpoint{0.477925in}{2.261077in}}%
\pgfpathlineto{\pgfqpoint{0.500979in}{2.255795in}}%
\pgfpathlineto{\pgfqpoint{0.526337in}{2.247578in}}%
\pgfpathlineto{\pgfqpoint{0.551696in}{2.236875in}}%
\pgfpathlineto{\pgfqpoint{0.577055in}{2.223735in}}%
\pgfpathlineto{\pgfqpoint{0.602413in}{2.208216in}}%
\pgfpathlineto{\pgfqpoint{0.630077in}{2.188655in}}%
\pgfpathlineto{\pgfqpoint{0.657741in}{2.166449in}}%
\pgfpathlineto{\pgfqpoint{0.687711in}{2.139542in}}%
\pgfpathlineto{\pgfqpoint{0.719985in}{2.107432in}}%
\pgfpathlineto{\pgfqpoint{0.754565in}{2.069672in}}%
\pgfpathlineto{\pgfqpoint{0.791451in}{2.025893in}}%
\pgfpathlineto{\pgfqpoint{0.830641in}{1.975836in}}%
\pgfpathlineto{\pgfqpoint{0.874443in}{1.916155in}}%
\pgfpathlineto{\pgfqpoint{0.925160in}{1.843010in}}%
\pgfpathlineto{\pgfqpoint{0.985099in}{1.752414in}}%
\pgfpathlineto{\pgfqpoint{1.072702in}{1.615468in}}%
\pgfpathlineto{\pgfqpoint{1.204106in}{1.410314in}}%
\pgfpathlineto{\pgfqpoint{1.270961in}{1.310323in}}%
\pgfpathlineto{\pgfqpoint{1.326289in}{1.231374in}}%
\pgfpathlineto{\pgfqpoint{1.377006in}{1.162681in}}%
\pgfpathlineto{\pgfqpoint{1.425418in}{1.100792in}}%
\pgfpathlineto{\pgfqpoint{1.471525in}{1.045446in}}%
\pgfpathlineto{\pgfqpoint{1.515326in}{0.996263in}}%
\pgfpathlineto{\pgfqpoint{1.559127in}{0.950467in}}%
\pgfpathlineto{\pgfqpoint{1.600623in}{0.910232in}}%
\pgfpathlineto{\pgfqpoint{1.642120in}{0.873043in}}%
\pgfpathlineto{\pgfqpoint{1.683616in}{0.838852in}}%
\pgfpathlineto{\pgfqpoint{1.725112in}{0.807580in}}%
\pgfpathlineto{\pgfqpoint{1.766608in}{0.779123in}}%
\pgfpathlineto{\pgfqpoint{1.808104in}{0.753358in}}%
\pgfpathlineto{\pgfqpoint{1.849600in}{0.730148in}}%
\pgfpathlineto{\pgfqpoint{1.893401in}{0.708253in}}%
\pgfpathlineto{\pgfqpoint{1.937202in}{0.688850in}}%
\pgfpathlineto{\pgfqpoint{1.983309in}{0.670909in}}%
\pgfpathlineto{\pgfqpoint{2.031721in}{0.654569in}}%
\pgfpathlineto{\pgfqpoint{2.082439in}{0.639920in}}%
\pgfpathlineto{\pgfqpoint{2.135461in}{0.627004in}}%
\pgfpathlineto{\pgfqpoint{2.190789in}{0.615808in}}%
\pgfpathlineto{\pgfqpoint{2.250728in}{0.605937in}}%
\pgfpathlineto{\pgfqpoint{2.317583in}{0.597246in}}%
\pgfpathlineto{\pgfqpoint{2.391354in}{0.589960in}}%
\pgfpathlineto{\pgfqpoint{2.476651in}{0.583878in}}%
\pgfpathlineto{\pgfqpoint{2.575780in}{0.579128in}}%
\pgfpathlineto{\pgfqpoint{2.697963in}{0.575591in}}%
\pgfpathlineto{\pgfqpoint{2.734848in}{0.574873in}}%
\pgfpathlineto{\pgfqpoint{2.734848in}{0.574873in}}%
\pgfusepath{stroke}%
\end{pgfscope}%
\begin{pgfscope}%
\pgfsetrectcap%
\pgfsetmiterjoin%
\pgfsetlinewidth{0.803000pt}%
\definecolor{currentstroke}{rgb}{0.000000,0.000000,0.000000}%
\pgfsetstrokecolor{currentstroke}%
\pgfsetdash{}{0pt}%
\pgfpathmoveto{\pgfqpoint{0.316667in}{0.571603in}}%
\pgfpathlineto{\pgfqpoint{0.316667in}{2.350000in}}%
\pgfusepath{stroke}%
\end{pgfscope}%
\begin{pgfscope}%
\pgfsetrectcap%
\pgfsetmiterjoin%
\pgfsetlinewidth{0.803000pt}%
\definecolor{currentstroke}{rgb}{0.000000,0.000000,0.000000}%
\pgfsetstrokecolor{currentstroke}%
\pgfsetdash{}{0pt}%
\pgfpathmoveto{\pgfqpoint{2.850000in}{0.571603in}}%
\pgfpathlineto{\pgfqpoint{2.850000in}{2.350000in}}%
\pgfusepath{stroke}%
\end{pgfscope}%
\begin{pgfscope}%
\pgfsetrectcap%
\pgfsetmiterjoin%
\pgfsetlinewidth{0.803000pt}%
\definecolor{currentstroke}{rgb}{0.000000,0.000000,0.000000}%
\pgfsetstrokecolor{currentstroke}%
\pgfsetdash{}{0pt}%
\pgfpathmoveto{\pgfqpoint{0.316667in}{0.571603in}}%
\pgfpathlineto{\pgfqpoint{2.850000in}{0.571603in}}%
\pgfusepath{stroke}%
\end{pgfscope}%
\begin{pgfscope}%
\pgfsetrectcap%
\pgfsetmiterjoin%
\pgfsetlinewidth{0.803000pt}%
\definecolor{currentstroke}{rgb}{0.000000,0.000000,0.000000}%
\pgfsetstrokecolor{currentstroke}%
\pgfsetdash{}{0pt}%
\pgfpathmoveto{\pgfqpoint{0.316667in}{2.350000in}}%
\pgfpathlineto{\pgfqpoint{2.850000in}{2.350000in}}%
\pgfusepath{stroke}%
\end{pgfscope}%
\begin{pgfscope}%
\pgfsetbuttcap%
\pgfsetmiterjoin%
\definecolor{currentfill}{rgb}{1.000000,1.000000,1.000000}%
\pgfsetfillcolor{currentfill}%
\pgfsetfillopacity{0.800000}%
\pgfsetlinewidth{1.003750pt}%
\definecolor{currentstroke}{rgb}{0.800000,0.800000,0.800000}%
\pgfsetstrokecolor{currentstroke}%
\pgfsetstrokeopacity{0.800000}%
\pgfsetdash{}{0pt}%
\pgfpathmoveto{\pgfqpoint{1.770025in}{0.977861in}}%
\pgfpathlineto{\pgfqpoint{2.752778in}{0.977861in}}%
\pgfpathquadraticcurveto{\pgfqpoint{2.780556in}{0.977861in}}{\pgfqpoint{2.780556in}{1.005639in}}%
\pgfpathlineto{\pgfqpoint{2.780556in}{2.252778in}}%
\pgfpathquadraticcurveto{\pgfqpoint{2.780556in}{2.280556in}}{\pgfqpoint{2.752778in}{2.280556in}}%
\pgfpathlineto{\pgfqpoint{1.770025in}{2.280556in}}%
\pgfpathquadraticcurveto{\pgfqpoint{1.742247in}{2.280556in}}{\pgfqpoint{1.742247in}{2.252778in}}%
\pgfpathlineto{\pgfqpoint{1.742247in}{1.005639in}}%
\pgfpathquadraticcurveto{\pgfqpoint{1.742247in}{0.977861in}}{\pgfqpoint{1.770025in}{0.977861in}}%
\pgfpathlineto{\pgfqpoint{1.770025in}{0.977861in}}%
\pgfpathclose%
\pgfusepath{stroke,fill}%
\end{pgfscope}%
\begin{pgfscope}%
\pgfsetrectcap%
\pgfsetroundjoin%
\pgfsetlinewidth{1.505625pt}%
\definecolor{currentstroke}{rgb}{0.579608,0.770196,0.873725}%
\pgfsetstrokecolor{currentstroke}%
\pgfsetdash{}{0pt}%
\pgfpathmoveto{\pgfqpoint{1.797803in}{2.168088in}}%
\pgfpathlineto{\pgfqpoint{1.936692in}{2.168088in}}%
\pgfpathlineto{\pgfqpoint{2.075580in}{2.168088in}}%
\pgfusepath{stroke}%
\end{pgfscope}%
\begin{pgfscope}%
\definecolor{textcolor}{rgb}{0.000000,0.000000,0.000000}%
\pgfsetstrokecolor{textcolor}%
\pgfsetfillcolor{textcolor}%
\pgftext[x=2.186692in,y=2.119477in,left,base]{\color{textcolor}\sffamily\fontsize{10.000000}{12.000000}\selectfont \(\displaystyle n=1\)}%
\end{pgfscope}%
\begin{pgfscope}%
\pgfsetrectcap%
\pgfsetroundjoin%
\pgfsetlinewidth{1.505625pt}%
\definecolor{currentstroke}{rgb}{0.290980,0.594510,0.789020}%
\pgfsetstrokecolor{currentstroke}%
\pgfsetdash{}{0pt}%
\pgfpathmoveto{\pgfqpoint{1.797803in}{1.964231in}}%
\pgfpathlineto{\pgfqpoint{1.936692in}{1.964231in}}%
\pgfpathlineto{\pgfqpoint{2.075580in}{1.964231in}}%
\pgfusepath{stroke}%
\end{pgfscope}%
\begin{pgfscope}%
\definecolor{textcolor}{rgb}{0.000000,0.000000,0.000000}%
\pgfsetstrokecolor{textcolor}%
\pgfsetfillcolor{textcolor}%
\pgftext[x=2.186692in,y=1.915620in,left,base]{\color{textcolor}\sffamily\fontsize{10.000000}{12.000000}\selectfont \(\displaystyle n=2\)}%
\end{pgfscope}%
\begin{pgfscope}%
\pgfsetrectcap%
\pgfsetroundjoin%
\pgfsetlinewidth{1.505625pt}%
\definecolor{currentstroke}{rgb}{0.090196,0.392941,0.670588}%
\pgfsetstrokecolor{currentstroke}%
\pgfsetdash{}{0pt}%
\pgfpathmoveto{\pgfqpoint{1.797803in}{1.760374in}}%
\pgfpathlineto{\pgfqpoint{1.936692in}{1.760374in}}%
\pgfpathlineto{\pgfqpoint{2.075580in}{1.760374in}}%
\pgfusepath{stroke}%
\end{pgfscope}%
\begin{pgfscope}%
\definecolor{textcolor}{rgb}{0.000000,0.000000,0.000000}%
\pgfsetstrokecolor{textcolor}%
\pgfsetfillcolor{textcolor}%
\pgftext[x=2.186692in,y=1.711763in,left,base]{\color{textcolor}\sffamily\fontsize{10.000000}{12.000000}\selectfont \(\displaystyle n=3\)}%
\end{pgfscope}%
\begin{pgfscope}%
\pgfsetrectcap%
\pgfsetroundjoin%
\pgfsetlinewidth{1.505625pt}%
\definecolor{currentstroke}{rgb}{0.031373,0.188235,0.419608}%
\pgfsetstrokecolor{currentstroke}%
\pgfsetdash{}{0pt}%
\pgfpathmoveto{\pgfqpoint{1.797803in}{1.556516in}}%
\pgfpathlineto{\pgfqpoint{1.936692in}{1.556516in}}%
\pgfpathlineto{\pgfqpoint{2.075580in}{1.556516in}}%
\pgfusepath{stroke}%
\end{pgfscope}%
\begin{pgfscope}%
\definecolor{textcolor}{rgb}{0.000000,0.000000,0.000000}%
\pgfsetstrokecolor{textcolor}%
\pgfsetfillcolor{textcolor}%
\pgftext[x=2.186692in,y=1.507905in,left,base]{\color{textcolor}\sffamily\fontsize{10.000000}{12.000000}\selectfont \(\displaystyle n=5\)}%
\end{pgfscope}%
\begin{pgfscope}%
\pgfsetrectcap%
\pgfsetroundjoin%
\pgfsetlinewidth{1.505625pt}%
\definecolor{currentstroke}{rgb}{0.031373,0.188235,0.419608}%
\pgfsetstrokecolor{currentstroke}%
\pgfsetdash{}{0pt}%
\pgfpathmoveto{\pgfqpoint{1.797803in}{1.352659in}}%
\pgfpathlineto{\pgfqpoint{1.936692in}{1.352659in}}%
\pgfpathlineto{\pgfqpoint{2.075580in}{1.352659in}}%
\pgfusepath{stroke}%
\end{pgfscope}%
\begin{pgfscope}%
\definecolor{textcolor}{rgb}{0.000000,0.000000,0.000000}%
\pgfsetstrokecolor{textcolor}%
\pgfsetfillcolor{textcolor}%
\pgftext[x=2.186692in,y=1.304048in,left,base]{\color{textcolor}\sffamily\fontsize{10.000000}{12.000000}\selectfont \(\displaystyle n=10\)}%
\end{pgfscope}%
\begin{pgfscope}%
\pgfsetrectcap%
\pgfsetroundjoin%
\pgfsetlinewidth{1.505625pt}%
\definecolor{currentstroke}{rgb}{1.000000,0.000000,0.000000}%
\pgfsetstrokecolor{currentstroke}%
\pgfsetdash{}{0pt}%
\pgfpathmoveto{\pgfqpoint{1.797803in}{1.110918in}}%
\pgfpathlineto{\pgfqpoint{1.936692in}{1.110918in}}%
\pgfpathlineto{\pgfqpoint{2.075580in}{1.110918in}}%
\pgfusepath{stroke}%
\end{pgfscope}%
\begin{pgfscope}%
\definecolor{textcolor}{rgb}{0.000000,0.000000,0.000000}%
\pgfsetstrokecolor{textcolor}%
\pgfsetfillcolor{textcolor}%
\pgftext[x=2.186692in,y=1.062307in,left,base]{\color{textcolor}\sffamily\fontsize{10.000000}{12.000000}\selectfont \(\displaystyle t \mapsto \mathrm{e}^{-t^2}\)}%
\end{pgfscope}%
\end{pgfpicture}%
\makeatother%
\endgroup%

    \caption{Illustration de la convergence simple de la suite $(f_n)_{n \in \N}$ vers $f$}
\end{marginfigure}

\begin{preuve}
\begin{enumerate}
\item La fonction $x \mapsto \e^{-x^2}$ est continue sur $[0, +\infty[$. D'après le théorème des croissances comparées, $\e^{-x^2} = o_{+\infty}\left(\frac{1}{x^2}\right)$. Ainsi, d'après le théorème de comparaison aux intégrales de Riemann, la fonction $x \mapsto \e^{-x^2}$ est intégrable sur $[0, +\infty[$.

\item
\begin{enumerate}
\item Pour tout $x \geq n$, $h_n(x) = \e^{-x} \leq \e^{-n} = h_n(n)$. De plus, la restriction de la fonction $h_n$ au segment $[0, n]$ est continue sur ce segment, donc elle y est bornée et atteint ses bornes. On note $x_n$ l'abscisse d'un point où $h_n$ atteint ce maximum.

\item En utilisant l'inégalité de convexité du logarithme, pour $x \in [0, n]$,
\[
g_n(x)
= \exp\left(n \ln\left(1 - \frac{x}{n}\right)\right)
\leq \e^{-x}.
\]
Ainsi, $h_n$ est à valeurs positives.

\item Pour tout $x \in [0, n]$,
\[
h_n'(x) = -\e^{-x} + \left(1 - \frac{x}{n}\right)^{n-1}.
\]
Ainsi, $h_n'(0) = 0$ et $h_n'(n) = - \e^{-n} < 0$.

Comme $h_n'(n) < 0$, alors $h_n$ est décroissante sur un voisinage de $n$. De plus, $h_n$ est positive et $h_n(0) = 0$. Ainsi, $x_n \in ]0, n[$. Comme $h_n$ est dérivable et atteint son maximum sur l'ouvert $]0, n[$, alors $h_n'(x_n) = 0$, soit
\[
\e^{-x_n} = \left(1 - \frac{x_n}{n}\right)^{n-1}.
\]

Alors,
\begin{align*}
h_n(x_n)
&= \e^{-x_n} - \left(1 - \frac{x_n}{n}\right)^n
= \e^{-x_n} - \left(1 - \frac{x_n}{n}\right) \e^{-x_n}\\
&= \frac{x_n \e^{-x_n}}{n}.
\end{align*}

\item La fonction $u \mapsto u \e^{-u}$ est dérivable et sa dérivée vaut $u \mapsto \e^{-u} (1 - u)$. Elle atteint donc son maximum en $1$ et la valeur de ce maximum est $\e^{-1}$.

D'après la question précédente, pour tout $x \in \Rp$,
\begin{align*}
\module{h_n(x)}
&\leq h_n(x_n)
\leq \frac{1}{n \e}.
\end{align*}

\item D'après la question précédente, pour tout $x$ réel positif,
\[
\module{\e^{-x^2} - \left(1 - \frac{x^2}{n}\right)^n} \indicatrice{[0,n]}(x)
\leq \frac{1}{n \e}.
\]

Ainsi,
\begin{align*}
\module{I_n - I}
&\leq \int_0^{+\infty} \module{h_n(x^2)} \d x\\
&\leq \int_0^{\sqrt{n}} \module{h_n(x^2)} \d x + \int_{\sqrt{n}}^{+\infty} \module{h_n(x^2)} \d x\\
&\leq \int_0^{\sqrt{n}} \frac{1}{n\e} \d x + \int_{\sqrt{n}}^{+\infty} \e^{-x^2} \d x\\
&\leq \frac{1}{\sqrt{n} \e} + \int_{\sqrt{n}}^{+\infty} \e^{-x^2} \d x.
\end{align*}

\item Comme $x \mapsto \e^{-x^2}$ est intégrable, alors
\[
\lim_{n\to+\infty} \int_{\sqrt{n}}^{+\infty} \e^{-x^2} \d x = 0.
\]

Ainsi, d'après le théorème d'encadrement,
\[
\lim_{n\to+\infty} I_n
= \lim_{n\to+\infty} \int_0^{\sqrt{n}} \left(1 - \frac{x^2}{n}\right)^n \d x
= \int_0^{+\infty} \e^{-x^2} \d x.
\]
\end{enumerate}

\item
\begin{enumerate}
\item Soit $t \in \R$. Soit $n \geq t^2$. Alors, $t \in [0, \sqrt{n}]$, soit
\[
f_n(t)
= \left(1 - \frac{t^2}{n}\right)^n
= \exp \left(n \ln\left(1 - \frac{t^2}{\sqrt{n}}\right)\right).
\]

D'après les équivalents classiques, $\ln\left(1 - \frac{t^2}{\sqrt{n}}\right) \sim -\frac{t^2}{n}$.

Ainsi, d'après la continuité de la fonction exponentielle en $t^2$,
\[
\lim_{n\to+\infty} f_n(t) = \e^{-t^2}.
\]

\item On vérifie les hypothèses du théorème.
\begin{itemize}
\item $t \mapsto \left(1 - \frac{x^2}{n}\right)^n \indicatrice{[0,\sqrt{n}[}(t) \in \mathscr{C}^-(\R_+)$.

\item D'après la question précédente, $f_n(t) \to \e^{-t^2}$.

\item $f \in \mathscr{C}^-(\R_+, \R_+)$.

\item En utilisant l'inégalité de convexité du logarithme, pour $t \in [0, \sqrt{n}]$,
\[
\abs{f_n(t)}
= \exp\left(n \ln\left(1 - \frac{t^2}{n}\right)\right)
\leq \e^{-t^2}.
\]
Ainsi, pour tout $t \in \R_+$, $\abs{f_n(t)} \leq f(t)$, qui est bien intégrable.
\end{itemize}

D'après le théorème de convergence dominée,
\[
\lim_{n\to+\infty} \int_{\R_+} f_n = \int_0^{+\infty} \e^{-t^2} \d t.
\]
\end{enumerate}

\item En effectuant le changement de variable $\phi : [0,\pi/2] \to [0,\sqrt{n}],\, t \mapsto \sqrt{n} \sin(t)$ qui est bien $\mathscr{C}^1$ et strictement croissant,
\begin{align*}
\int_0^{\sqrt{n}} \left(1 - \frac{x^2}{n}\right)^n \d x &= \int_0^{\pi/2} (1 - \sin^2(t))^n \sqrt{n} \cos(t) \d t \\
&= \sqrt{n} \int_0^{\pi/2} \cos^{2n+1}(t) \d t \\
&\to \frac{\sqrt{\pi}}{2}.
\end{align*}
Finalement, $I = \frac{\sqrt{\pi}}{2}$.
\end{enumerate}
\end{preuve}

\todoinline{J'ajouterais ici cette remarque et je supprimerais la partie sur les intégrales de Wallis}
\begin{remarque}
La convexité de la fonction exponentielle permet même de montrer que, pour tout entier $n > 0$ et tout réel $u \in \interoo{-n}{n}$, 
\[
\left(1 + \frac{u}{n} \right)^n \leqslant \e^u \leqslant \left( 1 - \frac{u}{n} \right)^{-n},
\]
puis que
\[
\int_0^{\sqrt{n}} \left( 1 - \frac{x^2}{n} \right)^n \d x \leqslant \int_0^{\sqrt{n}} \e^{-x^2} \d x \leqslant \int_0^{\sqrt{n}} \left( 1 + \frac{x^2}{n} \right)^{-n} \d x.
\]
En utilisant un changement de variables puis les intégrales de Wallis, on obtient alors l'encadement :
\[
\sqrt{n} \Wallis_{2n+1} \leqslant \int_0^{\sqrt{n}} \e^{-x^2} \d x \leqslant \sqrt{n} \Wallis_{2n-2}.
\]
\end{remarque}

\begin{remarque}
En raison de la parité de la fonction $x \mapsto \e^{-x^2}$, alors $\int_\R \e^{-x^2} \d x = \sqrt{\pi}$.
\end{remarque}

\todoinline{Ajouter une remarque pour son utilisation en probas ? Avec un dessin d'une planche de Galton ?}

\subsection{Moments de la gaussienne}

D'après \url{https://djalil.chafai.net/blog/2024/04/28/two-details-about-gaussians/}.

\todoinline{Je prends $\sigma = \frac{1}{\sqrt{2}}$ pour plus coller à la section précédente et simplifier les notations. On peut bien sûr revenir à un $\sigma$ qcq si tu préfères. J'ai rédigé la dérivation sous le signe intégral}

\begin{theo}{} Pour tout $n$ entier naturel,
    \[
    \int_\R x^{2n} \e^{-x^2} \d x
    = \frac{\sqrt{\pi}}{2^n} \prod_{k=1}^n (2k - 1)
    = \frac{\sqrt{\pi}}{4^n} \times n! \binom{2n}{n}.
    \]
\end{theo}

\begin{exercice}
\begin{enumerate}
\item Pour tout $\beta > 0$, déterminer la valeur de $\int_\R \e^{-\beta x^2} \d x$.

\item Calculer la dérivée $n$-ième de la fonction $f : \beta \mapsto \sqrt{\frac{\pi}{\beta}}$.

\item En déduire le résultat annoncé.
\end{enumerate}
\end{exercice}

\begin{preuve}
\begin{enumerate}
\item Le changement de variable affine $x \mapsto \frac{x}{\sqrt{\beta}}$ dans l'intégrale permet d'obtenir l'intégrale de Gauss. Ainsi,
\[
\int_\R \e^{-\beta x^2} \d x
= \int_\R \e^{-\beta u^2} \times \frac{1}{\sqrt{\beta}} \d u
= \sqrt{\frac{\pi}{\beta}}.
\]

\item Comme $f(\beta) = \sqrt{\pi} \beta^{-1/2}$, une récurrence permet d'obtenir :
\begin{align*}
f_n'(\beta)
&= \sqrt{\pi} (-1)^n \beta^{-\frac{2n+1}{2}} \prod_{k=1}^n \left(\frac{1}{2} + k - 1\right)\\
&= (-1)^n \sqrt{\frac{\pi}{\beta^{2n+1}}} \prod_{k=1}^n \frac{2k-1}{2}.
\end{align*}

\item Posons $g : (\beta, x) \mapsto \e^{-\beta x^2}$ et $I(\beta) = \int_{-\infty}^{+\infty} g(\beta, x) \d x$.
\begin{enumerate}
\item Pour tout $x \in \R$, la fonction $g(\cdot, x)$ est de classe $\mathscr{C}^\infty$ sur $\R_+^*$.

\item Pour tout $n$ entier naturel et $x \in \R$,
\[
\frac{\partial^n g}{\partial \beta^n}(\beta, x) = (-1)^n x^{2n} \e^{-\beta x^2}\]
donc $\frac{\partial^n g}{\partial \beta^n}(\beta, \cdot)$ est continue sur $\R$.

\item Pour tout $a > 0$, pour tout $\beta \in [a, +\infty[$ et pour tout $x \in \R$,
\[
\module{\frac{\partial^n g}{\partial \beta^n}(\beta, x)} = x^{2n} \e^{-a x^2}
\]
qui est une fonction intégrable.
\end{enumerate}

En utilisant le théorème de dérivation sous le signe intégral, la fonction $I(\cdot)$ est de classe $\mathscr{CC}^\infty$ et pour tout $n$ entier naturel,
\[
I^{(n)}(\beta)
= (-1)^n \int_\R x^{2n} \e^{-\beta x^2} \d x.
\]

Finalement, on obtient l'égalité :
\[
\int_\R x^{2n} \e^{-\beta x^2} \d x
= \sqrt{\frac{\pi}{\beta^{2n+1}}} \prod_{k=1}^n \frac{2k-1}{2}.
\]

Évaluer en $x = 1$ permet d'obtenir le résultat annoncé.
\end{enumerate}
\end{preuve}


% \begin{prop}{}
    % \[
    % \int_\R x^{2n} \frac{1}{\sqrt{2 \pi \sigma^2}} \e^{-\frac{x^2}{2 \sigma^2}} \d x = \sigma^{2n} \prod_{k=1}^n (2k - 1) = \sigma^{2n} (2n-1) !!
    % \]
% \end{prop}
% \begin{preuve}
    % La dérivée $n$-ième par rapport à $\beta$ de l'intégrale suivante
    % \[
    % \int_\R \e^{-\beta x^2} \d x = \sqrt{\frac{\pi}{\beta}},
    % \]
    % que l'on obtient aisément par un changement de variable linéaire dans \ref{eqIntGauss}, s'écrit
    % \[
    % (-1)^n \int_\R x^{2n} \e^{-\beta x^2} \d x = \sqrt{\pi} (-1)^n \beta^{-\frac{2n+1}{2}} \prod_{k=1}^n \left( \frac{1}{2} + k - 1 \right) = (-1)^n \sqrt{\frac{\pi}{\beta^{2n+1}}} \prod_{k=1}^n \left( \frac{1}{2} + k - 1 \right),
    % \]
    % et en prenant $\beta = \frac{1}{2 \sigma^2}$, on obtient
    % \[
    % \int_\R x^{2n} \e^{-\frac{x^2}{2 \sigma^2}} \d x = \sqrt{\pi (2 \sigma^2)^{2n+1}} \prod_{k=1}^n \left( \frac{2k-1}{2} \right)
    % \]
    % soit 
    % \[
    % \int_\R x^{2n} \frac{1}{\sqrt{2 \pi \sigma^2}} \e^{-\frac{x^2}{2 \sigma^2}} \d x = \sigma^{2n} \prod_{k=1}^n (2k - 1) = \sigma^{2n} (2n-1) !!.
    % \]
% \end{preuve}
% Cette dernière notation se nomme la \emph{double factorielle}.


\subsection{Calcul de l'intégrale de \textsc{Gauss} avec celle de \textsc{Wallis}}
\todoinline{À supprimer suite à la remarque précédente ?}
\marginnote[0cm]{Source : \href{https://fr.wikipedia.org/wiki/Intégrale_de_Wallis}{Intégrale de \textsc{Wallis} -- \textsf{wikipedia.org}}}
On peut aisément utiliser les intégrales de \textsc{Wallis} pour calculer l'intégrale de \text{Gauss}. \\
On utilise pour cela l'encadrement suivant, issu de la construction de la fonction exponentielle par la méthode d'\textsc{Euler}: pour tout entier $n > 0$ et tout réel $u \in ]-n, n[$, 
$$\left(1 + \frac{u}{n} \right)^n \leqslant \e^u \leqslant \left( 1 - \frac{u}{n} \right)^{-n}.$$
Posant alors $u = -x^2$, on obtient:
$$\int_0^{\sqrt{n}} \left( 1 - \frac{x^2}{n} \right)^n \d x \leqslant \int_0^{\sqrt{n}} \e^{-x^2} \d x \leqslant \int_0^{\sqrt{n}} \left( 1 + \frac{x^2}{n} \right)^{-n} \d x.$$
Or les intégrales d'encadrement sont liées aux intégrales de \textsc{Wallis}. Pour celle de gauche, il suffit de poser $x = \sqrt{n} \sin t$ ($t$ variant de $0$ à $\pi/2$). Quant à celle de droite, on peut poser $x = \sqrt{n} \tan t$ ($t$ variant de $0$ à $\pi/4$) puis majorer par l'intégrale de $0$ à $\pi/2$. On obtient ainsi:
$$\sqrt{n} \Wallis_{2n+1} \leqslant \int_0^{\sqrt{n}} \e^{-x^2} \d x \leqslant \sqrt{n} \Wallis_{2n-2}.$$
Par le théorème des gendarmes, on déduit alors de l'équivalent de $\Wallis_n$ que
$$\int_0^{+ \infty} \e^{-x^2} \d x = \frac{\sqrt{\pi}}{2}.$$



%---------------

\begin{exercice}
\begin{enumerate}
\item Montrer que
\[
\int_0^{\sqrt{n}} \left(1 - \frac{t^2}{n}\right)^n \d t \leq \int_0^{\sqrt{n}} \e^{-t^2} \d t \leq \int_0^{+\infty} \frac{\d t}{\left(1 + \frac{t^2}{n}\right)^n}.
\]

\item En déduire que $\int_0^{\sqrt{n}} \e^{-t^2} \d t \sim \sqrt{n} \int_0^{\pi/2} \cos^{2n+1}(\theta) \d \theta$.
{En utilisant les intégrales de {Wallis}, on montre que $\int_0^{+\infty} \e^{-t^2} \d t = \frac{\sqrt{\pi}}{2}$.}
\end{enumerate}
\end{exercice}

\begin{preuve}
\begin{enumerate}
\item Rappelons que $\ln(1 + x) \leq x$. Ainsi, $\left(1 - \frac{t^2}{n}\right)^n \leq \e^{-t^2}$. De même, $-\ln(1+t^2/n) \geq -t^2/n$ et $\e^{-\frac{t^2}{n}} \geq \left(1 + \frac{t^2}{n}\right)^{-n}$.

On obtient ainsi le résultat en intégrant entre $0$ et $\sqrt{n}$. De plus, $\left(1 + \frac{t^2}{n}\right)^{-n} = O(1/t^2)$ donc l'intégrale est convergente.

\item On pose $\varphi : t \mapsto \sqrt{n} \sin(t)$ dans la première intégrale et $\psi : t \mapsto \sqrt{n} \tan(t)$ dans la seconde. On obtient ainsi l'encadrement
\begin{align*}
\sqrt{n} \int_0^{\pi/2} \cos^{2n+1}(t) \d t \leq \int_0^{\sqrt{n}} \e^{-t^2} \d t &\leq \sqrt{n} \int_0^{\pi/2} \cos^{2n-2}(t) \d t\\
& \leq \sqrt{n} \int_0^{\pi/2} \cos^{2n-3}(t) \d t.
\end{align*}
En notant $I_n = \int_0^{\pi/2} \cos^{2n+1}(t) \d t$, comme la suite $(I_n)$ est décroissante. De plus, en utilisant une intégration par parties, on obtient une relation de récurrence puis $I_{n+1} \sim I_n$.

{On obtiendra ce résultat plus simplement en utilisant le théorème de convergence dominée.}
\end{enumerate}
\end{preuve}



% \todoinline{Pour avoir une application, on peut regarder la remarque dans le chapitre sur les polynômes d'Hermite ou alors calculer la transformée de Fourier d'une gaussienne.\\
% Dans le document bestiaire.pdf, on peut trouver :\\
% * une preuve par étude d'intégrale à paramètre en page 178\\
% * une preuve avec Wallis en p. 223 mais je l'ai déjà quelque part rédigée}

% \todoarmand{
% * Je ne vois pas de quelle remarque vous parlez \\
% * On peut calculer la transformée de Fourier d'une gaussienne, ça pourrait aussi être l'occasion de dire un mot sur ce qu'est la TF. \\
% * J'aime bien la preuve avec Wallis, il faudrait alors mettre cette section après celle sur Wallis. Il faut voir comment l'intégrer (c'est le cas de le dire) avec les autres exercices sur Wallis et Stirling pour ne pas trop se répéter.
% }

%-----------
\subsection{Transformée de Fourier}

\begin{theo}{}
Pour tout $x$ réel,
\[
\int_{-\infty}^{+\infty} \e^{i t x} \e^{-t^2} \d t
= \sqrt{\pi} \e^{-\frac{x^2}{4}}.
\]
\end{theo}

\begin{exercice}
On pose $f(x) = \int_{-\infty}^{+\infty} \e^{-t^2 + i t x} \d t$.
\begin{enumerate}
\item Montrer que la fonction $f$  est définie sur $\R$.

\item Montrer que $f$ est dérivable sur $\R$ et que $f'(x) = \int_{-\infty}^{+\infty} i t \e^{-t^2 + i t x} \d t$.

\item Montrer que $f$ vérifie l'équation différentielle $2 y' + x y = 0$.
{On pourra utiliser une intégration par parties.}

\item En déduire la valeur de $f$.
\end{enumerate}
\end{exercice}

\begin{preuve} On pose $g : (x, t) \mapsto \e^{-t^2 + i t x}$.
\begin{enumerate}
\item Pour tout $(x, t) \in \R^2$,  $\abs{g(x, t)} \leq \e^{-t^2}$. Ainsi, $g(x, \cdot)$ est une fonction intégrable pour tout $x$ réel.

\item Les hypothèses de régularité sont aisément vérifiables. De plus,
\[
\module{\frac{\partial g}{\partial x}(x, t)} = t \e^{-t^2}
\]
qyu est intégrable sur $\R$.

Ainsi, en appliquant le théorème de dérivation sous le signe intégral, la fonction $f$ est dérivable et
\[
f'(x) = \int_\R i t \e^{-t^2 + i t x} \d t.
\]

\item En utilisant une intégration par parties généralisée dont le crochet converge,
\begin{align*}
f'(x) &= \int_\R t \e^{-t^2} i \e^{i t x} \d t \\
&= \left[-\frac{1}{2} \e^{-t^2} i \e^{i t x}\right]_{-\infty}^{+\infty} - \int_\R \frac{x}{2} \e^{-t^2} \e^{i t x} \d t \\
f'(x) &= -\frac{x}{2} f(x).
\end{align*}

\item L'ensemble des solutions de l'équation du différentielle du premier ordre $2 y' + x y = 0$ est
\[
\left\{x \mapsto \lambda \e^{-\frac{x^2}{4}},\, \lambda \in \R\right\}.
\]

D'après le calcul de l'intégrale de Gauss, $f(0) = \sqrt{\pi}$. Ainsi,
\[
f(x) = \sqrt{\pi} \e^{-\frac{x^2}{4}}.
\]
\end{enumerate}
\end{preuve}


\subsection{À trier}

\todoinline{On met cet exercice avec les polynômes orthogonaux ?}

%---------------

\begin{exercice}
Polynômes d'{Hermite}
{RMS 2017 154 - Autres écoles}
{TPE}
{16}
Soit $f : x \mapsto \e^{-x^2}$. On rappelle que $\int_{-\infty}^{+\infty} f(x) \d x = \sqrt{\pi}$.
\begin{enumerate}
\item Montrer qu'il existe un polynôme $P_n$ tel que $f^{(n)}(x) = f(x) P_n(x)$. Préciser le degré, la parité et le coefficient dominant de $P_n$.

\item Montrer l'existence puis calculer $\int_{-\infty}^{+\infty} f(x) P_n(x) P_m(x) \d x$.
\end{enumerate}
\end{exercice}

\begin{preuve}
\begin{enumerate}
\item On raisonne par récurrence en remarquant que $P_0 = 1$ et $P_{n+1} = P_n' - 2 X P_n$. Ainsi, le degré de $P_n$ est $n$ et son coefficient dominant $(-2)^n$ et $P_n$ a même parité que $n$, i.e. $P_n(-X) = (-1)^n P_n(X)$.

\item Soient $0 < m \leq n$. Les intégrales sont bien définies car ce sont des $o(1/x^2)$ en $\pm\infty$. Comme les crochets tendent vers $0$, en utilisant les fonctions $t \mapsto f^{(n-1)}(t)$ et $t \mapsto P_m(t)$ qui sont de classe $\mathscr{C}^1$, on remarque que
\[
\int_\R f P_n f P_m = \int_\R f^{(n)} P_m = -\int_\R f^{(n-1)} P_m'.
\]
Ainsi, en itérant ce procédé, comme $m \leq n$,
\[
\int_\R f P_n P_m = (-1)^m \int_\R f^{(n-m)} P_m^{(m)}
= (-1)^m (-2)^m \int_\R f^{(n-m)}.
\]
Ainsi,
\begin{itemize}
\item Si $n - m = 0$, i.e. $m = n$, alors $\int_\R f P_n^2 = m! 2^m \sqrt{\pi}$.
\item Si $n - m \neq 0$, alors $\int_\R f P_n P_m = \left[f^{(n-m-1)}\right]_{-\infty}^{+\infty} = 0$.
\end{itemize}
\end{enumerate}
\end{preuve}


%%%%%%%%%%%%%%%%%%%%%

\todoinline{Je supprimerais l'exercice suivant}

%---------------

\begin{exercice}
{X-ENS}
{16}%
Soient $f$ et $g$ les fonctions définies pour tout $x \in \R_+$ par $f(x) = \int_0^1 \frac{e^{-(t^2+1) x^2}}{1 + t^2} \d t$ et $g(x) = \int_0^x e^{-t^2} \d t$.
\begin{enumerate}
\item Calculer $f(0)$ puis $\lim_{x\to+\infty} f(x)$.

\item Montrer que $f$ est de classe $\mathscr{C}^1$ sur $\R_+$ et que, pour tout $x \in \R_+$, $-2 g'(x) g(x) = f'(x)$.

\item En déduire $I = \int_0^{+\infty} e^{-t^2} \d t$.

Soit $h$ une fonction continue par morceaux, décroissante sur $\R_+$ telle que $\int_0^{+\infty} h(t) \d t$ soit convergente et non nulle.

\item Montrer que $h$ est à valeurs positives.

Pour tout réel positif $t$ non nul, on pose $S(t) = \sum_{n=1}^{+\infty} h(n t)$.
\item Montrer que $S$ existe.

\item Déterminer un équivalent de $S(t)$ lorsque $t$ tend vers $0^+$.

\item Déterminer un équivalent de $\sum_{n=1}^{+\infty} x^{n^2}$ lors que $x$ tend vers $1^-$.
\end{enumerate}
\end{exercice}

\begin{preuve}
\begin{enumerate}
\item D'après la définition, $f(0) = \int_0^1 \frac{1}{1 + t^2} \d t = \arctan(1) = \frac{\pi}{4}$.

On remarque que l'intégrande est majorée par $t \mapsto \frac{1}{1+ t^2}$ qui est intégrable donc en appliquant le théorème de convergence dominée,
\[
\lim_{x\to+\infty} f(x) = 0.
\]

\item Sur $[a, b]$, on majore la dérivée par $b^2$ qui est intégrable sur $[0, 1]$. Ainsi, d'après le théorème de dérivation sous le signe intégral,
\begin{align*}
f'(x) &= - \int_0^1 2 x e^{-(t^2+1) x^2} \d t \\
&= - 2 x e^{-x^2} \int_0^1 e^{-(t x)^2} \d t \\
&= - 2 x e^{-x^2} \int_0^x e^{-t^2} \frac{\d t}{x} \\
&= - 2 g'(x) g(x).
\end{align*}

\item D'après la question précédente,
\[
f(x) - f(0) = - (g(x)^2 - g(0)^2).
\]
Ainsi,
\[
\frac{\pi}{4} - f(x) = g(x)^2.
\]
La fonction $g$ étant à valeurs positives, $I = \frac{\sqrt{\pi}}{2}$.

\item $h$ est décroissante sur $\R_+$. Elle admet donc une limite en $+\infty$. Si cette limite (dans $\bar{\R}$) est égale à $\ell < 0$, alors $h(t) \leq \frac{\ell}{2}$ pour $t$ assez grand et $\int_0^{+\infty} h(t) \d t$ diverge. On raisonne de même pour $\ell > 0$. Ainsi, $h$ tend vers $0$ en $+\infty$ et $h$ est à valeurs positives.

\medskip

{2ème méthode (si $h$ est continue).} En utilisant le théorème des accroissements finis, $H(n+1) - H(n) = h(c_n)$ et le membre de gauche tend vers $0$ donc $\ell = 0$.

\item D'après la décroissance de $h$,
\begin{align*}
\sum_{n=1}^N h(n t) &= \sum_{n=1}^N (n t - (n-1)t) \frac{1}{t} h(n t) \\
&\leq \frac{1}{t} \sum_{n=1}^N \int_{(n-1)t}^{nt} h(u) \d u \\
&\leq \frac{1}{t} \int_0^{Nt} h(u) \d u.
\end{align*}
Ainsi, comme $\int_0^{+\infty} h(t) \d t$ converge, alors d'après les théorèmes sur les séries à termes positifs, $S$ converge.

\item D'après la question précédente, en utilisant une minoration,
\[
J - c t \leq t S(t) \leq J.
\]
Ainsi, comme $J \neq 0$, alors
\[
S(t) \sim_{t\to0} \frac{J}{t}.
\]

\item En posant $h(t) = e^{-t^2}$, la fonction $h$ est bien continue, décroissante et d'intégrale sur $\R_+$ convergente. Ainsi, d'après la question précédente, pour $x \in ]0, 1[$,
\[
S(\sqrt{-\ln(x)}) = \sum_{n=1}^{+\infty} x^{n^2}.
\]
Ainsi, d'après la question précédente,
\[
\sum_{n=1}^{+\infty} x^{n^2} \sim_1 \frac{\sqrt{\pi}}{2 \sqrt{\abs{\ln(x)}}}.
\]
\end{enumerate}
\end{preuve}


\section{Intégrale de \nom{Wallis}} \label{integrale_wallis}

\marginnote[0cm]{
Les intégrales de \nom{Wallis} ont été introduites par John \nom{Wallis} (1616--1703), notamment pour développer le nombre $\pi$ en un produit infini de rationnels; le \textsl{produit de \nom{Wallis}}, énoncé en 1656 dans son ouvrage \emph{Arithmetica infinitorum}. \href{https://fr.wikipedia.org/wiki/Intégrale_de_Wallis}{Source}.
}

\begin{defi}[Intégrale de \nom{Wallis}]
Pour tout $n$ entier naturel, on nomme \definir{intégrale de \nom{Wallis}} l'intégrale définie par
\[
\Wallis_n \defeq \int_{0}^{\frac{\pi}{2}} \sin(x)^n \d x.
\]
\end{defi}
\begin{prop}
De façon équivalente, on peut définir l'intégrale de \nom{Wallis} par 
\[
\Wallis_n = \int_{0}^{\frac{\pi}{2}} \cos(x)^n \d x.
\]
\end{prop}
\begin{elemdemo}
Effectuer le changement de variable $t = \frac{\pi}{2} - x$. 
\end{elemdemo}

%-----------
\subsection{Calcul de l'intégrale et démonstration de ses propriétés}

\begin{theo}{} \labprop{prop_wallis} Pour tout $n$ entier naturel, les intégrales de \nom{Wallis} vérifient les propriétés suivantes
\begin{enumerate}[label=(\roman*)]
    \item $(n + 2) \Wallis_{n+2} = (n + 1) \Wallis_n$,  
    \item $\displaystyle \Wallis_{2n} = \frac{\pi}{2^{2n+1}} \binom{2n}{n}$ et $\Wallis_{2n+1} = \frac{2^{2n} (n!)^2}{(2n+1)!}$,
    \item $\displaystyle \Wallis_n \Wallis_{n+1} = \frac{\pi}{2(n+1)}$,
    \item $\Wallis_{n+1} \sim \Wallis_n$ et $\displaystyle \Wallis_n \sim \sqrt{\frac{\pi}{2n}}$.
\end{enumerate}
\end{theo}
Démontrons l'ensemble de ces résultats grâce à l'exercice suivant.
\begin{exercice}\label{exo:propWallis}
\begin{questions}
\item Montrer que la suite $(\Wallis_n)_{n\in\N}$ est décroissante et minorée.

\item Pour tout $n$ entier naturel non nul, montrer que $(n + 2) \Wallis_{n+2} = (n + 1) \Wallis_n$.

\item Pour tout $p \in \N$, en déduire que $\Wallis_{2p} = \frac{1}{2^{2p}} \binom{2p}{p} \frac{\pi}{2}$.

\item Pour tout $p \in \N$, montrer de manière analogue que $\Wallis_{2p+1} = \frac{2^{2p} (p!)^2}{(2p+1)!}$.

\item En déduire que $\Wallis_{n+1} \sim \Wallis_n$.

\item Montrer que $\Wallis_n \Wallis_{n+1} = \frac{\pi}{2 (n + 1)}$.

\item En déduire que $\Wallis_n \sim \sqrt{\frac{\pi}{2n}}$.
\end{questions}
\end{exercice}

\begin{solution}
\begin{reponses}
\item Par linéarité de l'intégrale, pour tout $n \in \N$, $\Wallis_{n+1} - \Wallis_n = \int_0^{\frac{\pi}{2}} \sin(t)^n \big(\sin(t) - 1\big) \d t$. Or, pour tout $t \in \interff{0}{\frac{\pi}{2}}$, $0 \leqslant \sin(t) \leqslant 1$. Ainsi, l'intégrande est à valeurs négatives et en utilisant la croissance de l'intégrale, on obtient bien $0 \leqslant \Wallis_{n+1} \leqslant \Wallis_n$. La suite $(\Wallis_n)_{n\in\N}$ converge donc par le \theoremeutilise{théorème de la limite monotone}{theo:limitemonotone}.
\item Calculons $\Wallis_{n+2}$ en effectuant une intégration par parties. On pose $u(t) = - \cos(t)$ et $v(t) = \sin(t)^{n+1}$, toutes deux de classe $\mathscr{C}^1$ sur $\interff{0}{\frac{\pi}{2}}$ et on calcule
    \begin{align*}
        \Wallis_{n+2} &= \underbrace{\left[ -\cos(t) \sin(t)^{n+1} \right]_0^{\pi/2}}_{=0} + (n+1) \int_0^{\pi/2} \cos(t)^2 \sin(t)^n \d t \\
        &= (n+1) \int_0^{\frac{\pi}{2}} \big(1 - \sin(t)^2 \big) \sin(t)^n \d t \\
        &= (n+1) \Wallis_n - (n+1) \Wallis_{n+2} \\
        \text{soit } (n+2) \Wallis_{n+2} &= (n+1) \Wallis_n.
\end{align*}

\item On remarque que
\[
\Wallis_0 = \int_0^{\frac{\pi}{2}} 1 \d x = \frac{\pi}{2}.
\]
Soit $p \in \N$. Comme la fonction sinus est de signe constant et non identiquement nulle sur $\interff{0}{\frac{\pi}{2}}$, alors $\Wallis_k$ est non nulle pour tout $k$ entier naturel. Alors, d'après la relation de la question précédente,
\[
\frac{\Wallis_{2k}}{\Wallis_{2k-2}} = \frac{2k-1}{2k}
\]
et en utilisant un produit télescopique pour $k \in \interent{1}{p}$,
\[
\prod_{k=1}^p \frac{\Wallis_{2k}}{\Wallis_{2k-2}} = \prod_{k=1}^p \frac{2k-1}{2k} \quad \text{soit} \quad \Wallis_{2p} = \left[ \prod_{k=1}^p \frac{2k-1}{2k} \right] \Wallis_0.
\] 
Nous pouvons récrire le produit de la manière suivante
\begin{align*}
\Wallis_{2p} &= \frac{\prod\limits_{k=1}^p (2k-1)}{\prod\limits_{k=1}^{p} 2 k} \Wallis_0 = \frac{\left[\prod\limits_{k=1}^p (2k-1)\right] \times \left[\prod\limits_{k=1}^{p} (2k)\right]}{\left[\prod\limits_{k=1}^p (2k)\right]^2} \Wallis_0 = \frac{(2p)!}{2^{2p}(p!)^2} \frac{\pi}{2},
\end{align*}
ce qui fournit le résultat demandé. 

\item De manière analogue, on remarque que
\[
\Wallis_1
= \int_0^{\frac{\pi}{2}} \sin(t) \d t
= \left[-\cos(t)\right]_0^{\frac{\pi}{2}}
= 1.
\]

Ainsi, pour $p \in \N$,
\[
\frac{\Wallis_{2p+1}}{\Wallis_{2p-1}} = \frac{2p}{2p+1}
\]
et comme précédemment, en utilisant un produit télescopique, 
\begin{align*}
\prod_{k=1}^p \frac{\Wallis_{2k+1}}{\Wallis_{2k-1}} &= \prod_{k=1}^p \frac{2k}{2k+1}\\
\frac{\Wallis_{2p+1}}{\Wallis_1} &= \frac{\left[\prod\limits_{k=1}^p (2k)\right]^2}{\left[\prod\limits_{k=1}^p (2k+1)\right] \times \left[\prod\limits_{k=1}^p (2k)\right]}\\
\intertext{soit}
\Wallis_{2p+1} &= \frac{2^{2p} (p!)^2}{(2p+1)!}.
\end{align*}

\item D'après les questions précédentes, pour tout $n$ entier naturel non nul,
\begin{align*}
\Wallis_{n-1} &\leqslant \Wallis_n \leqslant \Wallis_{n+1}\\
\frac{\Wallis_{n-1}}{\Wallis_{n+1}} &\leqslant \frac{\Wallis_n}{\Wallis_{n+1}} \leqslant 1.
\end{align*}
Or, d'après la question précédente, $\frac{\Wallis_{n-1}}{\Wallis_{n+1}} = \frac{n+1}{n}$. Ainsi, d'après le \theoremeutilise{théorème d'encadrement}{theo:encadrement}, $\Wallis_n \sim \Wallis_{n+1}$.

\item D'après la question précédente,
\[
\Wallis_{2p} \Wallis_{2p+1} = \frac{\pi}{2 (2 p + 1)}
\quad \text{et} \quad
\Wallis_{2p+1} \Wallis_{2p+2} = \frac{\pi}{2 (2p+2)}.
\]
Ainsi, pour tout $n$ entier naturel,
\[
\Wallis_n \Wallis_{n+1} = \frac{\pi}{2 (n + 1)}.
\]

\item En utilisant les propriétés des équivalents, $\Wallis_n{}^2 \sim \frac{\pi}{2 n}$ soit
\[
\Wallis_n \sim \sqrt{\frac{\pi}{2 n}}.
\]
\end{reponses}
\end{solution}

%-----------
\subsection{Formule de \nom{Stirling}} \label{preuve_stirling}

\begin{theo}[Formule de \nom{Stirling}]
\[
n! \sim \sqrt{2 \pi n} \left(\frac{n}{\e}\right)^n
\]
\end{theo}

\begin{exercice}
Pour tout $n$ entier naturel non nul, on pose $u_n = \frac{n! \e^n}{n^{n + \frac{1}{2}}}$.
\begin{questions}
\item Montrer que $\left(n + \frac{1}{2}\right) \ln\mathopen{}\left(1 + \frac{1}{n}\right) - 1\sim \frac{1}{12 n^2}$.

\item En déduire que la série $\sum \ln \frac{u_{n+1}}{u_n}$ converge.

\item Montrer que la suite $(u_n)_{n\in\Ne}$ converge vers un réel $\ell$ strictement positif.

\item À l'aide des intégrales de \nom{Wallis}, déterminer la valeur~de~$\ell$.

\item En déduire la formule de \nom{Stirling}.
\end{questions}
\end{exercice}

\begin{solution}
\begin{reponses}
\item En utilisant les développements limités classiques,
\begin{align*}
\left(n + \frac{1}{2}\right) \ln\mathopen{}\left(1 + \frac{1}{n}\right) - 1
&= \left(n + \frac{1}{2}\right) \left(\frac{1}{n} - \frac{1}{2 n^2} + \frac{1}{3 n^3} + o\mathopen{}\left(\frac{1}{n^3}\right)\right) - 1\\
&= 1 - \frac{1}{2 n} + \frac{1}{3 n^2} + \frac{1}{2 n} - \frac{1}{4 n^2} + o\mathopen{}\left(\frac{1}{n^2}\right) - 1\\
&= \frac{1}{12 n^2} + o\mathopen{}\left(\frac{1}{n^2}\right).
\end{align*}

On obtient ainsi l'équivalent annoncé.

\item On remarque que
\[
\frac{u_{n+1}}{u_n}
= \frac{(n + 1)!\, \e^{n+1}}{(n+1)^{n+\frac{3}{2}}} \times \frac{n^{n+\frac{1}{2}}}{n!\, \e^n}
= \e \left(1 + \frac{1}{n}\right)^{-\mathopen{}\left(n + \frac{1}{2}\right)}.
\]
Ainsi, d'après la question précédente et le \theoremeutilise{théorème de comparaison aux séries de \nom{Riemann}}{theo:comparaisonseriesriemann}, la série de terme général $\ln\frac{u_{n+1}}{u_n}$ converge.

\item Pour tout $N \geqslant 2$, en reconnaissant une somme télescopique,
\[
\sum_{n=1}^{N-1} \ln\frac{u_{n+1}}{u_n} = \ln(u_N) - \ln(u_1).
\]

Ainsi, d'après la question précédente, la suite $(\ln(u_n))_{n\in\Ne}$ converge vers un réel $\tilde{\ell}$.

D'après la continuité de la fonction exponentielle, la suite $(u_n)_{n\in\Ne}$ converge vers $\ell = \e^{\tilde{\ell}}$ qui est bien un réel strictement positif.

Ainsi, $n! \sim \ell \left(\frac{n}{\e}\right)^n \sqrt{n}$.

\item D'après les résultats sur les intégrales de \nom{Wallis},
\[
\Wallis_{2p}
\sim \sqrt{\frac{\pi}{2\times 2p}}
\sim \sqrt{\frac{\pi}{4 p}}.
\]

Par ailleurs,
\begin{align*}
\Wallis_{2p}
= \frac{(2p)!}{2^{2p} (p!)^2} \frac{\pi}{2}
\sim \frac{\ell\, \e^{2p} (2p)^{2p} \sqrt{2p}}{\ell^2\, \e^{2p} (2p)^{2p} p} \times \frac{\pi}{2}
\sim \frac{\pi}{\ell \sqrt{2 p}}.
\end{align*}

Ainsi,
\begin{align*}
\sqrt{\frac{\pi}{4 p}} \sim \frac{\pi}{\ell \sqrt{2p}}
\quad \text{soit} \quad 
\ell = \sqrt{2 \pi}.
\end{align*}

\item D'après les questions précédentes, on obtient l'équivalent annoncé
\[
n! \sim \sqrt{2 \pi n} \left(\frac{n}{\e}\right)^n.
\]
\end{reponses}
\end{solution}

%-----------
\subsection{Séries génératrices}

% \todoinline{C'est rigolo ! Est-ce qu'on pourrait retrouver l'expression de $\Wallis_n$ à l'aide d'un produit de DSE ? Il y a aussi une application ici : https://math-os.com/coefficient-binomial-central/}

\begin{prop}{}
Pour tout $x \in \interoo{-1}{1}$, la série génératrice des termes impairs est
$$\sum_{p=0}^\infty \Wallis_{2p+1} x^{2p+1} = \frac{\arcsin x}{\sqrt{1-x^2}}.$$

Pour tout $x \in \interoo{-1}{1}$, la série génératrice des termes pairs est 
$$\sum_{p=0}^\infty \Wallis_{2p} x^{2p} = \frac{\pi}{2} \frac{1}{\sqrt{1-x^2}}.$$

Pour tout $x \in \interoo{-1}{1}$,
\[
\sum_{n=0}^{+\infty} \Wallis_n x^n = \frac{2}{\sqrt{1 - x^2}} \arctan\sqrt{\frac{x + 1}{x - 1}}.
\]
\end{prop}

% \todoinline{Citer \url{https://math-os.com/coefficient-binomial-central/}. Je réécris certaines parties pour préciser et détailler la démarche.}

\source{Exercice inspiré de l'article \href{https://math-os.com/coefficient-binomial-central/}{Coefficient Binomial Central : un aperçu} (Annexe 2) du site \href{https://math-os.com/bienvenue/}{math-os.com} de René \textsc{Adad}}
\begin{exercice}
On considère la formule dans le cas des termes impairs. Pour tout $p$ entier naturel, on pose $\fonctionligne[f_p]{t}{\cos(t)^{2p+1} x^{2p+1}}$.
\begin{questions}
\item Montrer que $\sum f_p$ converge normalement sur $\interff{0}{\frac{\pi}{2}}$.

\item En déduire que
\[
\sum_{p=0}^{+\infty} \Wallis_{2p+1} x^{2p+1} = \int_0^{\frac{\pi}{2}} \frac{x \cos(t)}{1 - x^2 \cos(t)^2} \d t.
\]

\item Utiliser le changement de variable $\fonctionligne[\varphi]{u}{\arcsin(u/x)}$ pour conclure.

\item Reprendre le raisonnement précédent avec le changement de variable $\fonctionligne[\varphi]{u}{\tan\frac{u}{2}}$ pour conclure dans le cas pair.

\item Reprendre les raisonnements précédents pour démontrer la dernière formule.
\end{questions}
\end{exercice}

\begin{solution}
\begin{reponses}
\item Comme la fonction cosinus est bornée par $1$,
\[
\norm{f_p}_{\infty} = \module{x}^{2p+1}.
\]

Or, pour tout $x \in \interoo{-1}{1}$, $\sum \module{x}^{2p+1}$ converge. Ainsi, $\sum \norm{f_p}_{\infty}$ converge.

\item Soit $x \in \interoo{-1}{1}$. D'après le \theoremeutilise{théorème d'interversion série / intégrale}{theo:interversionserieintegrale},
\begin{align*}
\sum_{p=0}^\infty \Wallis_{2p+1} x^{2p+1}
&= \sum_{p=0}^\infty x^{2p+1} \int_0^{\pi/2} \cos(t)^{2p+1} \d t\\
&= \int_0^{\pi/2} x\cos(t) \sum_{p=0}^\infty \Big(\big(x\cos(t) \big)^2 \Big)^p \d t \\
&= \int_0^{\pi/2} \frac{x \cos(t)}{1 - x^2 \cos(t)^2} \d t.
\end{align*}

\item Le cas $x = 0$ est trivial. Pour $x \neq 0$, la fonction $\fonctionens[\varphi]{\interff{0}{x}}{\interff{0}{\pi/2}},\, u \mapsto \arcsin\frac{u}{x}$ est de classe $\mathscr{C}^1$. De plus,
\[
\cos(\varphi(u)) = \sqrt{1 - \frac{u^2}{x^2}}
\quad \text{et} \quad 
\varphi'(u) = \frac{1}{x \sqrt{1 - \frac{u^2}{x^2}}}.
\]

Ainsi,
% On voit rapidement que le changement de variable $u = x \cos(t)$ ne permet pas d'aboutir mais ne remplaçant $\cos(t)^2$ par $1 - \sin(t)^2$ au dénominateur puis en posant $u = x \sin(t)$, il vient:
\begin{align*}
\sum_{p=0}^\infty \Wallis_{2p+1} x^{2p+1}
&= \int_0^x \frac{x \sqrt{1 - \frac{u^2}{x^2}}}{1 - x^2 \left(1 - \frac{u^2}{x^2}\right)} \times \frac{1}{x \sqrt{1 - \frac{u^2}{x^2}}}\d u \\
&= \int_0^x \frac{1}{1 - x^2 + u^2} \d u\\
&= \frac{1}{\sqrt{1 - x^2}} \int_0^x \frac{\frac{1}{\sqrt{1 - x^2}}}{1 + \left(\frac{u}{\sqrt{1-x^2}}\right)^2} \d u \\
&= \frac{1}{\sqrt{1 - x^2}} \left[ \arctan \frac{u}{\sqrt{1 - x^2}} \right]^x_0 \\
&= \frac{1}{\sqrt{1 - x^2}} \arctan \frac{x}{\sqrt{1 - x^2}}.
\end{align*}
On obtient le résultat annoncé en utilisant le fait que pour tout $x \in \interoo{-1}{1}$, $\arctan\mathopen{}\Big(\frac{x}{\sqrt{1-x^2}} \Big) = \arcsin(x)$.
\marginnote[-5cm]{
\begin{tikzpicture}[
  scale=1.5,
  my angle/.style={
    every pic quotes/.append style={text=cyan},
    draw=cyan,
    angle radius=1.5cm,
  }]
  \coordinate (C) at (-1.5,-1);
  \coordinate (A) at (1.5,-1);
  \coordinate (B) at (1.5,1);
  \pic [my angle, "$\alpha$"] {angle=A--C--B};
  \draw (C) -- node[above] {$1$} (B) -- node[right] {$x$} (A) -- node[below] {$\sqrt{1-x^2}$} (C);
  \draw (A) +(-.25,0) |- +(0,.25);
  % \pic [my angle, "$\beta$"] {angle=C--B--A};
\end{tikzpicture}
}

\item Pour le cas pair, $\big\Vert t \mapsto \cos(t)^{2p} x^{2p} \big\Vert_{\infty} = \abs{x}^{2p}$ et la série est normalement convergente sur $\interoo{-1}{1}$. Ainsi,
\begin{align*}
\sum_{p=0}^\infty \Wallis_{2p} x^{2p} &= \int_0^{\pi/2} \frac{1}{1 - x^2 \cos(t)^2} \d t \\
&= \int_0^1 \frac{1}{1 - x^2 \left(\frac{1-u^2}{1+u^2}\right)^2} \frac{2}{1 + u^2} \d u \quad \text{en posant } u = \tan \frac{t}{2} \\
&= \int_0^1 \frac{1 + u^2}{(1 + u^2)^2 - x^2(1-u^2)^2} \d u \\
&= \int_0^1 \frac{1 + u^2}{(1 + u^2 - x(1-u^2))(1 + u^2 + x(1-u^2))} \d u \\
&= \frac{1}{2} \int_0^1 \left[ \frac{1}{1 + u^2 - x(1-u^2)} + \frac{1}{1 + u^2 + x(1-u^2)}\right] \d u \\
&= \frac{1}{2} \int_0^1 \left[\frac{1}{1 - x + (1+x)u^2} + \frac{1}{1 + x + (1-x)u^2} \right] \d u \\
&= \frac{1}{\sqrt{1-x^2}} \arctan \sqrt{\frac{x+1}{x-1}} + \frac{1}{\sqrt{1-x^2}} \arctan \sqrt{\frac{x-1}{x+1}}\\
\sum_{p=0}^\infty \Wallis_{2p} x^{2p} &= \frac{\pi}{2} \frac{1}{\sqrt{1-x^2}} \quad \text{car $\arctan u + \arctan \frac{1}{u} = \frac{\pi}{2}$ pour tout $u > 0$}
\end{align*}

\item Pour la dernière formule, $\norm{t \mapsto \cos(t)^p x^p}_{\infty} = \module{x}^p$ et la série converge normalement sur $\interoo{-1}{1}$. Ainsi,
\begin{align*}
\sum_{n=0}^{+ \infty} \Wallis_n x^n &= \sum_{n=0}^{+ \infty} \left[ x^n \int_0^{\pi/2} \cos^n t \d t \right] = \int_0^{\pi/2} \left( \sum_{n=0}^{+\infty} x^n \cos^n t \right) \d t \\
&=\int_0^{\pi/2} \frac{1}{1 - x \cos t} \d t \\
&= \int_0^1 \frac{1}{1 - x \frac{1-u^2}{1+u^2}} \frac{2}{1 + u^2} \d u \quad \text{en posant } u = \tan \frac{t}{2} \\
&= 2 \int_0^1 \frac{1}{(1+x)u^2 + (1-x)} \d u \\
&= 2 \times \frac{1}{1+x} \times \frac{1}{\sqrt{\frac{1-x}{1+x}}} \left[ \arctan\mathopen{}\left( \frac{u}{\sqrt{\frac{1-x}{1+x}}} \right) \right]_0^1 \\
\sum_{n=0}^{+ \infty} \Wallis_n x^n &= \frac{2}{\sqrt{1-x^2}} \arctan \sqrt{\frac{x+1}{x-1}}.
\end{align*}
\end{reponses}
\end{solution}

%---------------

\begin{exercice}
\source{Oral : ENSAM - 2016}
Montrer la convergence et déterminer la somme de la série $\sum (-1)^n \Wallis_n$.
\end{exercice}

\begin{solution}
\begin{enumerate}
\item Posons $u_n = (-1)^n \Wallis_n$. D'après les propriétés des intégrales de \nom{Wallis} démontrées dans l'exercice \ref{exo:propWallis},
\begin{itemize}
\item pour tout $n \in \N$, $u_n u_{n+1} < 0$,
\item la suite de terme général $\abs{u_n} = \Wallis_n$ est décroissante,
\item la suite $(\Wallis_n)_{n \in \N}$ converge vers $0$.
\end{itemize}
Ainsi, d'après le \theoremeutilise{théorème des séries alternées}{theo:seriesalternees}, la série converge.

\item On pose $\fonctionligne[f_n]{t}{\sum\limits_{k=0}^n (-1)^k \cos(t)^k}$. On remarque que
\begin{itemize}
\item la fonction $f_n$ est continue sur~$\interff{0}{\pi/2}$,
\item la suite $(f_n)_{n \in \N}$ converge simplement vers la fonction $t \mapsto \frac{1}{1 + \cos(t)}$ qui est une fonction continue sur~$\interfo{0}{\pi/2}$,
\item d'après le \theoremeutilise{théorème des séries alternées}{theo:seriesalternees}, $\abs{f_n(t)} \leqslant \big|\cos(t)^{n+1}\big| \leqslant 1$, la fonction constante égale à $1$ étant intégrable sur $\interff{0}{\pi/2}$.
\end{itemize}
Ainsi, d'après le \theoremeutilise{théorème de convergence dominée}{theo:convergencedominee},
\[
\sum_{n=0}^{+\infty} (-1)^n \int_0^{\pi/2} \cos(t)^n \d t
= \int_0^{\pi/2} \frac{\d t}{1 + \cos(t)}
= \frac{1}{2} \int_0^{\pi/2} \frac{1}{\cos(t/2)^2} \d t
= 1.
\]
\end{enumerate}
\end{solution}

\begin{remarque}
Pour la majoration de la suite $(f_n)_{n \in \N}$, on peut également utiliser que
\[
\abs{f_n}
= \module{\frac{1 - (-1)^{n+1} \cos^{n+1}(t)}{1 + \cos(t)}}
\leqslant 2.
\]

\end{remarque}

% \begin{exercice}{}
% Calculer $\sum\limits_{n=0}^\infty (-1)^n \Wallis_n$.
% \end{exercice}

% \begin{preuve}
    % \marginnote[0cm]{Source : \href{http://exo7.emath.fr/ficpdf/fic00126.pdf}{Exercices de Jean-Louis \textsc{Rouget} (fic00126) -- \textsf{http://exo7.emath.fr}}}
    % D'après \vrefprop{prop_wallis}, $\Wallis_n \sim \sqrt{\frac{\pi}{2n}}$ et la règle de \textsc{d'Alembert} fournit $R = 1$. Soit $x \in ]-1, 1[$. \\
    % Pour tout $t \in \left[ 0, \frac{\pi}{2} \right]$ et tout entier naturel $n$, $|x^n \cos^n t| \leqslant |x|^n$. Comme la série numérique de terme général $|x|^n$ converge, la série de fonctions de terme général $t \mapsto x^n \cos^n t$ est normalement convergente et donc uniformément convergente sur le segment $\left[ 0, \frac{\pi}{2} \right]$. D'après le théorème d'intégration terme à terme sur un segment, 
    % \begin{align*}
        % \sum_{n=0}^{+ \infty} \Wallis_n x^n &= \sum_{n=0}^{+ \infty} \left[ x^n \int_0^{\pi/2} \cos^n t \d t \right] = \int_0^{\pi/2} \left( \sum_{n=0}^{+\infty} x^n \cos^n t \right) \d t \\
        % &=\int_0^{\pi/2} \frac{1}{1 - x \cos t} \d t \\
        % &= \int_0^1 \frac{1}{1 - x \frac{1-u^2}{1+u^2}} \frac{2}{1 + u^2} \d u \quad \text{en posant } u = \tan \frac{t}{2} \\
        % &= 2 \int_0^1 \frac{1}{(1+x)u^2 + (1-x)} \d u \\
        % &= 2 \times \frac{1}{1+x} \times \frac{1}{\sqrt{\frac{1-x}{1+x}}} \left[ \arctan \left( \frac{u}{\sqrt{\frac{1-x}{1+x}}} \right) \right]_0^1 \\
        % \sum_{n=0}^{+ \infty} \Wallis_n x^n &= \frac{2}{\sqrt{1-x^2}} \arctan \sqrt{\frac{x+1}{x-1}}.
    % \end{align*}
% \end{preuve}

\begin{comment}
\todoinline{Je trouve que ça fait un peu beaucoup ou alors on ne donne pas de preuve pour l'exercice suivant.}

\todoarmand{Nouvel exercice de }

\begin{exercice}
\marginpar[0cm]{Source : \cite{exos_oraux}}
    \textbf{Fonction de \textsc{Bessel} et intégrales de \textsc{Wallis}} \\
    Pour tout $x \in \R$, on note $J(x) = \displaystyle \int_0^{\pi/2} \cos\big(x \sin(t) \big) \d t$.
    \begin{enumerate}
        \item Montrer que $J$ est solution de l'équation différentielles $x y'' + y' + xy = 0$ $(E)$. 
        \item Déterminer les solutions développables en série entière de $(E)$, puis le développement en série entière de $J$. En déduire une expression des intégrales de \textsc{Wallis} d'indice pair. 
    \end{enumerate}
\end{exercice}
\end{comment}

\subsection{Volume d'une boule en dimension \texorpdfstring{$n$}{n}}

La partie suivante constitue une ouverture mais utilise des notions d'intégrales multiples qui dépassent le cadre du programme des classes préparatoires.

\begin{defi}{}
La boule unité de $\R^n$ est définie par
\[
\mathscr{B}_n = \ens[\Big]{(x_1, \dots, x_n) \in \R^n \tq \sum\limits_{i=1}^n x_i{}^2 \leqslant 1}.
\]
Le volume de $\mathscr{B}_n$ est défini par l'intégrale multiple :
\[
\mathscr{V}_n \defeq \idotsint_{x_1{}^2 + \cdots + x_n{}^2 \leqslant 1} \d x_1 \cdots \d x_n.
\]
\end{defi}

\begin{theo}{}
Pour tout $n$ entier naturel non nul,
\[
\mathscr{V}_{2n} = \frac{\pi^n}{n!}
\quad \text{et} \quad
\mathscr{V}_{2n+1} = \frac{2^{2n+1} \pi^n n!}{(2n+1)!}.
\]
\end{theo}

\begin{exercice}
Pour tout $n$ entier naturel non nul, on note
\[
I_n = \int_0^\pi \sin(t)^n \d t.
\]
\begin{questions}
\item Exprimer $I_n$ en fonction de l'intégrale de \nom{Wallis} $\Wallis_n$.

\item En déduire l'expression de $I_n$ en fonction de $n$.

\item Montrer que $\mathscr{V}_n = I_n \times \mathscr{V}_{n-1}$.

\item En déduire l'expression de $\mathscr{V}_n$ en fonction de $n$.
\end{questions}
\end{exercice}

\begin{solution}
\begin{reponses}
\item En utilisant les symétries de la fonction sinus,
\begin{align*}
I_n
% &= \int_0^{\pi} \sin(t)^n \d t\\
&= \int_0^{\frac{\pi}{2}} \sin(t)^n \d t + \int_{\frac{\pi}{2}}^\pi \sin(t)^n \d t
= 2 \int_0^{\frac{\pi}{2}} \sin(t)^n \d t
= 2 \Wallis_n.
\end{align*}

\item D'après la question précédente et les résultats de l'exercice \ref{exo:propWallis}, pour tout $n$ entier naturel,
\begin{align*}
I_{2n} = \frac{(2n)!}{2^{2n} (n!)^2} \frac{\pi}{2}
\quad \text{et} \quad
I_{2n+1} = \frac{2^{2n} (n!)^2}{(2n+1)!}.
\end{align*}

\item D'après la définition,
\begin{align*}
\mathscr{V}_n
&= \int_{-1}^1 \left(\int_{x_2{}^2 + \cdots + x_n{}^2 \leqslant 1 - x_1{}^2} \d x_2 \cdots \d x_n\right) \d x_1.
\end{align*}
Comme $\int_{x_2{}^2 + \cdots + x_n{}^2 \leqslant 1 - x_1{}^2} \d x_2 \cdots \d x_n$ est le volume de la boule de rayon $1 - x_1{}^2$, cette quantité est égale à $\big(1 - x_1{}^2\big)^{\frac{n-1}{2}} \mathscr{V}_{n-1}$. Ainsi,
\begin{align*}
\mathscr{V}_n
&= \mathscr{V}_{n-1} \int_{-1}^1 \big(1 - x_1{}^2\big)^{\frac{n-1}{2}} \d x_1
= \mathscr{V}_{n-1} \times I_n,
\end{align*}
où on a utilisé le changement de variables $x_1 = \cos(\theta)$.

\item D'après la question précédente,
\begin{align*}
\mathscr{V}_n
&= 2 \Wallis_n \mathscr{V}_{n-1}
= 2^{n-2} \prod_{k=3}^n \Wallis_k \times \mathscr{V}_2.
\end{align*}
Or, $\mathscr{V}_2$ est l'aire d'un disque de rayon $1$ qui vaut $\pi$. Alors,
\[
\mathscr{V}_n = \pi\, 2^{n-2} \prod_{k=3}^n \Wallis_k.
\]

Rappelons enfin que $\Wallis_k \Wallis_{k+1} = \frac{\pi}{2(k+1)}$. Ainsi, en distinguant selon la parité de $n$,
\begin{align*}
\mathscr{V}_{2n}
&= \pi\, 2^{2n-2} \prod_{k=3}^{2n} \Wallis_k
= \pi\, 2^{2n-2} \prod_{k=2}^{n} \frac{\pi}{2 \times 2 k}
= \frac{\pi\, 2^{2n-2} \pi^{n-1}}{2^{2(n-1)} \times n!}
= \frac{\pi^n}{n!}
\end{align*}
et
\begin{align*}
\mathscr{V}_{2n+1}
&= 2 \Wallis_{2n+1} \mathscr{V}_{2n}
= 2 \times \frac{2^{2n} (n!)^2}{(2n+1)!} \times \frac{\pi^n}{n!}
= \frac{2^{2n+1} \pi^n n!}{(2n+1)!}.
\end{align*}
\end{reponses}
\end{solution}

\begin{remarque}
Interprétons physiquement ces considérations. En physique statistique, on utilise fréquemment l'espace des phases comprenant comme dimensions les six coordonnées de position et de vitesse de chaque particule. lorsque celles-ci sont en très grand nombre, l'espace des phases a énormément de dimensions. Supposons que nous mesurions la vitesse de chacune des $n$ particules : la moyenne $V$ de ces vitesses donnera une mesure macroscopique comme la témpérature ou la pression (quantités liées à la vitesse). Reportons cette mesure sur chacun des axes d'un espace à $n$ dimensions : les points considérés seront approximativement sur une coquille de rayon $V$ et d'épaisseur $\frac{V}{\sqrt{n}}$ (voir ch. 13). On peut également considérer que les vitesses vont se répartir dans une sphère de rayon inférieur ou égal à la plus grande mesure obtenue $V_\mathrm{max}$, mais comme les points se retrouveront quasiment à la surface, la plupart des vitesses seront environ $V_\mathrm{max}$ qui sera en fait la vitesse moyenne : nous retrouvons là l'hypothèse ergodique de \nompropre{Boltzmann} et \nompropre{Maxwell}. \url{http://promenadesmaths.free.fr/fichiers_pdf/volume%20en%20dim%20n.pdf} \url{https://www.phy.ulaval.ca/fileadmin/phy/documents/Bacc_PHY/Nathalie/Espace-phase-pdf.pdf}
\end{remarque}




\subsection{\textsc{Grain de raisin}: Produit de \nom{Wallis}}

\marginnote[0cm]{
    L'expression \chevron{Grain de raisin} est une référence à l'extrait suivant \url{https://youtube.com/clip/UgkxzacmUJF7Qr3526JmMgMOwLirh_c-gddc?si=AUUGMWZPFH8aNZl7} d'une discussion entre Jean-Pierre \nom{Serre} et Christophe \nom{Ritzenthaler}.
    % \begin{figure}
    % \centering
    \begin{center}
    \includegraphics[width=1cm]{images/qrcode_grainderaisin.png}
    \end{center}
    % \end{figure}
}

\begin{prop}{Produit de \nom{Wallis}}
    Formule énoncée en 1656 par John \nom{Wallis}, dans son ouvrage \emph{Arithmetica infinitorum}
    $$\prod_{n=1}^{\infty} \frac{4n^2}{4n^2-1} = \frac{\pi}{2}.$$
\end{prop}

\begin{demo}
    Puisque $\Wallis_{2n} \sim \Wallis_{2n+1}$, 
    $$\lim_{n \to +\infty} \frac{\Wallis_{2n+1}}{\Wallis_{2n} / \frac{\pi}{2}} = \frac{\pi}{2}.$$
    Or d'après le calcul des intégrales de \nom{Wallis}:
    $$\frac{\Wallis_{2n+1}}{\Wallis_{2n} / \frac{\pi}{2}} = \frac{\frac{2^{2p}(p!)^2}{(2p+1)!}}{\frac{(2p)!}{2^{2p}(p!)^2}} = \prod_{k=1}^n \frac{4k^2}{4k^2-1}.$$
\end{demo}


    %-----------
% \subsection{À trier}

% \todoinline{Cet exercice, je le supprimerais ou alors le mettre avec les intégrales de Gauss ?}

% \begin{exercice}
    % \marginnote[0cm]{Source : \cite{fmaalouf}}
    % Pour $n \in \Ne$ et $R \in \Rpe$ on désigne par $V_n(R)$ le volume de la boule de $\R^n$ de centre $O$ et de rayon $R$, 
    % $$V_n(R) \defeq \idotsint_{x_1^2 + \cdots + x_n^2 \leqslant R^2} \d x_1 \cdots \d x_n.$$
    % Montrer que pour tout $p \in \Ne$, 
    % $$V_{2p}(R) = \frac{\pi^p R^{2p}}{p!}.$$
% \end{exercice}

% \todoinline{Ajouter une preuve.}
% \todoarmand{Je suis entrain de réfléchir à une démo, mais ça me paraît difficile sans la notion de jacobien.}


% \todoarmand{Les exercices 4 et 5 de \url{http://exo7.emath.fr/ficpdf/fic00143.pdf} donnent les liens entre calcul de la surface de la sphère unité de $\R^n$, son volume, la fonction Gamma et les intégrales de Wallis.}

% Exercices recopiés : 

% \begin{exercice}
    % \emph{Cet exercice fournit une autre méthode de calcul du volume de la boule unité $\mathscr{B}_n$ de $\R^n$ et de l'aire de la sphère $\mathscr{S}_{n-1} \subset \R^n$.} On conserve les notations de l'exercice précédent. 
    % \begin{enumerate}
        % \item Montrer que $\mathscr{V}_n = I_n \cdot \mathscr{V}_{n-1}$, où $I_n = \int_0^\pi \sin(t)^n \d t$. 
        % \item Vérifier que $I_n = \frac{n-1}{n}I_{n-2}$.
    % \end{enumerate}
% \end{exercice}


\section{Intégrales eulériennes}\label{secinteuleriennes}

\source{\href{https://lescoursdemathsdepjh.monsite-orange.fr/file/7bcfcf82249b1046f185e9a3495845cd.pdf}{Fonction eulériennes -- Pierre-Jean \nompropre{Hormière}}}
\todoarmand{Glossaire des mathématiciens : distinguer entre les Bernoulli}
De premières tentatives pour définir la factorielle de valeurs non entières remontent à James \nom{Stirling} et Daniel \nom{Bernoulli}. Dans une lettre à Christian \nom{Goldbach} du 13 octobre 1729, \nom{Euler} découvre (ou invente ?) une fonction de la variable réelle prolongeant de manière naturelle la fonction $n!$. D'abord introduite comme limite de produits, cette fonction fut plus tard présentée sous forme intégrale et reliée à des fonctions voisines.

Les fonctions eulériennes sont les plus importantes \say{ fonctions spéciales } de l'analyse classique, réelle et complexe. \nom{Legendre} les a nommées, classifiées et étudiées. Elles ont aussi été étudiées par \nom{Gauss}, \nom{Binet}, \textsc{Plana}, \textsc{Malmsten}, \textsc{Raabe}, \textsc{Weierstrass}, \textsc{Hankel}, H. \textsc{Bohr}, \textsc{Mollerup}, \textsc{Artin}, \ldots
\todoarmand{Utiliser la commande nom}

On trouvera une présentation du contexte historique de l'extension de la fonction factorielle dans l'article~\cite{davis1959}.

\todoarmand{
\url{https://people.math.osu.edu/gautam.42/S20/DavisGammaFunction.pdf} : un texte passionnant qui raconte en détail toute l'aventure intellectuelle autour de la fonction Gamma et qui explique (p. 18-20) la manière dont on peut construire des pseudo fonctions Gamma qui vérifient l'essentiel des propriétés d'un prolongement de la fonction $n!$.}

Il y a bien des façons de prolonger la fonction $n!$ au domaine réel, même en se limitant aux fonctions continues. Une idée naturelle est de partir de la formule $\displaystyle n! = \int_{0}^{+ \infty} t^n \e^{-t} \d t$. Cette forme intégrale de la factorielle suggère de considérer la fonction $\displaystyle F(x)=\int_{0}^{+ \infty} t^x \e^{-t} \d t$. Cette fonction, définie sur $\interoo{-1}{+\infty}$, prolonge intelligemment la factorielle, en ce sens qu'elle possède des propriétés nombreuses et cohérentes. Par commodité, on considère plutôt $\displaystyle \Gamma(x) = \int_{0}^{+ \infty} t^{x-1} \e^{-t} \d t$.
\todoarmand{Introduction à réécrire à partir de la ressource ci-dessus.}

\subsection{Fonction Gamma d'\nom{Euler}}\label{subsec:FonctionGammaEuler}

%\begin{marginfigure}[3cm]
%    \begin{tikzpicture}[]

\begin{axis}[
xmin = -4.9, xmax = 5.1, 
%ymin = -3.5, ymax = 3.5,  
restrict y to domain=-6:6,
axis lines = middle,
axis line style={-latex},  
xlabel={$x$}, 
ylabel={$\Gamma(x)$},
%enlarge x limits={upper={val=0.2}},
enlarge y limits=0.05,
x label style={at={(ticklabel* cs:1.00)}, inner sep=5pt, anchor=north},
y label style={at={(ticklabel* cs:1.00)}, inner sep=2pt, anchor=south east},
]

\addplot[color=red, samples=222, smooth, 
domain = 0:5] gnuplot{gamma(x)};

\foreach[evaluate={\N=\n-1}] \n in {0,...,-5}{%
\addplot[color=red, samples=555, smooth,  
domain = \n:\N] gnuplot{gamma(x)};
%
\addplot [domain=-6:6, samples=2, densely dashed, thin] (\N, x);
}%
\end{axis}
\end{tikzpicture}
%    \caption*{\centering Graphe de la fonction Gamma}
%\end{marginfigure}

\begin{defi}[Fonction Gamma d'\nom{Euler}]
    La \emph{fonction Gamma d'\nom{Euler}} est définie par: 
    $$\Gamma(x) \defeq \int_{0}^{+\infty} t^{x-1} \e^{-t} \d t.$$
\end{defi}

\begin{remarque}
    À un changement de variable près, la fonction $\Gamma$ est la \nameref{transformee_laplace} de la fonction $t \mapsto t^x$. 
\end{remarque} 

\begin{theo}[Propriétés fondamentales]
\begin{itemize}
\item La fonction $\Gamma$ est définie sur $\Rpe$.
\item Pour tout $x > 0$, $\Gamma(x+1) = x\Gamma(x)$.
En particulier, pour tout $n \in \N$, $\Gamma(n+1) = n!$. 
\end{itemize}
\end{theo}

\begin{demo}
\begin{itemize}
\item On pose $\fonctionligne[f_x]{t}{t^{x - 1} \e^{-t}}$.

La fonction $f_x$ est continue sur $\R_+^\star$. Comme $f_x$ est à valeurs positives, son intégrabilité est équivalente à la convergence de l'intégrale.

Comme $f_x(t) = o_{+\infty}\mathopen{}\left(\frac{1}{t^2}\right)$, d'après les \theoremeutilise{théorèmes de comparaison aux intégrales de \nom{Riemann}}{theo:comparaisonintegralesriemann}, $f_x$ est intégrable sur $\interfo{1}{+\infty}$.

Comme $f_x(t) \sim_0 \frac{1}{t^{1-x}}$, alors $f_x$ est intégrable sur $\interof{0}{1}$ si et seulement si $x > 0$.

Finalement, $\Gamma$ est définie sur $\Rpe$.

\item Les fonctions $t \mapsto \e^{-t}$ et $t \mapsto t^x$ sont de classe $\Cont^1$ sur $\Rpe$. D'après les croissances comparées, $\lim_{t\to0} t^x \e^{-t} = \lim_{t\to+\infty} t^x \e^{-t} = 0$. 
\item D'après la formule d'\theoremeutilise{intégration par parties généralisée}{theo:ippgeneralisees},
\[
\int_0^{+\infty} t^x \e^{-t} \d t = x \int_0^{+\infty} t^{x-1} \e^{-t} \d t
\]
et $\Gamma(x+1) = x \Gamma(x)$.

\item On remarque que $\Gamma(1) = 1$ et on montre par récurrence que $\Gamma(n+1) = n!$.
\end{itemize}
\end{demo}

% La démonstration est donnée dans \ref{prolongementFonctionGamma}.

% Cette fonction, introduite en 1729 par le mathématicien suisse, prolonge la fonction factorielle à l'ensemble des réels strictement positifs.

\begin{theo}[Régularité]
La fonction $\Gamma$ est de classe $\Cont^\infty$ sur $\Rpe$. De plus,
\[
\forall k \in \N,\quad \forall x \in \Rpe,\quad \Gamma^{(k)}(x) = \int_{0}^{+\infty} (\ln t)^k t^{x-1} \e^{-t} \d t.
\]
\end{theo}

\begin{demo}
\begin{itemize}
\item \textbf{Étude de la continuité.} Soit $0 < a < A$. On pose $\fonctionligne[f]{(x, t)}{t^{x-1} \e^{-t}}$.
\begin{itemize}
\item Soit $t \in \Rpe$. Alors, $f(\,\cdot\,, t)$ est continue sur $\interff{a}{A}$.

\item Soit $x \in \interff{a}{A}$. Alors, $f(x, \cdot)$ est continue sur $\Rpe$.

\item Pour tout $(x, t) \in \interff{a}{A} \times \Rpe$,
\[
\abs{f(x, t)} \leqslant \phi(t) = 
\begin{cases}
\frac{\e^{-t}}{t^{1-a}} &\text{ si } t \in \interof{0}{1},\\
t^{A-1} \e^{-t} &\text{ si } t \in \interfo{1}{+\infty}
\end{cases}
\]
La fonction $\phi$ est continue sur $\Rpe$. De plus, $\phi(t) \sim_0 \frac{1}{t^{1 - a}}$ et $\phi(t) = o_{+\infty}\mathopen{}\left(\frac{1}{t^2}\right)$, donc $\phi$ est intégrable sur~$\Rpe$.
\end{itemize}
Finalement, d'après le \theoremeutilise{théorème de continuité sous le signe intégral}{theo:continuitesoussigneintegral}, $\Gamma$ est continue sur $\interff{a}{A}$ pour tous $0 < a < A$ et $\Gamma$ est donc continue sur $\Rpe$.

\item On applique le \theoremeutilise{théorème de dérivation sous le signe intégral}{theo:derivationsoussigneintegrale} sur l'intervalle $\interff{a}{A}$:
\begin{itemize}
\item la fonction $\fonctionligne[f(\,\cdot\,, t)]{t}{t^{x - 1} \e^{-t}}$ est de classe $\mathscr{C}^k$ sur $\interff{a}{A}$.

\item la dérivée partielle $\partial_1^j f(x, t) = \ln(t)^j t^{x - 1} \e^{-t}$ et $t \mapsto \partial_1^j f(x, t)$ est continue sur $\Rpe$.

\item pour tout $x \in \interff{a}{A}$,
\[
\abs{\partial_1^j f(x, t)} \leqslant
\begin{cases}
\abs{\ln(t)}^j \e^{-t} t^{a-1} &\text{ si } t < 1 \\
\abs{\ln(t)}^j \e^{-t} t^{A-1} &\text{ si } t \geqslant 1
\end{cases}
\]

En notant $\phi_j$ cette fonction,
\begin{itemize}
\item $\phi_j$ est continue sur $\Rpe$.
\item $\phi_j(t) = o_{+\infty}\mathopen{}\left(\frac{1}{t^2}\right)$ et $\phi_j(t) = o_0(t^{a/2-1})$. Ainsi, $\phi_j$ est intégrable sur $\Rp$.
\end{itemize}
\end{itemize}
Finalement, $\Gamma$ est de classe $\mathscr{C}^\infty$ et
\[
\Gamma^{(j)}(x) = \int_0^{+\infty} (\ln t)^j t^{x-1} \e^{-t} \d t.
\]
\end{itemize}
\end{demo}

% \begin{elem_preuve}
    % Utiliser une domination locale sur un segment $[a, A] \subset \R_+^\star$ par la fonction:
    % $$\varphi_k:t \mapsto 
    % \begin{cases}
        % |\ln t |^k \e^{-t} t^{a-1} & \text{si } t \in ]0, 1], \\
        % |\ln t |^k \e^{-t} t^{A-1} & \text{si } t > 1.
    % \end{cases}
    % $$
% \end{elem_preuve}

\begin{exercice}
\source{\cite{fmaalouf}}
\begin{questions}
\item Montrer que $\Gamma(x) \isEquivTo{0^+} \frac{1}{x}$.

\item Montrer que $\Gamma$ est convexe.

\item Montrer que $\ln(\Gamma)$ est convexe.\\
\emph{Indication :} On pourra utiliser l'inégalité de \nom{Cauchy}--\nom{Schwarz}.
% et étudier ses variations.
\end{questions}
\end{exercice}

\begin{solution}
\begin{reponses}
\item Comme $\Gamma$ est continue en $1$, 
\[
\lim_{x\to0} \Gamma(x + 1) = \Gamma(1) = \int_0^{+\infty} \e^{-t} \d t = 1.
\]

De plus, pour tout $x > 0$, $\Gamma(x + 1) = x \Gamma(x)$. Ainsi,
\[
\Gamma(x) \sim_0 \frac{1}{x}.
\]

\item D'après l'étude de régularité précédente, pour tout $x > 0$,
\[
\Gamma''(x) = \int_0^{+\infty} (\ln t)^2 t^{x-1} \e^{-t} \d t.
\]
Ainsi, $\Gamma'' > 0$ et $\Gamma$ est convexe.

\item Comme $\Gamma$ est à valeurs strictement positives, alors la fonction $f = \ln(\Gamma)$ est bien définie. De plus, comme la fonction $\Gamma$ est de classe $\mathscr{C}^2$, alors $f$ est de classe $\mathscr{C}^2$ et
\begin{align*}
f' &= \frac{\Gamma'}{\Gamma},\,
%
f'' = \frac{\Gamma'' \Gamma - (\Gamma')^2}{\Gamma^2}.
\end{align*}

De plus, d'après l'inégalité de \nom{Cauchy}--\nom{Schwarz}, pour tout $x > 0$,
\begin{align*}
\Gamma'(x)^2
&= \left(\int_0^{+\infty} \ln(t) t^{x-1} \e^{-t} \d t\right)^2 \\
&= \left(\int_0^{+\infty} \left(t^{(x-1)/2} \e^{-t/2}\right) \times \left(\ln(t) t^{(x-1)/2} \e^{-t/2}\right) \d t\right)^2 \\
&\leq \left(\int_0^{+\infty} t^{x-1} \e^{-t} \d t\right) \times \left(\int_0^{+\infty} \ln(t)^2 t^{x-1} \e^{-t} \d t\right)^2 \\
&\leq \Gamma(x) \Gamma''(x).
\end{align*}
Ainsi, $f'' \geq 0$ et la fonction $\ln(\Gamma)$ est convexe.
\end{reponses}
\end{solution}

%..............
\subsubsection{Caractérisation de $\Gamma$}

\begin{theo}
\source{\cite{rudin2009}}
Soit $f$ une fonction définie sur $\Rpe$ telle que
\begin{enumerate}
\item pour tout $x > 0$, $f(x + 1) = x f(x)$,
\item $f(1) = 1$,
\item $\ln(f)$ est convexe.
\end{enumerate}
Alors, $f = \Gamma$ et
\[
\forall\, x > 0,\,
\Gamma(x) = \lim_{n\to+\infty} \frac{n! n^x}{x (x + 1) \cdots (x + n)}.
\]
\end{theo}

\begin{exercice}
On pose $g = \ln(f)$.
\begin{questions}
\item Pour tout $n \in \Ne$, déterminer $f(n)$.

\item Montrer que, si $f = \Gamma$ sur $\interof{0}{1}$, alors $f = \Gamma$ sur $\Rpe$.
\end{questions}

Soit $x \in \Rpe$ et $n \in \Ne$.
\begin{questions}[resume]
\item Montrer que
\[
g(n + 1) - g(n) \leq \frac{g(n + 1 + x) - g(n + 1)}{x} \leq g(n + 2) - g(n + 1).
\]

\item En déduire que
\[
x \ln(n) \leq g(x) + \ln\left(x (x + 1) \cdots (x + n)\right) - \ln(n!) \leq x \ln(n + 1).
\]

\item Conclure.
\end{questions}
\end{exercice}

\begin{solution}
\begin{reponses}
\item Comme $f(1) = 1$ et $f(n + 1) = n f(n)$, on montre par récurrence que, pour tout $n \in \Ne$, $f(n) = n!$.

\item D'après la question précédente, $f$ et $\Gamma$ sont égales sur $\Ne$.

Supposons que $f = \Gamma$ sur $\interof{0}{1}$. On montre alors par récurrence sur $n$ que $f = \Gamma$ sur $]n, n + 1]$ car, si $x \in ]n, n + 1]$, alors
\begin{align*}
f(x) = (x - 1) f(x - 1)
\text{ et }
\Gamma(x) = (x - 1) \Gamma(x - 1)
\end{align*}
et $x - 1 \in ]n-1, n]$.

\item Comme la fonction $g$ est convexe, son taux d'accroissement est croissant. En appliquant ce résultat à $t \mapsto \frac{g(n + 1) - g(t)}{n + 1 - t}$ successivement en $n$, $(n + 1 + x)$ et $(n + 2)$, alors
\begin{align*}
\frac{g(n + 1) - g(n)}{(n + 1) - n}
&\leq \frac{g(n + 1) - g(n + 1 + x)}{(n + 1) - (n + 1 + x)}
\leq \frac{g(n + 1) - g(n + 1)}{(n + 1) - (n + 2)}\\
g(n + 1) - g(n) &\leq \frac{g(n + 1 + x) - g(n + 1)}{x} \leq g(n + 2) - g(n + 1).
\end{align*}

\item Comme $g(n + 1) = \ln(n!)$, d'après l'inégalité précédente,
\begin{align*}
\ln(n) &\leq \frac{g(n + 1 + x) - \ln(n!)}{x} \leq ln(n + 1)\\
x \ln(n) &\leq g(n + 1 + x) - \ln(n!) \leq x \ln(n + 1).
\end{align*}

Enfin, d'après la propriété de $f$,
\begin{align*}
g(n + 1 + x)
&= \ln\left(f(n + 1 + x)\right)
= \ln\left((n + x) f(n + x)\right)
= \cdots
= \ln\left((n + x) (n + x - 1) \cdots x f(x)\right)\\
&= g(x) + \ln\left(x (x + 1) \cdots (x + n)\right).
\end{align*}

On obtient ainsi l'encadrement souhaité :
\[
x \ln(n)
\leq g(x) - \ln\frac{n!}{x (x + 1) \cdots (x + n)} \leq x \ln(n + 1).
\]

\item D'après la question précédente,
\[
0 \leq g(x) - \ln \frac{n! n^x}{x (x + 1) \cdots (x + n)} \leq x \ln\left(1 + \frac{1}{n}\right).
\]

Ainsi, d'après le \theoremeutilise{théorème d'encadrement}{theo:encadrement},
\[
\lim_{n\to+\infty} \frac{n! n^x}{x (x + 1) \cdots (x + n)} = g(x).
\]

La fonction $g$ est donc unique et égale à $\Gamma$.
\end{reponses}
\end{solution}

%..............
\subsubsection{$\Gamma$ et $\gamma$}

\begin{defi}{Constante $\gamma$ d'\nom{Euler}}
\source{Oral : Centrale 2 - 2021}
La limite suivante est bien définie et appelée constante $\gamma$ d'\nom{Euler} :
\[
\gamma = \lim_{n\to+\infty} \left(\ln(n) - \sum_{k=1}^n \frac{1}{k}\right).
\]
\end{defi}

\begin{theo}
\[
\Gamma'(1) = -\gamma.
\]
\end{theo}

\source{Patrice \nompropre{Lassère}, p. 210}
\begin{exercice}
Pour tout $n \in \Ne$, on note $H_n = \sum_{k=1}^n \frac{1}{k}$.
\begin{questions}
\item Montrer que 
\[
\Gamma'(1) = \lim_{x \to \infty} \int_0^n \left(1 - \frac{t}{n} \right)^n \ln(t) \d t
\]

\item En utilisant un changement de variable affine, montrer que
\[
\int_0^n \left(1 - \frac{t}{n}\right)^n \ln(t) \d t
= \frac{n}{n+1} \ln(n) + n \int_0^1 (1 - u)^n \ln(u) \d u.
\]


\item En utilisant une \theoremeutilise{intégration par parties généralisée}{theo:ippgeneralisees} puis une factorisation de $(1 - u)^{n+1} - 1$, montrer que
\[
(n+1) \int_0^1 (1 - u)^n \ln(u) \d u
= -\sum_{k=0}^n \int_0^1 (1 - u)^k \d t.
\]

\item En déduire que
\[
\int_0^n \left(1 - \frac{t}{n}\right)^n \ln(t) \d t
= \frac{n}{n + 1} (\ln(n) - H_{n+1}).
\]

\item Conclure.
\end{questions}
\end{exercice}

\begin{solution}
\begin{reponses}
\item La convexité de la fonction exponentielle assure que pour tout $t \in \interff{0}{n}$,
\[
\left(1 - \frac{t}{n}\right)^n \leq \e^{-t}.
\]

Posons $f : t \mapsto \ln(t) \e^{-t}$. La fonction $f$ est continue sur $\Rpe$, $f(t) \sim_0 \ln(t)$ et $f(t) = o_{+\infty}(1/t^2)$. Ainsi, la fonction $f$ est intégrable sur $\Rpe$.

On obtient la limite annoncée en utilisant le \theoremeutilise{théorème de convergence dominée}{theo:convergencedominee}.

\item En effectuant le changement de variable affine $u \mapsto n u$,
\begin{align*}
\int_0^n \left(1 - \frac{t}{n}\right)^n \ln(t) \d t
&= n \int_0^1 \left(1 - u\right)^n \ln(n u) \d u\\
&= n \ln(n) \int_0^1 \left(1 - u\right)^n \d u + n \int_0^1 (1 - u)^n \ln(u) \d u\\
&= \frac{n}{n+1} \ln(n) + n \int_0^1 (1 - u)^n \ln(u) \d u.
\end{align*}

\item Les fonctions $u \mapsto -\frac{(1 - u)^{n+1} - 1}{n+1}$ et $u \mapsto \ln(u)$ sont de classe $\mathscr{C}^1$ sur $]0, 1]$ et, d'après les croissances comparées, leur produit tend vers $0$ en $0$. Ainsi, d'après la formule d'\theoremeutilise{intégration par parties généralisée}{theo:ippgeneralisees} puis à l'aide de la formule de \nom{Bernoulli},
\begin{align*}
(n+1) \int_0^1 (1 - u)^n \ln(u) \d u
&= \int_0^1 \frac{\left(1 - u\right)^{n+1} - 1}{u} \d t
= \int_0^1 \frac{- u \sum_{k=0}^n (1 - u)^k}{u} \d t
= -\sum_{k=0}^n \int_0^1 (1 - u)^k \d t.
\end{align*}

\item Finalement,
\begin{align*}
\int_0^n \left(1 - \frac{t}{n}\right)^n \ln(t) \d t
&= \frac{n}{n+1} \left(\ln(n) - \sum_{k=0}^n \int_0^1 (1 - u)^k \d t\right)
= \frac{n}{n+1} \left(\ln(n) - \sum_{k=0}^n \frac{1}{k+1}\right)
= \frac{n}{n+1} \left(\ln(n) - H_{n+1}\right).\\
\end{align*}

\item La conclusion est alors immédiate.
\end{reponses}
\end{solution}


%..............
\subsubsection{Fonction $\Gamma$ et intégrale de \nom{Gauss}}
\marginnote[0cm]{\href{sec:intGauss}{\faLink~Intégrale de \nom{Gauss}}}

\begin{theo}
$\Gamma\mathopen{}\left(\frac{1}{2}\right) = \sqrt{\pi}$
et pour tout $n \in \Ne$,
$\Gamma\mathopen{}\left(n + \frac{1}{2}\right) = \frac{(2n)!}{4^n n!} \sqrt{\pi}$.
\end{theo}
% \begin{exercice}
    \source{\cite{fmaalouf}}
    % Caculer $\Gamma \left( \frac{1}{2} \right)$, puis $\Gamma \left( n + \frac{1}{2} \right)$ pour tout $n \in \Ne$.
% \end{exercice}

% \marginnote[0cm]{Source : \href{https://fr.wikipedia.org/wiki/Intégrale_de_Gauss#Calcul_de_l'intégrale_de_Gauss}{Intégrale de Gauss -- \textsf{wikipedia.org}}}

\begin{elemdemo}
On utilise le changement de variable $\mathscr{C}^1$ et bijectif $u \mapsto u^2$. Alors,
\begin{align*}
\Gamma\mathopen{}\left( \frac{1}{2} \right)
&= \int_0^{+ \infty} t^{\frac{1}{2}-1} \e^{-t} \d t
= 2 \int_0^{+ \infty} \e^{-u^2} \d u
= \int_{- \infty}^{+ \infty} \e^{-u^2} \d u
= \sqrt{\pi},
\end{align*}
d'après les résultats sur l'intégrale de \nom{Gauss}.

Le second point s'obtient en utilisant l'égalité
\[
\Gamma\left(n + \frac{1}{2}\right) = \frac{n + 2}{2} \Gamma\left(n - 1 + \frac{1}{2}\right).
\]
\end{elemdemo}

%-----------
\subsection{Fonction bêta}

\begin{defi}[Fonction bêta]
Pour tout $(x, y) \in (\Rpe)^2$, on définit
\[
B(x, y)
= \int_0^1 t^{x-1} (1 - t)^{y-1} \d t.
\]
\end{defi}

\begin{remarque}
On pose $f_{x,y} : t \mapsto t^{x-1} (1 - t)^{y-1}$.

La fonction $f_{x,y}$ est continue et à valeurs positives sur $\interoo{0}{1}$.

Comme $f_{x,y}(t) \sim_0 t^{x-1}$, la fonction $f_{x,y}$ est intégrable en $0$ si et seulement $x > 0$.

De même, $f_{x,y}(t) \sim_1 (1 - t)^{y-1}$ et la fonction $f_{x,y}$ est intégrable en $1$ si et seulement si $y > 0$.

La fonction bêta est donc bien définie sur $(\Rpe)^2$.
\end{remarque}


\begin{comment}
\begin{defi}[Fonction bêta]
Pour tout $(p,q) \in \N^2$, on note
$$B(p, q) \defeq \int_{0}^{1} t^p (1-t)^q \d t.$$
\end{defi}
\end{comment}

%..............
\subsubsection{Paramètres entiers}

\begin{theo}[Expression factiorelle de la fonction bêta]
Pour tout $(p,q) \in \N^2$,
    $$B(p+1, q+1) = \frac{p! q!}{(p + q + 1)!}.$$
\end{theo}

\source{
\begin{itemize}
    \item \href{https://fr.wikipedia.org/wiki/Intégrale_d'Euler}{Intégrale d'\nom{Euler} -- \textsf{wikipedia.org}}
    \item \cite{dieudonne1968} Chapitre IV, 3 Intégrales eulériennes, page 125.
\end{itemize}
}

\begin{exercice}
\begin{questions}
\item Déterminer $B(p+1, 1)$.

\item Pour tout $(p, q) \in \N \times \Ne$, montrer que $B(p+1, q+1) = \frac{q}{p + 1} B(p+2, q)$.

\item Conclure.
\end{questions}
\end{exercice}

\begin{solution}
\begin{reponses}
\item On a immédiatement $B(p+1, 1) = \int_0^1 t^p \d t = \frac{1}{p + 1}$.

\item On pose $\fonctionligne[u]{t}{\frac{1}{p+1} t^{p+1}}$ et $\fonctionligne[v]{t}{(1-t)^q}$, toutes deux de classe $\mathscr{C}^1$ sur $\interff{0}{1}$. Alors, d'après la formule d'intégration par parties, 
\begin{align*}
B(p+1, q+1)
&= \left[ \frac{1}{p+1}t^{p+1} \times (1-t)^q \right]_0^1 + \frac{q}{p+1} \int_0^1 t^{p+1} (1-t)^{q-1} \d t
= \frac{q}{p+1} B(p+2, q).
\end{align*}

On en déduit que 
\begin{align*}
B(p+1, q+1)
&= \frac{q}{p+1} \times \frac{q-1}{p+2} \times \cdots \times \frac{1}{p+q} B(p+q+1, 1)
= \frac{p! q!}{(p + q + 1)!}.
\end{align*}
\end{reponses}
\end{solution}

\begin{exercice}[Oral CentraleSupélec 2016]
Soit $p \in \Ne$. On dispose de $p$ urnes numérotées de $1$ à $p$. Chaque urne contient $p$ boules et, pour tout $i \in \interent{1}{p}$, l'urne numéro $i$ contient $i$ boules noires et $p - i$ boules blanches. On effectue l'expérience suivante : choisir au hasard une urne puis effectuer des tirages avec remise dans l'urne choisie. On note, pour $n \in \Ne$, $A_n$ l'événement \textit{on a effectué $2 n$ tirages et obtenu le même nombre de boules blanches que de noires}.

\begin{questions}
\item Exprimer $\mathbf{P}(A_n)$ sous forme d'une somme.

\item On note $b_{n,p}$ la probabilité que la prochaine boule tirée soit blanche sachant que $A_n$ est réalisé. Exprimer~$b_{n,p}$.
\item Calculer $\lim\limits_{p\to+\infty} b_{n,p}$.
\end{questions}
\end{exercice}

\begin{solution}
On note $U_i$ l'événement \textit{Choisir l'urne $i$}.
\begin{reponses}
\item Comme $(U_1,\ldots,U_p)$ est un système complet d'événements, en utilisant la \theoremeutilise{formule des probabilités totales}{theo:formuleprobabilitestotales},
\begin{align*}
\mathbf{P}(A_n) &= \sum_{i=1}^p \mathbf{P}_{U_i}(A_n) \mathbf{P}(U_i) \\
&= \sum_{i=1}^n \binom{2n}{n} \left(\frac{p-i}{p}\right)^n \left(\frac{i}{p}\right)^n.
\end{align*}

\item En utilisant le même raisonnement que précédemment,
\[
b_{n,p}
= \frac{\frac{1}{p} \sum\limits_{i=1}^p \left(1 - \frac{i}{p}\right)^{n+1} \left(\frac{i}{p}\right)^n}{\frac{1}{p} \sum\limits_{i=1}^p \left(1 - \frac{i}{p}\right)^{n} \left(\frac{i}{p}\right)^n} \cdot \frac{\binom{2n+1}{n+1}}{\binom{2n}{n}}.
\]

\item En utilisant les limites des sommes de \nom{Riemann},
\[
\lim\limits_{p\to+\infty} b_{n,p} = \frac{2n+1}{n+1} \cdot \frac{\int_0^1 (1 - x)^{n+1} x^n \d x}{\int_0^1 (1 - x)^n x^n \d x}.
\]
Cette limite vaut donc
\[
\frac{2n+1}{n+1} \cdot \frac{n! (n+1)!}{(2n+2)!} \cdot \frac{(2n+1)!}{n! n!}  = \frac{2n+1}{2n+2}.
\]
\end{reponses}
\end{solution}

%..............
\subsubsection{Des fonctions Bêta et des sommes}

\begin{exercice}
    \source{Clémentine \nompropre{Portal} (PCSI1, Collège Stanislas) feuille d'exo n°7, exo 11}
Soit $(p, q) \in \N^2$.

Déterminer une expression sans signe somme de $\sum\limits_{k=0}^q \binom{q}{k} \frac{(-1)^k}{p+k+1}$.
\end{exercice}

\begin{solution}
En utilisant la formule du binôme de \nom{Newton} puis la linéarité de l'intégrale,
\begin{align*}
B(p+1, q+1)
&= \int_0^1 t^p (1 - t)^q \d t\\
&= \int_0^1 t^p \left(\sum_{k=0}^q \binom{q}{k} (-1)^k t^k\right) \d t\\
&= \sum_{k=0}^q \binom{q}{k} (-1)^k \left(\int_0^1 t^{p + k} \d t\right)\\
&= \sum_{k=0}^q \binom{q}{k} \frac{(-1)^k}{p + k + 1}.
\end{align*}
Ainsi,
\[
\sum_{k=0}^q \binom{q}{k} \frac{(-1)^k}{p + k + 1}
= \frac{p! q!}{(p + q + 1)!}.
\]
\end{solution}

%---------------

\begin{exercice}[Oral Mines 2018]
Montrer que la série $\sum B(n, n)$ converge et déterminer sa somme.
\end{exercice}

\begin{solution}
D'après la définition, $0 \leqslant B(n, n) \leqslant \frac{1}{4^{n-1}}$. Ainsi, $\sum B(n, n)$ converge.

Comme $t^p (1 - t)^q \geqslant 0$ pour $t \in \interff{0}{1}$, d'après le \theoremeutilise{théorème d'interversion série / intégrale}{theo:interversionserieintegrale},
\begin{align*}
\sum_{n=0}^{+\infty} B(n+1, n+1)
&= \int_0^1 \left(\sum_{n=0}^{+\infty} t^n (1 - t)^n\right) \d t \\
&= \int_0^1 \frac{\d t}{1 - t(1 - t)}\\
&= \frac{4}{3} \int_0^1 \frac{\d t}{\left[\frac{2}{\sqrt{3}}(t - 1/2)\right]^2 + 1} \\
&= \frac{2}{\sqrt{3}} \left[\arctan\left(\frac{2}{\sqrt{3}} (t-1/2)\right)\right]_0^1 \\
&= \frac{4}{\sqrt{3}} \arctan\frac{1}{\sqrt{3}} \\
&= \frac{2\pi}{3\sqrt{3}}
\end{align*}
\end{solution}

% --------------

\begin{exercice}[Oral CentraleSupélec 2016]
Soit $p \geqslant 2$ et $q \in \N$. On pose $S(p) = \sum\limits_{n=p}^{+\infty} \binom{n}{p}^{-1}$.
\begin{questions}
\item Montrer l'existence de $S(p)$.

\item Montrer que $S_N(p) = \sum\limits_{n=p}^{N+p} \binom{n}{p}^{-1} = \sum\limits_{n=0}^N (n + p + 1) B(n+1, p+1)$.

\item En déduire que $S(p) = \frac{p}{p-1}$.
\end{questions}
\end{exercice}

\begin{solution}
\begin{reponses}
\item D'après la définition des coefficients binomiaux,
\[
\binom{n}{p}^{-1} = \frac{p (p-1) \cdots 2 \cdot 1}{n (n-1) \cdots (n-p+2) (n-p+1)} = o\mathopen{}\left(\frac{1}{n^2}\right).
\]
Ainsi, d'après les \theoremeutilise{théorèmes de comparaison des séries à termes positifs}{theo:comparaisonseriestermespositifs}, $S(p)$ est bien définie.

\item D'après le théorème précédent,
\[
B(p+1, q+1) = \frac{p! q!}{(p + q + 1) (p + q)!} = \frac{\binom{p+q}{p}^{-1}}{p + q + 1}.
\]

Ainsi, si $p \geqslant 1$,
\begin{align*}
S_N(p)
&= \sum_{n=0}^{N} \binom{n+p}{p}^{-1}
= \sum_{n=0}^{N} (n+p+1) B(n+1, p+1) \\
&= \int_0^1 \sum_{n=0}^N (n+1) t^n (1 -t)^p \d t + \int_0^1 \sum_{n=0}^N p\, t^n (1 - t)^p \d t.
\end{align*}

De plus,
\[
\lim_{N\to+\infty} \sum_{n=0}^N (n + 1) t^n (1 - t)^p = (1 - t)^{p-2}
\quad \text{et} \quad 
\lim_{N\to+\infty} \sum_{n=0}^N t^n (1 - t)^p = (1 - t)^{p-1}.
\]
Comme ces suites sont croissantes et que les fonctions limites sont intégrables sur $\interff{0}{1}$, d'après le \theoremeutilise{théorème de convergence dominée}{theo:convergencedominee},
\begin{align*}
S(p)
&= \int_0^1 (1 -t)^{p-2} \d t + \int_0^1 p (1 - t)^{p-1} \d t 
= \frac{1}{p-1} + p \frac{1}{p}
= \frac{p}{p-1}.
\end{align*}

% \gras{Remarque.} On peut obtenir directement :
% \[
% \binom{n}{p}^{-1} = \binom{n-1}{p-1}^{-1} - \frac{n-p}{n-p+1} \binom{n}{p-1}^{-1}.
% \]
% Ainsi,
% \begin{align*}
% S(p) &= \sum_{n=0}^{+\infty} \binom{n+p}{n}^{-1} = 1 + \sum_{n=1}^{+\infty} \binom{n+p-1}{n-1}^{-1} - \frac{p}{p+1} \sum_{n=1}^{+\infty} \binom{n+p}{p-1}^{-1} \\
% &= 1 + S(p) - \frac{p}{p+1} S(p+1).
% \end{align*}
\end{reponses}
\end{solution}

%..............
\subsubsection{Bêta, Gamma et \nom{Wallis}}

\begin{theo}
Soit $(x, y) \in (\Rpe)^2$.
\begin{enumerate}
\item $B(x, y) = B(y, x)$.

\item $B(1, y) = \frac{1}{y}$.

\item $B(x + 1, y) = \frac{x}{y} B(x, y + 1)$ et $B(x + 1, y) = B(x, y) - B(x, y + 1)$ soit
\[
B(x, y) = \frac{x}{x + y} B(x, y).
\]

\item $B(x, y) = \displaystyle\int_0^{+\infty} \frac{t^{x - 1}}{(1 + t)^{x + y}} \d t$.
\end{enumerate}
\end{theo}

\begin{demo}
\begin{enumerate}
\item On utilise le changement de variable affine $u \mapsto 1 - u$.

\item Simple calcul de primitive.

\item Pour la première égalité, on effectue une intégration par parties avec les fonctions $t \mapsto t^x$ et $t \mapsto -\frac{(1 - t)^y}{y}$, de classe $\mathscr{C}^1$ sur $\interff{0}{1}$.

Pour la deuxième égalité, on décompose $t^x = (1 - t + t) t^{x - 1}$.

Enfin, on résout le système linéaire.

\item On utilise le changement de variable $\mathscr{C}^1$ et bijectif $t \mapsto \frac{t}{1 - t}$.
\end{enumerate}
\end{demo}

\begin{theo}
Pour tout $(x, y) \in (\Rpe)^2$,
\[
B(x, y) = \frac{\Gamma(x) \Gamma(y)}{\Gamma(x + y)}.
\]
\end{theo}

\begin{demo}
On pose $f : x \mapsto \frac{B(x, y) \Gamma(x + y)}{\Gamma(y)}$.
\begin{itemize}
\item
\begin{align*}
f(1)
&= \frac{B(1, y) \Gamma(y + 1)}{\Gamma(y)}
= \frac{\frac{1}{y} y \Gamma(y)}{\Gamma(y)}
= 1.
\end{align*}

\item Pour $x > 0$,
\begin{align*}
f(x + 1)
&= \frac{B(x + 1, y) \Gamma(x + 1 + y)}{\Gamma(y)}
= \frac{\frac{x}{x + y} B(x, y) (x + y) \Gamma(x + y)}{\Gamma(y)}
= x f(x).
\end{align*}

\item D'après la définition, pour tout $x > 0$,
\[
\ln(f(x)) = \ln(B(x, y)) + \ln(\Gamma(x + y)) - \ln(\Gamma(y)).
\]
D'après les parties précédentes, $x \mapsto \ln(\Gamma(x + y))$ est convexe.

On montre de manière analogue, en utilisant l'\theoremeutilise{inégalité de \nom{Cauchy}--\nom{Schwarz}}{theo:inegalitecs}, que $x \mapsto \ln(B(x, y))$ est convexe.

Finalement, $\ln(f)$ est convexe.
\end{itemize}
Finalement, d'après la caractérisation de la fonction $\Gamma$,
\[
\forall\, (x, y) \in (\Re)^2,\,
\frac{B(x, y) \Gamma(x + y)}{\Gamma(y)} = \Gamma(x).
\]
\end{demo}

\begin{remarque}
Pour prouver cette relation entre $B$ et $\Gamma$, une autre méthode utilisant la méthode des rectangles est présentée dans le sujet \textsl{Mathématiques 1, session 2009, filière \textsc{mp}} du concours Centrale-Supélec \cite{cs_1_2009}. \\
Le sujet propose également une démonstration de la relation 
\[
B(x, 1 - x) = \Gamma(x) \Gamma(1 - x) = \frac{\pi}{\sin(\pi x)}.
\]
\end{remarque}

% \todoinline{Citer proprement (si nécessaire car tout ici est très classique : soit 
% \url{https://claude-gimenes.fr/mathematiques/calcul-integral/-i-integrales-euleriennes}
% }

\begin{theo}
Pour tout $n \in \N$,
\[
\Wallis_n = \frac{1}{2} B\mathopen{}\left(\frac{n+1}{2}, \frac{1}{2} \right) = \frac{1}{2} \frac{\Gamma\mathopen{}\left(\frac{n+1}{2}\right) \sqrt{\pi}}{\Gamma\mathopen{}\left(\frac{n}{2}+1\right)}
\]
\end{theo}

\begin{elemdemo}
Après le changement de variable $\sin x = \sqrt{t}$ dans l'intégrale de \nom{Wallis} $\Wallis_n$, on obtient
\[
\Wallis_n = \frac{1}{2} \int_0^1 t^{\frac{n-1}{2}} (1-t)^{-\frac{1}{2}} \d t
\]
\end{elemdemo}


\section{Théorème de \textsc{Fubini}}

\url{https://www.bibmath.net/dico/index.php?action=affiche&quoi=./f/fubini.html} pour quelques éléments historiques sur les intégrales multiples. 

\marginnote[0pt]{Ce théorème a été démontré par le mathématicien italien Guido \textsc{Fubini} (1879--1943) en 1907.}
\begin{theo}[\textsc{Fubini}]
    Soit $\fonctionligne[f]{\interff{a}{b} \times \interff{c}{d}}{\K}$ une application continue. Alors,
    \[
    \int_{a}^{b} \mathopen{}\bigg( \int_{c}^{d} f(x,y) \d y \bigg) \d x = \int_{c}^{d} \mathopen{}\bigg( \int_{a}^{b} f(x,y) \d x \bigg) \d y.
    \]
\end{theo}

\begin{marginfigure}[0cm]
    \centering
    % define colors 
\colorlet{fillbottom}{yellow!60}
\colorlet{filltop}{blue!20}
\colorlet{curvecolor}{cyan}
\colorlet{meshcolor}{cyan}

\todoinline{Je commente la commande shorthandoff car elle ne compile pas chez moi}


 \begin{tikzpicture}[
 %3d view={135}{60},
 x={(-0.6cm,-0.35cm)},z={(0,0.2cm)},y={(1cm,-.25cm)}, 
 %  x={(-0.6cm,-0.5cm)},z={(0,0.2cm)},y={(1cm,-.3cm)}, 
 scale=1.5, line cap=round, line join=round
 ]
% \shorthandoff{:;!?} % https://groups.google.com/g/fr.comp.text.tex/c/K2CEGtgU3YQ
 %===================================
 %  surface 1: z=xy with domain boundary x=1, y=x, y=0
 % surface 2:  z=x^2+y^2 with domain boundary y=x, y=1, x=0
 %===================================
 \tikzset{
    declare function={
        f(\u,\v)=(-(\u/1)^2+(\v/0.8)^2)/3+6.5;
        }
    }

    \def\l{2}
 
    \def\w{1}
    \def\offset{1/4}
    \def\eps{1/30}
    \def\rad{0.05}

    \def\a{0.2}
    \def\b{1.9}
    \def\c{0.3}
    \def\d{2.1}


    \def\N{10}
    \pgfmathsetmacro{\step}{(\b-\a)/(\N+2)}
    \pgfmathsetmacro{\y}{\c+6*\step}
    \def\ythick{1*\step}

    \def\opacoupe{0.3}

% set coordinates 
     \def \mxmin{0}\def \xdash{0} \def\mxmax{2.5}
     \def \mymin{0}\def \ydash{0} \def\mymax{2}
     \def \mzmin{0}\def \zdash{0} \def\mzmax{2}

%%%%%%%%%%%%%%%%%%%%%%%%%%%%%%%%%%%%
%%%%%%%%%%%%%%%%%%%%%%%%%%%%%%%%%%%%
    \coordinate (O) at (\l*1/2,\l*1/2,0);
    \coordinate (X) at (\l*1/4,\l*1/4,0);
    \coordinate (Y) at ({\l*(1-1/4)},{\l*(1-1/4)},0);    
    
    \draw[line width=\w pt,black, dashed] (\a,\c,0) -- (\b,\c,0) -- (\b,\d,0) -- (\a,\d,0) -- cycle;

    \draw[arrows = {-latex[slant=-0.85]}] (0,0,0)--(\l*1.1,0,0) node[left]{$x$};
    \draw[arrows = {-latex[slant=0.75]}] (0,0,0)--(0,\l*1.2,0) node[right]{$y$};
    \draw[-latex] (0,0,0) -- (0,0,7) node[pos = 1.1] {$z$};
     
%%%%%%%%%%%%%%%%%%%%%%%%%%%%%%%%%%%%
%%%%%%%%%%%%%%%%%%%%%%%%%%%%%%%%%%%%

   \draw[dashed] (\a,-\eps,0)[above left] node {$a$} --(\a,\c,0);
   \draw[dashed] (\b,-\eps,0)[above left] node {$b$} --(\b,\c,0);
   \draw[dashed] (-\eps,\c,0)[above right] node {$c$}--(\a,\c,0);
   \draw[dashed] (-\eps,\d,0)[above right] node {$d$}--(\a,\d,0);

   \draw[dashed] (\y,-\eps,0)[above] node {$x$}--(\y,\c,0);


% help lines
   \draw[thick,dashed] (\a,\c,0)--(\a,\c,{f(\a,\c)});
   \draw[thick,dashed] (\b,\c,0)--(\b,\c,{f(\b,\c)});
   \draw[thick,dashed] (\a,\d,0)--(\a,\d,{f(\a,\d)});
% coupe

\draw[thick,draw=red,fill=red!50!white,opacity=\opacoupe] 
   (\y,\c,0) --
   plot[domain=0:{f(\y,\c)},samples=50,smooth] (\y,\c,\x)
   --
   plot[domain=\c:\d,samples=50,smooth] ({\y},{\x},{f(\y,\x)})
   --
   plot[domain={f(\y,\d)}:0,samples=50,smooth] (\y,\d,\x)
   --
   cycle;
\draw[thick,draw=red,fill=red!50!white,opacity=\opacoupe] 
   (\y,\c,0) -- (\y+\ythick,\c,0) -- (\y+\ythick,\d,0) -- (\y,\d,0) -- cycle;
\draw[thick,draw=red,fill=red!50!white,opacity=\opacoupe] 
   (\y,\c,0) -- (\y+\ythick,\c,0) -- (\y+\ythick,\c,{f(\y+\ythick,\c)}) -- (\y,\c,{f(\y,\c)}) -- cycle;
\draw[thick,draw=red,fill=red!50!white,opacity=\opacoupe] 
   (\y+\ythick,\c,0) --
   plot[domain=0:{f(\y+\ythick,\c)},samples=50,smooth] (\y+\ythick,\c,\x)
   --
   plot[domain=\c:\d,samples=50,smooth] ({\y+\ythick},{\x},{f(\y+\ythick,\x)})
   --
   plot[domain={f(\y+\ythick,\d)}:0,samples=50,smooth] (\y+\ythick,\d,\x)
   --
   cycle;
\draw[thick,draw=red,fill=red!50!white,opacity=\opacoupe] 
   (\y,\d,0) -- (\y+\ythick,\d,0) -- (\y+\ythick,\d,{f(\y+\ythick,\d)}) -- (\y,\d,{f(\y,\d)}) -- cycle;

\draw[thick,draw=red,fill=red!50!white,opacity=\opacoupe] 
   (\y,\c,{f(\y,\c)})--
   plot[domain=\c:\d,samples=50,smooth] ({\y},{\x},{f(\y,\x)})
   --
   plot[variable=\Y,domain=\y:\y+\ythick,samples=50,smooth] ({\Y},\d,{f(\Y,\d)}) 
   --
   plot[domain=\d:\c,samples=50,smooth] (\y+\ythick,\x,{f(\y+\ythick,\x)}) 
   --
   plot[variable=\Y,domain=\y+\ythick:\y,samples=50,smooth] (\Y,\c,{f(\Y,\c)}) 
   --
   cycle;

   
\draw[thick,dashed] (\b,\d,0)--(\b,\d,{f(\b,\d)});

%  surface 1
  \draw[thick,draw=curvecolor,fill=filltop,opacity=0.4] 
   (\a,\c,{f(\a,\c)})--
   plot[domain=\a:\b,samples=50,smooth] ({\x},{\c},{f(\x,\c)})
   --
   plot[variable=\y,domain=\c:\d,samples=50,smooth] (\b,{\y},{f(\b,\y)}) 
   --
   plot[domain=\b:\a,samples=50,smooth] (\x,\d,{f(\x,\d)}) 
   --
   plot[variable=\y,domain=\d:\c,samples=50,smooth] (\a,\y,{f(\a,\y)}) 
   --
   cycle;
   
% surface 1: mesh lines  
\foreach \k [evaluate=\k as \x using \a + \k * \step] in {-1,...,\N} {
    \draw[meshcolor] plot[variable=\y,domain=\c:\d,samples=50,smooth] (\a+\x+0.1,\y,{f(\a+\x+0.1,\y)});
}
\foreach \k [evaluate=\k as \i using \c + \k * \step] in {-2,...,\N} {
    \draw[meshcolor] plot[domain=\a:\b,samples=50,smooth] (\x,\c+\i+0.1,{f(\x,\c+\i+0.1)});
}

 %======================
 \end{tikzpicture}

    \caption{Découpage vertical}
\end{marginfigure}
\begin{marginfigure}[5cm]
% \begin{figure}
    \centering
    \begin{tikzpicture}[
  x=(215:2em/sqrt 2), y=(0:2em), z=(90:2em),
  declare function={f(\x,\y)=((\x-3)^2+(-\y+3)^3)/8+3;}, 
  very thick, line join=round]
\draw [-stealth, black!75] (0,0,0) -- (5,0,0) node [below left] {$x$};
\draw [-stealth, black!75] (0,0,0) -- (0,5,0) node [below right] {$y$};
\draw [-stealth, black!75] (0,0,0) -- (0,0,5) node [right] {$z$};
\foreach \x in {1,...,4}
  \foreach \y [evaluate={\j=\x+.5; \i=\y+.5; \k=f(\j,\i);}] in {1,...,4}{
    \path [fill=black!50, draw=white] (\x, \y+1, 0) -- (\x+1, \y+1, 0) -- 
      (\x+1, \y+1, \k) -- (\x, \y+1, \k) -- cycle;
    \path [fill=black!25, draw=white] (\x+1, \y, 0) -- (\x+1, \y+1, 0) -- 
      (\x+1, \y+1, \k) -- (\x+1, \y, \k) -- cycle;
    \path [fill=black!10, draw=white] (\x, \y, \k)  -- (\x+1, \y, \k) -- 
      (\x+1, \y+1, \k) -- (\x, \y+1, \k) -- cycle;
  }
 \foreach \x in {1,...,4}
   \foreach \y in {1,...,4}{
 \draw [black, fill=black, fill opacity=0.125, 
    domain=0:1, samples=10, variable=\t] 
    plot (\x+\t, \y, {f(\x+\t,\y)}) -- 
    plot (\x+1, \y+\t, {f(\x+1,\y+\t)}) -- 
    plot (\x+1-\t, \y+1, {f(\x+1-\t,\y+1)}) --
    plot (\x, \y+1-\t, {f(\x,\y+1-\t)}) -- cycle;
  }
\end{tikzpicture}
    \caption{Découpage horizontal}
% \end{figure}
\end{marginfigure}

\begin{exercice}
    Pour tout $(x, t) \in \interff{a}{b} \times \interff{c}{d}$ on pose 
    $$\varphi(x, t) \defeq \int_{a}^{x} f(u, t) \d u.$$
    \begin{questions}
    \item Montrer que pour tout $x \in \interff{a}{b}$, l'application $t \mapsto \varphi(x, t)$ est continue sur $\interff{c}{d}$.
    \end{questions}

    On pose alors, pour tout $x  \in \interff{a}{b}$,
    $$\psi(x) \defeq \int_{c}^{d} \varphi(x, t) \d t.$$
    \begin{questions}[resume]
        \item Montrer que $\psi$ est de classe $\mathscr{C}^1$ sur $\interff{a}{b}$ et donner une expression de $\psi'$.
        \item En déduire que pour tout $x \in \interff{a}{b}$,
        \[
        \int_{a}^{x} \mathopen{}\bigg( \int_{c}^{d} f(u,t) \d t \bigg) \d u = \int_{c}^{d} \mathopen{}\bigg( \int_{a}^{x} f(u,t) \d u \bigg) \d t.
        \]
    \end{questions}
\end{exercice}

\todoarmand{Je propose les deux figures ci-contre (qui sont à peaufiner). Est-ce qu'on garde cette version de tranches épaisses à la "Riemann" ou bien on fait plutôt une version comme dans \url{https://math.hawaii.edu/~kcorea/courses/spring_2023/244/static/15.1-scan.pdf} p.7}

\todoinline{J'aime bien les tranches épaisses. Par contre, je remplacerais dans la légende horizontal/vertical par selon l'axe des abscisses/ordonnées (ou l'inverse!)}


\marginnote[2cm]{Source : correction du sujet Mines Maths 2 PSI 2021 par Doc Solus.} 
\begin{solution}
\begin{reponses}
\item Les hypothèses de régularité du théorème de continuité des intégrales à paramètre sont immédiates.
    
Pour vérifier l'hypothèse de domination, on constate que la fonction $f$ est continue sur $\interff{a}{b} \times \interff{c}{d}$ qui est une partie fermée bornée de $\R^2$. Ainsi, d'après le théorème des bornes, la fonction $f$ est bornée sur $\interff{a}{b} \times \interff{c}{d}$ par une constante $M \in \Rp$. Les fonctions constantes sont intégrables car l'intégrale s'effectue ici sur un segment.
        
\item On applique le théorème de dérivation des intégrales à paramètre à la fonction $x \mapsto \int_{c}^{d} \varphi(x, t) \d t$:
        \begin{itemize}
            \item Pour tout $t \in \interff{c}{d}$, la fonction $x \mapsto \varphi(x, t)$ est de classe $\mathscr{C}^1$ sur $\interff{a}{b}$ car c'est la primitive s'annulant en $a$ de la fonction continue $x \mapsto f(x, t)$. 
            \item Sa dérivée partielle s'écrit $\frac{\partial \varphi}{\partial x}(x, t) = f(x, t)$.
            \item La domination se fait par la même constante $M$ que précédemment. 
            \end{itemize}
            Ainsi,
            \[
            \forall x \in \interff{a}{b} \quad \psi'(x) = \int_c^d \frac{\partial \varphi}{\partial x}(x, t) \d t = \int_{c}^{d} f(x, t) \d t.
            \]
        \item Soit $x \in \interff{a}{b}$. D'une part,
        $$\psi(x) = \int_{c}^{d} \mathopen{}\bigg( \int_{a}^{x} f(u,t) \d u \bigg) \d t.$$
        D'autre part, d'après la question précédente et le théorème fondamental de l'analyse, 
        \begin{align*}
            \int_{a}^{x} \mathopen{}\bigg( \int_{c}^{d} f(u,t) \d t \bigg) \d u &= \int_{a}^{x} \psi'(u) \d u  = \psi(x) - \psi(a) \\
            \text{Or, } \psi(a) &= \int_{c}^{d} \varphi(a, t) \d t \\
            \text{et } \forall t \in \interff{c}{d}, \quad \varphi(a, t) &= \int_{a}^{a} f(u, t) \d u = 0
        \end{align*}
        d'où $\psi(a) = 0$ et le résultat. \\
        En particulier, pour $x = b$ on obtient le résultat final.
    \end{reponses}
\end{solution}    

\todoarmand{Une application possible : le produit de convolution. J'ai pour l'instant simplement repris le théorème de l'un de mes cours de première année.}

\todoinline{Le pb est que la démo précédente de Fubini ne fonctionne que sur un segment. J'aime bien le produit de convolution, on peut le garder pour la transfo de Laplace.}

\begin{theo}
    Soient $u$ et $v$ deux fonctions de $L^1(\R^d)$. \textcolor{red}{Pour presque tout} $x \in \R^d$, on peut définir
    \begin{equation}\label{defconvolution}
    (u \ast v)(x) = \int_{\R^d} u(x-y) v(y) \d y = \int_{\R^d} u(y) v(x-y) \d y.
    \end{equation}
    La fonction $u \ast v$ ainsi définie est appelée le \emph{produit de convolution} de $u$ et $v$. Elle appartient à $L^1(\R^d)$ et 
    \[
    \norm{u \ast v}_{L^1} \leqslant \norm{u}_{L^1} \norm{v}_{L^1}.
    \]
\end{theo}
\begin{demo}
    L'existence des intégrales \ref{defconvolution} n'est pas évidente : à $x$ fixé, il s'agit d'intégrer le produit de deux fonctions intégrables. Mais un tel produit n'est pas intégrable en général. Dans ce cas particulier, nous allons le déduire tu théorème de \textsc{Fubini}. On a
    \[
    \int_{\R^d \times \R^d} \module{u(x-y) v(y)} \d x \d y = \int_{\R^d} \module{v(y)} \left( \int_{\R^d} \module{u(x - y)} \d x \right) \d y.
    \]
    En effectuant le changement de variable $z = x - y$ (à $y$ fixé), il vient
    \[
    \int_{\R^d} \module{u(x-y)} \d x = \int_{\R^d} \module{u(z)} \d z = \norm{u}_{L^1}.
    \]
    Ainsi l'égalité précédente s'écrit
    \[
    \int_{\R^d \times \R^d} \module{u(x-y) v(y)} \d x \d y = \norm{u}_{L^1} \int_{\R^d} \module{v(y)} \d y = \norm{u}_{L^1} \norm{v}_{L^1}.
    \]
    Il s'ensuit que la fonction $(x,y) \mapsto u(x-y) v(y)$ appartient à $L^1(\R^d \times \R^d)$. Le théorème de \textsc{Fubini} nous permet donc de dire que pour presque tout $x$, la fonction $y \mapsto u(x-y) v(y)$ appartient à $L^1(\R^d)$, ce qui donne un sens à la première intégrale dans \ref{defconvolution}, la seconde s'en déduisant par le changement de variable $z = x - y$ (à $x$ fixé). Il reste à constater que 
    \[
    \norm{u \ast v}_{L^1} = \int_{\R^d} \module{\int_{\R^d} u(x-y)v(y) \d y} \d x \leqslant \int_{\R^d \times \R^d} \module{u(x-y) v(y)} \d x \d y,
    \]
    où l'on vient de voir que la dernière intégrale n'est autre que $\norm{u}_{L^1} \norm{v}_{L^1}$.
\end{demo}

\todoarmand{On pourrait ensuite proposer quelques exercices sur le calcul du produit de convolution de fonctions classiques. C'est aussi l'occasion de faire de jolies illustrations.}

\section{Transformée de \textsc{Laplace}} 
\label{transformee_laplace}

\todoinline{On pourrait tenter une illustration ? La transformée de Laplace d'une fonction échelon ?}
\todoarmand{Un thème assez développé sur la transformée de Laplace : \url{https://cahier-de-prepa.fr/ecg2-saintlouis/download?id=289}
}

La transformation de \textsc{Laplace} généralise la transformation de \textsc{Fourier} qui est également utilisée pour résoudre les équations différentielles : contrairement à cette dernière, elle tient compte des conditions initiales et peut ainsi être utilisée en théorie des vibrations mécaniques ou en électricité dans l'étude des régimes forcés sans négliger le régime transitoire. De manière générale, ses propriétés vis-à-vis de la dérivation permettent un traitement plus simple de certaines équations différentielles, et elle est de ce fait très utilisée en automatique.

Dans ce type d'analyse, la transformation de \textsc{Laplace} est souvent interprétée comme un passage du domaine temps, dans lequel les entrées et sorties sont des fonctions du temps, dans le domaine des fréquences, dans lequel les mêmes entrées et sorties sont des fonctions de la \say{ fréquence } (complexe) $p$. Ainsi, il est possible d'analyser simplement l'effet du système sur l'entrée pour donner la sortie en matière d'opérations algébriques simples (cf. théorie des fonctions de transfert en électronique ou en mécanique). 

\begin{defi}[Fonction causale]
\begin{comment}
\begin{itemize}
    \item On appelle \definir{fonction causale} une fonction définie sur $\R$ dont le support est borné à gauche en $0$ \emph{i.e.} $f$ est nulle pour tout $x < 0$. 
    \item On appelle fonction causale toute fonction définie sur $\R$, nulle sur $\interoo{-\infty}{0}$ et continue par morceaux sur $\interfo{0}{+\infty}$.
\end{itemize}
\end{comment}
Une fonction $f$ définie sur $\R$ est une \definir{fonction causale} si elle est continue par morceaux et nulle sur $\interfo{0}{+\infty}$.
\end{defi}


\begin{remarque}
La fonction de \textsc{Heaviside} est définie par $H = \indicatrice{\Rp}$. Étant donnée une fonction $g$ non causale, la fonction $f = H \times g$ est une fonction causale.
\end{remarque}

\begin{defi}[Transformée de \textsc{Laplace}]
    Soit $f$ une fonction causale. On note, lorsqu'elle converge, 
    $$\mathscr{L}(f)(p) \defeq \int_{0}^{+ \infty} \e^{-pt} f(t) \d t.$$
    La fonction $\mathscr{L}(f)$ est la \emph{transformée de \textsc{Laplace} de f}.
\end{defi}

\marginnote[0cm]{Sources : \cite{exos_oraux} + \cite{acamanes} (Exercice cerise Ch. 12)}
% \underline{Démonstration du théorème de la valeur finale:}
% \begin{itemize}
    % \item Généralisation classique du théorème des bornes $\leadsto$ $f$ est bornée
    % \item Changement de variable: $\varphi: u \mapsto \frac{u}{p}$
    % \item Caractérisation séquentielle de la limite
    % \item Théorème de convergence dominée
% \end{itemize}

\todoinline{Pointer vers CCP - MP - 2011}

%-----------
\subsection{Exemples de transformées de \textsc{Laplace}}

\begin{exercice}
Soient $\lambda \in \C$ et $n \in \N$. Pour chacune des fonctions suivantes, déterminer leur transformée de~\textsc{Laplace} en précisant le domaine de définition :
\begin{tasks}(3)
    \task $t \mapsto 1$.
    \task $t \mapsto \e^{\lambda t}$.
    \task $t \mapsto t^n$.
\end{tasks}
\end{exercice}

\begin{solution}
\begin{reponses}
\item La fonction $t \mapsto \e^{-pt}$ est intégrable sur $\Rp$ si et seulement si $p > 0$. Alors,
\[
\forall\, p > 0,\,
\mathscr{L}(f)(p) = \int_0^{+\infty} \e^{-pt} \d t = \frac{1}{p}.
\]

\item La fonction $t \mapsto \e^{-(\lambda-p)t}$ est intégrable sur $\Rp$ si et seulement si $p > \Reel(\lambda)$. Alors,
\[
\mathscr{L}(f)(p)
= \int_0^{+\infty} \e^{-(p-\lambda)t} \d t
= \frac{1}{p-\lambda}.
\]

\item Soit $n \in \Ne$ et $\fonctionligne[f_n]{t}{t^n \e^{-pt}}$.
\begin{itemize}
\item La fonction $f_n$ est continue sur $\Rpe$.
\item Si $p > 0$, alors $f(t) = o_{+\infty}(1/t^2)$ et $f$ est intégrable sur $\Rpe$.

Si $p \leqslant 0$, alors $\lim_{+\infty} f = +\infty$ et $f$ n'est pas intégrable sur $\Rpe$.
\end{itemize}
Ainsi, $f_n$ est intégrable si et seulement si $p > 0$. Une récurrence classique avec des intégrations par parties permet de montrer que
\[
\mathscr{L}(f)(p) = \frac{n!}{p^{n+1}}.
\]
\end{reponses}
\end{solution}

\todoarmand{Résoudre problème avec adjustbox{valign=c}}
\todoarmand{Ajouter les derniers graphes}
\begin{table}[h!]
    \centering
    % \renewcommand{\arraystretch}{3.0}
    % \setlength{\tabcolsep}{0.5em}
    \SetTblrInner{rowsep=3pt}
    \def\colplot{cyan}
    \def\a{-0.3}
    \def\b{3.8}
    \def\xmin{-0.1}
    \def\xmax{4.1}
    \def\amp{1}
    \def\puls{8}
    \def\damp{0.6}
    \def\samp{200}
    \begin{tblr}{
        colspec = {|cc|cc|}
    }
        \hline
        $f(t) \mathbf{1}_{t > 0}$ & $f(t)$ & $\mathcal{L}(f)(p)$ & \\ \hline\hline
        % Row 2
        % \adjustbox{valign=c}{%
            \begin{tikzpicture}[scale=0.6]
                \draw[-latex] (\xmin,0) -- (\xmax,0);
                \draw[-latex] (0,-0.2) -- (0,1.5);
                \draw[\colplot, thick, domain=\a:0] plot (\x, 0);
                \draw[\colplot, thick] (0,0) -- (0,1) -- (\b,1);
            \end{tikzpicture}%
        % } 
        & $ A $ & $\displaystyle \frac{A}{p} $ & 
        % \adjustbox{valign=c}{%
            \begin{tikzpicture}[scale=0.6]
                \draw[-latex] (\xmin,0) -- (\xmax,0);
                \draw[-latex] (0,-0.2) -- (0,1.5);
                \draw[\colplot, thick, domain=1/6:\b, samples=\samp] plot (\x, {1/(6*\x)});
            \end{tikzpicture}%
        % }
        \\ \hline

        % Row 3
        % \adjustbox{valign=c}{%
            \begin{tikzpicture}[scale=0.6]
                \draw[-latex] (\xmin,0) -- (\xmax,0);
                \draw[-latex] (0,-0.2) -- (0,1.5);
                \draw[\colplot, thick, domain=\a:0] plot (\x, 0);
                \draw[\colplot, thick] (0,0) -- (\b,1);
            \end{tikzpicture}%
        % } 
        & $ a\,t$ % \cdot u(t) $ 
        & $\displaystyle \frac{a}{p^2} $ & 
        % \adjustbox{valign=c}{%
            \begin{tikzpicture}[scale=0.6]
                \draw[-latex] (\xmin,0) -- (\xmax,0);
                \draw[-latex] (0,-0.2) -- (0,1.5);
                \draw[\colplot, thick, domain=0.40824829046:\b, samples=\samp] plot (\x, {1/(6*\x^2)});
            \end{tikzpicture}%
        % } 
        \\ \hline

        % Row 4
        % \adjustbox{valign=c}{%
            \begin{tikzpicture}[scale=0.6]
                \draw[-latex] (\xmin,0) -- (\xmax,0);
                \draw[-latex] (0,-0.2) -- (0,1.5);
                \draw[\colplot, thick, domain=\a:0] plot (\x, 0);
                \draw[\colplot, thick] (0,0) -- (0,\amp);
                \draw[\colplot, thick, domain=0:\b] plot (\x, {\amp*exp(-\damp*\x)});
            \end{tikzpicture}%
        % }
        & $ \mathrm{e}^{-at}$ % $ \cdot u(t) $ 
        & $\displaystyle \frac{1}{p + a} $ & 
        % \adjustbox{valign=c}{%
            \begin{tikzpicture}[scale=0.6]
                \draw[-latex] (\xmin,0) -- (\xmax,0);
                \draw[-latex] (2,-0.2) -- (2,1.5);
                \draw[dashed] (1/6,-0.2) -- (1/6,1.5);
                \node at (1/6, 0) {\footnotesize $a$};
                \draw[\colplot, thick, domain=1/3:\b, samples=\samp] plot (\x, {1/(6*(\x-1/6))});
            \end{tikzpicture}%
        % } 
        \\ \hline

        % Row 5
        % \adjustbox{valign=c}{%
            \begin{tikzpicture}[scale=0.6]
                \draw[-latex] (\xmin,0) -- (\xmax,0);
                \draw[-latex] (0,-0.2) -- (0,1.5);
                \draw[\colplot, thick, domain=\a:0] plot (\x, 0);
                \draw[\colplot, thick, domain=0:\b, samples=\samp] plot (\x, {5*\x*exp(-1.5*\x)});
            \end{tikzpicture}%
        % }
        & $ t\, \mathrm{e}^{-at}$ % $ \cdot u(t) $ 
        & $\displaystyle \frac{1}{(p + a)^2} $ & 
        % \adjustbox{valign=c}{%
            \begin{tikzpicture}[scale=0.6]
                \draw[-latex] (\xmin,0) -- (\xmax,0);
                \draw[-latex] (0,-0.2) -- (0,1.5);
            \end{tikzpicture}%
        % } 
        \\ \hline

        % Row 6
        % \adjustbox{valign=c}{%
            \begin{tikzpicture}[scale=0.6]
                \draw[-latex] (\xmin,0) -- (\xmax,0);
                \draw[-latex] (0,-0.65) -- (0,1.05);
                \draw[\colplot, thick, domain=\a:0] plot (\x, 0);
                \draw[\colplot, thick, domain=0:\b, samples=\samp] plot (\x, {\amp/1.5*sin(\puls*\x r)});
            \end{tikzpicture}%
        % } 
        & $ \sin(\omega t)$ % $ \cdot u(t) $ 
        & $\displaystyle \frac{\omega}{p^2 + \omega^2} $ & 
        % \adjustbox{valign=c}{%
            \begin{tikzpicture}[scale=0.6]
                \draw[-latex] (\xmin,0) -- (\xmax,0);
                \draw[-latex] (0,-0.65) -- (0,1.05);
            \end{tikzpicture}%
        % }
        \\ \hline

        % Row 7
        % \adjustbox{valign=c, width=0.22\textwidth}{%
            \begin{tikzpicture}[scale=0.6]
                \draw[-latex] (\xmin,0) -- (\xmax,0);
                \draw[-latex] (0,-0.65) -- (0,1.05);
                \draw[\colplot, thick, domain=\a:0] plot (\x, 0);
                \draw[\colplot, thick] (0,0) -- (0,\amp/1.5);
                \draw[\colplot, thick, domain=0:\b, samples=\samp] plot (\x, {\amp/1.5*cos(\puls*\x r)});
            \end{tikzpicture}%
        % } 
        & $ \cos(\omega t)$ % $ \cdot u(t) $ 
        & $\displaystyle \frac{p}{p^2 + \omega^2} $ & 
        % \adjustbox{valign=c}{%
            \begin{tikzpicture}[scale=0.6]
                \draw[-latex] (\xmin,0) -- (\xmax,0);
                \draw[-latex] (0,-0.65) -- (0,1.05);
            \end{tikzpicture}%
        % }
        \\ \hline

        % Row 8
        % \adjustbox{valign=c}{%
            \begin{tikzpicture}[scale=0.6]
                \draw[-latex] (\xmin,0) -- (\xmax,0);
                \draw[-latex] (0,-0.65) -- (0,1.05);
                \draw[\colplot, thick, domain=\a:0] plot (\x, 0);
                \draw[\colplot, thick, domain=0:\b, samples=\samp] plot (\x, {\amp/1.5*exp(-\damp*\x)*sin(\puls*\x r)});
            \end{tikzpicture}%
        % }
        & $ \mathrm{e}^{-at} \sin(\omega t)$ % $ \cdot u(t) $ 
        & $\displaystyle \frac{\omega}{(p + a)^2 + \omega^2} $ & 
        % \adjustbox{valign=c}{%
            \begin{tikzpicture}[scale=0.6]
                \draw[-latex] (\xmin,0) -- (\xmax,0);
                \draw[-latex] (0,-0.65) -- (0,1.05);
            \end{tikzpicture}%
        % }
        \\ \hline

        % Row 9
        % \adjustbox{valign=c}{%
            \begin{tikzpicture}[scale=0.6]
                \draw[-latex] (\xmin,0) -- (\xmax,0);
                \draw[-latex] (0,-0.65) -- (0,1.05);
                \draw[\colplot, thick, domain=\a:0] plot (\x, 0);
                \draw[\colplot, thick] (0,0) -- (0,\amp/1.5);
                \draw[\colplot, thick, domain=0:\b, samples=\samp] plot (\x, {\amp/1.5*exp(-\damp*\x)*cos(\puls*\x r)});
            \end{tikzpicture}%
        % }
        & $ \mathrm{e}^{-at} \cos(\omega t)$ % $ \cdot u(t) $ 
        & $\displaystyle \frac{p + a}{(p + a)^2 + \omega^2} $ & 
        % \adjustbox{valign=c}{%
            \begin{tikzpicture}[scale=0.6]
                \draw[-latex] (\xmin,0) -- (\xmax,0);
                \draw[-latex] (0,-0.65) -- (0,1.05);
            \end{tikzpicture}%
        % }
        \\ \hline
    \end{tblr}
    \caption{Transformées de \textsc{Laplace} usuelles}
    \end{table}

\begin{remarque}
À un changement de variable près, on reconnaît, pour la dernière transformée de Laplace, la fonction Gamma d'\textsc{Euler}.\marginnote[0pt]{\href{subsec:FonctionGammaEuler}{\faLink~Fonction Gamma d'\textsc{Euler}}}
\end{remarque}


\begin{theo}
\begin{enumerate}
\item Si la fonction $f$ est bornée, alors $\mathscr{L}(f)$ est définie et de classe $\mathscr{C}^\infty$ sur $\Rpe$.

\item Si $\int_0^{+\infty} f(t) \d t$ converge, alors $\mathscr{L}(f)$ est définie sur $\Rp$.

\item Si $f$ est continue uniquement sur $\Rpe$ et qu'il existe $p_0 > 0$ tel que, pour pour tout $p > p_0$, la fonction $t \mapsto \e^{-p t} f(t)$ est intégrable sur $\Rp$, alors $\mathscr{L}(f)$ est définie et continue sur $\interoo{p_0}{+\infty}$.
\end{enumerate}
\end{theo}

\begin{demo}
\begin{enumerate}
\item \marginnote[-7pt]{\theoremeutilise{Théorème de dérivation sous le signe intégral}}La fonction $f$ est bornée par une constante $M$. On utilise le théorème de dérivation sous le signe intégral. Notons $\fonctionligne[g]{(p, t)}{f(t) \e^{-pt}}$.
\begin{itemize}
\item Pour tout $t \in \Rpe$, la fonction $\fonctionligne[g(\,\cdot\,, t)]{p}{\e^{-p t} f(t)}$ est de classe $\mathscr{C}^\infty$ sur $\Rpe$ et
\[
\forall j \in \N,\quad \frac{\partial^j g}{\partial p^j} g(p, t) = (-1)^j t^j f(t) \e^{-pt}.
\]

\item Pour tout $p > 0$, la fonction $\fonctionligne[g(p,\,\cdot\,)]{t}{(-1)^j t^j \e^{-p t} f(t)}$ est continue sur $\interfo{0}{+\infty}$.

\item Soit $\tilde{p} > 0$. Alors, pour tout $p \geqslant \tilde{p}$,
\[
\abs{(-1)^j t^j f(t) \e^{-p t}} \leqslant M t^j \e^{- \tilde{p} t} M.
\]
De plus, la fonction $\fonctionligne[\phi_j]{t}{M t^j \e^{-\tilde{p} t}}$ est continue sur $\Rp$, et est négligeable devant $1/t^2$ en $+\infty$. Ainsi, $\phi_j$ est intégrable.
\end{itemize}
Ainsi, la transformée de \textsc{Laplace} $\mathscr{L}(f)$ est de classe $\mathscr{C}^\infty$ sur $\Rpe$.

\item Lorsque $p = 0$, l'intégrale $\int_0^{+\infty} \e^{-0\cdot t} f(t) \d t$ converge par hypothèse et $\mathscr{L}(f)(0)$ est donc bien définie.

\medskip

Soit $p > 0$. Comme $f$ est continue, on note $\fonctionligne[F]{x}{\int_0^x f(t) \d t}$ sa primitive qui s'annule en $0$.

Par hypothèse, $F$ possède une limite finie en $+\infty$. Ainsi, d'après le théorème d'intégration par parties, les intégrales $\int_0^{+\infty} \e^{-p t} f(t) \d t$ et $\int_0^{+\infty} \e^{-p t} F(t) \d t$ sont de même nature.

Enfin, comme $F$ est continue sur $\Rp$ et admet une limite finie en $+\infty$, alors $F$ est bornée sur $\Rp$. Il existe donc une constante $M$ telle que
\[
\module{\e^{-p t} F(t)} \leqslant M \e^{-p t}.
\]
Ainsi, la fonction $t \mapsto \e^{-p t} F(t)$ est intégrable, donc $\int_0^{+\infty} \e^{-p t} F(t) \d t$ converge et $\mathscr{L}(f)(p)$ est bien définie.

\item Soit $\fonctionligne[g]{(p, u)}{f(u) \e^{-p u}}$.
\begin{itemize}
\item $\forall t \in \Rpe$, $g(\,\cdot\,, u)$ est continue sur $\interoo{p_0}{+\infty}$.
\item $\forall p > p_0$, $g(p, \cdot)$ est continue sur $\Rpe$.
\item Soit $\tilde{p} > p_0$. Pour tout $p \geqslant \tilde{p}$ et $t \in \Rpe$,
\[
\abs{\e^{-p t} f(t)} \leqslant \e^{- \tilde{p} t} \abs{f(t)}.
\]
De plus, la fonction $t \mapsto f(t) \e^{-\tilde{p} t}$ est intégrable sur $\Rpe$ par hypothèse.
\end{itemize}
\marginnote[-7pt]{\theoremeutilise{Théorème de continuité sous le signe intégral}}D'après le théorème de continuité sous le signe intégral, la transformée $\mathscr{L}(f)$ est continue sur $\interoo{p_0}{+\infty}$.
\end{enumerate}
\end{demo}


%-----------
\subsection{Théorème de la valeur finale}

\begin{theo}[Théorème de la valeur finale]
On suppose qu'il existe un réel $\ell$ non nul tel que $\lim\limits_{x \to +\infty} f(x) = \ell$. Alors,
\[
\lim_{p\to 0} p\, \mathscr{L}(f)(p) = \ell.
\]
\end{theo}

\begin{exercice}
On se place sous les hypothèses du théorème.
\begin{questions}
\item Montrer que $\mathscr{L}(f)$ est définie sur $\Rpe$.

\item Montrer que $p \mathscr{L}(f)(p) = \int_0^{+\infty} \e^{-u} f\big(\frac{u}{p}\big) \d u$.

\item Montrer que, si $(p_n)$ est une suite de réels strictement positifs qui tend vers $0$, alors $\lim\limits_{n\to+\infty} p_n\, \mathscr{L}(f)(p_n) = \ell$.

\item Conclure.
\end{questions}
\end{exercice}

\begin{solution}
\begin{reponses}
\item Comme $f$ est continue sur $\Rp$ et possède une limite en $+\infty$, une généralisation classique du théorème des bornes assure que $f$ est bornée sur $\Rp$.\marginnote[0pt]{\theoremeutilise{Théorème des bornes}}Ainsi, d'après le théorème précédent, la transformée $\mathscr{L}(f)$ est bien définie sur $\Rpe$.

\item On effectue le changement de variable affine $\fonctionligne[\phi]{u}{\frac{u}{p}}$. Alors,
\[
p\, \mathscr{L}(f)(p) = \int_0^{+\infty} \e^{-u} f\mathopen{}\left(\frac{u}{p}\right) \d u.
\]

\item Posons $\fonctionligne[g_n]{u}{\e^{-u} f(u/p_n)}$.
\begin{itemize}
\item Pour tout $n \in \N$, la fonction $g_n$ est continue sur $\Rp$.

\item Soit $u > 0$. D'après les hypothèses sur la fonction $f$, $\lim\limits_{n\to+\infty} \e^{-u} f(u/p_n) = \e^{-u} \ell$. Ainsi, la suite de fonctions $(g_n)$ converge simplement sur $\Rpe$ vers $u \mapsto \e^{-u} \ell$.

\item Comme $f$ est bornée par une constante $M$, $\abs{g_n(u)} \leqslant M \e^{-u}$ et la fonction $u \mapsto \e^{-u}$ est intégrable sur $\Rpe$.
\end{itemize}
\marginnote[-7pt]{\theoremeutilise{Théorème de convergence dominée}}Ainsi, d'après le théorème de convergence dominée,
\[
\lim_{n\to+\infty} p_n\, \mathscr{L}(f)(p_n) = \ell \int_0^{+\infty} \e^{-u} \d u = \ell.
\]

\item \marginnote[-7pt]{\theoremeutilise{Caractérisation séquentielle de la limite}}En utilisant la caractérisation séquentielle de la limite,
\[
\lim_{p\to 0} p\, \mathscr{L}(f)(p) = \ell \quad \text{soit} \quad \mathscr{L}(f)(p) \sim_0 \frac{\ell}{p}.
\]
\end{reponses}
\end{solution}

%-----------
\subsection{Théorème de la valeur initiale}

\begin{theo}[Théorème de la valeur initiale]
On suppose que $f$ est continue sur $\Rpe$ et qu'il existe $p_0 > 0$ tel que, pour pour tout $p > p_0$, $t \mapsto \e^{-p t} f(t)$ est intégrable sur $\Rp$. On suppose de plus qu'il existe un réel $\ell$ tel que $\lim_{t\to0^+} f(t) = \ell$. Alors,
\[
\lim_{p\to+\infty} p\, \mathscr{L}(f)(p) = \ell.
\]
\end{theo}

\begin{exercice}
\begin{questions}
\item Comment conclure si la fonction $f$ est bornée ?
\end{questions}

On suppose dans la suite que $f$ n'est pas nécessairement bornée. Soit $\varepsilon > 0$.
\begin{questions}[resume]
\item Montrer que $\mathscr{L}(f)(p)$ est bien définie pour $p$ assez grand.

\item Montrer qu'il existe un réel $h$ tel que
\[
\int_0^h p \module{f(t) - \ell} \e^{-p t} \d t \leqslant \varepsilon.
\]

\item Montrer qu'il existe un réel $\tilde{p}$ et une constante $c$ tels que pour tout $p \geqslant \tilde{p}$,
\[
p \int_h^{+\infty} \module{f(t)} \e^{-p t} \d t
\leqslant c p\,\e^{-(p-\tilde{p}) h}.
\]

\item Conclure.
\end{questions}
\end{exercice}

\begin{solution}
\begin{reponses}
\item  Si la fonction $f$ est bornée, on peut appliquer la même méthode que pour le théorème de la valeur finale.

\item Comme $\lim_{t\to0^+} f(t) = \ell$, il existe $h > 0$ tel que
\[
\forall t \in \interof{0}{h},\quad \abs{f(t) - \ell} \leqslant \varepsilon.
\]
Ainsi,
\begin{align*}
\int_0^h p \abs{f(t) - \ell} \e^{-p t} \d t
&\leqslant \varepsilon\big(1 - \underbrace{\e^{-p h}}_{\geqslant 0}\big)
\leqslant \varepsilon
.
\end{align*}

\item Posons $\tilde{p} = p_0 + \frac{1}{2}$. Alors, pour $p \geqslant \tilde{p}$,
\begin{align*}
p \int_h^{+\infty} \module{f(t)} \e^{-p t} \d t
&= p \int_h^{+\infty} \module{f(t)} \e^{-(p - \tilde{p}) t - \tilde{p} t} \d t\\
&\leqslant p\, \e^{-(p-\tilde{p}) h} \int_h^{+\infty} \module{f(t)} \e^{- \tilde{p} t} \d t\\
&\leqslant c p\, \e^{-(p-\tilde{p}) h}.
\end{align*}

\item Finalement, en utilisant l'inégalité triangulaire et la relation de \textsc{Chasles}, pour $p \geqslant \tilde{p}$,
\begin{align*}
\abs{p\, \mathscr{L}(f)(p) - \ell}
&\leqslant \int_0^{+\infty} p \abs{f(t) - \ell} \e^{-p t} \d t \\
&\leqslant \int_0^h p \abs{f(t) - \ell} \e^{-pt} \d t
+ p \int_h^{+\infty} \abs{f(t)} \e^{-p t} \d t
+ p \int_h^{+\infty} \abs{\ell} \e^{- p t} \d t \\
&\leqslant \varepsilon + c p\, \e^{-(p-\tilde{p})h} + \module{\ell} \e^{-p h}.
\end{align*}

Comme $\lim\limits_{p\to+\infty} \e^{-p h} = \lim\limits_{p\to+\infty} p\, \e^{-(p-\tilde{p})h} = 0$, pour $p$ assez grand,
\[
\abs{p\, \mathscr{L}(f)(p) - \ell}
\leqslant 3 \varepsilon.
\]

Ainsi, d'après la définition de la limite,
\[
\lim_{p\,\to+\infty} p\, \mathscr{L}(f)(p) = \ell.
\]
\end{reponses}
\end{solution}


%--------------
\subsubsection{Méthode de Laplace}
\todoinline{Ajouter méthode de Laplace ? cf . Mines 2 - PC - 2017 - Un cas discret est traité, ici on traite un cas continu}

\todoinline{Phrase introductive - Voir Rouvière - Petit guide de calcul différentiel pour un théorème plus complet}

\begin{theo}{Méthode de Laplace}
Soient $f$ une fonction de classe $\mathscr{C}^2$ et intégrable sur $\Rp$. Alors,
\[
F(t) = \int_0^{+\infty} \e^{-t x^2} f(t) \d x
\sim_{t\to+\infty} \frac{\sqrt{\pi}}{2} \cdot \frac{f(0)}{\sqrt{t}}.
\]
\end{theo}

\begin{remarque}
Par rapport au théorème de la valeur initiale, la fonction $\e^{-t x}$ a été remplacée par $\e^{-t x^2}$. L'heuristique de la démonstration est identique, à savoir que dès que $t$ devient grand, seules les valeurs de $f$ proches de $0$ comptent. Une généralisation de ce résultat permet de retrouver la formule de \textsc{Stirling}. 
\end{remarque}

\begin{exercice}
\begin{questions}
\item Montrer que $F$ est bien définie sur $\Rp$.

\item À l'aide d'un changement de variable et du théorème de convergence dominée, montrer que
\[
\lim_{t\to+\infty} \sqrt{t} \int_0^1 \e^{-t x^2} f(x) \d x = f(0) \int_0^{+\infty} \e^{-u^2} \d u.
\]

\item Montrer que
\[
\lim_{t\to+\infty} \sqrt{t} \int_1^{+\infty} \e^{-t x^2} f(x) \d x = 0.
\]

\item Conclure.
\end{questions}
\end{exercice}

\begin{demo}
\begin{reponses}
\item Comme $\module{\e^{-t x^2} f(t)} \leq \module{f(t)}$, la fonction $F$ est bien définie.

\item Comme la fonction $f$ est continue sur $\interff{0}{1}$, elle est bornée par une constante $M$ sur ce segment. Alors, en utilisant le changement de variable affine $u \mapsto u/\sqrt{t}$,
\begin{align*}
\int_0^1 \e^{-t x^2} f(x) \d x
&= \int_0^{\sqrt{t}} \frac{\e^{-u^2}}{\sqrt{t}} f\left(\frac{u}{\sqrt{t}}\right) \d u\\
&= \frac{1}{\sqrt{t}} \int_0^{+\infty} \e^{-u^2} f\left(\frac{u}{\sqrt{t}}\right) \indicatrice{\interff{0}{\sqrt{t}}}(u) \d u.
\end{align*}
En utilisant une majoration par $u \mapsto M \e^{-u^2}$, le théorème de convergence dominée permet de conclure que
\[
\lim_{t\to+\infty} \sqrt{t} \int_0^1 \e^{-t x^2} f(x) \d x = f(0) \int_0^{+\infty} \e^{-u^2} \d u.
\]

\item De plus,
\begin{align*}
\module{\int_1^{+\infty} \e^{-t x^2} f(x) \d x}
&\leq \e^{-t} \int_1^{+\infty} \module{f(x)} \d x.
\end{align*}
Ainsi, comme $f$ est intégrable,
\[
\lim_{t\to+\infty} \sqrt{t} \int_1^{+\infty} \e^{-t x^2} f(x) \d x = 0.
\]

\item Finalement, en utilisant la relation de Chasles,
\[
\lim_{t\to+\infty} \sqrt{t} \int_0^{+\infty} \e^{-t x^2} f(x) \d x = f(0) \int_0^{+\infty} \e^{-u^2} \d u.
\]
On conclut en utilisant l'intégrale de Gauss.
\end{reponses}
\end{demo}



\section{Intégrale de \textsc{Frullani}}

\todoinline{Pourrait-on ajouter une illustration ici ? Faudrait voir ce que cela fait de tracer ces fonctions...}

\todoarmand{
Exercices 51, 52, 53 de \url{http://vonbuhren.free.fr/Prepa/Colles/integration_intervalle_quelconque.pdf}}

\begin{theo}
Soit $f : ]0, +\infty[ \to \R$ une fonction continue et $L$ et $\ell$ deux réels tels que
\[
\lim_{x\to+\infty} f(x) = L
\text{ et }
\lim_{x\to0} f(x) = \ell.
\]
Alors, $I = \int_0^{+\infty} \frac{f(a t) - f(b t)}{t} \d t$ converge et
\[
\int_0^{+\infty} \frac{f(a t) - f(b t)}{t} \d t = (\ell - L) \ln\frac{b}{a}.
\]
\end{theo}

\begin{exercice}
Soit $f$ une fonction satisfaisant les hypothèses du théorème et $0 < \epsilon \leq M$.
\begin{enumerate}
\item Montrer que
\[
\int_\epsilon^M \frac{f(a t) - f(b t)}{t} \d t
= \int_{a\epsilon}^{b\epsilon} \frac{f(t)}{t} \d t - \int_{a M}^{b M} \frac{f(t)}{t} \d t.
\]

\item Montrer que
\[
\lim_{\epsilon\to0} \int_{a\epsilon}^{b\epsilon} \frac{f(t) - \ell}{t} \d t = 0.
\]

\item Conclure.
\end{enumerate}
\end{exercice}

\begin{preuve}
\begin{enumerate}
\item La fonction $f$ étant continue sur $[\epsilon, M]$, les intégrales sont bien définies et on utilise la relation de Chasles puis les changements de variables $u \mapsto \frac{u}{a}$ et $u \mapsto \frac{u}{b}$ :
\begin{align*}
\int_\epsilon^M \frac{f(a t) - f(b t)}{t} \d t
&= \int_\epsilon^M \frac{f(a t)}{t} \d t - \int_\epsilon^M \frac{f(b t)}{t} \d t\\
&= \int_{a\epsilon}^{a M} \frac{f(u)}{\frac{u}{a}} a \d u - \int_{b\epsilon}^{b M} \frac{f(u)}{\frac{u}{b}} b \d u\\
&= \int_{a\epsilon}^{a M} \frac{f(u)}{u} \d u - \int_{b\epsilon}^{b M} \frac{f(u)}{u} \d u.
\end{align*}

\item Soit $\eta > 0$. Comme $\lim_{x \to 0} f(x) = \ell$, il existe $\delta > 0$ tel que pour tout $x \in [0, \delta]$,
\[
\module{f(x) - \ell} < \eta.
\]
Soit $\epsilon < \frac{\eta}{b}$. Alors, $[a\epsilon, b\epsilon] \subset [0, \delta]$ et
\begin{align*}
\module{\int_{a\epsilon}^{b\epsilon} \frac{f(t) - \ell}{t} \d t}
&\leq \int_{a\epsilon}^{b\epsilon} \frac{\module{f(t) - \ell}}{t} \d t\\
&\leq \eta \int_{a\epsilon}^{b\epsilon} \frac{1}{t} \d t\\
&\leq \eta \ln\frac{b}{a}.
\end{align*}

Ainsi, $\lim\limits_{\epsilon\to0} \module{\int_{a\epsilon}^{b\epsilon} \frac{f(t) - \ell}{t} \d t} = 0$.

\item On montre de manière analogue que
\[
\lim_{M\to+\infty} \frac{f(t) - L}{t} \d t = 0.
\]
Ainsi, $\int\limits_0^{+\infty} \frac{f(a t) - f(b t)}{t} \d t$ converge et
\[
\int_0^{+\infty} \frac{f(a t) - f(b t)}{t} \d t
= \ell \ln\frac{b}{a} - L \ln\frac{b}{a}.
\]
\end{enumerate}
\end{preuve}

\begin{exercice}
Soit $(a, b) \in \R^2$ avec $0<a<b$. Montrer que les intégrales suivantes sont convergentes et calculer leur valeur.
\begin{enumerate}
\item $I = \int_0^{+\infty} \frac{\e^{-ax} - \e^{-bx}}{x} \d x$.
\item $J = \int_0^{+\infty} \frac{\arctan(bx) - \arctan(ax)}{x} \d x$.
\item $K = \int_0^{+\infty} \frac{\tanh(3x) - \tanh(x)}{x} \d x$.
\item $L = \int_0^1 \frac{t - 1}{\ln(t)} \d t$.
\end{enumerate}
\end{exercice}

\todoinline{L'avant dernière est issue des oraux Mines 2016.}

\begin{preuve}
\begin{enumerate}
\item Comme $\lim_{x\to+\infty} \e^{-x} = 0$ et $\lim_{x\to0} \e^{-x} = 1$, alors $I$ converge et
\[
\int_0^{+\infty} \frac{\e^{-a x} - \e^{-b x}}{x} = \ln\frac{b}{a}.
\]

\item Comme $\lim_{x\to+\infty} \arctan(x) = \frac{\pi}{2}$ et $\lim_{x\to0} \arctan(x) = 0$, alors $J$ converge et
\[
\int_0^{+\infty} \frac{\arctan(b x) - \arctan(a x)}{x} \d x = -\frac{\pi}{2} \ln\frac{b}{a}
= \frac{\pi}{2} \ln\frac{a}{b}.
\]

\item Comme $\lim_{x\to+\infty} \tanh(x) = 1$ et $\lim_{x\to 0} \tanh(x) = 0$, alors $K$ converge et
\[
\int_0^{+\infty} \frac{\tanh(3 x) - \tanh(x)}{x} \d x
= -\ln\frac{1}{3}
= -\ln(3).
\]

\item Le changement de variables $u \mapsto \e^{-u}$ est $\mathscr{C}^1$ et bijectif. Ainsi, l'intégrale $L$ converge si et seulement si $\int_0^{+\infty} \frac{\e^{-u} - 1}{-u} \e^{-u} \d u$ converge. Cette intégrale s'écrit également $\int_0^{+\infty} \frac{\e^{-u} - \e^{-2u}}{u} \d u$.

En utilisant un raisonnement analogue à celui effectué pour l'intégrale $I$, l'intégrale $L$ converge et
\[
\int_0^1 \frac{t - 1}{\ln(t)} \d t
= \ln(2).
\]
\end{enumerate}
\end{preuve}

% \begin{exercice}
    % % % % Soit $\fonctionens{\interoo{0}{+\infty}}{\R}$ une fonction continue admettant une limite finie $L$ en $+\infty$ et une limite finie $\ell$ en $0$. On considère un couple $(a,b) \in \R^2$ vérifiant $0<a<b$ et les intégrales
    % \[
    % % I = \int_0^{+\infty} \frac{f(at) - f(bt)}{t} \d t \quad \text{et} \quad J = \int_0^1 \frac{t-1}{\ln(t)} \d t.
    % \]
    % \begin{enumerate}
        % % \item Montrer que l'intégrale $I$ est convergente et calculer sa valeur. 
        % % \item En déduire que l'intégrale $J$ converge et calculer sa valeur. 
    % \end{enumerate}
% \end{exercice}

% \begin{exercice}
    % % % Soient $\fonctionens{\interof{0}{+\infty}}{\R}$ une fonction continue, un couple $(a, b) \in \R^2$ avec $0<a<b$ et les intégrales
    % \[
    % % I = \int_1^{+\infty} \frac{f(t)}{t} \d t \quad \text{et} \quad J = \int_0^{+\infty} \frac{f(at)-f(bt)}{t} \d t.
    % \]
    % On suppose que l'intégrale $I$ est convergente. 
    % \begin{enumerate}
        % \item Montrer que pour tout $x \in \Rpe$, on a 
        % \[
        % % \int_x^{+\infty} \frac{f(at) - f(bt)}{t} \d t = \int_{ax}^{bx} \frac{f(t)}{t} \d t.
        % \]
        % % \item En déduire que l'intégrale $J$ converge et calculer sa valeur.
    % \end{enumerate}
    % \end{exercice}

% \todoinline{Ajout en vrac d'exercices ci-dessous}



% %---------------

% \begin{exercice}
% {Mines}
% {16}
% {RMS 728}
% % Montrer l'existence puis calculer la valeur de $\int_0^{+\infty} \frac{\tanh(3x) - \tanh(x)}{x} \d x$.
% \end{exercice}

% \begin{preuve}
% % La fonction $f$ est continue sur $\R_+^\ast$, prolongeable par continuité par $2$ en $0$. De plus, en $+\infty$,
% \[
% % \tanh(3 x) - \tanh(x) = \frac{1 - \e^{-6x}}{1 + \e^{-6 x}} - \frac{1 - \e^{-2x}}{1 + \e^{-2x}} = 2 \e^{-2x} + o(\e^{-2x}),
% \]
% % et $f$ est un $o(1/x^2)$. Ainsi, $f$ est intégrable sur $\R_+^\ast$.

% À l'aide d'une formule de changement de variables,
% \[
% % \int_0^M \frac{\tanh(3 x) - \tanh(x)}{x} \d x = \int_{M}^{3M} \frac{\tanh(x)}{x} \d x
% \]
% % Or, quand $M$ est grand, $\abs{\tanh(x) - 1} \leq \eta$ et $\int_M^{3M} \frac{\tanh(x) - 1}{x} \d x \leq \eta \ln(3)$.

% Finalement, $\int_0^{+\infty} \frac{\tanh(3x) - \tanh(x)}{x} \d x = \ln(3)$.
% \end{preuve}

\section{Version intégrale du lemme de \textsc{Cesàro}}

\todoarmand{Voir aussi la section \ref{variante_cesaro}.}

\todoinline{Je mettrais ceci dans un thème Cesaro, dans une partie analyse élémentaire en mettant la version avec les suites, la version $2^{-n} \sum_{k=0}^n \binom{n}{k} u_k \to \ell$ et la version $f(x + 1) - f(x) \to \ell$.}

\begin{itemize}
    \item Pour les applications : \url{http://luls55.free.fr/fichiers/Analyse1/SUITES/11Cesaro.pdf} (II) : déterminer $\lim\limits_{n \to +\infty} w_n$ où $$w_n = \prod_{k=1}^n \left(1 + \frac{2}{k}\right)^{k/n}$$
    \item \url{http://poiret.aurelien.free.fr/Musculation/Analyse/Theoremes%20Tauberiens/Theoremes%20Tauberiens.pdf}
    \item \url{https://math-os.com/histoire-lemme-cesaro/}
\end{itemize}


\begin{theo}
Soit $f$ une fonction continue sur $\Rp$ et $\ell \in \overline{\R}$ tels que $\lim\limits_{+\infty} f = \ell$. Alors 
\[
\lim_{x \to + \infty} \frac{1}{x} \int_0^x f(t) \d t = \ell.
\]
\end{theo}

\begin{exercice}
Soit $f$ une fonction continue sur $\Rp$ telle que $\lim\limits_{x\to+\infty} f(x) = \ell$. On suppose que $\ell \in \R$. Soit $\varepsilon > 0$.
\begin{questions}
\item Montrer qu'il existe $x_0 \in \Rp$ tel que
\[
\forall x \geqslant x_0,\quad \module{f(x) - \ell} \leq \varepsilon.
\]
\todoarmand{Est-ce qu'on garde la Q1) ?}

\item En déduire qu'il existe une constante $K$ telle que pour tout $x \geqslant x_0$,
\[
\module{\frac{1}{x} \int_0^x f(t) \d t - \ell} \leqslant \frac{K}{x} + \varepsilon.
\]

\item Conclure.

\item Démontrer le résultat lorsque $\ell = +\infty$.
\end{questions}
\end{exercice}

\begin{solution}
La démonstration est directement adaptée de celle de la version discrète. 
\begin{reponses}
\item Comme la fonction $f$ converge vers $\ell$ en $+ \infty$, il existe $x_0 \in \Rp$ tel que pour tout $x \geqslant x_0$, $|f(x) - \ell| \leqslant \varepsilon$. \\

\item Comme la fonction $f$ est continue sur le segment $\interff{0}{x_0}$, elle est bornée sur ce segment. Ainsi, en utilisant la relation de \textsc{Chasles},
\begin{align*}
\left| \frac{1}{x} \int_0^x f(t) \d t - \ell \right| &= \left| \frac{1}{x} \int_0^x \mathopen{}\big(f(t) - \ell\big) \d t \right| \\
\text{par l'inégalité triangulaire} &\leqslant \frac{1}{x} \int_0^x |f(t) - \ell| \d t \\
&\leqslant \frac{1}{x} \Bigg( \underbrace{\int_{0}^{x_0} |f(t) - \ell| \d t}_{\defeq K} + \int_{x_0}^{x} \underbrace{|f(t) - \ell|}_{\leqslant \varepsilon} \d t \Bigg) \\
&\leqslant \frac{K}{x} + \varepsilon
\end{align*}

\item Comme $\lim\limits_{x \to \infty} \frac{K}{x} = 0$, il existe $x_1 \in \Rp$ tel que pour tout $x \geqslant x_1$, $ \big| \frac{K}{x} \big| \leqslant \varepsilon$.

Ainsi pour tout $x \geqslant \max \{ x_0, x_1 \}$, 
$$\left| \frac{1}{x} \int_0^x f(t) \d t - \ell \right| \leqslant 2 \varepsilon.$$
On en déduit le résultat. 

\item On suppose que $\ell = +\infty$. Soit $M \in \Rp$. Comme $f$ tend vers $+\infty$ en $+\infty$, il existe $x_0 \in \Rp$ tel que
\[
\forall x \geqslant x_0,\quad f(x) \geqslant M.
\]

Comme $f$ est continue sur le segment $\interff{0}{x_0}$, elle est bornée par une constante $K$ sur ce segment. \\
Ainsi, en utilisant la relation de \textsc{Chasles} et l'inégalité triangulaire, pour tout $x \geqslant x_0$,
\begin{align*}
\frac{1}{x} \int_0^x f(t) \d t
&= \frac{1}{x} \int_0^{x_0} f(t) \d t + \frac{1}{x} \int_{x_0}^x f(t) \d t\\
&\geqslant -\frac{K}{x} + M (x - x_0).
\end{align*}

Comme $\lim\limits_{x\to+\infty} \frac{K}{x} = 0$ et $\lim\limits_{x\to+\infty} M (x - x_0) = +\infty$, il existe un réel $x_1$ tel que
\[
\forall x \geqslant x_1,\quad \frac{K}{x} \geqslant -\frac{1}{2} \quad \text{et} \quad M (x - x_0) \geqslant M + \frac{1}{2}.
\]
Ainsi, pour tout $x \geqslant \max\{x_0, x_1\}$,
\begin{align*}
\frac{1}{x} \int_0^x f(t) \d t
&\geqslant M.
\end{align*}
Finalement, 
\[
\lim\limits_{x\to+\infty} \frac{1}{x}\int_0^x f(t) \d t = +\infty.
\]
\end{reponses}
\end{solution}

\section{Normes $L^p$}
\ref{normes_lp_et_inegalites}


\todoinline{On peut aussi montrer l'inclusion des $L^p$ lorsqu'on intègre sur un segment, c'est ensuite utile dans la théorie de Lebesgue, dans le cadre des probabilités. Il y a une réciproque pour savoir quand $L^p \subset L^q$. J'ai des notes là-dessus, mais faut voir si c'est faisable niveau prépa.}


\subsection{Cas $p = +\infty$}

\subsubsection{Étude de la suite de terme général \texorpdfstring{$\left( \frac{1}{b-a} \int_{a}^{b} f(x)^n \d x \right)^{1/n}$}{égal à une intégrale}}
 \begin{exercice}
    \marginnote[0cm]{Source : \cite{acamanes} \href{https://acamanes.github.io/psi/psi_doc/exos_e01.pdf}{(Exercice 9. TD I)}}
    Soit $f$ une fonction supposée continue et positive sur $\interff{a}{b}$. Étudier la suite de terme général 
    $$u_n \defeq \left( \frac{1}{b-a} \int_{a}^{b} f(x)^n \d x \right)^{1/n}.$$
 \end{exercice}

\begin{preuve}
    \begin{itemize}
        \item La démarche consiste à encadrer le terme $u_n$. 
        \item \textbf{Majoration:} la fonction $f$ est continue et positive sur un segment donc est en particuler, elle y est bornée par un réel positif $M$. Montrons que $u_n \leqslant M$ pour tout $n \in \N$. \emph{ne pas oublier l’argument de la continuité lors du passage à l’intégrale}
        \item \textbf{Minoration:} soit $\varepsilon > 0$, soit $x_0$ tel que $f(x_0) = M$. Comme $f$ est continue en $x_0$, il existe $[c, d] \subset [a, b]$ tel que $x_0 \in [c, d]$ et pour tout $x \in [c, d]$, $f(x) \geqslant M - \varepsilon$ \emph{(un dessin permet de bien comprendre la stratégie)}.\\
        On peut ensuite montrer que $u_n \geqslant \left(\frac{d-c}{b-a} \right)^{1/n}(M-\varepsilon) \xrightarrow[n \to + \infty]{} M-\varepsilon$.
        \item Finalement, $u_n \displaystyle \longrightarrow M = \max_{[ a, b ]} f = \Ninf{f}$.
    \end{itemize}
\end{preuve}

\todoinline{J'ai un pb avec la clé inline dans le tikz suivant}


% \begin{figure}
    % \centering
    
% \begin{tikzpicture}

% \tikzset{>=latex} % for LaTeX arrow head

% \def\tick#1#2{\draw[thick] (#1)++(#2:0.12) --++ (#2-180:0.24)}
% \def\N{100} % number of samples

  % \def\xmax{3.5}
  % \def\ymax{3.5}
  % \def\xeps{3}
  % \def\yeps{1.8}
  % \coordinate (O) at (0,0);
  % \coordinate (C) at (0.927575,0);
  % \coordinate (A) at (0.5,0);
  % \coordinate (B) at (2.796001,0);
  % \coordinate (D) at (2.0700,0);
  % \coordinate (X0) at (1.63694,0);
  % \def\yMAX{2.587020}

  % % AXIS
  % \draw[->,thick]
    % (-0.1*\ymax,0) -- (\xmax,0) node[below] {$x$};
  % \draw[->,thick]
    % (0,-0.1*\ymax) -- (0,\ymax) node[below left] {$\textcolor{red}{f}(x)$}; %\langle{P}\rangle
  
  % % PLOT
  % \draw[xline,red,samples=\N,smooth,variable=\x,domain=-0.1:0.94*\xmax,thick]
    % plot(\x,{3/(1+(0.36*\x*\x-1)^2) -\x/4});
    
  % \draw[dashed,thin] (X0) --++ (0,\yMAX);
  % \draw[dashed,thin] (C) --++ (0,\yeps);
  % \draw[dashed,thin] (D) --++ (0,\yeps);
  
  % \draw[thin] (0,\yMAX) -- (\xmax,\yMAX) node[left] at (0,\yMAX) {$M$};
  % \draw[thin] (0,\yeps) -- (\xmax,\yeps);
  
  % \draw[<->] (\xeps,\yeps) -- (\xeps,\yMAX)
    % node[midway,scale=0.9] {\contour{white}{$\varepsilon$}};
    
  % \tick{X0}{90} node[below] {$x_0$};
  % \tick{A}{90} node[below] {$a$};
  % \tick{B}{90} node[below] {$b$};
  % \tick{C}{90} node[below] {$c$};
  % \tick{D}{90} node[below] {$d$};
  
% \end{tikzpicture}

    % \caption{Illustration à finir}
% \end{figure}

\subsection{Inclusions entre les $L^p(\Omega)$}

\begin{theo}{}
    Si $\Omega$ est de mesure $\module{\Omega}$ finie, alors pour $p<q$, $L^q(\Omega) \subset L^p(\Omega)$ et pour tout $f \in L^q(\Omega)$, $\norm{f}_{L^p} \leqslant \module{\Omega}^{\frac{1}{p} - \frac{1}{q}} \norm{f}_{L^q}$.
\end{theo}

\subsection{Cas $0 < p < 1$}

Lorsque l’exposant $p$ satisfait $0 < p < 1$, on constate qu’une inégalité triangulaire ne peut pas être satisfaite, ce qui justifie, pour bénéficier d’une structure naturelle
d’espace vectoriel, de se restreindre à supposer $1 < p < +\infty$. 

Exercice 2 de \url{https://www.imo.universite-paris-saclay.fr/~joel.merker/Enseignement/Integration/l-p-espaces.pdf}
\begin{exercice}
    On considère les espaces $L^p(\R^d)$ pour  $0 < p < +\infty$. Montrer que si l'on a 
    \[
    \norm{f + g}_{L^p} \leqslant  \norm{f}_{L^p} + \norm{g}_{L^p}
    \]
    pour toutes fonctions $f, g \in L^p(\R^d )$, alors nécessairement $p \geqslant 1$.
\end{exercice}

\subsection{Cas $p \to 0$}

%---------------

\begin{exercice}%

Soit $f \in \mathscr{C}([0,1],\R_+^\ast)$ et $I(y) = \int_0^1 f(x)^y \d x$.
\begin{enumerate}
\item Soit $g$ une fonction définie sur un voisinage de $0$ à valeurs dans $\R_+^\ast$, dérivable en $0$, vérifiant $g(0) = 1$. Déterminer $\lim_{y\to0} g(y)^{1/y}$.
\item Montrer que $I$ est une fonction dérivable sur $\R_+$.
\item En déduire que
\[
\lim_{y\to0} \left(\int_0^1 f(x)^y \d x\right)^{1/y} = \exp\left\{\int_0^1 \ln(f(x)) \d x\right\}.
\]
\end{enumerate}
\end{exercice}


\begin{preuve}
\begin{enumerate}
\item Comme $g$ est dérivable, d'après le théorème de Taylor-Young, $g(y) = 1 + y g'(0) + o(y)$. Ainsi,
\begin{align*}
g(y)^{1/y} &= \exp\left\{\frac{1}{y} \ln(g(0) + y g'(0) + o(y))\right\} \\
&= \exp\left\{g'(0) + o(1)\right\} \\
&\to \e^{g'(0)}.
\end{align*}

\item En posant $F: (x, y) \mapsto f(x)^y$, alors
\[
\abs{\frac{\partial F}{\partial y}(x, y)} = \abs{\ln(f(x)) f(x)^y}.
\]

La fonction $\ln \circ f$ étant bornée sur $[0, 1]$ car continue, on peut appliquer le théorème de dérivation sous le signe intégral sur $[0, a]$ par majoration par $M \cdot M^a$.

D'après le théorème de dérivation sous le signe intégral,
\begin{align*}
I'(y) &= \int_0^1 \ln(f(x)) f(x)^y \d x \\
I'(0) &= \int_0^1 \ln(f(x)) \d x
\end{align*}

\item Finalement,
\[
\lim_{y\to0} \left(\int_0^1 f(x)^y \d x\right)^{1/y} = \exp\left\{\int_0^1 \ln(f(x)) \d x\right\}.
\]
\end{enumerate}
\end{preuve}

\url{https://math.stackexchange.com/questions/2351581/convergence-question-about-lp-norm-when-p-tends-to-zero}


\section{Intégrales dépendant des bornes}

%-----------
\subsection{Une modification du logarithme intégral}
\todoarmand{Ajouter une note sur sa signification en théorie des nombres pour compter les nombres premiers.}

\begin{marginfigure}[-2cm]
    \centering
    %% Creator: Matplotlib, PGF backend
%%
%% To include the figure in your LaTeX document, write
%%   \input{<filename>.pgf}
%%
%% Make sure the required packages are loaded in your preamble
%%   \usepackage{pgf}
%%
%% Also ensure that all the required font packages are loaded; for instance,
%% the lmodern package is sometimes necessary when using math font.
%%   \usepackage{lmodern}
%%
%% Figures using additional raster images can only be included by \input if
%% they are in the same directory as the main LaTeX file. For loading figures
%% from other directories you can use the `import` package
%%   \usepackage{import}
%%
%% and then include the figures with
%%   \import{<path to file>}{<filename>.pgf}
%%
%% Matplotlib used the following preamble
%%   
%%   \usepackage{fontspec}
%%   \setmainfont{DejaVuSerif.ttf}[Path=\detokenize{/home/wayoff/.pyenv/versions/3.8.10/lib/python3.8/site-packages/matplotlib/mpl-data/fonts/ttf/}]
%%   \setsansfont{DejaVuSans.ttf}[Path=\detokenize{/home/wayoff/.pyenv/versions/3.8.10/lib/python3.8/site-packages/matplotlib/mpl-data/fonts/ttf/}]
%%   \setmonofont{DejaVuSansMono.ttf}[Path=\detokenize{/home/wayoff/.pyenv/versions/3.8.10/lib/python3.8/site-packages/matplotlib/mpl-data/fonts/ttf/}]
%%   \makeatletter\@ifpackageloaded{underscore}{}{\usepackage[strings]{underscore}}\makeatother
%%
\begingroup%
\makeatletter%
\begin{pgfpicture}%
\pgfpathrectangle{\pgfpointorigin}{\pgfqpoint{3.000000in}{2.000000in}}%
\pgfusepath{use as bounding box, clip}%
\begin{pgfscope}%
\pgfsetbuttcap%
\pgfsetmiterjoin%
\definecolor{currentfill}{rgb}{1.000000,1.000000,1.000000}%
\pgfsetfillcolor{currentfill}%
\pgfsetlinewidth{0.000000pt}%
\definecolor{currentstroke}{rgb}{1.000000,1.000000,1.000000}%
\pgfsetstrokecolor{currentstroke}%
\pgfsetdash{}{0pt}%
\pgfpathmoveto{\pgfqpoint{0.000000in}{0.000000in}}%
\pgfpathlineto{\pgfqpoint{3.000000in}{0.000000in}}%
\pgfpathlineto{\pgfqpoint{3.000000in}{2.000000in}}%
\pgfpathlineto{\pgfqpoint{0.000000in}{2.000000in}}%
\pgfpathlineto{\pgfqpoint{0.000000in}{0.000000in}}%
\pgfpathclose%
\pgfusepath{fill}%
\end{pgfscope}%
\begin{pgfscope}%
\pgfsetbuttcap%
\pgfsetmiterjoin%
\definecolor{currentfill}{rgb}{1.000000,1.000000,1.000000}%
\pgfsetfillcolor{currentfill}%
\pgfsetlinewidth{0.000000pt}%
\definecolor{currentstroke}{rgb}{0.000000,0.000000,0.000000}%
\pgfsetstrokecolor{currentstroke}%
\pgfsetstrokeopacity{0.000000}%
\pgfsetdash{}{0pt}%
\pgfpathmoveto{\pgfqpoint{0.619136in}{0.576079in}}%
\pgfpathlineto{\pgfqpoint{2.850000in}{0.576079in}}%
\pgfpathlineto{\pgfqpoint{2.850000in}{1.850000in}}%
\pgfpathlineto{\pgfqpoint{0.619136in}{1.850000in}}%
\pgfpathlineto{\pgfqpoint{0.619136in}{0.576079in}}%
\pgfpathclose%
\pgfusepath{fill}%
\end{pgfscope}%
\begin{pgfscope}%
\pgfpathrectangle{\pgfqpoint{0.619136in}{0.576079in}}{\pgfqpoint{2.230864in}{1.273921in}}%
\pgfusepath{clip}%
\pgfsetrectcap%
\pgfsetroundjoin%
\pgfsetlinewidth{0.803000pt}%
\definecolor{currentstroke}{rgb}{0.690196,0.690196,0.690196}%
\pgfsetstrokecolor{currentstroke}%
\pgfsetdash{}{0pt}%
\pgfpathmoveto{\pgfqpoint{0.718507in}{0.576079in}}%
\pgfpathlineto{\pgfqpoint{0.718507in}{1.850000in}}%
\pgfusepath{stroke}%
\end{pgfscope}%
\begin{pgfscope}%
\pgfsetbuttcap%
\pgfsetroundjoin%
\definecolor{currentfill}{rgb}{0.000000,0.000000,0.000000}%
\pgfsetfillcolor{currentfill}%
\pgfsetlinewidth{0.803000pt}%
\definecolor{currentstroke}{rgb}{0.000000,0.000000,0.000000}%
\pgfsetstrokecolor{currentstroke}%
\pgfsetdash{}{0pt}%
\pgfsys@defobject{currentmarker}{\pgfqpoint{0.000000in}{-0.048611in}}{\pgfqpoint{0.000000in}{0.000000in}}{%
\pgfpathmoveto{\pgfqpoint{0.000000in}{0.000000in}}%
\pgfpathlineto{\pgfqpoint{0.000000in}{-0.048611in}}%
\pgfusepath{stroke,fill}%
}%
\begin{pgfscope}%
\pgfsys@transformshift{0.718507in}{0.576079in}%
\pgfsys@useobject{currentmarker}{}%
\end{pgfscope}%
\end{pgfscope}%
\begin{pgfscope}%
\definecolor{textcolor}{rgb}{0.000000,0.000000,0.000000}%
\pgfsetstrokecolor{textcolor}%
\pgfsetfillcolor{textcolor}%
\pgftext[x=0.718507in,y=0.478857in,,top]{\color{textcolor}\sffamily\fontsize{10.000000}{12.000000}\selectfont \(\displaystyle 0\)}%
\end{pgfscope}%
\begin{pgfscope}%
\pgfpathrectangle{\pgfqpoint{0.619136in}{0.576079in}}{\pgfqpoint{2.230864in}{1.273921in}}%
\pgfusepath{clip}%
\pgfsetrectcap%
\pgfsetroundjoin%
\pgfsetlinewidth{0.803000pt}%
\definecolor{currentstroke}{rgb}{0.690196,0.690196,0.690196}%
\pgfsetstrokecolor{currentstroke}%
\pgfsetdash{}{0pt}%
\pgfpathmoveto{\pgfqpoint{1.226538in}{0.576079in}}%
\pgfpathlineto{\pgfqpoint{1.226538in}{1.850000in}}%
\pgfusepath{stroke}%
\end{pgfscope}%
\begin{pgfscope}%
\pgfsetbuttcap%
\pgfsetroundjoin%
\definecolor{currentfill}{rgb}{0.000000,0.000000,0.000000}%
\pgfsetfillcolor{currentfill}%
\pgfsetlinewidth{0.803000pt}%
\definecolor{currentstroke}{rgb}{0.000000,0.000000,0.000000}%
\pgfsetstrokecolor{currentstroke}%
\pgfsetdash{}{0pt}%
\pgfsys@defobject{currentmarker}{\pgfqpoint{0.000000in}{-0.048611in}}{\pgfqpoint{0.000000in}{0.000000in}}{%
\pgfpathmoveto{\pgfqpoint{0.000000in}{0.000000in}}%
\pgfpathlineto{\pgfqpoint{0.000000in}{-0.048611in}}%
\pgfusepath{stroke,fill}%
}%
\begin{pgfscope}%
\pgfsys@transformshift{1.226538in}{0.576079in}%
\pgfsys@useobject{currentmarker}{}%
\end{pgfscope}%
\end{pgfscope}%
\begin{pgfscope}%
\definecolor{textcolor}{rgb}{0.000000,0.000000,0.000000}%
\pgfsetstrokecolor{textcolor}%
\pgfsetfillcolor{textcolor}%
\pgftext[x=1.226538in,y=0.478857in,,top]{\color{textcolor}\sffamily\fontsize{10.000000}{12.000000}\selectfont \(\displaystyle 1/4\)}%
\end{pgfscope}%
\begin{pgfscope}%
\pgfpathrectangle{\pgfqpoint{0.619136in}{0.576079in}}{\pgfqpoint{2.230864in}{1.273921in}}%
\pgfusepath{clip}%
\pgfsetrectcap%
\pgfsetroundjoin%
\pgfsetlinewidth{0.803000pt}%
\definecolor{currentstroke}{rgb}{0.690196,0.690196,0.690196}%
\pgfsetstrokecolor{currentstroke}%
\pgfsetdash{}{0pt}%
\pgfpathmoveto{\pgfqpoint{1.734568in}{0.576079in}}%
\pgfpathlineto{\pgfqpoint{1.734568in}{1.850000in}}%
\pgfusepath{stroke}%
\end{pgfscope}%
\begin{pgfscope}%
\pgfsetbuttcap%
\pgfsetroundjoin%
\definecolor{currentfill}{rgb}{0.000000,0.000000,0.000000}%
\pgfsetfillcolor{currentfill}%
\pgfsetlinewidth{0.803000pt}%
\definecolor{currentstroke}{rgb}{0.000000,0.000000,0.000000}%
\pgfsetstrokecolor{currentstroke}%
\pgfsetdash{}{0pt}%
\pgfsys@defobject{currentmarker}{\pgfqpoint{0.000000in}{-0.048611in}}{\pgfqpoint{0.000000in}{0.000000in}}{%
\pgfpathmoveto{\pgfqpoint{0.000000in}{0.000000in}}%
\pgfpathlineto{\pgfqpoint{0.000000in}{-0.048611in}}%
\pgfusepath{stroke,fill}%
}%
\begin{pgfscope}%
\pgfsys@transformshift{1.734568in}{0.576079in}%
\pgfsys@useobject{currentmarker}{}%
\end{pgfscope}%
\end{pgfscope}%
\begin{pgfscope}%
\definecolor{textcolor}{rgb}{0.000000,0.000000,0.000000}%
\pgfsetstrokecolor{textcolor}%
\pgfsetfillcolor{textcolor}%
\pgftext[x=1.734568in,y=0.478857in,,top]{\color{textcolor}\sffamily\fontsize{10.000000}{12.000000}\selectfont \(\displaystyle 1/2\)}%
\end{pgfscope}%
\begin{pgfscope}%
\pgfpathrectangle{\pgfqpoint{0.619136in}{0.576079in}}{\pgfqpoint{2.230864in}{1.273921in}}%
\pgfusepath{clip}%
\pgfsetrectcap%
\pgfsetroundjoin%
\pgfsetlinewidth{0.803000pt}%
\definecolor{currentstroke}{rgb}{0.690196,0.690196,0.690196}%
\pgfsetstrokecolor{currentstroke}%
\pgfsetdash{}{0pt}%
\pgfpathmoveto{\pgfqpoint{2.242599in}{0.576079in}}%
\pgfpathlineto{\pgfqpoint{2.242599in}{1.850000in}}%
\pgfusepath{stroke}%
\end{pgfscope}%
\begin{pgfscope}%
\pgfsetbuttcap%
\pgfsetroundjoin%
\definecolor{currentfill}{rgb}{0.000000,0.000000,0.000000}%
\pgfsetfillcolor{currentfill}%
\pgfsetlinewidth{0.803000pt}%
\definecolor{currentstroke}{rgb}{0.000000,0.000000,0.000000}%
\pgfsetstrokecolor{currentstroke}%
\pgfsetdash{}{0pt}%
\pgfsys@defobject{currentmarker}{\pgfqpoint{0.000000in}{-0.048611in}}{\pgfqpoint{0.000000in}{0.000000in}}{%
\pgfpathmoveto{\pgfqpoint{0.000000in}{0.000000in}}%
\pgfpathlineto{\pgfqpoint{0.000000in}{-0.048611in}}%
\pgfusepath{stroke,fill}%
}%
\begin{pgfscope}%
\pgfsys@transformshift{2.242599in}{0.576079in}%
\pgfsys@useobject{currentmarker}{}%
\end{pgfscope}%
\end{pgfscope}%
\begin{pgfscope}%
\definecolor{textcolor}{rgb}{0.000000,0.000000,0.000000}%
\pgfsetstrokecolor{textcolor}%
\pgfsetfillcolor{textcolor}%
\pgftext[x=2.242599in,y=0.478857in,,top]{\color{textcolor}\sffamily\fontsize{10.000000}{12.000000}\selectfont \(\displaystyle 3/4\)}%
\end{pgfscope}%
\begin{pgfscope}%
\pgfpathrectangle{\pgfqpoint{0.619136in}{0.576079in}}{\pgfqpoint{2.230864in}{1.273921in}}%
\pgfusepath{clip}%
\pgfsetrectcap%
\pgfsetroundjoin%
\pgfsetlinewidth{0.803000pt}%
\definecolor{currentstroke}{rgb}{0.690196,0.690196,0.690196}%
\pgfsetstrokecolor{currentstroke}%
\pgfsetdash{}{0pt}%
\pgfpathmoveto{\pgfqpoint{2.750629in}{0.576079in}}%
\pgfpathlineto{\pgfqpoint{2.750629in}{1.850000in}}%
\pgfusepath{stroke}%
\end{pgfscope}%
\begin{pgfscope}%
\pgfsetbuttcap%
\pgfsetroundjoin%
\definecolor{currentfill}{rgb}{0.000000,0.000000,0.000000}%
\pgfsetfillcolor{currentfill}%
\pgfsetlinewidth{0.803000pt}%
\definecolor{currentstroke}{rgb}{0.000000,0.000000,0.000000}%
\pgfsetstrokecolor{currentstroke}%
\pgfsetdash{}{0pt}%
\pgfsys@defobject{currentmarker}{\pgfqpoint{0.000000in}{-0.048611in}}{\pgfqpoint{0.000000in}{0.000000in}}{%
\pgfpathmoveto{\pgfqpoint{0.000000in}{0.000000in}}%
\pgfpathlineto{\pgfqpoint{0.000000in}{-0.048611in}}%
\pgfusepath{stroke,fill}%
}%
\begin{pgfscope}%
\pgfsys@transformshift{2.750629in}{0.576079in}%
\pgfsys@useobject{currentmarker}{}%
\end{pgfscope}%
\end{pgfscope}%
\begin{pgfscope}%
\definecolor{textcolor}{rgb}{0.000000,0.000000,0.000000}%
\pgfsetstrokecolor{textcolor}%
\pgfsetfillcolor{textcolor}%
\pgftext[x=2.750629in,y=0.478857in,,top]{\color{textcolor}\sffamily\fontsize{10.000000}{12.000000}\selectfont \(\displaystyle 1\)}%
\end{pgfscope}%
\begin{pgfscope}%
\definecolor{textcolor}{rgb}{0.000000,0.000000,0.000000}%
\pgfsetstrokecolor{textcolor}%
\pgfsetfillcolor{textcolor}%
\pgftext[x=1.734568in,y=0.284413in,,top]{\color{textcolor}\sffamily\fontsize{10.000000}{12.000000}\selectfont \(\displaystyle x\)}%
\end{pgfscope}%
\begin{pgfscope}%
\pgfpathrectangle{\pgfqpoint{0.619136in}{0.576079in}}{\pgfqpoint{2.230864in}{1.273921in}}%
\pgfusepath{clip}%
\pgfsetrectcap%
\pgfsetroundjoin%
\pgfsetlinewidth{0.803000pt}%
\definecolor{currentstroke}{rgb}{0.690196,0.690196,0.690196}%
\pgfsetstrokecolor{currentstroke}%
\pgfsetdash{}{0pt}%
\pgfpathmoveto{\pgfqpoint{0.619136in}{0.633771in}}%
\pgfpathlineto{\pgfqpoint{2.850000in}{0.633771in}}%
\pgfusepath{stroke}%
\end{pgfscope}%
\begin{pgfscope}%
\pgfsetbuttcap%
\pgfsetroundjoin%
\definecolor{currentfill}{rgb}{0.000000,0.000000,0.000000}%
\pgfsetfillcolor{currentfill}%
\pgfsetlinewidth{0.803000pt}%
\definecolor{currentstroke}{rgb}{0.000000,0.000000,0.000000}%
\pgfsetstrokecolor{currentstroke}%
\pgfsetdash{}{0pt}%
\pgfsys@defobject{currentmarker}{\pgfqpoint{-0.048611in}{0.000000in}}{\pgfqpoint{-0.000000in}{0.000000in}}{%
\pgfpathmoveto{\pgfqpoint{-0.000000in}{0.000000in}}%
\pgfpathlineto{\pgfqpoint{-0.048611in}{0.000000in}}%
\pgfusepath{stroke,fill}%
}%
\begin{pgfscope}%
\pgfsys@transformshift{0.619136in}{0.633771in}%
\pgfsys@useobject{currentmarker}{}%
\end{pgfscope}%
\end{pgfscope}%
\begin{pgfscope}%
\definecolor{textcolor}{rgb}{0.000000,0.000000,0.000000}%
\pgfsetstrokecolor{textcolor}%
\pgfsetfillcolor{textcolor}%
\pgftext[x=0.452469in, y=0.581009in, left, base]{\color{textcolor}\sffamily\fontsize{10.000000}{12.000000}\selectfont \(\displaystyle 0\)}%
\end{pgfscope}%
\begin{pgfscope}%
\pgfpathrectangle{\pgfqpoint{0.619136in}{0.576079in}}{\pgfqpoint{2.230864in}{1.273921in}}%
\pgfusepath{clip}%
\pgfsetrectcap%
\pgfsetroundjoin%
\pgfsetlinewidth{0.803000pt}%
\definecolor{currentstroke}{rgb}{0.690196,0.690196,0.690196}%
\pgfsetstrokecolor{currentstroke}%
\pgfsetdash{}{0pt}%
\pgfpathmoveto{\pgfqpoint{0.619136in}{0.968475in}}%
\pgfpathlineto{\pgfqpoint{2.850000in}{0.968475in}}%
\pgfusepath{stroke}%
\end{pgfscope}%
\begin{pgfscope}%
\pgfsetbuttcap%
\pgfsetroundjoin%
\definecolor{currentfill}{rgb}{0.000000,0.000000,0.000000}%
\pgfsetfillcolor{currentfill}%
\pgfsetlinewidth{0.803000pt}%
\definecolor{currentstroke}{rgb}{0.000000,0.000000,0.000000}%
\pgfsetstrokecolor{currentstroke}%
\pgfsetdash{}{0pt}%
\pgfsys@defobject{currentmarker}{\pgfqpoint{-0.048611in}{0.000000in}}{\pgfqpoint{-0.000000in}{0.000000in}}{%
\pgfpathmoveto{\pgfqpoint{-0.000000in}{0.000000in}}%
\pgfpathlineto{\pgfqpoint{-0.048611in}{0.000000in}}%
\pgfusepath{stroke,fill}%
}%
\begin{pgfscope}%
\pgfsys@transformshift{0.619136in}{0.968475in}%
\pgfsys@useobject{currentmarker}{}%
\end{pgfscope}%
\end{pgfscope}%
\begin{pgfscope}%
\definecolor{textcolor}{rgb}{0.000000,0.000000,0.000000}%
\pgfsetstrokecolor{textcolor}%
\pgfsetfillcolor{textcolor}%
\pgftext[x=0.344444in, y=0.915713in, left, base]{\color{textcolor}\sffamily\fontsize{10.000000}{12.000000}\selectfont \(\displaystyle 0{,}2\)}%
\end{pgfscope}%
\begin{pgfscope}%
\pgfpathrectangle{\pgfqpoint{0.619136in}{0.576079in}}{\pgfqpoint{2.230864in}{1.273921in}}%
\pgfusepath{clip}%
\pgfsetrectcap%
\pgfsetroundjoin%
\pgfsetlinewidth{0.803000pt}%
\definecolor{currentstroke}{rgb}{0.690196,0.690196,0.690196}%
\pgfsetstrokecolor{currentstroke}%
\pgfsetdash{}{0pt}%
\pgfpathmoveto{\pgfqpoint{0.619136in}{1.303179in}}%
\pgfpathlineto{\pgfqpoint{2.850000in}{1.303179in}}%
\pgfusepath{stroke}%
\end{pgfscope}%
\begin{pgfscope}%
\pgfsetbuttcap%
\pgfsetroundjoin%
\definecolor{currentfill}{rgb}{0.000000,0.000000,0.000000}%
\pgfsetfillcolor{currentfill}%
\pgfsetlinewidth{0.803000pt}%
\definecolor{currentstroke}{rgb}{0.000000,0.000000,0.000000}%
\pgfsetstrokecolor{currentstroke}%
\pgfsetdash{}{0pt}%
\pgfsys@defobject{currentmarker}{\pgfqpoint{-0.048611in}{0.000000in}}{\pgfqpoint{-0.000000in}{0.000000in}}{%
\pgfpathmoveto{\pgfqpoint{-0.000000in}{0.000000in}}%
\pgfpathlineto{\pgfqpoint{-0.048611in}{0.000000in}}%
\pgfusepath{stroke,fill}%
}%
\begin{pgfscope}%
\pgfsys@transformshift{0.619136in}{1.303179in}%
\pgfsys@useobject{currentmarker}{}%
\end{pgfscope}%
\end{pgfscope}%
\begin{pgfscope}%
\definecolor{textcolor}{rgb}{0.000000,0.000000,0.000000}%
\pgfsetstrokecolor{textcolor}%
\pgfsetfillcolor{textcolor}%
\pgftext[x=0.344444in, y=1.250418in, left, base]{\color{textcolor}\sffamily\fontsize{10.000000}{12.000000}\selectfont \(\displaystyle 0{,}4\)}%
\end{pgfscope}%
\begin{pgfscope}%
\pgfpathrectangle{\pgfqpoint{0.619136in}{0.576079in}}{\pgfqpoint{2.230864in}{1.273921in}}%
\pgfusepath{clip}%
\pgfsetrectcap%
\pgfsetroundjoin%
\pgfsetlinewidth{0.803000pt}%
\definecolor{currentstroke}{rgb}{0.690196,0.690196,0.690196}%
\pgfsetstrokecolor{currentstroke}%
\pgfsetdash{}{0pt}%
\pgfpathmoveto{\pgfqpoint{0.619136in}{1.637884in}}%
\pgfpathlineto{\pgfqpoint{2.850000in}{1.637884in}}%
\pgfusepath{stroke}%
\end{pgfscope}%
\begin{pgfscope}%
\pgfsetbuttcap%
\pgfsetroundjoin%
\definecolor{currentfill}{rgb}{0.000000,0.000000,0.000000}%
\pgfsetfillcolor{currentfill}%
\pgfsetlinewidth{0.803000pt}%
\definecolor{currentstroke}{rgb}{0.000000,0.000000,0.000000}%
\pgfsetstrokecolor{currentstroke}%
\pgfsetdash{}{0pt}%
\pgfsys@defobject{currentmarker}{\pgfqpoint{-0.048611in}{0.000000in}}{\pgfqpoint{-0.000000in}{0.000000in}}{%
\pgfpathmoveto{\pgfqpoint{-0.000000in}{0.000000in}}%
\pgfpathlineto{\pgfqpoint{-0.048611in}{0.000000in}}%
\pgfusepath{stroke,fill}%
}%
\begin{pgfscope}%
\pgfsys@transformshift{0.619136in}{1.637884in}%
\pgfsys@useobject{currentmarker}{}%
\end{pgfscope}%
\end{pgfscope}%
\begin{pgfscope}%
\definecolor{textcolor}{rgb}{0.000000,0.000000,0.000000}%
\pgfsetstrokecolor{textcolor}%
\pgfsetfillcolor{textcolor}%
\pgftext[x=0.344444in, y=1.585122in, left, base]{\color{textcolor}\sffamily\fontsize{10.000000}{12.000000}\selectfont \(\displaystyle 0{,}6\)}%
\end{pgfscope}%
\begin{pgfscope}%
\definecolor{textcolor}{rgb}{0.000000,0.000000,0.000000}%
\pgfsetstrokecolor{textcolor}%
\pgfsetfillcolor{textcolor}%
\pgftext[x=0.288889in,y=1.213040in,,bottom,rotate=90.000000]{\color{textcolor}\sffamily\fontsize{10.000000}{12.000000}\selectfont \(\displaystyle f(x)\)}%
\end{pgfscope}%
\begin{pgfscope}%
\pgfpathrectangle{\pgfqpoint{0.619136in}{0.576079in}}{\pgfqpoint{2.230864in}{1.273921in}}%
\pgfusepath{clip}%
\pgfsetrectcap%
\pgfsetroundjoin%
\pgfsetlinewidth{1.505625pt}%
\definecolor{currentstroke}{rgb}{0.000000,0.000000,1.000000}%
\pgfsetstrokecolor{currentstroke}%
\pgfsetdash{}{0pt}%
\pgfpathmoveto{\pgfqpoint{0.720539in}{0.633985in}}%
\pgfpathlineto{\pgfqpoint{0.748989in}{0.638700in}}%
\pgfpathlineto{\pgfqpoint{0.793696in}{0.648483in}}%
\pgfpathlineto{\pgfqpoint{0.850595in}{0.663398in}}%
\pgfpathlineto{\pgfqpoint{0.911559in}{0.681670in}}%
\pgfpathlineto{\pgfqpoint{0.980651in}{0.704735in}}%
\pgfpathlineto{\pgfqpoint{1.053807in}{0.731528in}}%
\pgfpathlineto{\pgfqpoint{1.131028in}{0.762174in}}%
\pgfpathlineto{\pgfqpoint{1.212313in}{0.796828in}}%
\pgfpathlineto{\pgfqpoint{1.297662in}{0.835665in}}%
\pgfpathlineto{\pgfqpoint{1.387075in}{0.878871in}}%
\pgfpathlineto{\pgfqpoint{1.476489in}{0.924511in}}%
\pgfpathlineto{\pgfqpoint{1.569966in}{0.974701in}}%
\pgfpathlineto{\pgfqpoint{1.667508in}{1.029651in}}%
\pgfpathlineto{\pgfqpoint{1.765050in}{1.087129in}}%
\pgfpathlineto{\pgfqpoint{1.866656in}{1.149592in}}%
\pgfpathlineto{\pgfqpoint{1.968262in}{1.214610in}}%
\pgfpathlineto{\pgfqpoint{2.073933in}{1.284855in}}%
\pgfpathlineto{\pgfqpoint{2.183667in}{1.360555in}}%
\pgfpathlineto{\pgfqpoint{2.293402in}{1.438986in}}%
\pgfpathlineto{\pgfqpoint{2.407201in}{1.523132in}}%
\pgfpathlineto{\pgfqpoint{2.520999in}{1.610072in}}%
\pgfpathlineto{\pgfqpoint{2.638863in}{1.702997in}}%
\pgfpathlineto{\pgfqpoint{2.748597in}{1.792095in}}%
\pgfpathlineto{\pgfqpoint{2.748597in}{1.792095in}}%
\pgfusepath{stroke}%
\end{pgfscope}%
\begin{pgfscope}%
\pgfsetrectcap%
\pgfsetmiterjoin%
\pgfsetlinewidth{0.803000pt}%
\definecolor{currentstroke}{rgb}{0.000000,0.000000,0.000000}%
\pgfsetstrokecolor{currentstroke}%
\pgfsetdash{}{0pt}%
\pgfpathmoveto{\pgfqpoint{0.619136in}{0.576079in}}%
\pgfpathlineto{\pgfqpoint{0.619136in}{1.850000in}}%
\pgfusepath{stroke}%
\end{pgfscope}%
\begin{pgfscope}%
\pgfsetrectcap%
\pgfsetmiterjoin%
\pgfsetlinewidth{0.803000pt}%
\definecolor{currentstroke}{rgb}{0.000000,0.000000,0.000000}%
\pgfsetstrokecolor{currentstroke}%
\pgfsetdash{}{0pt}%
\pgfpathmoveto{\pgfqpoint{2.850000in}{0.576079in}}%
\pgfpathlineto{\pgfqpoint{2.850000in}{1.850000in}}%
\pgfusepath{stroke}%
\end{pgfscope}%
\begin{pgfscope}%
\pgfsetrectcap%
\pgfsetmiterjoin%
\pgfsetlinewidth{0.803000pt}%
\definecolor{currentstroke}{rgb}{0.000000,0.000000,0.000000}%
\pgfsetstrokecolor{currentstroke}%
\pgfsetdash{}{0pt}%
\pgfpathmoveto{\pgfqpoint{0.619136in}{0.576079in}}%
\pgfpathlineto{\pgfqpoint{2.850000in}{0.576079in}}%
\pgfusepath{stroke}%
\end{pgfscope}%
\begin{pgfscope}%
\pgfsetrectcap%
\pgfsetmiterjoin%
\pgfsetlinewidth{0.803000pt}%
\definecolor{currentstroke}{rgb}{0.000000,0.000000,0.000000}%
\pgfsetstrokecolor{currentstroke}%
\pgfsetdash{}{0pt}%
\pgfpathmoveto{\pgfqpoint{0.619136in}{1.850000in}}%
\pgfpathlineto{\pgfqpoint{2.850000in}{1.850000in}}%
\pgfusepath{stroke}%
\end{pgfscope}%
\end{pgfpicture}%
\makeatother%
\endgroup%

    \caption{Graphe de la fonction $f \colon x \mapsto \int_x^{x^2} \frac{\d t}{\ln(t)}$ sur l'intervalle $\interoo{0}{1}$}
\end{marginfigure}

\begin{prop}
Pour tout $x \in \interff{0}{1}$, on définit $f(x) = \int_x^{x^2} \frac{\d t}{\ln(t)}$.
\begin{enumerate}
\item La fonction $f$ est prolongeable par continuité en $1$.

\item La fonction $f$ est dérivable sur $\interff{0}{1}$.
\end{enumerate}
\end{prop}

\begin{remarque}
La fonction $x \mapsto \int_0^x \frac{1}{\ln(t)} \d t$ est la fonction \textsl{logarithme intégral}.
\end{remarque}

\begin{exercice}
\begin{questions}
\item En utilisant la concavité du logarithme, montrer que
\[
\forall\, t \in \big]x^2\,;1\big],\quad
\frac{\ln(x^2)}{x^2 - 1} (t - 1) \leqslant \ln(t).
\]

\item En déduire que
\[
\frac{(x + 1) (x - 1)}{2 \ln(x)} \ln\module{x + 1} \leqslant f(x) \leqslant \ln\module{x + 1}.
\]

\item Montrer que $f$ est prolongeable en par continuité en $0$.

\item Calculer $f'$ sur $\interof{0}{1}$ puis montrer que $f$ est dérivable en $0$.
\end{questions}
\end{exercice}


\begin{elemsolution}
\begin{reponses}
\item Soit $t \in \interfo{x^2}{1}$. Comme la fonction logarithme est concave, sa courbe représentative se situe en-dessous de sa tangente en $1$ et
\[
\ln(t) - \ln(1) \leqslant t - 1.
\]

\medskip

Toujours par concavité de la fonction logarithme, la corde reliant les points de coordonnées $\big(x^2, \ln\big(x^2\big)\big)$ et $(1, 0)$ se situe en-dessous de la courbe représentant le logarithme, soit :
\[
\frac{\ln(x^2) - \ln(1)}{x^2 - 1} (t - 1) \leqslant \ln(t).
\]

\item En utilisant l'encadrement précédent, pour tout $t \in \interoo{x^2}{1}$,
\begin{align*}
\frac{1}{t - 1} \leqslant \frac{1}{\ln(t)} &\leqslant \frac{x^2 - 1}{2 \ln(x)} \times \frac{1}{t - 1}\\
\frac{x^2 - 1}{2 \ln(x)} \int_x^{x^2} \frac{\d t}{t - 1} &\leqslant f(x) \leqslant \int_x^{x^2} \frac{\d t}{t - 1}\\
\frac{x^2 - 1}{2 \ln(x)} \ln \module{\frac{x^2 - 1}{x - 1}} &\leqslant f(x) \leqslant \ln\module{\frac{x^2 - 1}{x - 1}}\\
\frac{(x + 1) (x - 1)}{2 \ln(x)} \ln\module{x + 1} &\leqslant f(x) \leqslant \ln\module{x + 1}.
\end{align*}

Ainsi, d'après le \theoremeutilise{théorème d'encadrement}{theo:encadrement}, $\lim\limits_{x\to 1} f(x) = \ln(2)$.

\item En utilisant la dérivation par rapport aux bornes,
\begin{align*}
f'(x)
= 2 x \frac{1}{\ln(x^2)} - x \frac{1}{\ln(x)} 
= \frac{x - 1}{\ln(x)}.
\end{align*}

La fonction $f$ est prolongeable par continuité sur $\interof{0}{1}$ et dérivable sur $\interoo{0}{1}$. De plus, $\lim\limits_{x\to 1} f'(x) = 1$. D'après le \theoremeutilise{théorème de prolongement dérivable}{theo:prolongementderivable}, $f$ est dérivable en $1$ et $f'(1) = 1$.
\end{reponses}
\end{elemsolution}

%-----------
\subsection{Un développement asymptotique}

\todoarmand{Lien avec l'exponentielle intégrale}

\begin{prop}
On pose
\[
\fonctionligne[F]{x}{\int_x^{+\infty} \frac{\e^{-t}}{t} \d t}.
\]
Alors,
\[
F(x) = \frac{\e^{-x}}{x} + \frac{\e^{-x}}{x^2} + o_{+\infty}\mathopen{}\bigg(\frac{\e^{-x}}{x^2}\bigg)
\quad \text{et} \quad
F(x) \sim_0 -\ln(x).
\]
\end{prop}

\begin{exercice}
\begin{questions}
\item Déterminer l'ensemble de définition de $F$. Étudier brièvement le comportement de la fonction $F$ et tracer sa courbe représentative.

\item Montrer que $F(x) \sim_{+\infty} \frac{\e^{-x}}{x}$.

\item Montrer que $F(x) = \frac{\e^{-x}}{x} + \frac{\e^{-x}}{x^2} + o_{+\infty}\mathopen{}\left(\frac{\e^{-x}}{x^2}\right)$.
       
\item Montrer que $F(x) \sim_0 -\ln(x)$.
\end{questions}
\end{exercice}

\begin{marginfigure}[-8cm]
    \centering
    \begin{tikzpicture}
\begin{axis}[
    xtick={10,15},
    xticklabels={$10$, $15$},
    ytick={0,0.000005},
    yticklabels={$0$, $5 \cdot 10^{-6}$},
    scaled y ticks=false,
    xlabel=$x$,
    width=7cm,
    axis lines=middle,
    axis line style=thick,
    axis line style={-latex},
    % xlabel style={at={(axis description cs:1.05,0)}, anchor=north},
    % ylabel style={at={(axis description cs:0,1.05)}, anchor=south},
    xmin=9.5,
    xmax=20,
    ymin=0,
    ymax=0.0000055,
    restrict x to domain=10:20,
    legend style={
        draw=none,
        fill=none,
        font=\footnotesize,
        legend image code/.code={\node[anchor=west] {#1};}
    },
    every axis x label/.style={at={(current axis.right of origin)},anchor=north},
]

% Plot from the data file
\draw[color=myred, thick, domain=10:18, samples=100] plot (\x,{exp(-\x)/\x + exp(-\x)/(\x^2)});
\addlegendentry{\textcolor{myred}{$\displaystyle x \mapsto \frac{\e^{-x}}{x} + \frac{\e^{-x}}{x^2}$}}

\addplot[myblue, thick, smooth] table[x=x, y=y] {chapitres/integration/illustrations/i_16-un_developpement_asymptotique.dat};

\end{axis}
\end{tikzpicture}
    \caption{Représentation graphique de la fonction $F$ et des premiers termes de son développement asymptotique en $+\infty$}
\end{marginfigure}
\begin{marginfigure}[-1cm]
    \centering
    \begin{tikzpicture}
\begin{axis}[
    xtick={0.001,0.5},
    xticklabels={$10^{-3}$,$0{,}5$},
    ytick={10},
    yticklabels={$10$},
    scaled y ticks=false,
    xlabel=$x$,
    width=7cm,
    axis lines=middle,
    axis line style=thick,
    axis line style={-latex},
    % xlabel style={at={(axis description cs:1.05,0)}, anchor=north},
    % ylabel style={at={(axis description cs:0,1.05)}, anchor=south},
    xmin=0,
    xmax=0.55,
    ymin=0,
    ymax=11,
    restrict x to domain=0.0001:0.5,
    legend style={
        draw=none,
        fill=none,
        font=\footnotesize,
        legend image code/.code={\node[anchor=west] {#1};}
    },
    every axis x label/.style={at={(current axis.right of origin)},anchor=north},
]

% Plot from the data file
\draw[color=mygreen, thick, domain=0.0001:0.5, samples=100] plot (\x,{-ln(\x)});
\addlegendentry{\textcolor{mygreen}{$x \mapsto -\ln(x)$}}

\addplot[myblue, thick, smooth] table[x=x, y=y] {chapitres/integration/illustrations/i_16-un_developpement_asymptotique.dat};
% \addlegendentry{\textcolor{myblue}{$\displaystyle x \mapsto \int_x^{+\infty} \frac{\e^{-t}}{t} \d t$}}
\end{axis}
\end{tikzpicture}

    \caption{Représentation graphique de la fonction $F$ proche de $0$}
\end{marginfigure}

\begin{elemsolution}
\begin{reponses}
\item La fonction $\fonctionligne[f]{t}{\frac{\e^{-t}}{t}}$ est positive et continue sur $\Re$.

\begin{itemize}
\item D'après le \theoremeutilise{théorème des croissances comparées}{theo:croissancescomparees}, $\frac{\e^{-t}}{t} = o_{+\infty}\mathopen{}\left(\frac{1}{t^2}\right)$. Ainsi, d'après le \theoremeutilise{théorème de comparaison des intégrales de fonctions à valeurs positives}{theo:comparaisonintegralesfonctionsvaleurspositives}, $\int_1^{+\infty} \frac{\e^{-t}}{t} \d t$ converge.

\item Comme $f(t) \sim_0 \frac{1}{t}$, d'après le théorème de comparaison des intégrales de fonctions à valeurs positives, $\int_0^1 \frac{\e^{-t}}{t} \d t$ diverge.
\end{itemize}

Ainsi, le domaine de définition de $F$ est $\interoo{0}{+ \infty}$.

\item Les fonctions $\fonctionligne[u]{t}{-\e^{-t}}$ et $\fonctionligne[v]{t}{\frac{1}{t}}$ sont de classe $\Cont^1$ sur $\interfo{x}{+\infty}$ et $\lim\limits_{t\to+\infty} u(t) v(t) = 0$. Ainsi, d'après le \theoremeutilise{théorème d'intégration par parties}{theo:ippgeneralisees}, pour $x > 1$, 
\begin{align*}
F(x)
&= \int_x^{+ \infty} \frac{\e^{-t}}{t} \d t 
= \frac{\e^{-x}}{x} - \int_x^{+\infty} \frac{\e^{-t}}{t^2} \d t.
\end{align*}

De plus,
\begin{align*}
\int_x^{+\infty} \frac{\e^{-t}}{t^2} \d t
&\leqslant \e^{-x} \int_x^{+\infty} \frac{1}{t^2} \d t
\leqslant \frac{\e^{-x}}{x^2}.
\end{align*}

Ainsi,
$$\frac{\e^{-t}}{t^2} = o_{+\infty}\mathopen{}\bigg( \frac{\e^{-t}}{t} \bigg).$$

D'où,
\[
\int_x^{+\infty} \frac{\e^{-t}}{t} \d t \sim_{+\infty} \frac{\e^{-x}}{x}.
\]

\begin{remarque}
On aurait également pu utiliser les théorèmes d'intégration des relations de comparaison car $\e^{-x}/x = o(\e^{-x}/x^2)$.
\end{remarque}

\item On effectue une nouvelle intégration par parties avec $\fonctionligne[u]{t}{\e^{-t}}$ et $\fonctionligne[v]{t}{\frac{1}{t^2}}$. On obtient ainsi
\begin{align*}
F(x)
&= \frac{\e^{-x}}{x} + \frac{\e^{-x}}{x^2} - \int_x^{+\infty} \frac{2 \e^{-t}}{t^3} \d t.
\end{align*}

On montre alors comme précédemment que l'intégrable est négligeable, en $+\infty$, devant~$\frac{\e^{-x}}{x^2}$.

\item En utilisant la relation de \nom{Chasles},
\begin{align*}
F(x)
&= \int_x^1 \frac{\e^{-t}}{t} \d t + \int_1^{+\infty} \frac{\e^{-t}}{t} \d t.
\end{align*}

De plus, la fonction exponentielle étant convexe, pour tout $t > 0$,
\begin{align*}
1 - t &\leqslant \e^{-t} \leqslant 1\\
\frac{1}{t} - 1 &\leqslant \frac{\e^{-t}}{t} \leqslant \frac{1}{t}\\
-\ln(x) - (1 - x) &\leqslant \int_x^1 \frac{\e^{-t}}{t} \d t \leqslant -\ln(x).
\end{align*}

D'après le \theoremeutilise{théorème d'encadrement}{theo:encadrement}, $\int_x^1 \frac{\e^{-t}}{t} \d t \sim_0 -\ln(x)$ et
\[
F(x) \sim_0 -\ln(x).
\]
\end{reponses}
\end{elemsolution}

\section{Prolongement de fonctions (par des intégrales)}

\subsection{Prolongement par continuité d'une fonction à variable réelle / Régularité du sinus cardinal sur $\R$}

La fonction $\fonction[f]{\Re}{\R}{x}{\frac{\sin(x)}{x}}$ admet un prolongement par continuité en $0$, appelé \emph{sinus cardinal}. En effet, considérons la fonction
\[
    g(x) \defeq
    \begin{cases} 
        \frac{\sin x}{x} &\text{si } x \not= 0 \\ 
        1 &\text{sinon} 
    \end{cases}.
\]
Cette fonction est continue en $0$ puisque, pour tout $x \in \Re$,
\[
\module{g(x) - g(0)} = \module{\frac{\sin(x) - 0}{x - 0} - 1} \xrightarrow[x \to 0]{} \module{\sin'(0) - 1} = \module{\cos(0) - 1} = 0. 
\]
La fonction $g$ constitue donc un prolongement de la fonction $f$ en $0$. \\
De plus, en utilisant le développement en série entière de la fonction sinus, on peut écrire pour $x$ réel non nul, 
\[
g(x) = \sum\limits_{n=0}^{+ \infty} (-1)^n \frac{x^{2n}}{(2n+1)!},
\]
ce qui reste vrai pour $x = 0$. La fonction $g$ est donc développable en série entière sur $\R$ et en particulier, la fonction $g$ est de classe $\mathscr{C}^\infty$ sur $\R$.
\marginnote[0cm]{fic00126}

\subsection{Fonction Gamma d'\textsc{Euler}}\label{prolongementFonctionGamma}

On rappelle la définition de la fonction Gamma d'\textsc{Euler} (cf. \ref{secinteuleriennes}).
\begin{defi}[Fonction Gamma d'\textsc{Euler}]
    La \emph{fonction Gamma d'\textsc{Euler}} est définie par: 
    $$\Gamma(x) \defeq \int_{0}^{+\infty} t^{x-1} \e^{-t} \d t.$$
\end{defi}

\begin{prop}
    \begin{itemize}
        \item La fonction $\Gamma$ est définie si et seulement si $x>0$.
        \item Pour tout $x > 0$, $\Gamma(x+1) = x\Gamma(x)$. \\
        En particulier, pour tout $n \in \N$, $\Gamma(n+1) = n!$. 
    \end{itemize}
\end{prop}

\begin{demo}
    \begin{itemize}
        \item La fonction $f_x:t \mapsto t^{x-1} \e^{-t}$ est continue sur $]0, + \infty[$ comme produit de fonctions qui y sont continues. La fonction $f_x$ est donc intégrable sur tout segment de $]0, +\infty[$. Il reste à étudier son intégrabilité en $0$ et en $+ \infty$:
        \begin{itemize}
            \item[$\triangleright$] En $+\infty:$ par croissances comparées, $f_x(t) = o_{+\infty} \left(\frac{1}{t^2} \right)$. D'après le théorème de comparaison des fonctions à termes positifs, $f_x$ est intégrable au voisinage de $+\infty$.
            \item[$\triangleright$] En $0$: $f_x(t) \sim_0 t^{x-1}$ qui est intégrable d'après \textsc{Riemann} si et seulement si $1-x < 1$ i.e. si et seulement si $x > 0$.
        \end{itemize}

        \item Soit $x > 0$. Calculons $\Gamma(x+1)$ en effectuant une intégration par parties. Posons $u:t \mapsto \e^{-t}$ et $v:t \mapsto t^x$, toutes deux de classe $\mathscr{C}^1$ sur $\Rp$. Vérifions la convergence du \emph{crochet}:
        \begin{align*}
            \text{par croissances comparées} \quad & \lim_{t \to +\infty} u(t) v(t) = 0, \\
            \text{ comme } x > 0 \quad & \lim_{t \to 0} u(t) v(t) = 0.
        \end{align*}
        Ainsi, d'après le théorème d'intégration par parties généralisées, 
        $$\int_{0}^{+\infty} t^{x-1} \e^{-t} \d t = \underbrace{0}_{\mathclap{\text{crochet}}} - \int_{0}^{+\infty} xt^{x-1} (-\e^{-t}) \d t.$$
        soit 
        $$\Gamma(x+1) = x \Gamma(x).$$
        En particulier, $\Gamma(1) = 1$ et pour tout $n \in \Ne, \Gamma(n+1) = n \Gamma(n)$. Donc
        $$\forall n \in \Ne, \Gamma(n+1) = n!$$
    \end{itemize}
\end{demo}


\subsection{Fonction zêta de \textsc{Riemann}}\label{subsec:fonctionZeta}

\todoinline{Pourrait être mis dans une section : prolongement de fonctions par des intégrales ; au même titre que $\Gamma$ prolonge la factorielle. Il faudrait peut être mettre un mot sur l'unicité du prolongement analytique.}

\begin{prop}
Pour tout $s > 1$, on rappelle que $\zeta(s) = \sum_{n=1}^{+\infty} \frac{1}{n^s}$. Pour tout $s > 0$, posons $I(s) = \frac{s}{s - 1} - s \int_1^\infty \frac{x - \lfloor x\rfloor}{x^{s+1}} \d x$. Alors,
\[
\forall\, s > 1,\, \zeta(s) = I(s).
\]
La fonction $I$ prolonge ainsi $\zeta$ sur le segment $]0, 1]$.
\end{prop}

%---------------
\begin{exercice}[Fonction zêta de Riemann]

Source : Rudin Ex.16 p.131

Pour tout réel $s > 1$, on pose $\zeta(s) = \sum_{n=1}^\infty \frac{1}{n^s}$.
\begin{itemize}
\item Montrer que $\zeta(s) = s \int_1^\infty \frac{\lfloor x\rfloor}{x^{s + 1}} \d x$.
% \indic{Indication : Comparer l'intégrale sur le segment $[1,N]$ et la somme partielle.}

\item Montrer que $\zeta(s) = \frac{s}{s - 1} - s \int_1^\infty \frac{x - \lfloor x\rfloor}{x^{s+1}} \d x$.

\item Montrer que cette dernière intégrale converge pour tout $s > 0$.
\end{itemize}
\end{exercice}

\begin{solution}
\begin{itemize}
\item Si $M$ est un réel, alors
\begin{align*}
s \int_1^M \frac{\lfloor x\rfloor}{x^{s + 1}} \d x
&= s \int_1^{\lfloor M\rfloor} \frac{\lfloor x\rfloor}{x^{s + 1}} \d x + s \int_{\lfloor M\rfloor}^M \frac{\lfloor x\rfloor}{x^{s + 1}} \d x.
\end{align*}

Or, le second membre vaut
\[
s \lfloor M \rfloor \int_{\lfloor M\rfloor}^M \frac{1}{x^{s + 1}} \d x
= \frac{1}{M^s} - \frac{1}{\lfloor M \rfloor^s}.
\]

Il tend donc vers $0$ lorsque $M \to +\infty$. Pour étudier la convergence de l'intégrale on peut donc se limiter à l'étudier pour une suite d'entiers.

\item On utilise la relation de Chasles pour découper l'intervalle en morceaux sur lesquels la fonction partie entière est constante :
\begin{align*}
s \int_1^N \frac{\lfloor x\rfloor}{x^{s+1}} \d x &= \sum_{n=1}^{N-1} s \int_n^{n+1} \frac{\lfloor x\rfloor}{x^{s+1}} \d x\\
&= \sum_{n=1}^{N-1} n \left[-\frac{1}{(n+1)^s} + \frac{1}{n^s}\right] \\
&= \sum_{n=1}^{N-1} \left(\frac{1}{n^{s-1}} - \frac{n + 1 - 1}{(n+1)^{s-1}}\right) \\
&= \sum_{n=1}^{N-1} \frac{1}{n^{s-1}} - \sum_{n=1}^{N-1} \frac{1}{(n+1)^{s-1}} + \sum_{n=1}^{N-1} \frac{1}{(n+1)^s} \\
&= 1 - \frac{1}{N^{s-1}} + \sum_{n=2}^N \frac{1}{n^s}.
\end{align*}
Comme le membre de droite converge vers $\zeta(s)$ lorsque $N \to +\infty$, alors l'intégrale est convergente et on obtient la relation indiquée.

\item En utilisant la relation précédente,
\begin{align*}
s \int_1^M \frac{x - \lfloor x\rfloor}{x^{s+1}} \d x &= s \int_1^M x^{-s} \d x - s \int_1^M \frac{\lfloor x\rfloor}{x^{s+1}} \d x \\
&= \frac{s}{s-1} \left[1 - \frac{1}{M^{s-1}}\right] - s \int_1^M \frac{\lfloor x\rfloor}{x^{s+1}} \d x.
\end{align*}
Comme le membre de droite converge vers $\frac{s}{s-1} - \zeta(s)$, alors l'intégrale converge et on obtient la relation demandée.

\item D'après la définition de la fonction partie entière,
\[
\abs{\frac{x-\lfloor x\rfloor}{x^{s+1}}} \leq \frac{1}{x^{s+1}}
\]
Ainsi, cette intégrale est convergente dès que $s > 0$.
\end{itemize}
\end{solution}

%========
\section{Inégalités intégrales}

%-----------
\subsection{En utilisant l'inégalité de Cauchy-Schwarz}

%---------------

\todoinline{Source du concours : ENSAM 2016}

\todoinline{Si on a trop de matériel dans ce chapitre, on pourrait déplacer le contenu dans la partie espaces préhilbertiens.}

\begin{prop}%
Soit $f$ une fonction de classe $\mathscr{C}^2$ sur $\R_+$ et à valeurs réelles telle que $f(0) = 0$. On suppose que $f$ et $f''$ sont de carrés intégrables. Montrer que $f'$ est de carré intégrable et que
\[
\left(\int_0^{+\infty} (f'(t))^2\d t \right)^2
\leq
\left(\int_0^{+\infty} f^2(t) \d t\right)
\left(\int_0^{+\infty} (f''(t))^2 \d t\right).
\]
\end{prop}

\begin{exercice}
Soit $M \geq 0$.
\begin{questions}
\item Exprimer $\int_0^M (f'(t))^2 \d t$ en fonction de $\int_0^M f(t) f''(t) \d t$.

\item En déduire que $f'$ est de carré intégrable.

\item Montrer que $\lim\limits_{M\to+\infty} f(M) f'(M) = 0$.

\item Montrer que $\left(\int_0^M f(t) f''(t) \d t\right)^2 \leq \int_0^M f(t)^2 \d t \int_0^M f''(t)^2 \d $.

\item En déduire l'inégalité demandée.
\end{questions}
\end{exercice}

\begin{solution}
\begin{reponses}
\item Comme la fonction $f'$ est de classe $\mathscr{C}^1$ sur $[0, M]$, d'après la formule d'intégration par parties,
\[
\int_0^M f(t) f''(t) \d t = (f(M) f'(M) - f(0) f'(0)) - \int_0^M (f'(t))^2 \d t.
\]

\item Supposons par l'absurde que $f'$ ne soit pas de carré intégrable. Comme $M \mapsto \int_0^M (f'(t))^2 \d t$ est croissante, alors $\lim\limits_{M\to+\infty} \int_0^M (f'(t))^2 \d t = +\infty$. D'après la formule d'intégration par parties, comme $f$ est de carré intégrable, $\lim\limits_{M\to+\infty} f(M) f'(M) = +\infty$.

Alors, il existe un réel $M_0$ à partir duquel la fonctoin $f(M) f(M')$ est plus grande que $2$. Ainsi, pour $M \geq M_0$,
\[
f(M)^2 - f(M_0)^2 = 2 \int_{M_0}^M f(t) f'(t) \d t \geq 2 (M - M_0),
\]
et $f$ n'est pas de carré intégrable.

Finalement, $f'$ est bien de carré intégrable.

\item En reprenant la formule d'intégration par parties, comme $M \mapsto \int_0^M f(t) f''(t) \d t$ et $M \mapsto \int_0^M (f'(t))^2 \d t$ admettent une limite à l'infini, alors $M \mapsto f(M) f'(M)$ admet également une limite.

De plus, l'argument précédent assure que cette limite doit être nulle.

\item En utilisant l'inégalité de Cauchy-Schwarz sur $[0, M]$,
\[
\left(\int_0^M f(t) f''(t) \d t\right)^2 \leq \int_0^M f(t)^2 \d t \int_0^M (f''(t))^2 \d t.
\]

\item Comme $f(0) = 0$, alors
\[
\int_0^{+\infty} f(t) f''(t) \d t = - \int_0^{+\infty} (f'(t))^2 \d t.
\]
On conclut en passant à la limite dans l'inégalité de Cauchy-Schwarz.
\end{reponses}
\end{solution}

\begin{remarque}
Le cas d'égalité correspond au cas d'égalité dans l'inégalité de Cauchy-Schwarz. Ici,
\begin{itemize}
\item soit $f'' = 0$ et il existe $(a, b) \in \R^2$ tel que $f : t \mapsto a t + b$. Comme $f$ est de carré intégrable, alors $f \equiv 0$.

\item soit il existe $\lambda \in \R$ tel que $f = \lambda f''$. Alors,
\[
f \in \left\{t \mapsto A \cos(\omega t + \phi),\, t \mapsto A \cosh(\omega t + \phi),\, (\omega,\, \phi) \in \R^2\right\}.
\]
La seule fonction de carré intégrable de cet ensemble est la fonction nulle.
\end{itemize}
Ainsi, l'égalité est atteinte uniquement par la fonction nulle.
\end{remarque}

%---------------

\todoinline{Source : Mines 2017}

\begin{prop}
Soit $f \in \mathscr{C}^1(\R_+,\R)$ telle que $f$ et $f'$ soient de carré intégrable sur $\R_+$. Alors,
\[
\abs{f(0)}^4 \leq 2 \int_0^{+\infty} f^2(t) \d t \cdot \int_0^{+\infty} f'^2(t) \d t.
\]
\end{prop}

\begin{exercice}
Soit $M \geq 0$.
\begin{questions}
\item Montrer que $\int_0^M f(t) f'(t) \d t = \frac{f(M)^2 - f(0)^2}{2}$.

\item En déduire que $\lim\limits_{M\to+\infty} f(M) = 0$.

\item Conclure.
\end{questions}
\end{exercice}

\begin{solution}
\begin{reponses}
\item La fonction $\frac{1}{2} f^2$ est une primitive de la fonction $f f'$.

\item Comme $f$ et $f'$ sont de carré intégrable, alors $M \mapsto \int_0^M f(t) f'(t) \d t$ converge. Ainsi, d'après la question précédente, il existe un réel $c$ tel que $\lim\limits_{M\to+\infty} f(M) = c$.

Comme $f$ est de carré intégrable, on montre que $c = 0$.
\todoinline{Mettre une référence vers cet argument déjà développé dans un des premiers chapitres.}

\item En passant à la limite dans l'égalité précédente,
\[
2 \int_0^{+\infty} f(t) f'(t) \d t = f(0)^2.
\]

De plus, en utilisant l'inégalité de Cauchy-Schwarz, comme $f$ et $f'$ sont de carré intégrable,
\[
|f(0)|^2 \leq 2 \int_0^{+\infty} f(t)^2 \d t \cdot \int_0^{+\infty} (f'(t))^2 \d t.
\]
\end{reponses}
\end{solution}

%--------------
\subsubsection{Inégalité de \textsc{Poincaré}}

\begin{prop}[Inégalité de \textsc{Poincaré}]
Soit $f : [0, 1] \to \R$ une fonction de classe $\mathscr{C}^1$ telle que \mbox{$f(0) = f(1) = 0$}. Alors,
\[
\int_0^1 f(x)^2 \d x \leq \frac{1}{\pi^2} \int_0^1 f'(x)^2 \d x.
\]
\end{prop}

\begin{exercice}
Soit $f$ une fonction de classe $\mathscr{C}^1$ telle que $f(0) = f(1) = 0$.
\begin{questions}
\item Déterminer des équivalents de $x \mapsto \cotan(\pi x)$ en $x = 0$ puis en $x = 1$.
\end{questions}

On pose
\[
I = \int_0^1 f(x) f'(x) \cotan(\pi x) \d x.
\]

\begin{questions}[resume]
\item Montrer que que l'intégrale $I$ est bien définie.

\item Soit $0 < a < b < 1$. Appliquer une intégration par parties à l'intégrale
\[
\int_a^b f(x) f'(x) \cotan(\pi x) \d x.
\]

\item En déduire que l'on a
\[
2 \pi I = \pi^2 \int_0^1 f(x)^2 (1 + \cotan(\pi x)^2) \d x.
\]
\end{questions}

On considère l'intégrale $J$ définie par
\[
J = \int_0^1 \left(f'(x) - \pi f(x) \cotan(\pi x)\right)^2 \d x.
\]

\begin{questions}[resume]
\item Montrer que $J$ est bien définie.

\item Conclure.
\end{questions}
\end{exercice}

\begin{solution}
\begin{reponses}
\item En utilisant les équivalents classiques en $0$, $\cotan(\pi x) = \frac{\cos(\pi x)}{\sin(\pi x)} \sim_0 \frac{1}{\pi x}$.


En effectuant le changement de variable $u = 1 - x$,
\begin{align*}
\cotan(\pi x)
&= \cotan(\pi - \pi u)
% = \frac{\cos(\pi - \pi u)}{\sin(\pi - \pi u)}\\
% &= \frac{-\cos(-\pi u)}{\sin(\pi u)}\\
% &= -\frac{\cos(\pi u)}{\sin(\pi u)}\\
= - \cotan(\pi u)
\underset{u\to0}{\sim} - \frac{1}{\pi u}
\underset{x\to1}{\sim} -\frac{1}{\pi (1 - x)}.
\end{align*}

\item Posons $u : x \mapsto f(x) f'(x) \cotan(\pi x)$.
\begin{itemize}
\item La fonction $u$ est continue sur $]0, 1[$.

\item D'après la question précédente,
$f(x) f'(x) \cotan(\pi x) \sim_0 \frac{f(x) f'(x)}{\pi x}$.

Comme $f$ est dérivable en $0$ et $f(0) = 0$, alors $\lim\limits_{x\to 0} u(x) = \frac{f'(0)^2}{\pi}$ et $u$ est prolongeable par continuité en $0$.

\item D'après la question précédente,
$f(x) f'(x) \cotan(\pi x) \sim_1 \frac{f(x) f'(x)}{\pi (x - 1)}$.

Comme $f$ est dérivable en $1$ et $f(1) = 0$, alors $\lim\limits_{x\to 1} u(x) = \frac{f'(1)^2}{\pi}$ et $u$ est prolongeable par continuité en $1$.
\end{itemize}

Finalement, la fonction $u$ est prolongeable par continuité sur $[0, 1]$ et l'intégrale $\int_0^1 u(x) \d x$ est donc convergente.

\item Posons $u(x) = \frac{f(x)^2}{2}$ et $v(x) = \cotan(\pi x)$. Alors, $u'(x) = f'(x) f(x)$ et $v'(x) = -\pi (1 + \cotan^2(\pi x))$. Les fonctions $u$ et $v$ sont donc de classe $\mathscr{C}^1$ sur $[a, b]$. En utilisant la formule d'intégration par parties,
\begin{align*}
\int_a^b f(x) f'(x) \cotan(\pi x) \d x
&= \left[\frac{f(x)^2}{2} \cotan(\pi x)\right]_a^b + \cdots\\
&\qquad\cdots + \pi \int_a^b \frac{f(x)^2}{2} (1 + \cotan^2(\pi x)) \d x\\
&= \frac{f(a)^2}{2} \cotan(\pi a) - \frac{f(b)^2}{2} \cotan(\pi b) + \cdots\\
&\qquad\cdots + \pi \int_a^b \frac{f(x)^2}{2} (1 + \cotan(\pi x)^2) \d x.
\end{align*}

\item D'une part,
$
f(a)^2 \cotan(\pi a)
\sim_0 \frac{f(a)^2}{\pi a}
\to \frac{f'(0) f(0)}{\pi}
= 0$.


D'autre part, comme l'intégrale $I$ converge,
$\lim\limits_{a\to 0}
\int_a^b f(x) f'(x) \cotan(\pi x) \d x
\to \int_0^b f(x) f'(x) \cotan(\pi x) \d x$.

Ainsi, d'après les théorèmes d'addition des limites, la fonction \mbox{$a \mapsto \int_a^b f(x)^2 (1 + \cotan(\pi x)^2) \d x$} admet une limite en $0$ et
\begin{align*}
\int_0^b f(x) f'(x) \cotan(\pi x) \d x
&= - \frac{f(b)^2}{2} \cotan(\pi b) + \frac{\pi}{2} \int_0^1 f(x)^2 (1 + \cotan(\pi x)^2) \d x.
\end{align*}

\medskip

Un raisonnement analogue montre que
\[
\int_0^1 f(x) f'(x) \cotan(\pi x) \d x = \frac{\pi}{2} \int_0^1 f(x)^2 (1 + \cotan(\pi x)^2) \d x.
\]

Finalement, on obtient bien
\[
2 \pi I = \pi^2 \int_0^1 f(x)^2 (1 + \cotan(\pi x)^2) \d x.
\]

\item Nous avons déjà montré que $\lim\limits_{x\to0} \pi f(x) \cotan(\pi x) = f'(0)$ et $\lim\limits_{x\to1} \pi f(x) \cotan(\pi x) = f'(1)$.

Ainsi, $x \mapsto f'(x) - \pi f(x) \cotan(\pi x)$ est prolongeable par continuité par $0$ en $0$ et en $1$.

Comme cette fonction est par ailleurs continue sur $]0, 1[$, l'intégrale $J$ est bien convergente.

\item Comme $J$ est l'intégrale d'une fonction positive, alors $J \geq 0$. De plus, en utilisant la linéarité des intégrales convergentes,
\begin{align*}
0 \leq J &= \int_0^1 \left[f'(x)^2 - 2 \pi f(x) f'(x) \cotan(\pi x) + \pi^2 f(x)^2 \cotan(\pi x)^2\right] \d x\\
&= \int_0^1 f'(x)^2 \d x - 2 \pi I + \pi^2 \int_0^1 f(x)^2 \cotan(\pi x)^2 \d x\\
&= \int_0^1 f'(x)^2 \d x - \pi^2 \int_0^1 f(x)^2 (1 + \cotan(\pi x)^2) \d x + \cdots\\
&\qquad\qquad\qquad\qquad\qquad  \cdots + \pi^2 \int_0^1 f(x)^2 (1 + \cotan(\pi x)^2) \d x\\
&= \int_0^1 f'(x)^2 \d x - \pi^2 \int_0^1 f(x)^2 \d x.
\end{align*}

Finalement,
\[
\int_0^1 f(x)^2 \d x \leq \frac{1}{\pi^2} \int_0^1 f'(x)^2 \d x.
\]
\end{reponses}
\end{solution}

\section{Intégrale de \nom{Poisson} : une utilisation des sommes de \nom{Riemann}}

\todoarmand{
Un dm \url{http://alain.troesch.free.fr/2023/Fichiers/dm11.pdf} qui montre comment le calcul de cette intégrale peut se ramener à
l’intégrale du noyau de Poisson.
}

\todoinline{Trouver une référence pour le noyau de Poisson.}

%---------------

\begin{prop}[Intégrale de \nom{Poisson}]
Pour tout $x \in \Rp$, on pose $I(x) = \int_0^\pi \ln(1 - 2 x \cos t + x^2) \d t$. Alors, pour tout $x$ réel positif,
\begin{align*}
I(x) &= 2 \pi \ln(x).
\end{align*}
\end{prop}

\begin{exercice}\label{exercice:integralePoisson}
\begin{questions}
\item Montrer que l'intégrale $I$ est bien définie sur $\Rp$.

\item Soit $n \in \Ne$ et $x \neq 1$. Montrer que
\[
\prod_{k=0}^{n-1} \mathopen{}\left(1 - 2 \cos\mathopen{}\left(\frac{k\pi}{n}\right) x + x^2\right)
= \frac{x^{2n} - 1}{x + 1} (x - 1).
\]
% Déterminer une expression simple de $\sum\limits_{k=0}^{n-1} \ln\mathopen{}\big(1 - 2 x \cos\mathopen{}\big(\frac{k\pi}{n}\big) + x^2\big)$.

\item En déduire la valeur de $I(x)$ pour $x \neq 1$.

\item Montrer que $I(1) = 2 \pi\ln(2) + 4 \int_0^{\pi/2} \ln(\sin u) \d u$.

\item En déduire la valeur de $I(1)$.
\end{questions}
\end{exercice}


\begin{solution}
\begin{reponses}
\item En utilisant le discriminant réduit, $\cos(t)^2 - 1 < 0$ sur $\interoo{0}{\pi}$ donc le trinôme n'admet pas de racine et le logarithme est bien défini.

De plus, en $t = 0$, $1 - 2 x + x^2 = (1 - x)^2 > 0$ car $x \neq 1$.
En $t = \pi$, $1 + 2 x + x^2 = (1 + x)^2 > 0$ car $x \neq 1$.

Ainsi, $I(x)$ est bien définie pour $x \neq 1$.

\item En utilisant les factorisations classiques ainsi que les résultats sur les racines $n$-èmes de l'unité,
\begin{align*}
\prod_{k=0}^{n-1} \mathopen{}\left(1 - 2 \cos\mathopen{}\left(\frac{k\pi}{n}\right) x + x^2\right)
&= \prod_{k=0}^{n-1} \mathopen{}\left(x - \e^{\i k\pi/n}\right) \mathopen{}\left(x - \e^{-\i k\pi/n}\right) \\
&= \prod_{k=0}^{n-1} \mathopen{}\left(x - \e^{2\i k\pi/(2n)}\right) \prod_{k=n+1}^{2n} \mathopen{}\left(x - \e^{2\i k\pi/(2n)}\right) \\
&= \frac{x^{2n} - 1}{x + 1} (x - 1)
\end{align*}

\item D'après le calculs précédents,
\begin{align*}
S_n(f)
&\defeq \frac{\pi}{n}\sum\limits_{k=0}^{n-1} \ln\mathopen{}\left(1 - 2 x \cos\mathopen{}\left(\frac{k\pi}{n}\right) + x^2\right)
&= \frac{\pi}{n} \ln\mathopen{}\abs{x^{2n}-1} - \frac{\pi}{n} \ln \mathopen{}\abs{\frac{x+1}{x-1}} \\
&= 2 \pi \ln\mathopen{}\abs{x} + \frac{\pi}{n} \ln \mathopen{}\abs{1 - x^{-2n}} - \frac{\pi}{n} \ln \mathopen{}\abs{\frac{x+1}{x-1}}.
\end{align*}

Ainsi, en utilisant une somme de \nom{Riemann} de pas $\frac{\pi}{n}$, $\lim\limits_{n\to +\infty} S_n(f) = 2 \pi \ln\mathopen{}\abs{x}$.

Finalement, pour tout $x \neq 1$, $I(x) = 2 \pi \ln\mathopen{}\abs{x}$.
\item \marginnote[-7pt]{\hyperref[exercice:integraleEuler]{Intégrale d'\nom{Euler}}}D'après les propriétés des fonctions trigonométriques,
\begin{align*}
\ln(2 - 2 \cos t)
= \ln\mathopen{}\big(2(1 - \cos t)\big)
= \ln\mathopen{}\big(4 \sin(t/2)^2\big)
= \ln(4) + 2 \ln\mathopen{}\big(\sin(t/2)\big)
\end{align*}

\item En effectuant le changement de variable affine $\fonctionligne[\varphi]{u}{2 u}$,
\begin{align*}
\int_0^\pi \ln(2 - 2 \cos t) \d t
= 2 \pi \ln(2) + 2 \int_0^\pi \ln\mathopen{}\big(\sin(t/2)\big) \d t
= 2 \pi \ln(2) + 4 \int_0^{\pi/2} \ln(\sin u) \d u
\end{align*}
Or, $\int_0^{\pi/2} \ln(\sin u) \d u = \int_0^{\pi/2} \ln(\cos u) \d u$. Ainsi,
% \begin{comment}

\begin{align*}
\int_0^{\pi/2} \ln(\sin u) \d u + \int_0^{\pi/2} \ln(\cos u) \d u
&= \int_0^{\pi/2} \ln\mathopen{}\big(\sin(2 t)\big) \d t - \frac{\pi}{2} \ln(2) \\
&= \frac{1}{2} \int_0^\pi \ln(\sin t) \d t - \frac{\pi}{2} \ln(2) \\
&= \begin{multlined}[t]\frac{1}{2} \int_0^{\pi/2} \ln(\sin t) \d t + \frac{1}{2} \int_{\pi/2}^{\pi} \ln(\sin t) \d t \\
- \frac{\pi}{2} \ln(2)
\end{multlined}\\
&= \int_0^{\pi/2} \ln(\sin t) \d t - \frac{\pi}{2} \ln(2)
\end{align*}

Ainsi,
\[
\int_0^{\pi/2} \ln(\sin t) \d t = -\frac{\pi}{2} \ln(2)
\]

D'où $I(1) = 0$.
% \end{comment}
\end{reponses}
\end{solution}

\begin{marginfigure}[-5cm]
    \centering
    \includegraphics[scale=0.08]{illustrations/Journal_de_l_ecole_polytechnique_cahier_17_1815.png}
    % \includepdf[pages={3},scale=.4]{Journal_de_l_ecole_polytechnique_cahier_17_1815.pdf}
    % \caption{\chevron{Suite du mémoire sur les intégrales définies}, Journal de l'École polytechnique, Cahier 17, X (1815), 612-631 (% \url{https://gallica.bnf.fr/ark:/12148/bpt6k433673r/f614.item})}
\end{marginfigure}
\todoarmand{Problème avec la légende de la figure}

% \printbibliography[heading=bibintoc, title=Références]

% \printindex % Output the index

\end{document}
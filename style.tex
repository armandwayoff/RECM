%%%%%%%%%%%%%%%%%%%%%%%%%%%
%%%%%%%%%%%%%%%%%%%%%%%%%%%

\colorlet{myred}{red!85!black}
\colorlet{myblue}{blue!80!black}
\colorlet{mygreen}{green!70!black}
\colorlet{mycyan}{cyan!80!black}
\colorlet{myorange}{orange!90!black!80}
\colorlet{mypurple}{red!50!blue!90!black!80}

\colorlet{mylightred}{myred!20}
\colorlet{mylightblue}{myblue!20}
\colorlet{mylightgreen}{mygreen!20}

\colorlet{mydarkred}{myred!80!black}
\colorlet{mydarkblue}{myblue!60!black}

\colorlet{xcol}{blue!60!black}
\tikzstyle{xline}=[xcol,thick]

%%%%%%%%%%%%%%%%%%%%%%%%%%%
%%%%%%%%%%%%%%%%%%%%%%%%%%%

%\addtokomafont{title}{\color{Maroon}}
%\addtokomafont{part}{\color{Maroon}}
%\addtokomafont{chapter}{\color{Maroon}}
%\addtokomafont{section}{\color{Maroon}}
%\addtokomafont{subsection}{\color{Maroon}}
%\addtokomafont{subsubsection}{\color{Maroon}}
%\addtokomafont{paragraph}{\color{Maroon}}
%\addtokomafont{captionlabel}{\color{Blue}}
%\addtokomafont{pagenumber}{\color{Maroon}}

% Set KOMA fonts for book-specific elements
\addtokomafont{part}{\normalfont\fontsize{26pt}{28pt}\selectfont\scshape\bfseries}
\addtokomafont{partentry}{\normalfont\scshape\bfseries} 
\addtokomafont{chapter}{\normalfont\scshape\bfseries\fontsize{26pt}{28pt}\selectfont\color{BlueViolet}}
\addtokomafont{chapterentry}{\normalfont\bfseries}

% Set KOMA fonts for elements common to all classes
\addtokomafont{section}{\normalfont\bfseries\color{BlueViolet}}
\addtokomafont{subsection}{\normalfont\bfseries}
\addtokomafont{subsubsection}{\normalfont\bfseries}
\addtokomafont{paragraph}{\normalfont\bfseries}
\setkomafont{descriptionlabel}{\normalfont\bfseries}

\hypersetup{ % Set up hyperref options
	unicode, % Use unicode for links
	pdfborder={0 0 0}, % Suppress border around pdf
	%xetex,
	%pagebackref=true,
	%hyperfootnotes=false, % We already use footmisc
	bookmarksdepth=section,
	bookmarksopen=true, % Expand the bookmarks as soon as the pdf file is opened
	%bookmarksopenlevel=4,
	linktoc=all, % Toc entries and numbers links to pages
	breaklinks=true,
	colorlinks=true,
	%allcolors=DarkGreen,
    linkcolor = BlueViolet,
	anchorcolor=BlueViolet, 
	citecolor=myred,
	filecolor=myred,
	menucolor=myred,
	runcolor=gray,
	urlcolor=myred,
}

% Pour remplacer la version anglaise par défaut
\addto\captionsfrench{\def\tablename{Tableau}}

\renewcommand{\bfdefault}{b} % https://tex.stackexchange.com/questions/27843/level-of-boldness-changeable

\renewcommand{\sfdefault}{xcmss} % https://ctan.tetaneutral.net/fonts/sansmathfonts/sansmathfonts.pdf

%%%%%%%%%%%%%%%%%%%%%%%%%%%%%%%%%%%%%%%%%%%%%%%%%%%%%%%%%%%%%%
% https://tex.stackexchange.com/questions/88281/how-to-change-font-for-the-integral-symbol
% \makeatletter
% \def\upintkern@{\mkern-7mu\mathchoice{\mkern-3.5mu}{}{}{}}
% \def\upintdots@{\mathchoice{\mkern-4mu\@cdots\mkern-4mu}%
%  {{\cdotp}\mkern1.5mu{\cdotp}\mkern1.5mu{\cdotp}}%
%  {{\cdotp}\mkern1mu{\cdotp}\mkern1mu{\cdotp}}%
%  {{\cdotp}\mkern1mu{\cdotp}\mkern1mu{\cdotp}}}
% \newcommand{\upiint}{\DOTSI\protect\UpMultiIntegral{2}}
% \newcommand{\upiiint}{\DOTSI\protect\UpMultiIntegral{3}}
% \newcommand{\upiiiint}{\DOTSI\protect\UpMultiIntegral{4}}
% \newcommand{\upidotsint}{\DOTSI\protect\UpMultiIntegral{0}}
% \newcommand{\UpMultiIntegral}[1]{%
%   \edef\ints@c{\noexpand\upintop
%     \ifnum#1=\z@\noexpand\upintdots@\else\noexpand\upintkern@\fi
%     \ifnum#1>\tw@\noexpand\upintop\noexpand\upintkern@\fi
%     \ifnum#1>\thr@@\noexpand\upintop\noexpand\upintkern@\fi
%     \noexpand\upintop
%     \noexpand\ilimits@
%   }%
%   \futurelet\@let@token\ints@a
% }
% \makeatother

% \DeclareFontFamily{OMX}{mdbch}{}
% \DeclareFontShape{OMX}{mdbch}{m}{n}{ <->s * [0.8]  mdbchr7v }{}
% \DeclareFontShape{OMX}{mdbch}{b}{n}{ <->s * [0.8]  mdbchb7v }{}
% \DeclareFontShape{OMX}{mdbch}{bx}{n}{<->ssub * mdbch/b/n}{}

% \DeclareSymbolFont{uplargesymbols}{OMX}{mdbch}{m}{n}
% \SetSymbolFont{uplargesymbols}{bold}{OMX}{mdbch}{b}{n}
% \DeclareMathSymbol{\upintop}{\mathop}{uplargesymbols}{82}

% \makeatletter
% \newcommand{\upint}{\DOTSI\upintop\ilimits@}
% \makeatother

% \let\int\upint
% \let\idotsint\upidotsint
%%%%%%%%%%%%%%%%%%%%%%%%%%%%%%%%%%%%%%%%%%%%%%%%%%%%%%%%%%%%%

% https://tex.stackexchange.com/questions/124049/applying-options-to-already-loaded-package
% Author: Izaak Neutelings (August, 2017)


\tikzset{>=latex} % for LaTeX arrow head
\contourlength{1.2pt}
\usetikzlibrary{positioning,calc}
\usetikzlibrary{backgrounds}% required for 'inner frame sep'
%\usepackage{adjustbox} % add whitespace (trim)

% define gaussian pdf and cdf
\pgfmathdeclarefunction{gauss}{3}{%
  \pgfmathparse{1/(#3*sqrt(2*pi))*exp(-((#1-#2)^2)/(2*#3^2))}%
}
\colorlet{mydarkblue}{blue!30!black}

% to fill an area under function
\usepgfplotslibrary{fillbetween}
\usetikzlibrary{patterns}
\pgfplotsset{compat=1.12} % TikZ coordinates <-> axes coordinates
% https://tex.stackexchange.com/questions/240642/add-vertical-line-of-equation-x-2-and-shade-a-region-in-graph-by-pgfplots

% plot aspect ratio
%\def\axisdefaultwidth{8cm}
%\def\axisdefaultheight{6cm}

% number of sample points
\def\N{50}

\begin{tikzpicture}[scale=1]
  \message{Cumulative probability^^J}
  
  \def\B{0};
  \def\Bs{6.0};
  \def\xmax{\B+3.5*\Bs};
  \def\ymin{{-0.1*gauss(\B,\B,\Bs)}};
  \def\h{0.07*gauss(\B,\B,\Bs)};
  \def\a{\B-0.8*\Bs};
  
  \begin{axis}[every axis plot post/.append style={
               mark=none,
               domain={-1*(\xmax)}:{1*(\xmax)},samples=\N,smooth},
               % xmin={-1*(\xmax)}, 
               xmax={1.06*(\xmax)},
               ymin=\ymin, ymax={1.3*gauss(\B,\B,\Bs)},
               axis lines=middle,
               axis line style=thick,
               axis line style={-latex},
               ticks=none,
               xlabel=$x$,
               every axis x label/.style={at={(current axis.right of origin)},anchor=north},
               y=700pt,
               clip=false
              ]
    
    % PLOTS
    \addplot[blue,thick,name path=B] {gauss(x,\B,\Bs)};
    % FILL
    \path[name path=xaxis]
      (0,0) -- (\pgfkeysvalueof{/pgfplots/xmax},0);
    \addplot[blue!25, opacity=0.9] fill between[of=xaxis and B];
    % LINES
    \node[blue,above left] at ({-0.7*(\B+\Bs)},{1.2*gauss(\B+\Bs,\B,\Bs)}) {$x \mapsto \mathrm{e}^{-x^2}$};
    \node[blue!60!black] at ({0},{0.6*gauss(0.85*(\a),\B,\Bs)}) {$\sqrt{\pi}$};
    
  \end{axis}
\end{tikzpicture}
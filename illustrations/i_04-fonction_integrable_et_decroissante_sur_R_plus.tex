\begin{tikzpicture}

  % https://copyprogramming.com/howto/how-do-i-draw-arrows-at-coordinate-on-a-plot
  
  \def\xmax{4.5}
  \def\ymax{1.5}
  \def\ymin{-2}
  \def\xzero{2}
  \def\x{5}

  \def\colfonc{green!70!black}

    \begin{axis}[
        restrict x to domain=0:8,
        samples=100, % you don't need 1000, it only slows things down
        ticks=none,
        xmin = -1, xmax = \xmax+1,
        ymin = \ymin, ymax = \ymax,
        unbounded coords=jump,
        axis x line=middle,
        axis y line=middle,
        axis line style={-latex},
        % xlabel={$x$},
        % ylabel={$y$},
        x label style={
          at={(axis cs:5.1,0.2)},
          anchor=west,
        },
        every axis y label/.style={
          at={(axis cs:0,1.5)},
          anchor=south
        },
        legend style={
          at={(axis cs:-5.2,4)},
          anchor=west, font=\scriptsize
        },
        % ,declare function={f(\x)=2*e^(-\x^2/2)-(\x/7)^2-1;},
        ] 
\pgfmathdeclarefunction{f}{1}{%
\pgfmathparse{2*e^(-#1^2/2)-(#1/7)^2-1}%
}
      \addplot[name path=A,very thick,color=\colfonc, mark=none, domain=0:\xmax] {f(x)} node [above=2mm,near start] {$f$};
          
      % \addplot[mark=*,fill=white] coordinates {(\xzero,{f(\xzero)})};
      \draw[dashed] (axis cs:0,{f(\xzero)}) node[left=1mm] (l) {$f(x_0)$} -| (axis cs:2,0) node[above] (a) {$x_0$}; 
      \draw (axis cs:\x-0.5,0) node[above] {$x$};

    \path [name path=B] (0,0)--(\xmax, 0);
    \addplot [blue!20!white] fill between [of=A and B, soft clip={domain=\xzero:\x}];

    \fill[pattern color=black, pattern=north east lines] (\xzero,0) rectangle (\x-0.5,f(\xzero);
    
    \end{axis}

    % \draw[->, black] (5, 4) node[above] {$f(x_0) (x - x_0)$} to [out=-90,in=87] ($(4.5,2.6)$);
    \contourlength{2pt}
    \node at (4.5,2.6) {\contour{blue!20!white}{$f(x_0) (x - x_0)$}};

    \draw[->, black] (4, 1) node[left] {\color{blue}$\displaystyle \int_{x_0}^x f(t)\, \mathrm{d} t$} to [out=20,in=-110] ($(4.5,1.6)$);
    
\end{tikzpicture}
\begin{tikzpicture}[
    label/.style={black},
    vector/.style={ultra thick,-latex}
  ]

  \def\xmin{-1} \def\xmax{5}
  \def\ymin{-1} \def\ymax{5}
  \def\deltax{0.4} \def\deltay{0.3}
  \def\gridscale{3}
  
  \def\spacing{0.15} 

  \def\colTrigHB{red} \def\colTrigGD{blue} \def\colRect{green}
  \def\opac{0.3}
  \def\colx{violet} \def\coly{cyan}

  \begin{scope}
    \coordinate (origin) at (0,0);
    
    % repère droit
    % \draw [very thick,->] (\xmin,0) -- (\xmax,0);
    % \draw [very thick,->] (0,\ymin) -- (0,\ymax);
    \clip [draw] (\xmin,\ymin) rectangle (\xmax,\ymax);
    
    % quadrillage droit
    \draw[style=help lines] (\xmin-\xmax,\ymin-\ymax) grid[step=\deltay/\deltax] (-\xmin+\xmax,-\ymin+\ymax);
    
    % triangle haut
    \filldraw [\colTrigHB!50, fill opacity=\opac] (\deltax * \gridscale, \gridscale) -- (\gridscale + \deltax * \gridscale, \gridscale + \deltay * \gridscale) -- (\deltax * \gridscale, \gridscale + \deltay * \gridscale) -- cycle;
    
    % triangle bas
    \filldraw [\colTrigHB!50, fill opacity=\opac] (origin) -- (\gridscale, \deltay * \gridscale) -- (\gridscale, 0) -- cycle;
    
    % triangle gauche
    \filldraw [\colTrigGD!50, fill opacity=\opac] (origin) -- (0, \gridscale) -- (\deltax * \gridscale, \gridscale) -- cycle;
    
    % triangle droit
    \filldraw [\colTrigGD!50, fill opacity=\opac] (\gridscale, \deltay * \gridscale) -- (\gridscale + \deltax * \gridscale, \gridscale + \deltay * \gridscale) -- (\gridscale + \deltax * \gridscale, \deltay * \gridscale) -- cycle;
    
    % rectangle haut gauche
    \filldraw[\colRect!50,fill opacity=\opac] (\deltax * \gridscale, \gridscale) rectangle (0, \gridscale + \deltay * \gridscale);
    
    % rectangle bas droite
    \filldraw[\colRect!50,fill opacity=\opac] (\gridscale, 0) rectangle (\gridscale + \deltax * \gridscale, \deltay * \gridscale);
    
    % curly brackets
    \draw [thick, decorate,
    decoration = {calligraphic brace, mirror, raise=1pt, amplitude=5pt}] (origin) --  (\gridscale, 0) node[pos=0.5pt,below=5pt,black]{$\textcolor{\colx}{a}$};
    
    \draw [thick, decorate,
    decoration = {calligraphic brace, mirror, raise=1pt, amplitude=5pt}] (\gridscale, 0) -- (\gridscale + \deltax * \gridscale, 0) node[pos=0.5pt,below=5pt,black]{$\textcolor{\coly}{c}$};
    
    \draw [thick, decorate,
    decoration = {calligraphic brace, mirror, raise=1pt, amplitude=5pt}] (\gridscale + \deltax * \gridscale, 0) --  (\gridscale + \deltax * \gridscale, \deltay * \gridscale) node[pos=0.5pt,right=5pt,black]{$\textcolor{\colx}{b}$};
    
    \draw [thick, decorate,
    decoration = {calligraphic brace, mirror, raise=1pt, amplitude=5pt}] (\gridscale + \deltax * \gridscale, \deltay * \gridscale) -- (\gridscale + \deltax * \gridscale, \gridscale + \deltay * \gridscale) node[pos=0.5pt,right=5pt,black]{$\textcolor{\coly}{d}$};
    
    % aires
    \node at (\gridscale + \deltax * \gridscale / 2, \deltay * \gridscale / 2) {$\textcolor{black}{bc}$};
    \node at (\gridscale + \deltax * \gridscale / 1.5, \gridscale / 2 + \deltay * \gridscale / 2) {$\displaystyle \frac{dc}{2}$};
    \node at (\gridscale / 1.5, \deltay * \gridscale / 3) {$ab/2$};
    \node at (\gridscale / 2 + \deltax * \gridscale / 2, \gridscale / 2 + \deltay * \gridscale / 2) {\Huge $\mathcal{A}$};


    \pgftransformcm{1}{\deltay}{\deltax}{1}{\pgfpoint{0}{0}}

    % quadrillage oblique
    \draw[style=help lines,dashed] (\xmin-\xmax,\ymin-\ymax) grid[step=\gridscale] (-\xmin+\xmax,-\ymin+\ymax);

    % noeuds
    \foreach \x in {\xmin,...,\xmax}{
        \foreach \y in {\ymin,...,\ymax}{
            \node[draw,circle,inner sep=1pt,fill] at (\gridscale*\x,\gridscale*\y) {};
          }
      }
      
    \filldraw[fill=yellow,fill opacity=\opac] (origin) rectangle (\gridscale,\gridscale);

    \draw [vector, \colx] (origin) -- (\gridscale,0) node [label,right=0] {$\textcolor{\colx}{\vec{x}}$};
    \draw [vector, \coly] (origin) -- (0,\gridscale) node [label,above=0] {$\textcolor{\coly}{\vec{y}}$};
    
    % je rajoute le point de l'origine
    \node[draw,circle,inner sep=1pt,fill] at (origin) {};
    \end{scope}
\end{tikzpicture}